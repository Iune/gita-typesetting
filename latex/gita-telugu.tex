%!TEX program = xelatex
\documentclass{scrbook}
\KOMAoptions{paper=6in:9in}
\KOMAoptions{fontsize=11pt}
\KOMAoptions{toc=chapterentrydotfill}
\KOMAoptions{twoside=false}

\addtokomafont{disposition}{\rmfamily}
\renewcommand*{\raggedsection}{\centering}
% \usepackage[left=1in,right=1in,top=1in,bottom=1in,]{geometry}

% Formatting
\usepackage{multicol}
\usepackage{float}
\usepackage{hyperref}

% Fonts
\usepackage{fontspec}
\setmainfont[
    Path = assets/,
    UprightFont = Junicode.ttf,
    BoldFont = Junicode-Bold.ttf]{Junicode}
\newfontfamily{\telfont}[
    Path = assets/, 
    UprightFont = TiroTelugu-Regular.ttf,
    ItalicFont = TiroTelugu-Italic.ttf,
    Scale=MatchUppercase, Script=Telugu]{Tiro Telugu}


% Commands
\newcommand{\tel}[1]{\begin{telfont}{#1}\end{telfont}}
\newcommand{\romline}[1]{{#1}}
\newcommand{\natline}[1]{\tel{#1}}

% Variables

\title{Bhagavad Gita}
\subtitle{Typeset in Telugu}
\author{Aditya Duri}
\date{}

\begin{document}
\maketitle
\frontmatter

\tableofcontents
\newpage

\chapter{Dhyāna Ślokas}
\begin{table}[H]
\begin{tabular}{cl}
 & \romline{śrī vallabhā samāśliṣṭaṃ daśahastaṃ gajānanaṃ} \\
 & \romline{gaṇanātha mahaṃ vaṃde sarva siddhi pradāyakam}
\end{tabular}
\end{table}

\begin{table}[H]
\begin{tabular}{cl}
 & \romline{prapanna pārijātāya totravetraika pāṇaye} \\
 & \romline{jñānamudrāya kṛṣṇāya gītāṃṛtaduhe namaḥ}
\end{tabular}
\end{table}

\begin{table}[H]
\begin{tabular}{cl}
 & \romline{vyāsāya viṣṇu rūpāya vyāsa rūpāya viṣṇave} \\
 & \romline{namo vai brahma nidhaye vāsiṣṭhāya namo namaḥ}
\end{tabular}
\end{table}

\begin{table}[H]
\begin{tabular}{cl}
 & \romline{śrī kṛṣṇaśśaraṇaṃ mama} \\
 & \romline{śrī kṛṣṇaśśaraṇaṃ mama} \\
 & \romline{śrī kṛṣṇaśśaraṇaṃ mama}
\end{tabular}
\end{table}



\mainmatter
\chapter{Arjuna Viṣāda Yoga}
% \begin{multicols}{2}
\begin{table}[H]
\begin{tabular}{cl}
\textbf{1.0} & \natline{ఓం శ్రీ పరమాత్మనే నమః} \\
 & \natline{అథ ప్రథమోఽధ్యాయః} \\
 & \natline{అర్జునవిషాదయోగః}
\end{tabular}
\end{table}

\begin{table}[H]
\begin{tabular}{cl}
\textbf{1.1} & \natline{ధృతరాష్ట్ర ఉవాచ} \\
 & \natline{ధర్మక్షేత్రే కురుక్షేత్రే} \\
 & \natline{సమవేతా యుయుత్సవః |} \\
 & \natline{మామకాః పాణ్డవాశ్చైవ} \\
 & \natline{కిమకుర్వత సఞ్జయ ||}
\end{tabular}
\end{table}

\begin{table}[H]
\begin{tabular}{cl}
\textbf{1.2} & \natline{సఞ్జయ ఉవాచ} \\
 & \natline{దృష్ట్వా తు పాణ్డవానీకం} \\
 & \natline{వ్యూఢం దుర్యోధనస్తదా |} \\
 & \natline{ఆచార్యముపసఙ్గమ్య} \\
 & \natline{రాజా వచనమబ్రవీత్ ||}
\end{tabular}
\end{table}

\begin{table}[H]
\begin{tabular}{cl}
\textbf{1.3} & \natline{పశ్యైతాం పాణ్డుపుత్రాణామ్} \\
 & \natline{ఆచార్య మహతీం చమూమ్ |} \\
 & \natline{వ్యూఢాం ద్రుపదపుత్రేణ} \\
 & \natline{తవ శిష్యేణ ధీమతా ||}
\end{tabular}
\end{table}

\begin{table}[H]
\begin{tabular}{cl}
\textbf{1.4} & \natline{అత్ర శూరా మహేష్వాసాః} \\
 & \natline{భీమార్జునసమా యుధి |} \\
 & \natline{యుయుధానో విరాటశ్చ} \\
 & \natline{ద్రుపదశ్చ మహారథః ||}
\end{tabular}
\end{table}

\begin{table}[H]
\begin{tabular}{cl}
\textbf{1.5} & \natline{ధృష్టకేతుశ్చేకితానః} \\
 & \natline{కాశిరాజశ్చ వీర్యవాన్ |} \\
 & \natline{పురుజిత్కున్తిభోజశ్చ} \\
 & \natline{శైబ్యశ్చ నరపుఙ్గవః ||}
\end{tabular}
\end{table}

\begin{table}[H]
\begin{tabular}{cl}
\textbf{1.6} & \natline{యుధామన్యుశ్చ విక్రాన్తః} \\
 & \natline{ఉత్తమౌజాశ్చ వీర్యవాన్ |} \\
 & \natline{సౌభద్రో ద్రౌపదేయాశ్చ} \\
 & \natline{సర్వ ఏవ మహారథాః ||}
\end{tabular}
\end{table}

\begin{table}[H]
\begin{tabular}{cl}
\textbf{1.7} & \natline{అస్మాకం తు విశిష్టా యే} \\
 & \natline{తాన్నిబోధ ద్విజోత్తమ |} \\
 & \natline{నాయకా మమ సైన్యస్య} \\
 & \natline{సఞ్జ్ఞార్థం తాన్బ్రవీమి తే ||}
\end{tabular}
\end{table}

\begin{table}[H]
\begin{tabular}{cl}
\textbf{1.8} & \natline{భవాన్భీష్మశ్చ కర్ణశ్చ} \\
 & \natline{కృపశ్చ సమితిఞ్జయః |} \\
 & \natline{అశ్వత్థామా వికర్ణశ్చ} \\
 & \natline{సౌమదత్తిస్తథైవ చ ||}
\end{tabular}
\end{table}

\begin{table}[H]
\begin{tabular}{cl}
\textbf{1.9} & \natline{అన్యే చ బహవః శూరాః} \\
 & \natline{మదర్థే త్యక్తజీవితాః |} \\
 & \natline{నానాశస్త్రప్రహరణాః} \\
 & \natline{సర్వే యుద్ధవిశారదాః ||}
\end{tabular}
\end{table}

\begin{table}[H]
\begin{tabular}{cl}
\textbf{1.10} & \natline{అపర్యాప్తం తదస్మాకం} \\
 & \natline{బలం భీష్మాభిరక్షితమ్ |} \\
 & \natline{పర్యాప్తం త్విదమేతేషాం} \\
 & \natline{బలం భీమాభిరక్షితమ్ ||}
\end{tabular}
\end{table}

\begin{table}[H]
\begin{tabular}{cl}
\textbf{1.11} & \natline{అయనేషు చ సర్వేషు} \\
 & \natline{యథాభాగమవస్థితాః |} \\
 & \natline{భీష్మమేవాభిరక్షన్తు} \\
 & \natline{భవన్తః సర్వ ఏవ హి ||}
\end{tabular}
\end{table}

\begin{table}[H]
\begin{tabular}{cl}
\textbf{1.12} & \natline{తస్య సఞ్జనయన్హర్షం} \\
 & \natline{కురువృద్ధః పితామహః |} \\
 & \natline{సింహనాదం వినద్యోచ్చైః} \\
 & \natline{శఙ్ఖం దధ్మౌ ప్రతాపవాన్ ||}
\end{tabular}
\end{table}

\begin{table}[H]
\begin{tabular}{cl}
\textbf{1.13} & \natline{తతః శఙ్ఖాశ్చ భేర్యశ్చ} \\
 & \natline{పణవానకగోముఖాః |} \\
 & \natline{సహసైవాభ్యహన్యన్త} \\
 & \natline{స శబ్దస్తుములోఽభవత్ ||}
\end{tabular}
\end{table}

\begin{table}[H]
\begin{tabular}{cl}
\textbf{1.14} & \natline{తతః శ్వేతైర్హయైర్యుక్తే} \\
 & \natline{మహతి స్యన్దనే స్థితౌ |} \\
 & \natline{మాధవః పాణ్డవశ్చైవ} \\
 & \natline{దివ్యౌ శఙ్ఖౌ ప్రదధ్మతుః ||}
\end{tabular}
\end{table}

\begin{table}[H]
\begin{tabular}{cl}
\textbf{1.15} & \natline{పాఞ్చజన్యం హృషీకేశః} \\
 & \natline{దేవదత్తం ధనఞ్జయః |} \\
 & \natline{పౌణ్డ్రం దధ్మౌ మహాశఙ్ఖం} \\
 & \natline{భీమకర్మా వృకోదరః ||}
\end{tabular}
\end{table}

\begin{table}[H]
\begin{tabular}{cl}
\textbf{1.16} & \natline{అనన్తవిజయం రాజా} \\
 & \natline{కున్తీపుత్రో యుధిష్ఠిరః |} \\
 & \natline{నకులః సహదేవశ్చ} \\
 & \natline{సుఘోషమణిపుష్పకౌ ||}
\end{tabular}
\end{table}

\begin{table}[H]
\begin{tabular}{cl}
\textbf{1.17} & \natline{కాశ్యశ్చ పరమేష్వాసః} \\
 & \natline{శిఖణ్డీ చ మహారథః |} \\
 & \natline{ధృష్టద్యుమ్నో విరాటశ్చ} \\
 & \natline{సాత్యకిశ్చాపరాజితః ||}
\end{tabular}
\end{table}

\begin{table}[H]
\begin{tabular}{cl}
\textbf{1.18} & \natline{ద్రుపదో ద్రౌపదేయాశ్చ} \\
 & \natline{సర్వశః పృథివీపతే |} \\
 & \natline{సోఉభద్రశ్చ మహాబాహుః} \\
 & \natline{శఙ్ఖాన్దధ్ముః పృథక్పృథక్ ||}
\end{tabular}
\end{table}

\begin{table}[H]
\begin{tabular}{cl}
\textbf{1.19} & \natline{స ఘోషో ధార్తరాష్ట్రాణాం} \\
 & \natline{హృదయాని వ్యదారయత్ |} \\
 & \natline{నభశ్చ పృథివీం చైవ} \\
 & \natline{తుములో వ్యనునాదయన్ ||}
\end{tabular}
\end{table}

\begin{table}[H]
\begin{tabular}{cl}
\textbf{1.20} & \natline{అథ వ్యవస్థితాన్దృష్ట్వా} \\
 & \natline{ధార్తరాష్ట్రాన్ కపిధ్వజః |} \\
 & \natline{ప్రవృత్తే శస్త్రసమ్పాతే} \\
 & \natline{ధనురుద్యమ్య పాణ్డవః ||}
\end{tabular}
\end{table}

\begin{table}[H]
\begin{tabular}{cl}
\textbf{1.21} & \natline{హృషీకేశం తదా వాక్యమ్} \\
 & \natline{ఇదమాహ మహీపతే} \\
 & \natline{అర్జున ఉవాచ |} \\
 & \natline{సేనయోరుభయోర్మధ్యే} \\
 & \natline{రథం స్థాపయ మేఽచ్యుత ||}
\end{tabular}
\end{table}

\begin{table}[H]
\begin{tabular}{cl}
\textbf{1.22} & \natline{యావదేతాన్నిరీక్షేఽహం} \\
 & \natline{యోద్ధుకామానవస్థితాన్ |} \\
 & \natline{కైర్మయా సహ యోద్ధవ్యమ్} \\
 & \natline{అస్మిన్ రణసముద్యమే ||}
\end{tabular}
\end{table}

\begin{table}[H]
\begin{tabular}{cl}
\textbf{1.23} & \natline{యోత్స్యమానానవేక్షేఽహం} \\
 & \natline{య ఏతేఽత్ర సమాగతాః |} \\
 & \natline{ధార్తరాష్ట్రస్యదుర్బుద్ధేః} \\
 & \natline{యుద్ధే ప్రియచికీర్షవః ||}
\end{tabular}
\end{table}

\begin{table}[H]
\begin{tabular}{cl}
\textbf{1.24} & \natline{సఞ్జయ ఉవాచ} \\
 & \natline{ఏవముక్తో హృషీకేశః} \\
 & \natline{గుడాకేశేన భారత |} \\
 & \natline{సేనయోరుభయోర్మధ్యే} \\
 & \natline{స్థాపయిత్వా రథోత్తమమ్ ||}
\end{tabular}
\end{table}

\begin{table}[H]
\begin{tabular}{cl}
\textbf{1.25} & \natline{భీష్మద్రోణప్రముఖతః} \\
 & \natline{సర్వేషాం చ మహీక్షితామ్ |} \\
 & \natline{ఉవాచ పార్థ పశ్యైతాన్} \\
 & \natline{సమవేతాన్కురూనితి ||}
\end{tabular}
\end{table}

\begin{table}[H]
\begin{tabular}{cl}
\textbf{1.26} & \natline{తత్రాపశ్యత్స్థితాన్పార్థః} \\
 & \natline{పితౄనథ పితామహాన్ |} \\
 & \natline{ఆచార్యాన్మాతులాన్భ్రాతౄన్} \\
 & \natline{పుత్రాన్పౌత్రాన్సఖీంస్తథా ||}
\end{tabular}
\end{table}

\begin{table}[H]
\begin{tabular}{cl}
\textbf{1.27} & \natline{శ్వశురాన్సుహృదశ్చైవ} \\
 & \natline{సేనయోరుభయోరపి |} \\
 & \natline{తాన్సమీక్ష్య స కౌన్తేయః} \\
 & \natline{సర్వాన్బన్ధూనవస్థితాన్ ||}
\end{tabular}
\end{table}

\begin{table}[H]
\begin{tabular}{cl}
\textbf{1.28} & \natline{కృపయా పరయాఽఽవిష్టః} \\
 & \natline{విషీదన్నిదమబ్రవీత్} \\
 & \natline{అర్జున ఉవాచ |} \\
 & \natline{దృష్ట్వేమం స్వజనం కృష్ణ} \\
 & \natline{యుయుత్సుం సముపస్థితమ్ ||}
\end{tabular}
\end{table}

\begin{table}[H]
\begin{tabular}{cl}
\textbf{1.29} & \natline{సీదన్తి మమ గాత్రాణి} \\
 & \natline{ముఖం చ పరిశుష్యతి |} \\
 & \natline{వేపథుశ్చ శరీరే మే} \\
 & \natline{రోమహర్షశ్చ జాయతే ||}
\end{tabular}
\end{table}

\begin{table}[H]
\begin{tabular}{cl}
\textbf{1.30} & \natline{గాణ్డీవం స్రంసతే హస్తాత్} \\
 & \natline{త్వక్చైవ పరిదహ్యతే |} \\
 & \natline{న చ శక్నోమ్యవస్థాతుం} \\
 & \natline{భ్రమతీవ చ మే మనః ||}
\end{tabular}
\end{table}

\begin{table}[H]
\begin{tabular}{cl}
\textbf{1.31} & \natline{నిమిత్తాని చ పశ్యామి} \\
 & \natline{విపరీతాని కేశవ |} \\
 & \natline{న చ శ్రేయోఽనుపశ్యామి} \\
 & \natline{హత్వా స్వజనమాహవే ||}
\end{tabular}
\end{table}

\begin{table}[H]
\begin{tabular}{cl}
\textbf{1.32} & \natline{న కాఙ్క్షే విజయం కృష్ణ} \\
 & \natline{న చ రాజ్యం సుఖాని చ |} \\
 & \natline{కిం నో రాజ్యేన గోవిన్ద} \\
 & \natline{కిం భోగైర్జీవితేన వా ||}
\end{tabular}
\end{table}

\begin{table}[H]
\begin{tabular}{cl}
\textbf{1.33} & \natline{యేషామర్థే కాఙ్క్షితం నః} \\
 & \natline{రాజ్యం భోగాః సుఖాని చ |} \\
 & \natline{త ఇమేఽవస్థితా యుద్ధే} \\
 & \natline{ప్రాణాంస్త్యక్త్వా ధనాని చ ||}
\end{tabular}
\end{table}

\begin{table}[H]
\begin{tabular}{cl}
\textbf{1.34} & \natline{ఆచార్యాః పితరః పుత్రాః} \\
 & \natline{తథైవ చ పితామహాః |} \\
 & \natline{మాతులాః శ్వశురాః పౌత్రాః} \\
 & \natline{శ్యాలాః సమ్బన్ధినస్తథా ||}
\end{tabular}
\end{table}

\begin{table}[H]
\begin{tabular}{cl}
\textbf{1.35} & \natline{ఏతాన్న హన్తుమిచ్చామి} \\
 & \natline{ఘ్నతోఽపి మధుసూదన |} \\
 & \natline{అపి త్రైలోక్యరాజ్యస్య} \\
 & \natline{హేతోః కిం ను మహీకృతే ||}
\end{tabular}
\end{table}

\begin{table}[H]
\begin{tabular}{cl}
\textbf{1.36} & \natline{నిహత్య ధార్తరాష్ట్రాన్నః} \\
 & \natline{కా ప్రీతిః స్యాజ్జనార్దన |} \\
 & \natline{పాపమేవాశ్రయేదస్మాన్} \\
 & \natline{హత్వైతానాతతాయినః ||}
\end{tabular}
\end{table}

\begin{table}[H]
\begin{tabular}{cl}
\textbf{1.37} & \natline{తస్మాన్నార్హా వయం హన్తుం} \\
 & \natline{ధార్తరాష్ట్రాన్స్వబాన్ధవాన్ |} \\
 & \natline{స్వజనం హి కథం హత్వా} \\
 & \natline{సుఖినః స్యామ మాధవ ||}
\end{tabular}
\end{table}

\begin{table}[H]
\begin{tabular}{cl}
\textbf{1.38} & \natline{యద్యప్యేతే న పశ్యన్తి} \\
 & \natline{లోభోపహతచేతసః |} \\
 & \natline{కులక్షయకృతం దోషం} \\
 & \natline{మిత్రద్రోహే చ పాతకమ్ ||}
\end{tabular}
\end{table}

\begin{table}[H]
\begin{tabular}{cl}
\textbf{1.39} & \natline{కథం న జ్ఞేయమస్మాభిః} \\
 & \natline{పాపాదస్మాన్నివర్తితుమ్ |} \\
 & \natline{కులక్షయకృతం దోషం} \\
 & \natline{ప్రపశ్యద్భిర్జనార్దన ||}
\end{tabular}
\end{table}

\begin{table}[H]
\begin{tabular}{cl}
\textbf{1.40} & \natline{కులక్షయే ప్రణశ్యన్తి} \\
 & \natline{కులధర్మాః సనాతనాః |} \\
 & \natline{ధర్మే నష్టే కులం కృత్స్నమ్} \\
 & \natline{అధర్మోఽభిభవత్యుత ||}
\end{tabular}
\end{table}

\begin{table}[H]
\begin{tabular}{cl}
\textbf{1.41} & \natline{అధర్మాభిభవాత్కృష్ణ} \\
 & \natline{ప్రదుష్యన్తి కులస్త్రియః |} \\
 & \natline{స్త్రీషు దుష్టాసు వార్ష్ణేయ} \\
 & \natline{జాయతే వర్ణసఙ్కరః ||}
\end{tabular}
\end{table}

\begin{table}[H]
\begin{tabular}{cl}
\textbf{1.42} & \natline{సఙ్కరో నరకాయైవ} \\
 & \natline{కులఘ్నానాం కులస్య చ |} \\
 & \natline{పతన్తి పితరో హ్యేషాం} \\
 & \natline{లుప్తపిణ్డోదకక్రియాః ||}
\end{tabular}
\end{table}

\begin{table}[H]
\begin{tabular}{cl}
\textbf{1.43} & \natline{దోషైరేతైః కులఘ్నానాం} \\
 & \natline{వర్ణసఙ్కరకారకైః |} \\
 & \natline{ఉత్సాద్యన్తే జాతిధర్మాః} \\
 & \natline{కులధర్మాశ్చ శాశ్వతాః ||}
\end{tabular}
\end{table}

\begin{table}[H]
\begin{tabular}{cl}
\textbf{1.44} & \natline{ఉత్సన్నకులధర్మాణాం} \\
 & \natline{మనుష్యాణాం జనార్దన |} \\
 & \natline{నరకేఽనియతం వాసః} \\
 & \natline{భవతీత్యనుశుశ్రుమ ||}
\end{tabular}
\end{table}

\begin{table}[H]
\begin{tabular}{cl}
\textbf{1.45} & \natline{అహో బత మహత్పాపం} \\
 & \natline{కర్తుం వ్యవసితా వయమ్ |} \\
 & \natline{యద్రాజ్యసుఖలోభేన} \\
 & \natline{హన్తుం స్వజనముద్యతాః ||}
\end{tabular}
\end{table}

\begin{table}[H]
\begin{tabular}{cl}
\textbf{1.46} & \natline{యది మామప్రతీకారమ్} \\
 & \natline{అశస్త్రం శస్త్రపాణయః |} \\
 & \natline{ధార్తరాష్ట్రా రణే హన్యుః} \\
 & \natline{తన్మే క్షేమతరం భవేత్ ||}
\end{tabular}
\end{table}

\begin{table}[H]
\begin{tabular}{cl}
\textbf{1.47} & \natline{సఞ్జయ ఉవాచ} \\
 & \natline{ఏవముక్త్వాఽర్జునః సఙ్ఖ్యే} \\
 & \natline{రథోపస్థ ఉపావిశత్ |} \\
 & \natline{విసృజ్య సశరం చాపం} \\
 & \natline{శోకసంవిగ్నమానసః ||}
\end{tabular}
\end{table}


% \end{multicols}

\chapter{Sāṅkhya Yoga}
% \begin{multicols}{2}
\subsection*{2.0}
\begin{table}[H]
\centering
\begin{tabular}{ll}
\natline{ఓం శ్రీ పరమాత్మనే నమః} & \romline{oṃ śrī paramātmane namaḥ} \\
\natline{అథ ద్వితీయోఽధ్యాయః} & \romline{atha dvitīyo'dhyāyaḥ} \\
\natline{సాఙ్ఖ్యయోగః} & \romline{sāṅkhya-yogaḥ}
\end{tabular}
\end{table}

\subsection*{2.1}
\begin{table}[H]
\centering
\begin{tabular}{ll}
\natline{సంజయ ఉవాచ} & \romline{saṃjaya uvāca} \\
\natline{తం తథా కృపయావిష్టమ్} & \romline{taṃ tathā kṛpayāviṣṭam} \\
\natline{అశ్రుపూర్ణాకులేక్షణమ్} & \romline{aśru-pūrṇākulekṣaṇam} \\
\natline{విషీదంతమిదం వాక్యమ్} & \romline{viṣīdaṃtamidaṃ vākyam} \\
\natline{ఉవాచ మధుసూదనః} & \romline{uvāca madhusūdanaḥ}
\end{tabular}
\end{table}

\subsection*{2.2}
\begin{table}[H]
\centering
\begin{tabular}{ll}
\natline{శ్రీ భగవానువాచ} & \romline{śrī bhagavān-uvāca} \\
\natline{కుతస్త్వా కశ్మలమిదం} & \romline{kutastvā kaśmalamidaṃ} \\
\natline{విషమే సముపస్థితమ్} & \romline{viṣame samupasthitam} \\
\natline{అనార్యజుష్టమస్వర్గ్యమ్} & \romline{anārya-juṣṭamasvargyam} \\
\natline{అకీర్తికరమర్జున} & \romline{akīrti-karam-arjuna}
\end{tabular}
\end{table}

\subsection*{2.3}
\begin{table}[H]
\centering
\begin{tabular}{ll}
\natline{క్లైబ్యం మా స్మ గమః పార్థ} & \romline{klaibyaṃ mā sma gamaḥ pārtha} \\
\natline{నైతత్త్వయ్యుపపద్యతే} & \romline{naitat-tvayyupapadyate} \\
\natline{క్షుద్రం హృదయదౌర్బల్యం} & \romline{kṣudraṃ hṛdaya-daurbalyaṃ} \\
\natline{త్యక్త్వోత్తిష్ఠ పరంతప} & \romline{tyaktvottiṣṭha paraṃtapa}
\end{tabular}
\end{table}

\subsection*{2.4}
\begin{table}[H]
\centering
\begin{tabular}{ll}
\natline{అర్జున ఉవాచ} & \romline{arjuna uvāca} \\
\natline{కథం భీశ్మమహం సంఖ్యే} & \romline{kathaṃ bhīśmamahaṃ saṃkhye} \\
\natline{ద్రోణం చ మధుసూదన} & \romline{droṇaṃ ca madhusūdana} \\
\natline{ఇశుభిః ప్రతియోత్స్యామి} & \romline{iśubhiḥ pratiyotsyāmi} \\
\natline{పూజార్హావరిసూదన} & \romline{pūjārhāvarisūdana}
\end{tabular}
\end{table}

\subsection*{2.5}
\begin{table}[H]
\centering
\begin{tabular}{ll}
\natline{గురూనహత్వా హి మహానుభావాన్} & \romline{gurūnahatvā hi mahānubhāvān} \\
\natline{శ్రేయో భోక్తుం భైక్ష్యమపీహ లోకే} & \romline{śreyo bhoktuṃ bhaikṣyamapīha loke} \\
\natline{హత్వార్థకామంస్తు గురూనిహైవ} & \romline{hatvārtha-kāmaṃstu gurūnihaiva} \\
\natline{భుంజీయ భోగాన్ రుధిరప్రదిగ్ధాన్} & \romline{bhuṃjīya bhogān rudhira-pradigdhān}
\end{tabular}
\end{table}

\subsection*{2.6}
\begin{table}[H]
\centering
\begin{tabular}{ll}
\natline{న చైతద్విద్మః కతరన్నో గరీయః} & \romline{na caitadvidmaḥ kataranno garīyaḥ} \\
\natline{యద్వా జయేమ యది వా నో జయేయుః} & \romline{yadvā jayema yadi vā no jayeyuḥ} \\
\natline{యానేవ హత్వా న జిజీవిషామః} & \romline{yāneva hatvā na jijīviṣāmaḥ} \\
\natline{తేఽవస్థితాః ప్రముఖే ధార్తరాష్ట్రాః} & \romline{te'vasthitāḥ pramukhe dhārtarāṣṭrāḥ}
\end{tabular}
\end{table}

\subsection*{2.7}
\begin{table}[H]
\centering
\begin{tabular}{ll}
\natline{కార్పన్యదోషోపహతస్వభావః} & \romline{kārpanya-doṣopahata-svabhāvaḥ} \\
\natline{పృచ్ఛామి త్వాం ధర్మసమ్మూఢచేతాః} & \romline{pṛcchāmi tvāṃ dharma-sammūḍha-cetāḥ} \\
\natline{యచ్ఛ్రేయః స్యాన్నిశ్చితం బ్రూహి తన్మే} & \romline{yacchreyaḥ syānniścitaṃ brūhi tanme} \\
\natline{శిష్యస్తేఽహం శాధి మాం త్వాం ప్రపన్నమ్} & \romline{śiṣyaste'haṃ śādhi māṃ tvāṃ prapannam}
\end{tabular}
\end{table}

\subsection*{2.8}
\begin{table}[H]
\centering
\begin{tabular}{ll}
\natline{న హిప్రపశ్యామి మమాపనుద్యాద్} & \romline{na hi·prapaśyāmi mamāpanudyād} \\
\natline{యచ్ఛోకముచ్ఛోషణమిన్ద్రియాణామ్} & \romline{yacchokam-ucchoṣaṇam-indriyāṇām} \\
\natline{అవాప్య భూమావసపత్నమృద్ధం} & \romline{avāpya bhūmāv-asapatnamṛddhaṃ} \\
\natline{రాజ్యం సురాణామపి చాధిపత్యమ్} & \romline{rājyaṃ surāṇāmapi cādhipatyam}
\end{tabular}
\end{table}

\subsection*{2.9}
\begin{table}[H]
\centering
\begin{tabular}{ll}
\natline{సంజయ ఉవాచ} & \romline{saṃjaya uvāca} \\
\natline{ఏవముక్త్వా హృషీకేశం} & \romline{evam-uktvā hṛṣīkeśaṃ} \\
\natline{గుడాకేశః పరన్తపః} & \romline{guḍākeśaḥ parantapaḥ} \\
\natline{న యోత్స్య ఇతి గోవిందమ్} & \romline{na yotsya iti goviṃdam} \\
\natline{ఉక్త్వా తూష్ణీమ్ బభూవ హ} & \romline{uktvā tūṣṇīm babhūva ha}
\end{tabular}
\end{table}

\subsection*{2.10}
\begin{table}[H]
\centering
\begin{tabular}{ll}
\natline{తమువాచ హృషీకేశః} & \romline{tam-uvāca hṛṣīkeśaḥ} \\
\natline{ప్రహసన్నివ భారత} & \romline{prahasanniva bhārata} \\
\natline{సేనయోరుభయోర్మధ్యే} & \romline{senayorubhayor-madhye} \\
\natline{విషీదంతమిదం వచః} & \romline{viṣīdaṃtam-idaṃ vacaḥ}
\end{tabular}
\end{table}

\subsection*{2.11}
\begin{table}[H]
\centering
\begin{tabular}{ll}
\natline{శ్రీ భగవానువాచ} & \romline{śrī bhagavān-uvāca} \\
\natline{అశోచ్యానన్వశోచస్త్వం} & \romline{aśocyān-anvaśocas-tvaṃ} \\
\natline{ప్రజ్ఞావాదాంశ్చ భాషసే} & \romline{prajñā-vādāṃśca bhāṣase} \\
\natline{గతాసూనగతాసూంశ్చ} & \romline{gatāsūn-agatāsūṃś-ca} \\
\natline{నానుశోచన్తి పణ్డితాః} & \romline{nānuśocanti paṇḍitāḥ}
\end{tabular}
\end{table}

\subsection*{2.12}
\begin{table}[H]
\centering
\begin{tabular}{ll}
\natline{న త్వేవాహం జాతు నాసం} & \romline{na tvevāhaṃ jātu nāsaṃ} \\
\natline{న త్వం నేమే జనాధిపాః} & \romline{na tvaṃ neme janādhipāḥ} \\
\natline{న చైవ న భవిష్యామః} & \romline{na caiva na bhaviṣyāmaḥ} \\
\natline{సర్వే వయమతః పరమ్} & \romline{sarve vayamataḥ param}
\end{tabular}
\end{table}

\subsection*{2.13}
\begin{table}[H]
\centering
\begin{tabular}{ll}
\natline{దేహినోఽస్మిన్ యథా దేహే} & \romline{dehino'smin yathā dehe} \\
\natline{కౌమారం యౌవనం జరా} & \romline{kaumāraṃ yauvanaṃ jarā} \\
\natline{తథా దేహాంతరప్రాప్తిః} & \romline{tathā dehāṃtara-prāptiḥ} \\
\natline{ధీరస్తత్ర న ముహ్యతి} & \romline{dhīras-tatra na muhyati}
\end{tabular}
\end{table}

\subsection*{2.14}
\begin{table}[H]
\centering
\begin{tabular}{ll}
\natline{మాత్రాస్పర్శాస్తు కౌంతేయ} & \romline{mātrā-sparśās-tu kauṃteya} \\
\natline{శీతోష్ణసుఖదుఃఖదాః} & \romline{śītoṣṇa-sukha-duḥkha-dāḥ} \\
\natline{ఆగమాపాయినోఽనిత్యాః} & \romline{āgamāpāyino'nityāḥ} \\
\natline{తాంస్తితిక్షస్వ భారత} & \romline{tāṃs-titikṣasva bhārata}
\end{tabular}
\end{table}

\subsection*{2.15}
\begin{table}[H]
\centering
\begin{tabular}{ll}
\natline{యం హి న వ్యథయంత్యేతే} & \romline{yaṃ hi na vyathayaṃtyete} \\
\natline{పురుషం పురుషర్షభ} & \romline{puruṣaṃ puruṣarṣabha} \\
\natline{సమదుఃఖసుఖం ధీరం} & \romline{sama-duḥkha-sukhaṃ dhīraṃ} \\
\natline{సోఽమృతత్వాయ కల్పతే} & \romline{so'mṛtatvāya kalpate}
\end{tabular}
\end{table}

\subsection*{2.16}
\begin{table}[H]
\centering
\begin{tabular}{ll}
\natline{నాసతో విద్యతే భావః} & \romline{nāsato vidyate bhāvaḥ} \\
\natline{నాభావో విద్యతే సతః} & \romline{nābhāvo vidyate sataḥ} \\
\natline{ఉభయోరపి దృష్తోఽన్తః} & \romline{ubhayorapi dṛṣto'ntaḥ} \\
\natline{త్వనయోస్తత్త్వదర్శిభిః} & \romline{tvanayos-tattva-darśibhiḥ}
\end{tabular}
\end{table}

\subsection*{2.17}
\begin{table}[H]
\centering
\begin{tabular}{ll}
\natline{అవినాశి తు తద్విద్ధి} & \romline{avināśi tu tadviddhi} \\
\natline{యేన సర్వమిదం తతమ్} & \romline{yena sarvamidaṃ tatam} \\
\natline{వినాశమవ్యయస్యాస్య} & \romline{vināśam-avyayasyāsya} \\
\natline{న కశ్చిత్కర్తుమర్హతి} & \romline{na kaścit-kartum-arhati}
\end{tabular}
\end{table}

\subsection*{2.18}
\begin{table}[H]
\centering
\begin{tabular}{ll}
\natline{అంతవన్త ఇమే దేహాః} & \romline{aṃtavanta ime dehāḥ} \\
\natline{నిత్యస్యోక్తాః శరీరిణః} & \romline{nityasyoktāḥ śarīriṇaḥ} \\
\natline{అనాశినోఽప్రమేయస్య} & \romline{anāśino'prameyasya} \\
\natline{తస్మాద్యుధ్యస్వ భారత} & \romline{tasmād-yudhyasva bhārata}
\end{tabular}
\end{table}

\subsection*{2.19}
\begin{table}[H]
\centering
\begin{tabular}{ll}
\natline{య ఏనం వేత్తి హన్తారం} & \romline{ya enaṃ vetti hantāraṃ} \\
\natline{యశ్చైనం మన్యతే హతం} & \romline{yaścainaṃ manyate hataṃ} \\
\natline{ఉభౌ తౌ న విజానీతః} & \romline{ubhau tau na vijānītaḥ} \\
\natline{నాయం హన్తి న హన్యతే} & \romline{nāyaṃ hanti na hanyate}
\end{tabular}
\end{table}

\subsection*{2.20}
\begin{table}[H]
\centering
\begin{tabular}{ll}
\natline{న జాయతే మ్రియతే వా కదాచిత్} & \romline{na jāyate mriyate vā kadācit} \\
\natline{నాయం భూత్వా భవితా వా న భూయః} & \romline{nāyaṃ bhūtvā bhavitā vā na bhūyaḥ} \\
\natline{అజో నిత్యః శాశ్వతోఽయం పురాణః} & \romline{ajo nityaḥ śāśvato'yaṃ purāṇaḥ} \\
\natline{న హన్యతే హన్యమానే శరీరే} & \romline{na hanyate hanyamāne śarīre}
\end{tabular}
\end{table}

\subsection*{2.21}
\begin{table}[H]
\centering
\begin{tabular}{ll}
\natline{వేదావినాశినం నిత్యం} & \romline{vedāvināśinaṃ nityaṃ} \\
\natline{య ఏనమజమవ్యయమ్} & \romline{ya enamajam-avyayam} \\
\natline{కథం స పురుషః పార్థ} & \romline{kathaṃ sa puruṣaḥ pārtha} \\
\natline{కం ఘాతయతి హంతి కమ్} & \romline{kaṃ ghātayati haṃti kam}
\end{tabular}
\end{table}

\subsection*{2.22}
\begin{table}[H]
\centering
\begin{tabular}{ll}
\natline{వాసాంసి జీర్ణాని యథా విహాయ} & \romline{vāsāṃsi jīrṇāni yathā vihāya} \\
\natline{నవాని గృహ్ణాతి నరోఽపరాణి} & \romline{navāni gṛhṇāti naro'parāṇi} \\
\natline{తథా శరీరాణి విహాయ జీర్ణాని} & \romline{tathā śarīrāṇi vihāya jīrṇāni} \\
\natline{అన్యాని సంయాతి నవాని దేహీ} & \romline{anyāni saṃyāti navāni dehī}
\end{tabular}
\end{table}

\subsection*{2.23}
\begin{table}[H]
\centering
\begin{tabular}{ll}
\natline{నైనం ఛిన్దన్తి శస్త్రాణి} & \romline{nainaṃ chindanti śastrāṇi} \\
\natline{నైనం దహతి పావకః} & \romline{nainaṃ dahati pāvakaḥ} \\
\natline{న చైనం క్లేదయన్త్యాపః} & \romline{na cainaṃ kledayantyāpaḥ} \\
\natline{న శోషయతి మారుతః} & \romline{na śoṣayati mārutaḥ}
\end{tabular}
\end{table}

\subsection*{2.24}
\begin{table}[H]
\centering
\begin{tabular}{ll}
\natline{అచ్ఛేద్యోఽయమ్ అదాహ్యోఽయమ్} & \romline{acchedyo'yam adāhyo'yam} \\
\natline{అక్లేద్యోఽశోష్య ఏవ చ} & \romline{akledyo'śoṣya eva ca} \\
\natline{నిత్యః సర్వగతః స్థాణుః} & \romline{nityaḥ sarva-gataḥ sthāṇuḥ} \\
\natline{అచలోఽయం సనాతనః} & \romline{acalo'yaṃ sanātanaḥ}
\end{tabular}
\end{table}

\subsection*{2.25}
\begin{table}[H]
\centering
\begin{tabular}{ll}
\natline{అవ్యక్తోఽయమ్ అచిన్త్యోఽయమ్} & \romline{avyakto'yam acintyo'yam} \\
\natline{అవికార్యోఽయముచ్యతే} & \romline{avikāryo'yamucyate} \\
\natline{తస్మాదేవం విదిత్వైనం} & \romline{tasmādevaṃ viditvainaṃ} \\
\natline{నానుశోచితుమర్హసి} & \romline{nānuśocitumarhasi}
\end{tabular}
\end{table}

\subsection*{2.26}
\begin{table}[H]
\centering
\begin{tabular}{ll}
\natline{అథ చైనం నిత్యజాతం} & \romline{atha cainaṃ nityajātaṃ} \\
\natline{నిత్యం వా మన్యసే మృతమ్} & \romline{nityaṃ vā manyase mṛtam} \\
\natline{తథాఽపి త్వం మహాబాహో} & \romline{tathā'pi tvaṃ mahābāho} \\
\natline{నైవం శోచితుమర్హసి} & \romline{naivaṃ śocitumarhasi}
\end{tabular}
\end{table}

\subsection*{2.27}
\begin{table}[H]
\centering
\begin{tabular}{ll}
\natline{జాతస్య హిధ్రువో మృత్యుః} & \romline{jātasya hi·dhruvo mṛtyuḥ} \\
\natline{ధ్రువం జన్మ మృతస్య చ} & \romline{dhruvaṃ janma mṛtasya ca} \\
\natline{తస్మాదపరిహార్యేఽర్థే} & \romline{tasmādaparihārye'rthe} \\
\natline{న త్వం శోచితుమర్హసి} & \romline{na tvaṃ śocitumarhasi}
\end{tabular}
\end{table}

\subsection*{2.28}
\begin{table}[H]
\centering
\begin{tabular}{ll}
\natline{అవ్యక్తాదీని భూతాని} & \romline{avyaktādīni bhūtāni} \\
\natline{వ్యక్తమధ్యాని భారత} & \romline{vyaktamadhyāni bhārata} \\
\natline{అవ్యక్తనిధనాన్యేవ} & \romline{avyakta-nidhanānyeva} \\
\natline{తత్ర కా పరిదేవనా} & \romline{tatra kā paridevanā}
\end{tabular}
\end{table}

\subsection*{2.29}
\begin{table}[H]
\centering
\begin{tabular}{ll}
\natline{ఆశ్చర్యవత్ పశ్యతి కశ్చిదేనమ్} & \romline{āścaryavat paśyati kaścidenam} \\
\natline{ఆశ్చర్యవద్ వదతి తథైవ cఆన్యః} & \romline{āścaryavad vadati tathaiva cānyaḥ} \\
\natline{ఆశ్చర్యవచ్చైనమన్యః శృణోతి} & \romline{āścaryavaccainamanyaḥ śṛṇoti} \\
\natline{శ్రుత్వాప్యేనం వేద న చైవ కశ్చిత్} & \romline{śrutvāpyenaṃ veda na caiva kaścit}
\end{tabular}
\end{table}

\subsection*{2.30}
\begin{table}[H]
\centering
\begin{tabular}{ll}
\natline{దేహీ నిత్యమవధ్యోఽయం} & \romline{dehī nityamavadhyo'yaṃ} \\
\natline{దేహే సర్వస్య భారత} & \romline{dehe sarvasya bhārata} \\
\natline{తస్మాత్సర్వాణి భూతాని} & \romline{tasmātsarvāṇi bhūtāni} \\
\natline{న త్వం శోచితుమర్హసి} & \romline{na tvaṃ śocitumarhasi}
\end{tabular}
\end{table}


% \end{multicols}

\chapter{Karma Yoga}
% \begin{multicols}{2}
\begin{table}[H]
\begin{tabular}{cl}
\textbf{3.0} & \romline{oṃ śrī paramātmane namaḥ} \\
 & \romline{atha tṛtīyo'dhyāyaḥ} \\
 & \romline{karma-yogaḥ}
\end{tabular}
\end{table}

\begin{table}[H]
\begin{tabular}{cl}
\textbf{3.1} & \romline{arjuna uvāca} \\
 & \romline{jyāyasī cetkarmaṇaste} \\
 & \romline{matā buddhirjanārdana |} \\
 & \romline{tatkiṃ karmaṇi ghore māṃ} \\
 & \romline{niyojayasi keśava ||}
\end{tabular}
\end{table}

\begin{table}[H]
\begin{tabular}{cl}
\textbf{3.2} & \romline{vyāmiśreṇeva vākyena} \\
 & \romline{buddhiṃ mohayasīva me |} \\
 & \romline{tadekaṃ vada niścitya} \\
 & \romline{yena śreyo'hamāpnuyām ||}
\end{tabular}
\end{table}

\begin{table}[H]
\begin{tabular}{cl}
\textbf{3.3} & \romline{srī bhagavānuvāca} \\
 & \romline{loke'smin-dvividhā niṣṭhā} \\
 & \romline{purā proktā mayā'nagha |} \\
 & \romline{jñānayogena sāṅkhyānāṃ} \\
 & \romline{karmayogena yoginām ||}
\end{tabular}
\end{table}

\begin{table}[H]
\begin{tabular}{cl}
\textbf{3.4} & \romline{na karmaṇāmanārambhāt} \\
 & \romline{naiṣkarmyaṃ puruṣo'śnute |} \\
 & \romline{na ca sannyasanādeva} \\
 & \romline{siddhiṃ samadhigacchati ||}
\end{tabular}
\end{table}

\begin{table}[H]
\begin{tabular}{cl}
\textbf{3.5} & \romline{na hi kaścit-kṣaṇamapi} \\
 & \romline{jātu tiṣṭhatya-karmakṛt |} \\
 & \romline{kāryate hyavaśaḥ karma} \\
 & \romline{sarvaḥ prakṛti-jairguṇaiḥ ||}
\end{tabular}
\end{table}

\begin{table}[H]
\begin{tabular}{cl}
\textbf{3.6} & \romline{karmendriyāṇi saṃyamya} \\
 & \romline{ya āste manasā smaran |} \\
 & \romline{indriyārthān-vimūḍhātmā} \\
 & \romline{mithyācāraḥ sa ucyate ||}
\end{tabular}
\end{table}

\begin{table}[H]
\begin{tabular}{cl}
\textbf{3.7} & \romline{yastvindriyāṇi manasā} \\
 & \romline{niyamyārabhate'rjuna |} \\
 & \romline{karmendriyaiḥ karmayogam} \\
 & \romline{asaktaḥ sa viśiṣyate ||}
\end{tabular}
\end{table}

\begin{table}[H]
\begin{tabular}{cl}
\textbf{3.8} & \romline{niyataṃ kuru karma tvaṃ} \\
 & \romline{karma jyāyo hyakarmaṇaḥ |} \\
 & \romline{śarīra-yātrā'pi ca te} \\
 & \romline{na prasiddhyeda-karmaṇaḥ ||}
\end{tabular}
\end{table}

\begin{table}[H]
\begin{tabular}{cl}
\textbf{3.9} & \romline{yajñārthāt-karmaṇo'nyatra} \\
 & \romline{loko'yaṃ karmabandhanaḥ |} \\
 & \romline{tadarthaṃ karma kaunteya} \\
 & \romline{muktasaṅgaḥ samācara ||}
\end{tabular}
\end{table}

\begin{table}[H]
\begin{tabular}{cl}
\textbf{3.10} & \romline{sahayajñāḥ prajāḥ sṛṣṭvā} \\
 & \romline{purovāca·prajāpatiḥ |} \\
 & \romline{anena·prasaviṣyadhvaṃ} \\
 & \romline{eṣa vo'stviṣṭa-kāmadhuk ||}
\end{tabular}
\end{table}

\begin{table}[H]
\begin{tabular}{cl}
\textbf{3.11} & \romline{devān-bhāvayatā'nena} \\
 & \romline{te devā bhāvayantu vaḥ |} \\
 & \romline{parasparaṃ bhāvayantaḥ} \\
 & \romline{śreyaḥ paramavāpsyatha ||}
\end{tabular}
\end{table}

\begin{table}[H]
\begin{tabular}{cl}
\textbf{3.12} & \romline{iṣṭānbhogānhi vo devāḥ} \\
 & \romline{dāsyante yajñabhāvitāḥ |} \\
 & \romline{tairdattān-apradāyaibhyaḥ} \\
 & \romline{yo bhuṅkte stena eva saḥ ||}
\end{tabular}
\end{table}

\begin{table}[H]
\begin{tabular}{cl}
\textbf{3.13} & \romline{yajñaśiṣṭāśinaḥ santaḥ} \\
 & \romline{mucyante sarvakilbiṣaiḥ |} \\
 & \romline{bhuñjate te tvaghaṃ pāpāḥ} \\
 & \romline{ye pacantyātmakāraṇāt ||}
\end{tabular}
\end{table}

\begin{table}[H]
\begin{tabular}{cl}
\textbf{3.14} & \romline{annādbhavanti bhūtāni} \\
 & \romline{parjanyādannasambhavaḥ |} \\
 & \romline{yajñādbhavati parjanyaḥ} \\
 & \romline{yajñaḥ karmasamudbhavaḥ ||}
\end{tabular}
\end{table}

\begin{table}[H]
\begin{tabular}{cl}
\textbf{3.15} & \romline{karma brahmodbhavaṃ viddhi} \\
 & \romline{brahmākṣarasamudbhavam |} \\
 & \romline{tasmātsarvagataṃ brahma} \\
 & \romline{nityaṃ yajñe pratiṣṭhitam ||}
\end{tabular}
\end{table}

\begin{table}[H]
\begin{tabular}{cl}
\textbf{3.16} & \romline{evaṃ pravartitaṃ cakraṃ} \\
 & \romline{nānuvartayatīha yaḥ |} \\
 & \romline{aghāyurindriyārāmaḥ} \\
 & \romline{moghaṃ pārtha sa jīvati ||}
\end{tabular}
\end{table}

\begin{table}[H]
\begin{tabular}{cl}
\textbf{3.17} & \romline{yastvātmaratireva syāt} \\
 & \romline{ātmatṛptaśca mānavaḥ |} \\
 & \romline{ātmanyeva ca santuṣṭaḥ} \\
 & \romline{tasya kāryaṃ na vidyate ||}
\end{tabular}
\end{table}

\begin{table}[H]
\begin{tabular}{cl}
\textbf{3.18} & \romline{naiva tasya kṛtenārthaḥ} \\
 & \romline{nākṛteneha kaścana |} \\
 & \romline{na cāsya sarvabhūteṣu} \\
 & \romline{kaścidarthavyapāśrayaḥ ||}
\end{tabular}
\end{table}

\begin{table}[H]
\begin{tabular}{cl}
\textbf{3.19} & \romline{tasmādasaktaḥ satataṃ} \\
 & \romline{kāryaṃ karma samācara |} \\
 & \romline{asakto hyācarankarma} \\
 & \romline{paramāpnoti pūruṣaḥ ||}
\end{tabular}
\end{table}

\begin{table}[H]
\begin{tabular}{cl}
\textbf{3.20} & \romline{karmaṇaiva hi saṃsiddhiṃ} \\
 & \romline{āsthitā janakādayaḥ |} \\
 & \romline{lokasaṅgrahamevāpi} \\
 & \romline{sampaśyankartumarhasi ||}
\end{tabular}
\end{table}


% \end{multicols}

\chapter{Jñāna Yoga}
% \begin{multicols}{2}
\begin{table}[H]
\begin{tabular}{cl}
\textbf{4.0} & \natline{ఓం శ్రీ పరమాత్మనే నమః} \\
 & \natline{అథ చతుర్థోఽధ్యాయః} \\
 & \natline{జ్ఞానయోగః}
\end{tabular}
\end{table}

\begin{table}[H]
\begin{tabular}{cl}
\textbf{4.1} & \natline{శ్రీ భగవానువాచ} \\
 & \natline{ఇమం వివస్వతే యోగం} \\
 & \natline{ప్రోక్తవానహమవ్యయమ్ |} \\
 & \natline{వివస్వాన్మనవే ప్రాహ} \\
 & \natline{మనురిక్ష్వాకవేఽబ్రవీత్ ||}
\end{tabular}
\end{table}

\begin{table}[H]
\begin{tabular}{cl}
\textbf{4.2} & \natline{ఏవం పరమ్పరాప్రాప్తమ్} \\
 & \natline{ఇమం రాజర్షయో విదుః |} \\
 & \natline{స కాలేనేహ మహతా} \\
 & \natline{యోగో నష్టః పరన్తప ||}
\end{tabular}
\end{table}

\begin{table}[H]
\begin{tabular}{cl}
\textbf{4.3} & \natline{స ఏవాయం మయా తేఽద్య} \\
 & \natline{యోగః ప్రోక్తః పురాతనః |} \\
 & \natline{భక్తోఽసి మే సఖా చేతి} \\
 & \natline{రహస్యం హ్యేతదుత్తమమ్ ||}
\end{tabular}
\end{table}

\begin{table}[H]
\begin{tabular}{cl}
\textbf{4.4} & \natline{అర్జున ఉవాచ} \\
 & \natline{అపరం భవతో జన్మ} \\
 & \natline{పరం జన్మ వివస్వతః |} \\
 & \natline{కథమేతద్విజానీయాం} \\
 & \natline{త్వమాదౌ ప్రోక్తవానితి ||}
\end{tabular}
\end{table}

\begin{table}[H]
\begin{tabular}{cl}
\textbf{4.5} & \natline{శ్రీ భగవానువాచ} \\
 & \natline{బహూని మే వ్యతీతాని} \\
 & \natline{జన్మాని తవ చార్జున |} \\
 & \natline{తాన్యహం వేద సర్వాణి} \\
 & \natline{న త్వం వేత్థ పరన్తప ||}
\end{tabular}
\end{table}

\begin{table}[H]
\begin{tabular}{cl}
\textbf{4.6} & \natline{అజోఽపి సన్నవ్యయాత్మా} \\
 & \natline{భూతానామీశ్వరోఽపి సన్ |} \\
 & \natline{ప్రకృతిం స్వామధిష్ఠాయ} \\
 & \natline{సమ్భవామ్యాత్మమాయయా ||}
\end{tabular}
\end{table}

\begin{table}[H]
\begin{tabular}{cl}
\textbf{4.7} & \natline{యదా యదా హి ధర్మస్య} \\
 & \natline{గ్లానిర్భవతి భారత |} \\
 & \natline{అభ్యుత్థానమధర్మస్య} \\
 & \natline{తదాఽఽత్మానం సృజామ్యహమ్ ||}
\end{tabular}
\end{table}

\begin{table}[H]
\begin{tabular}{cl}
\textbf{4.8} & \natline{పరిత్రాణాయ సాధూనాం} \\
 & \natline{వినాశాయ చ దుష్కృతామ్ |} \\
 & \natline{ధర్మసంస్థాపనార్థాయ} \\
 & \natline{సమ్భవామి యుగే యుగే ||}
\end{tabular}
\end{table}

\begin{table}[H]
\begin{tabular}{cl}
\textbf{4.9} & \natline{జన్మ కర్మ చ మే దివ్యమ్} \\
 & \natline{ఏవం యో వేత్తి తత్త్వతః |} \\
 & \natline{త్యక్త్వా దేహం పునర్జన్మ} \\
 & \natline{నైతి మామేతి సోఽర్జున ||}
\end{tabular}
\end{table}

\begin{table}[H]
\begin{tabular}{cl}
\textbf{4.10} & \natline{వీతరాగభయక్రోధాః} \\
 & \natline{మన్మయా మాముపాశ్రితాః |} \\
 & \natline{బహవో జ్ఞానతపసా} \\
 & \natline{పూతా మద్భావమాగతాః ||}
\end{tabular}
\end{table}

\begin{table}[H]
\begin{tabular}{cl}
\textbf{4.11} & \natline{యే యథా మాం ప్రపద్యన్తే} \\
 & \natline{తాంస్తథైవ భజామ్యహమ్ |} \\
 & \natline{మమ వర్త్మానువర్తన్తే} \\
 & \natline{మనుష్యాః పార్థ సర్వశః ||}
\end{tabular}
\end{table}

\begin{table}[H]
\begin{tabular}{cl}
\textbf{4.12} & \natline{కాఙ్క్షన్తః కర్మణాం సిద్ధిం} \\
 & \natline{యజన్త ఇహ దేవతాః |} \\
 & \natline{క్షిప్రం హి మానుషే లోకే} \\
 & \natline{సిద్ధిర్భవతి కర్మజా ||}
\end{tabular}
\end{table}

\begin{table}[H]
\begin{tabular}{cl}
\textbf{4.13} & \natline{చాతుర్వర్ణ్యం మయా సృష్టం} \\
 & \natline{గుణకర్మవిభాగశః |} \\
 & \natline{తస్య కర్తారమపి మాం} \\
 & \natline{విద్ధ్యకర్తారమవ్యయమ్ ||}
\end{tabular}
\end{table}

\begin{table}[H]
\begin{tabular}{cl}
\textbf{4.14} & \natline{న మాం కర్మాణి లిమ్పన్తి} \\
 & \natline{న మే కర్మఫలే స్పృహా |} \\
 & \natline{ఇతి మాం యోఽభిజానాతి} \\
 & \natline{కర్మభిర్న స బధ్యతే ||}
\end{tabular}
\end{table}

\begin{table}[H]
\begin{tabular}{cl}
\textbf{4.15} & \natline{ఏవం జ్ఞాత్వా కృతం కర్మ} \\
 & \natline{పూర్వైరపి ముముక్షుభిః |} \\
 & \natline{కురు కర్మైవ తస్మాత్త్వం} \\
 & \natline{పూర్వైః పూర్వతరం కృతమ్ ||}
\end{tabular}
\end{table}

\begin{table}[H]
\begin{tabular}{cl}
\textbf{4.16} & \natline{కిం కర్మ కిమకర్మేతి} \\
 & \natline{కవయోఽప్యత్ర మోహితాః |} \\
 & \natline{తత్తే కర్మ ప్రవక్ష్యామి} \\
 & \natline{యజ్జ్ఞాత్వా మోక్ష్యసేఽశుభాత్ ||}
\end{tabular}
\end{table}

\begin{table}[H]
\begin{tabular}{cl}
\textbf{4.17} & \natline{కర్మణో హ్యపి బోద్ధవ్యం} \\
 & \natline{బోద్ధవ్యం చ వికర్మణః |} \\
 & \natline{అకర్మణశ్చ బోద్ధవ్యం} \\
 & \natline{గహనా కర్మణో గతిః ||}
\end{tabular}
\end{table}

\begin{table}[H]
\begin{tabular}{cl}
\textbf{4.18} & \natline{కర్మణ్యకర్మ యః పశ్యేత్} \\
 & \natline{అకర్మణిచ కర్మ యః |} \\
 & \natline{స బుద్ధిమాన్మనుష్యేషు} \\
 & \natline{స యుక్తః కృత్స్నకర్మకృత్ ||}
\end{tabular}
\end{table}

\begin{table}[H]
\begin{tabular}{cl}
\textbf{4.19} & \natline{యస్య సర్వే సమారమ్భాః} \\
 & \natline{కామసఙ్కల్పవర్జితాః |} \\
 & \natline{జ్ఞానాగ్నిదగ్ధకర్మాణం} \\
 & \natline{తమాహుః పణ్డితం బుధాః ||}
\end{tabular}
\end{table}

\begin{table}[H]
\begin{tabular}{cl}
\textbf{4.20} & \natline{త్యక్త్వా కర్మఫలాసఙ్గం} \\
 & \natline{నిత్యతృప్తో నిరాశ్రయః |} \\
 & \natline{కర్మణ్యభిప్రవృత్తోఽపి} \\
 & \natline{నైవ కిఞ్చిత్కరోతి సః ||}
\end{tabular}
\end{table}

\begin{table}[H]
\begin{tabular}{cl}
\textbf{4.21} & \natline{నిరాశీర్యతచిత్తాత్మా} \\
 & \natline{త్యక్తసర్వ పరిగ్రహః |} \\
 & \natline{శారీరం కేవలం కర్మ} \\
 & \natline{కుర్వన్నాప్నోతి కిల్బిషమ్ ||}
\end{tabular}
\end{table}

\begin{table}[H]
\begin{tabular}{cl}
\textbf{4.22} & \natline{యదృచ్ఛాలాభసన్తుష్టః} \\
 & \natline{ద్వన్ద్వాతీతో విమత్సరః |} \\
 & \natline{సమః సిద్ధావసిద్ధౌ చ} \\
 & \natline{కృత్వాపి న నిబధ్యతే ||}
\end{tabular}
\end{table}

\begin{table}[H]
\begin{tabular}{cl}
\textbf{4.23} & \natline{గతసఙ్గస్య ముక్తస్య} \\
 & \natline{జ్ఞానావస్థితచేతసః |} \\
 & \natline{యజ్ఞాయాచరతః కర్మ} \\
 & \natline{సమగ్రం ప్రవిలీయతే ||}
\end{tabular}
\end{table}

\begin{table}[H]
\begin{tabular}{cl}
\textbf{4.24} & \natline{బ్రహ్మార్పణం బ్రహ్మ హవిః} \\
 & \natline{బ్రహ్మాగ్నౌ బ్రహ్మణా హుతమ్ |} \\
 & \natline{బ్రహ్మైవ తేన గన్తవ్యం} \\
 & \natline{బ్రహ్మకర్మసమాధినా ||}
\end{tabular}
\end{table}

\begin{table}[H]
\begin{tabular}{cl}
\textbf{4.25} & \natline{దైవమేవాపరే యజ్ఞం} \\
 & \natline{యోగినః పర్యుపాసతే |} \\
 & \natline{బ్రహ్మాగ్నావపరే యజ్ఞం} \\
 & \natline{యజ్ఞేనైవోపజుహ్వతి ||}
\end{tabular}
\end{table}

\begin{table}[H]
\begin{tabular}{cl}
\textbf{4.26} & \natline{శ్రోత్రాదీనీన్ద్రియాణ్యన్యే} \\
 & \natline{సంయమాగ్నిషు జుహ్వతి |} \\
 & \natline{శబ్దాదీన్విషయానన్యే} \\
 & \natline{ఇన్ద్రియాగ్నిషు జుహ్వతి ||}
\end{tabular}
\end{table}

\begin{table}[H]
\begin{tabular}{cl}
\textbf{4.27} & \natline{సర్వాణీన్ద్రియకర్మాణి} \\
 & \natline{ప్రాణకర్మాణి చాపరే |} \\
 & \natline{ఆత్మసంయమయోగాగ్నౌ} \\
 & \natline{జుహ్వతి జ్ఞానదీపితే ||}
\end{tabular}
\end{table}

\begin{table}[H]
\begin{tabular}{cl}
\textbf{4.28} & \natline{ద్రవ్యయజ్ఞాస్తపోయజ్ఞాః} \\
 & \natline{యోగయజ్ఞాస్తథాఽపరే |} \\
 & \natline{స్వాధ్యాయజ్ఞానయజ్ఞాశ్చ} \\
 & \natline{యతయః సంశితవ్రతాః ||}
\end{tabular}
\end{table}

\begin{table}[H]
\begin{tabular}{cl}
\textbf{4.29} & \natline{అపానే జుహ్వతి ప్రాణం} \\
 & \natline{ప్రాణేఽపానం తథాపరే |} \\
 & \natline{ప్రాణాపానగతీ రుద్ధ్వా} \\
 & \natline{ప్రాణాయామపరాయణాః ||}
\end{tabular}
\end{table}

\begin{table}[H]
\begin{tabular}{cl}
\textbf{4.30} & \natline{అపరే నియతాహారాః} \\
 & \natline{ప్రాణాన్ప్రాణేషు జుహ్వతి |} \\
 & \natline{సర్వేఽప్యేతే యజ్ఞవిదః} \\
 & \natline{యజ్ఞక్షపితకల్మషాః ||}
\end{tabular}
\end{table}

\begin{table}[H]
\begin{tabular}{cl}
\textbf{4.31} & \natline{యజ్ఞశిష్టామృతభుజః} \\
 & \natline{యాన్తి బ్రహ్మ సనాతనమ్ |} \\
 & \natline{నాయం లోకోఽస్త్యయజ్ఞస్య} \\
 & \natline{కుతోఽన్యః కురుసత్తమ ||}
\end{tabular}
\end{table}

\begin{table}[H]
\begin{tabular}{cl}
\textbf{4.32} & \natline{ఏవం బహువిధా యజ్ఞాః} \\
 & \natline{వితతా బ్రహ్మణో ముఖే |} \\
 & \natline{కర్మజాన్విద్ధి తాన్సర్వాన్} \\
 & \natline{ఏవం జ్ఞాత్వా విమోక్ష్యసే ||}
\end{tabular}
\end{table}

\begin{table}[H]
\begin{tabular}{cl}
\textbf{4.33} & \natline{శ్రేయాన్ద్రవ్యమయాద్యజ్ఞాత్} \\
 & \natline{జ్ఞానయజ్ఞః పరన్తప |} \\
 & \natline{సర్వం కర్మాఖిలం పార్థ} \\
 & \natline{జ్ఞానే పరిసమాప్యతే ||}
\end{tabular}
\end{table}

\begin{table}[H]
\begin{tabular}{cl}
\textbf{4.34} & \natline{తద్విద్ధి ప్రణిపాతేన} \\
 & \natline{పరిప్రశ్నేన సేవయా |} \\
 & \natline{ఉపదేక్ష్యన్తి తే జ్ఞానం} \\
 & \natline{జ్ఞానినస్తత్త్వదర్శినః ||}
\end{tabular}
\end{table}

\begin{table}[H]
\begin{tabular}{cl}
\textbf{4.35} & \natline{యజ్జ్ఞాత్వా న పునర్మోహమ్} \\
 & \natline{ఏవం యాస్యసి పాణ్డవ |} \\
 & \natline{యేన భూతాన్యశేషేణ} \\
 & \natline{ద్రక్ష్యస్యాత్మన్యథో మయి ||}
\end{tabular}
\end{table}

\begin{table}[H]
\begin{tabular}{cl}
\textbf{4.36} & \natline{అపి చేదసి పాపేభ్యః} \\
 & \natline{సర్వేభ్యః పాపకృత్తమః |} \\
 & \natline{సర్వం జ్ఞానప్లవేనైవ} \\
 & \natline{వృజినం సన్తరిష్యసి ||}
\end{tabular}
\end{table}

\begin{table}[H]
\begin{tabular}{cl}
\textbf{4.37} & \natline{యథైధాంసి సమిద్ధోఽగ్నిః} \\
 & \natline{భస్మసాత్కురుతేఽర్జున |} \\
 & \natline{జ్ఞానాగ్నిః సర్వకర్మాణి} \\
 & \natline{భస్మసాత్కురుతే తథా ||}
\end{tabular}
\end{table}

\begin{table}[H]
\begin{tabular}{cl}
\textbf{4.38} & \natline{న హి జ్ఞానేన సదృశం} \\
 & \natline{పవిత్రమిహ విద్యతే |} \\
 & \natline{తత్స్వయం యోగసంసిద్ధః} \\
 & \natline{కాలేనాత్మని విన్దతి ||}
\end{tabular}
\end{table}

\begin{table}[H]
\begin{tabular}{cl}
\textbf{4.39} & \natline{శ్రద్ధావాన్ లభతే జ్ఞానం} \\
 & \natline{తత్పరః సంయతేన్ద్రియః |} \\
 & \natline{జ్ఞానం లబ్ధ్వా పరాం శాన్తిమ్} \\
 & \natline{అచిరేణాధిగచ్ఛతి ||}
\end{tabular}
\end{table}

\begin{table}[H]
\begin{tabular}{cl}
\textbf{4.40} & \natline{అజ్ఞశ్చాశ్రద్దధానశ్చ} \\
 & \natline{సంశయాత్మా వినశ్యతి |} \\
 & \natline{నాయం లోకోఽస్తి న పరః} \\
 & \natline{న సుఖం సంశయాత్మనః ||}
\end{tabular}
\end{table}

\begin{table}[H]
\begin{tabular}{cl}
\textbf{4.41} & \natline{యోగసన్న్యస్తకర్మాణం} \\
 & \natline{జ్ఞానసఞ్ఛిన్నసంశయమ్ |} \\
 & \natline{ఆత్మవన్తం న కర్మాణి} \\
 & \natline{నిబధ్నన్తి ధనఞ్జయ ||}
\end{tabular}
\end{table}

\begin{table}[H]
\begin{tabular}{cl}
\textbf{4.42} & \natline{తస్మాదజ్ఞానసమ్భూతం} \\
 & \natline{హృత్స్థం జ్ఞానాసినాత్మనః |} \\
 & \natline{ఛిత్త్వైనం సంశయం యోగమ్} \\
 & \natline{ఆతిష్ఠోత్తిష్ఠ భారత ||}
\end{tabular}
\end{table}


% \end{multicols}

\chapter{Karma-Sannyāsa Yoga}
% \begin{multicols}{2}
\begin{table}[H]
\begin{tabular}{cl}
\textbf{5.0} & \natline{ఓం శ్రీ పరమాత్మనే నమః} \\
 & \natline{అథ పఞ్చమోఽధ్యాయః} \\
 & \natline{కర్మసన్న్యాసయోగః}
\end{tabular}
\end{table}

\begin{table}[H]
\begin{tabular}{cl}
\textbf{5.1} & \natline{అర్జున ఉవాచ} \\
 & \natline{సన్న్యాసం కర్మణాం కృష్ణ} \\
 & \natline{పునర్యోగం చ శంససి |} \\
 & \natline{యచ్ఛ్రేయ ఏతయోరేకం} \\
 & \natline{తన్మే బ్రూహి సునిశ్చితమ్ ||}
\end{tabular}
\end{table}

\begin{table}[H]
\begin{tabular}{cl}
\textbf{5.2} & \natline{శ్రీ భగవానువాచ} \\
 & \natline{సన్న్యాసః కర్మయోగశ్చ} \\
 & \natline{నిశ్శ్రేయసకరావుభౌ |} \\
 & \natline{తయోస్తు కర్మసన్న్యాసాత్} \\
 & \natline{కర్మయోగో విశిష్యతే ||}
\end{tabular}
\end{table}

\begin{table}[H]
\begin{tabular}{cl}
\textbf{5.3} & \natline{జ్ఞేయః స నిత్యసన్న్యాసీ} \\
 & \natline{యో న ద్వేష్టి న కాఙ్క్షతి |} \\
 & \natline{నిర్ద్వన్ద్వో హి మహాబాహో} \\
 & \natline{సుఖం బన్ధాత్ప్రముచ్యతే ||}
\end{tabular}
\end{table}

\begin{table}[H]
\begin{tabular}{cl}
\textbf{5.4} & \natline{సాఙ్ఖ్యయోగౌ పృథగ్బాలాః} \\
 & \natline{ప్రవదన్తి న పణ్డితాః |} \\
 & \natline{ఏకమప్యాస్థితః సమ్యక్} \\
 & \natline{ఉభయోర్విన్దతే ఫలమ్ ||}
\end{tabular}
\end{table}

\begin{table}[H]
\begin{tabular}{cl}
\textbf{5.5} & \natline{యత్సాఙ్ఖ్యైః ప్రాప్యతే స్థానం} \\
 & \natline{తద్యోగైరపి గమ్యతే |} \\
 & \natline{ఏకం సాఙ్ఖ్యం చ యోగం చ} \\
 & \natline{యః పశ్యతి స పశ్యతి ||}
\end{tabular}
\end{table}

\begin{table}[H]
\begin{tabular}{cl}
\textbf{5.6} & \natline{సన్న్యాసస్తు మహాబాహో} \\
 & \natline{దుఃఖమాప్తుమయోగతః |} \\
 & \natline{యోగయుక్తో మునిర్బ్రహ్మ} \\
 & \natline{నచిరేణాధిగచ్ఛతి ||}
\end{tabular}
\end{table}

\begin{table}[H]
\begin{tabular}{cl}
\textbf{5.7} & \natline{యోగయుక్తో విశుద్ధాత్మా} \\
 & \natline{విజితాత్మా జితేన్ద్రియః |} \\
 & \natline{సర్వభూతాత్మభూతాత్మా} \\
 & \natline{కుర్వన్నపి న లిప్యతే ||}
\end{tabular}
\end{table}

\begin{table}[H]
\begin{tabular}{cl}
\textbf{5.8} & \natline{నైవ కిఞ్చిత్కరోమీతి} \\
 & \natline{యుక్తో మన్యేత తత్త్వవిత్ |} \\
 & \natline{పశ్యన్శృణ్వన్ స్పృశఞ్జిఘ్రన్} \\
 & \natline{అశ్నన్గచ్ఛన్స్వపన్శ్వసన్ ||}
\end{tabular}
\end{table}

\begin{table}[H]
\begin{tabular}{cl}
\textbf{5.9} & \natline{ప్రలపన్ విసృజన్ గృహ్ణన్} \\
 & \natline{ఉన్మిషన్నిమిషన్నపి |} \\
 & \natline{ఇన్ద్రియాణీన్ద్రియార్థేషు} \\
 & \natline{వర్తన్త ఇతి ధారయన్ ||}
\end{tabular}
\end{table}

\begin{table}[H]
\begin{tabular}{cl}
\textbf{5.10} & \natline{బ్రహ్మణ్యాధాయ కర్మాణి} \\
 & \natline{సఙ్గం త్యక్త్వా కరోతి యః |} \\
 & \natline{లిప్యతే న స పాపేన} \\
 & \natline{పద్మపత్రమివామ్భసా ||}
\end{tabular}
\end{table}

\begin{table}[H]
\begin{tabular}{cl}
\textbf{5.11} & \natline{కాయేన మనసా బుద్ధ్యా} \\
 & \natline{కేవలైరిన్ద్రియైరపి |} \\
 & \natline{యోగినః కర్మ కుర్వన్తి} \\
 & \natline{సఙ్గం త్యక్త్వాత్మశుద్ధయే ||}
\end{tabular}
\end{table}

\begin{table}[H]
\begin{tabular}{cl}
\textbf{5.12} & \natline{యుక్తః కర్మఫలం త్యక్త్వా} \\
 & \natline{శాన్తిమాప్నోతి నైష్ఠికీమ్ |} \\
 & \natline{అయుక్తః కామకారేణ} \\
 & \natline{ఫలే సక్తో నిబధ్యతే ||}
\end{tabular}
\end{table}

\begin{table}[H]
\begin{tabular}{cl}
\textbf{5.13} & \natline{సర్వకర్మాణి మనసా} \\
 & \natline{సన్న్యస్యాస్తే సుఖం వశీ |} \\
 & \natline{నవద్వారే పురే దేహీ} \\
 & \natline{నైవ కుర్వన్న కారయన్ ||}
\end{tabular}
\end{table}

\begin{table}[H]
\begin{tabular}{cl}
\textbf{5.14} & \natline{న కర్తృత్వం న కర్మాణి} \\
 & \natline{లోకస్య సృజతి ప్రభుః |} \\
 & \natline{న కర్మఫలసంయోగం} \\
 & \natline{స్వభావస్తు ప్రవర్తతే ||}
\end{tabular}
\end{table}

\begin{table}[H]
\begin{tabular}{cl}
\textbf{5.15} & \natline{నాదత్తే కస్యచిత్పాపం} \\
 & \natline{న చైవ సుకృతం విభుః |} \\
 & \natline{అజ్ఞానేనావృతం జ్ఞానం} \\
 & \natline{తేన ముహ్యన్తి జన్తవః ||}
\end{tabular}
\end{table}

\begin{table}[H]
\begin{tabular}{cl}
\textbf{5.16} & \natline{జ్ఞానేన తు తదజ్ఞానం} \\
 & \natline{యేషాం నాశితమాత్మనః |} \\
 & \natline{తేషామాదిత్యవజ్జ్ఞానం} \\
 & \natline{ప్రకాశయతి తత్పరమ్ ||}
\end{tabular}
\end{table}

\begin{table}[H]
\begin{tabular}{cl}
\textbf{5.17} & \natline{తద్బుద్ధయస్తదాత్మానః} \\
 & \natline{తన్నిష్ఠాస్తత్పరాయణాః |} \\
 & \natline{గచ్ఛన్త్యపునరావృత్తిం} \\
 & \natline{జ్ఞాననిర్ధూతకల్మషాః ||}
\end{tabular}
\end{table}

\begin{table}[H]
\begin{tabular}{cl}
\textbf{5.18} & \natline{విద్యావినయసమ్పన్నే} \\
 & \natline{బ్రాహ్మణే గవి హస్తిని |} \\
 & \natline{శుని చైవ శ్వపాకే చ} \\
 & \natline{పణ్డితాః సమదర్శినః ||}
\end{tabular}
\end{table}

\begin{table}[H]
\begin{tabular}{cl}
\textbf{5.19} & \natline{ఇహైవ తైర్జితః సర్గః} \\
 & \natline{యేషాం సామ్యే స్థితం మనః |} \\
 & \natline{నిర్దోషం హి సమం బ్రహ్మ} \\
 & \natline{తస్మాత్ బ్రహ్మణి తే స్థితాః ||}
\end{tabular}
\end{table}

\begin{table}[H]
\begin{tabular}{cl}
\textbf{5.20} & \natline{న ప్రహృష్యేత్ప్రియం ప్రాప్య} \\
 & \natline{నోద్విజేత్ప్రాప్య చాప్రియమ్ |} \\
 & \natline{స్థిరబుద్ధిరసమ్మూఢః} \\
 & \natline{బ్రహ్మవిత్ బ్రహ్మణి స్థితః ||}
\end{tabular}
\end{table}

\begin{table}[H]
\begin{tabular}{cl}
\textbf{5.21} & \natline{బాహ్యస్పర్శేష్వసక్తాత్మా} \\
 & \natline{విన్దత్యాత్మని యత్ సుఖమ్ |} \\
 & \natline{స బ్రహ్మయోగయుక్తాత్మా} \\
 & \natline{సుఖమక్షయమశ్నుతే ||}
\end{tabular}
\end{table}

\begin{table}[H]
\begin{tabular}{cl}
\textbf{5.22} & \natline{యే హి సంస్పర్శజా భోగాః} \\
 & \natline{దుఃఖయోనయ ఏవ తే |} \\
 & \natline{ఆద్యన్తవన్తః కౌన్తేయ} \\
 & \natline{న తేషు రమతే బుధః ||}
\end{tabular}
\end{table}

\begin{table}[H]
\begin{tabular}{cl}
\textbf{5.23} & \natline{శక్నోతీహైవ యః సోఢుం} \\
 & \natline{ప్రాక్ శరీరవిమోక్షణాత్ |} \\
 & \natline{కామక్రోధోద్భవం వేగం} \\
 & \natline{స యుక్తః స సుఖీ నరః ||}
\end{tabular}
\end{table}

\begin{table}[H]
\begin{tabular}{cl}
\textbf{5.24} & \natline{యోఽన్తఃసుఖోఽన్తరారామః} \\
 & \natline{తథాన్తర్జ్యోతిరేవ యః |} \\
 & \natline{స యోగీ బ్రహ్మనిర్వాణం} \\
 & \natline{బ్రహ్మభూతోఽధిగచ్ఛతి ||}
\end{tabular}
\end{table}

\begin{table}[H]
\begin{tabular}{cl}
\textbf{5.25} & \natline{లభన్తే బ్రహ్మనిర్వాణమ్} \\
 & \natline{ఋషయః క్షీణకల్మషాః |} \\
 & \natline{ఛిన్నద్వైధా యతాత్మానః} \\
 & \natline{సర్వభూతహితే రతాః ||}
\end{tabular}
\end{table}

\begin{table}[H]
\begin{tabular}{cl}
\textbf{5.26} & \natline{కామక్రోధవియుక్తానాం} \\
 & \natline{యతీనాం యతచేతసామ్ |} \\
 & \natline{అభితో బ్రహ్మనిర్వాణం} \\
 & \natline{వర్తతే విదితాత్మనామ్ ||}
\end{tabular}
\end{table}

\begin{table}[H]
\begin{tabular}{cl}
\textbf{5.27} & \natline{స్పర్శాన్ కృత్వా బహిర్బాహ్యాన్} \\
 & \natline{చక్షుశ్చైవాన్తరే భ్రువోః |} \\
 & \natline{ప్రాణాపానౌ సమౌ కృత్వా} \\
 & \natline{నాసాభ్యన్తరచారిణౌ ||}
\end{tabular}
\end{table}

\begin{table}[H]
\begin{tabular}{cl}
\textbf{5.28} & \natline{యతేన్ద్రియమనోబుద్ధిః} \\
 & \natline{మునిర్మోక్షపరాయణః |} \\
 & \natline{విగతేచ్ఛాభయక్రోధః} \\
 & \natline{యః సదా ముక్త ఏవ సః ||}
\end{tabular}
\end{table}

\begin{table}[H]
\begin{tabular}{cl}
\textbf{5.29} & \natline{భోక్తారం యజ్ఞతపసాం} \\
 & \natline{సర్వలోకమహేశ్వరమ్ |} \\
 & \natline{సుహృదం సర్వభూతానాం} \\
 & \natline{జ్ఞాత్వా మాం శాన్తిమృచ్ఛతి। ||}
\end{tabular}
\end{table}


% \end{multicols}

\chapter{Ātma-Saṃyama Yoga}
% \begin{multicols}{2}
\begin{table}[H]
\begin{tabular}{cl}
\textbf{6.0} & \romline{oṃ śrī paramātmane namaḥ} \\
 & \romline{atha ṣaṣṭho'dhyāyaḥ} \\
 & \romline{ātma-saṃyama-yogaḥ}
\end{tabular}
\end{table}

\begin{table}[H]
\begin{tabular}{cl}
\textbf{6.1} & \romline{śrī bhagavānuvāca} \\
 & \romline{anāśritaḥ karma-phalaṃ} \\
 & \romline{kāryaṃ karma karoti yaḥ |} \\
 & \romline{sa sannyāsī ca yogī ca} \\
 & \romline{na niragnirna cākriyaḥ ||}
\end{tabular}
\end{table}

\begin{table}[H]
\begin{tabular}{cl}
\textbf{6.2} & \romline{yaṃ sannyā-samiti·prāhuḥ} \\
 & \romline{yogaṃ taṃ viddhi pāṇḍava |} \\
 & \romline{na hya-sannyasta-saṅkalpaḥ} \\
 & \romline{yogī bhavati kaścana ||}
\end{tabular}
\end{table}

\begin{table}[H]
\begin{tabular}{cl}
\textbf{6.3} & \romline{āru-rukṣor-muneryogaṃ} \\
 & \romline{karma kāraṇa-mucyate |} \\
 & \romline{yogā-rūḍhasya tasyaiva} \\
 & \romline{śamaḥ kāraṇa-mucyate ||}
\end{tabular}
\end{table}

\begin{table}[H]
\begin{tabular}{cl}
\textbf{6.4} & \romline{yadā hi nendriyārtheṣu} \\
 & \romline{na karma-svanuṣajjate |} \\
 & \romline{sarva-saṅkalpa-sannyāsī} \\
 & \romline{yogā-rūḍhasta-docyate ||}
\end{tabular}
\end{table}

\begin{table}[H]
\begin{tabular}{cl}
\textbf{6.5} & \romline{uddhare-dātmanā''tmānaṃ} \\
 & \romline{nātmānama-vasādayet |} \\
 & \romline{ātmaiva hyātmano bandhuḥ} \\
 & \romline{ātmaiva ripurātmanaḥ ||}
\end{tabular}
\end{table}

\begin{table}[H]
\begin{tabular}{cl}
\textbf{6.6} & \romline{bandhurātmā''t-manastasya} \\
 & \romline{yenāt-maivāt-manā jitaḥ |} \\
 & \romline{anāt-manastu śatrutve} \\
 & \romline{varte-tātmaiva śatruvat ||}
\end{tabular}
\end{table}

\begin{table}[H]
\begin{tabular}{cl}
\textbf{6.7} & \romline{jitāt-manaḥ praśāntasya} \\
 & \romline{para-mātmā samāhitaḥ |} \\
 & \romline{śītoṣṇa-sukha-duḥkheṣu} \\
 & \romline{tathā mānāpa-mānayoḥ ||}
\end{tabular}
\end{table}

\begin{table}[H]
\begin{tabular}{cl}
\textbf{6.8} & \romline{jñāna-vijñāna-tṛptātmā} \\
 & \romline{kūṭastho vijitendriyaḥ |} \\
 & \romline{yukta ityucyate yogī} \\
 & \romline{samaloṣṭāś-makāñcanaḥ ||}
\end{tabular}
\end{table}

\begin{table}[H]
\begin{tabular}{cl}
\textbf{6.9} & \romline{suhṛn-mitrāryu-dāsīna-} \\
 & \romline{madhya-stha-dveṣya-bandhuṣu |} \\
 & \romline{sādhuṣvapi ca pāpeṣu} \\
 & \romline{samabuddhir-viśiṣyate ||}
\end{tabular}
\end{table}

\begin{table}[H]
\begin{tabular}{cl}
\textbf{6.10} & \romline{yogī yuñjīta satatam} \\
 & \romline{ātmānaṃ rahasi sthitaḥ |} \\
 & \romline{ekākī yatacittātmā} \\
 & \romline{nirāśīra-parigrahaḥ ||}
\end{tabular}
\end{table}

\begin{table}[H]
\begin{tabular}{cl}
\textbf{6.11} & \romline{śucau deśe pratiṣṭhāpya} \\
 & \romline{sthira-māsana-mātmanaḥ |} \\
 & \romline{nātyuc-chritaṃ nāti-nīcaṃ} \\
 & \romline{cailājina-kuśottaram ||}
\end{tabular}
\end{table}

\begin{table}[H]
\begin{tabular}{cl}
\textbf{6.12} & \romline{tatraikāgraṃ manaḥ kṛtvā} \\
 & \romline{yata-cittendriya-kriyaḥ |} \\
 & \romline{upa-viśyāsane yuñjyāt} \\
 & \romline{yogamāt-mavi-śuddhaye ||}
\end{tabular}
\end{table}

\begin{table}[H]
\begin{tabular}{cl}
\textbf{6.13} & \romline{samaṃ kāyaśiro-grīvaṃ} \\
 & \romline{dhārayan-nacalaṃ sthiraḥ |} \\
 & \romline{samprekṣya nāsi-kāgraṃ svaṃ} \\
 & \romline{diśaścānava-lokayan ||}
\end{tabular}
\end{table}

\begin{table}[H]
\begin{tabular}{cl}
\textbf{6.14} & \romline{praśāntātmā vigata-bhīḥ} \\
 & \romline{brahmacāri-vrate sthitaḥ |} \\
 & \romline{manaḥ saṃyamya maccittaḥ} \\
 & \romline{yukta āsīta matparaḥ ||}
\end{tabular}
\end{table}

\begin{table}[H]
\begin{tabular}{cl}
\textbf{6.15} & \romline{yuñjan-nevaṃ sadā''tmānaṃ} \\
 & \romline{yogī niyata-mānasaḥ |} \\
 & \romline{śāntiṃ nirvāṇa-paramāṃ} \\
 & \romline{matsaṃsthā-madhi-gacchati ||}
\end{tabular}
\end{table}

\begin{table}[H]
\begin{tabular}{cl}
\textbf{6.16} & \romline{nātya-śnatastu yogo'sti} \\
 & \romline{na caikānta-manaśnataḥ |} \\
 & \romline{na cāti svapna-śīlasya} \\
 & \romline{jāgrato naiva cārjuna ||}
\end{tabular}
\end{table}

\begin{table}[H]
\begin{tabular}{cl}
\textbf{6.17} & \romline{yuktā-hāra-vihārasya} \\
 & \romline{yukta-ceṣṭasya karmasu |} \\
 & \romline{yukta-svapnāva-bodhasya} \\
 & \romline{yogo bhavati duḥkhahā ||}
\end{tabular}
\end{table}

\begin{table}[H]
\begin{tabular}{cl}
\textbf{6.18} & \romline{yadā viniyataṃ cittam} \\
 & \romline{ātmanyevā-vatiṣṭhate |} \\
 & \romline{nisspṛhaḥ sarva-kāmebhyaḥ} \\
 & \romline{yukta ityucyate tadā ||}
\end{tabular}
\end{table}

\begin{table}[H]
\begin{tabular}{cl}
\textbf{6.19} & \romline{yathā dīpo nivā-tasthaḥ} \\
 & \romline{neṅgate sopamā smṛtā |} \\
 & \romline{yogino yatacittasya} \\
 & \romline{yuñjato yogomātmanaḥ ||}
\end{tabular}
\end{table}

\begin{table}[H]
\begin{tabular}{cl}
\textbf{6.20} & \romline{yatro-paramate cittaṃ} \\
 & \romline{niruddhaṃ yoga-sevayā |} \\
 & \romline{yatra caivāt-manā''t-mānaṃ} \\
 & \romline{pasyan-nātmani tuṣyati ||}
\end{tabular}
\end{table}

\begin{table}[H]
\begin{tabular}{cl}
\textbf{6.21} & \romline{sukhamāt-yantikaṃ yattat} \\
 & \romline{buddhi-grāhya-matīndriyam |} \\
 & \romline{vetti yatra na caivāyaṃ} \\
 & \romline{sthitaścalati tattvataḥ ||}
\end{tabular}
\end{table}

\begin{table}[H]
\begin{tabular}{cl}
\textbf{6.22} & \romline{yaṃ labdhvā cāparaṃ lābhaṃ} \\
 & \romline{manyate nādhikaṃ tataḥ |} \\
 & \romline{yasmin sthito na duḥkhena} \\
 & \romline{guruṇāpi vicālyate ||}
\end{tabular}
\end{table}

\begin{table}[H]
\begin{tabular}{cl}
\textbf{6.23} & \romline{taṃ vidyāt duḥkha-saṃyoga-} \\
 & \romline{viyogaṃ yogasañjñitam |} \\
 & \romline{sa niścayena yoktavyaḥ} \\
 & \romline{yogo'nirviṇṇa-cetasā ||}
\end{tabular}
\end{table}

\begin{table}[H]
\begin{tabular}{cl}
\textbf{6.24} & \romline{saṅkalpa-prabhavān-kāmān} \\
 & \romline{tyaktvā sarvāna-śeṣataḥ |} \\
 & \romline{manasaivendriya-grāmaṃ} \\
 & \romline{viniyamya samantataḥ ||}
\end{tabular}
\end{table}

\begin{table}[H]
\begin{tabular}{cl}
\textbf{6.25} & \romline{śanaiḥ śanai-ruparamet} \\
 & \romline{buddhyā dhṛti-gṛhītayā |} \\
 & \romline{ātmasaṃsthaṃ manaḥ kṛtvā} \\
 & \romline{na kiñcidapi cintayet ||}
\end{tabular}
\end{table}

\begin{table}[H]
\begin{tabular}{cl}
\textbf{6.26} & \romline{yato yato niścarati} \\
 & \romline{manaścañcala-masthiram |} \\
 & \romline{tatastato niyam-yaitat} \\
 & \romline{ātmanyeva vaśaṃ nayet ||}
\end{tabular}
\end{table}

\begin{table}[H]
\begin{tabular}{cl}
\textbf{6.27} & \romline{praśānta-manasaṃ hyenaṃ} \\
 & \romline{yoginaṃ sukha-muttamam |} \\
 & \romline{upaiti śānta-rajasaṃ} \\
 & \romline{brahma-bhūta-makalmaṣam ||}
\end{tabular}
\end{table}

\begin{table}[H]
\begin{tabular}{cl}
\textbf{6.28} & \romline{yuñjannevaṃ sadā''t-mānaṃ} \\
 & \romline{yogī vigata-kalmaṣaḥ |} \\
 & \romline{sukhena brahma-saṃsparśaṃ} \\
 & \romline{atyantaṃ sukkha-maśnute ||}
\end{tabular}
\end{table}

\begin{table}[H]
\begin{tabular}{cl}
\textbf{6.29} & \romline{sarva-bhūtastha-mātmānaṃ} \\
 & \romline{sarva-bhūtāni cātmani |} \\
 & \romline{īkṣate yoga-yuktātmā} \\
 & \romline{sarvatra samadarśanaḥ ||}
\end{tabular}
\end{table}

\begin{table}[H]
\begin{tabular}{cl}
\textbf{6.30} & \romline{yo māṃ paśyati sarvatra} \\
 & \romline{sarvaṃ ca mayi paśyati |} \\
 & \romline{tasyāhaṃ na·praṇaśyāmi} \\
 & \romline{sa ca me na·praṇaśyati ||}
\end{tabular}
\end{table}

\begin{table}[H]
\begin{tabular}{cl}
\textbf{6.31} & \romline{sarva-bhūta-sthitaṃ yo māṃ} \\
 & \romline{bhajatye-katva-māsthitaḥ |} \\
 & \romline{sarvathā vartamāno'pi} \\
 & \romline{sa yogī mayi vartate ||}
\end{tabular}
\end{table}

\begin{table}[H]
\begin{tabular}{cl}
\textbf{6.32} & \romline{ātmau-pamyena sarvatra} \\
 & \romline{samaṃ paśyati yo'rjuna |} \\
 & \romline{sukhaṃ vā yadi vā duḥkhaṃ} \\
 & \romline{sa yogī paramo matatḥ ||}
\end{tabular}
\end{table}

\begin{table}[H]
\begin{tabular}{cl}
\textbf{6.33} & \romline{arjuna uvāca} \\
 & \romline{yo'yaṃ yogastvayā proktaḥ} \\
 & \romline{sāmyena madhu-sūdana |} \\
 & \romline{etas-yāhaṃ na paśyāmi} \\
 & \romline{cañcalatvāt sthitiṃ sthirām ||}
\end{tabular}
\end{table}

\begin{table}[H]
\begin{tabular}{cl}
\textbf{6.34} & \romline{cañcalaṃ hi manaḥ kṛṣṇa} \\
 & \romline{pramāthi balavad-dṛḍham |} \\
 & \romline{tasyāhaṃ nigrahaṃ manye} \\
 & \romline{vāyoriva suduṣkaram ||}
\end{tabular}
\end{table}

\begin{table}[H]
\begin{tabular}{cl}
\textbf{6.35} & \romline{śrī bhagavānuvāca} \\
 & \romline{asaṃśayaṃ mahābāho} \\
 & \romline{mano durnigrahaṃ calam |} \\
 & \romline{abhyāsena tu kaunteya} \\
 & \romline{vairāgyeṇa ca gṛhyate ||}
\end{tabular}
\end{table}

\begin{table}[H]
\begin{tabular}{cl}
\textbf{6.36} & \romline{asaṃ-yatāt-manā yogaḥ} \\
 & \romline{duṣprāpa iti me matiḥ |} \\
 & \romline{vaśyāt-manā tu yatatā} \\
 & \romline{śakyo'vāptu-mupāyataḥ ||}
\end{tabular}
\end{table}

\begin{table}[H]
\begin{tabular}{cl}
\textbf{6.37} & \romline{arjuna uvāca} \\
 & \romline{ayatiḥ śraddha-yopetaḥ} \\
 & \romline{yogāccalita-mānasaḥ |} \\
 & \romline{aprāpya yoga-saṃsiddhiṃ} \\
 & \romline{kāṃ gatiṃ kṛṣṇa gacchati ||}
\end{tabular}
\end{table}

\begin{table}[H]
\begin{tabular}{cl}
\textbf{6.38} & \romline{kaccinno-bhaya-vibhraṣṭaḥ} \\
 & \romline{cinnā-bhra-miva naśyati |} \\
 & \romline{apratiṣṭho mahābāho} \\
 & \romline{vimūḍho brahmaṇaḥ pathi ||}
\end{tabular}
\end{table}

\begin{table}[H]
\begin{tabular}{cl}
\textbf{6.39} & \romline{etanme saṃśayaṃ kṛṣṇa} \\
 & \romline{chettu-marhasya-śeṣataḥ |} \\
 & \romline{tvadanyaḥ saṃśa-yasyāsya} \\
 & \romline{chettā na hyupa-padyate ||}
\end{tabular}
\end{table}

\begin{table}[H]
\begin{tabular}{cl}
\textbf{6.40} & \romline{śrī bhagavānuvāca} \\
 & \romline{pārtha naiveha nāmutra} \\
 & \romline{vināśastasya vidyate |} \\
 & \romline{na hi kalyāṇa-kṛt-kaścit} \\
 & \romline{durgatiṃ tāta gacchati ||}
\end{tabular}
\end{table}


% \end{multicols}

\chapter{Jñāna-Vijñāna Yoga}
% \begin{multicols}{2}
\begin{table}[H]
\begin{tabular}{cl}
\textbf{7.0} & \natline{ओं श्री परमात्मने नमः} \\
 & \natline{अथ सप्तमोऽध्यायः} \\
 & \natline{ज्ञानविज्ञानयोगः}
\end{tabular}
\end{table}

\begin{table}[H]
\begin{tabular}{cl}
\textbf{7.1} & \natline{श्री भगवानुवाच} \\
 & \natline{मय्यासक्तमनाः पार्थ} \\
 & \natline{योगं युञ्जन्मदाश्रयः |} \\
 & \natline{असंशयं समग्रं मां} \\
 & \natline{यथा ज्ञास्यसि तच्छृणु ||}
\end{tabular}
\end{table}

\begin{table}[H]
\begin{tabular}{cl}
\textbf{7.2} & \natline{ज्ञानं तेऽहं सविज्ञानम्} \\
 & \natline{इदं वक्ष्याम्यशेषतः |} \\
 & \natline{यज्ज्ञात्वा नेह भूयोऽन्यत्} \\
 & \natline{ज्ञातव्यमवशिष्यते ||}
\end{tabular}
\end{table}

\begin{table}[H]
\begin{tabular}{cl}
\textbf{7.3} & \natline{मनुष्याणां सहस्रेषु} \\
 & \natline{कश्चिद्यतति सिद्धये |} \\
 & \natline{यततामपि सिद्धानां} \\
 & \natline{कस्चिन्मां वेत्ति तत्त्वतह् ||}
\end{tabular}
\end{table}

\begin{table}[H]
\begin{tabular}{cl}
\textbf{7.4} & \natline{भूमिरापोऽनलो वायुः} \\
 & \natline{खं मनो बुद्धिरेव च |} \\
 & \natline{अहङ्कार इतीयं मे} \\
 & \natline{भिन्ना प्रकृतिरष्टधा ||}
\end{tabular}
\end{table}

\begin{table}[H]
\begin{tabular}{cl}
\textbf{7.5} & \natline{अपरेयमितस्त्वन्यां} \\
 & \natline{प्रकृतिं विद्धि मे पराम् |} \\
 & \natline{जीवभूतां महाबाहो} \\
 & \natline{ययेदं धार्यते जगत् ||}
\end{tabular}
\end{table}

\begin{table}[H]
\begin{tabular}{cl}
\textbf{7.6} & \natline{एतद्योनीनि भूतानि} \\
 & \natline{सर्वाणीत्युपधारय |} \\
 & \natline{अहं कृत्स्नस्य जगतः} \\
 & \natline{प्रभवः प्रलयस्तथा ||}
\end{tabular}
\end{table}

\begin{table}[H]
\begin{tabular}{cl}
\textbf{7.7} & \natline{मत्तः परतरं नान्यत्} \\
 & \natline{किञ्चिदस्ति धनञ्जय |} \\
 & \natline{मयि सर्वमिदं प्रोतं} \\
 & \natline{सूत्रे मणिगणा इव ||}
\end{tabular}
\end{table}

\begin{table}[H]
\begin{tabular}{cl}
\textbf{7.8} & \natline{रसोऽहमप्सु कौन्तेय} \\
 & \natline{प्रभाऽस्मि शशिसूर्ययोः |} \\
 & \natline{प्रणवः सर्ववेदेषु} \\
 & \natline{शब्दः खे पौरुषं नृषु ||}
\end{tabular}
\end{table}

\begin{table}[H]
\begin{tabular}{cl}
\textbf{7.9} & \natline{पुण्यो गन्धः पृथिव्यां च} \\
 & \natline{तेजश्चास्मि विभावसौ |} \\
 & \natline{जीवनं सर्वभुतेषु} \\
 & \natline{तपश्चास्मि तपस्विषु ||}
\end{tabular}
\end{table}

\begin{table}[H]
\begin{tabular}{cl}
\textbf{7.10} & \natline{बीजं मां सर्वभूतानां} \\
 & \natline{विद्धि पार्थ सनातनम् |} \\
 & \natline{बुद्धिर्बुद्धिमतामस्मि} \\
 & \natline{तेजस्तेजस्विनामहम् ||}
\end{tabular}
\end{table}

\begin{table}[H]
\begin{tabular}{cl}
\textbf{7.11} & \natline{बलं बलवतां चाहं} \\
 & \natline{कामरागविवर्जितम् |} \\
 & \natline{धर्माविरुद्धो भूतेषु} \\
 & \natline{कामोऽस्मि भरतर्षभ ||}
\end{tabular}
\end{table}

\begin{table}[H]
\begin{tabular}{cl}
\textbf{7.12} & \natline{ये चैव सात्त्विका भावाः} \\
 & \natline{राजसास्तामसाश्च ये |} \\
 & \natline{मत्त एवेति तान्विद्धि} \\
 & \natline{न त्वहं तेषु ते मयि ||}
\end{tabular}
\end{table}

\begin{table}[H]
\begin{tabular}{cl}
\textbf{7.13} & \natline{त्रिभिर्गुणमयैर्भावैः} \\
 & \natline{एभिः सर्वमिदं जगत् |} \\
 & \natline{मोहितं नाभिजानाति} \\
 & \natline{मामेभ्यः परमव्ययम् ||}
\end{tabular}
\end{table}

\begin{table}[H]
\begin{tabular}{cl}
\textbf{7.14} & \natline{दैवी ह्येषा गुनामयी} \\
 & \natline{मम माय दुरत्यया |} \\
 & \natline{मामेव ये प्रपद्यन्ते} \\
 & \natline{मायामेतां तरन्ति ते ||}
\end{tabular}
\end{table}

\begin{table}[H]
\begin{tabular}{cl}
\textbf{7.15} & \natline{न मां दुष्कृतिनो मूढाः} \\
 & \natline{प्रपद्यन्ते नराधमाः |} \\
 & \natline{माययाऽपहृतज्ञानाः} \\
 & \natline{आसुरं भावमाश्रिताः ||}
\end{tabular}
\end{table}

\begin{table}[H]
\begin{tabular}{cl}
\textbf{7.16} & \natline{चतुर्विधा भजन्ते मां} \\
 & \natline{जनाः सुकृतिनोऽर्जुन |} \\
 & \natline{आर्तो जिज्ञासुरर्थार्थी} \\
 & \natline{ज्ञानी च भरतर्षभ ||}
\end{tabular}
\end{table}

\begin{table}[H]
\begin{tabular}{cl}
\textbf{7.17} & \natline{तेषां ज्ञानी नित्ययुक्तः} \\
 & \natline{एकभक्तिर्विशिष्यते |} \\
 & \natline{प्रियो हि ज्ञानिनोऽत्यर्थम्} \\
 & \natline{अहं स च मम प्रियः ||}
\end{tabular}
\end{table}

\begin{table}[H]
\begin{tabular}{cl}
\textbf{7.18} & \natline{उदाराः सर्व एवैते} \\
 & \natline{ज्ञानी त्वात्मैव मे मतम् |} \\
 & \natline{आस्थितः स हि युक्तात्मा} \\
 & \natline{मामेवानुत्तमां गतिम् ||}
\end{tabular}
\end{table}

\begin{table}[H]
\begin{tabular}{cl}
\textbf{7.19} & \natline{बहूनां जन्मनामन्ते} \\
 & \natline{ज्ञानवान्मां प्रपद्यते |} \\
 & \natline{वासुदेवः सर्वमिति} \\
 & \natline{स महात्मा सुदुर्लभः ||}
\end{tabular}
\end{table}

\begin{table}[H]
\begin{tabular}{cl}
\textbf{7.20} & \natline{कामैस्तैस्तैर्हृतज्ञानाः} \\
 & \natline{प्रपद्यन्तेऽन्यदेवताः |} \\
 & \natline{तं तं नियममास्थाय} \\
 & \natline{प्रकृत्या नियताः स्वया ||}
\end{tabular}
\end{table}

\begin{table}[H]
\begin{tabular}{cl}
\textbf{7.21} & \natline{यो यो यां यां तनुं भक्तः} \\
 & \natline{श्रद्धयार्चितुमिच्छति |} \\
 & \natline{तस्य तस्याचलां श्रद्धां} \\
 & \natline{तामेव विदधाम्यहम् ||}
\end{tabular}
\end{table}

\begin{table}[H]
\begin{tabular}{cl}
\textbf{7.22} & \natline{स तया श्रद्धया युक्तः} \\
 & \natline{तस्याराधनमीहते |} \\
 & \natline{लभते च ततः कामान्} \\
 & \natline{मयैव विहितान्हि तान् ||}
\end{tabular}
\end{table}

\begin{table}[H]
\begin{tabular}{cl}
\textbf{7.23} & \natline{अन्तवत्तु फलं तेषां} \\
 & \natline{तद्भवत्यल्पमेधसां |} \\
 & \natline{देवान्देवयजो यान्ति} \\
 & \natline{मद्भक्ता यान्ति मामपि ||}
\end{tabular}
\end{table}

\begin{table}[H]
\begin{tabular}{cl}
\textbf{7.24} & \natline{अव्यक्तं व्यक्तिमापन्नं} \\
 & \natline{मन्यन्ते मामबुद्धयः |} \\
 & \natline{परं भावमजानन्तः} \\
 & \natline{ममाव्ययमनुत्तमन् ||}
\end{tabular}
\end{table}

\begin{table}[H]
\begin{tabular}{cl}
\textbf{7.25} & \natline{नाहां प्रकाशः सर्वस्य} \\
 & \natline{योगमायासमावृतः |} \\
 & \natline{मूढोऽयं नाभिजानाति} \\
 & \natline{लोको मामजमव्ययम् ||}
\end{tabular}
\end{table}

\begin{table}[H]
\begin{tabular}{cl}
\textbf{7.26} & \natline{वेदाहं समतीतानि} \\
 & \natline{वर्तमानानि चार्जुन |} \\
 & \natline{भविष्याणि च भूतानि} \\
 & \natline{मां तु वेद न कश्चन ||}
\end{tabular}
\end{table}

\begin{table}[H]
\begin{tabular}{cl}
\textbf{7.27} & \natline{इच्छाद्वेषसमुत्थेन} \\
 & \natline{द्वन्द्वमोहेन भारत |} \\
 & \natline{सर्वभूतानि सम्मोहं} \\
 & \natline{सर्गे यान्ति परन्तप ||}
\end{tabular}
\end{table}

\begin{table}[H]
\begin{tabular}{cl}
\textbf{7.28} & \natline{येषां त्वन्तगतं पापम्} \\
 & \natline{जनानां पुण्यकर्मणाम् |} \\
 & \natline{ते द्वन्द्वमोहनिर्मुक्ताः} \\
 & \natline{भजन्ते मां दृढव्रताः ||}
\end{tabular}
\end{table}

\begin{table}[H]
\begin{tabular}{cl}
\textbf{7.29} & \natline{जरामरणमोक्षाय} \\
 & \natline{मामाश्रित्य यतन्ति ये |} \\
 & \natline{ते ब्रह्म तद्विदुः कृत्स्नम्} \\
 & \natline{अध्यात्मं कर्म चाखिलम् ||}
\end{tabular}
\end{table}

\begin{table}[H]
\begin{tabular}{cl}
\textbf{7.30} & \natline{साधिभूताधिदैवं मां} \\
 & \natline{साधियज्ञं चे ये विदुः |} \\
 & \natline{प्रयाणकालेऽपि च मां} \\
 & \natline{ते विदुर्युक्तचेतसः ||}
\end{tabular}
\end{table}


% \end{multicols}

\chapter{Akṣara-Parabrahma Yoga}
% \begin{multicols}{2}
\begin{table}[H]
\begin{tabular}{cl}
\textbf{8.0} & \natline{ओं श्री परमात्मने नमः} \\
 & \natline{अथ अष्टमोऽध्यायः} \\
 & \natline{अक्षरपरब्रह्मयोगः}
\end{tabular}
\end{table}

\begin{table}[H]
\begin{tabular}{cl}
\textbf{8.1} & \natline{अर्जुन उवाच} \\
 & \natline{किं तद्ब्रह्म किमध्यात्मं} \\
 & \natline{किं कर्म पुरुषोत्तम |} \\
 & \natline{अधिभूतं च किं प्रोक्तम्} \\
 & \natline{अधिदैवं किमुच्यते ||}
\end{tabular}
\end{table}

\begin{table}[H]
\begin{tabular}{cl}
\textbf{8.2} & \natline{अधियज्ञः कथं कोऽत्र} \\
 & \natline{देहेऽस्मिन्मधुसूदन |} \\
 & \natline{प्रयाणकाले च कथं} \\
 & \natline{ज्ञेयोऽसि नियतात्मभिः ||}
\end{tabular}
\end{table}

\begin{table}[H]
\begin{tabular}{cl}
\textbf{8.3} & \natline{श्री भगवानुवाच} \\
 & \natline{अक्षरम् ब्रह्म परमं} \\
 & \natline{स्वभावोऽध्यात्ममुच्यते |} \\
 & \natline{भूतभावोद्भवकरः} \\
 & \natline{विसर्गः कर्मसञ्ज्ञितः ||}
\end{tabular}
\end{table}

\begin{table}[H]
\begin{tabular}{cl}
\textbf{8.4} & \natline{अधिभूतं क्षरो भावः} \\
 & \natline{पुरुषश्चाधिदैवतम् |} \\
 & \natline{अधियज्ञोऽहमेवात्र} \\
 & \natline{देहे देहभृतां वर ||}
\end{tabular}
\end{table}

\begin{table}[H]
\begin{tabular}{cl}
\textbf{8.5} & \natline{अन्तकाले च मामेव} \\
 & \natline{स्मरन्मुक्त्वा कलेवरम् |} \\
 & \natline{यः प्रयाति स मद्भावं} \\
 & \natline{याति नास्त्यत्र संशयः ||}
\end{tabular}
\end{table}

\begin{table}[H]
\begin{tabular}{cl}
\textbf{8.6} & \natline{यं यं वापि स्मरन्भावं} \\
 & \natline{त्यजत्यन्ते कलेवरम् |} \\
 & \natline{तं तमेवैति कौन्तेय} \\
 & \natline{सदा तद्भावभावितः ||}
\end{tabular}
\end{table}

\begin{table}[H]
\begin{tabular}{cl}
\textbf{8.7} & \natline{तस्मात्सर्वेषु कालेषु} \\
 & \natline{मामनुस्मर युध्य च |} \\
 & \natline{मय्यर्पितमनोबुद्धिः} \\
 & \natline{मामेवैष्यस्यसंशयम् ||}
\end{tabular}
\end{table}

\begin{table}[H]
\begin{tabular}{cl}
\textbf{8.8} & \natline{अभ्यासयोगयुक्तेन} \\
 & \natline{चेतसा नान्यगामिना |} \\
 & \natline{परमं पुरुषं दिव्यं} \\
 & \natline{याति पार्थानुचिन्तयन् ||}
\end{tabular}
\end{table}

\begin{table}[H]
\begin{tabular}{cl}
\textbf{8.9} & \natline{कविं पुराणमनुशासितारम्} \\
 & \natline{अणोरणीयांसमनुस्मरेद्यः |} \\
 & \natline{सर्वस्य धातारमचिन्त्यरूपम्} \\
 & \natline{आदित्यवर्णं तमसः परस्तात् ||}
\end{tabular}
\end{table}

\begin{table}[H]
\begin{tabular}{cl}
\textbf{8.10} & \natline{प्रयाणकाले मनसाऽचलेन} \\
 & \natline{भक्त्या युक्तो योगबलेन चैव |} \\
 & \natline{भ्रुवोर्मध्ये प्राणमावेश्य सम्यक्} \\
 & \natline{स तं परं पुरुषमुपैति दिव्यं ||}
\end{tabular}
\end{table}

\begin{table}[H]
\begin{tabular}{cl}
\textbf{8.11} & \natline{यदक्षरं वेदविदो वदन्ति} \\
 & \natline{विशन्ति यद्यतयो वीतरागाः |} \\
 & \natline{यदिच्छन्तो ब्रह्मचर्यं चरन्ति} \\
 & \natline{तत्ते पदं सङ्ग्रहेण प्रवक्ष्ये ||}
\end{tabular}
\end{table}

\begin{table}[H]
\begin{tabular}{cl}
\textbf{8.12} & \natline{सर्वद्वाराणि संयम्य} \\
 & \natline{मनो हृदि निरुध्य च |} \\
 & \natline{मूर्ध्न्याधायात्मनः प्राणम्} \\
 & \natline{आस्थितो योगधारणां ||}
\end{tabular}
\end{table}

\begin{table}[H]
\begin{tabular}{cl}
\textbf{8.13} & \natline{ओमित्येकाक्षरं ब्रह्म} \\
 & \natline{व्याहरन्मामनुस्मरन् |} \\
 & \natline{यः प्रयाति त्यजन्देहं} \\
 & \natline{स याति परमां गतिम् ||}
\end{tabular}
\end{table}

\begin{table}[H]
\begin{tabular}{cl}
\textbf{8.14} & \natline{अनन्यचेताः सततं} \\
 & \natline{यो मां स्मरति नित्यशः |} \\
 & \natline{तस्याहं सुलभः पार्थ} \\
 & \natline{नित्ययुक्तस्य योगिनः ||}
\end{tabular}
\end{table}

\begin{table}[H]
\begin{tabular}{cl}
\textbf{8.15} & \natline{मामुपेत्य पुनर्जन्म} \\
 & \natline{दुःखालयमशाश्वतम् |} \\
 & \natline{नाप्नुवन्ति महात्मानः} \\
 & \natline{संसिद्धिं परमां गताः ||}
\end{tabular}
\end{table}

\begin{table}[H]
\begin{tabular}{cl}
\textbf{8.16} & \natline{आब्रह्मभुवनाल्लोकाः} \\
 & \natline{पुनरावर्तिनोऽर्जुन |} \\
 & \natline{मामुपेत्य तु कौन्तेय} \\
 & \natline{पुनर्जन्म न विद्यते ||}
\end{tabular}
\end{table}

\begin{table}[H]
\begin{tabular}{cl}
\textbf{8.17} & \natline{सहस्रयुगपर्यन्तम्} \\
 & \natline{अहर्यद्ब्रह्मणो विदुः |} \\
 & \natline{रात्रिं युगसहस्रान्तां} \\
 & \natline{तेऽहोरात्रविदो जनाः ||}
\end{tabular}
\end{table}

\begin{table}[H]
\begin{tabular}{cl}
\textbf{8.18} & \natline{अव्यक्ताद्व्यक्तयः सर्वाः} \\
 & \natline{प्रभवन्त्यहरागमे |} \\
 & \natline{रात्र्यागमे प्रलीयन्ते} \\
 & \natline{तत्रैवाव्यक्तसञ्ज्ञके ||}
\end{tabular}
\end{table}

\begin{table}[H]
\begin{tabular}{cl}
\textbf{8.19} & \natline{भूतग्रामः स एवायं} \\
 & \natline{भूत्वा भूत्वा प्रलीयते |} \\
 & \natline{रात्र्यागमेऽवशः पार्थ} \\
 & \natline{प्रभवत्यहरागमे ||}
\end{tabular}
\end{table}

\begin{table}[H]
\begin{tabular}{cl}
\textbf{8.20} & \natline{परस्तस्मात्तु भावोऽन्यः} \\
 & \natline{अव्यक्तोऽव्यक्तात्सनातनः |} \\
 & \natline{यः स सर्वेषु भूतेषु} \\
 & \natline{नश्यत्सु न विनश्यति ||}
\end{tabular}
\end{table}

\begin{table}[H]
\begin{tabular}{cl}
\textbf{8.21} & \natline{अव्यक्तोऽक्षर इत्युक्तः} \\
 & \natline{तमाहुः परमां गतिम् |} \\
 & \natline{यं प्राप्य न निवर्तन्ते} \\
 & \natline{तद्धाम परमं मम ||}
\end{tabular}
\end{table}

\begin{table}[H]
\begin{tabular}{cl}
\textbf{8.22} & \natline{पुरुषः स परः पार्थ} \\
 & \natline{भक्त्या लभ्यस्त्वनन्यया |} \\
 & \natline{यस्यान्तः स्थानि भूतानि} \\
 & \natline{येन सर्वमिदं ततम् ||}
\end{tabular}
\end{table}

\begin{table}[H]
\begin{tabular}{cl}
\textbf{8.23} & \natline{यत्र काले त्वनावृतिम्} \\
 & \natline{आवृतिं चैव योगिनः |} \\
 & \natline{प्रयाता यान्ति तं कालं} \\
 & \natline{वक्ष्यामि भरतर्षभ ||}
\end{tabular}
\end{table}

\begin{table}[H]
\begin{tabular}{cl}
\textbf{8.24} & \natline{अग्निर्ज्योतिरहः शुक्लः} \\
 & \natline{षण्मासा उत्तरायणम् |} \\
 & \natline{तत्र प्रयाता गच्छन्ति} \\
 & \natline{ब्रह्म ब्रह्मविदो जनाः ||}
\end{tabular}
\end{table}

\begin{table}[H]
\begin{tabular}{cl}
\textbf{8.25} & \natline{धूमो रात्रिस्तथा कृष्णः} \\
 & \natline{षण्मासा दक्षिणायनम् |} \\
 & \natline{तत्र चान्द्रमसं ज्योतिः} \\
 & \natline{योगी प्राप्य निवर्तते ||}
\end{tabular}
\end{table}

\begin{table}[H]
\begin{tabular}{cl}
\textbf{8.26} & \natline{शुक्लकृष्णे गती ह्येते} \\
 & \natline{जगतः शाश्वते मते |} \\
 & \natline{एकया यात्यनावृतिम्} \\
 & \natline{अन्ययाऽऽवर्तते पुनः ||}
\end{tabular}
\end{table}

\begin{table}[H]
\begin{tabular}{cl}
\textbf{8.27} & \natline{नैते सृती पार्थ जानन्} \\
 & \natline{योगी मुह्यति कश्चन |} \\
 & \natline{तस्मात्सर्वेषु कालेषु} \\
 & \natline{योगयुक्तो भवार्जुन ||}
\end{tabular}
\end{table}

\begin{table}[H]
\begin{tabular}{cl}
\textbf{8.28} & \natline{वेदेषु यज्ञेषु तपस्सु चैव} \\
 & \natline{दानेषु यत् पुण्यफलं प्रदिष्टम् |} \\
 & \natline{अत्येति तत्सर्वमिदं विदित्वा} \\
 & \natline{योगी परं स्थानमुपैति चाद्यम् ||}
\end{tabular}
\end{table}


% \end{multicols}

\chapter{Rājavidyā-Rājaguhya Yoga}
% \begin{multicols}{2}
\begin{table}[H]
\begin{tabular}{cl}
\textbf{9.0} & \natline{ఓం శ్రీ పరమాత్మనే నమః} \\
 & \natline{అథ నవమోఽధ్యాయః} \\
 & \natline{రాజవిద్యారాజగుహ్యయోగః}
\end{tabular}
\end{table}

\begin{table}[H]
\begin{tabular}{cl}
\textbf{9.1} & \natline{శ్రీ భగవానువాచ} \\
 & \natline{ఇదం తు తే గుహ్యతమమ్} \\
 & \natline{ప్రవక్ష్యామ్యనసూయవే |} \\
 & \natline{జ్ఞానం విజ్ఞానసహితం} \\
 & \natline{యజ్జ్ఞాత్వా మోక్ష్యసేఽశుభాత్ ||}
\end{tabular}
\end{table}

\begin{table}[H]
\begin{tabular}{cl}
\textbf{9.2} & \natline{రాజవిద్యా రాజగుహ్యం} \\
 & \natline{పవిత్రమిదముత్తమమ్ |} \\
 & \natline{ప్రత్యక్షావగమం ధర్మ్యం} \\
 & \natline{సుసుఖం కర్తుమవ్యయమ్ ||}
\end{tabular}
\end{table}

\begin{table}[H]
\begin{tabular}{cl}
\textbf{9.3} & \natline{అశ్రద్దధానాః పురుషాః} \\
 & \natline{ధర్మస్యాస్య పరన్తప |} \\
 & \natline{అప్రాప్య మాం నివర్తన్తే} \\
 & \natline{మృత్యుసంసారవర్త్మని ||}
\end{tabular}
\end{table}

\begin{table}[H]
\begin{tabular}{cl}
\textbf{9.4} & \natline{మయా తతమిదం సర్వం} \\
 & \natline{జగదవ్యక్తమూర్తినా |} \\
 & \natline{మత్స్థాని సర్వభూతాని} \\
 & \natline{న చాహం తేష్వవస్థితః ||}
\end{tabular}
\end{table}

\begin{table}[H]
\begin{tabular}{cl}
\textbf{9.5} & \natline{న చ మత్స్థాని భూతాని} \\
 & \natline{పశ్య మే యోగమైశ్వరమ్ |} \\
 & \natline{భూతభృన్న చ భూతస్థః} \\
 & \natline{మమాత్మా భూతభావనః ||}
\end{tabular}
\end{table}

\begin{table}[H]
\begin{tabular}{cl}
\textbf{9.6} & \natline{యథాఽఽకాశస్థితో నిత్యం} \\
 & \natline{వాయుః సర్వత్రగో మహాన్ |} \\
 & \natline{తథా సర్వాణి భూతాని} \\
 & \natline{మత్స్థానీత్యుపధారయ ||}
\end{tabular}
\end{table}

\begin{table}[H]
\begin{tabular}{cl}
\textbf{9.7} & \natline{సర్వభూతాని కౌన్తేయ} \\
 & \natline{ప్రకృతిం యాన్తి మామికామ్ |} \\
 & \natline{కల్పక్షయే పునస్తాని} \\
 & \natline{కల్పాదౌ విసృజామ్యహమ్ ||}
\end{tabular}
\end{table}

\begin{table}[H]
\begin{tabular}{cl}
\textbf{9.8} & \natline{ప్రకృతిం స్వామవష్టభ్య} \\
 & \natline{విసృజామి పునః పునః |} \\
 & \natline{భూతగ్రామమిమం కృత్స్నమ్} \\
 & \natline{అవశం ప్రకృతేర్వశాత్ ||}
\end{tabular}
\end{table}

\begin{table}[H]
\begin{tabular}{cl}
\textbf{9.9} & \natline{న చ మాం తాని కర్మాణి} \\
 & \natline{నిబధ్నన్తి ధనఞ్జయ |} \\
 & \natline{ఉదాసీనవదాసీనమ్} \\
 & \natline{అసక్తం తేషు కర్మసు ||}
\end{tabular}
\end{table}

\begin{table}[H]
\begin{tabular}{cl}
\textbf{9.10} & \natline{మయాధ్యక్షేణ ప్రకృతిః} \\
 & \natline{సూయతే సచరాచరమ్ |} \\
 & \natline{హేతునాఽనేన కౌన్తేయ} \\
 & \natline{జగద్విపరివర్తతే ||}
\end{tabular}
\end{table}

\begin{table}[H]
\begin{tabular}{cl}
\textbf{9.11} & \natline{అవజానన్తి మాం మూఢాః} \\
 & \natline{మానుషీం తనుమాశ్రితమ్ |} \\
 & \natline{పరం భావమజానన్తః} \\
 & \natline{మమ భూతమహేశ్వరమ్ ||}
\end{tabular}
\end{table}

\begin{table}[H]
\begin{tabular}{cl}
\textbf{9.12} & \natline{మోఘాశా మోఘకర్మాణః} \\
 & \natline{మోఘజ్ఞానా విచేతసః |} \\
 & \natline{రాక్షసీమాసురీం చైవ} \\
 & \natline{ప్రకృతిం మోహినీం శ్రితాః ||}
\end{tabular}
\end{table}

\begin{table}[H]
\begin{tabular}{cl}
\textbf{9.13} & \natline{మహాత్మానస్తు మాం పార్థ} \\
 & \natline{దైవీం ప్రకృతిమాశ్రితాః |} \\
 & \natline{భజన్త్యనన్యమనసః} \\
 & \natline{జ్ఞాత్వా భూతాదిమవ్యయమ్ ||}
\end{tabular}
\end{table}

\begin{table}[H]
\begin{tabular}{cl}
\textbf{9.14} & \natline{సతతం కీర్తయన్తో మాం} \\
 & \natline{యతన్తశ్చ దృఢవ్రతాః |} \\
 & \natline{నమస్యన్తశ్చ మామ్ భక్త్యా} \\
 & \natline{నిత్యయుక్తా ఉపాసతే ||}
\end{tabular}
\end{table}

\begin{table}[H]
\begin{tabular}{cl}
\textbf{9.15} & \natline{జ్ఞానయజ్ఞేన చాప్యన్యే} \\
 & \natline{యజన్తో మాముపాసతే |} \\
 & \natline{ఏకత్వేన పృథక్త్వేన} \\
 & \natline{బహుధా విశ్వతోముఖమ్ ||}
\end{tabular}
\end{table}

\begin{table}[H]
\begin{tabular}{cl}
\textbf{9.16} & \natline{అహం క్రతురహం యజ్ఞః} \\
 & \natline{స్వధాహమహమౌషధమ్ |} \\
 & \natline{మన్త్రోఽహమహమేవాజ్యమ్} \\
 & \natline{అహమగ్నిరహం హుతమ్ ||}
\end{tabular}
\end{table}

\begin{table}[H]
\begin{tabular}{cl}
\textbf{9.17} & \natline{పితాఽహమస్య జగతః} \\
 & \natline{మాతా ధాతా పితామహః |} \\
 & \natline{వేద్యం పవిత్రమోఙ్కారః} \\
 & \natline{ఋక్సామ యజురేవ చ ||}
\end{tabular}
\end{table}

\begin{table}[H]
\begin{tabular}{cl}
\textbf{9.18} & \natline{గతిర్భర్తా ప్రభుః సాక్షీ} \\
 & \natline{నివాసః శరణం సుహృత్ |} \\
 & \natline{ప్రభవః ప్రలయః స్థానం} \\
 & \natline{నిధానం బీజమవ్యయమ్ ||}
\end{tabular}
\end{table}

\begin{table}[H]
\begin{tabular}{cl}
\textbf{9.19} & \natline{తపామ్యహమహం వర్షం} \\
 & \natline{నిగృహ్ణామ్యుత్సృజామి చ |} \\
 & \natline{అమృతం చైవ మృత్యుశ్చ} \\
 & \natline{సదసచ్చాహమర్జున ||}
\end{tabular}
\end{table}

\begin{table}[H]
\begin{tabular}{cl}
\textbf{9.20} & \natline{త్రైవిద్యా మాం సోమపాః పూతపాపాః} \\
 & \natline{యజ్ఞైరిష్ట్వా స్వర్గతిం ప్రార్థయన్తే |} \\
 & \natline{తే పుణ్యమాసాద్య సురేన్ద్రలోకమ్} \\
 & \natline{అశ్నన్తి దివ్యాన్దివి దేవభోగాన్ ||}
\end{tabular}
\end{table}

\begin{table}[H]
\begin{tabular}{cl}
\textbf{9.21} & \natline{తే తం భుక్త్వా స్వర్గలోకం విశాలం} \\
 & \natline{క్షీణే పుణ్యే మర్త్యలోకం విశన్తి |} \\
 & \natline{ఏవం త్రయీధర్మమనుప్రపన్నాః} \\
 & \natline{గతాగతం కామకామా లభన్తే ||}
\end{tabular}
\end{table}

\begin{table}[H]
\begin{tabular}{cl}
\textbf{9.22} & \natline{అనన్యాశ్చిన్తయన్తో మాం} \\
 & \natline{యే జనాః పర్యుపాసతే |} \\
 & \natline{తేషాం నిత్యాభియుక్తానాం} \\
 & \natline{యోగక్షేమం వహామ్యహమ్ ||}
\end{tabular}
\end{table}

\begin{table}[H]
\begin{tabular}{cl}
\textbf{9.23} & \natline{యేఽప్యన్యదేవతా భక్తాః} \\
 & \natline{యజన్తే శ్రద్ధయాన్వితాః |} \\
 & \natline{తేఽపి మామేవ కౌన్తేయ} \\
 & \natline{యజన్త్యవిధిపూర్వకమ్ ||}
\end{tabular}
\end{table}

\begin{table}[H]
\begin{tabular}{cl}
\textbf{9.24} & \natline{అహం హి సర్వయజ్ఞానాం} \\
 & \natline{భోక్తా చ ప్రభురేవ చ |} \\
 & \natline{న తు మామభిజానన్తి} \\
 & \natline{తత్త్వేనాతశ్చ్యవన్తి తే ||}
\end{tabular}
\end{table}

\begin{table}[H]
\begin{tabular}{cl}
\textbf{9.25} & \natline{యాన్తి దేవవ్రతా దేవాణ్} \\
 & \natline{పితౄన్ యాన్తి పితృవ్రతాః |} \\
 & \natline{భూతాని యాన్తి భూతేజ్యాః} \\
 & \natline{యాన్తి మద్యాజినోఽపి మామ్ ||}
\end{tabular}
\end{table}

\begin{table}[H]
\begin{tabular}{cl}
\textbf{9.26} & \natline{పత్రం పుష్పం ఫలం తోయం} \\
 & \natline{యో మే భక్త్యా ప్రయచ్ఛతి |} \\
 & \natline{తదహం భక్త్యుపహృతం} \\
 & \natline{అశ్నామి ప్రయతాత్మనః ||}
\end{tabular}
\end{table}

\begin{table}[H]
\begin{tabular}{cl}
\textbf{9.27} & \natline{యత్కరోషి యదశ్నాసి} \\
 & \natline{యజ్జుహోషి దదాసి యత్ |} \\
 & \natline{యత్తపస్యసి కౌన్తేయ} \\
 & \natline{తత్కురుష్వ మదర్పణమ్ ||}
\end{tabular}
\end{table}

\begin{table}[H]
\begin{tabular}{cl}
\textbf{9.28} & \natline{శుభాశుభఫలైరేవం} \\
 & \natline{మోక్ష్యసే కర్మబన్ధనైః |} \\
 & \natline{సన్న్యాసయోగయుక్తాత్మా} \\
 & \natline{విముక్తో మాముపైష్యసి ||}
\end{tabular}
\end{table}

\begin{table}[H]
\begin{tabular}{cl}
\textbf{9.29} & \natline{సమోఽహం సర్వభూతేషు} \\
 & \natline{న మే ద్వేష్యోఽస్తి న ప్రియః |} \\
 & \natline{యే భజన్తి తు మాం భక్త్యా} \\
 & \natline{మయి తే తేషు చాప్యహమ్ ||}
\end{tabular}
\end{table}

\begin{table}[H]
\begin{tabular}{cl}
\textbf{9.30} & \natline{అపి చేత్సుదురాచారః} \\
 & \natline{భజతే మామనన్యభాక్ |} \\
 & \natline{సాధురేవ స మన్తవ్యః} \\
 & \natline{సమ్యగ్వ్యవసితో హి సః ||}
\end{tabular}
\end{table}

\begin{table}[H]
\begin{tabular}{cl}
\textbf{9.31} & \natline{క్షిప్రమ్ భవతి ధర్మాత్మా} \\
 & \natline{శశ్వచ్ఛాన్తిం నిగచ్ఛతి |} \\
 & \natline{కౌన్తేయ ప్రతిజానీహి} \\
 & \natline{న మే భక్తః ప్రణశ్యతి ||}
\end{tabular}
\end{table}

\begin{table}[H]
\begin{tabular}{cl}
\textbf{9.32} & \natline{మామ్ హి పార్థ వ్యపాశ్రిత్య} \\
 & \natline{యేఽపి స్యుః పాపయోనయః |} \\
 & \natline{స్త్రియో వైశ్యాస్తథా శూద్రాః} \\
 & \natline{తేఽపి యాన్తి పరాం గతిమ్ ||}
\end{tabular}
\end{table}

\begin{table}[H]
\begin{tabular}{cl}
\textbf{9.33} & \natline{కిం పునర్బ్రాహ్మణాః పుణ్యాః} \\
 & \natline{భక్తా రాజర్షయస్తథా |} \\
 & \natline{అనిత్యమసుఖం లోకమ్} \\
 & \natline{ఇమం ప్రాప్య భజస్వ మామ్ ||}
\end{tabular}
\end{table}

\begin{table}[H]
\begin{tabular}{cl}
\textbf{9.34} & \natline{మన్మనా భవ మద్భక్తః} \\
 & \natline{మద్యాజీ మాం నమస్కురు |} \\
 & \natline{మామేవైష్యసి యుక్త్వైవమ్} \\
 & \natline{ఆత్మానం మత్పరాయణః ||}
\end{tabular}
\end{table}


% \end{multicols}

\chapter{Vibhūti Yoga}
% \begin{multicols}{2}
\begin{table}[H]
\begin{tabular}{cl}
 & \natline{श्री परमात्मने नमः} \\
 & \natline{अथ दशमोऽध्यायः} \\
 & \natline{विभुतियोगः}
\end{tabular}
\end{table}

\begin{table}[H]
\begin{tabular}{cl}
\textbf{10.1} & \natline{श्री भगवानुवाच} \\
 & \natline{भूय एव महाबाहो} \\
 & \natline{शृणु मे परमं वचः |} \\
 & \natline{यत्तेऽहं प्रीयमाणाय} \\
 & \natline{वक्ष्यामि हितकाम्यया ||}
\end{tabular}
\end{table}

\begin{table}[H]
\begin{tabular}{cl}
\textbf{10.2} & \natline{न मे विदुः सुरगणाः} \\
 & \natline{प्रभवं न महर्षयः |} \\
 & \natline{अहमादिर्हि देवानां} \\
 & \natline{महर्षीणां च सर्वशः ||}
\end{tabular}
\end{table}

\begin{table}[H]
\begin{tabular}{cl}
\textbf{10.3} & \natline{यो मामजमनादिं च} \\
 & \natline{वेत्ति लोकमहेश्वरम् |} \\
 & \natline{असम्मूढः स मर्त्येषु} \\
 & \natline{सर्वपापैः प्रमुच्यते ||}
\end{tabular}
\end{table}

\begin{table}[H]
\begin{tabular}{cl}
\textbf{10.4} & \natline{बुद्धिर्ज्ञानमसम्मोहः} \\
 & \natline{क्षमा सत्यं दमः शमः |} \\
 & \natline{सुखं दुःखं भवोऽभावः} \\
 & \natline{भयं चाभयमेव च ||}
\end{tabular}
\end{table}

\begin{table}[H]
\begin{tabular}{cl}
\textbf{10.5} & \natline{अहिंसा समता तुष्टिः} \\
 & \natline{तपो दानं यशोऽयशः |} \\
 & \natline{भवन्ति भावा भूतानां} \\
 & \natline{मत्त एव पृथग्विधाः ||}
\end{tabular}
\end{table}

\begin{table}[H]
\begin{tabular}{cl}
\textbf{10.6} & \natline{महर्षयः सप्त पूर्वे} \\
 & \natline{चत्वारो मनवस्तथा |} \\
 & \natline{मद्भावा मानसा जाताः} \\
 & \natline{येषां लोक इमाः प्रजाः ||}
\end{tabular}
\end{table}

\begin{table}[H]
\begin{tabular}{cl}
\textbf{10.7} & \natline{एतां विभूतिं योगं च} \\
 & \natline{मम यो वेत्ति तत्त्वतः |} \\
 & \natline{सोऽविकम्पेन योगेन} \\
 & \natline{युज्यते नात्र संशयः ||}
\end{tabular}
\end{table}

\begin{table}[H]
\begin{tabular}{cl}
\textbf{10.8} & \natline{अहं सर्वस्य प्रभवः} \\
 & \natline{मत्तः सर्वं प्रवर्तते |} \\
 & \natline{इति मत्वा भजन्ते मां} \\
 & \natline{बुधा भावसमन्विताः ||}
\end{tabular}
\end{table}

\begin{table}[H]
\begin{tabular}{cl}
\textbf{10.9} & \natline{मच्चित्ता मद्गतप्राणाः} \\
 & \natline{बोधयन्तः परस्परम् |} \\
 & \natline{कथयन्तश्च मां नित्यं} \\
 & \natline{तुष्यन्ति च रमन्ति च ||}
\end{tabular}
\end{table}

\begin{table}[H]
\begin{tabular}{cl}
\textbf{10.10} & \natline{तेषां सततयुक्तानां} \\
 & \natline{भजतां प्रीतिपूर्वकम् |} \\
 & \natline{ददामि बुद्धियोगं तं} \\
 & \natline{येन मामुपयान्ति ते ||}
\end{tabular}
\end{table}

\begin{table}[H]
\begin{tabular}{cl}
\textbf{10.11} & \natline{तेषामेवानुकम्पार्थम्} \\
 & \natline{अहमज्ञानजं तमः |} \\
 & \natline{नाशयाम्यात्मभावस्थः} \\
 & \natline{ज्ञानदीपेन भास्वता ||}
\end{tabular}
\end{table}

\begin{table}[H]
\begin{tabular}{cl}
\textbf{10.12} & \natline{अर्जुन उवाच} \\
 & \natline{परं ब्रह्म परं धाम} \\
 & \natline{पवित्रं परमं भवान् |} \\
 & \natline{पुरुषं शाश्वतं दिव्यम्} \\
 & \natline{आदिदेवमजं विभुम् ||}
\end{tabular}
\end{table}

\begin{table}[H]
\begin{tabular}{cl}
\textbf{10.13} & \natline{आहुस्त्वामृषयः सर्वे} \\
 & \natline{देवर्षिर्नारदस्तथा |} \\
 & \natline{असितो देवलो व्यासः} \\
 & \natline{स्वयं चैव ब्रवीषि मे ||}
\end{tabular}
\end{table}

\begin{table}[H]
\begin{tabular}{cl}
\textbf{10.14} & \natline{सर्वमेतदृतं मन्ये} \\
 & \natline{यन्मां वदसि केशव |} \\
 & \natline{न हि ते भगवन्व्यक्तिं} \\
 & \natline{विदुर्देवा न दानवाः ||}
\end{tabular}
\end{table}

\begin{table}[H]
\begin{tabular}{cl}
\textbf{10.15} & \natline{स्वयमेवात्मनाऽऽत्मानं} \\
 & \natline{वेत्थ त्वं पुरुषोत्तम |} \\
 & \natline{भूतभावन भूतेश} \\
 & \natline{देवदेव जगत्पते ||}
\end{tabular}
\end{table}

\begin{table}[H]
\begin{tabular}{cl}
\textbf{10.16} & \natline{वक्तुमर्हस्यशेषेण} \\
 & \natline{दिव्या ह्यात्मविभूतयः |} \\
 & \natline{याभिर्विभूतिभिर्लोकान्} \\
 & \natline{इमांस्त्वं व्याप्य तिष्ठसि ||}
\end{tabular}
\end{table}

\begin{table}[H]
\begin{tabular}{cl}
\textbf{10.17} & \natline{कथं विद्यामहं योगिन्} \\
 & \natline{त्वां सदा परिचिन्तयन् |} \\
 & \natline{केषु केषु च भावेषु} \\
 & \natline{चिन्त्योऽसि भगवन्मया ||}
\end{tabular}
\end{table}

\begin{table}[H]
\begin{tabular}{cl}
\textbf{10.18} & \natline{विस्तरेणात्मनो योगं} \\
 & \natline{विभूतिं च जनार्दन |} \\
 & \natline{भूयः कथय तृप्तिर्हि} \\
 & \natline{शृण्वतो नास्ति मेऽमृतम् ||}
\end{tabular}
\end{table}

\begin{table}[H]
\begin{tabular}{cl}
\textbf{10.19} & \natline{श्री भगवानुवाच} \\
 & \natline{हन्त ते कथयिष्यामि} \\
 & \natline{दिव्या ह्यात्मविभूतयः |} \\
 & \natline{प्राधान्यतः कुरुश्रेष्ठ} \\
 & \natline{नास्त्यन्तो विस्तरस्य मे ||}
\end{tabular}
\end{table}

\begin{table}[H]
\begin{tabular}{cl}
\textbf{10.20} & \natline{अहमात्मा गुडाकेश} \\
 & \natline{सर्वभूताशयस्थितः |} \\
 & \natline{अहमादिश्च मध्यं च} \\
 & \natline{भूतानामन्त एव च ||}
\end{tabular}
\end{table}

\begin{table}[H]
\begin{tabular}{cl}
\textbf{10.21} & \natline{आदित्यानामहम् विष्णुः} \\
 & \natline{ज्योतिषां रविरंशुमान् |} \\
 & \natline{मरीचिर्मरुतामस्मि} \\
 & \natline{नक्षत्राणामहं शशी ||}
\end{tabular}
\end{table}

\begin{table}[H]
\begin{tabular}{cl}
\textbf{10.22} & \natline{वेदानां सामवेदोऽस्मि} \\
 & \natline{देवानामस्मि वासवः |} \\
 & \natline{इन्द्रियाणां मनश्चास्मि} \\
 & \natline{भूतानामस्मि चेतना ||}
\end{tabular}
\end{table}

\begin{table}[H]
\begin{tabular}{cl}
\textbf{10.23} & \natline{रुद्राणां शङ्करश्चास्मि} \\
 & \natline{वित्तेशो यक्षरक्षसाम् |} \\
 & \natline{वसूनां पावकश्चास्मि} \\
 & \natline{मेरुः शिखरिणामहम् ||}
\end{tabular}
\end{table}

\begin{table}[H]
\begin{tabular}{cl}
\textbf{10.24} & \natline{पुरोधसां च मुख्यं मां} \\
 & \natline{विद्धि पार्थ बृहस्पतिम् |} \\
 & \natline{सेनानीनामहं स्कन्दः} \\
 & \natline{सरसामस्मि सागरः ||}
\end{tabular}
\end{table}

\begin{table}[H]
\begin{tabular}{cl}
\textbf{10.25} & \natline{महर्षीणां भृगुरहं} \\
 & \natline{गिरामस्म्येकमक्षरम् |} \\
 & \natline{यज्ञानां जपयज्ञोऽस्मि} \\
 & \natline{स्थावराणां हिमालयः ||}
\end{tabular}
\end{table}

\begin{table}[H]
\begin{tabular}{cl}
\textbf{10.26} & \natline{अश्वत्थः सर्ववृक्षाणां} \\
 & \natline{देवर्षीणां च नारदः |} \\
 & \natline{गन्धर्वाणां चित्ररथः} \\
 & \natline{सिद्धानां कपिलो मुनिः ||}
\end{tabular}
\end{table}

\begin{table}[H]
\begin{tabular}{cl}
\textbf{10.27} & \natline{उच्चैः श्रवसमश्वानां} \\
 & \natline{विद्धि माममृतोद्भवम् |} \\
 & \natline{ऐरावतं गजेन्द्राणां} \\
 & \natline{नराणां च नराधिपम् ||}
\end{tabular}
\end{table}

\begin{table}[H]
\begin{tabular}{cl}
\textbf{10.28} & \natline{आयुधानामहं वज्रं} \\
 & \natline{धेनूनामस्मि कामधुक् |} \\
 & \natline{प्रजनश्चास्मि कन्दर्पः} \\
 & \natline{सर्पाणामस्मि वासुकिः ||}
\end{tabular}
\end{table}

\begin{table}[H]
\begin{tabular}{cl}
\textbf{10.29} & \natline{अनन्तश्चास्मि नागानां} \\
 & \natline{वरुणो यादसामहम् |} \\
 & \natline{पितॄणामर्यमा चास्मि} \\
 & \natline{यमः संयमतामहम् ||}
\end{tabular}
\end{table}

\begin{table}[H]
\begin{tabular}{cl}
\textbf{10.30} & \natline{प्रह्लादश्चास्मि दैत्यानां} \\
 & \natline{कालः कलयतामहम् |} \\
 & \natline{मृगाणां च मृगेन्द्रोऽहं} \\
 & \natline{वैनतेयश्च पक्षिणाम् ||}
\end{tabular}
\end{table}

\begin{table}[H]
\begin{tabular}{cl}
\textbf{10.31} & \natline{पवनः पवतामस्मि} \\
 & \natline{रामः शस्त्रभृतामहम् |} \\
 & \natline{झषाणां मकरश्चास्मि} \\
 & \natline{स्रोतसामस्मि जाह्नवी ||}
\end{tabular}
\end{table}

\begin{table}[H]
\begin{tabular}{cl}
\textbf{10.32} & \natline{सर्गाणामादिरन्तश्च} \\
 & \natline{मध्यं चैवाहमर्जुन |} \\
 & \natline{अध्यात्मविद्या विद्यानां} \\
 & \natline{वादः प्रवदतामहम् ||}
\end{tabular}
\end{table}

\begin{table}[H]
\begin{tabular}{cl}
\textbf{10.33} & \natline{अक्षराणामकारोऽस्मि} \\
 & \natline{द्वन्द्वः सामासिकस्य च |} \\
 & \natline{अहमेवाक्षयः कालः} \\
 & \natline{धाताऽहं विश्वतोमुखः ||}
\end{tabular}
\end{table}

\begin{table}[H]
\begin{tabular}{cl}
\textbf{10.34} & \natline{मृत्युः सर्वहरश्चाहम्} \\
 & \natline{उद्भवश्च भविष्यताम् |} \\
 & \natline{कीर्तिः श्रीर्वाक्च नारीणां} \\
 & \natline{स्मृतिर्मेधा धृतिः क्षमा ||}
\end{tabular}
\end{table}

\begin{table}[H]
\begin{tabular}{cl}
\textbf{10.35} & \natline{बृहत्साम तथा साम्नां} \\
 & \natline{गायत्री छन्दसामहम् |} \\
 & \natline{मासानां मार्गशीर्षोऽहम्} \\
 & \natline{ऋतूनां कुसुमाकरः ||}
\end{tabular}
\end{table}

\begin{table}[H]
\begin{tabular}{cl}
\textbf{10.36} & \natline{द्यूतं छलयतामस्मि} \\
 & \natline{तेजस्तेजस्विनामहम् |} \\
 & \natline{जयोऽस्मि व्यवसायोऽस्मि} \\
 & \natline{सत्त्वं सत्त्ववतामहम् ||}
\end{tabular}
\end{table}

\begin{table}[H]
\begin{tabular}{cl}
\textbf{10.37} & \natline{वृष्णीनां वासुदेवोऽस्मि} \\
 & \natline{पाण्डवानां धनञ्जयः |} \\
 & \natline{मुनीनामप्यहं व्यासः} \\
 & \natline{कवीनामुशना कविः ||}
\end{tabular}
\end{table}

\begin{table}[H]
\begin{tabular}{cl}
\textbf{10.38} & \natline{दण्डो दमयतामस्मि} \\
 & \natline{नीतिरस्मि जिगीषताम् |} \\
 & \natline{मौनं चैवास्मि गुह्यानां} \\
 & \natline{ज्ञानं ज्ञानवतामहम् ||}
\end{tabular}
\end{table}

\begin{table}[H]
\begin{tabular}{cl}
\textbf{10.39} & \natline{यच्चापि सर्वभूतानां} \\
 & \natline{बीजं तदहमर्जुन |} \\
 & \natline{न तदस्ति विना यत्स्यात्} \\
 & \natline{मया भूतं चराचरम् ||}
\end{tabular}
\end{table}

\begin{table}[H]
\begin{tabular}{cl}
\textbf{10.40} & \natline{नान्तोऽस्ति मम दिव्यानां} \\
 & \natline{विभूतीनां परन्तप |} \\
 & \natline{एष तूद्देशतः प्रोक्तः} \\
 & \natline{विभूतेर्विस्तरो मया ||}
\end{tabular}
\end{table}

\begin{table}[H]
\begin{tabular}{cl}
\textbf{10.41} & \natline{यद्यद्विभूतिमत्सत्त्वं} \\
 & \natline{श्रीमदूर्जितमेव वा |} \\
 & \natline{तत्तदेवावगच्छ त्वं} \\
 & \natline{मम तेजोऽम्शसम्भवम् ||}
\end{tabular}
\end{table}

\begin{table}[H]
\begin{tabular}{cl}
\textbf{10.42} & \natline{अथवा बहुनैतेन} \\
 & \natline{किं ज्ञातेन तवार्जुन |} \\
 & \natline{विष्टभ्याहमिदं कृत्स्नम्} \\
 & \natline{एकांशेन स्थितो जगत् ||}
\end{tabular}
\end{table}

\begin{table}[H]
\begin{tabular}{cl}
 & \natline{श्रीमद्भगवद्गीतासु उपनिषत्सु} \\
 & \natline{ब्रह्मविद्यायां योगशास्त्रे} \\
 & \natline{श्रीकृष्णार्जुन संवादे} \\
 & \natline{विभुतियोगो नाम} \\
 & \natline{दशमोध्यायः}
\end{tabular}
\end{table}


% \end{multicols}

\chapter{Viśvarūpa-Sandarśana Yoga}
% \begin{multicols}{2}
\begin{table}[H]
\begin{tabular}{cl}
\textbf{11.0} & \natline{ఓం శ్రీ పరమాత్మనే నమః} \\
 & \natline{అథ ఏకాదశోఽధ్యాయః} \\
 & \natline{విశ్వరూప సన్దర్శన యోగః}
\end{tabular}
\end{table}

\begin{table}[H]
\begin{tabular}{cl}
\textbf{11.1} & \natline{అర్జున ఉవాచ} \\
 & \natline{మదనుగ్రహాయ పరమం} \\
 & \natline{గుహ్యమధ్యాత్మసఞ్జ్ఞితమ్ |} \\
 & \natline{యత్త్వయోక్తం వచస్తేన} \\
 & \natline{మోహోఽయం విగతో మమ ||}
\end{tabular}
\end{table}

\begin{table}[H]
\begin{tabular}{cl}
\textbf{11.2} & \natline{భవాప్యయౌ హి భూతానాం} \\
 & \natline{శ్రుతౌ విస్తరశో మయా |} \\
 & \natline{త్వత్తః కమలపత్రాక్ష} \\
 & \natline{మాహాత్మ్యమపి చావ్యయమ్ ||}
\end{tabular}
\end{table}

\begin{table}[H]
\begin{tabular}{cl}
\textbf{11.3} & \natline{ఏవమేతద్యథాఽఽత్థ త్వమ్} \\
 & \natline{ఆత్మానం పరమేశ్వర |} \\
 & \natline{ద్రష్టుమిచ్ఛామి తే రూపమ్} \\
 & \natline{ఐశ్వరమ్ పురుషోత్తమ ||}
\end{tabular}
\end{table}

\begin{table}[H]
\begin{tabular}{cl}
\textbf{11.4} & \natline{మన్యసే యది తచ్ఛక్యం} \\
 & \natline{మయా ద్రష్టుమితి ప్రభో |} \\
 & \natline{యోగేశ్వర తతో మే త్వం} \\
 & \natline{దర్శయాత్మానమవ్యయమ్ ||}
\end{tabular}
\end{table}

\begin{table}[H]
\begin{tabular}{cl}
\textbf{11.5} & \natline{శ్రీ భగవానువాచ} \\
 & \natline{పశ్య మే పార్థ రూపాణి} \\
 & \natline{శతశోఽథ సహస్రశః |} \\
 & \natline{నానావిధాని దివ్యాని} \\
 & \natline{నానావర్ణాకృతీని చ ||}
\end{tabular}
\end{table}

\begin{table}[H]
\begin{tabular}{cl}
\textbf{11.6} & \natline{పశ్యాదిత్యాన్వసూన్రుద్రాన్} \\
 & \natline{అశ్వినౌ మరుతస్తథా |} \\
 & \natline{బహూన్యదృష్టపూర్వాణి} \\
 & \natline{పశ్యాశ్చర్యాణి భారత ||}
\end{tabular}
\end{table}

\begin{table}[H]
\begin{tabular}{cl}
\textbf{11.7} & \natline{ఇహైకస్థం జగత్కృత్స్నం} \\
 & \natline{పశ్యాద్య సచరాచరమ్ |} \\
 & \natline{మమ దేహే గుడాకేశ} \\
 & \natline{యచ్చాన్యత్ ద్రష్టుమిచ్ఛసి ||}
\end{tabular}
\end{table}

\begin{table}[H]
\begin{tabular}{cl}
\textbf{11.8} & \natline{న తు మాం శక్యసే ద్రష్టుమ్} \\
 & \natline{అనేనైవ స్వచక్షుషా |} \\
 & \natline{దివ్యం దదామి తే చక్షుః} \\
 & \natline{పశ్య మే యోగమైశ్వరమ్ ||}
\end{tabular}
\end{table}

\begin{table}[H]
\begin{tabular}{cl}
\textbf{11.9} & \natline{సఞ్జయ ఉవాచ} \\
 & \natline{ఏవముక్త్వా తతో రాజన్} \\
 & \natline{మహాయోగేశ్వరో హరిః |} \\
 & \natline{దర్శయామాస పార్థాయ} \\
 & \natline{పరమం రూపమైశ్వరమ్ ||}
\end{tabular}
\end{table}

\begin{table}[H]
\begin{tabular}{cl}
\textbf{11.10} & \natline{అనేకవక్త్రనయనమ్} \\
 & \natline{అనేకాద్భుతదర్శనమ్ |} \\
 & \natline{అనేకదివ్యాభరణం} \\
 & \natline{దివ్యానేకోద్యతాయుధమ్ ||}
\end{tabular}
\end{table}

\begin{table}[H]
\begin{tabular}{cl}
\textbf{11.11} & \natline{దివ్యమాల్యామ్బరధరం} \\
 & \natline{దివ్యగన్ధానులేపనమ్ |} \\
 & \natline{సర్వాశ్చర్యమయం దేవమ్} \\
 & \natline{అనన్తం విశ్వతోముఖమ్ ||}
\end{tabular}
\end{table}

\begin{table}[H]
\begin{tabular}{cl}
\textbf{11.12} & \natline{దివి సూర్యసహస్రస్య} \\
 & \natline{భవేద్యుగపదుత్థితా |} \\
 & \natline{యది భాః సదృశీ సా స్యాత్} \\
 & \natline{భాసస్తస్య మహాత్మనః ||}
\end{tabular}
\end{table}

\begin{table}[H]
\begin{tabular}{cl}
\textbf{11.13} & \natline{తత్రైకస్థం జగత్కృత్స్నం} \\
 & \natline{ప్రవిభక్తమనేకధా |} \\
 & \natline{అపశ్యద్దేవదేవస్య} \\
 & \natline{శరీరే పాణ్డవస్తదా ||}
\end{tabular}
\end{table}

\begin{table}[H]
\begin{tabular}{cl}
\textbf{11.14} & \natline{తతః స విస్మయావిష్టః} \\
 & \natline{హృష్టరోమా ధనఞ్జయః |} \\
 & \natline{ప్రణమ్య శిరసా దేవం} \\
 & \natline{కృతాఞ్జలిరభాషత ||}
\end{tabular}
\end{table}

\begin{table}[H]
\begin{tabular}{cl}
\textbf{11.15} & \natline{అర్జున ఉవాచ} \\
 & \natline{పశ్యామి దేవాంస్తవ దేవ దేహే} \\
 & \natline{సర్వాంస్తథా భూతవిశేషసఙ్ఘాన్ |} \\
 & \natline{బ్రహ్మాణమీశం కమలాసనస్థమ్} \\
 & \natline{ఋషీంశ్చ సర్వానురగాంశ్చ దివ్యాన్ ||}
\end{tabular}
\end{table}

\begin{table}[H]
\begin{tabular}{cl}
\textbf{11.16} & \natline{అనేకబాహూదరవక్త్రనేత్రం} \\
 & \natline{పశ్యామి త్వా సర్వతోఽనన్తరూపమ్ |} \\
 & \natline{నాన్తం న మధ్యం న పునస్తవాదిం} \\
 & \natline{పశ్యామి విశ్వేశ్వర విశ్వరూప ||}
\end{tabular}
\end{table}

\begin{table}[H]
\begin{tabular}{cl}
\textbf{11.17} & \natline{కిరీటినం గదినం చక్రిణం చ} \\
 & \natline{తేజోరాశిం సర్వతో దీప్తిమన్తమ్ |} \\
 & \natline{పశ్యామి త్వాం దుర్నిరీక్ష్యం సమన్తాత్} \\
 & \natline{దీప్తానలార్కద్యుతిమప్రమేయమ్ ||}
\end{tabular}
\end{table}

\begin{table}[H]
\begin{tabular}{cl}
\textbf{11.18} & \natline{త్వమక్షరం పరమం వేదితవ్యం} \\
 & \natline{త్వమస్య విశ్వస్య పరం నిధానమ్ |} \\
 & \natline{త్వమవ్యయః శాశ్వతధర్మగోప్తా} \\
 & \natline{సనాతనస్త్వం పురుషో మతో మే ||}
\end{tabular}
\end{table}

\begin{table}[H]
\begin{tabular}{cl}
\textbf{11.19} & \natline{అనాదిమధ్యాన్తమనన్తవీర్యమ్} \\
 & \natline{అనన్తబాహుం శశిసూర్యనేత్రమ్ |} \\
 & \natline{పశ్యామి త్వాం దీప్తహుతాశవక్త్రం} \\
 & \natline{స్వతేజసా విశ్వమిదం తపన్తమ్ ||}
\end{tabular}
\end{table}

\begin{table}[H]
\begin{tabular}{cl}
\textbf{11.20} & \natline{ద్యావాపృథివ్యోరిదమన్తరం హి} \\
 & \natline{వ్యాప్తం త్వయైకేన దిశశ్చ సర్వాః |} \\
 & \natline{దృష్ట్వాద్భుతం రూపమిదం తవోగ్రం} \\
 & \natline{లోకత్రయం ప్రవ్యథితం మహాత్మన్ ||}
\end{tabular}
\end{table}

\begin{table}[H]
\begin{tabular}{cl}
\textbf{11.21} & \natline{అమీ హి త్వా సురసఙ్ఘా విశన్తి} \\
 & \natline{కేచిద్భీతాః ప్రాఞ్జలయో గృణన్తి |} \\
 & \natline{స్వస్తీత్యుక్త్వా మహర్షిసిద్ధసఙ్ఘాః} \\
 & \natline{స్తువన్తి త్వాం స్తుతిభిః పుష్కలాభిః ||}
\end{tabular}
\end{table}

\begin{table}[H]
\begin{tabular}{cl}
\textbf{11.22} & \natline{రుద్రాదిత్యా వసవో యే చ సాధ్యాః} \\
 & \natline{విశ్వేఽశ్వినౌ మరుతశ్చోష్మపాశ్చ |} \\
 & \natline{గన్ధర్వయక్షాసురసిద్ధసఙ్ఘాః} \\
 & \natline{వీక్షన్తే త్వాం విస్మితాశ్చైవ సర్వే ||}
\end{tabular}
\end{table}

\begin{table}[H]
\begin{tabular}{cl}
\textbf{11.23} & \natline{రూపం మహత్తే బహువక్త్ర నేత్రం} \\
 & \natline{మహాబాహో బహుబాహూరుపాదమ్ |} \\
 & \natline{బహూదరం బహుదంష్ట్రాకరాలం} \\
 & \natline{దృష్ట్వా లోకాః ప్రవ్యథితాస్తథాఽహమ్ ||}
\end{tabular}
\end{table}

\begin{table}[H]
\begin{tabular}{cl}
\textbf{11.24} & \natline{నభః స్పృశం దీప్తమనేకవర్ణం} \\
 & \natline{వ్యాత్తాననం దీప్తవిశాలనేత్రమ్ |} \\
 & \natline{దృష్ట్వా హి త్వాం ప్రవ్యథితాన్తరాత్మా} \\
 & \natline{ధృతిం న విన్దామి శమం చ విష్ణో ||}
\end{tabular}
\end{table}

\begin{table}[H]
\begin{tabular}{cl}
\textbf{11.25} & \natline{దంష్ట్రాకరాలాని చ తే ముఖాని} \\
 & \natline{దృష్ట్వైవ కాలానలసన్నిభాని |} \\
 & \natline{దిశో న జానే న లభే చ శర్మ} \\
 & \natline{ప్రసీద దేవేశ జగన్నివాస ||}
\end{tabular}
\end{table}

\begin{table}[H]
\begin{tabular}{cl}
\textbf{11.26} & \natline{అమీ చ త్వాం ధృతరాష్ట్రస్య పుత్రాః} \\
 & \natline{సర్వే సహైవావనిపాలసఙ్ఘైః |} \\
 & \natline{భీష్మో ద్రోణః సూతపుత్రస్తథాఽసౌ} \\
 & \natline{సహాస్మదీయైరపి యోధముఖ్యైః ||}
\end{tabular}
\end{table}

\begin{table}[H]
\begin{tabular}{cl}
\textbf{11.27} & \natline{వక్త్రాణి తే త్వరమాణా విశన్తి} \\
 & \natline{దంష్ట్రాకరాలాని భయానకాని |} \\
 & \natline{కేచిద్విలగ్నా దశనాన్తరేషు} \\
 & \natline{సన్దృశ్యన్తే చూర్ణితైరుత్తమాఙ్గైః ||}
\end{tabular}
\end{table}

\begin{table}[H]
\begin{tabular}{cl}
\textbf{11.28} & \natline{యథా నదీనాం బహవోఽమ్బువేగాః} \\
 & \natline{సముద్రమేవాభిముఖా ద్రవన్తి |} \\
 & \natline{తథా తవామీ నరలోకవీరాః} \\
 & \natline{విశన్తి వక్త్రాణ్యభివిజ్వలన్తి ||}
\end{tabular}
\end{table}

\begin{table}[H]
\begin{tabular}{cl}
\textbf{11.29} & \natline{యథా ప్రదీప్తం జ్వలనం పతఙ్గాః} \\
 & \natline{విశన్తి నాశాయ సమృద్ధవేగాః |} \\
 & \natline{తథైవ నాశాయ విశన్తి లోకాః} \\
 & \natline{తవాపి వక్త్రాణి సమృద్ధవేగాః ||}
\end{tabular}
\end{table}

\begin{table}[H]
\begin{tabular}{cl}
\textbf{11.30} & \natline{లేలిహ్యసే గ్రసమానః సమన్తాత్} \\
 & \natline{లోకాన్సమగ్రాన్వదనైర్జ్వలద్భిః |} \\
 & \natline{తేజోభిరాపూర్య జగత్సమగ్రం} \\
 & \natline{భాసస్తవోగ్రాః ప్రతపన్తి విష్ణో ||}
\end{tabular}
\end{table}

\begin{table}[H]
\begin{tabular}{cl}
\textbf{11.31} & \natline{ఆఖ్యాహి మే కో భవానుగ్రరూపః} \\
 & \natline{నమోఽస్తు తే దేవవర ప్రసీద |} \\
 & \natline{విజ్ఞాతుమిచ్ఛామి భవన్తమాద్యం} \\
 & \natline{న హి ప్రజానామి తవ ప్రవృత్తిమ్ ||}
\end{tabular}
\end{table}

\begin{table}[H]
\begin{tabular}{cl}
\textbf{11.32} & \natline{శ్రీ భగవానువాచ} \\
 & \natline{కాలోఽస్మి లోకక్షయకృత్ప్రవృద్ధః} \\
 & \natline{లోకాన్సమాహర్తుమిహ ప్రవృత్తః |} \\
 & \natline{ఋతేఽపి త్వా న భవిష్యన్తి సర్వే} \\
 & \natline{యేఽవస్థితాః ప్రత్యనీకేషు యోధాః ||}
\end{tabular}
\end{table}

\begin{table}[H]
\begin{tabular}{cl}
\textbf{11.33} & \natline{తస్మాత్త్వముత్తిష్ఠ యశో లభస్వ} \\
 & \natline{జిత్వా శత్రూన్భుఙ్క్ష్వ రాజ్యం సమృద్ధమ్ |} \\
 & \natline{మయైవైతే నిహతాః పూర్వమేవ} \\
 & \natline{నిమిత్తమాత్రం భవ సవ్యసాచిన్ ||}
\end{tabular}
\end{table}

\begin{table}[H]
\begin{tabular}{cl}
\textbf{11.34} & \natline{ద్రోణం చ భీష్మం చ జయద్రథం చ} \\
 & \natline{కర్ణం తథాన్యానపి యోధవీరాన్ |} \\
 & \natline{మయా హతాంస్త్వం జహి మా వ్యథిష్ఠాః} \\
 & \natline{యుధ్యస్వ జేతాసి రణే సపత్నాన్ ||}
\end{tabular}
\end{table}

\begin{table}[H]
\begin{tabular}{cl}
\textbf{11.35} & \natline{సఞ్జయ ఉవాచ} \\
 & \natline{ఏతచ్ఛ్రుత్వా వచనం కేశవస్య} \\
 & \natline{కృతాఞ్జలిర్వేపమానః కిరీటీ |} \\
 & \natline{నమస్కృత్వా భూయ ఏవాహ కృష్ణం} \\
 & \natline{సగద్గదం భీతభీతః ప్రణమ్య ||}
\end{tabular}
\end{table}

\begin{table}[H]
\begin{tabular}{cl}
\textbf{11.36} & \natline{అర్జున ఉవాచ} \\
 & \natline{స్థానే హృషీకేశ తవ ప్రకీర్త్యా} \\
 & \natline{జగత్ప్రహృష్యత్యనురజ్యతే చ |} \\
 & \natline{రక్షాంసి భీతాని దిశో ద్రవన్తి} \\
 & \natline{సర్వే నమస్యన్తి చ సిద్ధసఙ్ఘాః ||}
\end{tabular}
\end{table}

\begin{table}[H]
\begin{tabular}{cl}
\textbf{11.37} & \natline{కస్మాచ్చ తే న నమేరన్మహాత్మన్} \\
 & \natline{గరీయసే బ్రహ్మణోఽప్యాదికర్త్రే |} \\
 & \natline{అనన్త దేవేశ జగన్నివాస} \\
 & \natline{త్వమక్షరం సదసత్తత్పరం యత్ ||}
\end{tabular}
\end{table}

\begin{table}[H]
\begin{tabular}{cl}
\textbf{11.38} & \natline{త్వమాదిదేవః పురుషః పురాణః} \\
 & \natline{త్వమస్య విశ్వస్య పరం నిధానమ్ |} \\
 & \natline{వేత్తాఽసి వేద్యం చ పరం చ ధామ} \\
 & \natline{త్వయా తతం విశ్వమనన్తరూప ||}
\end{tabular}
\end{table}

\begin{table}[H]
\begin{tabular}{cl}
\textbf{11.39} & \natline{వాయుర్యమోఽగ్నిర్వరుణః శశాఙ్కః} \\
 & \natline{ప్రజాపతిస్త్వం ప్రపితామహశ్చ |} \\
 & \natline{నమో నమస్తేఽస్తు సహస్రకృత్వః} \\
 & \natline{పునశ్చ భూయోఽపి నమో నమస్తే ||}
\end{tabular}
\end{table}

\begin{table}[H]
\begin{tabular}{cl}
\textbf{11.40} & \natline{నమః పురస్తాదథ పృష్ఠతస్తే} \\
 & \natline{నమోఽస్తు తే సర్వత ఏవ సర్వ |} \\
 & \natline{అనన్తవీర్యామితవిక్రమస్త్వం} \\
 & \natline{సర్వం సమాప్నోషి తతోఽసి సర్వః ||}
\end{tabular}
\end{table}

\begin{table}[H]
\begin{tabular}{cl}
\textbf{11.41} & \natline{సఖేతి మత్వా ప్రసభం యదుక్తం} \\
 & \natline{హే కృష్ణ హే యాదవ హే సఖేతి |} \\
 & \natline{అజానతా మహిమానం తవేదం} \\
 & \natline{మయా ప్రమాదాత్ప్రణయేన వాఽపి ||}
\end{tabular}
\end{table}

\begin{table}[H]
\begin{tabular}{cl}
\textbf{11.42} & \natline{యచ్చాపహాసార్థమసత్కృతోఽసి} \\
 & \natline{విహారశయ్యాసనభోజనేషు |} \\
 & \natline{ఏకోఽథవాప్యచ్యుత తత్సమక్షం} \\
 & \natline{తత్క్షామయే త్వామహమప్రమేయమ్ ||}
\end{tabular}
\end{table}

\begin{table}[H]
\begin{tabular}{cl}
\textbf{11.43} & \natline{పితాసి లోకస్య చరాచరస్య} \\
 & \natline{త్వమస్య పూజ్యశ్చ గురుర్గరీయాన్ |} \\
 & \natline{న త్వత్సమోఽస్త్యభ్యధికః కుతోఽన్యః} \\
 & \natline{లోకత్రయేఽప్యప్రతిమప్రభావ ||}
\end{tabular}
\end{table}

\begin{table}[H]
\begin{tabular}{cl}
\textbf{11.44} & \natline{తస్మాత్ప్రణమ్య ప్రణిధాయ కాయం} \\
 & \natline{ప్రసాదయే త్వామహమీశమీడ్యమ్ |} \\
 & \natline{పితేవ పుత్రస్య సఖేవ సఖ్యుః} \\
 & \natline{ప్రియః ప్రియాయార్హసి దేవ సోఢుమ్ ||}
\end{tabular}
\end{table}

\begin{table}[H]
\begin{tabular}{cl}
\textbf{11.45} & \natline{అదృష్టపూర్వం హృషితోఽస్మి దృష్ట్వా} \\
 & \natline{భయేన చ ప్రవ్యథితం మనో మే |} \\
 & \natline{తదేవ మే దర్శయ దేవరూపం} \\
 & \natline{ప్రసీద దేవేశ జగన్నివాస ||}
\end{tabular}
\end{table}

\begin{table}[H]
\begin{tabular}{cl}
\textbf{11.46} & \natline{కిరీటినం గదినం చక్రహస్తమ్} \\
 & \natline{ఇచ్ఛామి త్వాం ద్రష్టుమహం తథైవ |} \\
 & \natline{తేనైవ రూపేణ చతుర్భుజేన} \\
 & \natline{సహస్రబాహో భవ విశ్వమూర్తే ||}
\end{tabular}
\end{table}

\begin{table}[H]
\begin{tabular}{cl}
\textbf{11.47} & \natline{శ్రీ భగవానువాచ} \\
 & \natline{మయా ప్రసన్నేన తవార్జునేదం} \\
 & \natline{రూపం పరం దర్శితమాత్మయోగాత్ |} \\
 & \natline{తేజోమయం విశ్వమనన్తమాద్యం} \\
 & \natline{యన్మే త్వదన్యేన న దృష్టపూర్వమ్ ||}
\end{tabular}
\end{table}

\begin{table}[H]
\begin{tabular}{cl}
\textbf{11.48} & \natline{న వేదయజ్ఞాధ్యయనైర్న దానైః} \\
 & \natline{న చ క్రియాభిర్న తపోభిరుగ్రైః |} \\
 & \natline{ఏవంరూపః శక్య అహం నృలోకే} \\
 & \natline{ద్రష్టుం త్వదన్యేన కురుప్రవీర ||}
\end{tabular}
\end{table}

\begin{table}[H]
\begin{tabular}{cl}
\textbf{11.49} & \natline{మా తే వ్యథా మా చ విమూఢభావః} \\
 & \natline{దృష్ట్వా రూపం ఘోరమీదృఙ్మమేదమ్ |} \\
 & \natline{వ్యపేతభీః ప్రీతమనాః పునస్త్వం} \\
 & \natline{తదేవ మే రూపమిదం ప్రపశ్య ||}
\end{tabular}
\end{table}

\begin{table}[H]
\begin{tabular}{cl}
\textbf{11.50} & \natline{సఞ్జయ ఉవాచ} \\
 & \natline{ఇత్యర్జునం వాసుదేవస్తథోక్త్వా} \\
 & \natline{స్వకం రూపం దర్శయామాస భూయః |} \\
 & \natline{ఆశ్వాసయామాస చ భీతమేనం} \\
 & \natline{భూత్వా పునః సౌమ్యవపుర్మహాత్మా ||}
\end{tabular}
\end{table}

\begin{table}[H]
\begin{tabular}{cl}
\textbf{11.51} & \natline{అర్జున ఉవాచ} \\
 & \natline{దృష్ట్వేదం మానుషం రూపం} \\
 & \natline{తవ సౌమ్యం జనార్దన |} \\
 & \natline{ఇదానీమస్మి సంవృత్తః} \\
 & \natline{సచేతాః ప్రకృతిం గతః ||}
\end{tabular}
\end{table}

\begin{table}[H]
\begin{tabular}{cl}
\textbf{11.52} & \natline{శ్రీ భగవానువాచ} \\
 & \natline{సుదుర్దర్శమిదం రూపం} \\
 & \natline{దృష్టవానసి యన్మమ |} \\
 & \natline{దేవా అప్యస్య రూపస్య} \\
 & \natline{నిత్యం దర్శనకాఙ్క్షిణః ||}
\end{tabular}
\end{table}

\begin{table}[H]
\begin{tabular}{cl}
\textbf{11.53} & \natline{నాహం వేదైర్న తపసా} \\
 & \natline{న దానేన న చేజ్యయా |} \\
 & \natline{శక్య ఏవంవిధో ద్రష్టుం} \\
 & \natline{దృష్టవానసి మాం యథా ||}
\end{tabular}
\end{table}

\begin{table}[H]
\begin{tabular}{cl}
\textbf{11.54} & \natline{భక్త్యా త్వనన్యయా శక్యః} \\
 & \natline{అహమేవంవిధోఽర్జున |} \\
 & \natline{జ్ఞాతుం ద్రష్టుం చ తత్త్వేన} \\
 & \natline{ప్రవేష్టుం చ పరన్తప ||}
\end{tabular}
\end{table}

\begin{table}[H]
\begin{tabular}{cl}
\textbf{11.55} & \natline{మత్కర్మకృన్మత్పరమః} \\
 & \natline{మద్భక్తః సఙ్గవర్జితః |} \\
 & \natline{నిర్వైరః సర్వభూతేషు} \\
 & \natline{యః స మామేతి పాణ్డవ ||}
\end{tabular}
\end{table}


% \end{multicols}

\chapter{Bhakti Yoga}
% \begin{multicols}{2}
\begin{table}[H]
\begin{tabular}{cl}
\textbf{12.0} & \romline{oṃ śrī paramātmane namaḥ} \\
 & \romline{atha dvādaśo'dhyāyaḥ} \\
 & \romline{bhakti-yogaḥ}
\end{tabular}
\end{table}

\begin{table}[H]
\begin{tabular}{cl}
\textbf{12.1} & \romline{arjuna uvāca} \\
 & \romline{evaṃ satatayuktā ye} \\
 & \romline{bhaktāstvāṃ paryupāsate |} \\
 & \romline{ye cāpyakṣaram-avyaktaṃ} \\
 & \romline{teṣāṃ ke yogavittamāḥ ||}
\end{tabular}
\end{table}

\begin{table}[H]
\begin{tabular}{cl}
\textbf{12.2} & \romline{śrī bhagavān uvāca} \\
 & \romline{mayyāveśya mano ye māṃ} \\
 & \romline{nityayuktā upāsate |} \\
 & \romline{śraddhayā parayopetāḥ} \\
 & \romline{te me yuktatamā matāḥ ||}
\end{tabular}
\end{table}

\begin{table}[H]
\begin{tabular}{cl}
\textbf{12.3} & \romline{ye tvakṣaram-anirdeśyam} \\
 & \romline{avyaktaṃ paryupāsate |} \\
 & \romline{sarvatragam-acintyam ca} \\
 & \romline{kūṭastha-macalam dhruvaṃ ||}
\end{tabular}
\end{table}

\begin{table}[H]
\begin{tabular}{cl}
\textbf{12.4} & \romline{sanniyamyendriya-grāmaṃ} \\
 & \romline{sarvatra samabuddhayaḥ |} \\
 & \romline{te prāpnuvanti māmeva} \\
 & \romline{sarvabhūtahite ratāḥ ||}
\end{tabular}
\end{table}

\begin{table}[H]
\begin{tabular}{cl}
\textbf{12.5} & \romline{kleśo'dhika-tarasteṣām} \\
 & \romline{avyaktāsakta-cetasām |} \\
 & \romline{avyaktā hi gatirduḥkhaṃ} \\
 & \romline{dehavad-bhiravāpyate ||}
\end{tabular}
\end{table}

\begin{table}[H]
\begin{tabular}{cl}
\textbf{12.6} & \romline{ye tu sarvāṇi karmāṇi} \\
 & \romline{mayi sannyasya matparāḥ |} \\
 & \romline{ananyenaiva yogena} \\
 & \romline{māṃ dhyāyanta upāsate ||}
\end{tabular}
\end{table}

\begin{table}[H]
\begin{tabular}{cl}
\textbf{12.7} & \romline{teṣāmahaṃ samuddhartā} \\
 & \romline{mṛtyu-saṃsāra-sāgarāt |} \\
 & \romline{bhavāmi nacirāt-pārtha} \\
 & \romline{mayyāveśita-cetasām ||}
\end{tabular}
\end{table}

\begin{table}[H]
\begin{tabular}{cl}
\textbf{12.8} & \romline{mayyeva mana ādhatsva} \\
 & \romline{mayi buddhiṃ niveśaya |} \\
 & \romline{nivasiṣyasi mayyeva} \\
 & \romline{ata ūrdhvaṃ na samśayaḥ ||}
\end{tabular}
\end{table}

\begin{table}[H]
\begin{tabular}{cl}
\textbf{12.9} & \romline{atha cittaṃ samādhātuṃ} \\
 & \romline{na śaknoṣi mayi sthiram |} \\
 & \romline{abhyāsayogena tataḥ} \\
 & \romline{māmicchāptuṃ dhanañjaya ||}
\end{tabular}
\end{table}

\begin{table}[H]
\begin{tabular}{cl}
\textbf{12.10} & \romline{abhyāse'pyasamartho'si} \\
 & \romline{mat-karma-paramo bhava |} \\
 & \romline{madarthamapi karmāṇi} \\
 & \romline{kurvan-siddhi-mavāpsyasi ||}
\end{tabular}
\end{table}

\begin{table}[H]
\begin{tabular}{cl}
\textbf{12.11} & \romline{athaitadapyaśakto'si} \\
 & \romline{kartuṃ mad-yogam-āśritaḥ |} \\
 & \romline{sarva-karma-phala-tyāgaṃ} \\
 & \romline{tataḥ kuru yatātmavān ||}
\end{tabular}
\end{table}

\begin{table}[H]
\begin{tabular}{cl}
\textbf{12.12} & \romline{śreyo hi·jñānam-abhyāsāt} \\
 & \romline{jñānāddhyānaṃ viśiṣyate |} \\
 & \romline{dhyānāt-karma-phala-tyāgaḥ} \\
 & \romline{tyāgācchāntiranantaram ||}
\end{tabular}
\end{table}

\begin{table}[H]
\begin{tabular}{cl}
\textbf{12.13} & \romline{adveṣṭā sarvabhūtānāṃ} \\
 & \romline{maitraḥ karuṇa eva ca |} \\
 & \romline{nirmamo nirahankāraḥ} \\
 & \romline{sama-duḥkha-sukhaḥ·kṣamī ||}
\end{tabular}
\end{table}

\begin{table}[H]
\begin{tabular}{cl}
\textbf{12.14} & \romline{santuṣṭaḥ satataṃ yogī} \\
 & \romline{yatātmā dṛḍha-niścayaḥ |} \\
 & \romline{mayyarpitamanobuddhiḥ} \\
 & \romline{yo madbhaktaḥ sa me priyaḥ ||}
\end{tabular}
\end{table}

\begin{table}[H]
\begin{tabular}{cl}
\textbf{12.15} & \romline{yasmānnodvijate lokaḥ} \\
 & \romline{lokānnodvijate ca yaḥ |} \\
 & \romline{harṣāmarṣa-bhayodvegaiḥ} \\
 & \romline{mukto yaḥ sa ca me priyaḥ ||}
\end{tabular}
\end{table}

\begin{table}[H]
\begin{tabular}{cl}
\textbf{12.16} & \romline{anapekṣaḥ śucirdakṣaḥ} \\
 & \romline{udāsīno gatavyathaḥ |} \\
 & \romline{sarvārambha-parityāgī} \\
 & \romline{yo madbhaktaḥ sa me priyaḥ ||}
\end{tabular}
\end{table}

\begin{table}[H]
\begin{tabular}{cl}
\textbf{12.17} & \romline{yo na hṛṣyati na·dveṣṭi} \\
 & \romline{na śocati na kāñkṣati |} \\
 & \romline{śubhāśubha-parityāgī} \\
 & \romline{bhaktimānyaḥ sa me priyaḥ ||}
\end{tabular}
\end{table}

\begin{table}[H]
\begin{tabular}{cl}
\textbf{12.18} & \romline{samaḥ śatrau ca mitre ca} \\
 & \romline{tathā mānāpamānayoḥ |} \\
 & \romline{śītoṣṇa-sukha-duḥkheṣu} \\
 & \romline{samaḥ sañga-vivarjitaḥ ||}
\end{tabular}
\end{table}

\begin{table}[H]
\begin{tabular}{cl}
\textbf{12.19} & \romline{tulya-nindā-stutirmaunī} \\
 & \romline{santuṣṭo yena kenacit |} \\
 & \romline{aniketaḥ sthiramatiḥ} \\
 & \romline{bhaktimānme priyo naraḥ ||}
\end{tabular}
\end{table}

\begin{table}[H]
\begin{tabular}{cl}
\textbf{12.20} & \romline{ye tu dharmyāmṛtamidaṃ} \\
 & \romline{yathoktaṃ paryupāsate |} \\
 & \romline{śraddadhānā matparamāḥ} \\
 & \romline{bhaktāste'tīva me priyāḥ ||}
\end{tabular}
\end{table}


% \end{multicols}

\chapter{Kṣetra-Kṣetrajña-Vibhāga Yoga}
% \begin{multicols}{2}
\begin{table}[H]
\begin{tabular}{cl}
\textbf{13.0} & \natline{ओं श्री परमात्मने नमः} \\
 & \natline{अथ त्रयोदशोऽध्यायः} \\
 & \natline{क्षेत्रक्षेत्रज्ञविभागयोगः}
\end{tabular}
\end{table}

\begin{table}[H]
\begin{tabular}{cl}
\textbf{13.1} & \natline{अर्जुन उवाच} \\
 & \natline{प्रकृतिं पुरुषं चैव} \\
 & \natline{क्षेत्रं क्षेत्रज्ञमेव च |} \\
 & \natline{एतत् वेदितुमिच्छामि} \\
 & \natline{ज्ञानं ज्ञेयं च केशव ||}
\end{tabular}
\end{table}

\begin{table}[H]
\begin{tabular}{cl}
\textbf{13.2} & \natline{श्री भगवानुवाच} \\
 & \natline{इदं शरीरं कौन्तेय} \\
 & \natline{क्षेत्रमित्यभिधीयते |} \\
 & \natline{एतद्यो वेत्ति तं प्राहुः} \\
 & \natline{क्षेत्रज्ञ इति तद्विदः ||}
\end{tabular}
\end{table}

\begin{table}[H]
\begin{tabular}{cl}
\textbf{13.3} & \natline{क्षेत्रज्ञं चापि मां विद्धि} \\
 & \natline{सर्वक्षेत्रेषु भारत |} \\
 & \natline{क्षेत्रक्षेत्रज्ञयोर्ज्ञानं} \\
 & \natline{यत्तज्ज्ञानं मतं मम ||}
\end{tabular}
\end{table}

\begin{table}[H]
\begin{tabular}{cl}
\textbf{13.4} & \natline{तत्क्षेत्रं यच्च यादृक्च} \\
 & \natline{यद्विकारि यतश्च यत् |} \\
 & \natline{स च यो यत्प्रभावश्च} \\
 & \natline{तत्समासेन मे शृणु ||}
\end{tabular}
\end{table}

\begin{table}[H]
\begin{tabular}{cl}
\textbf{13.5} & \natline{ऋषिभिर्बहुधा गीतं} \\
 & \natline{छन्दोभिर्विविधैः पृथक् |} \\
 & \natline{ब्रह्मसूत्रपदैश्चैव} \\
 & \natline{हेतुमद्भिर्विनिश्चितैः ||}
\end{tabular}
\end{table}

\begin{table}[H]
\begin{tabular}{cl}
\textbf{13.6} & \natline{महाभूतान्यहङ्कारः} \\
 & \natline{बुद्धिरव्यक्तमेव च |} \\
 & \natline{इन्द्रियाणि दशैकं च} \\
 & \natline{पञ्च चेन्द्रियगोचराः ||}
\end{tabular}
\end{table}

\begin{table}[H]
\begin{tabular}{cl}
\textbf{13.7} & \natline{इच्च्हा द्वेषः सुखं दुःखं} \\
 & \natline{सङ्घातश्चेतना धृतिः |} \\
 & \natline{एतत्क्षेत्रं समासेन} \\
 & \natline{सविकारमुदाहृतम् ||}
\end{tabular}
\end{table}

\begin{table}[H]
\begin{tabular}{cl}
\textbf{13.8} & \natline{अमानित्वमदंभित्वम्} \\
 & \natline{अहिंसा क्षान्तिरार्जवम् |} \\
 & \natline{आचार्योपासनं शौचं} \\
 & \natline{स्थैर्यमात्मविनिग्रहः ||}
\end{tabular}
\end{table}

\begin{table}[H]
\begin{tabular}{cl}
\textbf{13.9} & \natline{इन्द्रियार्थेषु वैराग्यम्} \\
 & \natline{अनहङ्कार एव च |} \\
 & \natline{जन्ममृत्युजराव्याधि} \\
 & \natline{दुःखदोषानुदर्शनम् ||}
\end{tabular}
\end{table}

\begin{table}[H]
\begin{tabular}{cl}
\textbf{13.10} & \natline{असक्तिरनभिष्वङ्गः} \\
 & \natline{पुत्रदारगृहादिषु |} \\
 & \natline{नित्यं च समचित्तत्वम्} \\
 & \natline{इष्टानिष्टोपपत्तिषु ||}
\end{tabular}
\end{table}

\begin{table}[H]
\begin{tabular}{cl}
\textbf{13.11} & \natline{मयि चानन्ययोगेन} \\
 & \natline{भक्तिरव्यभिचारिणी |} \\
 & \natline{विविक्तदेशसेवित्वम्} \\
 & \natline{अरतिर्जनसंसदि ||}
\end{tabular}
\end{table}

\begin{table}[H]
\begin{tabular}{cl}
\textbf{13.12} & \natline{अध्यात्मज्ञाननित्यत्वं} \\
 & \natline{तत्त्वज्ञानार्थदर्शनम् |} \\
 & \natline{एतज्ज्ञानमिति प्रोक्तम्} \\
 & \natline{अज्ञानं यदतोऽन्यथा ||}
\end{tabular}
\end{table}

\begin{table}[H]
\begin{tabular}{cl}
\textbf{13.13} & \natline{ज्ञेयं यत्तत्प्रवक्ष्यामि} \\
 & \natline{यज्ज्ञात्वाऽमृतमश्नुते |} \\
 & \natline{अनादिमत्परं ब्रह्म} \\
 & \natline{न सत्तन्नासदुच्यते ||}
\end{tabular}
\end{table}

\begin{table}[H]
\begin{tabular}{cl}
\textbf{13.14} & \natline{सर्वतः पाणिपादं तत्} \\
 & \natline{सर्वतोऽक्षिशिरोमुखम् |} \\
 & \natline{सर्वतः श्रुतिमल्लोके} \\
 & \natline{सर्वमावृत्य तिष्ठति ||}
\end{tabular}
\end{table}

\begin{table}[H]
\begin{tabular}{cl}
\textbf{13.15} & \natline{सर्वेन्द्रियगुणाभासं} \\
 & \natline{सर्वेन्द्रियविवर्जितम् |} \\
 & \natline{असक्तं सर्वभृच्चैव} \\
 & \natline{निर्गुणं गुणभोक्तृ च ||}
\end{tabular}
\end{table}

\begin{table}[H]
\begin{tabular}{cl}
\textbf{13.16} & \natline{बहिरन्तश्च भूतानाम्} \\
 & \natline{अcअरं चरमेव च |} \\
 & \natline{सूक्ष्मत्वात्तदविज्ञेयं} \\
 & \natline{दूरस्थं चान्तिके च तत् ||}
\end{tabular}
\end{table}

\begin{table}[H]
\begin{tabular}{cl}
\textbf{13.17} & \natline{अविभक्तं च भूतेषु} \\
 & \natline{विभक्तमिव च स्थितम् |} \\
 & \natline{भूतभर्तृ च तज्ज्ञेयं} \\
 & \natline{ग्रसिष्णु प्रभविष्णु च ||}
\end{tabular}
\end{table}

\begin{table}[H]
\begin{tabular}{cl}
\textbf{13.18} & \natline{ज्योतिषामपि तज्ज्योतिः} \\
 & \natline{तमसः परमुच्यते |} \\
 & \natline{ज्ञानं ज्ञेयं ज्ञानगम्यं} \\
 & \natline{हृदि सर्वस्य विष्ठितम् ||}
\end{tabular}
\end{table}

\begin{table}[H]
\begin{tabular}{cl}
\textbf{13.19} & \natline{इति क्षेत्रं तथा ज्ञानं} \\
 & \natline{ज्ञेयं चोक्तं समासतः |} \\
 & \natline{मद्भक्त एतद्विज्ञाय} \\
 & \natline{मद्भावायोपपद्यते ||}
\end{tabular}
\end{table}

\begin{table}[H]
\begin{tabular}{cl}
\textbf{13.20} & \natline{प्रकृतिं पुरुषं चैव} \\
 & \natline{विद्ध्यनादी उभावपि |} \\
 & \natline{विकारांश्च गुणांश्चैव} \\
 & \natline{विद्धि प्रकृतिसम्भवान् ||}
\end{tabular}
\end{table}

\begin{table}[H]
\begin{tabular}{cl}
\textbf{13.21} & \natline{कार्यकरणकर्तृत्वे} \\
 & \natline{हेतुः प्रकृतिरुच्यते |} \\
 & \natline{पुरुषः सुखदुःखानां} \\
 & \natline{भोक्तृत्वे हेतुरुच्यते ||}
\end{tabular}
\end{table}

\begin{table}[H]
\begin{tabular}{cl}
\textbf{13.22} & \natline{पुरुषः प्रकृतिस्थो हि} \\
 & \natline{भुङ्क्ते प्रकृतिजान्गुणान् |} \\
 & \natline{कारणं गुणसङ्गोऽस्य} \\
 & \natline{सदसद्योनिजन्मसु ||}
\end{tabular}
\end{table}

\begin{table}[H]
\begin{tabular}{cl}
\textbf{13.23} & \natline{उपद्रष्टाऽनुमन्ता च} \\
 & \natline{भर्ता भोक्ता महेश्वरः |} \\
 & \natline{परमात्मेति चाप्युक्तः} \\
 & \natline{देहेऽस्मिन्पुरुषः परः ||}
\end{tabular}
\end{table}

\begin{table}[H]
\begin{tabular}{cl}
\textbf{13.24} & \natline{य एवं वेत्ति पुरुषं} \\
 & \natline{प्रकृतिं च गुणैः सह |} \\
 & \natline{सर्वथा वर्तमानोऽपि} \\
 & \natline{न स भूयोऽभिजायते ||}
\end{tabular}
\end{table}

\begin{table}[H]
\begin{tabular}{cl}
\textbf{13.25} & \natline{ध्यानेनात्मनि पश्यन्ति} \\
 & \natline{केचिदात्मानमात्मना |} \\
 & \natline{अन्ये साङ्ख्येन योगेन} \\
 & \natline{कर्मयोगेन चापरे ||}
\end{tabular}
\end{table}

\begin{table}[H]
\begin{tabular}{cl}
\textbf{13.26} & \natline{अन्ये त्वेवमजानन्तः} \\
 & \natline{श्रुत्वाऽन्येभ्य उपासते |} \\
 & \natline{तेऽपि चातितरन्त्येव} \\
 & \natline{मृत्युं श्रुतिपरायणाः ||}
\end{tabular}
\end{table}

\begin{table}[H]
\begin{tabular}{cl}
\textbf{13.27} & \natline{यावत्सञ्जायते किञ्चित्} \\
 & \natline{सत्त्वं स्थावरजङ्गमम् |} \\
 & \natline{क्षेत्रक्षेत्रज्ञसंयोगात्} \\
 & \natline{तद्विद्धि भरतर्षभ ||}
\end{tabular}
\end{table}

\begin{table}[H]
\begin{tabular}{cl}
\textbf{13.28} & \natline{समं सर्वेषु भूतेषु} \\
 & \natline{तिष्ठन्तं परमेश्वरम् |} \\
 & \natline{विनश्यत्स्वविनश्यन्तं} \\
 & \natline{यः पश्यति स पश्यति ||}
\end{tabular}
\end{table}

\begin{table}[H]
\begin{tabular}{cl}
\textbf{13.29} & \natline{समं पश्यन्हि सर्वत्र} \\
 & \natline{समवस्थितमीश्वरम् |} \\
 & \natline{न हिनस्त्यात्मनाऽऽत्मानं} \\
 & \natline{ततो याति परां गतिम् ||}
\end{tabular}
\end{table}

\begin{table}[H]
\begin{tabular}{cl}
\textbf{13.30} & \natline{प्रकृत्यैव च कर्माणि} \\
 & \natline{क्रियमाणानि सर्वशः |} \\
 & \natline{यः पश्यति तथाऽऽत्मानम्} \\
 & \natline{अकर्तारं स पश्यति ||}
\end{tabular}
\end{table}

\begin{table}[H]
\begin{tabular}{cl}
\textbf{13.31} & \natline{यदा भूतपृथग्भावम्} \\
 & \natline{एकस्थमनुपश्यति |} \\
 & \natline{तत एव च विस्तारं} \\
 & \natline{ब्रह्म सम्पद्यते तदा ||}
\end{tabular}
\end{table}

\begin{table}[H]
\begin{tabular}{cl}
\textbf{13.32} & \natline{अनादित्वान्निर्गुणत्वात्} \\
 & \natline{परमात्मायमव्ययः |} \\
 & \natline{शरीरस्थोऽपि कौन्तेय} \\
 & \natline{न करोति न लिप्यते ||}
\end{tabular}
\end{table}

\begin{table}[H]
\begin{tabular}{cl}
\textbf{13.33} & \natline{यथा सर्वगतं सौक्ष्म्यात्} \\
 & \natline{आकाशं नोपलिप्यते |} \\
 & \natline{सर्वत्रावस्थितो देहे} \\
 & \natline{तथाऽऽत्मा नोपलिप्यते ||}
\end{tabular}
\end{table}

\begin{table}[H]
\begin{tabular}{cl}
\textbf{13.34} & \natline{यथा प्रकाशयत्येकः} \\
 & \natline{कृत्स्नं लोकमिमं रविः |} \\
 & \natline{क्षेत्रं क्षेत्री तथा कृत्स्नं} \\
 & \natline{प्रकाशयति भारत ||}
\end{tabular}
\end{table}

\begin{table}[H]
\begin{tabular}{cl}
\textbf{13.35} & \natline{क्षेत्रक्षेत्रज्ञयोरेवम्} \\
 & \natline{अन्तरं ज्ञानचक्षुषा |} \\
 & \natline{भूतप्रकृतिमोक्षं च} \\
 & \natline{ये विदुर्यान्ति ते परम् ||}
\end{tabular}
\end{table}


% \end{multicols}

\chapter{Guṇatraya-Vibhāga Yoga}
% \begin{multicols}{2}
\begin{table}[H]
\begin{tabular}{cl}
\textbf{14.0} & \romline{oṃ śrī paramātmane namaḥ} \\
 & \romline{atha caturdaśo'dhyāyaḥ} \\
 & \romline{guṇa-trayavibhāga-yogaḥ}
\end{tabular}
\end{table}

\begin{table}[H]
\begin{tabular}{cl}
\textbf{14.1} & \romline{śrī bhagavānuvāca} \\
 & \romline{paraṃ bhūyaḥ pravakṣyāmi} \\
 & \romline{jñānānāṃ jñāna-muttamam |} \\
 & \romline{yaj-jñātvā munayaḥ sarve} \\
 & \romline{parāṃ siddhi-mito gatāḥ ||}
\end{tabular}
\end{table}

\begin{table}[H]
\begin{tabular}{cl}
\textbf{14.2} & \romline{idaṃ jñāna-mupāśritya} \\
 & \romline{mama sādharmya-māgatāḥ |} \\
 & \romline{sarge'pi nopajāyante} \\
 & \romline{pralaye na·vyathanti ca ||}
\end{tabular}
\end{table}

\begin{table}[H]
\begin{tabular}{cl}
\textbf{14.3} & \romline{mama yonir-mahad-brahma} \\
 & \romline{tasmin-garbhaṃ dadhāmyaham |} \\
 & \romline{sambhavaḥ sarva-bhūtānāṃ} \\
 & \romline{tato bhavati bhārata ||}
\end{tabular}
\end{table}

\begin{table}[H]
\begin{tabular}{cl}
\textbf{14.4} & \romline{sarva-yoniṣu kaunteya} \\
 & \romline{mūrtayaḥ sambha-vanti yāḥ |} \\
 & \romline{tāsāṃ brahma mahad-yoniḥ} \\
 & \romline{ahaṃ bīja-pradaḥ pitā ||}
\end{tabular}
\end{table}

\begin{table}[H]
\begin{tabular}{cl}
\textbf{14.5} & \romline{sattvaṃ rajas-tama iti} \\
 & \romline{guṇāḥ prakṛti-sambhavāḥ |} \\
 & \romline{nibadhnanti mahābāho} \\
 & \romline{dehe dehina-mavyayam ||}
\end{tabular}
\end{table}

\begin{table}[H]
\begin{tabular}{cl}
\textbf{14.6} & \romline{tatra sattvaṃ nirmala-tvāt} \\
 & \romline{prakāśa-kama-nāmayam |} \\
 & \romline{sukha-saṅgena badhnāti} \\
 & \romline{jñāna-saṅgena cānagha ||}
\end{tabular}
\end{table}

\begin{table}[H]
\begin{tabular}{cl}
\textbf{14.7} & \romline{rajo rāgātmakaṃ viddhi} \\
 & \romline{tṛṣṇā-saṅga-samudbhavam |} \\
 & \romline{tanni-badhnāti kaunteya} \\
 & \romline{karma-saṅgena dehinam ||}
\end{tabular}
\end{table}

\begin{table}[H]
\begin{tabular}{cl}
\textbf{14.8} & \romline{tamastva-jñāna-jaṃ viddhi} \\
 & \romline{mohanaṃ sarva-dehinām |} \\
 & \romline{pramādālasya-nidrābhiḥ} \\
 & \romline{tanni-badhnāti bhārata ||}
\end{tabular}
\end{table}

\begin{table}[H]
\begin{tabular}{cl}
\textbf{14.9} & \romline{sattvaṃ sukhe sañjayati} \\
 & \romline{rajaḥ karmaṇi bhārata |} \\
 & \romline{jñāna-māvṛtya tu tamaḥ} \\
 & \romline{pramāde sañjaya-tyuta ||}
\end{tabular}
\end{table}

\begin{table}[H]
\begin{tabular}{cl}
\textbf{14.10} & \romline{rajas-tamaścā-bhibhūya} \\
 & \romline{sattvaṃ bhavati bhārata |} \\
 & \romline{rajaḥ sattvaṃ tamaścaiva} \\
 & \romline{tamaḥ sattvaṃ rajas-tathā ||}
\end{tabular}
\end{table}

\begin{table}[H]
\begin{tabular}{cl}
\textbf{14.11} & \romline{sarva-dvāreṣu dehe'smin} \\
 & \romline{prakāśa upajāyate |} \\
 & \romline{jñānaṃ yadā tadā vidyāt} \\
 & \romline{vivṛddhaṃ sattvami-tyuta ||}
\end{tabular}
\end{table}

\begin{table}[H]
\begin{tabular}{cl}
\textbf{14.12} & \romline{lobhaḥ pravṛtti-rārambhaḥ} \\
 & \romline{karmaṇā-maśamaḥ spṛhā |} \\
 & \romline{rajas-yetāni jāyante} \\
 & \romline{vivṛddhe bhara-tarṣabha ||}
\end{tabular}
\end{table}

\begin{table}[H]
\begin{tabular}{cl}
\textbf{14.13} & \romline{aprakāśo'pravṛttiśca} \\
 & \romline{pramādo moha eva ca |} \\
 & \romline{tamasyetāni jāyante} \\
 & \romline{vivṛddhe kuru-nandana ||}
\end{tabular}
\end{table}

\begin{table}[H]
\begin{tabular}{cl}
\textbf{14.14} & \romline{yadā sattve pravṛddhe tu} \\
 & \romline{pralayaṃ yāti dehabhṛt |} \\
 & \romline{tadotta-mavidāṃ lokān} \\
 & \romline{amalān-pratipadyate ||}
\end{tabular}
\end{table}

\begin{table}[H]
\begin{tabular}{cl}
\textbf{14.15} & \romline{rajasi·pralayaṃ gatvā} \\
 & \romline{karma-saṅgiṣu jāyate |} \\
 & \romline{tathā pralīna-stamasi} \\
 & \romline{mūḍha-yoniṣu jāyate ||}
\end{tabular}
\end{table}

\begin{table}[H]
\begin{tabular}{cl}
\textbf{14.16} & \romline{karmaṇaḥ sukṛta-syāhuḥ} \\
 & \romline{sāttvikaṃ nirmalaṃ phalam |} \\
 & \romline{raja-sastu phalaṃ duḥkham} \\
 & \romline{ajñānaṃ tamasaḥ phalam ||}
\end{tabular}
\end{table}

\begin{table}[H]
\begin{tabular}{cl}
\textbf{14.17} & \romline{sattvāt-sañjāyate jñānaṃ} \\
 & \romline{rajaso lobha eva ca |} \\
 & \romline{pramāda-mohau tamasaḥ} \\
 & \romline{bhavato'jñāna-meva ca ||}
\end{tabular}
\end{table}

\begin{table}[H]
\begin{tabular}{cl}
\textbf{14.18} & \romline{ūrdhvaṃ gacchanti sattva-sthāḥ} \\
 & \romline{madhye tiṣṭhanti rājasāḥ |} \\
 & \romline{jaghanya-guṇa-vṛtti-sthāḥ} \\
 & \romline{adho gacchanti tāmasāḥ ||}
\end{tabular}
\end{table}

\begin{table}[H]
\begin{tabular}{cl}
\textbf{14.19} & \romline{nānyaṃ guṇebhyaḥ kartāraṃ} \\
 & \romline{yadā draṣṭā'nupaśyati |} \\
 & \romline{guṇebhyaśca paraṃ vetti} \\
 & \romline{madbhāvaṃ so'dhi-gacchati ||}
\end{tabular}
\end{table}

\begin{table}[H]
\begin{tabular}{cl}
\textbf{14.20} & \romline{guṇā-netā-natītya·trīn} \\
 & \romline{dehī deha-samud-bhavān |} \\
 & \romline{janma-mṛtyu-jarā-duḥkhaiḥ} \\
 & \romline{vimukto'mṛta-maśnute ||}
\end{tabular}
\end{table}

\begin{table}[H]
\begin{tabular}{cl}
\textbf{14.21} & \romline{arjuna uvāca} \\
 & \romline{kairliṅgai-strīn-guṇā-netān} \\
 & \romline{atīto bhavati·prabho |} \\
 & \romline{kimācāraḥ kathaṃ caitān} \\
 & \romline{trīn-guṇā-nati-vartate ||}
\end{tabular}
\end{table}

\begin{table}[H]
\begin{tabular}{cl}
\textbf{14.22} & \romline{śrī bhagavānuvāca} \\
 & \romline{prakāśaṃ ca·pravṛttiṃ ca} \\
 & \romline{mohameva ca pāṇḍava |} \\
 & \romline{na·dveṣṭi sam-pravṛttāni} \\
 & \romline{na nivṛttāni kāṅkṣati ||}
\end{tabular}
\end{table}

\begin{table}[H]
\begin{tabular}{cl}
\textbf{14.23} & \romline{udāsīnava-dāsīnaḥ} \\
 & \romline{guṇairyo na vicālyate |} \\
 & \romline{guṇā vartanta ityeva} \\
 & \romline{yo'va-tiṣṭhati neṅgate ||}
\end{tabular}
\end{table}

\begin{table}[H]
\begin{tabular}{cl}
\textbf{14.24} & \romline{sama-duḥkha-sukhaḥ svasthaḥ} \\
 & \romline{samaloṣṭāś-makāñcanaḥ |} \\
 & \romline{tulya-priyāpriyo dhīraḥ} \\
 & \romline{tulya-nindātma-saṃstutiḥ ||}
\end{tabular}
\end{table}

\begin{table}[H]
\begin{tabular}{cl}
\textbf{14.25} & \romline{mānāpa-mānayo-stulyaḥ} \\
 & \romline{tulyo mitrā-ripakṣayoḥ |} \\
 & \romline{sarvā-rambha-pari-tyāgī} \\
 & \romline{guṇā-tītaḥ sa ucyate ||}
\end{tabular}
\end{table}

\begin{table}[H]
\begin{tabular}{cl}
\textbf{14.26} & \romline{māṃ ca yo'vya-bhicāreṇa} \\
 & \romline{bhakti-yogena sevate |} \\
 & \romline{sa guṇān-samatī-tyaitān} \\
 & \romline{brahma-bhūyāya kalpate ||}
\end{tabular}
\end{table}

\begin{table}[H]
\begin{tabular}{cl}
\textbf{14.27} & \romline{brahmaṇo hi·pratiṣṭhā'ham} \\
 & \romline{amṛtas-yāvya-yasya ca |} \\
 & \romline{śāśva-tasya ca dharmasya} \\
 & \romline{sukhasyai-kāntikasya ca ||}
\end{tabular}
\end{table}


% \end{multicols}

\chapter{Puruṣottama-Prāpti Yoga}
% \begin{multicols}{2}
\begin{table}[H]
\begin{tabular}{cl}
 & \natline{श्री परमात्मने नमः} \\
 & \natline{अथ पञ्चदशोऽध्यायः} \\
 & \natline{पुरुषोत्तमप्रप्तियोगः}
\end{tabular}
\end{table}

\begin{table}[H]
\begin{tabular}{cl}
\textbf{15.1} & \natline{श्रीभगवान् उवाच} \\
 & \natline{ऊर्ध्वमूलमधः* शाखम्} \\
 & \natline{अश्वत्थं प्राहुरव्ययम् |} \\
 & \natline{छन्दांसि यस्य पर्णानि} \\
 & \natline{यस्तं वेद स वेदवित् ||}
\end{tabular}
\end{table}

\begin{table}[H]
\begin{tabular}{cl}
\textbf{15.2} & \natline{अधश्चोर्ध्वं प्रसृतास्तस्य शाखाः} \\
 & \natline{गुणप्रवृद्धा विषयप्रवालाः |} \\
 & \natline{अधश्च मूलान्यनुसन्ततानि} \\
 & \natline{कर्मानुबन्धीनि मनुष्यलोके ||}
\end{tabular}
\end{table}

\begin{table}[H]
\begin{tabular}{cl}
\textbf{15.3} & \natline{न रूपमस्येह तथोपलभ्यते} \\
 & \natline{नान्तो न चादिर्न च संप्रतिष्ठा |} \\
 & \natline{अश्वत्थमेनं सुविरूढमूलम्} \\
 & \natline{असङ्गशस्त्रेण दृढेन छित्त्वा ||}
\end{tabular}
\end{table}

\begin{table}[H]
\begin{tabular}{cl}
\textbf{15.4} & \natline{ततः पदं तत्परिमार्गितव्यं} \\
 & \natline{यस्मिन्गता न निवर्तन्ति भूयः |} \\
 & \natline{तमेव चाद्यं पुरुषं प्रपद्ये} \\
 & \natline{यतः प्रवृत्तिः प्रसृता पुराणी ||}
\end{tabular}
\end{table}

\begin{table}[H]
\begin{tabular}{cl}
\textbf{15.5} & \natline{निर्मानमोहा जितसङ्गदोषाः} \\
 & \natline{अध्यात्मनित्या विनिवृत्तकामाः |} \\
 & \natline{द्वन्द्वैर्विमुक्ताः सुखदुःख सञ्ज्ञैः} \\
 & \natline{गच्छन्त्यमूढाः पदमव्ययं तत् ||}
\end{tabular}
\end{table}

\begin{table}[H]
\begin{tabular}{cl}
\textbf{15.6} & \natline{न तद्भासयते सूर्यः} \\
 & \natline{न शशाङ्को न पावकः |} \\
 & \natline{यद्गत्वा न निवर्तन्ते} \\
 & \natline{तद्धाम परमं मम ||}
\end{tabular}
\end{table}

\begin{table}[H]
\begin{tabular}{cl}
\textbf{15.7} & \natline{ममैवांशो जीवलोके} \\
 & \natline{जीवभूतः सनातनः |} \\
 & \natline{मनः षष्ठानीन्द्रियाणि} \\
 & \natline{प्रकृतिस्थानि कर्षति ||}
\end{tabular}
\end{table}

\begin{table}[H]
\begin{tabular}{cl}
\textbf{15.8} & \natline{शरीरं यदवाप्नोति} \\
 & \natline{यच्चाप्युत्क्रामतीश्वरः |} \\
 & \natline{गृहीत्वैतानि संयाति} \\
 & \natline{वायुर्गन्धानिवाशयात् ||}
\end{tabular}
\end{table}

\begin{table}[H]
\begin{tabular}{cl}
\textbf{15.9} & \natline{श्रोत्रं चक्षुः स्पर्शनं च} \\
 & \natline{रसनं घ्राणमेव च |} \\
 & \natline{अधिष्ठाय मनश्चायं} \\
 & \natline{विषयानुपसेवते ||}
\end{tabular}
\end{table}

\begin{table}[H]
\begin{tabular}{cl}
\textbf{15.10} & \natline{उत्क्रामन्तं स्थितं वाऽपि} \\
 & \natline{भुञ्जानं वा गुणान्वितम् |} \\
 & \natline{विमूढा नानुपश्यन्ति} \\
 & \natline{पश्यन्ति ज्ञानचक्षुषः ||}
\end{tabular}
\end{table}

\begin{table}[H]
\begin{tabular}{cl}
\textbf{15.11} & \natline{यतन्तो योगिनश्चैनं} \\
 & \natline{पश्यन्त्यात्मन्यवस्थितम् |} \\
 & \natline{यतन्तोऽप्यकृतात्मानः} \\
 & \natline{नैनं पश्यन्त्यचेतसः ||}
\end{tabular}
\end{table}

\begin{table}[H]
\begin{tabular}{cl}
\textbf{15.12} & \natline{यदादित्यगतं तेजः} \\
 & \natline{जगद्भासयतेऽखिलम् |} \\
 & \natline{यच्चन्द्रमसि यच्चाग्नौ} \\
 & \natline{तत्तेजो विद्धि मामकम् ||}
\end{tabular}
\end{table}

\begin{table}[H]
\begin{tabular}{cl}
\textbf{15.13} & \natline{गामाविश्य च भूतानि} \\
 & \natline{धारयाम्यहमोजसा |} \\
 & \natline{पुष्णामि चौषधीः सर्वाः} \\
 & \natline{सोमो भूत्वा रसात्मकः ||}
\end{tabular}
\end{table}

\begin{table}[H]
\begin{tabular}{cl}
\textbf{15.14} & \natline{अहं वैश्वानरो भूत्वा} \\
 & \natline{प्राणिनां देहमाश्रितः |} \\
 & \natline{प्राणापानसमायुक्तः} \\
 & \natline{पचाम्यन्नं चतुर्विधम् ||}
\end{tabular}
\end{table}

\begin{table}[H]
\begin{tabular}{cl}
\textbf{15.15} & \natline{सर्वस्य चाहं हृदि सन्निविष्टः} \\
 & \natline{मत्तः स्मृतिर्ज्ञानमपोहनं च |} \\
 & \natline{वेदैश्च सर्वैरहमेव वेद्यः} \\
 & \natline{वेदान्तकृद्वेदविदेव चाहम् ||}
\end{tabular}
\end{table}

\begin{table}[H]
\begin{tabular}{cl}
\textbf{15.16} & \natline{द्वाविमौ पुरुषौ लोके} \\
 & \natline{क्षरश्चाक्षर एव च |} \\
 & \natline{क्षरः सर्वाणि भूतानि} \\
 & \natline{कूटस्थोऽक्षर उच्यते ||}
\end{tabular}
\end{table}

\begin{table}[H]
\begin{tabular}{cl}
\textbf{15.17} & \natline{उत्तमः पुरुषस्त्वन्यः} \\
 & \natline{परमात्मेत्युदाहृतः |} \\
 & \natline{यो लोकत्रयमाविश्य} \\
 & \natline{बिभर्त्यव्यय ईश्वरः ||}
\end{tabular}
\end{table}

\begin{table}[H]
\begin{tabular}{cl}
\textbf{15.18} & \natline{यस्मात्क्षरमतीतोऽहम्} \\
 & \natline{अक्षरादपि चोत्तमः |} \\
 & \natline{अतोऽस्मि लोके वेदे च} \\
 & \natline{प्रथितः पुरुषोत्तमः ||}
\end{tabular}
\end{table}

\begin{table}[H]
\begin{tabular}{cl}
\textbf{15.19} & \natline{यो मामेवमसम्मूढः} \\
 & \natline{जानाति पुरुषोत्तमम् |} \\
 & \natline{स सर्वविद्भजति मां} \\
 & \natline{सर्वभावेन भारत ||}
\end{tabular}
\end{table}

\begin{table}[H]
\begin{tabular}{cl}
\textbf{15.20} & \natline{इति गुह्यतमं शास्त्रम्} \\
 & \natline{इदमुक्तं मयाऽनघ |} \\
 & \natline{एतद्बुद्ध्वा बुद्धिमान्स्यात्} \\
 & \natline{कृतकृत्यश्च भारत ||}
\end{tabular}
\end{table}

\begin{table}[H]
\begin{tabular}{cl}
 & \natline{श्रीमद्भगवद्गीतासु उपनिषत्सु} \\
 & \natline{ब्रह्मविद्यायां योगशास्त्रे} \\
 & \natline{श्रीकृष्णार्जुन संवादे} \\
 & \natline{पुरुषोत्तमप्रप्तियोगो नाम} \\
 & \natline{पञ्चदशोध्यायः}
\end{tabular}
\end{table}


% \end{multicols}

\chapter{Daivāsura-Sampad-Vibhāga Yoga}
% \begin{multicols}{2}
\subsection*{16.0}
\begin{table}[H]
\begin{tabular}{l}
\natline{ఓం శ్రీ పరమాత్మనే నమః} \\
\natline{అథ షోదశోఽధ్యాయః} \\
\natline{దైవాసురసమ్పద్విభాగయోగః}
\end{tabular}
\end{table}

\subsection*{16.1}
\begin{table}[H]
\begin{tabular}{l}
\natline{శ్రీ భగవానువాచ} \\
\natline{అభయం సత్త్వసంశుద్ధిః} \\
\natline{జ్ఞానయోగవ్యవస్థితిః} \\
\natline{దానం దమశ్చ యజ్ఞశ్చ} \\
\natline{స్వాధ్యాయస్తప ఆర్జవమ్}
\end{tabular}
\end{table}

\subsection*{16.2}
\begin{table}[H]
\begin{tabular}{l}
\natline{అహింసా సత్యమక్రోధః} \\
\natline{త్యాగః శాన్తిరపైశునమ్} \\
\natline{దయా భూతేష్వలోలుప్త్వం} \\
\natline{మార్దవం హ్రీరచాపలమ్}
\end{tabular}
\end{table}

\subsection*{16.3}
\begin{table}[H]
\begin{tabular}{l}
\natline{తేజః క్షమా ధృతిః శౌచమ్} \\
\natline{అద్రోహో నాతిమానితా} \\
\natline{భవన్తి సమ్పదం దైవీమ్} \\
\natline{అభిజాతస్య భారత}
\end{tabular}
\end{table}

\subsection*{16.4}
\begin{table}[H]
\begin{tabular}{l}
\natline{దమ్భో దర్పోఽభిమానశ్చ} \\
\natline{క్రోధః పారుష్యమేవ చ} \\
\natline{అజ్ఞానం చాభిజాతస్య} \\
\natline{పార్థ సమ్పదమాసురీమ్}
\end{tabular}
\end{table}

\subsection*{16.5}
\begin{table}[H]
\begin{tabular}{l}
\natline{దైవీ సమ్పద్విమోక్షాయ} \\
\natline{నిబన్ధాయాసురీ మతా} \\
\natline{మా శుచః సమ్పదం దైవీమ్} \\
\natline{అభిజాతోఽసి పాణ్డవ}
\end{tabular}
\end{table}

\subsection*{16.6}
\begin{table}[H]
\begin{tabular}{l}
\natline{ద్వౌ భూతసర్గౌ లోకేఽస్మిన్} \\
\natline{దైవ ఆసుర ఏవ చ} \\
\natline{దైవో విస్తరశః ప్రోక్తః} \\
\natline{ఆసురం పార్థ మే శృణు}
\end{tabular}
\end{table}

\subsection*{16.7}
\begin{table}[H]
\begin{tabular}{l}
\natline{ప్రవృత్తిం చ నివృత్తిం చ} \\
\natline{జనా న విదురాసురాః} \\
\natline{న శౌచం నాపి చాచారః} \\
\natline{న సత్యం తేషు విద్యతే}
\end{tabular}
\end{table}

\subsection*{16.8}
\begin{table}[H]
\begin{tabular}{l}
\natline{అసత్యమప్రతిష్ఠం తే} \\
\natline{జగదాహురనీశ్వరమ్} \\
\natline{అపరస్పరసమ్భూతం} \\
\natline{కిమన్యత్కామహైతుకమ్}
\end{tabular}
\end{table}

\subsection*{16.9}
\begin{table}[H]
\begin{tabular}{l}
\natline{ఏతాం దృష్టిమవష్టభ్య} \\
\natline{నష్టాత్మానోఽల్పబుద్ధయః} \\
\natline{ప్రభవన్త్యుగ్రకర్మాణః} \\
\natline{క్షయాయ జగతోఽహితాః}
\end{tabular}
\end{table}

\subsection*{16.10}
\begin{table}[H]
\begin{tabular}{l}
\natline{కామమాశ్రిత్య దుష్పూరం} \\
\natline{దమ్భమానమదాన్వితాః} \\
\natline{మోహాద్గృహీత్వాసద్గ్రాహాన్} \\
\natline{ప్రవర్తన్తేఽశుచివ్రతాః}
\end{tabular}
\end{table}

\subsection*{16.11}
\begin{table}[H]
\begin{tabular}{l}
\natline{చిన్తామపరిమేయాం చ} \\
\natline{ప్రలయాన్తాముపాశ్రితాః} \\
\natline{కామోపభోగపరమాః} \\
\natline{ఏతావదితి నిశ్చితాః}
\end{tabular}
\end{table}

\subsection*{16.12}
\begin{table}[H]
\begin{tabular}{l}
\natline{ఆశాపాశశతైర్బద్ధాః} \\
\natline{కామక్రోధపరాయణాః} \\
\natline{ఈహన్తే కామభోగార్థమ్} \\
\natline{అన్యాయేనార్థసఞ్చయాన్}
\end{tabular}
\end{table}

\subsection*{16.13}
\begin{table}[H]
\begin{tabular}{l}
\natline{ఇదమద్య మయా లబ్ధమ్} \\
\natline{ఇమం ప్రాప్స్యే మనోరథమ్} \\
\natline{ఇదమస్తీదమపి మే} \\
\natline{భవిష్యతి పునర్ధనమ్}
\end{tabular}
\end{table}

\subsection*{16.14}
\begin{table}[H]
\begin{tabular}{l}
\natline{అసౌ మయా హతః శతృః} \\
\natline{హనిష్యే చాపరానపి} \\
\natline{ఈశ్వరోఽహమహం భోగీ} \\
\natline{సిద్ధోఽహం బలవాన్సుఖీ}
\end{tabular}
\end{table}

\subsection*{16.15}
\begin{table}[H]
\begin{tabular}{l}
\natline{ఆఢ్యోఽభిజనవానస్మి} \\
\natline{కోఽన్యోఽస్తి సదృశో మయా} \\
\natline{యక్ష్యే దాస్యామి మోదిష్యే} \\
\natline{ఇత్యజ్ఞానవిమోహితాః}
\end{tabular}
\end{table}

\subsection*{16.16}
\begin{table}[H]
\begin{tabular}{l}
\natline{అనేకచిత్తవిభ్రాన్తాః} \\
\natline{మోహజాలసమావృతాః} \\
\natline{ప్రసక్తాః కామభోగేషు} \\
\natline{పతన్తి నరకేఽశుచౌ}
\end{tabular}
\end{table}

\subsection*{16.17}
\begin{table}[H]
\begin{tabular}{l}
\natline{ఆత్మసమ్భావితాః స్తబ్ధాః} \\
\natline{ధనమానమదాన్వితాః} \\
\natline{యజన్తే నామయజ్ఞైస్తే} \\
\natline{దమ్భేనావిధిపూర్వకమ్}
\end{tabular}
\end{table}

\subsection*{16.18}
\begin{table}[H]
\begin{tabular}{l}
\natline{అహఙ్కారం బలం దర్పం} \\
\natline{కామం క్రోధం చ సంశ్రితాః} \\
\natline{మామాత్మపరదేహేషు} \\
\natline{ప్రద్విషన్తోఽభ్యసూయకాః}
\end{tabular}
\end{table}

\subsection*{16.19}
\begin{table}[H]
\begin{tabular}{l}
\natline{తానహం ద్విషతః క్రూరాన్} \\
\natline{సంసారేషు నరాధమాన్} \\
\natline{క్షిపామ్యజస్రమశుభాన్} \\
\natline{ఆసురీష్వేవ యోనిషు}
\end{tabular}
\end{table}

\subsection*{16.20}
\begin{table}[H]
\begin{tabular}{l}
\natline{ఆసురీం యోనిమాపన్నాః} \\
\natline{మూఢా జన్మని జన్మని} \\
\natline{మామప్రాప్యైవ కౌన్తేయ} \\
\natline{తతో యాన్త్యధమాం గతిమ్}
\end{tabular}
\end{table}

\subsection*{16.21}
\begin{table}[H]
\begin{tabular}{l}
\natline{త్రివిధం నరకస్యేదం} \\
\natline{ద్వారం నాశనమాత్మనః} \\
\natline{కామః క్రోధస్తథా లోభః} \\
\natline{తస్మాదేతత్త్రయం త్యజేత్}
\end{tabular}
\end{table}

\subsection*{16.22}
\begin{table}[H]
\begin{tabular}{l}
\natline{ఏతైర్విముక్తః కౌన్తేయ} \\
\natline{తమోద్వారైస్త్రిభిర్నరః} \\
\natline{ఆచరత్యాత్మనః శ్రేయః} \\
\natline{తతో యాతి పరాం గతిమ్}
\end{tabular}
\end{table}

\subsection*{16.23}
\begin{table}[H]
\begin{tabular}{l}
\natline{యః శాస్త్రవిధిముత్సృజ్య} \\
\natline{వర్తతే కామకారతః} \\
\natline{న స సిద్ధిమవాప్నోతి} \\
\natline{న సుఖం న పరాం గతిమ్}
\end{tabular}
\end{table}

\subsection*{16.24}
\begin{table}[H]
\begin{tabular}{l}
\natline{తస్మాచ్ఛాస్త్రం ప్రమాణం తే} \\
\natline{కార్యాకార్యవ్యవస్థితౌ} \\
\natline{జ్ఞాత్వా శాస్త్రవిధానోక్తం} \\
\natline{కర్మ కర్తుమిహార్హసి}
\end{tabular}
\end{table}


% \end{multicols}

\chapter{Śraddhātraya-Vibhāga Yoga}
% \begin{multicols}{2}
\begin{table}[H]
\begin{tabular}{cl}
\textbf{17.0} & \natline{ఓం శ్రీ పరమాత్మనే నమః} \\
 & \natline{అథ సప్తదశోఽధ్యాయః} \\
 & \natline{శ్రద్ధాత్రయవిభాగ యోగః}
\end{tabular}
\end{table}

\begin{table}[H]
\begin{tabular}{cl}
\textbf{17.1} & \natline{అర్జున ఉవాచ} \\
 & \natline{యే శాస్త్రవిధిముత్సృజ్య} \\
 & \natline{యజన్తే శ్రద్ధయాన్వితాః |} \\
 & \natline{తేషాం నిష్ఠా తు కా కృష్ణ} \\
 & \natline{సత్త్వమాహో రజస్తమః ||}
\end{tabular}
\end{table}

\begin{table}[H]
\begin{tabular}{cl}
\textbf{17.2} & \natline{శ్రీ భగవానువాచ} \\
 & \natline{త్రివిధా భవతి శ్రద్ధా} \\
 & \natline{దేహినాం సా స్వభావజా |} \\
 & \natline{సాత్త్వికీ రాజసీ చైవ} \\
 & \natline{తామసీ చేతి తాం శృణు ||}
\end{tabular}
\end{table}

\begin{table}[H]
\begin{tabular}{cl}
\textbf{17.3} & \natline{సత్త్వానురూపా సర్వస్య} \\
 & \natline{శ్రద్ధా భవతి భారత |} \\
 & \natline{శ్ర్రద్ధామయోఽయం పురుషః} \\
 & \natline{యో యcచ్రద్ధః స ఏవ సః ||}
\end{tabular}
\end{table}

\begin{table}[H]
\begin{tabular}{cl}
\textbf{17.4} & \natline{యజన్తే సాత్త్వికా దేవాన్} \\
 & \natline{యక్షరక్షాంసి రాజసాః |} \\
 & \natline{ప్రేతాన్ భూతగణాంశ్చాన్యే} \\
 & \natline{యజన్తే తామసా జనాః ||}
\end{tabular}
\end{table}

\begin{table}[H]
\begin{tabular}{cl}
\textbf{17.5} & \natline{అశాస్త్రవిహితం ఘోరం} \\
 & \natline{తప్యతే యే తపో జనాః |} \\
 & \natline{దమ్భాహన్కారసమ్యుతాః} \\
 & \natline{కామరాగబలాన్వితాః ||}
\end{tabular}
\end{table}

\begin{table}[H]
\begin{tabular}{cl}
\textbf{17.6} & \natline{కర్శయన్తః శరీరస్థం} \\
 & \natline{భూతగ్రామమcఏతసః |} \\
 & \natline{మాం చైవాన్తః శరీరస్థం} \\
 & \natline{తాన్ విద్ధ్యాసురనిశ్చయాన్ ||}
\end{tabular}
\end{table}

\begin{table}[H]
\begin{tabular}{cl}
\textbf{17.7} & \natline{ఆహారస్త్వపి సర్వస్య} \\
 & \natline{త్రివిధో భవతి ప్రియః |} \\
 & \natline{యజ్ఞస్తపస్తథా దానం} \\
 & \natline{తేషాం భేదమిమమ్ శృణు ||}
\end{tabular}
\end{table}

\begin{table}[H]
\begin{tabular}{cl}
\textbf{17.8} & \natline{ఆయుస్సత్త్వబలారోగ్య} \\
 & \natline{సుఖప్రీతివివర్ధనాః |} \\
 & \natline{రస్యాః స్నిగ్ధాః స్థిరా హృద్యాః} \\
 & \natline{ఆహారాః సాత్త్వికప్రియాః ||}
\end{tabular}
\end{table}

\begin{table}[H]
\begin{tabular}{cl}
\textbf{17.9} & \natline{కట్వమ్లలవణాత్యుష్ణ} \\
 & \natline{తీక్ష్ణరూక్షవిదాహినః |} \\
 & \natline{అహారా రాజసస్యేష్టాః} \\
 & \natline{దుఃఖశోకామయప్రదాః ||}
\end{tabular}
\end{table}

\begin{table}[H]
\begin{tabular}{cl}
\textbf{17.10} & \natline{యాతయామం గతరసం} \\
 & \natline{పూతి పర్యుషితం చ యత్ |} \\
 & \natline{ఉచ్ఛిష్టమపి చామేధ్యం} \\
 & \natline{భోజనం తామసప్రియమ్ ||}
\end{tabular}
\end{table}

\begin{table}[H]
\begin{tabular}{cl}
\textbf{17.11} & \natline{అఫలాకాఙ్క్షిభిర్యజ్ఞః} \\
 & \natline{విధిదృష్టో య ఇజ్యతే |} \\
 & \natline{యష్టవ్యమేవేతి మనః} \\
 & \natline{సమాధాయ స సాత్త్వికః ||}
\end{tabular}
\end{table}

\begin{table}[H]
\begin{tabular}{cl}
\textbf{17.12} & \natline{అభిసన్ధాయ తు ఫలం} \\
 & \natline{దమ్భార్థమపి చైవ యత్ |} \\
 & \natline{ఇజ్యతే భరతశ్రేష్ఠ} \\
 & \natline{తం యజ్ఞం విద్ధి రాజసమ్ ||}
\end{tabular}
\end{table}

\begin{table}[H]
\begin{tabular}{cl}
\textbf{17.13} & \natline{విధిహీనమసృష్టాన్నం} \\
 & \natline{మన్త్రహీనమదక్షిణమ్ |} \\
 & \natline{శ్రద్ధావిరహితం యజ్ఞం} \\
 & \natline{తామసం పరిచక్షతే ||}
\end{tabular}
\end{table}

\begin{table}[H]
\begin{tabular}{cl}
\textbf{17.14} & \natline{దేవద్విజగురుప్రాజ్ఞ} \\
 & \natline{ప్ū̀జనం శౌచమార్జవమ్ |} \\
 & \natline{బ్రహ్మచర్యమహింసా చ} \\
 & \natline{శారీరం తప ఉచ్యతే ||}
\end{tabular}
\end{table}

\begin{table}[H]
\begin{tabular}{cl}
\textbf{17.15} & \natline{అనుద్వేగకరం వాక్యం} \\
 & \natline{సత్యం ప్రియహితం చ యత్ |} \\
 & \natline{స్వాధ్యాయాభ్యసనం చైవ} \\
 & \natline{వాఙ్మయం తప ఉచ్యతే ||}
\end{tabular}
\end{table}

\begin{table}[H]
\begin{tabular}{cl}
\textbf{17.16} & \natline{మనః ప్రసాదః సౌమ్యత్వం} \\
 & \natline{మౌనమాత్మవినిగ్రహః |} \\
 & \natline{భావసంశుద్ధిరిత్యేతత్} \\
 & \natline{తపో మానసముచ్యతే ||}
\end{tabular}
\end{table}

\begin{table}[H]
\begin{tabular}{cl}
\textbf{17.17} & \natline{శ్రద్ధయా పరయా తప్తం} \\
 & \natline{తపస్తత్ త్రివిధం నరైః |} \\
 & \natline{అఫలాకాఙ్క్షిభిర్యుక్తైః} \\
 & \natline{సత్త్వికం పరిచక్షతే ||}
\end{tabular}
\end{table}

\begin{table}[H]
\begin{tabular}{cl}
\textbf{17.18} & \natline{సత్కారమానపూజార్థం} \\
 & \natline{తపో దమ్భేన చైవ యత్ |} \\
 & \natline{క్రియతే తదిహ ప్రోక్తం} \\
 & \natline{రాజసం చలమధ్రువమ్ ||}
\end{tabular}
\end{table}

\begin{table}[H]
\begin{tabular}{cl}
\textbf{17.19} & \natline{మూఢగ్రాహేణ్ā́త్మనో యత్} \\
 & \natline{పీడయా క్రియతే తపః |} \\
 & \natline{పర్రస్యోత్సాదనార్థం వా} \\
 & \natline{తత్తామసముదాహ్Rఋతమ్ ||}
\end{tabular}
\end{table}

\begin{table}[H]
\begin{tabular}{cl}
\textbf{17.20} & \natline{దాతవ్యమితి యద్దానం} \\
 & \natline{దీయతేఽనుపకారిణే |} \\
 & \natline{దేశే కాలే చ పాత్రే చ} \\
 & \natline{తద్దానం సాత్త్వికం స్మృతమ్ ||}
\end{tabular}
\end{table}

\begin{table}[H]
\begin{tabular}{cl}
\textbf{17.21} & \natline{యత్తు ప్రత్యుపకారార్థం} \\
 & \natline{ఫలముద్దిశ్య వా పునః |} \\
 & \natline{దీయతే చ పరిక్లిష్టం} \\
 & \natline{తద్దానం రాజసం స్మృతమ్ ||}
\end{tabular}
\end{table}

\begin{table}[H]
\begin{tabular}{cl}
\textbf{17.22} & \natline{అదేశకాలే యద్దానమ్} \\
 & \natline{అపాత్రేభ్యశ్చ దీయతే |} \\
 & \natline{అసత్కృతమవజ్ఞాతం} \\
 & \natline{తత్తామసముదాహృతమ్ ||}
\end{tabular}
\end{table}

\begin{table}[H]
\begin{tabular}{cl}
\textbf{17.23} & \natline{ఓం తత్సదితి నిర్దేశః} \\
 & \natline{బ్రహ్మణస్త్రివిధః స్మృతః |} \\
 & \natline{బ్రాహ్మణాస్తేన వేదాశ్చ} \\
 & \natline{యజ్ఞాశ్చ విహితాః పురా ||}
\end{tabular}
\end{table}

\begin{table}[H]
\begin{tabular}{cl}
\textbf{17.24} & \natline{తస్మాదోమిత్యుదాహృత్య} \\
 & \natline{యజ్ఞదానతపఃక్రియాః |} \\
 & \natline{ప్రవర్తన్తే విధానోక్తాః} \\
 & \natline{సతతం బ్రహ్మవాదినామ్ ||}
\end{tabular}
\end{table}

\begin{table}[H]
\begin{tabular}{cl}
\textbf{17.25} & \natline{తదిత్యనభిసన్ధాయ} \\
 & \natline{ఫలం యజ్ఞతపఃక్రియాః |} \\
 & \natline{దానక్రియాశ్చ వివిధాః} \\
 & \natline{క్రియన్తే మోక్షకాఙ్క్షిభిః ||}
\end{tabular}
\end{table}

\begin{table}[H]
\begin{tabular}{cl}
\textbf{17.26} & \natline{సద్భావే సాధుభావే చ} \\
 & \natline{సదిత్యేతత్ప్రయుజ్యతే |} \\
 & \natline{ప్రశస్తే కర్మణి తథా} \\
 & \natline{సచ్చ్హబ్దః పార్థ యుజ్యతే ||}
\end{tabular}
\end{table}

\begin{table}[H]
\begin{tabular}{cl}
\textbf{17.27} & \natline{యజ్ఞే తపసి దానే చ} \\
 & \natline{స్థితిః సదితి చోచ్యతే |} \\
 & \natline{కర్మ చైవ తదర్థీయం} \\
 & \natline{సదిత్యేవాభిధీయతే ||}
\end{tabular}
\end{table}

\begin{table}[H]
\begin{tabular}{cl}
\textbf{17.28} & \natline{అశ్రద్ధయా హుతం దత్తం} \\
 & \natline{తపస్తప్తం కృతం చ యత్ |} \\
 & \natline{అసదిత్యుచ్యతే పార్థ} \\
 & \natline{న చ తత్ప్రేత్య నో ఇహ ||}
\end{tabular}
\end{table}


% \end{multicols}

\chapter{Mokṣa-Sannyāsa Yoga}
% \begin{multicols}{2}
\begin{table}[H]
\begin{tabular}{cl}
\textbf{18.0} & \natline{ఓం శ్రీ పరమాత్మనే నమః} \\
 & \natline{అథ అష్టాదశోఽధ్యాయః} \\
 & \natline{మోక్షసన్న్యాస యోగః}
\end{tabular}
\end{table}

\begin{table}[H]
\begin{tabular}{cl}
\textbf{18.1} & \natline{అర్జున ఉవాచ} \\
 & \natline{సన్న్యాసస్య మహాబాహో} \\
 & \natline{తత్త్వమిచ్ఛామి వేదితుమ్ |} \\
 & \natline{త్యాగస్య చ హృషికేశ} \\
 & \natline{పృథక్కేశినిషూదన ||}
\end{tabular}
\end{table}

\begin{table}[H]
\begin{tabular}{cl}
\textbf{18.2} & \natline{శ్రీ భగవనువాచ} \\
 & \natline{కామ్యానాం కర్మణాం న్యాసం} \\
 & \natline{సన్న్యాసం కవయో విదుః |} \\
 & \natline{సర్వకర్మఫలత్యాగం} \\
 & \natline{ప్రాహుస్త్యాగం విచక్షణాః ||}
\end{tabular}
\end{table}

\begin{table}[H]
\begin{tabular}{cl}
\textbf{18.3} & \natline{త్యాజ్యం దోషవదిత్యేకే} \\
 & \natline{కర్మ ప్రాహుర్మనీషిణః |} \\
 & \natline{యజ్ఞదానతపః కర్మ} \\
 & \natline{న త్యాజ్యమితి చాపరే ||}
\end{tabular}
\end{table}

\begin{table}[H]
\begin{tabular}{cl}
\textbf{18.4} & \natline{నిశ్చయం శృణు మే తత్ర} \\
 & \natline{త్యాగే భరతసత్తమ |} \\
 & \natline{త్యాగో హి పురుషవ్యాఘ్ర} \\
 & \natline{త్రివిధః సమ్ప్రకీర్తితః ||}
\end{tabular}
\end{table}

\begin{table}[H]
\begin{tabular}{cl}
\textbf{18.5} & \natline{యజ్ఞదానతపఃకర్మ} \\
 & \natline{న త్యాజ్యం కార్యమేవ తత్ |} \\
 & \natline{యజ్ఞో దానం తపశ్చైవ} \\
 & \natline{పావనాని మనీషిణామ్ ||}
\end{tabular}
\end{table}

\begin{table}[H]
\begin{tabular}{cl}
\textbf{18.6} & \natline{ఏతాన్యపి తు కర్మాణి} \\
 & \natline{సఙ్గం త్యక్త్వా ఫలాని చ |} \\
 & \natline{కర్తవ్యానీతి మే పార్థ} \\
 & \natline{నిశ్చితం మతముత్తమమ్ ||}
\end{tabular}
\end{table}

\begin{table}[H]
\begin{tabular}{cl}
\textbf{18.7} & \natline{నియతస్య తు సన్న్యాసః} \\
 & \natline{కర్మణో నోపపద్యతే |} \\
 & \natline{మోహాత్తస్య పరిత్యాగః} \\
 & \natline{తామసః పరికీర్తితః ||}
\end{tabular}
\end{table}

\begin{table}[H]
\begin{tabular}{cl}
\textbf{18.8} & \natline{దుఃఖమిత్యేవ యత్కర్మ} \\
 & \natline{కాయక్లేశభయాత్త్యజేత్ |} \\
 & \natline{స కృత్వా రాజసం త్యాగం} \\
 & \natline{నైవ త్యాగఫలం లభేత్ ||}
\end{tabular}
\end{table}

\begin{table}[H]
\begin{tabular}{cl}
\textbf{18.9} & \natline{కార్యమిత్యేవ యత్కర్మ} \\
 & \natline{నియతం క్రియతేఽర్జున |} \\
 & \natline{సఙ్గం త్యక్త్వా ఫలం చైవ} \\
 & \natline{స త్యాగః సాత్త్వికో మతః ||}
\end{tabular}
\end{table}

\begin{table}[H]
\begin{tabular}{cl}
\textbf{18.10} & \natline{న ద్వేష్ట్యకుశలం కర్మ} \\
 & \natline{కుశలే నానుషజ్జతే |} \\
 & \natline{త్యాగీ సత్త్వసమావిష్టః} \\
 & \natline{మేధావీ ఛిన్నసంశయః ||}
\end{tabular}
\end{table}

\begin{table}[H]
\begin{tabular}{cl}
\textbf{18.11} & \natline{న హి దేహభృతా శక్యం} \\
 & \natline{త్యక్తుం కర్మాణ్యశేషతః |} \\
 & \natline{యస్తు కర్మఫలత్యాగీ} \\
 & \natline{స త్యాగీత్యభిధీయతే ||}
\end{tabular}
\end{table}

\begin{table}[H]
\begin{tabular}{cl}
\textbf{18.12} & \natline{అనిష్టమిష్టం మిశ్రం చ} \\
 & \natline{త్రివిధం కర్మణః ఫలమ్ |} \\
 & \natline{భవత్యత్యాగినాం ప్రేత్య} \\
 & \natline{న తు సన్న్యాసినాం క్వచిత్ ||}
\end{tabular}
\end{table}

\begin{table}[H]
\begin{tabular}{cl}
\textbf{18.13} & \natline{పఞ్చైతాని మహాబాహో} \\
 & \natline{కారణాని నిబోధ మే |} \\
 & \natline{సాఙ్ఖ్యే కృతాన్తే ప్రోక్తాని} \\
 & \natline{సిద్ధయే సర్వకర్మణామ్ ||}
\end{tabular}
\end{table}

\begin{table}[H]
\begin{tabular}{cl}
\textbf{18.14} & \natline{అధిష్ఠానం తథా కర్తా} \\
 & \natline{కరణం చ పృథగ్విధమ్ |} \\
 & \natline{వివిధాశ్చ పృథక్చేష్టాః} \\
 & \natline{దైవం చైవాత్ర పఞ్చమమ్ ||}
\end{tabular}
\end{table}

\begin{table}[H]
\begin{tabular}{cl}
\textbf{18.15} & \natline{శరీరవాఙ్మనోభిర్యత్} \\
 & \natline{కర్మ ప్రారభతే నరః |} \\
 & \natline{న్యాయ్యం వా విపరీతం వా} \\
 & \natline{పఞ్చైతే తస్య హేతవః ||}
\end{tabular}
\end{table}

\begin{table}[H]
\begin{tabular}{cl}
\textbf{18.16} & \natline{తత్రైవం సతి కర్తారమ్} \\
 & \natline{ఆత్మానం కేవలం తు యః |} \\
 & \natline{పశ్యత్యకృతబుద్ధిత్వాత్} \\
 & \natline{న స పశ్యతి దుర్మతిః ||}
\end{tabular}
\end{table}

\begin{table}[H]
\begin{tabular}{cl}
\textbf{18.17} & \natline{యస్య నాహఙ్కృతో భావః} \\
 & \natline{బుద్ధిర్యస్య న లిప్యతే |} \\
 & \natline{హత్వాఽపి స ఇమాల్లోకాన్} \\
 & \natline{న హన్తి న నిబధ్యతే ||}
\end{tabular}
\end{table}

\begin{table}[H]
\begin{tabular}{cl}
\textbf{18.18} & \natline{జ్ఞానం జ్ఞేయం పరిజ్ఞాతా} \\
 & \natline{త్రివిధా కర్మచోదనా |} \\
 & \natline{కరణం కర్మ కర్తేతి} \\
 & \natline{త్రివిధః కర్మసఙ్గ్రహః ||}
\end{tabular}
\end{table}

\begin{table}[H]
\begin{tabular}{cl}
\textbf{18.19} & \natline{జ్ఞానం కర్మ చ కర్తా చ} \\
 & \natline{త్రిధైవ గుణభేదతః |} \\
 & \natline{ప్రోచ్యతే గుణసఙ్ఖ్యానే} \\
 & \natline{యథావచ్ఛృణు తాన్యపి ||}
\end{tabular}
\end{table}

\begin{table}[H]
\begin{tabular}{cl}
\textbf{18.20} & \natline{సర్వభూతేషు యేనైకం} \\
 & \natline{భావమవ్యయమీక్షతే |} \\
 & \natline{అవిభక్తం విభక్తేషు} \\
 & \natline{తజ్జ్ఞానం విద్ధి సాత్త్వికమ్ ||}
\end{tabular}
\end{table}

\begin{table}[H]
\begin{tabular}{cl}
\textbf{18.21} & \natline{పృథక్త్వేన తు యజ్జ్ఞానం} \\
 & \natline{నానాభావాన్ పృథగ్విధాన్ |} \\
 & \natline{వేత్తి సర్వేషు భూతేషు} \\
 & \natline{తజ్జ్ఞానం విద్ధి రాజసమ్ ||}
\end{tabular}
\end{table}

\begin{table}[H]
\begin{tabular}{cl}
\textbf{18.22} & \natline{యత్తు కృత్స్నవదేకస్మిన్} \\
 & \natline{కార్యే సక్తమహైతుకమ్ |} \\
 & \natline{అతత్త్వార్థవదల్పం చ} \\
 & \natline{తత్తామసముదాహృతమ్ ||}
\end{tabular}
\end{table}

\begin{table}[H]
\begin{tabular}{cl}
\textbf{18.23} & \natline{నియతం సఙ్గరహితమ్} \\
 & \natline{అరాగద్వేషతః కృతమ్ |} \\
 & \natline{అఫలప్రేప్సునా కర్మ} \\
 & \natline{యత్తత్సాత్త్వికముచ్యతే ||}
\end{tabular}
\end{table}

\begin{table}[H]
\begin{tabular}{cl}
\textbf{18.24} & \natline{యత్తు కామేప్సునా కర్మ} \\
 & \natline{సాహఙ్కారేణ వా పునః |} \\
 & \natline{క్రియతే బహులాయాసం} \\
 & \natline{తద్రాజసముదాహృతమ్ ||}
\end{tabular}
\end{table}

\begin{table}[H]
\begin{tabular}{cl}
\textbf{18.25} & \natline{అనుబన్ధం క్షయం హింసామ్} \\
 & \natline{అనపేక్ష్య చ పౌరుషమ్ |} \\
 & \natline{మోహాదారభ్యతే కర్మ} \\
 & \natline{యత్తత్తామసముచ్యతే ||}
\end{tabular}
\end{table}

\begin{table}[H]
\begin{tabular}{cl}
\textbf{18.26} & \natline{ముక్తసఙ్గోఽనహంవాదీ} \\
 & \natline{ధృత్యుత్సాహసమన్వితః |} \\
 & \natline{సిద్ధ్యసిద్ధ్యోర్నిర్వికారః} \\
 & \natline{కర్తా సాత్త్విక ఉచ్యతే ||}
\end{tabular}
\end{table}

\begin{table}[H]
\begin{tabular}{cl}
\textbf{18.27} & \natline{రాగీ కర్మఫలప్రేప్సుః} \\
 & \natline{లుబ్ధో హింసాత్మకోఽశుచిః |} \\
 & \natline{హర్షశోకాన్వితః కర్తా} \\
 & \natline{రాజసః పరికీర్తితః ||}
\end{tabular}
\end{table}

\begin{table}[H]
\begin{tabular}{cl}
\textbf{18.28} & \natline{అయుక్తః ప్రాకృతః స్తబ్ధః} \\
 & \natline{శఠో నైష్కృతికోఽలసః |} \\
 & \natline{విషాదీ దీర్ఘసూత్రీ చ} \\
 & \natline{కర్తా తామస ఉచ్యతే ||}
\end{tabular}
\end{table}

\begin{table}[H]
\begin{tabular}{cl}
\textbf{18.29} & \natline{బుద్ధేర్భేదం ధృతేశ్చైవ} \\
 & \natline{గుణతస్త్రివిధం శృణు |} \\
 & \natline{ప్రోచ్యమానమశేషేణ} \\
 & \natline{పృథక్త్వేన ధనఞ్జయ ||}
\end{tabular}
\end{table}

\begin{table}[H]
\begin{tabular}{cl}
\textbf{18.30} & \natline{ప్రవృత్తిం చ నివృత్తిం చ} \\
 & \natline{కార్యాకార్యే భయాభయే |} \\
 & \natline{బన్ధం మోక్షం చ యా వేత్తి} \\
 & \natline{బుద్ధిః సా పార్థ సాత్త్వికీ ||}
\end{tabular}
\end{table}

\begin{table}[H]
\begin{tabular}{cl}
\textbf{18.31} & \natline{యయా ధర్మమధర్మం చ} \\
 & \natline{కార్యం చాకార్యమేవ చ |} \\
 & \natline{అయథావత్ప్రజానాతి} \\
 & \natline{బుద్ధిః సా పార్థ రాజసీ ||}
\end{tabular}
\end{table}

\begin{table}[H]
\begin{tabular}{cl}
\textbf{18.32} & \natline{అధర్మం ధర్మమితి యా} \\
 & \natline{మన్యతే తమసాఽఽవృతా |} \\
 & \natline{సర్వార్థాన్విపరీతాంశ్చ} \\
 & \natline{బుద్ధిః సా పార్థ తామసీ ||}
\end{tabular}
\end{table}

\begin{table}[H]
\begin{tabular}{cl}
\textbf{18.33} & \natline{ధృత్యా యయా ధారయతే} \\
 & \natline{మనః ప్రాణేన్ద్రియక్రియాః |} \\
 & \natline{యోగేనావ్యభిచారిణ్యా} \\
 & \natline{ధృతిః సా పార్థ సాత్త్వికీ ||}
\end{tabular}
\end{table}

\begin{table}[H]
\begin{tabular}{cl}
\textbf{18.34} & \natline{యయా తు ధర్మకామార్థాన్} \\
 & \natline{ధృత్యా ధారయతేఽర్జున |} \\
 & \natline{ప్రసఙ్గేన ఫలాకాఙ్క్షీ} \\
 & \natline{ధృతిః సా పార్థ రాజసీ ||}
\end{tabular}
\end{table}

\begin{table}[H]
\begin{tabular}{cl}
\textbf{18.35} & \natline{యయా స్వప్నం భయం శోకం} \\
 & \natline{విషాదం మదమేవ చ |} \\
 & \natline{న విముఞ్చతి దుర్మేధాః} \\
 & \natline{ధృతిః సా తామసీ మతా ||}
\end{tabular}
\end{table}

\begin{table}[H]
\begin{tabular}{cl}
\textbf{18.36} & \natline{సుఖం త్విదానీం త్రివిధం} \\
 & \natline{శృణు మే భరతర్షభ |} \\
 & \natline{అభ్యాసాద్రమతే యత్ర} \\
 & \natline{దుఃఖాన్తం చ నిగచ్ఛతి ||}
\end{tabular}
\end{table}

\begin{table}[H]
\begin{tabular}{cl}
\textbf{18.37} & \natline{యత్తదగ్రే విషమివ} \\
 & \natline{పరిణామేఽమృతోపమమ్ |} \\
 & \natline{తత్సుఖం సాత్త్వికం ప్రోక్తమ్} \\
 & \natline{ఆత్మబుద్ధిప్రసాదజమ్ ||}
\end{tabular}
\end{table}

\begin{table}[H]
\begin{tabular}{cl}
\textbf{18.38} & \natline{విషయేన్ద్రియసంయోగాత్} \\
 & \natline{యత్తదగ్రేఽమృతోపమమ్ |} \\
 & \natline{పరిణామే విషమివ} \\
 & \natline{తత్సుఖం రాజసం స్మృతమ్ ||}
\end{tabular}
\end{table}

\begin{table}[H]
\begin{tabular}{cl}
\textbf{18.39} & \natline{యదగ్రే చానుబన్ధే చ} \\
 & \natline{సుఖం మోహనమాత్మనః |} \\
 & \natline{నిద్రాలస్యప్రమాదోత్థం} \\
 & \natline{తత్తామసముదాహృతమ్ ||}
\end{tabular}
\end{table}

\begin{table}[H]
\begin{tabular}{cl}
\textbf{18.40} & \natline{న తదస్తి పృథివ్యాం వా} \\
 & \natline{దివి దేవేషు వా పునః |} \\
 & \natline{సత్త్వం ప్రకృతిజైర్ముక్తం} \\
 & \natline{యదేభిః స్యాత్త్రిభిర్గుణైః ||}
\end{tabular}
\end{table}

\begin{table}[H]
\begin{tabular}{cl}
\textbf{18.41} & \natline{బ్రాహ్మణక్షత్రియవిశాం} \\
 & \natline{శూద్రాణాం చ పరన్తప |} \\
 & \natline{కర్మాణి ప్రవిభక్తాని} \\
 & \natline{స్వభావప్రభవైర్గుణైః ||}
\end{tabular}
\end{table}

\begin{table}[H]
\begin{tabular}{cl}
\textbf{18.42} & \natline{శమో దమస్తపః శౌచం} \\
 & \natline{షాన్తిరార్జవమేవ చ |} \\
 & \natline{జ్ఞానం విజ్ఞానమాస్తిక్యం} \\
 & \natline{బ్రహ్మకర్మ స్వభావజమ్ ||}
\end{tabular}
\end{table}

\begin{table}[H]
\begin{tabular}{cl}
\textbf{18.43} & \natline{శౌర్యం తేజో ధృతిర్దాక్ష్యం} \\
 & \natline{యుద్ధే చాప్యపలాయనమ్ |} \\
 & \natline{దానమీశ్వరభావశ్చ} \\
 & \natline{క్షాత్రం కర్మ స్వభావజమ్ ||}
\end{tabular}
\end{table}


% \end{multicols}

\backmatter
\chapter{Closing Ślokas}
\begin{table}[H]
\begin{tabular}{l}
\romline{tatsaditi śrīmad-bhagavadgītāsu} \\
\romline{upaniṣatsu brahmavidyāyāṃ yogaśāstre} \\
\romline{śrīkṛṣṇārjuna saṃvāde} \\
\romline{(bhakti-yogonāma dvādaśodhyāyaḥ)}
\end{tabular}
\end{table}

\begin{table}[H]
\begin{tabular}{l}
\romline{sarvadharmān-parityajya} \\
\romline{māmekaṃ śaraṇaṃ vraja} \\
\romline{aham tvā sarvapāpebhyaḥ} \\
\romline{mokṣayiṣyāmi mā śucaḥ}
\end{tabular}
\end{table}

\begin{table}[H]
\begin{tabular}{l}
\romline{yataḥ pravṛttir-bhūtānāṃ} \\
\romline{yena sarvam-idaṃ tatam} \\
\romline{sva-karmaṇā tam-abhyarcya} \\
\romline{siddhiṃ viṃdati mānavaḥ}
\end{tabular}
\end{table}

\begin{table}[H]
\begin{tabular}{l}
\romline{yatra yogeśvaraḥ kṛṣṇaḥ} \\
\romline{yatra pārtho dhanur-dharaḥ} \\
\romline{tatra śrīr-vijayo bhūtiḥ} \\
\romline{dhruvā nītir-matir-mama}
\end{tabular}
\end{table}

\begin{table}[H]
\begin{tabular}{l}
\romline{śrī kṛṣṇaśśaraṇaṃ mama} \\
\romline{śrī kṛṣṇaśśaraṇaṃ mama} \\
\romline{śrī kṛṣṇaśśaraṇaṃ mama}
\end{tabular}
\end{table}



\end{document}