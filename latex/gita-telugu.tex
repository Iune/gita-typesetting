%!TEX program = xelatex
\documentclass{scrbook}
\KOMAoptions{paper=6in:9in}
\KOMAoptions{fontsize=11pt}
\KOMAoptions{toc=chapterentrydotfill}
\KOMAoptions{twoside=false}

\addtokomafont{disposition}{\rmfamily}
\renewcommand*{\raggedsection}{\centering}
% \usepackage[left=1in,right=1in,top=1in,bottom=1in,]{geometry}

% Formatting
\usepackage{multicol}
\usepackage{float}
\usepackage{hyperref}

% Fonts
\usepackage{fontspec}
\setmainfont[
    Path = assets/,
    UprightFont = Junicode.ttf,
    BoldFont = Junicode-Bold.ttf]{Junicode}
\newfontfamily{\telfont}[
    Path = assets/, 
    UprightFont = TiroTelugu-Regular.ttf,
    ItalicFont = TiroTelugu-Italic.ttf,
    Scale=MatchUppercase, Script=Telugu]{Tiro Telugu}


% Commands
\newcommand{\tel}[1]{\begin{telfont}{#1}\end{telfont}}
\newcommand{\romline}[1]{{#1}}
\newcommand{\natline}[1]{\tel{#1}}

% Variables

\title{Bhagavad Gita}
\subtitle{Typeset in Telugu}
\author{Aditya Duri}
\date{}

\begin{document}
\maketitle
\frontmatter

\tableofcontents
\newpage

\chapter{Dhyāna Ślokas}
\begin{table}[H]
\begin{tabular}{cl}
 & \natline{श्री वल्लभा समाश्लिष्टं} \\
 & \natline{दशहस्तं गजाननं |} \\
 & \natline{गणनाथ महं वंदे} \\
 & \natline{सर्व सिद्धि प्रदायकम् ||}
\end{tabular}
\end{table}

\begin{table}[H]
\begin{tabular}{cl}
 & \natline{प्रपन्न पारिजाताय} \\
 & \natline{तोत्रवेत्रैक पाणये |} \\
 & \natline{ज्ञानमुद्राय कृष्णाय} \\
 & \natline{गीतांऋतदुहे नमः ||}
\end{tabular}
\end{table}

\begin{table}[H]
\begin{tabular}{cl}
 & \natline{व्यासाय विष्णु रूपाय} \\
 & \natline{व्यास रूपाय विष्णवे |} \\
 & \natline{नमो वै ब्रह्म निधये} \\
 & \natline{वासिष्ठाय नमो नमः ||}
\end{tabular}
\end{table}

\begin{table}[H]
\begin{tabular}{cl}
 & \natline{सर्वोपनिषदो गावः} \\
 & \natline{दोग्धा गोपालनन्दनः |} \\
 & \natline{पार्थोवत्सः सुधीर्भोक्ता} \\
 & \natline{दुग्धं गीतामृतं महत् ||}
\end{tabular}
\end{table}

\begin{table}[H]
\begin{tabular}{cl}
 & \natline{सच्चिदानन्द रुपाय} \\
 & \natline{विश्वोत्पत्यादिहेतवे |} \\
 & \natline{तापत्रयविनाशाय} \\
 & \natline{श्री कृष्णाय नमो नमः ||}
\end{tabular}
\end{table}

\begin{table}[H]
\begin{tabular}{cl}
 & \natline{श्री कृष्णश्शरणं मम} \\
 & \natline{श्री कृष्णश्शरणं मम} \\
 & \natline{श्री कृष्णश्शरणं मम}
\end{tabular}
\end{table}



\mainmatter
\chapter{Arjuna Viṣāda Yoga}
% \begin{multicols}{2}
\begin{table}[H]
\begin{tabular}{cl}
 & \natline{శ్రీ పరమాత్మనే నమః} \\
 & \natline{అథ ప్రథమోఽధ్యాయః} \\
 & \natline{అర్జునవిషాదయోగః}
\end{tabular}
\end{table}

\begin{table}[H]
\begin{tabular}{cl}
\textbf{1.1} & \natline{ధృతరాష్ట్ర ఉవాచ} \\
 & \natline{ధర్మక్షేత్రే కురుక్షేత్రే} \\
 & \natline{సమవేతా యుయుత్సవః |} \\
 & \natline{మామకాః పాణ్డవాశ్చైవ} \\
 & \natline{కిమకుర్వత సఞ్జయ ||}
\end{tabular}
\end{table}

\begin{table}[H]
\begin{tabular}{cl}
\textbf{1.2} & \natline{సఞ్జయ ఉవాచ} \\
 & \natline{దృష్ట్వా తు పాణ్డవానీకం} \\
 & \natline{వ్యూఢం దుర్యోధనస్తదా |} \\
 & \natline{ఆచార్యముపసఙ్గమ్య} \\
 & \natline{రాజా వచనమబ్రవీత్ ||}
\end{tabular}
\end{table}

\begin{table}[H]
\begin{tabular}{cl}
\textbf{1.3} & \natline{పశ్యైతాం పాణ్డుపుత్రాణామ్} \\
 & \natline{ఆచార్య మహతీం చమూమ్ |} \\
 & \natline{వ్యూఢాం ద్రుపదపుత్రేణ} \\
 & \natline{తవ శిష్యేణ ధీమతా ||}
\end{tabular}
\end{table}

\begin{table}[H]
\begin{tabular}{cl}
\textbf{1.4} & \natline{అత్ర శూరా మహేష్వాసాః} \\
 & \natline{భీమార్జునసమా యుధి |} \\
 & \natline{యుయుధానో విరాటశ్చ} \\
 & \natline{ద్రుపదశ్చ మహారథః ||}
\end{tabular}
\end{table}

\begin{table}[H]
\begin{tabular}{cl}
\textbf{1.5} & \natline{ధృష్టకేతుశ్చేకితానః} \\
 & \natline{కాశిరాజశ్చ వీర్యవాన్ |} \\
 & \natline{పురుజిత్కున్తిభోజశ్చ} \\
 & \natline{శైబ్యశ్చ నరపుఙ్గవః ||}
\end{tabular}
\end{table}

\begin{table}[H]
\begin{tabular}{cl}
\textbf{1.6} & \natline{యుధామన్యుశ్చ విక్రాన్తః} \\
 & \natline{ఉత్తమౌజాశ్చ వీర్యవాన్ |} \\
 & \natline{సౌభద్రో ద్రౌపదేయాశ్చ} \\
 & \natline{సర్వ ఏవ మహారథాః ||}
\end{tabular}
\end{table}

\begin{table}[H]
\begin{tabular}{cl}
\textbf{1.7} & \natline{అస్మాకం తు విశిష్టా యే} \\
 & \natline{తాన్నిబోధ ద్విజోత్తమ |} \\
 & \natline{నాయకా మమ సైన్యస్య} \\
 & \natline{సఞ్జ్ఞార్థం తాన్బ్రవీమి తే ||}
\end{tabular}
\end{table}

\begin{table}[H]
\begin{tabular}{cl}
\textbf{1.8} & \natline{భవాన్భీష్మశ్చ కర్ణశ్చ} \\
 & \natline{కృపశ్చ సమితిఞ్జయః |} \\
 & \natline{అశ్వత్థామా వికర్ణశ్చ} \\
 & \natline{సౌమదత్తిస్తథైవ చ ||}
\end{tabular}
\end{table}

\begin{table}[H]
\begin{tabular}{cl}
\textbf{1.9} & \natline{అన్యే చ బహవః శూరాః} \\
 & \natline{మదర్థే త్యక్తజీవితాః |} \\
 & \natline{నానాశస్త్రప్రహరణాః} \\
 & \natline{సర్వే యుద్ధవిశారదాః ||}
\end{tabular}
\end{table}

\begin{table}[H]
\begin{tabular}{cl}
\textbf{1.10} & \natline{అపర్యాప్తం తదస్మాకం} \\
 & \natline{బలం భీష్మాభిరక్షితమ్ |} \\
 & \natline{పర్యాప్తం త్విదమేతేషాం} \\
 & \natline{బలం భీమాభిరక్షితమ్ ||}
\end{tabular}
\end{table}

\begin{table}[H]
\begin{tabular}{cl}
\textbf{1.11} & \natline{అయనేషు చ సర్వేషు} \\
 & \natline{యథాభాగమవస్థితాః |} \\
 & \natline{భీష్మమేవాభిరక్షన్తు} \\
 & \natline{భవన్తః సర్వ ఏవ హి ||}
\end{tabular}
\end{table}

\begin{table}[H]
\begin{tabular}{cl}
\textbf{1.12} & \natline{తస్య సఞ్జనయన్హర్షం} \\
 & \natline{కురువృద్ధః పితామహః |} \\
 & \natline{సింహనాదం వినద్యోచ్చైః} \\
 & \natline{శఙ్ఖం దధ్మౌ ప్రతాపవాన్ ||}
\end{tabular}
\end{table}

\begin{table}[H]
\begin{tabular}{cl}
\textbf{1.13} & \natline{తతః శఙ్ఖాశ్చ భేర్యశ్చ} \\
 & \natline{పణవానకగోముఖాః |} \\
 & \natline{సహసైవాభ్యహన్యన్త} \\
 & \natline{స శబ్దస్తుములోఽభవత్ ||}
\end{tabular}
\end{table}

\begin{table}[H]
\begin{tabular}{cl}
\textbf{1.14} & \natline{తతః శ్వేతైర్హయైర్యుక్తే} \\
 & \natline{మహతి స్యన్దనే స్థితౌ |} \\
 & \natline{మాధవః పాణ్డవశ్చైవ} \\
 & \natline{దివ్యౌ శఙ్ఖౌ ప్రదధ్మతుః ||}
\end{tabular}
\end{table}

\begin{table}[H]
\begin{tabular}{cl}
\textbf{1.15} & \natline{పాఞ్చజన్యం హృషీకేశః} \\
 & \natline{దేవదత్తం ధనఞ్జయః |} \\
 & \natline{పౌణ్డ్రం దధ్మౌ మహాశఙ్ఖం} \\
 & \natline{భీమకర్మా వృకోదరః ||}
\end{tabular}
\end{table}

\begin{table}[H]
\begin{tabular}{cl}
\textbf{1.16} & \natline{అనన్తవిజయం రాజా} \\
 & \natline{కున్తీపుత్రో యుధిష్ఠిరః |} \\
 & \natline{నకులః సహదేవశ్చ} \\
 & \natline{సుఘోషమణిపుష్పకౌ ||}
\end{tabular}
\end{table}

\begin{table}[H]
\begin{tabular}{cl}
\textbf{1.17} & \natline{కాశ్యశ్చ పరమేష్వాసః} \\
 & \natline{శిఖణ్డీ చ మహారథః |} \\
 & \natline{ధృష్టద్యుమ్నో విరాటశ్చ} \\
 & \natline{సాత్యకిశ్చాపరాజితః ||}
\end{tabular}
\end{table}

\begin{table}[H]
\begin{tabular}{cl}
\textbf{1.18} & \natline{ద్రుపదో ద్రౌపదేయాశ్చ} \\
 & \natline{సర్వశః పృథివీపతే |} \\
 & \natline{సోఉభద్రశ్చ మహాబాహుః} \\
 & \natline{శఙ్ఖాన్దధ్ముః పృథక్పృథక్ ||}
\end{tabular}
\end{table}

\begin{table}[H]
\begin{tabular}{cl}
\textbf{1.19} & \natline{స ఘోషో ధార్తరాష్ట్రాణాం} \\
 & \natline{హృదయాని వ్యదారయత్ |} \\
 & \natline{నభశ్చ పృథివీం చైవ} \\
 & \natline{తుములో వ్యనునాదయన్ ||}
\end{tabular}
\end{table}

\begin{table}[H]
\begin{tabular}{cl}
\textbf{1.20} & \natline{అథ వ్యవస్థితాన్దృష్ట్వా} \\
 & \natline{ధార్తరాష్ట్రాన్ కపిధ్వజః |} \\
 & \natline{ప్రవృత్తే శస్త్రసమ్పాతే} \\
 & \natline{ధనురుద్యమ్య పాణ్డవః ||}
\end{tabular}
\end{table}

\begin{table}[H]
\begin{tabular}{cl}
\textbf{1.21} & \natline{హృషీకేశం తదా వాక్యమ్} \\
 & \natline{ఇదమాహ మహీపతే} \\
 & \natline{అర్జున ఉవాచ} \\
 & \natline{సేనయోరుభయోర్మధ్యే} \\
 & \natline{రథం స్థాపయ మేఽచ్యుత}
\end{tabular}
\end{table}

\begin{table}[H]
\begin{tabular}{cl}
\textbf{1.22} & \natline{యావదేతాన్నిరీక్షేఽహం} \\
 & \natline{యోద్ధుకామానవస్థితాన్ |} \\
 & \natline{కైర్మయా సహ యోద్ధవ్యమ్} \\
 & \natline{అస్మిన్ రణసముద్యమే ||}
\end{tabular}
\end{table}

\begin{table}[H]
\begin{tabular}{cl}
\textbf{1.23} & \natline{యోత్స్యమానానవేక్షేఽహం} \\
 & \natline{య ఏతేఽత్ర సమాగతాః |} \\
 & \natline{ధార్తరాష్ట్రస్యదుర్బుద్ధేః} \\
 & \natline{యుద్ధే ప్రియచికీర్షవః ||}
\end{tabular}
\end{table}

\begin{table}[H]
\begin{tabular}{cl}
\textbf{1.24} & \natline{సఞ్జయ ఉవాచ} \\
 & \natline{ఏవముక్తో హృషీకేశః} \\
 & \natline{గుడాకేశేన భారత |} \\
 & \natline{సేనయోరుభయోర్మధ్యే} \\
 & \natline{స్థాపయిత్వా రథోత్తమమ్ ||}
\end{tabular}
\end{table}

\begin{table}[H]
\begin{tabular}{cl}
\textbf{1.25} & \natline{భీష్మద్రోణప్రముఖతః} \\
 & \natline{సర్వేషాం చ మహీక్షితామ్ |} \\
 & \natline{ఉవాచ పార్థ పశ్యైతాన్} \\
 & \natline{సమవేతాన్కురూనితి ||}
\end{tabular}
\end{table}

\begin{table}[H]
\begin{tabular}{cl}
\textbf{1.26} & \natline{తత్రాపశ్యత్స్థితాన్పార్థః} \\
 & \natline{పితౄనథ పితామహాన్ |} \\
 & \natline{ఆచార్యాన్మాతులాన్భ్రాతౄన్} \\
 & \natline{పుత్రాన్పౌత్రాన్సఖీంస్తథా ||}
\end{tabular}
\end{table}

\begin{table}[H]
\begin{tabular}{cl}
\textbf{1.27} & \natline{శ్వశురాన్సుహృదశ్చైవ} \\
 & \natline{సేనయోరుభయోరపి |} \\
 & \natline{తాన్సమీక్ష్య స కౌన్తేయః} \\
 & \natline{సర్వాన్బన్ధూనవస్థితాన్ ||}
\end{tabular}
\end{table}

\begin{table}[H]
\begin{tabular}{cl}
\textbf{1.28} & \natline{కృపయా పరయాఽఽవిష్టః} \\
 & \natline{విషీదన్నిదమబ్రవీత్} \\
 & \natline{అర్జున ఉవాచ |} \\
 & \natline{దృష్ట్వేమం స్వజనం కృష్ణ} \\
 & \natline{యుయుత్సుం సముపస్థితమ్ ||}
\end{tabular}
\end{table}

\begin{table}[H]
\begin{tabular}{cl}
\textbf{1.29} & \natline{సీదన్తి మమ గాత్రాణి} \\
 & \natline{ముఖం చ పరిశుష్యతి |} \\
 & \natline{వేపథుశ్చ శరీరే మే} \\
 & \natline{రోమహర్షశ్చ జాయతే ||}
\end{tabular}
\end{table}

\begin{table}[H]
\begin{tabular}{cl}
\textbf{1.30} & \natline{గాణ్డీవం స్రంసతే హస్తాత్} \\
 & \natline{త్వక్చైవ పరిదహ్యతే |} \\
 & \natline{న చ శక్నోమ్యవస్థాతుం} \\
 & \natline{భ్రమతీవ చ మే మనః ||}
\end{tabular}
\end{table}

\begin{table}[H]
\begin{tabular}{cl}
\textbf{1.31} & \natline{నిమిత్తాని చ పశ్యామి} \\
 & \natline{విపరీతాని కేశవ |} \\
 & \natline{న చ శ్రేయోఽనుపశ్యామి} \\
 & \natline{హత్వా స్వజనమాహవే ||}
\end{tabular}
\end{table}

\begin{table}[H]
\begin{tabular}{cl}
\textbf{1.32} & \natline{న కాఙ్క్షే విజయం కృష్ణ} \\
 & \natline{న చ రాజ్యం సుఖాని చ |} \\
 & \natline{కిం నో రాజ్యేన గోవిన్ద} \\
 & \natline{కిం భోగైర్జీవితేన వా ||}
\end{tabular}
\end{table}

\begin{table}[H]
\begin{tabular}{cl}
\textbf{1.33} & \natline{యేషామర్థే కాఙ్క్షితం నః} \\
 & \natline{రాజ్యం భోగాః సుఖాని చ |} \\
 & \natline{త ఇమేఽవస్థితా యుద్ధే} \\
 & \natline{ప్రాణాంస్త్యక్త్వా ధనాని చ ||}
\end{tabular}
\end{table}

\begin{table}[H]
\begin{tabular}{cl}
\textbf{1.34} & \natline{ఆచార్యాః పితరః పుత్రాః} \\
 & \natline{తథైవ చ పితామహాః |} \\
 & \natline{మాతులాః శ్వశురాః పౌత్రాః} \\
 & \natline{శ్యాలాః సమ్బన్ధినస్తథా ||}
\end{tabular}
\end{table}

\begin{table}[H]
\begin{tabular}{cl}
\textbf{1.35} & \natline{ఏతాన్న హన్తుమిచ్చామి} \\
 & \natline{ఘ్నతోఽపి మధుసూదన |} \\
 & \natline{అపి త్రైలోక్యరాజ్యస్య} \\
 & \natline{హేతోః కిం ను మహీకృతే ||}
\end{tabular}
\end{table}

\begin{table}[H]
\begin{tabular}{cl}
\textbf{1.36} & \natline{నిహత్య ధార్తరాష్ట్రాన్నః} \\
 & \natline{కా ప్రీతిః స్యాజ్జనార్దన |} \\
 & \natline{పాపమేవాశ్రయేదస్మాన్} \\
 & \natline{హత్వైతానాతతాయినః ||}
\end{tabular}
\end{table}

\begin{table}[H]
\begin{tabular}{cl}
\textbf{1.37} & \natline{తస్మాన్నార్హా వయం హన్తుం} \\
 & \natline{ధార్తరాష్ట్రాన్స్వబాన్ధవాన్ |} \\
 & \natline{స్వజనం హి కథం హత్వా} \\
 & \natline{సుఖినః స్యామ మాధవ ||}
\end{tabular}
\end{table}

\begin{table}[H]
\begin{tabular}{cl}
\textbf{1.38} & \natline{యద్యప్యేతే న పశ్యన్తి} \\
 & \natline{లోభోపహతచేతసః |} \\
 & \natline{కులక్షయకృతం దోషం} \\
 & \natline{మిత్రద్రోహే చ పాతకమ్ ||}
\end{tabular}
\end{table}

\begin{table}[H]
\begin{tabular}{cl}
\textbf{1.39} & \natline{కథం న జ్ఞేయమస్మాభిః} \\
 & \natline{పాపాదస్మాన్నివర్తితుమ్ |} \\
 & \natline{కులక్షయకృతం దోషం} \\
 & \natline{ప్రపశ్యద్భిర్జనార్దన ||}
\end{tabular}
\end{table}

\begin{table}[H]
\begin{tabular}{cl}
\textbf{1.40} & \natline{కులక్షయే ప్రణశ్యన్తి} \\
 & \natline{కులధర్మాః సనాతనాః |} \\
 & \natline{ధర్మే నష్టే కులం కృత్స్నమ్} \\
 & \natline{అధర్మోఽభిభవత్యుత ||}
\end{tabular}
\end{table}

\begin{table}[H]
\begin{tabular}{cl}
\textbf{1.41} & \natline{అధర్మాభిభవాత్కృష్ణ} \\
 & \natline{ప్రదుష్యన్తి కులస్త్రియః |} \\
 & \natline{స్త్రీషు దుష్టాసు వార్ష్ణేయ} \\
 & \natline{జాయతే వర్ణసఙ్కరః ||}
\end{tabular}
\end{table}

\begin{table}[H]
\begin{tabular}{cl}
\textbf{1.42} & \natline{సఙ్కరో నరకాయైవ} \\
 & \natline{కులఘ్నానాం కులస్య చ |} \\
 & \natline{పతన్తి పితరో హ్యేషాం} \\
 & \natline{లుప్తపిణ్డోదకక్రియాః ||}
\end{tabular}
\end{table}

\begin{table}[H]
\begin{tabular}{cl}
\textbf{1.43} & \natline{దోషైరేతైః కులఘ్నానాం} \\
 & \natline{వర్ణసఙ్కరకారకైః |} \\
 & \natline{ఉత్సాద్యన్తే జాతిధర్మాః} \\
 & \natline{కులధర్మాశ్చ శాశ్వతాః ||}
\end{tabular}
\end{table}

\begin{table}[H]
\begin{tabular}{cl}
\textbf{1.44} & \natline{ఉత్సన్నకులధర్మాణాం} \\
 & \natline{మనుష్యాణాం జనార్దన |} \\
 & \natline{నరకేఽనియతం వాసః} \\
 & \natline{భవతీత్యనుశుశ్రుమ ||}
\end{tabular}
\end{table}

\begin{table}[H]
\begin{tabular}{cl}
\textbf{1.45} & \natline{అహో బత మహత్పాపం} \\
 & \natline{కర్తుం వ్యవసితా వయమ్ |} \\
 & \natline{యద్రాజ్యసుఖలోభేన} \\
 & \natline{హన్తుం స్వజనముద్యతాః ||}
\end{tabular}
\end{table}

\begin{table}[H]
\begin{tabular}{cl}
\textbf{1.46} & \natline{యది మామప్రతీకారమ్} \\
 & \natline{అశస్త్రం శస్త్రపాణయః |} \\
 & \natline{ధార్తరాష్ట్రా రణే హన్యుః} \\
 & \natline{తన్మే క్షేమతరం భవేత్ ||}
\end{tabular}
\end{table}

\begin{table}[H]
\begin{tabular}{cl}
\textbf{1.47} & \natline{సఞ్జయ ఉవాచ} \\
 & \natline{ఏవముక్త్వాఽర్జునః సఙ్ఖ్యే} \\
 & \natline{రథోపస్థ ఉపావిశత్ |} \\
 & \natline{విసృజ్య సశరం చాపం} \\
 & \natline{శోకసంవిగ్నమానసః ||}
\end{tabular}
\end{table}

\begin{table}[H]
\begin{tabular}{cl}
 & \natline{శ్రీమద్భగవద్గీతాసు ఉపనిషత్సు} \\
 & \natline{బ్రహ్మవిద్యాయాం యోగశాస్త్రే} \\
 & \natline{శ్రీకృష్ణార్జున సంవాదే} \\
 & \natline{అర్జునవిషాదయోగోనామ} \\
 & \natline{ప్రథమోధ్యాయః}
\end{tabular}
\end{table}


% \end{multicols}

\chapter{Sāṅkhya Yoga}
% \begin{multicols}{2}
\subsection*{2.0}
\begin{table}[H]
\centering
\begin{tabular}{ll}
\natline{ओं श्री परमात्मने नमः} & \romline{oṃ śrī paramātmane namaḥ} \\
\natline{अथ द्वितीयोऽध्यायः} & \romline{atha dvitīyo'dhyāyaḥ} \\
\natline{साण्ख्ययोगः} & \romline{sāṅkhya-yogaḥ}
\end{tabular}
\end{table}

\subsection*{2.1}
\begin{table}[H]
\centering
\begin{tabular}{ll}
\natline{संजय उवाच} & \romline{saṃjaya uvāca} \\
\natline{तं तथा कृपयाविष्टम्} & \romline{taṃ tathā kṛpayāviṣṭam} \\
\natline{अश्रुपूर्णाकुलेक्षणम्} & \romline{aśru-pūrṇākulekṣaṇam} \\
\natline{विषीदंतमिदं वाक्यम्} & \romline{viṣīdaṃtamidaṃ vākyam} \\
\natline{उवाच मधुसूदनः} & \romline{uvāca madhusūdanaḥ}
\end{tabular}
\end{table}

\subsection*{2.2}
\begin{table}[H]
\centering
\begin{tabular}{ll}
\natline{श्री भगवानुवाच} & \romline{śrī bhagavān-uvāca} \\
\natline{कुतस्त्वा कश्मलमिदं} & \romline{kutastvā kaśmalamidaṃ} \\
\natline{विषमे समुपस्थितम्} & \romline{viṣame samupasthitam} \\
\natline{अनार्यजुष्टमस्वर्ग्यम्} & \romline{anārya-juṣṭamasvargyam} \\
\natline{अकीर्तिकरमर्जुन} & \romline{akīrti-karam-arjuna}
\end{tabular}
\end{table}

\subsection*{2.3}
\begin{table}[H]
\centering
\begin{tabular}{ll}
\natline{क्लैब्यं मा स्म गमः पार्थ} & \romline{klaibyaṃ mā sma gamaḥ pārtha} \\
\natline{नैतत्त्वय्युपपद्यते} & \romline{naitat-tvayyupapadyate} \\
\natline{क्षुद्रं हृदयदौर्बल्यं} & \romline{kṣudraṃ hṛdaya-daurbalyaṃ} \\
\natline{त्यक्त्वोत्तिष्ठ परंतप} & \romline{tyaktvottiṣṭha paraṃtapa}
\end{tabular}
\end{table}

\subsection*{2.4}
\begin{table}[H]
\centering
\begin{tabular}{ll}
\natline{अर्जुन उवाच} & \romline{arjuna uvāca} \\
\natline{कथं भीश्ममहं संख्ये} & \romline{kathaṃ bhīśmamahaṃ saṃkhye} \\
\natline{द्रोणं च मधुसूदन} & \romline{droṇaṃ ca madhusūdana} \\
\natline{इशुभिः प्रतियोत्स्यामि} & \romline{iśubhiḥ pratiyotsyāmi} \\
\natline{पूजार्हावरिसूदन} & \romline{pūjārhāvarisūdana}
\end{tabular}
\end{table}

\subsection*{2.5}
\begin{table}[H]
\centering
\begin{tabular}{ll}
\natline{गुरूनहत्वा हि महानुभावान्} & \romline{gurūnahatvā hi mahānubhāvān} \\
\natline{श्रेयो भोक्तुं भैक्ष्यमपीह लोके} & \romline{śreyo bhoktuṃ bhaikṣyamapīha loke} \\
\natline{हत्वार्थकामंस्तु गुरूनिहैव} & \romline{hatvārtha-kāmaṃstu gurūnihaiva} \\
\natline{भुंजीय भोगान् रुधिरप्रदिग्धान्} & \romline{bhuṃjīya bhogān rudhira-pradigdhān}
\end{tabular}
\end{table}

\subsection*{2.6}
\begin{table}[H]
\centering
\begin{tabular}{ll}
\natline{न चैतद्विद्मः कतरन्नो गरीयः} & \romline{na caitadvidmaḥ kataranno garīyaḥ} \\
\natline{यद्वा जयेम यदि वा नो जयेयुः} & \romline{yadvā jayema yadi vā no jayeyuḥ} \\
\natline{यानेव हत्वा न जिजीविषामः} & \romline{yāneva hatvā na jijīviṣāmaḥ} \\
\natline{तेऽवस्थिताः प्रमुखे धार्तराष्ट्राः} & \romline{te'vasthitāḥ pramukhe dhārtarāṣṭrāḥ}
\end{tabular}
\end{table}

\subsection*{2.7}
\begin{table}[H]
\centering
\begin{tabular}{ll}
\natline{कार्पन्यदोषोपहतस्वभावः} & \romline{kārpanya-doṣopahata-svabhāvaḥ} \\
\natline{पृच्छामि त्वां धर्मसम्मूढचेताः} & \romline{pṛcchāmi tvāṃ dharma-sammūḍha-cetāḥ} \\
\natline{यच्छ्रेयः स्यान्निश्चितं ब्रूहि तन्मे} & \romline{yacchreyaḥ syānniścitaṃ brūhi tanme} \\
\natline{शिष्यस्तेऽहं शाधि मां त्वां प्रपन्नम्} & \romline{śiṣyaste'haṃ śādhi māṃ tvāṃ prapannam}
\end{tabular}
\end{table}

\subsection*{2.8}
\begin{table}[H]
\centering
\begin{tabular}{ll}
\natline{न हिप्रपश्यामि ममापनुद्याद्} & \romline{na hi\textasciitilde{}prapaśyāmi mamāpanudyād} \\
\natline{यच्छोकमुच्छोषणमिन्द्रियाणाम्} & \romline{yacchokam-ucchoṣaṇam-indriyāṇām} \\
\natline{अवाप्य भूमावसपत्नमृद्धं} & \romline{avāpya bhūmāv-asapatnamṛddhaṃ} \\
\natline{राज्यं सुराणामपि चाधिपत्यम्} & \romline{rājyaṃ surāṇāmapi cādhipatyam}
\end{tabular}
\end{table}

\subsection*{2.9}
\begin{table}[H]
\centering
\begin{tabular}{ll}
\natline{संजय उवाच} & \romline{saṃjaya uvāca} \\
\natline{एवमुक्त्वा हृषीकेशं} & \romline{evam-uktvā hṛṣīkeśaṃ} \\
\natline{गुडाकेशः परन्तपः} & \romline{guḍākeśaḥ parantapaḥ} \\
\natline{न योत्स्य इति गोविंदम्} & \romline{na yotsya iti goviṃdam} \\
\natline{उक्त्वा तूष्णीम् बभूव ह} & \romline{uktvā tūṣṇīm babhūva ha}
\end{tabular}
\end{table}

\subsection*{2.10}
\begin{table}[H]
\centering
\begin{tabular}{ll}
\natline{तमुवाच हृषीकेशः} & \romline{tam-uvāca hṛṣīkeśaḥ} \\
\natline{प्रहसन्निव भारत} & \romline{prahasanniva bhārata} \\
\natline{सेनयोरुभयोर्मध्ये} & \romline{senayorubhayor-madhye} \\
\natline{विषीदंतमिदं वचः} & \romline{viṣīdaṃtam-idaṃ vacaḥ}
\end{tabular}
\end{table}

\subsection*{2.11}
\begin{table}[H]
\centering
\begin{tabular}{ll}
\natline{श्री भगवानुवाच} & \romline{śrī bhagavān-uvāca} \\
\natline{अशोच्यानन्वशोचस्त्वं} & \romline{aśocyān-anvaśocas-tvaṃ} \\
\natline{प्रज्ञावादांश्च भाषसे} & \romline{prajñā-vādāṃśca bhāṣase} \\
\natline{गतासूनगतासूंश्च} & \romline{gatāsūn-agatāsūṃś-ca} \\
\natline{नानुशोचन्ति पण्डिताः} & \romline{nānuśocanti paṇḍitāḥ}
\end{tabular}
\end{table}

\subsection*{2.12}
\begin{table}[H]
\centering
\begin{tabular}{ll}
\natline{न त्वेवाहं जातु नासं} & \romline{na tvevāhaṃ jātu nāsaṃ} \\
\natline{न त्वं नेमे जनाधिपाः} & \romline{na tvaṃ neme janādhipāḥ} \\
\natline{न चैव न भविष्यामः} & \romline{na caiva na bhaviṣyāmaḥ} \\
\natline{सर्वे वयमतः परम्} & \romline{sarve vayamataḥ param}
\end{tabular}
\end{table}

\subsection*{2.13}
\begin{table}[H]
\centering
\begin{tabular}{ll}
\natline{देहिनोऽस्मिन् यथा देहे} & \romline{dehino'smin yathā dehe} \\
\natline{कौमारं यौवनं जरा} & \romline{kaumāraṃ yauvanaṃ jarā} \\
\natline{तथा देहांतरप्राप्तिः} & \romline{tathā dehāṃtara-prāptiḥ} \\
\natline{धीरस्तत्र न मुह्यति} & \romline{dhīras-tatra na muhyati}
\end{tabular}
\end{table}

\subsection*{2.14}
\begin{table}[H]
\centering
\begin{tabular}{ll}
\natline{मात्रास्पर्शास्तु कौंतेय} & \romline{mātrā-sparśās-tu kauṃteya} \\
\natline{शीतोष्णसुखदुःखदाः} & \romline{śītoṣṇa-sukha-duḥkha-dāḥ} \\
\natline{आगमापायिनोऽनित्याः} & \romline{āgamāpāyino'nityāḥ} \\
\natline{तांस्तितिक्षस्व भारत} & \romline{tāṃs-titikṣasva bhārata}
\end{tabular}
\end{table}

\subsection*{2.15}
\begin{table}[H]
\centering
\begin{tabular}{ll}
\natline{यं हि न व्यथयंत्येते} & \romline{yaṃ hi na vyathayaṃtyete} \\
\natline{पुरुषं पुरुषर्षभ} & \romline{puruṣaṃ puruṣarṣabha} \\
\natline{समदुःखसुखं धीरं} & \romline{sama-duḥkha-sukhaṃ dhīraṃ} \\
\natline{सोऽमृतत्वाय कल्पते} & \romline{so'mṛtatvāya kalpate}
\end{tabular}
\end{table}

\subsection*{2.16}
\begin{table}[H]
\centering
\begin{tabular}{ll}
\natline{नासतो विद्यते भावः} & \romline{nāsato vidyate bhāvaḥ} \\
\natline{नाभावो विद्यते सतः} & \romline{nābhāvo vidyate sataḥ} \\
\natline{उभयोरपि दृष्तोऽन्तः} & \romline{ubhayorapi dṛṣto'ntaḥ} \\
\natline{त्वनयोस्तत्त्वदर्शिभिः} & \romline{tvanayos-tattva-darśibhiḥ}
\end{tabular}
\end{table}

\subsection*{2.17}
\begin{table}[H]
\centering
\begin{tabular}{ll}
\natline{अविनाशि तु तद्विद्धि} & \romline{avināśi tu tadviddhi} \\
\natline{येन सर्वमिदं ततम्} & \romline{yena sarvamidaṃ tatam} \\
\natline{विनाशमव्ययस्यास्य} & \romline{vināśam-avyayasyāsya} \\
\natline{न कश्चित्कर्तुमर्हति} & \romline{na kaścit-kartum-arhati}
\end{tabular}
\end{table}

\subsection*{2.18}
\begin{table}[H]
\centering
\begin{tabular}{ll}
\natline{अंतवन्त इमे देहाः} & \romline{aṃtavanta ime dehāḥ} \\
\natline{नित्यस्योक्ताः शरीरिणः} & \romline{nityasyoktāḥ śarīriṇaḥ} \\
\natline{अनाशिनोऽप्रमेयस्य} & \romline{anāśino'prameyasya} \\
\natline{तस्माद्युध्यस्व भारत} & \romline{tasmād-yudhyasva bhārata}
\end{tabular}
\end{table}

\subsection*{2.19}
\begin{table}[H]
\centering
\begin{tabular}{ll}
\natline{य एनं वेत्ति हन्तारं} & \romline{ya enaṃ vetti hantāraṃ} \\
\natline{यश्चैनं मन्यते हतं} & \romline{yaścainaṃ manyate hataṃ} \\
\natline{उभौ तौ न विजानीतः} & \romline{ubhau tau na vijānītaḥ} \\
\natline{नायं हन्ति न हन्यते} & \romline{nāyaṃ hanti na hanyate}
\end{tabular}
\end{table}

\subsection*{2.20}
\begin{table}[H]
\centering
\begin{tabular}{ll}
\natline{न जायते म्रियते वा कदाचित्} & \romline{na jāyate mriyate vā kadācit} \\
\natline{नायं भूत्वा भविता वा न भूयः} & \romline{nāyaṃ bhūtvā bhavitā vā na bhūyaḥ} \\
\natline{अजो नित्यः शाश्वतोऽयं पुराणः} & \romline{ajo nityaḥ śāśvato'yaṃ purāṇaḥ} \\
\natline{न हन्यते हन्यमाने शरीरे} & \romline{na hanyate hanyamāne śarīre}
\end{tabular}
\end{table}

\subsection*{2.21}
\begin{table}[H]
\centering
\begin{tabular}{ll}
\natline{वेदाविनाशिनं नित्यं} & \romline{vedāvināśinaṃ nityaṃ} \\
\natline{य एनमजमव्ययम्} & \romline{ya enamajam-avyayam} \\
\natline{कथं स पुरुषः पार्थ} & \romline{kathaṃ sa puruṣaḥ pārtha} \\
\natline{कं घातयति हंति कम्} & \romline{kaṃ ghātayati haṃti kam}
\end{tabular}
\end{table}

\subsection*{2.22}
\begin{table}[H]
\centering
\begin{tabular}{ll}
\natline{वासांसि जीर्णानि यथा विहाय} & \romline{vāsāṃsi jīrṇāni yathā vihāya} \\
\natline{नवानि गृह्णाति नरोऽपराणि} & \romline{navāni gṛhṇāti naro'parāṇi} \\
\natline{तथा शरीराणि विहाय जीर्णानि} & \romline{tathā śarīrāṇi vihāya jīrṇāni} \\
\natline{अन्यानि संयाति नवानि देही} & \romline{anyāni saṃyāti navāni dehī}
\end{tabular}
\end{table}

\subsection*{2.23}
\begin{table}[H]
\centering
\begin{tabular}{ll}
\natline{नैनं छिन्दन्ति शस्त्राणि} & \romline{nainaṃ chindanti śastrāṇi} \\
\natline{नैनं दहति पावकः} & \romline{nainaṃ dahati pāvakaḥ} \\
\natline{न चैनं क्लेदयन्त्यापः} & \romline{na cainaṃ kledayantyāpaḥ} \\
\natline{न शोषयति मारुतः} & \romline{na śoṣayati mārutaḥ}
\end{tabular}
\end{table}

\subsection*{2.24}
\begin{table}[H]
\centering
\begin{tabular}{ll}
\natline{अच्छेद्योऽयम् अदाह्योऽयम्} & \romline{acchedyo'yam adāhyo'yam} \\
\natline{अक्लेद्योऽशोष्य एव च} & \romline{akledyo'śoṣya eva ca} \\
\natline{नित्यः सर्वगतः स्थाणुः} & \romline{nityaḥ sarva-gataḥ sthāṇuḥ} \\
\natline{अचलोऽयं सनातनः} & \romline{acalo'yaṃ sanātanaḥ}
\end{tabular}
\end{table}

\subsection*{2.25}
\begin{table}[H]
\centering
\begin{tabular}{ll}
\natline{अव्यक्तोऽयम् अचिन्त्योऽयम्} & \romline{avyakto'yam acintyo'yam} \\
\natline{अविकार्योऽयमुच्यते} & \romline{avikāryo'yamucyate} \\
\natline{तस्मादेवं विदित्वैनं} & \romline{tasmādevaṃ viditvainaṃ} \\
\natline{नानुशोचितुमर्हसि} & \romline{nānuśocitumarhasi}
\end{tabular}
\end{table}

\subsection*{2.26}
\begin{table}[H]
\centering
\begin{tabular}{ll}
\natline{अथ चैनं नित्यजातं} & \romline{atha cainaṃ nityajātaṃ} \\
\natline{नित्यं वा मन्यसे मृतम्} & \romline{nityaṃ vā manyase mṛtam} \\
\natline{तथाऽपि त्वं महाबाहो} & \romline{tathā'pi tvaṃ mahābāho} \\
\natline{नैवं शोचितुमर्हसि} & \romline{naivaṃ śocitumarhasi}
\end{tabular}
\end{table}

\subsection*{2.27}
\begin{table}[H]
\centering
\begin{tabular}{ll}
\natline{जातस्य हिध्रुवो मृत्युः} & \romline{jātasya hi\textasciitilde{}dhruvo mṛtyuḥ} \\
\natline{ध्रुवं जन्म मृतस्य च} & \romline{dhruvaṃ janma mṛtasya ca} \\
\natline{तस्मादपरिहार्येऽर्थे} & \romline{tasmādaparihārye'rthe} \\
\natline{न त्वं शोचितुमर्हसि} & \romline{na tvaṃ śocitumarhasi}
\end{tabular}
\end{table}

\subsection*{2.28}
\begin{table}[H]
\centering
\begin{tabular}{ll}
\natline{अव्यक्तादीनि भूतानि} & \romline{avyaktādīni bhūtāni} \\
\natline{व्यक्तमध्यानि भारत} & \romline{vyaktamadhyāni bhārata} \\
\natline{अव्यक्तनिधनान्येव} & \romline{avyakta-nidhanānyeva} \\
\natline{तत्र का परिदेवना} & \romline{tatra kā paridevanā}
\end{tabular}
\end{table}

\subsection*{2.29}
\begin{table}[H]
\centering
\begin{tabular}{ll}
\natline{आश्चर्यवत् पश्यति कश्चिदेनम्} & \romline{āścaryavat paśyati kaścidenam} \\
\natline{आश्चर्यवद् वदति तथैव cआन्यः} & \romline{āścaryavad vadati tathaiva cānyaḥ} \\
\natline{आश्चर्यवच्चैनमन्यः शृणोति} & \romline{āścaryavaccainamanyaḥ śṛṇoti} \\
\natline{श्रुत्वाप्येनं वेद न चैव कश्चित्} & \romline{śrutvāpyenaṃ veda na caiva kaścit}
\end{tabular}
\end{table}

\subsection*{2.30}
\begin{table}[H]
\centering
\begin{tabular}{ll}
\natline{देही नित्यमवध्योऽयं} & \romline{dehī nityamavadhyo'yaṃ} \\
\natline{देहे सर्वस्य भारत} & \romline{dehe sarvasya bhārata} \\
\natline{तस्मात्सर्वाणि भूतानि} & \romline{tasmātsarvāṇi bhūtāni} \\
\natline{न त्वं शोचितुमर्हसि} & \romline{na tvaṃ śocitumarhasi}
\end{tabular}
\end{table}


% \end{multicols}

\chapter{Karma Yoga}
% \begin{multicols}{2}
\begin{table}[H]
\begin{tabular}{cl}
\textbf{3.0} & \natline{ఓం శ్రీ పరమాత్మనే నమః} \\
 & \natline{అథ తృతీయోఽధ్యాయః} \\
 & \natline{కర్మయోగః}
\end{tabular}
\end{table}

\begin{table}[H]
\begin{tabular}{cl}
\textbf{3.1} & \natline{అర్జున ఉవాచ} \\
 & \natline{జ్యాయసీ చేత్కర్మణస్తే} \\
 & \natline{మతా బుద్ధిర్జనార్దన |} \\
 & \natline{తత్కిం కర్మణి ఘోరే మాం} \\
 & \natline{నియోజయసి కేశవ ||}
\end{tabular}
\end{table}

\begin{table}[H]
\begin{tabular}{cl}
\textbf{3.2} & \natline{వ్యామిశ్రేణేవ వాక్యేన} \\
 & \natline{బుద్ధిం మోహయసీవ మే |} \\
 & \natline{తదేకం వద నిశ్చిత్య} \\
 & \natline{యేన శ్రేయోఽహమాప్నుయామ్ ||}
\end{tabular}
\end{table}

\begin{table}[H]
\begin{tabular}{cl}
\textbf{3.3} & \natline{స్రీ భగవానువాచ} \\
 & \natline{లోకేఽస్మిన్ద్వివిధా నిష్ఠా} \\
 & \natline{పురా ప్రోక్తా మయాఽనఘ |} \\
 & \natline{జ్ఞానయోగేన సాఙ్ఖ్యానాం} \\
 & \natline{కర్మయోగేన యోగినామ్ ||}
\end{tabular}
\end{table}

\begin{table}[H]
\begin{tabular}{cl}
\textbf{3.4} & \natline{న కర్మణామనారమ్భాత్} \\
 & \natline{నైష్కర్మ్యం పురుషోఽశ్నుతే |} \\
 & \natline{న చ సన్న్యసనాదేవ} \\
 & \natline{సిద్ధిం సమధిగచ్ఛతి ||}
\end{tabular}
\end{table}

\begin{table}[H]
\begin{tabular}{cl}
\textbf{3.5} & \natline{న హి కశ్చిత్క్షణమపి} \\
 & \natline{జాతు తిష్ఠత్యకర్మకృత్ |} \\
 & \natline{కార్యతే హ్యవశః కర్మ} \\
 & \natline{సర్వః ప్రకృతిజైర్గుణైః ||}
\end{tabular}
\end{table}

\begin{table}[H]
\begin{tabular}{cl}
\textbf{3.6} & \natline{కర్మేన్ద్రియాణి సంయమ్య} \\
 & \natline{య ఆస్తే మనసా స్మరన్ |} \\
 & \natline{ఇన్ద్రియార్థాన్విమూఢాత్మా} \\
 & \natline{మిథ్యాచారః స ఉచ్యతే ||}
\end{tabular}
\end{table}

\begin{table}[H]
\begin{tabular}{cl}
\textbf{3.7} & \natline{యస్త్విన్ద్రియాణి మనసా} \\
 & \natline{నియమ్యారభతేఽర్జున |} \\
 & \natline{కర్మేన్ద్రియైః కర్మయోగమ్} \\
 & \natline{అసక్తః స విశిష్యతే ||}
\end{tabular}
\end{table}

\begin{table}[H]
\begin{tabular}{cl}
\textbf{3.8} & \natline{నియతం కురు కర్మ త్వం} \\
 & \natline{కర్మ జ్యాయో హ్యకర్మణః |} \\
 & \natline{శరీరయాత్రాఽపి చ తే} \\
 & \natline{న ప్రసిద్ధ్యేదకర్మణః ||}
\end{tabular}
\end{table}

\begin{table}[H]
\begin{tabular}{cl}
\textbf{3.9} & \natline{యజ్ఞార్థాత్కర్మణోఽన్యత్ర} \\
 & \natline{లోకోఽయం కర్మబన్ధనః |} \\
 & \natline{తదర్థం కర్మ కౌన్తేయ} \\
 & \natline{ముక్తసఙ్గః సమాచర ||}
\end{tabular}
\end{table}

\begin{table}[H]
\begin{tabular}{cl}
\textbf{3.10} & \natline{సహయజ్ఞాః ప్రజాః సృష్ట్వా} \\
 & \natline{పురోవాచ ప్రజాపతిః |} \\
 & \natline{అనేన ప్రసవిష్యధ్వం} \\
 & \natline{ఏష వోఽస్త్విష్టకామధుక్ ||}
\end{tabular}
\end{table}

\begin{table}[H]
\begin{tabular}{cl}
\textbf{3.11} & \natline{దేవాన్భావయతాఽనేన} \\
 & \natline{తే దేవా భావయన్తు వః |} \\
 & \natline{పరస్పరం భావయన్తః} \\
 & \natline{శ్రేయః పరమవాప్స్యథ ||}
\end{tabular}
\end{table}

\begin{table}[H]
\begin{tabular}{cl}
\textbf{3.12} & \natline{ఇష్టాన్భోగాన్హి వో దేవాః} \\
 & \natline{దాస్యన్తే యజ్ఞభావితాః |} \\
 & \natline{తైర్దత్తానప్రదాయైభ్యః} \\
 & \natline{యో భుఙ్క్తే స్తేన ఏవ సః ||}
\end{tabular}
\end{table}

\begin{table}[H]
\begin{tabular}{cl}
\textbf{3.13} & \natline{యజ్ఞశిష్టాశినః సన్తః} \\
 & \natline{ముచ్యన్తే సర్వకిల్బిషైః |} \\
 & \natline{భుఞ్జతే తే త్వఘం పాపాః} \\
 & \natline{యే పచన్త్యాత్మకారణాత్ ||}
\end{tabular}
\end{table}

\begin{table}[H]
\begin{tabular}{cl}
\textbf{3.14} & \natline{అన్నాద్భవన్తి భూతాని} \\
 & \natline{పర్జన్యాదన్నసమ్భవః |} \\
 & \natline{యజ్ఞాద్భవతి పర్జన్యః} \\
 & \natline{యజ్ఞః కర్మసముద్భవః ||}
\end{tabular}
\end{table}

\begin{table}[H]
\begin{tabular}{cl}
\textbf{3.15} & \natline{కర్మ బ్రహ్మోద్భవం విద్ధి} \\
 & \natline{బ్రహ్మాక్షరసముద్భవమ్ |} \\
 & \natline{తస్మాత్సర్వగతం బ్రహ్మ} \\
 & \natline{నిత్యం యజ్ఞే ప్రతిష్ఠితమ్ ||}
\end{tabular}
\end{table}

\begin{table}[H]
\begin{tabular}{cl}
\textbf{3.16} & \natline{ఏవం ప్రవర్తితం చక్రం} \\
 & \natline{నానువర్తయతీహ యః |} \\
 & \natline{అఘాయురిన్ద్రియారామః} \\
 & \natline{మోఘం పార్థ స జీవతి ||}
\end{tabular}
\end{table}

\begin{table}[H]
\begin{tabular}{cl}
\textbf{3.17} & \natline{యస్త్వాత్మరతిరేవ స్యాత్} \\
 & \natline{ఆత్మతృప్తశ్చ మానవః |} \\
 & \natline{ఆత్మన్యేవ చ సన్తుష్టః} \\
 & \natline{తస్య కార్యం న విద్యతే ||}
\end{tabular}
\end{table}

\begin{table}[H]
\begin{tabular}{cl}
\textbf{3.18} & \natline{నైవ తస్య కృతేనార్థః} \\
 & \natline{నాకృతేనేహ కశ్చన |} \\
 & \natline{న చాస్య సర్వభూతేషు} \\
 & \natline{కశ్చిదర్థవ్యపాశ్రయః ||}
\end{tabular}
\end{table}

\begin{table}[H]
\begin{tabular}{cl}
\textbf{3.19} & \natline{తస్మాదసక్తః సతతం} \\
 & \natline{కార్యం కర్మ సమాచర |} \\
 & \natline{అసక్తో హ్యాచరన్కర్మ} \\
 & \natline{పరమాప్నోతి పూరుషః ||}
\end{tabular}
\end{table}

\begin{table}[H]
\begin{tabular}{cl}
\textbf{3.20} & \natline{కర్మణైవ హి సంసిద్ధిం} \\
 & \natline{ఆస్థితా జనకాదయః |} \\
 & \natline{లోకసఙ్గ్రహమేవాపి} \\
 & \natline{సమ్పశ్యన్కర్తుమర్హసి ||}
\end{tabular}
\end{table}


% \end{multicols}

\chapter{Jñāna Yoga}
% \begin{multicols}{2}
\begin{table}[H]
\begin{tabular}{cl}
 & \romline{śrī paramātmane namaḥ} \\
 & \romline{atha caturtho'dhyāyaḥ} \\
 & \romline{jñāna-yogaḥ}
\end{tabular}
\end{table}

\begin{table}[H]
\begin{tabular}{cl}
\textbf{4.1} & \romline{śrī bhagavānuvāca} \\
 & \romline{imaṃ vivasvate yogaṃ} \\
 & \romline{proktavā-naha-mavyayam |} \\
 & \romline{vivasvān-manave prāha} \\
 & \romline{manurikṣvā-kave'bravīt ||}
\end{tabular}
\end{table}

\begin{table}[H]
\begin{tabular}{cl}
\textbf{4.2} & \romline{evaṃ paramparā-prāptam} \\
 & \romline{imaṃ rājarṣayo viduḥ |} \\
 & \romline{sa kāleneha mahatā} \\
 & \romline{yogo naṣṭaḥ parantapa ||}
\end{tabular}
\end{table}

\begin{table}[H]
\begin{tabular}{cl}
\textbf{4.3} & \romline{sa evāyaṃ mayā te'dya} \\
 & \romline{yogaḥ proktaḥ purātanaḥ |} \\
 & \romline{bhakto'si me sakhā ceti} \\
 & \romline{rahasyaṃ hyeta-duttamam ||}
\end{tabular}
\end{table}

\begin{table}[H]
\begin{tabular}{cl}
\textbf{4.4} & \romline{arjuna uvāca} \\
 & \romline{aparaṃ bhavato janma} \\
 & \romline{paraṃ janma vivasvataḥ |} \\
 & \romline{kathameta-dvijānīyāṃ} \\
 & \romline{tvamādau proktavāniti ||}
\end{tabular}
\end{table}

\begin{table}[H]
\begin{tabular}{cl}
\textbf{4.5} & \romline{śrī bhagavānuvāca} \\
 & \romline{bahūni me vyatītāni} \\
 & \romline{janmāni tava cārjuna |} \\
 & \romline{tānyahaṃ veda sarvāṇi} \\
 & \romline{na tvaṃ vettha parantapa ||}
\end{tabular}
\end{table}

\begin{table}[H]
\begin{tabular}{cl}
\textbf{4.6} & \romline{ajo'pi san-navya-yātmā} \\
 & \romline{bhūtānā-mīśvaro'pi san |} \\
 & \romline{prakṛtiṃ svāma-dhiṣṭhāya} \\
 & \romline{sambhavā-myāt-mamā-yayā ||}
\end{tabular}
\end{table}

\begin{table}[H]
\begin{tabular}{cl}
\textbf{4.7} & \romline{yadā yadā hi dharmasya} \\
 & \romline{glānir-bhavati bhārata |} \\
 & \romline{abhyutthāna-madharmasya} \\
 & \romline{tadā''tmānaṃ sṛjāmyaham ||}
\end{tabular}
\end{table}

\begin{table}[H]
\begin{tabular}{cl}
\textbf{4.8} & \romline{paritrāṇāya sādhūnāṃ} \\
 & \romline{vināśāya ca duṣkṛtām |} \\
 & \romline{dharma-saṃsthāpa-nārthāya} \\
 & \romline{sambhavāmi yuge yuge ||}
\end{tabular}
\end{table}

\begin{table}[H]
\begin{tabular}{cl}
\textbf{4.9} & \romline{janma karma ca me divyam} \\
 & \romline{evaṃ yo vetti tattvataḥ |} \\
 & \romline{tyaktvā dehaṃ punarjanma} \\
 & \romline{naiti māmeti so'rjuna ||}
\end{tabular}
\end{table}

\begin{table}[H]
\begin{tabular}{cl}
\textbf{4.10} & \romline{vītarāga-bhaya-krodhāḥ} \\
 & \romline{manmayā māmu-pāśritāḥ |} \\
 & \romline{bahavo jñāna-tapasā} \\
 & \romline{pūtā madbhāva-māgatāḥ ||}
\end{tabular}
\end{table}

\begin{table}[H]
\begin{tabular}{cl}
\textbf{4.11} & \romline{ye yathā māṃ prapadyante} \\
 & \romline{tāṃstathaiva bhajāmyaham |} \\
 & \romline{mama vartmānu-vartante} \\
 & \romline{manuṣyāḥ pārtha sarvaśaḥ ||}
\end{tabular}
\end{table}

\begin{table}[H]
\begin{tabular}{cl}
\textbf{4.12} & \romline{kāṅkṣantaḥ karmaṇāṃ siddhiṃ} \\
 & \romline{yajanta iha devatāḥ |} \\
 & \romline{kṣipraṃ hi mānuṣe loke} \\
 & \romline{siddhir-bhavati karmajā ||}
\end{tabular}
\end{table}

\begin{table}[H]
\begin{tabular}{cl}
\textbf{4.13} & \romline{cātur-varṇyaṃ mayā sṛṣṭaṃ} \\
 & \romline{guṇa-karma-vibhāgaśaḥ |} \\
 & \romline{tasya kartāra-mapi māṃ} \\
 & \romline{viddhya-kartāra-mavyayam ||}
\end{tabular}
\end{table}

\begin{table}[H]
\begin{tabular}{cl}
\textbf{4.14} & \romline{na māṃ karmāṇi limpanti} \\
 & \romline{na me karmaphale spṛhā |} \\
 & \romline{iti māṃ yo'bhijānāti} \\
 & \romline{karmabhirna sa badhyate ||}
\end{tabular}
\end{table}

\begin{table}[H]
\begin{tabular}{cl}
\textbf{4.15} & \romline{evaṃ jñātvā kṛtaṃ karma} \\
 & \romline{pūrvairapi mumukṣubhiḥ |} \\
 & \romline{kuru karmaiva tasmāttvaṃ} \\
 & \romline{pūrvaiḥ pūrva-taraṃ kṛtam ||}
\end{tabular}
\end{table}

\begin{table}[H]
\begin{tabular}{cl}
\textbf{4.16} & \romline{kiṃ karma kima-karmeti} \\
 & \romline{kavayo'pyatra mohitāḥ |} \\
 & \romline{tatte karma·pravakṣyāmi} \\
 & \romline{yaj-jñātvā mokṣyase'śubhāt ||}
\end{tabular}
\end{table}

\begin{table}[H]
\begin{tabular}{cl}
\textbf{4.17} & \romline{karmaṇo hyapi boddhavyaṃ} \\
 & \romline{boddhavyaṃ ca vikarmaṇaḥ |} \\
 & \romline{akarmaṇaśca boddhavyaṃ} \\
 & \romline{gahanā karmaṇo gatiḥ ||}
\end{tabular}
\end{table}

\begin{table}[H]
\begin{tabular}{cl}
\textbf{4.18} & \romline{karmaṇya-karma yaḥ paśyet} \\
 & \romline{akarmaṇi-ca karma yaḥ |} \\
 & \romline{sa buddhimān-manuṣyeṣu} \\
 & \romline{sa yuktaḥ kṛtsna-karmakṛt ||}
\end{tabular}
\end{table}

\begin{table}[H]
\begin{tabular}{cl}
\textbf{4.19} & \romline{yasya sarve samā-rambhāḥ} \\
 & \romline{kāma-saṅkalpa-varjitāḥ |} \\
 & \romline{jñānāgni-dagdha-karmāṇaṃ} \\
 & \romline{tamāhuḥ paṇḍitaṃ budhāḥ ||}
\end{tabular}
\end{table}

\begin{table}[H]
\begin{tabular}{cl}
\textbf{4.20} & \romline{tyaktvā karma-phalāsaṅgaṃ} \\
 & \romline{nitya-tṛpto nirāśrayaḥ |} \\
 & \romline{karmaṇyabhi-pravṛtto'pi} \\
 & \romline{naiva kiñcitkaroti saḥ ||}
\end{tabular}
\end{table}

\begin{table}[H]
\begin{tabular}{cl}
\textbf{4.21} & \romline{nirāśīr-yatacittātmā} \\
 & \romline{tyaktasarva parigrahaḥ |} \\
 & \romline{śārīraṃ kevalaṃ karma} \\
 & \romline{kurvannāpnoti kilbiṣam ||}
\end{tabular}
\end{table}

\begin{table}[H]
\begin{tabular}{cl}
\textbf{4.22} & \romline{yadṛcchālā-bhasantuṣṭaḥ} \\
 & \romline{dvandvātīto vimatsaraḥ |} \\
 & \romline{samaḥ siddhāva-siddhau ca} \\
 & \romline{kṛtvāpi na nibadhyate ||}
\end{tabular}
\end{table}

\begin{table}[H]
\begin{tabular}{cl}
\textbf{4.23} & \romline{gatasaṅgasya muktasya} \\
 & \romline{jñānā-vasthita-cetasaḥ |} \\
 & \romline{yajñā-yācarataḥ karma} \\
 & \romline{samagraṃ pravilīyate ||}
\end{tabular}
\end{table}

\begin{table}[H]
\begin{tabular}{cl}
\textbf{4.24} & \romline{brahmārpaṇaṃ brahma haviḥ} \\
 & \romline{brahmāgnau brahmaṇā hutam |} \\
 & \romline{brahmaiva tena gantavyaṃ} \\
 & \romline{brahma-karma-samādhinā ||}
\end{tabular}
\end{table}

\begin{table}[H]
\begin{tabular}{cl}
\textbf{4.25} & \romline{daiva-mevāpare yajñaṃ} \\
 & \romline{yoginaḥ paryupāsate |} \\
 & \romline{brahmāgnā-vapare yajñaṃ} \\
 & \romline{yajñe-naivopa-juhvati ||}
\end{tabular}
\end{table}

\begin{table}[H]
\begin{tabular}{cl}
\textbf{4.26} & \romline{śrotrā-dīnīndriyāṇ-yanye} \\
 & \romline{saṃya-māgniṣu juhvati |} \\
 & \romline{śabdādīn-viṣayān-anye} \\
 & \romline{indriyā-gniṣu juhvati ||}
\end{tabular}
\end{table}

\begin{table}[H]
\begin{tabular}{cl}
\textbf{4.27} & \romline{sarvāṇ-īndriya-karmāṇi} \\
 & \romline{prāṇa-karmāṇi cāpare |} \\
 & \romline{ātma-saṃyama-yogāgnau} \\
 & \romline{juhvati·jñāna-dīpite ||}
\end{tabular}
\end{table}

\begin{table}[H]
\begin{tabular}{cl}
\textbf{4.28} & \romline{dravya-yajñāstapo-yajñāḥ} \\
 & \romline{yoga-yajñās-tathā'pare |} \\
 & \romline{svādhyā-yajñā-nayajñāśca} \\
 & \romline{yatayaḥ saṃ-śita-vratāḥ ||}
\end{tabular}
\end{table}

\begin{table}[H]
\begin{tabular}{cl}
\textbf{4.29} & \romline{apāne juhvati·prāṇaṃ} \\
 & \romline{prāṇe'pānaṃ tathāpare |} \\
 & \romline{prāṇāpāna-gatī ruddhvā} \\
 & \romline{prāṇāyāma-parāyaṇāḥ ||}
\end{tabular}
\end{table}

\begin{table}[H]
\begin{tabular}{cl}
\textbf{4.30} & \romline{apare niyatāhārāḥ} \\
 & \romline{prāṇān-prāṇeṣu juhvati |} \\
 & \romline{sarve'pyete yajña-vidaḥ} \\
 & \romline{yajña-kṣapita-kalmaṣāḥ ||}
\end{tabular}
\end{table}

\begin{table}[H]
\begin{tabular}{cl}
\textbf{4.31} & \romline{yajña-śiṣṭāmṛta-bhujaḥ} \\
 & \romline{yānti·brahma sanātanam |} \\
 & \romline{nāyaṃ loko'stya-yajñasya} \\
 & \romline{kuto'nyaḥ kurusattama ||}
\end{tabular}
\end{table}

\begin{table}[H]
\begin{tabular}{cl}
\textbf{4.32} & \romline{evaṃ bahuvidhā yajñāḥ} \\
 & \romline{vitatā brahmaṇo mukhe |} \\
 & \romline{karma-jānviddhi tānsarvān} \\
 & \romline{evaṃ jñātvā vimokṣyase ||}
\end{tabular}
\end{table}

\begin{table}[H]
\begin{tabular}{cl}
\textbf{4.33} & \romline{śreyān-dravya-mayād-yajñāt} \\
 & \romline{jñāna-yajñaḥ parantapa |} \\
 & \romline{sarvaṃ karmākhilaṃ pārtha} \\
 & \romline{jñāne parisa-māpyate ||}
\end{tabular}
\end{table}

\begin{table}[H]
\begin{tabular}{cl}
\textbf{4.34} & \romline{tadviddhi·praṇipātena} \\
 & \romline{pari-praśnena sevayā |} \\
 & \romline{upadekṣyanti te·jñānaṃ} \\
 & \romline{jñānina-stattva-darśinaḥ ||}
\end{tabular}
\end{table}

\begin{table}[H]
\begin{tabular}{cl}
\textbf{4.35} & \romline{yaj-jñātvā na punarmoham} \\
 & \romline{evaṃ yāsyasi pāṇḍava |} \\
 & \romline{yena bhūtānya-śeṣeṇa} \\
 & \romline{drakṣyasyāt-manyatho mayi ||}
\end{tabular}
\end{table}

\begin{table}[H]
\begin{tabular}{cl}
\textbf{4.36} & \romline{api cedasi pāpebhyaḥ} \\
 & \romline{sarvebhyaḥ pāpakṛttamaḥ |} \\
 & \romline{sarvaṃ jñāna-plavenaiva} \\
 & \romline{vṛjinaṃ santariṣyasi ||}
\end{tabular}
\end{table}

\begin{table}[H]
\begin{tabular}{cl}
\textbf{4.37} & \romline{yathai-dhāṃsi samiddho'gniḥ} \\
 & \romline{bhasmasāt-kurute'rjuna |} \\
 & \romline{jñānāgniḥ sarva-karmāṇi} \\
 & \romline{bhasmasāt-kurute tathā ||}
\end{tabular}
\end{table}

\begin{table}[H]
\begin{tabular}{cl}
\textbf{4.38} & \romline{na hi·jñānena sadṛśaṃ} \\
 & \romline{pavitra-miha vidyate |} \\
 & \romline{tat-svayaṃ yoga-saṃsiddhaḥ} \\
 & \romline{kālenātmani vindati ||}
\end{tabular}
\end{table}

\begin{table}[H]
\begin{tabular}{cl}
\textbf{4.39} & \romline{śraddhāvān labhate jñānaṃ} \\
 & \romline{tatparaḥ saṃ-yatendriyaḥ |} \\
 & \romline{jñānaṃ labdhvā parāṃ śāntim} \\
 & \romline{acireṇādhi-gacchati ||}
\end{tabular}
\end{table}

\begin{table}[H]
\begin{tabular}{cl}
\textbf{4.40} & \romline{ajñaścā-śraddadhānaśca} \\
 & \romline{saṃśayātmā vinaśyati |} \\
 & \romline{nāyaṃ loko'sti na paraḥ} \\
 & \romline{na sukhaṃ saṃśayātmanaḥ ||}
\end{tabular}
\end{table}

\begin{table}[H]
\begin{tabular}{cl}
\textbf{4.41} & \romline{yoga-sannyasta-karmāṇaṃ} \\
 & \romline{jñāna-sañchinna-saṃśayam |} \\
 & \romline{ātma-vantaṃ na karmāṇi} \\
 & \romline{nibadhnanti dhanañjaya ||}
\end{tabular}
\end{table}

\begin{table}[H]
\begin{tabular}{cl}
\textbf{4.42} & \romline{tasmāda-jñāna-sambhūtaṃ} \\
 & \romline{hṛtsthaṃ jñānā-sinātmanaḥ |} \\
 & \romline{chittvainaṃ saṃśayaṃ yogam} \\
 & \romline{ātiṣṭhottiṣṭha bhārata ||}
\end{tabular}
\end{table}

\begin{table}[H]
\begin{tabular}{cl}
 & \romline{śrīmad-bhagavad-gītāsu upaniṣatsu} \\
 & \romline{brahma-vidyāyāṃ yogaśāstre} \\
 & \romline{śrīkṛṣṇārjuna saṃvāde} \\
 & \romline{jñāna-yogonāma} \\
 & \romline{caturtho-dhyāyaḥ}
\end{tabular}
\end{table}


% \end{multicols}

\chapter{Karma-Sannyāsa Yoga}
% \begin{multicols}{2}
\begin{table}[H]
\begin{tabular}{cl}
 & \natline{श्री परमात्मने नमः} \\
 & \natline{अथ पञ्चमोऽध्यायः} \\
 & \natline{कर्मसन्न्यासयोगः}
\end{tabular}
\end{table}

\begin{table}[H]
\begin{tabular}{cl}
\textbf{5.1} & \natline{अर्जुन उवाच} \\
 & \natline{सन्न्यासं कर्मणां कृष्ण} \\
 & \natline{पुनर्योगं च शंससि |} \\
 & \natline{यच्छ्रेय एतयोरेकं} \\
 & \natline{तन्मे ब्रूहि सुनिश्चितम् ||}
\end{tabular}
\end{table}

\begin{table}[H]
\begin{tabular}{cl}
\textbf{5.2} & \natline{श्री भगवानुवाच} \\
 & \natline{सन्न्यासः कर्मयोगश्च} \\
 & \natline{निश्श्रेयसकरावुभौ |} \\
 & \natline{तयोस्तु कर्मसन्न्यासात्} \\
 & \natline{कर्मयोगो विशिष्यते ||}
\end{tabular}
\end{table}

\begin{table}[H]
\begin{tabular}{cl}
\textbf{5.3} & \natline{ज्ञेयः स नित्यसन्न्यासी} \\
 & \natline{यो न द्वेष्टि न काङ्क्षति |} \\
 & \natline{निर्द्वन्द्वो हि महाबाहो} \\
 & \natline{सुखं बन्धात्प्रमुच्यते ||}
\end{tabular}
\end{table}

\begin{table}[H]
\begin{tabular}{cl}
\textbf{5.4} & \natline{साङ्ख्ययोगौ पृथग्बालाः} \\
 & \natline{प्रवदन्ति न पण्डिताः |} \\
 & \natline{एकमप्यास्थितः सम्यक्} \\
 & \natline{उभयोर्विन्दते फलम् ||}
\end{tabular}
\end{table}

\begin{table}[H]
\begin{tabular}{cl}
\textbf{5.5} & \natline{यत्साङ्ख्यैः प्राप्यते स्थानं} \\
 & \natline{तद्योगैरपि गम्यते |} \\
 & \natline{एकं साङ्ख्यं च योगं च} \\
 & \natline{यः पश्यति स पश्यति ||}
\end{tabular}
\end{table}

\begin{table}[H]
\begin{tabular}{cl}
\textbf{5.6} & \natline{सन्न्यासस्तु महाबाहो} \\
 & \natline{दुःखमाप्तुमयोगतः |} \\
 & \natline{योगयुक्तो मुनिर्ब्रह्म} \\
 & \natline{नचिरेणाधिगच्छति ||}
\end{tabular}
\end{table}

\begin{table}[H]
\begin{tabular}{cl}
\textbf{5.7} & \natline{योगयुक्तो विशुद्धात्मा} \\
 & \natline{विजितात्मा जितेन्द्रियः |} \\
 & \natline{सर्वभूतात्मभूतात्मा} \\
 & \natline{कुर्वन्नपि न लिप्यते ||}
\end{tabular}
\end{table}

\begin{table}[H]
\begin{tabular}{cl}
\textbf{5.8} & \natline{नैव किञ्चित्करोमीति} \\
 & \natline{युक्तो मन्येत तत्त्ववित् |} \\
 & \natline{पश्यन्शृण्वन् स्पृशञ्जिघ्रन्} \\
 & \natline{अश्नन्गच्छन्स्वपन्श्वसन् ||}
\end{tabular}
\end{table}

\begin{table}[H]
\begin{tabular}{cl}
\textbf{5.9} & \natline{प्रलपन् विसृजन् गृह्णन्} \\
 & \natline{उन्मिषन्निमिषन्नपि |} \\
 & \natline{इन्द्रियाणीन्द्रियार्थेषु} \\
 & \natline{वर्तन्त इति धारयन् ||}
\end{tabular}
\end{table}

\begin{table}[H]
\begin{tabular}{cl}
\textbf{5.10} & \natline{ब्रह्मण्याधाय कर्माणि} \\
 & \natline{सङ्गं त्यक्त्वा करोति यः |} \\
 & \natline{लिप्यते न स पापेन} \\
 & \natline{पद्मपत्रमिवाम्भसा ||}
\end{tabular}
\end{table}

\begin{table}[H]
\begin{tabular}{cl}
\textbf{5.11} & \natline{कायेन मनसा बुद्ध्या} \\
 & \natline{केवलैरिन्द्रियैरपि |} \\
 & \natline{योगिनः कर्म कुर्वन्ति} \\
 & \natline{सङ्गं त्यक्त्वात्मशुद्धये ||}
\end{tabular}
\end{table}

\begin{table}[H]
\begin{tabular}{cl}
\textbf{5.12} & \natline{युक्तः कर्मफलं त्यक्त्वा} \\
 & \natline{शान्तिमाप्नोति नैष्ठिकीम् |} \\
 & \natline{अयुक्तः कामकारेण} \\
 & \natline{फले सक्तो निबध्यते ||}
\end{tabular}
\end{table}

\begin{table}[H]
\begin{tabular}{cl}
\textbf{5.13} & \natline{सर्वकर्माणि मनसा} \\
 & \natline{सन्न्यस्यास्ते सुखं वशी |} \\
 & \natline{नवद्वारे पुरे देही} \\
 & \natline{नैव कुर्वन्न कारयन् ||}
\end{tabular}
\end{table}

\begin{table}[H]
\begin{tabular}{cl}
\textbf{5.14} & \natline{न कर्तृत्वं न कर्माणि} \\
 & \natline{लोकस्य सृजति प्रभुः |} \\
 & \natline{न कर्मफलसंयोगं} \\
 & \natline{स्वभावस्तु प्रवर्तते ||}
\end{tabular}
\end{table}

\begin{table}[H]
\begin{tabular}{cl}
\textbf{5.15} & \natline{नादत्ते कस्यचित्पापं} \\
 & \natline{न चैव सुकृतं विभुः |} \\
 & \natline{अज्ञानेनावृतं ज्ञानं} \\
 & \natline{तेन मुह्यन्ति जन्तवः ||}
\end{tabular}
\end{table}

\begin{table}[H]
\begin{tabular}{cl}
\textbf{5.16} & \natline{ज्ञानेन तु तदज्ञानं} \\
 & \natline{येषां नाशितमात्मनः |} \\
 & \natline{तेषामादित्यवज्ज्ञानं} \\
 & \natline{प्रकाशयति तत्परम् ||}
\end{tabular}
\end{table}

\begin{table}[H]
\begin{tabular}{cl}
\textbf{5.17} & \natline{तद्बुद्धयस्तदात्मानः} \\
 & \natline{तन्निष्ठास्तत्परायणाः |} \\
 & \natline{गच्छन्त्यपुनरावृत्तिं} \\
 & \natline{ज्ञाननिर्धूतकल्मषाः ||}
\end{tabular}
\end{table}

\begin{table}[H]
\begin{tabular}{cl}
\textbf{5.18} & \natline{विद्याविनयसम्पन्ने} \\
 & \natline{ब्राह्मणे गवि हस्तिनि |} \\
 & \natline{शुनि चैव श्वपाके च} \\
 & \natline{पण्डिताः समदर्शिनः ||}
\end{tabular}
\end{table}

\begin{table}[H]
\begin{tabular}{cl}
\textbf{5.19} & \natline{इहैव तैर्जितः सर्गः} \\
 & \natline{येषां साम्ये स्थितं मनः |} \\
 & \natline{निर्दोषं हि समं ब्रह्म} \\
 & \natline{तस्मात् ब्रह्मणि ते स्थिताः ||}
\end{tabular}
\end{table}

\begin{table}[H]
\begin{tabular}{cl}
\textbf{5.20} & \natline{न प्रहृष्येत्प्रियं प्राप्य} \\
 & \natline{नोद्विजेत्प्राप्य चाप्रियम् |} \\
 & \natline{स्थिरबुद्धिरसम्मूढः} \\
 & \natline{ब्रह्मवित् ब्रह्मणि स्थितः ||}
\end{tabular}
\end{table}

\begin{table}[H]
\begin{tabular}{cl}
\textbf{5.21} & \natline{बाह्यस्पर्शेष्वसक्तात्मा} \\
 & \natline{विन्दत्यात्मनि यत् सुखम् |} \\
 & \natline{स ब्रह्मयोगयुक्तात्मा} \\
 & \natline{सुखमक्षयमश्नुते ||}
\end{tabular}
\end{table}

\begin{table}[H]
\begin{tabular}{cl}
\textbf{5.22} & \natline{ये हि संस्पर्शजा भोगाः} \\
 & \natline{दुःखयोनय एव ते |} \\
 & \natline{आद्यन्तवन्तः कौन्तेय} \\
 & \natline{न तेषु रमते बुधः ||}
\end{tabular}
\end{table}

\begin{table}[H]
\begin{tabular}{cl}
\textbf{5.23} & \natline{शक्नोतीहैव यः सोढुं} \\
 & \natline{प्राक् शरीरविमोक्षणात् |} \\
 & \natline{कामक्रोधोद्भवं वेगं} \\
 & \natline{स युक्तः स सुखी नरः ||}
\end{tabular}
\end{table}

\begin{table}[H]
\begin{tabular}{cl}
\textbf{5.24} & \natline{योऽन्तःसुखोऽन्तरारामः} \\
 & \natline{तथान्तर्ज्योतिरेव यः |} \\
 & \natline{स योगी ब्रह्मनिर्वाणं} \\
 & \natline{ब्रह्मभूतोऽधिगच्छति ||}
\end{tabular}
\end{table}

\begin{table}[H]
\begin{tabular}{cl}
\textbf{5.25} & \natline{लभन्ते ब्रह्मनिर्वाणम्} \\
 & \natline{ऋषयः क्षीणकल्मषाः |} \\
 & \natline{छिन्नद्वैधा यतात्मानः} \\
 & \natline{सर्वभूतहिते रताः ||}
\end{tabular}
\end{table}

\begin{table}[H]
\begin{tabular}{cl}
\textbf{5.26} & \natline{कामक्रोधवियुक्तानां} \\
 & \natline{यतीनां यतचेतसाम् |} \\
 & \natline{अभितो ब्रह्मनिर्वाणं} \\
 & \natline{वर्तते विदितात्मनाम् ||}
\end{tabular}
\end{table}

\begin{table}[H]
\begin{tabular}{cl}
\textbf{5.27} & \natline{स्पर्शान् कृत्वा बहिर्बाह्यान्} \\
 & \natline{चक्षुश्चैवान्तरे भ्रुवोः |} \\
 & \natline{प्राणापानौ समौ कृत्वा} \\
 & \natline{नासाभ्यन्तरचारिणौ ||}
\end{tabular}
\end{table}

\begin{table}[H]
\begin{tabular}{cl}
\textbf{5.28} & \natline{यतेन्द्रियमनोबुद्धिः} \\
 & \natline{मुनिर्मोक्षपरायणः |} \\
 & \natline{विगतेच्छाभयक्रोधः} \\
 & \natline{यः सदा मुक्त एव सः ||}
\end{tabular}
\end{table}

\begin{table}[H]
\begin{tabular}{cl}
\textbf{5.29} & \natline{भोक्तारं यज्ञतपसां} \\
 & \natline{सर्वलोकमहेश्वरम् |} \\
 & \natline{सुहृदं सर्वभूतानां} \\
 & \natline{ज्ञात्वा मां शान्तिमृच्छति। ||}
\end{tabular}
\end{table}

\begin{table}[H]
\begin{tabular}{cl}
 & \natline{श्रीमद्भगवद्गीतासु उपनिषत्सु} \\
 & \natline{ब्रह्मविद्यायां योगशास्त्रे} \\
 & \natline{श्रीकृष्णार्जुन संवादे} \\
 & \natline{कर्मसन्न्यासयोगो नाम} \\
 & \natline{पञ्चमोध्यायः}
\end{tabular}
\end{table}


% \end{multicols}

\chapter{Ātma-Saṃyama Yoga}
% \begin{multicols}{2}
\begin{table}[H]
\begin{tabular}{cl}
 & \romline{śrī paramātmane namaḥ} \\
 & \romline{atha ṣaṣṭho'dhyāyaḥ} \\
 & \romline{ātmasaṃyamayogaḥ}
\end{tabular}
\end{table}

\begin{table}[H]
\begin{tabular}{cl}
\textbf{6.1} & \romline{śrī bhagavānuvāca} \\
 & \romline{anāśritaḥ karmaphalaṃ} \\
 & \romline{kāryaṃ karma karoti yaḥ |} \\
 & \romline{sa sannyāsī ca yogī ca} \\
 & \romline{na niragnirna cākriyaḥ ||}
\end{tabular}
\end{table}

\begin{table}[H]
\begin{tabular}{cl}
\textbf{6.2} & \romline{yaṃ sannyāsamiti prāhuḥ} \\
 & \romline{yogaṃ taṃ viddhi pāṇḍava |} \\
 & \romline{na hyasannyastasaṅkalpaḥ} \\
 & \romline{yogī bhavati kaścana ||}
\end{tabular}
\end{table}

\begin{table}[H]
\begin{tabular}{cl}
\textbf{6.3} & \romline{ārurukṣormuneryogaṃ} \\
 & \romline{karma kāraṇamucyate |} \\
 & \romline{yogārūḍhasya tasyaiva} \\
 & \romline{śamaḥ kāraṇamucyate ||}
\end{tabular}
\end{table}

\begin{table}[H]
\begin{tabular}{cl}
\textbf{6.4} & \romline{yadā hi nendriyārtheṣu} \\
 & \romline{na karmasvanuṣajjate |} \\
 & \romline{sarvasaṅkalpasannyāsī} \\
 & \romline{yogārūḍhastadocyate ||}
\end{tabular}
\end{table}

\begin{table}[H]
\begin{tabular}{cl}
\textbf{6.5} & \romline{uddharedātmanā''tmānaṃ} \\
 & \romline{nātmānamavasādayet |} \\
 & \romline{ātmaiva hyātmano bandhuḥ} \\
 & \romline{ātmaiva ripurātmanaḥ ||}
\end{tabular}
\end{table}

\begin{table}[H]
\begin{tabular}{cl}
\textbf{6.6} & \romline{bandhurātmā''tmanastasya} \\
 & \romline{yenātmaivātmanā jitaḥ |} \\
 & \romline{anātmanastu śatrutve} \\
 & \romline{vartetātmaiva śatruvat ||}
\end{tabular}
\end{table}

\begin{table}[H]
\begin{tabular}{cl}
\textbf{6.7} & \romline{jitātmanaḥ praśāntasya} \\
 & \romline{paramātmā samāhitaḥ |} \\
 & \romline{śītoṣṇasukhaduḥkheṣu} \\
 & \romline{tathā mānāpamānayoḥ ||}
\end{tabular}
\end{table}

\begin{table}[H]
\begin{tabular}{cl}
\textbf{6.8} & \romline{jñānavijñānatṛptātmā} \\
 & \romline{kūṭastho vijitendriyaḥ |} \\
 & \romline{yukta ityucyate yogī} \\
 & \romline{samaloṣṭāśmakāñcanaḥ ||}
\end{tabular}
\end{table}

\begin{table}[H]
\begin{tabular}{cl}
\textbf{6.9} & \romline{suhṛnmitrāryudāsīna} \\
 & \romline{madhyasthadveṣyabandhuṣu |} \\
 & \romline{sādhuṣvapi ca pāpeṣu} \\
 & \romline{samabuddhirviśiṣyate ||}
\end{tabular}
\end{table}

\begin{table}[H]
\begin{tabular}{cl}
\textbf{6.10} & \romline{yogī yuñjīta satatam} \\
 & \romline{ātmānaṃ rahasi sthitaḥ |} \\
 & \romline{ekākī yatacittātmā} \\
 & \romline{nirāśīraparigrahaḥ ||}
\end{tabular}
\end{table}

\begin{table}[H]
\begin{tabular}{cl}
\textbf{6.11} & \romline{śucau deśe pratiṣṭhāpya} \\
 & \romline{sthiramāsanamātmanaḥ |} \\
 & \romline{nātyucchritaṃ nātinīcaṃ} \\
 & \romline{cailājinakuśottaram ||}
\end{tabular}
\end{table}

\begin{table}[H]
\begin{tabular}{cl}
\textbf{6.12} & \romline{tatraikāgraṃ manaḥ kṛtvā} \\
 & \romline{yatacittendriyakriyaḥ |} \\
 & \romline{upaviśyāsane yuñjyāt} \\
 & \romline{yogamātmaviśuddhaye ||}
\end{tabular}
\end{table}

\begin{table}[H]
\begin{tabular}{cl}
\textbf{6.13} & \romline{samaṃ kāyaśirogrīvaṃ} \\
 & \romline{dhārayannacalaṃ sthiraḥ |} \\
 & \romline{samprekṣya nāsikāgraṃ svaṃ} \\
 & \romline{diśaścānavalokayan ||}
\end{tabular}
\end{table}

\begin{table}[H]
\begin{tabular}{cl}
\textbf{6.14} & \romline{praśāntātmā vigatabhīḥ} \\
 & \romline{brahmacārivrate sthitaḥ |} \\
 & \romline{manaḥ saṃyamya maccittaḥ} \\
 & \romline{yukta āsīta matparaḥ ||}
\end{tabular}
\end{table}

\begin{table}[H]
\begin{tabular}{cl}
\textbf{6.15} & \romline{yuñjannevaṃ sadā''tmānaṃ} \\
 & \romline{yogī niyatamānasaḥ |} \\
 & \romline{śāntiṃ nirvāṇaparamāṃ} \\
 & \romline{matsaṃsthāmadhigacchati ||}
\end{tabular}
\end{table}

\begin{table}[H]
\begin{tabular}{cl}
\textbf{6.16} & \romline{nātyaśnatastu yogo'sti} \\
 & \romline{na caikāntamanaśnataḥ |} \\
 & \romline{na cāti svapnaśīlasya} \\
 & \romline{jāgrato naiva cārjuna ||}
\end{tabular}
\end{table}

\begin{table}[H]
\begin{tabular}{cl}
\textbf{6.17} & \romline{yuktāhāravihārasya} \\
 & \romline{yuktaceṣṭasya karmasu |} \\
 & \romline{yuktasvapnāvabodhasya} \\
 & \romline{yogo bhavati duḥkhahā ||}
\end{tabular}
\end{table}

\begin{table}[H]
\begin{tabular}{cl}
\textbf{6.18} & \romline{yadā viniyataṃ cittam} \\
 & \romline{ātmanyevāvatiṣṭhate |} \\
 & \romline{nisspṛhaḥ sarvakāmebhyaḥ} \\
 & \romline{yukta ityucyate tadā ||}
\end{tabular}
\end{table}

\begin{table}[H]
\begin{tabular}{cl}
\textbf{6.19} & \romline{yathā dīpo nivātasthaḥ} \\
 & \romline{neṅgate sopamā smṛtā |} \\
 & \romline{yogino yatacittasya} \\
 & \romline{yuñjato yogomātmanaḥ ||}
\end{tabular}
\end{table}

\begin{table}[H]
\begin{tabular}{cl}
\textbf{6.20} & \romline{yatroparamate cittaṃ} \\
 & \romline{niruddhaṃ yogasevayā |} \\
 & \romline{yatra caivātmanā''tmānaṃ} \\
 & \romline{pasyannātmani tuṣyati ||}
\end{tabular}
\end{table}

\begin{table}[H]
\begin{tabular}{cl}
\textbf{6.21} & \romline{sukhamātyantikaṃ yattat} \\
 & \romline{buddhigrāhyamatīndriyam |} \\
 & \romline{vetti yatra na caivāyaṃ} \\
 & \romline{sthitaścalati tattvataḥ ||}
\end{tabular}
\end{table}

\begin{table}[H]
\begin{tabular}{cl}
\textbf{6.22} & \romline{yaṃ labdhvā cāparaṃ lābhaṃ} \\
 & \romline{manyate nādhikaṃ tataḥ |} \\
 & \romline{yasmin sthito na duḥkhena} \\
 & \romline{guruṇāpi vicālyate ||}
\end{tabular}
\end{table}

\begin{table}[H]
\begin{tabular}{cl}
\textbf{6.23} & \romline{taṃ vidyāt duḥkhasaṃyoga} \\
 & \romline{viyogaṃ yogasañjñitam |} \\
 & \romline{sa niścayena yoktavyaḥ} \\
 & \romline{yogo'nirviṇṇacetasā ||}
\end{tabular}
\end{table}

\begin{table}[H]
\begin{tabular}{cl}
\textbf{6.24} & \romline{saṅkalpaprabhavānkāmān} \\
 & \romline{tyaktvā sarvānaśeṣataḥ |} \\
 & \romline{manasaivendriyagrāmaṃ} \\
 & \romline{viniyamya samantataḥ ||}
\end{tabular}
\end{table}

\begin{table}[H]
\begin{tabular}{cl}
\textbf{6.25} & \romline{śanaiḥ śanairuparamet} \\
 & \romline{buddhyā dhṛtigṛhītayā |} \\
 & \romline{ātmasaṃsthaṃ manaḥ kṛtvā} \\
 & \romline{na kiñcidapi cintayet ||}
\end{tabular}
\end{table}

\begin{table}[H]
\begin{tabular}{cl}
\textbf{6.26} & \romline{yato yato niścarati} \\
 & \romline{manaścañcalamasthiram |} \\
 & \romline{tatastato niyamyaitat} \\
 & \romline{ātmanyeva vaśaṃ nayet ||}
\end{tabular}
\end{table}

\begin{table}[H]
\begin{tabular}{cl}
\textbf{6.27} & \romline{praśāntamanasaṃ hyenaṃ} \\
 & \romline{yoginaṃ sukhamuttamam |} \\
 & \romline{upaiti śāntarajasaṃ} \\
 & \romline{brahmabhūtamakalmaṣam ||}
\end{tabular}
\end{table}

\begin{table}[H]
\begin{tabular}{cl}
\textbf{6.28} & \romline{yuñjannevaṃ sadā''tmānaṃ} \\
 & \romline{yogī vigatakalmaṣaḥ |} \\
 & \romline{sukhena brahmasaṃsparśaṃ} \\
 & \romline{atyantaṃ sukhamaśnute ||}
\end{tabular}
\end{table}

\begin{table}[H]
\begin{tabular}{cl}
\textbf{6.29} & \romline{sarvabhūtasthamātmānaṃ} \\
 & \romline{sarvabhūtāni cātmani |} \\
 & \romline{īkṣate yogayuktātmā} \\
 & \romline{sarvatra samadarśanaḥ ||}
\end{tabular}
\end{table}

\begin{table}[H]
\begin{tabular}{cl}
\textbf{6.30} & \romline{yo māṃ paśyati sarvatra} \\
 & \romline{sarvaṃ ca mayi paśyati |} \\
 & \romline{tasyāhaṃ na praṇaśyāmi} \\
 & \romline{sa ca me na praṇaśyati ||}
\end{tabular}
\end{table}

\begin{table}[H]
\begin{tabular}{cl}
\textbf{6.31} & \romline{sarvabhūtasthitaṃ yo māṃ} \\
 & \romline{bhajatyekatvamāsthitaḥ |} \\
 & \romline{sarvathā vartamāno'pi} \\
 & \romline{sa yogī mayi vartate ||}
\end{tabular}
\end{table}

\begin{table}[H]
\begin{tabular}{cl}
\textbf{6.32} & \romline{ātmaupamyena sarvatra} \\
 & \romline{samaṃ paśyati yo'rjuna |} \\
 & \romline{sukhaṃ vā yadi vā duḥkhaṃ} \\
 & \romline{sa yogī paramo mataḥ ||}
\end{tabular}
\end{table}

\begin{table}[H]
\begin{tabular}{cl}
\textbf{6.33} & \romline{arjuna uvāca} \\
 & \romline{yo'yaṃ yogastvayā proktaḥ} \\
 & \romline{sāmyena madhusūdana |} \\
 & \romline{etasyāhaṃ na paśyāmi} \\
 & \romline{cañcalatvāt sthitiṃ sthirām ||}
\end{tabular}
\end{table}

\begin{table}[H]
\begin{tabular}{cl}
\textbf{6.34} & \romline{cañcalaṃ hi manaḥ kṛṣṇa} \\
 & \romline{pramāthi balavaddṛḍham |} \\
 & \romline{tasyāhaṃ nigrahaṃ manye} \\
 & \romline{vāyoriva suduṣkaram ||}
\end{tabular}
\end{table}

\begin{table}[H]
\begin{tabular}{cl}
\textbf{6.35} & \romline{śrī bhagavānuvāca} \\
 & \romline{asaṃśayaṃ mahābāho} \\
 & \romline{mano durnigrahaṃ calam |} \\
 & \romline{abhyāsena tu kaunteya} \\
 & \romline{vairāgyeṇa ca gṛhyate ||}
\end{tabular}
\end{table}

\begin{table}[H]
\begin{tabular}{cl}
\textbf{6.36} & \romline{asaṃyatātmanā yogaḥ} \\
 & \romline{duṣprāpa iti me matiḥ |} \\
 & \romline{vaśyātmanā tu yatatā} \\
 & \romline{śakyo'vāptumupāyataḥ ||}
\end{tabular}
\end{table}

\begin{table}[H]
\begin{tabular}{cl}
\textbf{6.37} & \romline{arjuna uvāca} \\
 & \romline{ayatiḥ śraddhayopetaḥ} \\
 & \romline{yogāccalitamānasaḥ |} \\
 & \romline{aprāpya yogasaṃsiddhiṃ} \\
 & \romline{kāṃ gatiṃ kṛṣṇa gacchati ||}
\end{tabular}
\end{table}

\begin{table}[H]
\begin{tabular}{cl}
\textbf{6.38} & \romline{kaccinnobhayavibhraṣṭaḥ} \\
 & \romline{chinnābhramiva naśyati |} \\
 & \romline{apratiṣṭho mahābāho} \\
 & \romline{vimūḍho brahmaṇaḥ pathi ||}
\end{tabular}
\end{table}

\begin{table}[H]
\begin{tabular}{cl}
\textbf{6.39} & \romline{etanme saṃśayaṃ kṛṣṇa} \\
 & \romline{chettumarhasyaśeṣataḥ |} \\
 & \romline{tvadanyaḥ saṃśayasyāsya} \\
 & \romline{chettā na hyupapadyate ||}
\end{tabular}
\end{table}

\begin{table}[H]
\begin{tabular}{cl}
\textbf{6.40} & \romline{śrī bhagavānuvāca} \\
 & \romline{pārtha naiveha nāmutra} \\
 & \romline{vināśastasya vidyate |} \\
 & \romline{na hi kalyāṇakṛtkaścit} \\
 & \romline{durgatiṃ tāta gacchati ||}
\end{tabular}
\end{table}

\begin{table}[H]
\begin{tabular}{cl}
\textbf{6.41} & \romline{prāpya puṇyakṛtāṃ lokāṇ} \\
 & \romline{uṣitvā śāśvatīḥ samāḥ |} \\
 & \romline{śucīnāṃ śrīmatāṃ gehe} \\
 & \romline{yogabhraṣṭo'bhijāyate ||}
\end{tabular}
\end{table}

\begin{table}[H]
\begin{tabular}{cl}
\textbf{6.42} & \romline{athavā yogināmeva} \\
 & \romline{kule bhavati dhīmatām |} \\
 & \romline{etaddhi durlabhataraṃ} \\
 & \romline{loke janma yadīdṛśam ||}
\end{tabular}
\end{table}

\begin{table}[H]
\begin{tabular}{cl}
\textbf{6.43} & \romline{tatra taṃ buddhisaṃyogaṃ} \\
 & \romline{labhate paurvadehikam |} \\
 & \romline{yatate ca tato bhūyaḥ} \\
 & \romline{saṃsiddhau kurunandana ||}
\end{tabular}
\end{table}

\begin{table}[H]
\begin{tabular}{cl}
\textbf{6.44} & \romline{purvābhyāsena tenaiva} \\
 & \romline{hriyate hyavaśo'pi saḥ |} \\
 & \romline{jijñāsurapi yogasya} \\
 & \romline{śabdabrahmātivartate ||}
\end{tabular}
\end{table}

\begin{table}[H]
\begin{tabular}{cl}
\textbf{6.45} & \romline{prayatnādyatamānastu} \\
 & \romline{yogī saṃśuddhakilbiṣaḥ |} \\
 & \romline{anekajanmasaṃsiddhaḥ} \\
 & \romline{tato yāti parāṃ gatim ||}
\end{tabular}
\end{table}

\begin{table}[H]
\begin{tabular}{cl}
\textbf{6.46} & \romline{tapasvibhyo'dhiko yogī} \\
 & \romline{jñānibhyo'pi mato'dhikaḥ |} \\
 & \romline{karmibhyaścādhiko yogī} \\
 & \romline{tasmādyogī bhavārjuna ||}
\end{tabular}
\end{table}

\begin{table}[H]
\begin{tabular}{cl}
\textbf{6.47} & \romline{yogināmapi sarveṣāṃ} \\
 & \romline{madgatenāntarātmanā |} \\
 & \romline{śraddhāvānbhajate yo māṃ} \\
 & \romline{sa me yuktatamo mataḥ ||}
\end{tabular}
\end{table}

\begin{table}[H]
\begin{tabular}{cl}
 & \romline{śrīmadbhagavadgītāsu upaniṣatsu} \\
 & \romline{brahmavidyāyāṃ yogaśāstre} \\
 & \romline{śrīkṛṣṇārjuna saṃvāde} \\
 & \romline{ātmasaṃyamayogo nāma} \\
 & \romline{ṣaṣṭhodhyāyaḥ}
\end{tabular}
\end{table}


% \end{multicols}

\chapter{Jñāna-Vijñāna Yoga}
% \begin{multicols}{2}
\begin{table}[H]
\begin{tabular}{cl}
 & \natline{శ్రీ పరమాత్మనే నమః} \\
 & \natline{అథ సప్తమోఽధ్యాయః} \\
 & \natline{జ్ఞానవిజ్ఞానయోగః}
\end{tabular}
\end{table}

\begin{table}[H]
\begin{tabular}{cl}
\textbf{7.1} & \natline{శ్రీ భగవానువాచ} \\
 & \natline{మయ్యాసక్తమనాః పార్థ} \\
 & \natline{యోగం యుఞ్జన్మదాశ్రయః |} \\
 & \natline{అసంశయం సమగ్రం మాం} \\
 & \natline{యథా జ్ఞాస్యసి తచ్ఛృణు ||}
\end{tabular}
\end{table}

\begin{table}[H]
\begin{tabular}{cl}
\textbf{7.2} & \natline{జ్ఞానం తేఽహం సవిజ్ఞానమ్} \\
 & \natline{ఇదం వక్ష్యామ్యశేషతః |} \\
 & \natline{యజ్జ్ఞాత్వా నేహ భూయోఽన్యత్} \\
 & \natline{జ్ఞాతవ్యమవశిష్యతే ||}
\end{tabular}
\end{table}

\begin{table}[H]
\begin{tabular}{cl}
\textbf{7.3} & \natline{మనుష్యాణాం సహస్రేషు} \\
 & \natline{కశ్చిద్యతతి సిద్ధయే |} \\
 & \natline{యతతామపి సిద్ధానాం} \\
 & \natline{కస్చిన్మాం వేత్తి తత్త్వతహ్ ||}
\end{tabular}
\end{table}

\begin{table}[H]
\begin{tabular}{cl}
\textbf{7.4} & \natline{భూమిరాపోఽనలో వాయుః} \\
 & \natline{ఖం మనో బుద్ధిరేవ చ |} \\
 & \natline{అహఙ్కార ఇతీయం మే} \\
 & \natline{భిన్నా ప్రకృతిరష్టధా ||}
\end{tabular}
\end{table}

\begin{table}[H]
\begin{tabular}{cl}
\textbf{7.5} & \natline{అపరేయమితస్త్వన్యాం} \\
 & \natline{ప్రకృతిం విద్ధి మే పరామ్ |} \\
 & \natline{జీవభూతాం మహాబాహో} \\
 & \natline{యయేదం ధార్యతే జగత్ ||}
\end{tabular}
\end{table}

\begin{table}[H]
\begin{tabular}{cl}
\textbf{7.6} & \natline{ఏతద్యోనీని భూతాని} \\
 & \natline{సర్వాణీత్యుపధారయ |} \\
 & \natline{అహం కృత్స్నస్య జగతః} \\
 & \natline{ప్రభవః ప్రలయస్తథా ||}
\end{tabular}
\end{table}

\begin{table}[H]
\begin{tabular}{cl}
\textbf{7.7} & \natline{మత్తః పరతరం నాన్యత్} \\
 & \natline{కిఞ్చిదస్తి ధనఞ్జయ |} \\
 & \natline{మయి సర్వమిదం ప్రోతం} \\
 & \natline{సూత్రే మణిగణా ఇవ ||}
\end{tabular}
\end{table}

\begin{table}[H]
\begin{tabular}{cl}
\textbf{7.8} & \natline{రసోఽహమప్సు కౌన్తేయ} \\
 & \natline{ప్రభాఽస్మి శశిసూర్యయోః |} \\
 & \natline{ప్రణవః సర్వవేదేషు} \\
 & \natline{శబ్దః ఖే పౌరుషం నృషు ||}
\end{tabular}
\end{table}

\begin{table}[H]
\begin{tabular}{cl}
\textbf{7.9} & \natline{పుణ్యో గన్ధః పృథివ్యాం చ} \\
 & \natline{తేజశ్చాస్మి విభావసౌ |} \\
 & \natline{జీవనం సర్వభుతేషు} \\
 & \natline{తపశ్చాస్మి తపస్విషు ||}
\end{tabular}
\end{table}

\begin{table}[H]
\begin{tabular}{cl}
\textbf{7.10} & \natline{బీజం మాం సర్వభూతానాం} \\
 & \natline{విద్ధి పార్థ సనాతనమ్ |} \\
 & \natline{బుద్ధిర్బుద్ధిమతామస్మి} \\
 & \natline{తేజస్తేజస్వినామహమ్ ||}
\end{tabular}
\end{table}

\begin{table}[H]
\begin{tabular}{cl}
\textbf{7.11} & \natline{బలం బలవతాం చాహం} \\
 & \natline{కామరాగవివర్జితమ్ |} \\
 & \natline{ధర్మావిరుద్ధో భూతేషు} \\
 & \natline{కామోఽస్మి భరతర్షభ ||}
\end{tabular}
\end{table}

\begin{table}[H]
\begin{tabular}{cl}
\textbf{7.12} & \natline{యే చైవ సాత్త్వికా భావాః} \\
 & \natline{రాజసాస్తామసాశ్చ యే |} \\
 & \natline{మత్త ఏవేతి తాన్విద్ధి} \\
 & \natline{న త్వహం తేషు తే మయి ||}
\end{tabular}
\end{table}

\begin{table}[H]
\begin{tabular}{cl}
\textbf{7.13} & \natline{త్రిభిర్గుణమయైర్భావైః} \\
 & \natline{ఏభిః సర్వమిదం జగత్ |} \\
 & \natline{మోహితం నాభిజానాతి} \\
 & \natline{మామేభ్యః పరమవ్యయమ్ ||}
\end{tabular}
\end{table}

\begin{table}[H]
\begin{tabular}{cl}
\textbf{7.14} & \natline{దైవీ హ్యేషా గునామయీ} \\
 & \natline{మమ మాయ దురత్యయా |} \\
 & \natline{మామేవ యే ప్రపద్యన్తే} \\
 & \natline{మాయామేతాం తరన్తి తే ||}
\end{tabular}
\end{table}

\begin{table}[H]
\begin{tabular}{cl}
\textbf{7.15} & \natline{న మాం దుష్కృతినో మూఢాః} \\
 & \natline{ప్రపద్యన్తే నరాధమాః |} \\
 & \natline{మాయయాఽపహృతజ్ఞానాః} \\
 & \natline{ఆసురం భావమాశ్రితాః ||}
\end{tabular}
\end{table}

\begin{table}[H]
\begin{tabular}{cl}
\textbf{7.16} & \natline{చతుర్విధా భజన్తే మాం} \\
 & \natline{జనాః సుకృతినోఽర్జున |} \\
 & \natline{ఆర్తో జిజ్ఞాసురర్థార్థీ} \\
 & \natline{జ్ఞానీ చ భరతర్షభ ||}
\end{tabular}
\end{table}

\begin{table}[H]
\begin{tabular}{cl}
\textbf{7.17} & \natline{తేషాం జ్ఞానీ నిత్యయుక్తః} \\
 & \natline{ఏకభక్తిర్విశిష్యతే |} \\
 & \natline{ప్రియో హి జ్ఞానినోఽత్యర్థమ్} \\
 & \natline{అహం స చ మమ ప్రియః ||}
\end{tabular}
\end{table}

\begin{table}[H]
\begin{tabular}{cl}
\textbf{7.18} & \natline{ఉదారాః సర్వ ఏవైతే} \\
 & \natline{జ్ఞానీ త్వాత్మైవ మే మతమ్ |} \\
 & \natline{ఆస్థితః స హి యుక్తాత్మా} \\
 & \natline{మామేవానుత్తమాం గతిమ్ ||}
\end{tabular}
\end{table}

\begin{table}[H]
\begin{tabular}{cl}
\textbf{7.19} & \natline{బహూనాం జన్మనామన్తే} \\
 & \natline{జ్ఞానవాన్మాం ప్రపద్యతే |} \\
 & \natline{వాసుదేవః సర్వమితి} \\
 & \natline{స మహాత్మా సుదుర్లభః ||}
\end{tabular}
\end{table}

\begin{table}[H]
\begin{tabular}{cl}
\textbf{7.20} & \natline{కామైస్తైస్తైర్హృతజ్ఞానాః} \\
 & \natline{ప్రపద్యన్తేఽన్యదేవతాః |} \\
 & \natline{తం తం నియమమాస్థాయ} \\
 & \natline{ప్రకృత్యా నియతాః స్వయా ||}
\end{tabular}
\end{table}

\begin{table}[H]
\begin{tabular}{cl}
\textbf{7.21} & \natline{యో యో యాం యాం తనుం భక్తః} \\
 & \natline{శ్రద్ధయార్చితుమిచ్ఛతి |} \\
 & \natline{తస్య తస్యాచలాం శ్రద్ధాం} \\
 & \natline{తామేవ విదధామ్యహమ్ ||}
\end{tabular}
\end{table}

\begin{table}[H]
\begin{tabular}{cl}
\textbf{7.22} & \natline{స తయా శ్రద్ధయా యుక్తః} \\
 & \natline{తస్యారాధనమీహతే |} \\
 & \natline{లభతే చ తతః కామాన్} \\
 & \natline{మయైవ విహితాన్హి తాన్ ||}
\end{tabular}
\end{table}

\begin{table}[H]
\begin{tabular}{cl}
\textbf{7.23} & \natline{అన్తవత్తు ఫలం తేషాం} \\
 & \natline{తద్భవత్యల్పమేధసాం |} \\
 & \natline{దేవాన్దేవయజో యాన్తి} \\
 & \natline{మద్భక్తా యాన్తి మామపి ||}
\end{tabular}
\end{table}

\begin{table}[H]
\begin{tabular}{cl}
\textbf{7.24} & \natline{అవ్యక్తం వ్యక్తిమాపన్నం} \\
 & \natline{మన్యన్తే మామబుద్ధయః |} \\
 & \natline{పరం భావమజానన్తః} \\
 & \natline{మమావ్యయమనుత్తమన్ ||}
\end{tabular}
\end{table}

\begin{table}[H]
\begin{tabular}{cl}
\textbf{7.25} & \natline{నాహం ప్రకాశః సర్వస్య} \\
 & \natline{యోగమాయాసమావృతః |} \\
 & \natline{మూఢోఽయం నాభిజానాతి} \\
 & \natline{లోకో మామజమవ్యయమ్ ||}
\end{tabular}
\end{table}

\begin{table}[H]
\begin{tabular}{cl}
\textbf{7.26} & \natline{వేదాహం సమతీతాని} \\
 & \natline{వర్తమానాని చార్జున |} \\
 & \natline{భవిష్యాణి చ భూతాని} \\
 & \natline{మాం తు వేద న కశ్చన ||}
\end{tabular}
\end{table}

\begin{table}[H]
\begin{tabular}{cl}
\textbf{7.27} & \natline{ఇచ్ఛాద్వేషసముత్థేన} \\
 & \natline{ద్వన్ద్వమోహేన భారత |} \\
 & \natline{సర్వభూతాని సమ్మోహం} \\
 & \natline{సర్గే యాన్తి పరన్తప ||}
\end{tabular}
\end{table}

\begin{table}[H]
\begin{tabular}{cl}
\textbf{7.28} & \natline{యేషాం త్వన్తగతం పాపమ్} \\
 & \natline{జనానాం పుణ్యకర్మణామ్ |} \\
 & \natline{తే ద్వన్ద్వమోహనిర్ముక్తాః} \\
 & \natline{భజన్తే మాం దృఢవ్రతాః ||}
\end{tabular}
\end{table}

\begin{table}[H]
\begin{tabular}{cl}
\textbf{7.29} & \natline{జరామరణమోక్షాయ} \\
 & \natline{మామాశ్రిత్య యతన్తి యే |} \\
 & \natline{తే బ్రహ్మ తద్విదుః కృత్స్నమ్} \\
 & \natline{అధ్యాత్మం కర్మ చాఖిలమ్ ||}
\end{tabular}
\end{table}

\begin{table}[H]
\begin{tabular}{cl}
\textbf{7.30} & \natline{సాధిభూతాధిదైవం మాం} \\
 & \natline{సాధియజ్ఞం చే యే విదుః |} \\
 & \natline{ప్రయాణకాలేఽపి చ మాం} \\
 & \natline{తే విదుర్యుక్తచేతసః ||}
\end{tabular}
\end{table}

\begin{table}[H]
\begin{tabular}{cl}
 & \natline{శ్రీమద్భగవద్గీతాసు ఉపనిషత్సు} \\
 & \natline{బ్రహ్మవిద్యాయాం యోగశాస్త్రే} \\
 & \natline{శ్రీకృష్ణార్జున సంవాదే} \\
 & \natline{జ్ఞానవిజ్ఞానయోగోనామ} \\
 & \natline{సప్తమోధ్యాయః}
\end{tabular}
\end{table}


% \end{multicols}

\chapter{Akṣara-Parabrahma Yoga}
% \begin{multicols}{2}
\begin{table}[H]
\begin{tabular}{cl}
\textbf{8.0} & \natline{ఓం శ్రీ పరమాత్మనే నమః} \\
 & \natline{అథ అష్టమోఽధ్యాయః} \\
 & \natline{అక్షరపరబ్రహ్మయోగః}
\end{tabular}
\end{table}

\begin{table}[H]
\begin{tabular}{cl}
\textbf{8.1} & \natline{అర్జున ఉవాచ} \\
 & \natline{కిం తద్బ్రహ్మ కిమధ్యాత్మం} \\
 & \natline{కిం కర్మ పురుషోత్తమ |} \\
 & \natline{అధిభూతం చ కిం ప్రోక్తమ్} \\
 & \natline{అధిదైవం కిముచ్యతే ||}
\end{tabular}
\end{table}

\begin{table}[H]
\begin{tabular}{cl}
\textbf{8.2} & \natline{అధియజ్ఞః కథం కోఽత్ర} \\
 & \natline{దేహేఽస్మిన్మధుసూదన |} \\
 & \natline{ప్రయాణకాలే చ కథం} \\
 & \natline{జ్ఞేయోఽసి నియతాత్మభిః ||}
\end{tabular}
\end{table}

\begin{table}[H]
\begin{tabular}{cl}
\textbf{8.3} & \natline{శ్రీ భగవానువాచ} \\
 & \natline{అక్షరమ్ బ్రహ్మ పరమం} \\
 & \natline{స్వభావోఽధ్యాత్మముచ్యతే |} \\
 & \natline{భూతభావోద్భవకరః} \\
 & \natline{విసర్గః కర్మసఞ్జ్ఞితః ||}
\end{tabular}
\end{table}


% \end{multicols}

\chapter{Rājavidyā-Rājaguhya Yoga}
% \begin{multicols}{2}
\begin{table}[H]
\begin{tabular}{cl}
 & \romline{śrī paramātmane namaḥ} \\
 & \romline{atha navamo'dhyāyaḥ} \\
 & \romline{rājavidyārājaguhyayogaḥ}
\end{tabular}
\end{table}

\begin{table}[H]
\begin{tabular}{cl}
\textbf{9.1} & \romline{śrī bhagavānuvāca} \\
 & \romline{idaṃ tu te guhyatamam} \\
 & \romline{pravakṣyāmyanasūyave |} \\
 & \romline{jñānaṃ vijñānasahitaṃ} \\
 & \romline{yajjñātvā mokṣyase'śubhāt ||}
\end{tabular}
\end{table}

\begin{table}[H]
\begin{tabular}{cl}
\textbf{9.2} & \romline{rājavidyā rājaguhyaṃ} \\
 & \romline{pavitramidamuttamam |} \\
 & \romline{pratyakṣāvagamaṃ dharmyaṃ} \\
 & \romline{susukhaṃ kartumavyayam ||}
\end{tabular}
\end{table}

\begin{table}[H]
\begin{tabular}{cl}
\textbf{9.3} & \romline{aśraddadhānāḥ puruṣāḥ} \\
 & \romline{dharmasyāsya parantapa |} \\
 & \romline{aprāpya māṃ nivartante} \\
 & \romline{mṛtyusaṃsāravartmani ||}
\end{tabular}
\end{table}

\begin{table}[H]
\begin{tabular}{cl}
\textbf{9.4} & \romline{mayā tatamidaṃ sarvaṃ} \\
 & \romline{jagadavyaktamūrtinā |} \\
 & \romline{matsthāni sarvabhūtāni} \\
 & \romline{na cāhaṃ teṣvavasthitaḥ ||}
\end{tabular}
\end{table}

\begin{table}[H]
\begin{tabular}{cl}
\textbf{9.5} & \romline{na ca matsthāni bhūtāni} \\
 & \romline{paśya me yogamaiśvaram |} \\
 & \romline{bhūtabhṛnna ca bhūtasthaḥ} \\
 & \romline{mamātmā bhūtabhāvanaḥ ||}
\end{tabular}
\end{table}

\begin{table}[H]
\begin{tabular}{cl}
\textbf{9.6} & \romline{yathā''kāśasthito nityaṃ} \\
 & \romline{vāyuḥ sarvatrago mahān |} \\
 & \romline{tathā sarvāṇi bhūtāni} \\
 & \romline{matsthānītyupadhāraya ||}
\end{tabular}
\end{table}

\begin{table}[H]
\begin{tabular}{cl}
\textbf{9.7} & \romline{sarvabhūtāni kaunteya} \\
 & \romline{prakṛtiṃ yānti māmikām |} \\
 & \romline{kalpakṣaye punastāni} \\
 & \romline{kalpādau visṛjāmyaham ||}
\end{tabular}
\end{table}

\begin{table}[H]
\begin{tabular}{cl}
\textbf{9.8} & \romline{prakṛtiṃ svāmavaṣṭabhya} \\
 & \romline{visṛjāmi punaḥ punaḥ |} \\
 & \romline{bhūtagrāmamimaṃ kṛtsnam} \\
 & \romline{avaśaṃ prakṛtervaśāt ||}
\end{tabular}
\end{table}

\begin{table}[H]
\begin{tabular}{cl}
\textbf{9.9} & \romline{na ca māṃ tāni karmāṇi} \\
 & \romline{nibadhnanti dhanañjaya |} \\
 & \romline{udāsīnavadāsīnam} \\
 & \romline{asaktaṃ teṣu karmasu ||}
\end{tabular}
\end{table}

\begin{table}[H]
\begin{tabular}{cl}
\textbf{9.10} & \romline{mayādhyakṣeṇa prakṛtiḥ} \\
 & \romline{sūyate sacarācaram |} \\
 & \romline{hetunā'nena kaunteya} \\
 & \romline{jagadviparivartate ||}
\end{tabular}
\end{table}

\begin{table}[H]
\begin{tabular}{cl}
\textbf{9.11} & \romline{avajānanti māṃ mūḍhāḥ} \\
 & \romline{mānuṣīṃ tanumāśritam |} \\
 & \romline{paraṃ bhāvamajānantaḥ} \\
 & \romline{mama bhūtamaheśvaram ||}
\end{tabular}
\end{table}

\begin{table}[H]
\begin{tabular}{cl}
\textbf{9.12} & \romline{moghāśā moghakarmāṇaḥ} \\
 & \romline{moghajñānā vicetasaḥ |} \\
 & \romline{rākṣasīmāsurīṃ caiva} \\
 & \romline{prakṛtiṃ mohinīṃ śritāḥ ||}
\end{tabular}
\end{table}

\begin{table}[H]
\begin{tabular}{cl}
\textbf{9.13} & \romline{mahātmānastu māṃ pārtha} \\
 & \romline{daivīṃ prakṛtimāśritāḥ |} \\
 & \romline{bhajantyananyamanasaḥ} \\
 & \romline{jñātvā bhūtādimavyayam ||}
\end{tabular}
\end{table}

\begin{table}[H]
\begin{tabular}{cl}
\textbf{9.14} & \romline{satataṃ kīrtayanto māṃ} \\
 & \romline{yatantaśca dṛḍhavratāḥ |} \\
 & \romline{namasyantaśca mām bhaktyā} \\
 & \romline{nityayuktā upāsate ||}
\end{tabular}
\end{table}

\begin{table}[H]
\begin{tabular}{cl}
\textbf{9.15} & \romline{jñānayajñena cāpyanye} \\
 & \romline{yajanto māmupāsate |} \\
 & \romline{ekatvena pṛthaktvena} \\
 & \romline{bahudhā viśvatomukham ||}
\end{tabular}
\end{table}

\begin{table}[H]
\begin{tabular}{cl}
\textbf{9.16} & \romline{ahaṃ kraturahaṃ yajñaḥ} \\
 & \romline{svadhāhamahamauṣadham |} \\
 & \romline{mantro'hamahamevājyam} \\
 & \romline{ahamagnirahaṃ hutam ||}
\end{tabular}
\end{table}

\begin{table}[H]
\begin{tabular}{cl}
\textbf{9.17} & \romline{pitā'hamasya jagataḥ} \\
 & \romline{mātā dhātā pitāmahaḥ |} \\
 & \romline{vedyaṃ pavitramoṅkāraḥ} \\
 & \romline{ṛksāma yajureva ca ||}
\end{tabular}
\end{table}

\begin{table}[H]
\begin{tabular}{cl}
\textbf{9.18} & \romline{gatirbhartā prabhuḥ sākṣī} \\
 & \romline{nivāsaḥ śaraṇaṃ suhṛt |} \\
 & \romline{prabhavaḥ pralayaḥ sthānaṃ} \\
 & \romline{nidhānaṃ bījamavyayam ||}
\end{tabular}
\end{table}

\begin{table}[H]
\begin{tabular}{cl}
\textbf{9.19} & \romline{tapāmyahamahaṃ varṣaṃ} \\
 & \romline{nigṛhṇāmyutsṛjāmi ca |} \\
 & \romline{amṛtaṃ caiva mṛtyuśca} \\
 & \romline{sadasaccāhamarjuna ||}
\end{tabular}
\end{table}

\begin{table}[H]
\begin{tabular}{cl}
\textbf{9.20} & \romline{traividyā māṃ somapāḥ pūtapāpāḥ} \\
 & \romline{yajñairiṣṭvā svargatiṃ prārthayante |} \\
 & \romline{te puṇyamāsādya surendralokam} \\
 & \romline{aśnanti divyāndivi devabhogān ||}
\end{tabular}
\end{table}

\begin{table}[H]
\begin{tabular}{cl}
\textbf{9.21} & \romline{te taṃ bhuktvā svargalokaṃ viśālaṃ} \\
 & \romline{kṣīṇe puṇye martyalokaṃ viśanti |} \\
 & \romline{evaṃ trayīdharmamanuprapannāḥ} \\
 & \romline{gatāgataṃ kāmakāmā labhante ||}
\end{tabular}
\end{table}

\begin{table}[H]
\begin{tabular}{cl}
\textbf{9.22} & \romline{ananyāścintayanto māṃ} \\
 & \romline{ye janāḥ paryupāsate |} \\
 & \romline{teṣāṃ nityābhiyuktānāṃ} \\
 & \romline{yogakṣemaṃ vahāmyaham ||}
\end{tabular}
\end{table}

\begin{table}[H]
\begin{tabular}{cl}
\textbf{9.23} & \romline{ye'pyanyadevatā bhaktāḥ} \\
 & \romline{yajante śraddhayānvitāḥ |} \\
 & \romline{te'pi māmeva kaunteya} \\
 & \romline{yajantyavidhipūrvakam ||}
\end{tabular}
\end{table}

\begin{table}[H]
\begin{tabular}{cl}
\textbf{9.24} & \romline{ahaṃ hi sarvayajñānāṃ} \\
 & \romline{bhoktā ca prabhureva ca |} \\
 & \romline{na tu māmabhijānanti} \\
 & \romline{tattvenātaścyavanti te ||}
\end{tabular}
\end{table}

\begin{table}[H]
\begin{tabular}{cl}
\textbf{9.25} & \romline{yānti devavratā devāṇ} \\
 & \romline{pitṝn yānti pitṛvratāḥ |} \\
 & \romline{bhūtāni yānti bhūtejyāḥ} \\
 & \romline{yānti madyājino'pi mām ||}
\end{tabular}
\end{table}

\begin{table}[H]
\begin{tabular}{cl}
\textbf{9.26} & \romline{patraṃ puṣpaṃ phalaṃ toyaṃ} \\
 & \romline{yo me bhaktyā prayacchati |} \\
 & \romline{tadahaṃ bhaktyupahṛtaṃ} \\
 & \romline{aśnāmi prayatātmanaḥ ||}
\end{tabular}
\end{table}

\begin{table}[H]
\begin{tabular}{cl}
\textbf{9.27} & \romline{yatkaroṣi yadaśnāsi} \\
 & \romline{yajjuhoṣi dadāsi yat |} \\
 & \romline{yattapasyasi kaunteya} \\
 & \romline{tatkuruṣva madarpaṇam ||}
\end{tabular}
\end{table}

\begin{table}[H]
\begin{tabular}{cl}
\textbf{9.28} & \romline{śubhāśubhaphalairevaṃ} \\
 & \romline{mokṣyase karmabandhanaiḥ |} \\
 & \romline{sannyāsayogayuktātmā} \\
 & \romline{vimukto māmupaiṣyasi ||}
\end{tabular}
\end{table}

\begin{table}[H]
\begin{tabular}{cl}
\textbf{9.29} & \romline{samo'haṃ sarvabhūteṣu} \\
 & \romline{na me dveṣyo'sti na priyaḥ |} \\
 & \romline{ye bhajanti tu māṃ bhaktyā} \\
 & \romline{mayi te teṣu cāpyaham ||}
\end{tabular}
\end{table}

\begin{table}[H]
\begin{tabular}{cl}
\textbf{9.30} & \romline{api cetsudurācāraḥ} \\
 & \romline{bhajate māmananyabhāk |} \\
 & \romline{sādhureva sa mantavyaḥ} \\
 & \romline{samyagvyavasito hi saḥ ||}
\end{tabular}
\end{table}

\begin{table}[H]
\begin{tabular}{cl}
\textbf{9.31} & \romline{kṣipram bhavati dharmātmā} \\
 & \romline{śaśvacchāntiṃ nigacchati |} \\
 & \romline{kaunteya pratijānīhi} \\
 & \romline{na me bhaktaḥ praṇaśyati ||}
\end{tabular}
\end{table}

\begin{table}[H]
\begin{tabular}{cl}
\textbf{9.32} & \romline{mām hi pārtha vyapāśritya} \\
 & \romline{ye'pi syuḥ pāpayonayaḥ |} \\
 & \romline{striyo vaiśyāstathā śūdrāḥ} \\
 & \romline{te'pi yānti parāṃ gatim ||}
\end{tabular}
\end{table}

\begin{table}[H]
\begin{tabular}{cl}
\textbf{9.33} & \romline{kiṃ punarbrāhmaṇāḥ puṇyāḥ} \\
 & \romline{bhaktā rājarṣayastathā |} \\
 & \romline{anityamasukhaṃ lokam} \\
 & \romline{imaṃ prāpya bhajasva mām ||}
\end{tabular}
\end{table}

\begin{table}[H]
\begin{tabular}{cl}
\textbf{9.34} & \romline{manmanā bhava madbhaktaḥ} \\
 & \romline{madyājī māṃ namaskuru |} \\
 & \romline{māmevaiṣyasi yuktvaivam} \\
 & \romline{ātmānaṃ matparāyaṇaḥ ||}
\end{tabular}
\end{table}

\begin{table}[H]
\begin{tabular}{cl}
 & \romline{śrīmadbhagavadgītāsu upaniṣatsu} \\
 & \romline{brahmavidyāyāṃ yogaśāstre} \\
 & \romline{śrīkṛṣṇārjuna saṃvāde} \\
 & \romline{rājavidyārājaguhyayogo nāma} \\
 & \romline{navamodhyāyaḥ}
\end{tabular}
\end{table}


% \end{multicols}

\chapter{Vibhūti Yoga}
% \begin{multicols}{2}
\begin{table}[H]
\begin{tabular}{cl}
\textbf{10.0} & \natline{ఓం శ్రీ పరమాత్మనే నమః} \\
 & \natline{అథ దశమోఽధ్యాయః} \\
 & \natline{విభుతియోగః}
\end{tabular}
\end{table}

\begin{table}[H]
\begin{tabular}{cl}
\textbf{10.1} & \natline{శ్రీ భగవానువాచ} \\
 & \natline{భూయ ఏవ మహాబాహో} \\
 & \natline{శృణు మే పరమం వచః |} \\
 & \natline{యత్తేఽహం ప్రీయమాణాయ} \\
 & \natline{వక్ష్యామి హితకామ్యయా ||}
\end{tabular}
\end{table}

\begin{table}[H]
\begin{tabular}{cl}
\textbf{10.2} & \natline{న మే విదుః సురగణాః} \\
 & \natline{ప్రభవం న మహర్షయః |} \\
 & \natline{అహమాదిర్హి దేవానాం} \\
 & \natline{మహర్షీణాం చ సర్వశః ||}
\end{tabular}
\end{table}

\begin{table}[H]
\begin{tabular}{cl}
\textbf{10.3} & \natline{యో మామజమనాదిం చ} \\
 & \natline{వేత్తి లోకమహేశ్వరమ్ |} \\
 & \natline{అసమ్మూఢః స మర్త్యేషు} \\
 & \natline{సర్వపాపైః ప్రముచ్యతే ||}
\end{tabular}
\end{table}

\begin{table}[H]
\begin{tabular}{cl}
\textbf{10.4} & \natline{బుద్ధిర్జ్ఞానమసమ్మోహః} \\
 & \natline{క్షమా సత్యం దమః శమః |} \\
 & \natline{సుఖం దుఃఖం భవోఽభావః} \\
 & \natline{భయం చాభయమేవ చ ||}
\end{tabular}
\end{table}

\begin{table}[H]
\begin{tabular}{cl}
\textbf{10.5} & \natline{అహింసా సమతా తుష్టిః} \\
 & \natline{తపో దానం యశోఽయశః |} \\
 & \natline{భవన్తి భావా భూతానాం} \\
 & \natline{మత్త ఏవ పృథగ్విధాః ||}
\end{tabular}
\end{table}

\begin{table}[H]
\begin{tabular}{cl}
\textbf{10.6} & \natline{మహర్షయః సప్త పూర్వే} \\
 & \natline{చత్వారో మనవస్తథా |} \\
 & \natline{మద్భావా మానసా జాతాః} \\
 & \natline{యేషాం లోక ఇమాః ప్రజాః ||}
\end{tabular}
\end{table}

\begin{table}[H]
\begin{tabular}{cl}
\textbf{10.7} & \natline{ఏతాం విభూతిం యోగం చ} \\
 & \natline{మమ యో వేత్తి తత్త్వతః |} \\
 & \natline{సోఽవికమ్పేన యోగేన} \\
 & \natline{యుజ్యతే నాత్ర సంశయః ||}
\end{tabular}
\end{table}

\begin{table}[H]
\begin{tabular}{cl}
\textbf{10.8} & \natline{అహం సర్వస్య ప్రభవః} \\
 & \natline{మత్తః సర్వం ప్రవర్తతే |} \\
 & \natline{ఇతి మత్వా భజన్తే మాం} \\
 & \natline{బుధా భావసమన్వితాః ||}
\end{tabular}
\end{table}

\begin{table}[H]
\begin{tabular}{cl}
\textbf{10.9} & \natline{మచ్చిత్తా మద్గతప్రాణాః} \\
 & \natline{బోధయన్తః పరస్పరమ్ |} \\
 & \natline{కథయన్తశ్చ మాం నిత్యం} \\
 & \natline{తుష్యన్తి చ రమన్తి చ ||}
\end{tabular}
\end{table}

\begin{table}[H]
\begin{tabular}{cl}
\textbf{10.10} & \natline{తేషాం సతతయుక్తానాం} \\
 & \natline{భజతాం ప్రీతిపూర్వకమ్ |} \\
 & \natline{దదామి బుద్ధియోగం తం} \\
 & \natline{యేన మాముపయాన్తి తే ||}
\end{tabular}
\end{table}

\begin{table}[H]
\begin{tabular}{cl}
\textbf{10.11} & \natline{తేషామేవానుకమ్పార్థమ్} \\
 & \natline{అహమజ్ఞానజం తమః |} \\
 & \natline{నాశయామ్యాత్మభావస్థః} \\
 & \natline{జ్ఞానదీపేన భాస్వతా ||}
\end{tabular}
\end{table}

\begin{table}[H]
\begin{tabular}{cl}
\textbf{10.12} & \natline{అర్జున ఉవాచ} \\
 & \natline{పరం బ్రహ్మ పరం ధామ} \\
 & \natline{పవిత్రం పరమం భవాన్ |} \\
 & \natline{పురుషం శాశ్వతం దివ్యమ్} \\
 & \natline{ఆదిదేవమజం విభుమ్ ||}
\end{tabular}
\end{table}

\begin{table}[H]
\begin{tabular}{cl}
\textbf{10.13} & \natline{ఆహుస్త్వామృషయః సర్వే} \\
 & \natline{దేవర్షిర్నారదస్తథా |} \\
 & \natline{అసితో దేవలో వ్యాసః} \\
 & \natline{స్వయం చైవ బ్రవీషి మే ||}
\end{tabular}
\end{table}

\begin{table}[H]
\begin{tabular}{cl}
\textbf{10.14} & \natline{సర్వమేతదృతం మన్యే} \\
 & \natline{యన్మాం వదసి కేశవ |} \\
 & \natline{న హి తే భగవన్వ్యక్తిం} \\
 & \natline{విదుర్దేవా న దానవాః ||}
\end{tabular}
\end{table}

\begin{table}[H]
\begin{tabular}{cl}
\textbf{10.15} & \natline{స్వయమేవాత్మనాఽఽత్మానం} \\
 & \natline{వేత్థ త్వం పురుషోత్తమ |} \\
 & \natline{భూతభావన భూతేశ} \\
 & \natline{దేవదేవ జగత్పతే ||}
\end{tabular}
\end{table}

\begin{table}[H]
\begin{tabular}{cl}
\textbf{10.16} & \natline{వక్తుమర్హస్యశేషేణ} \\
 & \natline{దివ్యా హ్యాత్మవిభూతయః |} \\
 & \natline{యాభిర్విభూతిభిర్లోకాన్} \\
 & \natline{ఇమాంస్త్వం వ్యాప్య తిష్ఠసి ||}
\end{tabular}
\end{table}

\begin{table}[H]
\begin{tabular}{cl}
\textbf{10.17} & \natline{కథం విద్యామహం యోగిన్} \\
 & \natline{త్వాం సదా పరిచిన్తయన్ |} \\
 & \natline{కేషు కేషు చ భావేషు} \\
 & \natline{చిన్త్యోఽసి భగవన్మయా ||}
\end{tabular}
\end{table}

\begin{table}[H]
\begin{tabular}{cl}
\textbf{10.18} & \natline{విస్తరేణాత్మనో యోగం} \\
 & \natline{విభూతిం చ జనార్దన |} \\
 & \natline{భూయః కథయ తృప్తిర్హి} \\
 & \natline{శృణ్వతో నాస్తి మేఽమృతమ్ ||}
\end{tabular}
\end{table}

\begin{table}[H]
\begin{tabular}{cl}
\textbf{10.19} & \natline{శ్రీ భగవానువాచ} \\
 & \natline{హన్త తే కథయిష్యామి} \\
 & \natline{దివ్యా హ్యాత్మవిభూతయః |} \\
 & \natline{ప్రాధాన్యతః కురుశ్రేష్ఠ} \\
 & \natline{నాస్త్యన్తో విస్తరస్య మే ||}
\end{tabular}
\end{table}

\begin{table}[H]
\begin{tabular}{cl}
\textbf{10.20} & \natline{అహమాత్మా గుడాకేశ} \\
 & \natline{సర్వభూతాశయస్థితః |} \\
 & \natline{అహమాదిశ్చ మధ్యం చ} \\
 & \natline{భూతానామన్త ఏవ చ ||}
\end{tabular}
\end{table}

\begin{table}[H]
\begin{tabular}{cl}
\textbf{10.21} & \natline{ఆదిత్యానామహమ్ విష్ణుః} \\
 & \natline{జ్యోతిషాం రవిరంశుమాన్ |} \\
 & \natline{మరీచిర్మరుతామస్మి} \\
 & \natline{నక్షత్రాణామహం శశీ ||}
\end{tabular}
\end{table}

\begin{table}[H]
\begin{tabular}{cl}
\textbf{10.22} & \natline{వేదానాం సామవేదోఽస్మి} \\
 & \natline{దేవానామస్మి వాసవః |} \\
 & \natline{ఇన్ద్రియాణాం మనశ్చాస్మి} \\
 & \natline{భూతానామస్మి చేతనా ||}
\end{tabular}
\end{table}

\begin{table}[H]
\begin{tabular}{cl}
\textbf{10.23} & \natline{రుద్రాణాం శఙ్కరశ్చాస్మి} \\
 & \natline{విత్తేశో యక్షరక్షసామ్ |} \\
 & \natline{వసూనాం పావకశ్చాస్మి} \\
 & \natline{మేరుః శిఖరిణామహమ్ ||}
\end{tabular}
\end{table}

\begin{table}[H]
\begin{tabular}{cl}
\textbf{10.24} & \natline{పురోధసాం చ ముఖ్యం మాం} \\
 & \natline{విద్ధి పార్థ బృహస్పతిమ్ |} \\
 & \natline{సేనానీనామహం స్కన్దః} \\
 & \natline{సరసామస్మి సాగరః ||}
\end{tabular}
\end{table}

\begin{table}[H]
\begin{tabular}{cl}
\textbf{10.25} & \natline{మహర్షీణాం భృగురహం} \\
 & \natline{గిరామస్మ్యేకమక్షరమ్ |} \\
 & \natline{యజ్ఞానాం జపయజ్ఞోఽస్మి} \\
 & \natline{స్థావరాణాం హిమాలయః ||}
\end{tabular}
\end{table}

\begin{table}[H]
\begin{tabular}{cl}
\textbf{10.26} & \natline{అశ్వత్థః సర్వవృక్షాణాం} \\
 & \natline{దేవర్షీణాం చ నారదః |} \\
 & \natline{గన్ధర్వాణాం చిత్రరథః} \\
 & \natline{సిద్ధానాం కపిలో మునిః ||}
\end{tabular}
\end{table}

\begin{table}[H]
\begin{tabular}{cl}
\textbf{10.27} & \natline{ఉచ్చైః శ్రవసమశ్వానాం} \\
 & \natline{విద్ధి మామమృతోద్భవమ్ |} \\
 & \natline{ఐరావతం గజేన్ద్రాణాం} \\
 & \natline{నరాణాం చ నరాధిపమ్ ||}
\end{tabular}
\end{table}

\begin{table}[H]
\begin{tabular}{cl}
\textbf{10.28} & \natline{ఆయుధానామహం వజ్రం} \\
 & \natline{ధేనూనామస్మి కామధుక్ |} \\
 & \natline{ప్రజనశ్చాస్మి కన్దర్పః} \\
 & \natline{సర్పాణామస్మి వాసుకిః ||}
\end{tabular}
\end{table}

\begin{table}[H]
\begin{tabular}{cl}
\textbf{10.29} & \natline{అనన్తశ్చాస్మి నాగానాం} \\
 & \natline{వరుణో యాదసామహమ్ |} \\
 & \natline{పితౄణామర్యమా చాస్మి} \\
 & \natline{యమః సంయమతామహమ్ ||}
\end{tabular}
\end{table}

\begin{table}[H]
\begin{tabular}{cl}
\textbf{10.30} & \natline{ప్రహ్లాదశ్చాస్మి దైత్యానాం} \\
 & \natline{కాలః కలయతామహమ్ |} \\
 & \natline{మృగాణాం చ మృగేన్ద్రోఽహం} \\
 & \natline{వైనతేయశ్చ పక్షిణామ్ ||}
\end{tabular}
\end{table}

\begin{table}[H]
\begin{tabular}{cl}
\textbf{10.31} & \natline{పవనః పవతామస్మి} \\
 & \natline{రామః శస్త్రభృతామహమ్ |} \\
 & \natline{ఝషాణాం మకరశ్చాస్మి} \\
 & \natline{స్రోతసామస్మి జాహ్నవీ ||}
\end{tabular}
\end{table}

\begin{table}[H]
\begin{tabular}{cl}
\textbf{10.32} & \natline{సర్గాణామాదిరన్తశ్చ} \\
 & \natline{మధ్యం చైవాహమర్జున |} \\
 & \natline{అధ్యాత్మవిద్యా విద్యానాం} \\
 & \natline{వాదః ప్రవదతామహమ్ ||}
\end{tabular}
\end{table}

\begin{table}[H]
\begin{tabular}{cl}
\textbf{10.33} & \natline{అక్షరాణామకారోఽస్మి} \\
 & \natline{ద్వన్ద్వః సామాసికస్య చ |} \\
 & \natline{అహమేవాక్షయః కాలః} \\
 & \natline{ధాతాఽహం విశ్వతోముఖః ||}
\end{tabular}
\end{table}

\begin{table}[H]
\begin{tabular}{cl}
\textbf{10.34} & \natline{మృత్యుః సర్వహరశ్చాహమ్} \\
 & \natline{ఉద్భవశ్చ భవిష్యతామ్ |} \\
 & \natline{కీర్తిః శ్రీర్వాక్చ నారీణాం} \\
 & \natline{స్మృతిర్మేధా ధృతిః క్షమా ||}
\end{tabular}
\end{table}

\begin{table}[H]
\begin{tabular}{cl}
\textbf{10.35} & \natline{బృహత్సామ తథా సామ్నాం} \\
 & \natline{గాయత్రీ ఛన్దసామహమ్ |} \\
 & \natline{మాసానాం మార్గశీర్షోఽహమ్} \\
 & \natline{ఋతూనాం కుసుమాకరః ||}
\end{tabular}
\end{table}

\begin{table}[H]
\begin{tabular}{cl}
\textbf{10.36} & \natline{ద్యూతం ఛలయతామస్మి} \\
 & \natline{తేజస్తేజస్వినామహమ్ |} \\
 & \natline{జయోఽస్మి వ్యవసాయోఽస్మి} \\
 & \natline{సత్త్వం సత్త్వవతామహమ్ ||}
\end{tabular}
\end{table}

\begin{table}[H]
\begin{tabular}{cl}
\textbf{10.37} & \natline{వృష్ణీనాం వాసుదేవోఽస్మి} \\
 & \natline{పాణ్డవానాం ధనఞ్జయః |} \\
 & \natline{మునీనామప్యహం వ్యాసః} \\
 & \natline{కవీనాముశనా కవిః ||}
\end{tabular}
\end{table}

\begin{table}[H]
\begin{tabular}{cl}
\textbf{10.38} & \natline{దణ్డో దమయతామస్మి} \\
 & \natline{నీతిరస్మి జిగీషతామ్ |} \\
 & \natline{మౌనం చైవాస్మి గుహ్యానాం} \\
 & \natline{జ్ఞానం జ్ఞానవతామహమ్ ||}
\end{tabular}
\end{table}

\begin{table}[H]
\begin{tabular}{cl}
\textbf{10.39} & \natline{యచ్చాపి సర్వభూతానాం} \\
 & \natline{బీజం తదహమర్జున |} \\
 & \natline{న తదస్తి వినా యత్స్యాత్} \\
 & \natline{మయా భూతం చరాచరమ్ ||}
\end{tabular}
\end{table}

\begin{table}[H]
\begin{tabular}{cl}
\textbf{10.40} & \natline{నాన్తోఽస్తి మమ దివ్యానాం} \\
 & \natline{విభూతీనాం పరన్తప |} \\
 & \natline{ఏష తూద్దేశతః ప్రోక్తః} \\
 & \natline{విభూతేర్విస్తరో మయా ||}
\end{tabular}
\end{table}

\begin{table}[H]
\begin{tabular}{cl}
\textbf{10.41} & \natline{యద్యద్విభూతిమత్సత్త్వం} \\
 & \natline{శ్రీమదూర్జితమేవ వా |} \\
 & \natline{తత్తదేవావగచ్ఛ త్వం} \\
 & \natline{మమ తేజోఽమ్శసమ్భవమ్ ||}
\end{tabular}
\end{table}

\begin{table}[H]
\begin{tabular}{cl}
\textbf{10.42} & \natline{అథవా బహునైతేన} \\
 & \natline{కిం జ్ఞాతేన తవార్జున |} \\
 & \natline{విష్టభ్యాహమిదం కృత్స్నమ్} \\
 & \natline{ఏకాంశేన స్థితో జగత్ ||}
\end{tabular}
\end{table}


% \end{multicols}

\chapter{Viśvarūpa-Sandarśana Yoga}
% \begin{multicols}{2}
\begin{table}[H]
\begin{tabular}{cl}
 & \romline{śrī paramātmane namaḥ} \\
 & \romline{atha ekādaśo'dhyāyaḥ} \\
 & \romline{viśvarūpasandarśanayogaḥ}
\end{tabular}
\end{table}

\begin{table}[H]
\begin{tabular}{cl}
\textbf{11.1} & \romline{arjuna uvāca} \\
 & \romline{madanugrahāya paramaṃ} \\
 & \romline{guhyamadhyātmasañjñitam |} \\
 & \romline{yattvayoktaṃ vacastena} \\
 & \romline{moho'yaṃ vigato mama ||}
\end{tabular}
\end{table}

\begin{table}[H]
\begin{tabular}{cl}
\textbf{11.2} & \romline{bhavāpyayau hi bhūtānāṃ} \\
 & \romline{śrutau vistaraśo mayā |} \\
 & \romline{tvattaḥ kamalapatrākṣa} \\
 & \romline{māhātmyamapi cāvyayam ||}
\end{tabular}
\end{table}

\begin{table}[H]
\begin{tabular}{cl}
\textbf{11.3} & \romline{evametadyathā''ttha tvam} \\
 & \romline{ātmānaṃ parameśvara |} \\
 & \romline{draṣṭumicchāmi te rūpam} \\
 & \romline{aiśvaram puruṣottama ||}
\end{tabular}
\end{table}

\begin{table}[H]
\begin{tabular}{cl}
\textbf{11.4} & \romline{manyase yadi tacchakyaṃ} \\
 & \romline{mayā draṣṭumiti prabho |} \\
 & \romline{yogeśvara tato me tvaṃ} \\
 & \romline{darśayātmānamavyayam ||}
\end{tabular}
\end{table}

\begin{table}[H]
\begin{tabular}{cl}
\textbf{11.5} & \romline{śrī bhagavānuvāca} \\
 & \romline{paśya me pārtha rūpāṇi} \\
 & \romline{śataśo'tha sahasraśaḥ |} \\
 & \romline{nānāvidhāni divyāni} \\
 & \romline{nānāvarṇākṛtīni ca ||}
\end{tabular}
\end{table}

\begin{table}[H]
\begin{tabular}{cl}
\textbf{11.6} & \romline{paśyādityānvasūnrudrān} \\
 & \romline{aśvinau marutastathā |} \\
 & \romline{bahūnyadṛṣṭapūrvāṇi} \\
 & \romline{paśyāścaryāṇi bhārata ||}
\end{tabular}
\end{table}

\begin{table}[H]
\begin{tabular}{cl}
\textbf{11.7} & \romline{ihaikasthaṃ jagatkṛtsnaṃ} \\
 & \romline{paśyādya sacarācaram |} \\
 & \romline{mama dehe guḍākeśa} \\
 & \romline{yaccānyat draṣṭumicchasi ||}
\end{tabular}
\end{table}

\begin{table}[H]
\begin{tabular}{cl}
\textbf{11.8} & \romline{na tu māṃ śakyase draṣṭum} \\
 & \romline{anenaiva svacakṣuṣā |} \\
 & \romline{divyaṃ dadāmi te cakṣuḥ} \\
 & \romline{paśya me yogamaiśvaram ||}
\end{tabular}
\end{table}

\begin{table}[H]
\begin{tabular}{cl}
\textbf{11.9} & \romline{sañjaya uvāca} \\
 & \romline{evamuktvā tato rājan} \\
 & \romline{mahāyogeśvaro hariḥ |} \\
 & \romline{darśayāmāsa pārthāya} \\
 & \romline{paramaṃ rūpamaiśvaram ||}
\end{tabular}
\end{table}

\begin{table}[H]
\begin{tabular}{cl}
\textbf{11.10} & \romline{anekavaktranayanam} \\
 & \romline{anekādbhutadarśanam |} \\
 & \romline{anekadivyābharaṇaṃ} \\
 & \romline{divyānekodyatāyudham ||}
\end{tabular}
\end{table}

\begin{table}[H]
\begin{tabular}{cl}
\textbf{11.11} & \romline{divyamālyāmbaradharaṃ} \\
 & \romline{divyagandhānulepanam |} \\
 & \romline{sarvāścaryamayaṃ devam} \\
 & \romline{anantaṃ viśvatomukham ||}
\end{tabular}
\end{table}

\begin{table}[H]
\begin{tabular}{cl}
\textbf{11.12} & \romline{divi sūryasahasrasya} \\
 & \romline{bhavedyugapadutthitā |} \\
 & \romline{yadi bhāḥ sadṛśī sā syāt} \\
 & \romline{bhāsastasya mahātmanaḥ ||}
\end{tabular}
\end{table}

\begin{table}[H]
\begin{tabular}{cl}
\textbf{11.13} & \romline{tatraikasthaṃ jagatkṛtsnaṃ} \\
 & \romline{pravibhaktamanekadhā |} \\
 & \romline{apaśyaddevadevasya} \\
 & \romline{śarīre pāṇḍavastadā ||}
\end{tabular}
\end{table}

\begin{table}[H]
\begin{tabular}{cl}
\textbf{11.14} & \romline{tataḥ sa vismayāviṣṭaḥ} \\
 & \romline{hṛṣṭaromā dhanañjayaḥ |} \\
 & \romline{praṇamya śirasā devaṃ} \\
 & \romline{kṛtāñjalirabhāṣata ||}
\end{tabular}
\end{table}

\begin{table}[H]
\begin{tabular}{cl}
\textbf{11.15} & \romline{arjuna uvāca} \\
 & \romline{paśyāmi devāṃstava deva dehe} \\
 & \romline{sarvāṃstathā bhūtaviśeṣasaṅghān |} \\
 & \romline{brahmāṇamīśaṃ kamalāsanastham} \\
 & \romline{ṛṣīṃśca sarvānuragāṃśca divyān ||}
\end{tabular}
\end{table}

\begin{table}[H]
\begin{tabular}{cl}
\textbf{11.16} & \romline{anekabāhūdaravaktranetraṃ} \\
 & \romline{paśyāmi tvā sarvato'nantarūpam |} \\
 & \romline{nāntaṃ na madhyaṃ na punastavādiṃ} \\
 & \romline{paśyāmi viśveśvara viśvarūpa ||}
\end{tabular}
\end{table}

\begin{table}[H]
\begin{tabular}{cl}
\textbf{11.17} & \romline{kirīṭinaṃ gadinaṃ cakriṇaṃ ca} \\
 & \romline{tejorāśiṃ sarvato dīptimantam |} \\
 & \romline{paśyāmi tvāṃ durnirīkṣyaṃ samantāt} \\
 & \romline{dīptānalārkadyutimaprameyam ||}
\end{tabular}
\end{table}

\begin{table}[H]
\begin{tabular}{cl}
\textbf{11.18} & \romline{tvamakṣaraṃ paramaṃ veditavyaṃ} \\
 & \romline{tvamasya viśvasya paraṃ nidhānam |} \\
 & \romline{tvamavyayaḥ śāśvatadharmagoptā} \\
 & \romline{sanātanastvaṃ puruṣo mato me ||}
\end{tabular}
\end{table}

\begin{table}[H]
\begin{tabular}{cl}
\textbf{11.19} & \romline{anādimadhyāntamanantavīryam} \\
 & \romline{anantabāhuṃ śaśisūryanetram |} \\
 & \romline{paśyāmi tvāṃ dīptahutāśavaktraṃ} \\
 & \romline{svatejasā viśvamidaṃ tapantam ||}
\end{tabular}
\end{table}

\begin{table}[H]
\begin{tabular}{cl}
\textbf{11.20} & \romline{dyāvāpṛthivyoridamantaraṃ hi} \\
 & \romline{vyāptaṃ tvayaikena diśaśca sarvāḥ |} \\
 & \romline{dṛṣṭvādbhutaṃ rūpamidaṃ tavograṃ} \\
 & \romline{lokatrayaṃ pravyathitaṃ mahātman ||}
\end{tabular}
\end{table}

\begin{table}[H]
\begin{tabular}{cl}
\textbf{11.21} & \romline{amī hi tvā surasaṅghā viśanti} \\
 & \romline{kecidbhītāḥ prāñjalayo gṛṇanti |} \\
 & \romline{svastītyuktvā maharṣisiddhasaṅghāḥ} \\
 & \romline{stuvanti tvāṃ stutibhiḥ puṣkalābhiḥ ||}
\end{tabular}
\end{table}

\begin{table}[H]
\begin{tabular}{cl}
\textbf{11.22} & \romline{rudrādityā vasavo ye ca sādhyāḥ} \\
 & \romline{viśve'śvinau marutaścoṣmapāśca |} \\
 & \romline{gandharvayakṣāsurasiddhasaṅghāḥ} \\
 & \romline{vīkṣante tvāṃ vismitāścaiva sarve ||}
\end{tabular}
\end{table}

\begin{table}[H]
\begin{tabular}{cl}
\textbf{11.23} & \romline{rūpaṃ mahatte bahuvaktra netraṃ} \\
 & \romline{mahābāho bahubāhūrupādam |} \\
 & \romline{bahūdaraṃ bahudaṃṣṭrākarālaṃ} \\
 & \romline{dṛṣṭvā lokāḥ pravyathitāstathā'ham ||}
\end{tabular}
\end{table}

\begin{table}[H]
\begin{tabular}{cl}
\textbf{11.24} & \romline{nabhaḥ spṛśaṃ dīptamanekavarṇaṃ} \\
 & \romline{vyāttānanaṃ dīptaviśālanetram |} \\
 & \romline{dṛṣṭvā hi tvāṃ pravyathitāntarātmā} \\
 & \romline{dhṛtiṃ na vindāmi śamaṃ ca viṣṇo ||}
\end{tabular}
\end{table}

\begin{table}[H]
\begin{tabular}{cl}
\textbf{11.25} & \romline{daṃṣṭrākarālāni ca te mukhāni} \\
 & \romline{dṛṣṭvaiva kālānalasannibhāni |} \\
 & \romline{diśo na jāne na labhe ca śarma} \\
 & \romline{prasīda deveśa jagannivāsa ||}
\end{tabular}
\end{table}

\begin{table}[H]
\begin{tabular}{cl}
\textbf{11.26} & \romline{amī ca tvāṃ dhṛtarāṣṭrasya putrāḥ} \\
 & \romline{sarve sahaivāvanipālasaṅghaiḥ |} \\
 & \romline{bhīṣmo droṇaḥ sūtaputrastathā'sau} \\
 & \romline{sahāsmadīyairapi yodhamukhyaiḥ ||}
\end{tabular}
\end{table}

\begin{table}[H]
\begin{tabular}{cl}
\textbf{11.27} & \romline{vaktrāṇi te tvaramāṇā viśanti} \\
 & \romline{daṃṣṭrākarālāni bhayānakāni |} \\
 & \romline{kecidvilagnā daśanāntareṣu} \\
 & \romline{sandṛśyante cūrṇitairuttamāṅgaiḥ ||}
\end{tabular}
\end{table}

\begin{table}[H]
\begin{tabular}{cl}
\textbf{11.28} & \romline{yathā nadīnāṃ bahavo'mbuvegāḥ} \\
 & \romline{samudramevābhimukhā dravanti |} \\
 & \romline{tathā tavāmī naralokavīrāḥ} \\
 & \romline{viśanti vaktrāṇyabhivijvalanti ||}
\end{tabular}
\end{table}

\begin{table}[H]
\begin{tabular}{cl}
\textbf{11.29} & \romline{yathā pradīptaṃ jvalanaṃ pataṅgāḥ} \\
 & \romline{viśanti nāśāya samṛddhavegāḥ |} \\
 & \romline{tathaiva nāśāya viśanti lokāḥ} \\
 & \romline{tavāpi vaktrāṇi samṛddhavegāḥ ||}
\end{tabular}
\end{table}

\begin{table}[H]
\begin{tabular}{cl}
\textbf{11.30} & \romline{lelihyase grasamānaḥ samantāt} \\
 & \romline{lokānsamagrānvadanairjvaladbhiḥ |} \\
 & \romline{tejobhirāpūrya jagatsamagraṃ} \\
 & \romline{bhāsastavogrāḥ pratapanti viṣṇo ||}
\end{tabular}
\end{table}

\begin{table}[H]
\begin{tabular}{cl}
\textbf{11.31} & \romline{ākhyāhi me ko bhavānugrarūpaḥ} \\
 & \romline{namo'stu te devavara prasīda |} \\
 & \romline{vijñātumicchāmi bhavantamādyaṃ} \\
 & \romline{na hi prajānāmi tava pravṛttim ||}
\end{tabular}
\end{table}

\begin{table}[H]
\begin{tabular}{cl}
\textbf{11.32} & \romline{śrī bhagavānuvāca} \\
 & \romline{kālo'smi lokakṣayakṛtpravṛddhaḥ} \\
 & \romline{lokānsamāhartumiha pravṛttaḥ |} \\
 & \romline{ṛte'pi tvā na bhaviṣyanti sarve} \\
 & \romline{ye'vasthitāḥ pratyanīkeṣu yodhāḥ ||}
\end{tabular}
\end{table}

\begin{table}[H]
\begin{tabular}{cl}
\textbf{11.33} & \romline{tasmāttvamuttiṣṭha yaśo labhasva} \\
 & \romline{jitvā śatrūnbhuṅkṣva rājyaṃ samṛddham |} \\
 & \romline{mayaivaite nihatāḥ pūrvameva} \\
 & \romline{nimittamātraṃ bhava savyasācin ||}
\end{tabular}
\end{table}

\begin{table}[H]
\begin{tabular}{cl}
\textbf{11.34} & \romline{droṇaṃ ca bhīṣmaṃ ca jayadrathaṃ ca} \\
 & \romline{karṇaṃ tathānyānapi yodhavīrān |} \\
 & \romline{mayā hatāṃstvaṃ jahi mā vyathiṣṭhāḥ} \\
 & \romline{yudhyasva jetāsi raṇe sapatnān ||}
\end{tabular}
\end{table}

\begin{table}[H]
\begin{tabular}{cl}
\textbf{11.35} & \romline{sañjaya uvāca} \\
 & \romline{etacchrutvā vacanaṃ keśavasya} \\
 & \romline{kṛtāñjalirvepamānaḥ kirīṭī |} \\
 & \romline{namaskṛtvā bhūya evāha kṛṣṇaṃ} \\
 & \romline{sagadgadaṃ bhītabhītaḥ praṇamya ||}
\end{tabular}
\end{table}

\begin{table}[H]
\begin{tabular}{cl}
\textbf{11.36} & \romline{arjuna uvāca} \\
 & \romline{sthāne hṛṣīkeśa tava prakīrtyā} \\
 & \romline{jagatprahṛṣyatyanurajyate ca |} \\
 & \romline{rakṣāṃsi bhītāni diśo dravanti} \\
 & \romline{sarve namasyanti ca siddhasaṅghāḥ ||}
\end{tabular}
\end{table}

\begin{table}[H]
\begin{tabular}{cl}
\textbf{11.37} & \romline{kasmācca te na nameranmahātman} \\
 & \romline{garīyase brahmaṇo'pyādikartre |} \\
 & \romline{ananta deveśa jagannivāsa} \\
 & \romline{tvamakṣaraṃ sadasattatparaṃ yat ||}
\end{tabular}
\end{table}

\begin{table}[H]
\begin{tabular}{cl}
\textbf{11.38} & \romline{tvamādidevaḥ puruṣaḥ purāṇaḥ} \\
 & \romline{tvamasya viśvasya paraṃ nidhānam |} \\
 & \romline{vettā'si vedyaṃ ca paraṃ ca dhāma} \\
 & \romline{tvayā tataṃ viśvamanantarūpa ||}
\end{tabular}
\end{table}

\begin{table}[H]
\begin{tabular}{cl}
\textbf{11.39} & \romline{vāyuryamo'gnirvaruṇaḥ śaśāṅkaḥ} \\
 & \romline{prajāpatistvaṃ prapitāmahaśca |} \\
 & \romline{namo namaste'stu sahasrakṛtvaḥ} \\
 & \romline{punaśca bhūyo'pi namo namaste ||}
\end{tabular}
\end{table}

\begin{table}[H]
\begin{tabular}{cl}
\textbf{11.40} & \romline{namaḥ purastādatha pṛṣṭhataste} \\
 & \romline{namo'stu te sarvata eva sarva |} \\
 & \romline{anantavīryāmitavikramastvaṃ} \\
 & \romline{sarvaṃ samāpnoṣi tato'si sarvaḥ ||}
\end{tabular}
\end{table}

\begin{table}[H]
\begin{tabular}{cl}
\textbf{11.41} & \romline{sakheti matvā prasabhaṃ yaduktaṃ} \\
 & \romline{he kṛṣṇa he yādava he sakheti |} \\
 & \romline{ajānatā mahimānaṃ tavedaṃ} \\
 & \romline{mayā pramādātpraṇayena vā'pi ||}
\end{tabular}
\end{table}

\begin{table}[H]
\begin{tabular}{cl}
\textbf{11.42} & \romline{yaccāpahāsārthamasatkṛto'si} \\
 & \romline{vihāraśayyāsanabhojaneṣu |} \\
 & \romline{eko'thavāpyacyuta tatsamakṣaṃ} \\
 & \romline{tatkṣāmaye tvāmahamaprameyam ||}
\end{tabular}
\end{table}

\begin{table}[H]
\begin{tabular}{cl}
\textbf{11.43} & \romline{pitāsi lokasya carācarasya} \\
 & \romline{tvamasya pūjyaśca gururgarīyān |} \\
 & \romline{na tvatsamo'styabhyadhikaḥ kuto'nyaḥ} \\
 & \romline{lokatraye'pyapratimaprabhāva ||}
\end{tabular}
\end{table}

\begin{table}[H]
\begin{tabular}{cl}
\textbf{11.44} & \romline{tasmātpraṇamya praṇidhāya kāyaṃ} \\
 & \romline{prasādaye tvāmahamīśamīḍyam |} \\
 & \romline{piteva putrasya sakheva sakhyuḥ} \\
 & \romline{priyaḥ priyāyārhasi deva soḍhum ||}
\end{tabular}
\end{table}

\begin{table}[H]
\begin{tabular}{cl}
\textbf{11.45} & \romline{adṛṣṭapūrvaṃ hṛṣito'smi dṛṣṭvā} \\
 & \romline{bhayena ca pravyathitaṃ mano me |} \\
 & \romline{tadeva me darśaya devarūpaṃ} \\
 & \romline{prasīda deveśa jagannivāsa ||}
\end{tabular}
\end{table}

\begin{table}[H]
\begin{tabular}{cl}
\textbf{11.46} & \romline{kirīṭinaṃ gadinaṃ cakrahastam} \\
 & \romline{icchāmi tvāṃ draṣṭumahaṃ tathaiva |} \\
 & \romline{tenaiva rūpeṇa caturbhujena} \\
 & \romline{sahasrabāho bhava viśvamūrte ||}
\end{tabular}
\end{table}

\begin{table}[H]
\begin{tabular}{cl}
\textbf{11.47} & \romline{śrī bhagavānuvāca} \\
 & \romline{mayā prasannena tavārjunedaṃ} \\
 & \romline{rūpaṃ paraṃ darśitamātmayogāt |} \\
 & \romline{tejomayaṃ viśvamanantamādyaṃ} \\
 & \romline{yanme tvadanyena na dṛṣṭapūrvam ||}
\end{tabular}
\end{table}

\begin{table}[H]
\begin{tabular}{cl}
\textbf{11.48} & \romline{na vedayajñādhyayanairna dānaiḥ} \\
 & \romline{na ca kriyābhirna tapobhirugraiḥ |} \\
 & \romline{evaṃrūpaḥ śakya ahaṃ nṛloke} \\
 & \romline{draṣṭuṃ tvadanyena kurupravīra ||}
\end{tabular}
\end{table}

\begin{table}[H]
\begin{tabular}{cl}
\textbf{11.49} & \romline{mā te vyathā mā ca vimūḍhabhāvaḥ} \\
 & \romline{dṛṣṭvā rūpaṃ ghoramīdṛṅmamedam |} \\
 & \romline{vyapetabhīḥ prītamanāḥ punastvaṃ} \\
 & \romline{tadeva me rūpamidaṃ prapaśya ||}
\end{tabular}
\end{table}

\begin{table}[H]
\begin{tabular}{cl}
\textbf{11.50} & \romline{sañjaya uvāca} \\
 & \romline{ityarjunaṃ vāsudevastathoktvā} \\
 & \romline{svakaṃ rūpaṃ darśayāmāsa bhūyaḥ |} \\
 & \romline{āśvāsayāmāsa ca bhītamenaṃ} \\
 & \romline{bhūtvā punaḥ saumyavapurmahātmā ||}
\end{tabular}
\end{table}

\begin{table}[H]
\begin{tabular}{cl}
\textbf{11.51} & \romline{arjuna uvāca} \\
 & \romline{dṛṣṭvedaṃ mānuṣaṃ rūpaṃ} \\
 & \romline{tava saumyaṃ janārdana |} \\
 & \romline{idānīmasmi saṃvṛttaḥ} \\
 & \romline{sacetāḥ prakṛtiṃ gataḥ ||}
\end{tabular}
\end{table}

\begin{table}[H]
\begin{tabular}{cl}
\textbf{11.52} & \romline{śrī bhagavānuvāca} \\
 & \romline{sudurdarśamidaṃ rūpaṃ} \\
 & \romline{dṛṣṭavānasi yanmama |} \\
 & \romline{devā apyasya rūpasya} \\
 & \romline{nityaṃ darśanakāṅkṣiṇaḥ ||}
\end{tabular}
\end{table}

\begin{table}[H]
\begin{tabular}{cl}
\textbf{11.53} & \romline{nāhaṃ vedairna tapasā} \\
 & \romline{na dānena na cejyayā |} \\
 & \romline{śakya evaṃvidho draṣṭuṃ} \\
 & \romline{dṛṣṭavānasi māṃ yathā ||}
\end{tabular}
\end{table}

\begin{table}[H]
\begin{tabular}{cl}
\textbf{11.54} & \romline{bhaktyā tvananyayā śakyaḥ} \\
 & \romline{ahamevaṃvidho'rjuna |} \\
 & \romline{jñātuṃ draṣṭuṃ ca tattvena} \\
 & \romline{praveṣṭuṃ ca parantapa ||}
\end{tabular}
\end{table}

\begin{table}[H]
\begin{tabular}{cl}
\textbf{11.55} & \romline{matkarmakṛnmatparamaḥ} \\
 & \romline{madbhaktaḥ saṅgavarjitaḥ |} \\
 & \romline{nirvairaḥ sarvabhūteṣu} \\
 & \romline{yaḥ sa māmeti pāṇḍava ||}
\end{tabular}
\end{table}

\begin{table}[H]
\begin{tabular}{cl}
 & \romline{śrīmadbhagavadgītāsu upaniṣatsu} \\
 & \romline{brahmavidyāyāṃ yogaśāstre} \\
 & \romline{śrīkṛṣṇārjuna saṃvāde} \\
 & \romline{viśvarūpasandarśanayogo nāma} \\
 & \romline{ekādaśodhyāyaḥ}
\end{tabular}
\end{table}


% \end{multicols}

\chapter{Bhakti Yoga}
% \begin{multicols}{2}
\subsection*{12.0}
\begin{table}[H]
\begin{tabular}{l}
\natline{ओं श्री परमात्मने नमः} \\
\natline{अथ द्वादशोऽध्यायः} \\
\natline{भक्तियोगः}
\end{tabular}
\end{table}

\subsection*{12.1}
\begin{table}[H]
\begin{tabular}{l}
\natline{अर्जुन उवाच} \\
\natline{एवं सततयुक्ता ये} \\
\natline{भक्तास्त्वां पर्युपासते} \\
\natline{ये चाप्यक्षरमव्यक्तं} \\
\natline{तेषां के योगवित्तमाः}
\end{tabular}
\end{table}

\subsection*{12.2}
\begin{table}[H]
\begin{tabular}{l}
\natline{श्री भगवान् उवाच} \\
\natline{मय्यावेश्य मनो ये मां} \\
\natline{नित्ययुक्ता उपासते} \\
\natline{श्रद्धया परयोपेताः} \\
\natline{ते मे युक्ततमा मताः}
\end{tabular}
\end{table}

\subsection*{12.3}
\begin{table}[H]
\begin{tabular}{l}
\natline{ये त्वक्षरमनिर्देश्यम्} \\
\natline{अव्यक्तं पर्युपासते} \\
\natline{सर्वत्रगमचिन्त्यम् च} \\
\natline{कूटस्थमचलम् ध्रुवं}
\end{tabular}
\end{table}

\subsection*{12.4}
\begin{table}[H]
\begin{tabular}{l}
\natline{सन्नियम्येन्द्रियग्रामं} \\
\natline{सर्वत्र समबुद्धयः} \\
\natline{ते प्राप्नुवन्ति मामेव} \\
\natline{सर्वभूतहिते रताः}
\end{tabular}
\end{table}

\subsection*{12.5}
\begin{table}[H]
\begin{tabular}{l}
\natline{क्लेशोऽधिकतरस्तेषाम्} \\
\natline{अव्यक्तासक्तचेतसाम्} \\
\natline{अव्यक्ता हि गतिर्दुःखं} \\
\natline{देहवद्भिरवाप्यते}
\end{tabular}
\end{table}

\subsection*{12.6}
\begin{table}[H]
\begin{tabular}{l}
\natline{ये तु सर्वाणि कर्माणि} \\
\natline{मयि सन्न्यस्य मत्पराः} \\
\natline{अनन्येनैव योगेन} \\
\natline{मां ध्यायन्त उपासते}
\end{tabular}
\end{table}

\subsection*{12.7}
\begin{table}[H]
\begin{tabular}{l}
\natline{तेषामहं समुद्धर्ता} \\
\natline{मृत्युसंसारसागरात्} \\
\natline{भवामि नcइरात्पार्थ} \\
\natline{मय्यावेशितचेतसाम्}
\end{tabular}
\end{table}

\subsection*{12.8}
\begin{table}[H]
\begin{tabular}{l}
\natline{मय्येव मन आधत्स्व} \\
\natline{मयि बुद्धिं निवेशय} \\
\natline{निवसिष्यसि मय्येव} \\
\natline{अत ऊर्ध्वं न सम्शयः}
\end{tabular}
\end{table}

\subsection*{12.9}
\begin{table}[H]
\begin{tabular}{l}
\natline{अथ चित्तं समाधातुं} \\
\natline{न शक्नोषि मयि स्थिरम्} \\
\natline{अभ्यासयोगेन ततः} \\
\natline{मामिच्छाप्तुं धनञ्जय}
\end{tabular}
\end{table}

\subsection*{12.10}
\begin{table}[H]
\begin{tabular}{l}
\natline{अभ्यासेऽप्यसमर्थोऽसि} \\
\natline{मत्कर्मपरमो भव} \\
\natline{मदर्थमपि कर्माणि} \\
\natline{कुर्वन्सिद्धिमवाप्स्यसि}
\end{tabular}
\end{table}

\subsection*{12.11}
\begin{table}[H]
\begin{tabular}{l}
\natline{अथैतदप्यशक्तोऽसि} \\
\natline{कर्तुं मद्योगमाश्रितः} \\
\natline{सर्वकर्मफलत्यागं} \\
\natline{ततः कुरु यतात्मवान्}
\end{tabular}
\end{table}

\subsection*{12.12}
\begin{table}[H]
\begin{tabular}{l}
\natline{श्रेयो हि ज्ञानमभ्यासात्} \\
\natline{ज्ञानाद्ध्यानं विशिष्यते} \\
\natline{ध्यानात्कर्मफलत्यागः} \\
\natline{त्यागाच्छान्तिरनन्तरम्}
\end{tabular}
\end{table}

\subsection*{12.13}
\begin{table}[H]
\begin{tabular}{l}
\natline{अद्वेष्टा सर्वभूतानां} \\
\natline{मैत्रः करुण एव च} \\
\natline{निर्ममो निरहन्कारः} \\
\natline{समदुःखसुखः क्षमी}
\end{tabular}
\end{table}

\subsection*{12.14}
\begin{table}[H]
\begin{tabular}{l}
\natline{सन्तुष्टः सततं योगी} \\
\natline{यतात्मा दृढनिश्चयः} \\
\natline{मय्यर्पितमनोबुद्धिः} \\
\natline{यो मद्भक्तः स मे प्रियः}
\end{tabular}
\end{table}

\subsection*{12.15}
\begin{table}[H]
\begin{tabular}{l}
\natline{यस्मान्नोद्विजते लोकः} \\
\natline{लोकान्नोद्विजते च यः} \\
\natline{हर्षामर्षभयोद्वेगैः} \\
\natline{मुक्तो यः स च मे प्रियः}
\end{tabular}
\end{table}

\subsection*{12.16}
\begin{table}[H]
\begin{tabular}{l}
\natline{अनपेक्षः शुचिर्दक्षः} \\
\natline{उदासीनो गतव्यथः} \\
\natline{सर्वारम्भपरित्यागी} \\
\natline{यो मद्भक्तः स मे प्रियः}
\end{tabular}
\end{table}

\subsection*{12.17}
\begin{table}[H]
\begin{tabular}{l}
\natline{यो न हृष्यति न द्वेष्टि} \\
\natline{न शोचति न काञ्क्षति} \\
\natline{शुभाशुभपरित्यागी} \\
\natline{भक्तिमान्यः स मे प्रियः}
\end{tabular}
\end{table}

\subsection*{12.18}
\begin{table}[H]
\begin{tabular}{l}
\natline{समः शत्रौ च मित्रे च} \\
\natline{तथा मानापमानयोः} \\
\natline{शीतोष्णसुखदुःखेषु} \\
\natline{समः सञ्गविवर्जितः}
\end{tabular}
\end{table}

\subsection*{12.19}
\begin{table}[H]
\begin{tabular}{l}
\natline{तुल्यनिन्दास्तुतिर्मौनी} \\
\natline{सन्तुष्टो येन केनचित्} \\
\natline{अनिकेतः स्थिरमतिः} \\
\natline{भक्तिमान्मे प्रियो नरः}
\end{tabular}
\end{table}

\subsection*{12.20}
\begin{table}[H]
\begin{tabular}{l}
\natline{ये तु धर्म्यामृतमिदं} \\
\natline{यथोक्तं पर्युपासते} \\
\natline{श्रद्दधाना मत्परमाः} \\
\natline{भक्तास्तेऽतीव मे प्रियाः}
\end{tabular}
\end{table}


% \end{multicols}

\chapter{Kṣetra-Kṣetrajña-Vibhāga Yoga}
% \begin{multicols}{2}
\begin{table}[H]
\begin{tabular}{cl}
 & \natline{శ్రీ పరమాత్మనే నమః} \\
 & \natline{అథ త్రయోదశోఽధ్యాయః} \\
 & \natline{క్షేత్రక్షేత్రజ్ఞవిభాగయోగః}
\end{tabular}
\end{table}

\begin{table}[H]
\begin{tabular}{cl}
\textbf{13.1} & \natline{అర్జున ఉవాచ} \\
 & \natline{ప్రకృతిం పురుషం చైవ} \\
 & \natline{క్షేత్రం క్షేత్రజ్ఞమేవ చ |} \\
 & \natline{ఏతత్ వేదితుమిచ్ఛామి} \\
 & \natline{జ్ఞానం జ్ఞేయం చ కేశవ ||}
\end{tabular}
\end{table}

\begin{table}[H]
\begin{tabular}{cl}
\textbf{13.2} & \natline{శ్రీ భగవానువాచ} \\
 & \natline{ఇదం శరీరం కౌన్తేయ} \\
 & \natline{క్షేత్రమిత్యభిధీయతే |} \\
 & \natline{ఏతద్యో వేత్తి తం ప్రాహుః} \\
 & \natline{క్షేత్రజ్ఞ ఇతి తద్విదః ||}
\end{tabular}
\end{table}

\begin{table}[H]
\begin{tabular}{cl}
\textbf{13.3} & \natline{క్షేత్రజ్ఞం చాపి మాం విద్ధి} \\
 & \natline{సర్వక్షేత్రేషు భారత |} \\
 & \natline{క్షేత్రక్షేత్రజ్ఞయోర్జ్ఞానం} \\
 & \natline{యత్తజ్జ్ఞానం మతం మమ ||}
\end{tabular}
\end{table}

\begin{table}[H]
\begin{tabular}{cl}
\textbf{13.4} & \natline{తత్క్షేత్రం యచ్చ యాదృక్చ} \\
 & \natline{యద్వికారి యతశ్చ యత్ |} \\
 & \natline{స చ యో యత్ప్రభావశ్చ} \\
 & \natline{తత్సమాసేన మే శృణు ||}
\end{tabular}
\end{table}

\begin{table}[H]
\begin{tabular}{cl}
\textbf{13.5} & \natline{ఋషిభిర్బహుధా గీతం} \\
 & \natline{ఛన్దోభిర్వివిధైః పృథక్ |} \\
 & \natline{బ్రహ్మసూత్రపదైశ్చైవ} \\
 & \natline{హేతుమద్భిర్వినిశ్చితైః ||}
\end{tabular}
\end{table}

\begin{table}[H]
\begin{tabular}{cl}
\textbf{13.6} & \natline{మహాభూతాన్యహఙ్కారః} \\
 & \natline{బుద్ధిరవ్యక్తమేవ చ |} \\
 & \natline{ఇన్ద్రియాణి దశైకం చ} \\
 & \natline{పఞ్చ చేన్ద్రియగోచరాః ||}
\end{tabular}
\end{table}

\begin{table}[H]
\begin{tabular}{cl}
\textbf{13.7} & \natline{ఇచ్ఛా ద్వేషః సుఖం దుఃఖం} \\
 & \natline{సఙ్ఘాతశ్చేతనా ధృతిః |} \\
 & \natline{ఏతత్క్షేత్రం సమాసేన} \\
 & \natline{సవికారముదాహృతమ్ ||}
\end{tabular}
\end{table}

\begin{table}[H]
\begin{tabular}{cl}
\textbf{13.8} & \natline{అమానిత్వమదంభిత్వమ్} \\
 & \natline{అహింసా క్షాన్తిరార్జవమ్ |} \\
 & \natline{ఆచార్యోపాసనం శౌచం} \\
 & \natline{స్థైర్యమాత్మవినిగ్రహః ||}
\end{tabular}
\end{table}

\begin{table}[H]
\begin{tabular}{cl}
\textbf{13.9} & \natline{ఇన్ద్రియార్థేషు వైరాగ్యమ్} \\
 & \natline{అనహఙ్కార ఏవ చ |} \\
 & \natline{జన్మమృత్యుజరావ్యాధి} \\
 & \natline{దుఃఖదోషానుదర్శనమ్ ||}
\end{tabular}
\end{table}

\begin{table}[H]
\begin{tabular}{cl}
\textbf{13.10} & \natline{అసక్తిరనభిష్వఙ్గః} \\
 & \natline{పుత్రదారగృహాదిషు |} \\
 & \natline{నిత్యం చ సమచిత్తత్వమ్} \\
 & \natline{ఇష్టానిష్టోపపత్తిషు ||}
\end{tabular}
\end{table}

\begin{table}[H]
\begin{tabular}{cl}
\textbf{13.11} & \natline{మయి చానన్యయోగేన} \\
 & \natline{భక్తిరవ్యభిచారిణీ |} \\
 & \natline{వివిక్తదేశసేవిత్వమ్} \\
 & \natline{అరతిర్జనసంసది ||}
\end{tabular}
\end{table}

\begin{table}[H]
\begin{tabular}{cl}
\textbf{13.12} & \natline{అధ్యాత్మజ్ఞాననిత్యత్వం} \\
 & \natline{తత్త్వజ్ఞానార్థదర్శనమ్ |} \\
 & \natline{ఏతజ్జ్ఞానమితి ప్రోక్తమ్} \\
 & \natline{అజ్ఞానం యదతోఽన్యథా ||}
\end{tabular}
\end{table}

\begin{table}[H]
\begin{tabular}{cl}
\textbf{13.13} & \natline{జ్ఞేయం యత్తత్ప్రవక్ష్యామి} \\
 & \natline{యజ్జ్ఞాత్వాఽమృతమశ్నుతే |} \\
 & \natline{అనాదిమత్పరం బ్రహ్మ} \\
 & \natline{న సత్తన్నాసదుచ్యతే ||}
\end{tabular}
\end{table}

\begin{table}[H]
\begin{tabular}{cl}
\textbf{13.14} & \natline{సర్వతః పాణిపాదం తత్} \\
 & \natline{సర్వతోఽక్షిశిరోముఖమ్ |} \\
 & \natline{సర్వతః శ్రుతిమల్లోకే} \\
 & \natline{సర్వమావృత్య తిష్ఠతి ||}
\end{tabular}
\end{table}

\begin{table}[H]
\begin{tabular}{cl}
\textbf{13.15} & \natline{సర్వేన్ద్రియగుణాభాసం} \\
 & \natline{సర్వేన్ద్రియవివర్జితమ్ |} \\
 & \natline{అసక్తం సర్వభృచ్చైవ} \\
 & \natline{నిర్గుణం గుణభోక్తృ చ ||}
\end{tabular}
\end{table}

\begin{table}[H]
\begin{tabular}{cl}
\textbf{13.16} & \natline{బహిరన్తశ్చ భూతానామ్} \\
 & \natline{అచరం చరమేవ చ |} \\
 & \natline{సూక్ష్మత్వాత్తదవిజ్ఞేయం} \\
 & \natline{దూరస్థం చాన్తికే చ తత్ ||}
\end{tabular}
\end{table}

\begin{table}[H]
\begin{tabular}{cl}
\textbf{13.17} & \natline{అవిభక్తం చ భూతేషు} \\
 & \natline{విభక్తమివ చ స్థితమ్ |} \\
 & \natline{భూతభర్తృ చ తజ్జ్ఞేయం} \\
 & \natline{గ్రసిష్ణు ప్రభవిష్ణు చ ||}
\end{tabular}
\end{table}

\begin{table}[H]
\begin{tabular}{cl}
\textbf{13.18} & \natline{జ్యోతిషామపి తజ్జ్యోతిః} \\
 & \natline{తమసః పరముచ్యతే |} \\
 & \natline{జ్ఞానం జ్ఞేయం జ్ఞానగమ్యం} \\
 & \natline{హృది సర్వస్య విష్ఠితమ్ ||}
\end{tabular}
\end{table}

\begin{table}[H]
\begin{tabular}{cl}
\textbf{13.19} & \natline{ఇతి క్షేత్రం తథా జ్ఞానం} \\
 & \natline{జ్ఞేయం చోక్తం సమాసతః |} \\
 & \natline{మద్భక్త ఏతద్విజ్ఞాయ} \\
 & \natline{మద్భావాయోపపద్యతే ||}
\end{tabular}
\end{table}

\begin{table}[H]
\begin{tabular}{cl}
\textbf{13.20} & \natline{ప్రకృతిం పురుషం చైవ} \\
 & \natline{విద్ధ్యనాదీ ఉభావపి |} \\
 & \natline{వికారాంశ్చ గుణాంశ్చైవ} \\
 & \natline{విద్ధి ప్రకృతిసమ్భవాన్ ||}
\end{tabular}
\end{table}

\begin{table}[H]
\begin{tabular}{cl}
\textbf{13.21} & \natline{కార్యకరణకర్తృత్వే} \\
 & \natline{హేతుః ప్రకృతిరుచ్యతే |} \\
 & \natline{పురుషః సుఖదుఃఖానాం} \\
 & \natline{భోక్తృత్వే హేతురుచ్యతే ||}
\end{tabular}
\end{table}

\begin{table}[H]
\begin{tabular}{cl}
\textbf{13.22} & \natline{పురుషః ప్రకృతిస్థో హి} \\
 & \natline{భుఙ్క్తే ప్రకృతిజాన్గుణాన్ |} \\
 & \natline{కారణం గుణసఙ్గోఽస్య} \\
 & \natline{సదసద్యోనిజన్మసు ||}
\end{tabular}
\end{table}

\begin{table}[H]
\begin{tabular}{cl}
\textbf{13.23} & \natline{ఉపద్రష్టాఽనుమన్తా చ} \\
 & \natline{భర్తా భోక్తా మహేశ్వరః |} \\
 & \natline{పరమాత్మేతి చాప్యుక్తః} \\
 & \natline{దేహేఽస్మిన్పురుషః పరః ||}
\end{tabular}
\end{table}

\begin{table}[H]
\begin{tabular}{cl}
\textbf{13.24} & \natline{య ఏవం వేత్తి పురుషం} \\
 & \natline{ప్రకృతిం చ గుణైః సహ |} \\
 & \natline{సర్వథా వర్తమానోఽపి} \\
 & \natline{న స భూయోఽభిజాయతే ||}
\end{tabular}
\end{table}

\begin{table}[H]
\begin{tabular}{cl}
\textbf{13.25} & \natline{ధ్యానేనాత్మని పశ్యన్తి} \\
 & \natline{కేచిదాత్మానమాత్మనా |} \\
 & \natline{అన్యే సాఙ్ఖ్యేన యోగేన} \\
 & \natline{కర్మయోగేన చాపరే ||}
\end{tabular}
\end{table}

\begin{table}[H]
\begin{tabular}{cl}
\textbf{13.26} & \natline{అన్యే త్వేవమజానన్తః} \\
 & \natline{శ్రుత్వాఽన్యేభ్య ఉపాసతే |} \\
 & \natline{తేఽపి చాతితరన్త్యేవ} \\
 & \natline{మృత్యుం శ్రుతిపరాయణాః ||}
\end{tabular}
\end{table}

\begin{table}[H]
\begin{tabular}{cl}
\textbf{13.27} & \natline{యావత్సఞ్జాయతే కిఞ్చిత్} \\
 & \natline{సత్త్వం స్థావరజఙ్గమమ్ |} \\
 & \natline{క్షేత్రక్షేత్రజ్ఞసంయోగాత్} \\
 & \natline{తద్విద్ధి భరతర్షభ ||}
\end{tabular}
\end{table}

\begin{table}[H]
\begin{tabular}{cl}
\textbf{13.28} & \natline{సమం సర్వేషు భూతేషు} \\
 & \natline{తిష్ఠన్తం పరమేశ్వరమ్ |} \\
 & \natline{వినశ్యత్స్వవినశ్యన్తం} \\
 & \natline{యః పశ్యతి స పశ్యతి ||}
\end{tabular}
\end{table}

\begin{table}[H]
\begin{tabular}{cl}
\textbf{13.29} & \natline{సమం పశ్యన్హి సర్వత్ర} \\
 & \natline{సమవస్థితమీశ్వరమ్ |} \\
 & \natline{న హినస్త్యాత్మనాఽఽత్మానం} \\
 & \natline{తతో యాతి పరాం గతిమ్ ||}
\end{tabular}
\end{table}

\begin{table}[H]
\begin{tabular}{cl}
\textbf{13.30} & \natline{ప్రకృత్యైవ చ కర్మాణి} \\
 & \natline{క్రియమాణాని సర్వశః |} \\
 & \natline{యః పశ్యతి తథాఽఽత్మానమ్} \\
 & \natline{అకర్తారం స పశ్యతి ||}
\end{tabular}
\end{table}

\begin{table}[H]
\begin{tabular}{cl}
\textbf{13.31} & \natline{యదా భూతపృథగ్భావమ్} \\
 & \natline{ఏకస్థమనుపశ్యతి |} \\
 & \natline{తత ఏవ చ విస్తారం} \\
 & \natline{బ్రహ్మ సమ్పద్యతే తదా ||}
\end{tabular}
\end{table}

\begin{table}[H]
\begin{tabular}{cl}
\textbf{13.32} & \natline{అనాదిత్వాన్నిర్గుణత్వాత్} \\
 & \natline{పరమాత్మాయమవ్యయః |} \\
 & \natline{శరీరస్థోఽపి కౌన్తేయ} \\
 & \natline{న కరోతి న లిప్యతే ||}
\end{tabular}
\end{table}

\begin{table}[H]
\begin{tabular}{cl}
\textbf{13.33} & \natline{యథా సర్వగతం సౌక్ష్మ్యాత్} \\
 & \natline{ఆకాశం నోపలిప్యతే |} \\
 & \natline{సర్వత్రావస్థితో దేహే} \\
 & \natline{తథాఽఽత్మా నోపలిప్యతే ||}
\end{tabular}
\end{table}

\begin{table}[H]
\begin{tabular}{cl}
\textbf{13.34} & \natline{యథా ప్రకాశయత్యేకః} \\
 & \natline{కృత్స్నం లోకమిమం రవిః |} \\
 & \natline{క్షేత్రం క్షేత్రీ తథా కృత్స్నం} \\
 & \natline{ప్రకాశయతి భారత ||}
\end{tabular}
\end{table}

\begin{table}[H]
\begin{tabular}{cl}
\textbf{13.35} & \natline{క్షేత్రక్షేత్రజ్ఞయోరేవమ్} \\
 & \natline{అన్తరం జ్ఞానచక్షుషా |} \\
 & \natline{భూతప్రకృతిమోక్షం చ} \\
 & \natline{యే విదుర్యాన్తి తే పరమ్ ||}
\end{tabular}
\end{table}

\begin{table}[H]
\begin{tabular}{cl}
 & \natline{శ్రీమద్భగవద్గీతాసు ఉపనిషత్సు} \\
 & \natline{బ్రహ్మవిద్యాయాం యోగశాస్త్రే} \\
 & \natline{శ్రీకృష్ణార్జున సంవాదే} \\
 & \natline{క్షేత్రక్షేత్రజ్ఞవిభాగయోగో నామ} \\
 & \natline{త్రయోదశోధ్యాయః}
\end{tabular}
\end{table}


% \end{multicols}

\chapter{Guṇatraya-Vibhāga Yoga}
% \begin{multicols}{2}
\begin{table}[H]
\begin{tabular}{cl}
\textbf{14.0} & \natline{ओं श्री परमात्मने नमः} \\
 & \natline{अथ चतुर्दशोऽध्यायः} \\
 & \natline{गुणत्रयविभागयोगः}
\end{tabular}
\end{table}

\begin{table}[H]
\begin{tabular}{cl}
\textbf{14.1} & \natline{श्री भगवानुवाच} \\
 & \natline{परं भूयः प्रवक्ष्यामि} \\
 & \natline{ज्ञानानां ज्ञानमुत्तमम् |} \\
 & \natline{यज्ज्ञात्वा मुनयः सर्वे} \\
 & \natline{परां सिद्धिमितो गताः ||}
\end{tabular}
\end{table}

\begin{table}[H]
\begin{tabular}{cl}
\textbf{14.2} & \natline{इदं ज्ञानमुपाश्रित्य} \\
 & \natline{मम साधर्म्यमागताः |} \\
 & \natline{सर्गेऽपि नोपजायन्ते} \\
 & \natline{प्रलये न व्यथन्ति च ||}
\end{tabular}
\end{table}

\begin{table}[H]
\begin{tabular}{cl}
\textbf{14.3} & \natline{मम योनिर्महद्ब्रह्म} \\
 & \natline{तस्मिन्गर्भं दधाम्यहम् |} \\
 & \natline{सम्भवः सर्वभूतानां} \\
 & \natline{ततो भवति भारत ||}
\end{tabular}
\end{table}

\begin{table}[H]
\begin{tabular}{cl}
\textbf{14.4} & \natline{सर्वयोनिषु कौन्तेय} \\
 & \natline{मूर्तयः सम्भवन्ति याः |} \\
 & \natline{तासां ब्रह्म महद्योनिः} \\
 & \natline{अहं बीजप्रदः पिता ||}
\end{tabular}
\end{table}

\begin{table}[H]
\begin{tabular}{cl}
\textbf{14.5} & \natline{सत्त्वं रजस्तम इति} \\
 & \natline{गुणाः प्रकृतिसम्भवाः |} \\
 & \natline{निबध्नन्ति महाबाहो} \\
 & \natline{देहे देहिनमव्ययम् ||}
\end{tabular}
\end{table}

\begin{table}[H]
\begin{tabular}{cl}
\textbf{14.6} & \natline{तत्र सत्त्वं निर्मलत्वात्} \\
 & \natline{प्रकाशकमनामयम् |} \\
 & \natline{सुखसङ्गेन बध्नाति} \\
 & \natline{ज्ञानसङ्गेन चानघ ||}
\end{tabular}
\end{table}

\begin{table}[H]
\begin{tabular}{cl}
\textbf{14.7} & \natline{रजो रागात्मकं विद्धि} \\
 & \natline{तृष्णासङ्गसमुद्भवम् |} \\
 & \natline{तन्निबध्नाति कौन्तेय} \\
 & \natline{कर्मसङ्गेन देहिनम् ||}
\end{tabular}
\end{table}

\begin{table}[H]
\begin{tabular}{cl}
\textbf{14.8} & \natline{तमस्त्वज्ञानजं विद्धि} \\
 & \natline{मोहनं सर्वदेहिनाम् |} \\
 & \natline{प्रमादालस्यनिद्राभिः} \\
 & \natline{तन्निबध्नाति भारत ||}
\end{tabular}
\end{table}

\begin{table}[H]
\begin{tabular}{cl}
\textbf{14.9} & \natline{सत्त्वं सुखे सञ्जयति} \\
 & \natline{रजः कर्मणि भारत |} \\
 & \natline{ज्ञानमावृत्य तु तमः} \\
 & \natline{प्रमादे सञ्जयत्युत ||}
\end{tabular}
\end{table}

\begin{table}[H]
\begin{tabular}{cl}
\textbf{14.10} & \natline{रजस्तमश्चाभिभूय} \\
 & \natline{सत्त्वं भवति भारत |} \\
 & \natline{रजः सत्त्वं तमश्चैव} \\
 & \natline{तमः सत्त्वं रजस्तथा ||}
\end{tabular}
\end{table}

\begin{table}[H]
\begin{tabular}{cl}
\textbf{14.11} & \natline{सर्वद्वारेषु देहेऽस्मिन्} \\
 & \natline{प्रकाश उपजायते |} \\
 & \natline{ज्ञानं यदा तदा विद्यात्} \\
 & \natline{विवृद्धं सत्त्वमित्युत ||}
\end{tabular}
\end{table}

\begin{table}[H]
\begin{tabular}{cl}
\textbf{14.12} & \natline{लोभः प्रवृत्तिरारम्भः} \\
 & \natline{कर्मणामशमः स्पृहा |} \\
 & \natline{रजस्येतानि जायन्ते} \\
 & \natline{विवृद्धे भरतर्षभ ||}
\end{tabular}
\end{table}

\begin{table}[H]
\begin{tabular}{cl}
\textbf{14.13} & \natline{अप्रकाशोऽप्रवृत्तिश्च} \\
 & \natline{प्रमादो मोह एव च |} \\
 & \natline{तमस्येतानि जायन्ते} \\
 & \natline{विवृद्धे कुरुनन्दन ||}
\end{tabular}
\end{table}

\begin{table}[H]
\begin{tabular}{cl}
\textbf{14.14} & \natline{यदा सत्त्वे प्रवृद्धे तु} \\
 & \natline{प्रलयं याति देहभृत् |} \\
 & \natline{तदोत्तमविदां लोकान्} \\
 & \natline{अमलान्प्रतिपद्यते ||}
\end{tabular}
\end{table}

\begin{table}[H]
\begin{tabular}{cl}
\textbf{14.15} & \natline{रजसि प्रलयं गत्वा} \\
 & \natline{कर्मसङ्गिषु जायते |} \\
 & \natline{तथा प्रलीनस्तमसि} \\
 & \natline{मूढयोनिषु जायते ||}
\end{tabular}
\end{table}

\begin{table}[H]
\begin{tabular}{cl}
\textbf{14.16} & \natline{कर्मणः सुकृतस्याहुः} \\
 & \natline{सात्त्विकं निर्मलं फलम् |} \\
 & \natline{रजसस्तु फलं दुःखम्} \\
 & \natline{अज्ञानं तमसः फलम् ||}
\end{tabular}
\end{table}

\begin{table}[H]
\begin{tabular}{cl}
\textbf{14.17} & \natline{सत्त्वात्सञ्जायते ज्ञानं} \\
 & \natline{रजसो लोभ एव च |} \\
 & \natline{प्रमादमोहौ तमसः} \\
 & \natline{भवतोऽज्ञानमेव च ||}
\end{tabular}
\end{table}

\begin{table}[H]
\begin{tabular}{cl}
\textbf{14.18} & \natline{ऊर्ध्वं गच्छन्ति सत्त्वस्थाः} \\
 & \natline{मध्ये तिष्ठन्ति राजसाः |} \\
 & \natline{जघन्यगुणवृत्तिस्थाः} \\
 & \natline{अधो गच्छन्ति तामसाः ||}
\end{tabular}
\end{table}

\begin{table}[H]
\begin{tabular}{cl}
\textbf{14.19} & \natline{नान्यं गुणेभ्यः कर्तारं} \\
 & \natline{यदा द्रष्टाऽनुपश्यति |} \\
 & \natline{गुणेभ्यश्च परं वेत्ति} \\
 & \natline{मद्भावं सोऽधिगच्छति ||}
\end{tabular}
\end{table}

\begin{table}[H]
\begin{tabular}{cl}
\textbf{14.20} & \natline{गुणानेतानतीत्य त्रीन्} \\
 & \natline{देही देहसमुद्भवान् |} \\
 & \natline{जन्ममृत्युजरादुःखैः} \\
 & \natline{विमुक्तोऽमृतमश्नुते ||}
\end{tabular}
\end{table}

\begin{table}[H]
\begin{tabular}{cl}
\textbf{14.21} & \natline{अर्जुन उवाच} \\
 & \natline{कैर्लिङ्गैस्त्रीन्गुणानेतान्} \\
 & \natline{अतीतो भवति प्रभो |} \\
 & \natline{किमाचारः कथं चैतान्} \\
 & \natline{त्रीन्गुणानतिवर्तते ||}
\end{tabular}
\end{table}

\begin{table}[H]
\begin{tabular}{cl}
\textbf{14.22} & \natline{श्री भगवानुवाच} \\
 & \natline{प्रकाशं च प्रवृत्तिं च} \\
 & \natline{मोहमेव च पाण्डव |} \\
 & \natline{न द्वेष्टि सम्प्रवृत्तानि} \\
 & \natline{न निवृत्तानि काङ्क्षति ||}
\end{tabular}
\end{table}

\begin{table}[H]
\begin{tabular}{cl}
\textbf{14.23} & \natline{उदासीनवदासीनः} \\
 & \natline{गुणैर्यो न विचाल्यते |} \\
 & \natline{गुणा वर्तन्त इत्येव} \\
 & \natline{योऽवतिष्ठति नेङ्गते ||}
\end{tabular}
\end{table}

\begin{table}[H]
\begin{tabular}{cl}
\textbf{14.24} & \natline{समदुःखसुखः स्वस्थः} \\
 & \natline{समलोष्टाश्मकाञ्चनः |} \\
 & \natline{तुल्यप्रियाप्रियो धीरः} \\
 & \natline{तुल्यनिन्दात्मसंस्तुतिः ||}
\end{tabular}
\end{table}

\begin{table}[H]
\begin{tabular}{cl}
\textbf{14.25} & \natline{मानापमानयोस्तुल्यः} \\
 & \natline{तुल्यो मित्रारिपक्षयोः |} \\
 & \natline{सर्वारम्भपरित्यागी} \\
 & \natline{गुणातीतः स उच्यते ||}
\end{tabular}
\end{table}

\begin{table}[H]
\begin{tabular}{cl}
\textbf{14.26} & \natline{मां च योऽव्यभिचारेण} \\
 & \natline{भक्तियोगेन सेवते |} \\
 & \natline{स गुणान्समतीत्यैतान्} \\
 & \natline{ब्रह्मभूयाय कल्पते ||}
\end{tabular}
\end{table}

\begin{table}[H]
\begin{tabular}{cl}
\textbf{14.27} & \natline{ब्रह्मणो हि प्रतिष्ठाऽहम्} \\
 & \natline{अमृतस्याव्ययस्य च |} \\
 & \natline{शाश्वतस्य च धर्मस्य} \\
 & \natline{सुखस्यैकान्तिकस्य च ||}
\end{tabular}
\end{table}


% \end{multicols}

\chapter{Puruṣottama-Prāpti Yoga}
% \begin{multicols}{2}
\subsection*{15.0}
\begin{table}[H]
\begin{tabular}{l}
\natline{ఓం శ్రీ పరమాత్మనే నమః} \\
\natline{అథ పఞ్చదశోఽధ్యాయః} \\
\natline{పురుషోత్తమప్రప్తియోగః}
\end{tabular}
\end{table}

\subsection*{15.1}
\begin{table}[H]
\begin{tabular}{l}
\natline{శ్రీభగవాన్ ఉవాచ} \\
\natline{ఊర్ధ్వమూలమధః* శాఖమ్} \\
\natline{అశ్వత్థం ప్రాహురవ్యయమ్} \\
\natline{ఛన్దాంసి యస్య పర్ణాని} \\
\natline{యస్తం వేద స వేదవిత్}
\end{tabular}
\end{table}

\subsection*{15.2}
\begin{table}[H]
\begin{tabular}{l}
\natline{అధశ్చోర్ధ్వం ప్రసృతాస్తస్య శాఖాః} \\
\natline{గుణప్రవృద్ధా విషయప్రవాలాః} \\
\natline{అధశ్చ మూలాన్యనుసన్తతాని} \\
\natline{కర్మానుబన్ధీని మనుష్యలోకే}
\end{tabular}
\end{table}

\subsection*{15.3}
\begin{table}[H]
\begin{tabular}{l}
\natline{న రూపమస్యేహ తథోపలభ్యతే} \\
\natline{నాన్తో న చాదిర్న చ సంప్రతిష్ఠా} \\
\natline{అశ్వత్థమేనం సువిరూఢమూలమ్} \\
\natline{అసఙ్గశస్త్రేణ దృఢేన ఛిత్త్వా}
\end{tabular}
\end{table}

\subsection*{15.4}
\begin{table}[H]
\begin{tabular}{l}
\natline{తతః పదం తత్పరిమార్గితవ్యం} \\
\natline{యస్మిన్గతా న నివర్తన్తి భూయః} \\
\natline{తమేవ చాద్యం పురుషం ప్రపద్యే} \\
\natline{యతః ప్రవృత్తిః ప్రసృతా పురాణీ}
\end{tabular}
\end{table}

\subsection*{15.5}
\begin{table}[H]
\begin{tabular}{l}
\natline{నిర్మానమోహా జితసఙ్గదోషాః} \\
\natline{అధ్యాత్మనిత్యా వినివృత్తకామాః} \\
\natline{ద్వన్ద్వైర్విముక్తాః సుఖదుఃఖ సఞ్జ్ఞైః} \\
\natline{గచ్ఛన్త్యమూఢాః పదమవ్యయం తత్}
\end{tabular}
\end{table}

\subsection*{15.6}
\begin{table}[H]
\begin{tabular}{l}
\natline{న తద్భాసయతే సూర్యః} \\
\natline{న శశాఙ్కో న పావకః} \\
\natline{యద్గత్వా న నివర్తన్తే} \\
\natline{తద్ధామ పరమం మమ}
\end{tabular}
\end{table}

\subsection*{15.7}
\begin{table}[H]
\begin{tabular}{l}
\natline{మమైవాంశో జీవలోకే} \\
\natline{జీవభూతః సనాతనః} \\
\natline{మనః షష్ఠానీన్ద్రియాణి} \\
\natline{ప్రకృతిస్థాని కర్షతి}
\end{tabular}
\end{table}

\subsection*{15.8}
\begin{table}[H]
\begin{tabular}{l}
\natline{శరీరం యదవాప్నోతి} \\
\natline{యcచాప్యుత్క్రామతీశ్వరః} \\
\natline{గృహీత్వైతాని సంయాతి} \\
\natline{వాయుర్గన్ధానివాశయాత్}
\end{tabular}
\end{table}

\subsection*{15.9}
\begin{table}[H]
\begin{tabular}{l}
\natline{శ్రోత్రం చక్షుః స్పర్శనం చ} \\
\natline{రసనం ఘ్రాణమేవ చ} \\
\natline{అధిష్ఠాయ మనశ్చాయం} \\
\natline{విషయానుపసేవతే}
\end{tabular}
\end{table}

\subsection*{15.10}
\begin{table}[H]
\begin{tabular}{l}
\natline{ఉత్క్రామన్తం స్థితం వాఽపి} \\
\natline{భుఞ్జానం వా గుణాన్వితమ్} \\
\natline{విమూఢా నానుపశ్యన్తి} \\
\natline{పశ్యన్తి జ్ñఆనచక్షుషః}
\end{tabular}
\end{table}


% \end{multicols}

\chapter{Daivāsura-Sampad-Vibhāga Yoga}
% \begin{multicols}{2}
\begin{table}[H]
\begin{tabular}{cl}
\textbf{16.0} & \romline{oṃ śrī paramātmane namaḥ} \\
 & \romline{atha ṣodaśo'dhyāyaḥ} \\
 & \romline{daivāsura-sampad-vibhāga-yogaḥ}
\end{tabular}
\end{table}

\begin{table}[H]
\begin{tabular}{cl}
\textbf{16.1} & \romline{śrī bhagavānuvāca} \\
 & \romline{abhayaṃ sattvasaṃśuddhiḥ} \\
 & \romline{jñānayogavyavasthitiḥ |} \\
 & \romline{dānaṃ damaśca yajñaśca} \\
 & \romline{svādhyāyastapa ārjavam ||}
\end{tabular}
\end{table}

\begin{table}[H]
\begin{tabular}{cl}
\textbf{16.2} & \romline{ahiṃsā satyamakrodhaḥ} \\
 & \romline{tyāgaḥ śāntirapaiśunam |} \\
 & \romline{dayā bhūteṣvaloluptvaṃ} \\
 & \romline{mārdavaṃ hrīracāpalam ||}
\end{tabular}
\end{table}

\begin{table}[H]
\begin{tabular}{cl}
\textbf{16.3} & \romline{tejaḥ kṣamā dhṛtiḥ śaucam} \\
 & \romline{adroho nātimānitā |} \\
 & \romline{bhavanti sampadaṃ daivīm} \\
 & \romline{abhijātasya bhārata ||}
\end{tabular}
\end{table}

\begin{table}[H]
\begin{tabular}{cl}
\textbf{16.4} & \romline{dambho darpo'bhimānaśca} \\
 & \romline{krodhaḥ pāruṣyameva ca |} \\
 & \romline{ajñānaṃ cābhijātasya} \\
 & \romline{pārtha sampadamāsurīm ||}
\end{tabular}
\end{table}

\begin{table}[H]
\begin{tabular}{cl}
\textbf{16.5} & \romline{daivī sampadvimokṣāya} \\
 & \romline{nibandhāyāsurī matā |} \\
 & \romline{mā śucaḥ sampadaṃ daivīm} \\
 & \romline{abhijāto'si pāṇḍava ||}
\end{tabular}
\end{table}

\begin{table}[H]
\begin{tabular}{cl}
\textbf{16.6} & \romline{dvau bhūtasargau loke'smin} \\
 & \romline{daiva āsura eva ca |} \\
 & \romline{daivo vistaraśaḥ proktaḥ} \\
 & \romline{āsuraṃ pārtha me śṛṇu ||}
\end{tabular}
\end{table}

\begin{table}[H]
\begin{tabular}{cl}
\textbf{16.7} & \romline{pravṛttiṃ ca nivṛttiṃ ca} \\
 & \romline{janā na vidurāsurāḥ |} \\
 & \romline{na śaucaṃ nāpi cācāraḥ} \\
 & \romline{na satyaṃ teṣu vidyate ||}
\end{tabular}
\end{table}

\begin{table}[H]
\begin{tabular}{cl}
\textbf{16.8} & \romline{asatyamapratiṣṭhaṃ te} \\
 & \romline{jagadāhuranīśvaram |} \\
 & \romline{aparasparasambhūtaṃ} \\
 & \romline{kimanyatkāmahaitukam ||}
\end{tabular}
\end{table}

\begin{table}[H]
\begin{tabular}{cl}
\textbf{16.9} & \romline{etāṃ dṛṣṭimavaṣṭabhya} \\
 & \romline{naṣṭātmāno'lpa-buddhayaḥ |} \\
 & \romline{prabhavantyugra-karmāṇaḥ} \\
 & \romline{kṣayāya jagato'hitāḥ ||}
\end{tabular}
\end{table}

\begin{table}[H]
\begin{tabular}{cl}
\textbf{16.10} & \romline{kāma-māśritya duṣpūraṃ} \\
 & \romline{dambha-māna-madānvitāḥ |} \\
 & \romline{mohād-gṛhītvā-sadgrāhān} \\
 & \romline{pravartante'śucivratāḥ ||}
\end{tabular}
\end{table}

\begin{table}[H]
\begin{tabular}{cl}
\textbf{16.11} & \romline{cintām-aparimeyāṃ ca} \\
 & \romline{pralayāntām-upāśritāḥ |} \\
 & \romline{kāmopabhoga-paramāḥ} \\
 & \romline{etāvaditi niścitāḥ ||}
\end{tabular}
\end{table}

\begin{table}[H]
\begin{tabular}{cl}
\textbf{16.12} & \romline{āśā-pāśa-śatairbaddhāḥ} \\
 & \romline{kāma-krodha-parāyaṇāḥ |} \\
 & \romline{īhante kāma-bhogārtham} \\
 & \romline{anyāyenārtha-sañcayān ||}
\end{tabular}
\end{table}

\begin{table}[H]
\begin{tabular}{cl}
\textbf{16.13} & \romline{idamadya mayā labdham} \\
 & \romline{imaṃ prāpsye manoratham |} \\
 & \romline{idamastīdamapi me} \\
 & \romline{bhaviṣyati punardhanam ||}
\end{tabular}
\end{table}

\begin{table}[H]
\begin{tabular}{cl}
\textbf{16.14} & \romline{asau mayā hataḥ śatṛḥ} \\
 & \romline{haniṣye cāparānapi |} \\
 & \romline{īśvaro'hamahaṃ bhogī} \\
 & \romline{siddho'haṃ balavānsukhī ||}
\end{tabular}
\end{table}

\begin{table}[H]
\begin{tabular}{cl}
\textbf{16.15} & \romline{āḍhyo'bhijanavānasmi} \\
 & \romline{ko'nyo'sti sadṛśo mayā |} \\
 & \romline{yakṣye dāsyāmi modiṣye} \\
 & \romline{ityajñānavimohitāḥ ||}
\end{tabular}
\end{table}

\begin{table}[H]
\begin{tabular}{cl}
\textbf{16.16} & \romline{aneka-citta-vibhrāntāḥ} \\
 & \romline{moha-jāla-samāvṛtāḥ |} \\
 & \romline{prasaktāḥ kāma-bhogeṣu} \\
 & \romline{patanti narake'śucau ||}
\end{tabular}
\end{table}

\begin{table}[H]
\begin{tabular}{cl}
\textbf{16.17} & \romline{ātmasambhāvitāḥ stabdhāḥ} \\
 & \romline{dhanamānamadānvitāḥ |} \\
 & \romline{yajante nāmayajñaiste} \\
 & \romline{dambhenāvidhipūrvakam ||}
\end{tabular}
\end{table}

\begin{table}[H]
\begin{tabular}{cl}
\textbf{16.18} & \romline{ahaṅkāraṃ balaṃ darpaṃ} \\
 & \romline{kāmaṃ krodhaṃ ca saṃśritāḥ |} \\
 & \romline{māmātmaparadeheṣu} \\
 & \romline{pradviṣanto'bhyasūyakāḥ ||}
\end{tabular}
\end{table}

\begin{table}[H]
\begin{tabular}{cl}
\textbf{16.19} & \romline{tānahaṃ dviṣataḥ krūrān} \\
 & \romline{saṃsāreṣu narādhamān |} \\
 & \romline{kṣipā-myajasra-maśubhān} \\
 & \romline{āsurīṣveva yoniṣu ||}
\end{tabular}
\end{table}

\begin{table}[H]
\begin{tabular}{cl}
\textbf{16.20} & \romline{āsurīṃ yonimāpannāḥ} \\
 & \romline{mūḍhā janmani janmani |} \\
 & \romline{māmaprāpyaiva kaunteya} \\
 & \romline{tato yāntyadhamāṃ gatim ||}
\end{tabular}
\end{table}

\begin{table}[H]
\begin{tabular}{cl}
\textbf{16.21} & \romline{trividhaṃ narakasyedaṃ} \\
 & \romline{dvāraṃ nāśanamātmanaḥ |} \\
 & \romline{kāmaḥ krodhastathā lobhaḥ} \\
 & \romline{tasmādetattrayaṃ tyajet ||}
\end{tabular}
\end{table}

\begin{table}[H]
\begin{tabular}{cl}
\textbf{16.22} & \romline{etairvimuktaḥ kaunteya} \\
 & \romline{tamodvāraistribhirnaraḥ |} \\
 & \romline{ācaratyātmanaḥ śreyaḥ} \\
 & \romline{tato yāti parāṃ gatim ||}
\end{tabular}
\end{table}

\begin{table}[H]
\begin{tabular}{cl}
\textbf{16.23} & \romline{yaḥ śāstra-vidhimutsṛjya} \\
 & \romline{vartate kāmakārataḥ |} \\
 & \romline{na sa siddhimavāpnoti} \\
 & \romline{na sukhaṃ na parāṃ gatim ||}
\end{tabular}
\end{table}

\begin{table}[H]
\begin{tabular}{cl}
\textbf{16.24} & \romline{tasmācchāstraṃ pramāṇaṃ te} \\
 & \romline{kāryākārya-vyavasthitau |} \\
 & \romline{jñātvā śāstra-vidhānoktaṃ} \\
 & \romline{karma kartum-ihārhasi ||}
\end{tabular}
\end{table}


% \end{multicols}

\chapter{Śraddhātraya-Vibhāga Yoga}
% \begin{multicols}{2}
\begin{table}[H]
\begin{tabular}{cl}
 & \natline{శ్రీ పరమాత్మనే నమః} \\
 & \natline{అథ సప్తదశోఽధ్యాయః} \\
 & \natline{శ్రద్ధాత్రయవిభాగయోగః}
\end{tabular}
\end{table}

\begin{table}[H]
\begin{tabular}{cl}
\textbf{17.1} & \natline{అర్జున ఉవాచ} \\
 & \natline{యే శాస్త్రవిధిముత్సృజ్య} \\
 & \natline{యజన్తే శ్రద్ధయాన్వితాః |} \\
 & \natline{తేషాం నిష్ఠా తు కా కృష్ణ} \\
 & \natline{సత్త్వమాహో రజస్తమః ||}
\end{tabular}
\end{table}

\begin{table}[H]
\begin{tabular}{cl}
\textbf{17.2} & \natline{శ్రీ భగవానువాచ} \\
 & \natline{త్రివిధా భవతి శ్రద్ధా} \\
 & \natline{దేహినాం సా స్వభావజా |} \\
 & \natline{సాత్త్వికీ రాజసీ చైవ} \\
 & \natline{తామసీ చేతి తాం శృణు ||}
\end{tabular}
\end{table}

\begin{table}[H]
\begin{tabular}{cl}
\textbf{17.3} & \natline{సత్త్వానురూపా సర్వస్య} \\
 & \natline{శ్రద్ధా భవతి భారత |} \\
 & \natline{శ్రద్ధామయోఽయం పురుషః} \\
 & \natline{యో యచ్చ్రద్ధః స ఏవ సః ||}
\end{tabular}
\end{table}

\begin{table}[H]
\begin{tabular}{cl}
\textbf{17.4} & \natline{యజన్తే సాత్త్వికా దేవాన్} \\
 & \natline{యక్షరక్షాంసి రాజసాః |} \\
 & \natline{ప్రేతాన్ భూతగణాంశ్చాన్యే} \\
 & \natline{యజన్తే తామసా జనాః ||}
\end{tabular}
\end{table}

\begin{table}[H]
\begin{tabular}{cl}
\textbf{17.5} & \natline{అశాస్త్రవిహితం ఘోరం} \\
 & \natline{తప్యన్తే యే తపో జనాః |} \\
 & \natline{దమ్భాహన్కారసమ్యుక్తాః} \\
 & \natline{కామరాగబలాన్వితాః ||}
\end{tabular}
\end{table}

\begin{table}[H]
\begin{tabular}{cl}
\textbf{17.6} & \natline{కర్శయన్తః శరీరస్థం} \\
 & \natline{భూతగ్రామమచేతసః |} \\
 & \natline{మాం చైవాన్తః శరీరస్థం} \\
 & \natline{తాన్ విద్ధ్యాసురనిశ్చయాన్ ||}
\end{tabular}
\end{table}

\begin{table}[H]
\begin{tabular}{cl}
\textbf{17.7} & \natline{ఆహారస్త్వపి సర్వస్య} \\
 & \natline{త్రివిధో భవతి ప్రియః |} \\
 & \natline{యజ్ఞస్తపస్తథా దానం} \\
 & \natline{తేషాం భేదమిమమ్ శృణు ||}
\end{tabular}
\end{table}

\begin{table}[H]
\begin{tabular}{cl}
\textbf{17.8} & \natline{ఆయుస్సత్త్వబలారోగ్య} \\
 & \natline{సుఖప్రీతివివర్ధనాః |} \\
 & \natline{రస్యాః స్నిగ్ధాః స్థిరా హృద్యాః} \\
 & \natline{ఆహారాః సాత్త్వికప్రియాః ||}
\end{tabular}
\end{table}

\begin{table}[H]
\begin{tabular}{cl}
\textbf{17.9} & \natline{కట్వమ్లలవణాత్యుష్ణ} \\
 & \natline{తీక్ష్ణరూక్షవిదాహినః |} \\
 & \natline{ఆహారా రాజసస్యేష్టాః} \\
 & \natline{దుఃఖశోకామయప్రదాః ||}
\end{tabular}
\end{table}

\begin{table}[H]
\begin{tabular}{cl}
\textbf{17.10} & \natline{యాతయామం గతరసం} \\
 & \natline{పూతి పర్యుషితం చ యత్ |} \\
 & \natline{ఉచ్ఛిష్టమపి చామేధ్యం} \\
 & \natline{భోజనం తామసప్రియమ్ ||}
\end{tabular}
\end{table}

\begin{table}[H]
\begin{tabular}{cl}
\textbf{17.11} & \natline{అఫలాకాఙ్క్షిభిర్యజ్ఞః} \\
 & \natline{విధిదృష్టో య ఇజ్యతే |} \\
 & \natline{యష్టవ్యమేవేతి మనః} \\
 & \natline{సమాధాయ స సాత్త్వికః ||}
\end{tabular}
\end{table}

\begin{table}[H]
\begin{tabular}{cl}
\textbf{17.12} & \natline{అభిసన్ధాయ తు ఫలం} \\
 & \natline{దమ్భార్థమపి చైవ యత్ |} \\
 & \natline{ఇజ్యతే భరతశ్రేష్ఠ} \\
 & \natline{తం యజ్ఞం విద్ధి రాజసమ్ ||}
\end{tabular}
\end{table}

\begin{table}[H]
\begin{tabular}{cl}
\textbf{17.13} & \natline{విధిహీనమసృష్టాన్నం} \\
 & \natline{మన్త్రహీనమదక్షిణమ్ |} \\
 & \natline{శ్రద్ధావిరహితం యజ్ఞం} \\
 & \natline{తామసం పరిచక్షతే ||}
\end{tabular}
\end{table}

\begin{table}[H]
\begin{tabular}{cl}
\textbf{17.14} & \natline{దేవద్విజగురుప్రాజ్ఞ} \\
 & \natline{ప్ūజనం శౌచమార్జవమ్ |} \\
 & \natline{బ్రహ్మచర్యమహింసా చ} \\
 & \natline{శారీరం తప ఉచ్యతే ||}
\end{tabular}
\end{table}

\begin{table}[H]
\begin{tabular}{cl}
\textbf{17.15} & \natline{అనుద్వేగకరం వాక్యం} \\
 & \natline{సత్యం ప్రియహితం చ యత్ |} \\
 & \natline{స్వాధ్యాయాభ్యసనం చైవ} \\
 & \natline{వాఙ్మయం తప ఉచ్యతే ||}
\end{tabular}
\end{table}

\begin{table}[H]
\begin{tabular}{cl}
\textbf{17.16} & \natline{మనః ప్రసాదః సౌమ్యత్వం} \\
 & \natline{మౌనమాత్మవినిగ్రహః |} \\
 & \natline{భావసంశుద్ధిరిత్యేతత్} \\
 & \natline{తపో మానసముచ్యతే ||}
\end{tabular}
\end{table}

\begin{table}[H]
\begin{tabular}{cl}
\textbf{17.17} & \natline{శ్రద్ధయా పరయా తప్తం} \\
 & \natline{తపస్తత్ త్రివిధం నరైః |} \\
 & \natline{అఫలాకాఙ్క్షిభిర్యుక్తైః} \\
 & \natline{సత్త్వికం పరిచక్షతే ||}
\end{tabular}
\end{table}

\begin{table}[H]
\begin{tabular}{cl}
\textbf{17.18} & \natline{సత్కారమానపూజార్థం} \\
 & \natline{తపో దమ్భేన చైవ యత్ |} \\
 & \natline{క్రియతే తదిహ ప్రోక్తం} \\
 & \natline{రాజసం చలమధ్రువమ్ ||}
\end{tabular}
\end{table}

\begin{table}[H]
\begin{tabular}{cl}
\textbf{17.19} & \natline{మూఢగ్రాహేణ్ā́త్మనో యత్} \\
 & \natline{పీడయా క్రియతే తపః |} \\
 & \natline{పర్రస్యోత్సాదనార్థం వా} \\
 & \natline{తత్తామసముదాహృతమ్ ||}
\end{tabular}
\end{table}

\begin{table}[H]
\begin{tabular}{cl}
\textbf{17.20} & \natline{దాతవ్యమితి యద్దానం} \\
 & \natline{దీయతేఽనుపకారిణే |} \\
 & \natline{దేశే కాలే చ పాత్రే చ} \\
 & \natline{తద్దానం సాత్త్వికం స్మృతమ్ ||}
\end{tabular}
\end{table}

\begin{table}[H]
\begin{tabular}{cl}
\textbf{17.21} & \natline{యత్తు ప్రత్యుపకారార్థం} \\
 & \natline{ఫలముద్దిశ్య వా పునః |} \\
 & \natline{దీయతే చ పరిక్లిష్టం} \\
 & \natline{తద్దానం రాజసం స్మృతమ్ ||}
\end{tabular}
\end{table}

\begin{table}[H]
\begin{tabular}{cl}
\textbf{17.22} & \natline{అదేశకాలే యద్దానమ్} \\
 & \natline{అపాత్రేభ్యశ్చ దీయతే |} \\
 & \natline{అసత్కృతమవజ్ఞాతం} \\
 & \natline{తత్తామసముదాహృతమ్ ||}
\end{tabular}
\end{table}

\begin{table}[H]
\begin{tabular}{cl}
\textbf{17.23} & \natline{ఓం తత్సదితి నిర్దేశః} \\
 & \natline{బ్రహ్మణస్త్రివిధః స్మృతః |} \\
 & \natline{బ్రాహ్మణాస్తేన వేదాశ్చ} \\
 & \natline{యజ్ఞాశ్చ విహితాః పురా ||}
\end{tabular}
\end{table}

\begin{table}[H]
\begin{tabular}{cl}
\textbf{17.24} & \natline{తస్మాదోమిత్యుదాహృత్య} \\
 & \natline{యజ్ఞదానతపఃక్రియాః |} \\
 & \natline{ప్రవర్తన్తే విధానోక్తాః} \\
 & \natline{సతతం బ్రహ్మవాదినామ్ ||}
\end{tabular}
\end{table}

\begin{table}[H]
\begin{tabular}{cl}
\textbf{17.25} & \natline{తదిత్యనభిసన్ధాయ} \\
 & \natline{ఫలం యజ్ఞతపఃక్రియాః |} \\
 & \natline{దానక్రియాశ్చ వివిధాః} \\
 & \natline{క్రియన్తే మోక్షకాఙ్క్షిభిః ||}
\end{tabular}
\end{table}

\begin{table}[H]
\begin{tabular}{cl}
\textbf{17.26} & \natline{సద్భావే సాధుభావే చ} \\
 & \natline{సదిత్యేతత్ప్రయుజ్యతే |} \\
 & \natline{ప్రశస్తే కర్మణి తథా} \\
 & \natline{సచ్ఛబ్దః పార్థ యుజ్యతే ||}
\end{tabular}
\end{table}

\begin{table}[H]
\begin{tabular}{cl}
\textbf{17.27} & \natline{యజ్ఞే తపసి దానే చ} \\
 & \natline{స్థితిః సదితి చోచ్యతే |} \\
 & \natline{కర్మ చైవ తదర్థీయం} \\
 & \natline{సదిత్యేవాభిధీయతే ||}
\end{tabular}
\end{table}

\begin{table}[H]
\begin{tabular}{cl}
\textbf{17.28} & \natline{అశ్రద్ధయా హుతం దత్తం} \\
 & \natline{తపస్తప్తం కృతం చ యత్ |} \\
 & \natline{అసదిత్యుచ్యతే పార్థ} \\
 & \natline{న చ తత్ప్రేత్య నో ఇహ ||}
\end{tabular}
\end{table}

\begin{table}[H]
\begin{tabular}{cl}
 & \natline{శ్రీమద్భగవద్గీతాసు ఉపనిషత్సు} \\
 & \natline{బ్రహ్మవిద్యాయాం యోగశాస్త్రే} \\
 & \natline{శ్రీకృష్ణార్జున సంవాదే} \\
 & \natline{శ్రద్ధాత్రయవిభాగయోగో నామ} \\
 & \natline{సప్తదశోధ్యాయః}
\end{tabular}
\end{table}


% \end{multicols}

\chapter{Mokṣa-Sannyāsa Yoga}
% \begin{multicols}{2}
\begin{table}[H]
\begin{tabular}{cl}
\textbf{18.0} & \natline{ओं श्री परमात्मने नमः} \\
 & \natline{अथ अष्टादशोऽध्यायः} \\
 & \natline{मोक्षसन्न्यास योगः}
\end{tabular}
\end{table}

\begin{table}[H]
\begin{tabular}{cl}
\textbf{18.1} & \natline{अर्जुन उवाच} \\
 & \natline{सन्न्यासस्य महाबाहो} \\
 & \natline{तत्त्वमिच्छामि वेदितुम् |} \\
 & \natline{त्यागस्य च हृषिकेश} \\
 & \natline{पृथक्केशिनिषूदन ||}
\end{tabular}
\end{table}

\begin{table}[H]
\begin{tabular}{cl}
\textbf{18.2} & \natline{श्री भगवनुवाच} \\
 & \natline{काम्यानां कर्मणां न्यासं} \\
 & \natline{सन्न्यासं कवयो विदुः |} \\
 & \natline{सर्वकर्मफलत्यागं} \\
 & \natline{प्राहुस्त्यागं विचक्षणाः ||}
\end{tabular}
\end{table}

\begin{table}[H]
\begin{tabular}{cl}
\textbf{18.3} & \natline{त्याज्यं दोषवदित्येके} \\
 & \natline{कर्म प्राहुर्मनीषिणः |} \\
 & \natline{यज्ञदानतपः कर्म} \\
 & \natline{न त्याज्यमिति चापरे ||}
\end{tabular}
\end{table}

\begin{table}[H]
\begin{tabular}{cl}
\textbf{18.4} & \natline{निश्चयं शृणु मे तत्र} \\
 & \natline{त्यागे भरतसत्तम |} \\
 & \natline{त्यागो हि पुरुषव्याघ्र} \\
 & \natline{त्रिविधः सम्प्रकीर्तितः ||}
\end{tabular}
\end{table}

\begin{table}[H]
\begin{tabular}{cl}
\textbf{18.5} & \natline{यज्ञदानतपःकर्म} \\
 & \natline{न त्याज्यं कार्यमेव तत् |} \\
 & \natline{यज्ञो दानं तपश्चैव} \\
 & \natline{पावनानि मनीषिणाम् ||}
\end{tabular}
\end{table}

\begin{table}[H]
\begin{tabular}{cl}
\textbf{18.6} & \natline{एतान्यपि तु कर्माणि} \\
 & \natline{सङ्गं त्यक्त्वा फलानि च |} \\
 & \natline{कर्तव्यानीति मे पार्थ} \\
 & \natline{निश्चितं मतमुत्तमम् ||}
\end{tabular}
\end{table}

\begin{table}[H]
\begin{tabular}{cl}
\textbf{18.7} & \natline{नियतस्य तु सन्न्यासः} \\
 & \natline{कर्मणो नोपपद्यते |} \\
 & \natline{मोहात्तस्य परित्यागः} \\
 & \natline{तामसः परिकीर्तितः ||}
\end{tabular}
\end{table}

\begin{table}[H]
\begin{tabular}{cl}
\textbf{18.8} & \natline{दुःखमित्येव यत्कर्म} \\
 & \natline{कायक्लेशभयात्त्यजेत् |} \\
 & \natline{स कृत्वा राजसं त्यागं} \\
 & \natline{नैव त्यागफलं लभेत् ||}
\end{tabular}
\end{table}

\begin{table}[H]
\begin{tabular}{cl}
\textbf{18.9} & \natline{कार्यमित्येव यत्कर्म} \\
 & \natline{नियतं क्रियतेऽर्जुन |} \\
 & \natline{सङ्गं त्यक्त्वा फलं चैव} \\
 & \natline{स त्यागः सात्त्विको मतः ||}
\end{tabular}
\end{table}

\begin{table}[H]
\begin{tabular}{cl}
\textbf{18.10} & \natline{न द्वेष्ट्यकुशलं कर्म} \\
 & \natline{कुशले नानुषज्जते |} \\
 & \natline{त्यागी सत्त्वसमाविष्टः} \\
 & \natline{मेधावी छिन्नसंशयः ||}
\end{tabular}
\end{table}

\begin{table}[H]
\begin{tabular}{cl}
\textbf{18.11} & \natline{न हि देहभृता शक्यं} \\
 & \natline{त्यक्तुं कर्माण्यशेषतः |} \\
 & \natline{यस्तु कर्मफलत्यागी} \\
 & \natline{स त्यागीत्यभिधीयते ||}
\end{tabular}
\end{table}

\begin{table}[H]
\begin{tabular}{cl}
\textbf{18.12} & \natline{अनिष्टमिष्टं मिश्रं च} \\
 & \natline{त्रिविधं कर्मणः फलम् |} \\
 & \natline{भवत्यत्यागिनां प्रेत्य} \\
 & \natline{न तु सन्न्यासिनां क्वचित् ||}
\end{tabular}
\end{table}

\begin{table}[H]
\begin{tabular}{cl}
\textbf{18.13} & \natline{पञ्चैतानि महाबाहो} \\
 & \natline{कारणानि निबोध मे |} \\
 & \natline{साङ्ख्ये कृतान्ते प्रोक्तानि} \\
 & \natline{सिद्धये सर्वकर्मणाम् ||}
\end{tabular}
\end{table}

\begin{table}[H]
\begin{tabular}{cl}
\textbf{18.14} & \natline{अधिष्ठानं तथा कर्ता} \\
 & \natline{करणं च पृथग्विधम् |} \\
 & \natline{विविधाश्च पृथक्चेष्टाः} \\
 & \natline{दैवं चैवात्र पञ्चमम् ||}
\end{tabular}
\end{table}

\begin{table}[H]
\begin{tabular}{cl}
\textbf{18.15} & \natline{शरीरवाङ्मनोभिर्यत्} \\
 & \natline{कर्म प्रारभते नरः |} \\
 & \natline{न्याय्यं वा विपरीतं वा} \\
 & \natline{पञ्चैते तस्य हेतवः ||}
\end{tabular}
\end{table}

\begin{table}[H]
\begin{tabular}{cl}
\textbf{18.16} & \natline{तत्रैवं सति कर्तारम्} \\
 & \natline{आत्मानं केवलं तु यः |} \\
 & \natline{पश्यत्यकृतबुद्धित्वात्} \\
 & \natline{न स पश्यति दुर्मतिः ||}
\end{tabular}
\end{table}

\begin{table}[H]
\begin{tabular}{cl}
\textbf{18.17} & \natline{यस्य नाहङ्कृतो भावः} \\
 & \natline{बुद्धिर्यस्य न लिप्यते |} \\
 & \natline{हत्वाऽपि स इमाल्लोकान्} \\
 & \natline{न हन्ति न निबध्यते ||}
\end{tabular}
\end{table}

\begin{table}[H]
\begin{tabular}{cl}
\textbf{18.18} & \natline{ज्ञानं ज्ञेयं परिज्ञाता} \\
 & \natline{त्रिविधा कर्मचोदना |} \\
 & \natline{करणं कर्म कर्तेति} \\
 & \natline{त्रिविधः कर्मसङ्ग्रहः ||}
\end{tabular}
\end{table}

\begin{table}[H]
\begin{tabular}{cl}
\textbf{18.19} & \natline{ज्ञानं कर्म च कर्ता च} \\
 & \natline{त्रिधैव गुणभेदतः |} \\
 & \natline{प्रोच्यते गुणसङ्ख्याने} \\
 & \natline{यथावच्छृणु तान्यपि ||}
\end{tabular}
\end{table}

\begin{table}[H]
\begin{tabular}{cl}
\textbf{18.20} & \natline{सर्वभूतेषु येनैकं} \\
 & \natline{भावमव्ययमीक्षते |} \\
 & \natline{अविभक्तं विभक्तेषु} \\
 & \natline{तज्ज्ञानं विद्धि सात्त्विकम् ||}
\end{tabular}
\end{table}

\begin{table}[H]
\begin{tabular}{cl}
\textbf{18.21} & \natline{पृथक्त्वेन तु यज्ज्ञानं} \\
 & \natline{नानाभावान् पृथग्विधान् |} \\
 & \natline{वेत्ति सर्वेषु भूतेषु} \\
 & \natline{तज्ज्ञानं विद्धि राजसम् ||}
\end{tabular}
\end{table}

\begin{table}[H]
\begin{tabular}{cl}
\textbf{18.22} & \natline{यत्तु कृत्स्नवदेकस्मिन्} \\
 & \natline{कार्ये सक्तमहैतुकम् |} \\
 & \natline{अतत्त्वार्थवदल्पं च} \\
 & \natline{तत्तामसमुदाहृतम् ||}
\end{tabular}
\end{table}

\begin{table}[H]
\begin{tabular}{cl}
\textbf{18.23} & \natline{नियतं सङ्गरहितम्} \\
 & \natline{अरागद्वेषतः कृतम् |} \\
 & \natline{अफलप्रेप्सुना कर्म} \\
 & \natline{यत्तत्सात्त्विकमुच्यते ||}
\end{tabular}
\end{table}

\begin{table}[H]
\begin{tabular}{cl}
\textbf{18.24} & \natline{यत्तु कामेप्सुना कर्म} \\
 & \natline{साहङ्कारेण वा पुनः |} \\
 & \natline{क्रियते बहुलायासं} \\
 & \natline{तद्राजसमुदाहृतम् ||}
\end{tabular}
\end{table}

\begin{table}[H]
\begin{tabular}{cl}
\textbf{18.25} & \natline{अनुबन्धं क्षयं हिंसाम्} \\
 & \natline{अनपेक्ष्य च पौरुषम् |} \\
 & \natline{मोहादारभ्यते कर्म} \\
 & \natline{यत्तत्तामसमुच्यते ||}
\end{tabular}
\end{table}

\begin{table}[H]
\begin{tabular}{cl}
\textbf{18.26} & \natline{मुक्तसङ्गोऽनहंवादी} \\
 & \natline{धृत्युत्साहसमन्वितः |} \\
 & \natline{सिद्ध्यसिद्ध्योर्निर्विकारः} \\
 & \natline{कर्ता सात्त्विक उच्यते ||}
\end{tabular}
\end{table}

\begin{table}[H]
\begin{tabular}{cl}
\textbf{18.27} & \natline{रागी कर्मफलप्रेप्सुः} \\
 & \natline{लुब्धो हिंसात्मकोऽशुचिः |} \\
 & \natline{हर्षशोकान्वितः कर्ता} \\
 & \natline{राजसः परिकीर्तितः ||}
\end{tabular}
\end{table}

\begin{table}[H]
\begin{tabular}{cl}
\textbf{18.28} & \natline{अयुक्तः प्राकृतः स्तब्धः} \\
 & \natline{शठो नैष्कृतिकोऽलसः |} \\
 & \natline{विषादी दीर्घसूत्री च} \\
 & \natline{कर्ता तामस उच्यते ||}
\end{tabular}
\end{table}

\begin{table}[H]
\begin{tabular}{cl}
\textbf{18.29} & \natline{बुद्धेर्भेदं धृतेश्चैव} \\
 & \natline{गुणतस्त्रिविधं शृणु |} \\
 & \natline{प्रोच्यमानमशेषेण} \\
 & \natline{पृथक्त्वेन धनञ्जय ||}
\end{tabular}
\end{table}

\begin{table}[H]
\begin{tabular}{cl}
\textbf{18.30} & \natline{प्रवृत्तिं च निवृत्तिं च} \\
 & \natline{कार्याकार्ये भयाभये |} \\
 & \natline{बन्धं मोक्षं च या वेत्ति} \\
 & \natline{बुद्धिः सा पार्थ सात्त्विकी ||}
\end{tabular}
\end{table}

\begin{table}[H]
\begin{tabular}{cl}
\textbf{18.31} & \natline{यया धर्ममधर्मं च} \\
 & \natline{कार्यं चाकार्यमेव च |} \\
 & \natline{अयथावत्प्रजानाति} \\
 & \natline{बुद्धिः सा पार्थ राजसी ||}
\end{tabular}
\end{table}

\begin{table}[H]
\begin{tabular}{cl}
\textbf{18.32} & \natline{अधर्मं धर्ममिति या} \\
 & \natline{मन्यते तमसाऽऽवृता |} \\
 & \natline{सर्वार्थान्विपरीतांश्च} \\
 & \natline{बुद्धिः सा पार्थ तामसी ||}
\end{tabular}
\end{table}

\begin{table}[H]
\begin{tabular}{cl}
\textbf{18.33} & \natline{धृत्या यया धारयते} \\
 & \natline{मनः प्राणेन्द्रियक्रियाः |} \\
 & \natline{योगेनाव्यभिचारिण्या} \\
 & \natline{धृतिः सा पार्थ सात्त्विकी ||}
\end{tabular}
\end{table}

\begin{table}[H]
\begin{tabular}{cl}
\textbf{18.34} & \natline{यया तु धर्मकामार्थान्} \\
 & \natline{धृत्या धारयतेऽर्जुन |} \\
 & \natline{प्रसङ्गेन फलाकाङ्क्षी} \\
 & \natline{धृतिः सा पार्थ राजसी ||}
\end{tabular}
\end{table}

\begin{table}[H]
\begin{tabular}{cl}
\textbf{18.35} & \natline{यया स्वप्नं भयं शोकं} \\
 & \natline{विषादं मदमेव च |} \\
 & \natline{न विमुञ्चति दुर्मेधाः} \\
 & \natline{धृतिः सा तामसी मता ||}
\end{tabular}
\end{table}

\begin{table}[H]
\begin{tabular}{cl}
\textbf{18.36} & \natline{सुखं त्विदानीं त्रिविधं} \\
 & \natline{शृणु मे भरतर्षभ |} \\
 & \natline{अभ्यासाद्रमते यत्र} \\
 & \natline{दुःखान्तं च निगच्छति ||}
\end{tabular}
\end{table}

\begin{table}[H]
\begin{tabular}{cl}
\textbf{18.37} & \natline{यत्तदग्रे विषमिव} \\
 & \natline{परिणामेऽमृतोपमम् |} \\
 & \natline{तत्सुखं सात्त्विकं प्रोक्तम्} \\
 & \natline{आत्मबुद्धिप्रसादजम् ||}
\end{tabular}
\end{table}

\begin{table}[H]
\begin{tabular}{cl}
\textbf{18.38} & \natline{विषयेन्द्रियसंयोगात्} \\
 & \natline{यत्तदग्रेऽमृतोपमम् |} \\
 & \natline{परिणामे विषमिव} \\
 & \natline{तत्सुखं राजसं स्मृतम् ||}
\end{tabular}
\end{table}

\begin{table}[H]
\begin{tabular}{cl}
\textbf{18.39} & \natline{यदग्रे चानुबन्धे च} \\
 & \natline{सुखं मोहनमात्मनः |} \\
 & \natline{निद्रालस्यप्रमादोत्थं} \\
 & \natline{तत्तामसमुदाहृतम् ||}
\end{tabular}
\end{table}

\begin{table}[H]
\begin{tabular}{cl}
\textbf{18.40} & \natline{न तदस्ति पृथिव्यां वा} \\
 & \natline{दिवि देवेषु वा पुनः |} \\
 & \natline{सत्त्वं प्रकृतिजैर्मुक्तं} \\
 & \natline{यदेभिः स्यात्त्रिभिर्गुणैः ||}
\end{tabular}
\end{table}

\begin{table}[H]
\begin{tabular}{cl}
\textbf{18.41} & \natline{ब्राह्मणक्षत्रियविशां} \\
 & \natline{शूद्राणां च परन्तप |} \\
 & \natline{कर्माणि प्रविभक्तानि} \\
 & \natline{स्वभावप्रभवैर्गुणैः ||}
\end{tabular}
\end{table}

\begin{table}[H]
\begin{tabular}{cl}
\textbf{18.42} & \natline{शमो दमस्तपः शौचं} \\
 & \natline{षान्तिरार्जवमेव च |} \\
 & \natline{ज्ञानं विज्ञानमास्तिक्यं} \\
 & \natline{ब्रह्मकर्म स्वभावजम् ||}
\end{tabular}
\end{table}

\begin{table}[H]
\begin{tabular}{cl}
\textbf{18.43} & \natline{शौर्यं तेजो धृतिर्दाक्ष्यं} \\
 & \natline{युद्धे चाप्यपलायनम् |} \\
 & \natline{दानमीश्वरभावश्च} \\
 & \natline{क्षात्रं कर्म स्वभावजम् ||}
\end{tabular}
\end{table}

\begin{table}[H]
\begin{tabular}{cl}
\textbf{18.44} & \natline{कृषिगौरक्ष्यवाणिज्यं} \\
 & \natline{वैश्यकर्म स्वभावजम् |} \\
 & \natline{परिचर्यात्मकं कर्म} \\
 & \natline{शूद्रस्यापि स्वभावजम् ||}
\end{tabular}
\end{table}

\begin{table}[H]
\begin{tabular}{cl}
\textbf{18.45} & \natline{स्वे स्वे कर्मण्यभिरतः} \\
 & \natline{संसिद्धिं लभते नरः |} \\
 & \natline{स्वकर्मनिरतः सिद्धिं} \\
 & \natline{यथा विन्दति तच्छृणु ||}
\end{tabular}
\end{table}

\begin{table}[H]
\begin{tabular}{cl}
\textbf{18.46} & \natline{यतः प्रवृत्तिर्भूतानां} \\
 & \natline{येन सर्वमिदं ततम् |} \\
 & \natline{स्वकर्मणा तमभ्यर्च्य} \\
 & \natline{सिद्धिं विन्दति मानवः ||}
\end{tabular}
\end{table}

\begin{table}[H]
\begin{tabular}{cl}
\textbf{18.47} & \natline{श्रेयान्स्वधर्मो विगुणः} \\
 & \natline{परधर्मात्स्वनुष्ठितात् |} \\
 & \natline{स्वभावनियतं कर्म} \\
 & \natline{कुर्वन्नाप्नोति किल्बिषम् ||}
\end{tabular}
\end{table}

\begin{table}[H]
\begin{tabular}{cl}
\textbf{18.48} & \natline{सहजं कर्म कौन्तेय} \\
 & \natline{सदोषमपि न त्यजेत् |} \\
 & \natline{सर्वारम्भा हि दोषेण} \\
 & \natline{धूमेनाग्निरिवावृताः ||}
\end{tabular}
\end{table}

\begin{table}[H]
\begin{tabular}{cl}
\textbf{18.49} & \natline{असक्तबुद्धिः सर्वत्र} \\
 & \natline{जितात्मा विगतस्पृहः |} \\
 & \natline{नैष्कर्म्यसिद्धिं परमां} \\
 & \natline{सन्न्यासेनाधिगच्छति ||}
\end{tabular}
\end{table}

\begin{table}[H]
\begin{tabular}{cl}
\textbf{18.50} & \natline{सिद्धिं प्राप्तो यथा ब्रह्म} \\
 & \natline{तथाऽऽप्नोति निबोध मे |} \\
 & \natline{समासेनैव कौन्तेय} \\
 & \natline{निष्ठा ज्ञानस्य या परा ||}
\end{tabular}
\end{table}

\begin{table}[H]
\begin{tabular}{cl}
\textbf{18.51} & \natline{बुद्ध्या विशुद्धया युक्तः} \\
 & \natline{धृत्याऽऽत्मानं नियम्य च |} \\
 & \natline{शब्दादीन्विषयांस्त्यक्त्वा} \\
 & \natline{रागद्वेषौ व्युदस्य च ||}
\end{tabular}
\end{table}

\begin{table}[H]
\begin{tabular}{cl}
\textbf{18.52} & \natline{विविक्तसेवी लघ्वाशी} \\
 & \natline{यतवाक्कायमानसः |} \\
 & \natline{ध्यानयोगपरो नित्यं} \\
 & \natline{वैराग्यं समुपाश्रितः ||}
\end{tabular}
\end{table}

\begin{table}[H]
\begin{tabular}{cl}
\textbf{18.53} & \natline{अहङ्कारं बलं दर्पं} \\
 & \natline{कामं क्रोधं परिग्रहम् |} \\
 & \natline{विमुच्य निर्ममः शान्तः} \\
 & \natline{ब्रह्मभूयाय कल्पते ||}
\end{tabular}
\end{table}

\begin{table}[H]
\begin{tabular}{cl}
\textbf{18.54} & \natline{ब्रह्मभूतः प्रसन्नात्मा} \\
 & \natline{न शोचति न काङ्क्षति |} \\
 & \natline{समः सर्वेषु भूतेषु} \\
 & \natline{मद्भक्तिं लभते पराम् ||}
\end{tabular}
\end{table}

\begin{table}[H]
\begin{tabular}{cl}
\textbf{18.55} & \natline{भक्त्या मामभिजानाति} \\
 & \natline{यावान्यश्चास्मि तत्त्वतः |} \\
 & \natline{ततो मां तत्त्वतो ज्ञात्वा} \\
 & \natline{विशते तदनन्तरम् ||}
\end{tabular}
\end{table}

\begin{table}[H]
\begin{tabular}{cl}
\textbf{18.56} & \natline{सर्वकर्माण्यपि सदा} \\
 & \natline{कुर्वाणो मद्व्यपाश्रयः |} \\
 & \natline{मत्प्रसादादवाप्नोति} \\
 & \natline{शाश्वतं पदमव्ययम् ||}
\end{tabular}
\end{table}

\begin{table}[H]
\begin{tabular}{cl}
\textbf{18.57} & \natline{चेतसा सर्वकर्माणि} \\
 & \natline{मयि सन्न्यस्य मत्परः |} \\
 & \natline{बुद्धियोगमुपाश्रित्य} \\
 & \natline{मच्चित्तः सततं भव ||}
\end{tabular}
\end{table}

\begin{table}[H]
\begin{tabular}{cl}
\textbf{18.58} & \natline{मच्चित्तः सर्वदुर्गाणि} \\
 & \natline{मत्प्रसादात्तरिष्यसि |} \\
 & \natline{अथ चेत्त्वमहङ्कारात्} \\
 & \natline{न श्रोष्यसि विनङ्क्ष्यसि ||}
\end{tabular}
\end{table}

\begin{table}[H]
\begin{tabular}{cl}
\textbf{18.59} & \natline{यदहङ्कारमाश्रित्य} \\
 & \natline{न योत्स्य इति मन्यसे |} \\
 & \natline{मिथ्यैष व्यवसायस्ते} \\
 & \natline{प्रकृतिस्त्वां नियोक्ष्यति ||}
\end{tabular}
\end{table}

\begin{table}[H]
\begin{tabular}{cl}
\textbf{18.60} & \natline{स्वभावजेन कौन्तेय} \\
 & \natline{निबद्धः स्वेन कर्मणा |} \\
 & \natline{कर्तुं नेच्छसि यन्मोहात्} \\
 & \natline{करिष्यस्यवशोऽपि तत् ||}
\end{tabular}
\end{table}

\begin{table}[H]
\begin{tabular}{cl}
\textbf{18.61} & \natline{ईश्वरः सर्वभूतानां} \\
 & \natline{हृद्देशेऽर्जुन तिष्ठति |} \\
 & \natline{भ्रामयन्सर्वभूतानि} \\
 & \natline{यन्त्रारूढानि मायया ||}
\end{tabular}
\end{table}

\begin{table}[H]
\begin{tabular}{cl}
\textbf{18.62} & \natline{तमेव शरणं गच्छ} \\
 & \natline{सर्वभावेन भारत |} \\
 & \natline{तत्प्रसादात्परां शान्तिं} \\
 & \natline{स्थानं प्राप्स्यसि शाश्वतम् ||}
\end{tabular}
\end{table}

\begin{table}[H]
\begin{tabular}{cl}
\textbf{18.63} & \natline{इति ते ज्ञानमाख्यातं} \\
 & \natline{गुह्याद्गुह्यतरं मया |} \\
 & \natline{विमृश्यैतदशेषेण} \\
 & \natline{यथेच्छसि तथा कुरु ||}
\end{tabular}
\end{table}

\begin{table}[H]
\begin{tabular}{cl}
\textbf{18.64} & \natline{सर्वगुह्यतमं भूयः} \\
 & \natline{शृणु मे परमं वचः |} \\
 & \natline{इष्टोऽसि मे दृढमिति} \\
 & \natline{ततो वक्ष्यामि ते हितम् ||}
\end{tabular}
\end{table}

\begin{table}[H]
\begin{tabular}{cl}
\textbf{18.65} & \natline{मन्मना भव मद्भक्तः} \\
 & \natline{मद्याजी मां नमस्कुरु |} \\
 & \natline{मामेवैष्यसि सत्यं ते} \\
 & \natline{प्रतिजाने प्रियोऽसि मे ||}
\end{tabular}
\end{table}

\begin{table}[H]
\begin{tabular}{cl}
\textbf{18.66} & \natline{सर्वधर्मान्परित्यज्य} \\
 & \natline{मामेकं शरणं व्रज |} \\
 & \natline{अहं त्वा सर्वपापेभ्यः} \\
 & \natline{मोक्षयिष्यामि मा शुचः ||}
\end{tabular}
\end{table}

\begin{table}[H]
\begin{tabular}{cl}
\textbf{18.67} & \natline{इदं ते नातपस्काय} \\
 & \natline{नाभक्ताय कदाचन |} \\
 & \natline{न चाशुश्रूषवे वाच्यं} \\
 & \natline{न च मां योऽभ्यसूयति ||}
\end{tabular}
\end{table}

\begin{table}[H]
\begin{tabular}{cl}
\textbf{18.68} & \natline{य इमं परमं गुह्यं} \\
 & \natline{मद्भक्तेष्वभिधास्यति |} \\
 & \natline{भक्तिं मयि परां कृत्वा} \\
 & \natline{मामेवैष्यत्यसंशयः ||}
\end{tabular}
\end{table}

\begin{table}[H]
\begin{tabular}{cl}
\textbf{18.69} & \natline{न च तस्मान्मनुष्येषु} \\
 & \natline{कश्चिन्मे प्रियकृत्तमः |} \\
 & \natline{भविता न च मे तस्मात्} \\
 & \natline{अन्यः प्रियतरो भुवि ||}
\end{tabular}
\end{table}

\begin{table}[H]
\begin{tabular}{cl}
\textbf{18.70} & \natline{अध्येष्यते च य इमं} \\
 & \natline{धर्म्यं संवादमावयोः |} \\
 & \natline{ज्ञानयज्ञेन तेनाहम्} \\
 & \natline{इष्टः स्यामिति मे मतिः ||}
\end{tabular}
\end{table}

\begin{table}[H]
\begin{tabular}{cl}
\textbf{18.71} & \natline{श्रद्धावाननसूयश्च} \\
 & \natline{शृणुयादपि यो नरः |} \\
 & \natline{सोऽपि मुक्तः शुभाल्लोकान्} \\
 & \natline{प्राप्नुयात्पुण्यकर्मणाम् ||}
\end{tabular}
\end{table}

\begin{table}[H]
\begin{tabular}{cl}
\textbf{18.72} & \natline{कच्चिदेतच्छ्रुतं पार्थ} \\
 & \natline{त्वयैकाग्रेण चेतसा |} \\
 & \natline{कच्चिदज्ञानसम्मोहः} \\
 & \natline{प्रनष्टस्ते धनञ्जय ||}
\end{tabular}
\end{table}

\begin{table}[H]
\begin{tabular}{cl}
\textbf{18.73} & \natline{अर्जुन उवाच} \\
 & \natline{नष्टो मोहः स्मृतिर्लब्धा} \\
 & \natline{त्वत्प्रसादान्मयाऽच्युत |} \\
 & \natline{स्थितोऽस्मि गतसन्देहः} \\
 & \natline{करिष्ये वचनं तव ||}
\end{tabular}
\end{table}

\begin{table}[H]
\begin{tabular}{cl}
\textbf{18.74} & \natline{सञ्जय उवाच} \\
 & \natline{इत्यहं वासुदेवस्य} \\
 & \natline{पार्थस्य च महात्मनः |} \\
 & \natline{संवादमिममश्रौषम्} \\
 & \natline{अद्भुतं रोमहर्षणम् ||}
\end{tabular}
\end{table}

\begin{table}[H]
\begin{tabular}{cl}
\textbf{18.75} & \natline{व्यासप्रसादाच्छ्रुतवान्} \\
 & \natline{इमं गुह्यतमं परम् |} \\
 & \natline{योगं योगेश्वरात्कृष्णात्} \\
 & \natline{साक्षात्कथयतः स्वयम् ||}
\end{tabular}
\end{table}

\begin{table}[H]
\begin{tabular}{cl}
\textbf{18.76} & \natline{राजन् संस्मृत्य संस्मृत्य} \\
 & \natline{संवादमिममद्भुतम् |} \\
 & \natline{केशवार्जुनयोः पुण्यं} \\
 & \natline{हृष्यामि च मुहुर्मुहुः ||}
\end{tabular}
\end{table}

\begin{table}[H]
\begin{tabular}{cl}
\textbf{18.77} & \natline{तच्च संस्मृत्य संस्मृत्य} \\
 & \natline{रूपमत्यद्भुतं हरेः |} \\
 & \natline{विस्मयो मे महान्राजन्} \\
 & \natline{हृष्यामि च पुनः पुनः ||}
\end{tabular}
\end{table}

\begin{table}[H]
\begin{tabular}{cl}
\textbf{18.78} & \natline{यत्र योगेश्वरः कृष्णः} \\
 & \natline{यत्र पार्थो धनुर्धरः |} \\
 & \natline{तत्र श्रीर्विजयो भूतिः} \\
 & \natline{ध्रुवा नीतिर्मतिर्मम ||}
\end{tabular}
\end{table}


% \end{multicols}

\backmatter
\chapter{Closing Ślokas}
\begin{table}[H]
\begin{tabular}{l}
\romline{tatsaditi śrīmad-bhagavadgītāsu} \\
\romline{upaniṣatsu brahmavidyāyāṃ yogaśāstre} \\
\romline{śrīkṛṣṇārjuna saṃvāde} \\
\romline{(bhakti-yogonāma dvādaśodhyāyaḥ)}
\end{tabular}
\end{table}

\begin{table}[H]
\begin{tabular}{l}
\romline{sarvadharmān-parityajya} \\
\romline{māmekaṃ śaraṇaṃ vraja} \\
\romline{aham tvā sarvapāpebhyaḥ} \\
\romline{mokṣayiṣyāmi mā śucaḥ}
\end{tabular}
\end{table}

\begin{table}[H]
\begin{tabular}{l}
\romline{yataḥ pravṛttir-bhūtānāṃ} \\
\romline{yena sarvam-idaṃ tatam} \\
\romline{sva-karmaṇā tam-abhyarcya} \\
\romline{siddhiṃ viṃdati mānavaḥ}
\end{tabular}
\end{table}

\begin{table}[H]
\begin{tabular}{l}
\romline{yatra yogeśvaraḥ kṛṣṇaḥ} \\
\romline{yatra pārtho dhanur-dharaḥ} \\
\romline{tatra śrīr-vijayo bhūtiḥ} \\
\romline{dhruvā nītir-matir-mama}
\end{tabular}
\end{table}

\begin{table}[H]
\begin{tabular}{l}
\romline{śrī kṛṣṇaśśaraṇaṃ mama} \\
\romline{śrī kṛṣṇaśśaraṇaṃ mama} \\
\romline{śrī kṛṣṇaśśaraṇaṃ mama}
\end{tabular}
\end{table}



\end{document}