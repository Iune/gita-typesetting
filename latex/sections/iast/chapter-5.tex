\begin{table}[H]
\begin{tabular}{cl}
 & \romline{śrī paramātmane namaḥ} \\
 & \romline{atha pañcamo'dhyāyaḥ} \\
 & \romline{karmasannyāsayogaḥ}
\end{tabular}
\end{table}

\begin{table}[H]
\begin{tabular}{cl}
\textbf{5.1} & \romline{arjuna uvāca} \\
 & \romline{sannyāsaṃ karmaṇāṃ kṛṣṇa} \\
 & \romline{punaryogaṃ ca śaṃsasi |} \\
 & \romline{yacchreya etayorekaṃ} \\
 & \romline{tanme brūhi suniścitam ||}
\end{tabular}
\end{table}

\begin{table}[H]
\begin{tabular}{cl}
\textbf{5.2} & \romline{śrī bhagavānuvāca} \\
 & \romline{sannyāsaḥ karmayogaśca} \\
 & \romline{niśśreyasakarāvubhau |} \\
 & \romline{tayostu karmasannyāsāt} \\
 & \romline{karmayogo viśiṣyate ||}
\end{tabular}
\end{table}

\begin{table}[H]
\begin{tabular}{cl}
\textbf{5.3} & \romline{jñeyaḥ sa nityasannyāsī} \\
 & \romline{yo na dveṣṭi na kāṅkṣati |} \\
 & \romline{nirdvandvo hi mahābāho} \\
 & \romline{sukhaṃ bandhātpramucyate ||}
\end{tabular}
\end{table}

\begin{table}[H]
\begin{tabular}{cl}
\textbf{5.4} & \romline{sāṅkhyayogau pṛthagbālāḥ} \\
 & \romline{pravadanti na paṇḍitāḥ |} \\
 & \romline{ekamapyāsthitaḥ samyak} \\
 & \romline{ubhayorvindate phalam ||}
\end{tabular}
\end{table}

\begin{table}[H]
\begin{tabular}{cl}
\textbf{5.5} & \romline{yatsāṅkhyaiḥ prāpyate sthānaṃ} \\
 & \romline{tadyogairapi gamyate |} \\
 & \romline{ekaṃ sāṅkhyaṃ ca yogaṃ ca} \\
 & \romline{yaḥ paśyati sa paśyati ||}
\end{tabular}
\end{table}

\begin{table}[H]
\begin{tabular}{cl}
\textbf{5.6} & \romline{sannyāsastu mahābāho} \\
 & \romline{duḥkhamāptumayogataḥ |} \\
 & \romline{yogayukto munirbrahma} \\
 & \romline{nacireṇādhigacchati ||}
\end{tabular}
\end{table}

\begin{table}[H]
\begin{tabular}{cl}
\textbf{5.7} & \romline{yogayukto viśuddhātmā} \\
 & \romline{vijitātmā jitendriyaḥ |} \\
 & \romline{sarvabhūtātmabhūtātmā} \\
 & \romline{kurvannapi na lipyate ||}
\end{tabular}
\end{table}

\begin{table}[H]
\begin{tabular}{cl}
\textbf{5.8} & \romline{naiva kiñcitkaromīti} \\
 & \romline{yukto manyeta tattvavit |} \\
 & \romline{paśyanśṛṇvan spṛśañjighran} \\
 & \romline{aśnangacchansvapanśvasan ||}
\end{tabular}
\end{table}

\begin{table}[H]
\begin{tabular}{cl}
\textbf{5.9} & \romline{pralapan visṛjan gṛhṇan} \\
 & \romline{unmiṣannimiṣannapi |} \\
 & \romline{indriyāṇīndriyārtheṣu} \\
 & \romline{vartanta iti dhārayan ||}
\end{tabular}
\end{table}

\begin{table}[H]
\begin{tabular}{cl}
\textbf{5.10} & \romline{brahmaṇyādhāya karmāṇi} \\
 & \romline{saṅgaṃ tyaktvā karoti yaḥ |} \\
 & \romline{lipyate na sa pāpena} \\
 & \romline{padmapatramivāmbhasā ||}
\end{tabular}
\end{table}

\begin{table}[H]
\begin{tabular}{cl}
\textbf{5.11} & \romline{kāyena manasā buddhyā} \\
 & \romline{kevalairindriyairapi |} \\
 & \romline{yoginaḥ karma kurvanti} \\
 & \romline{saṅgaṃ tyaktvātmaśuddhaye ||}
\end{tabular}
\end{table}

\begin{table}[H]
\begin{tabular}{cl}
\textbf{5.12} & \romline{yuktaḥ karmaphalaṃ tyaktvā} \\
 & \romline{śāntimāpnoti naiṣṭhikīm |} \\
 & \romline{ayuktaḥ kāmakāreṇa} \\
 & \romline{phale sakto nibadhyate ||}
\end{tabular}
\end{table}

\begin{table}[H]
\begin{tabular}{cl}
\textbf{5.13} & \romline{sarvakarmāṇi manasā} \\
 & \romline{sannyasyāste sukhaṃ vaśī |} \\
 & \romline{navadvāre pure dehī} \\
 & \romline{naiva kurvanna kārayan ||}
\end{tabular}
\end{table}

\begin{table}[H]
\begin{tabular}{cl}
\textbf{5.14} & \romline{na kartṛtvaṃ na karmāṇi} \\
 & \romline{lokasya sṛjati prabhuḥ |} \\
 & \romline{na karmaphalasaṃyogaṃ} \\
 & \romline{svabhāvastu pravartate ||}
\end{tabular}
\end{table}

\begin{table}[H]
\begin{tabular}{cl}
\textbf{5.15} & \romline{nādatte kasyacitpāpaṃ} \\
 & \romline{na caiva sukṛtaṃ vibhuḥ |} \\
 & \romline{ajñānenāvṛtaṃ jñānaṃ} \\
 & \romline{tena muhyanti jantavaḥ ||}
\end{tabular}
\end{table}

\begin{table}[H]
\begin{tabular}{cl}
\textbf{5.16} & \romline{jñānena tu tadajñānaṃ} \\
 & \romline{yeṣāṃ nāśitamātmanaḥ |} \\
 & \romline{teṣāmādityavajjñānaṃ} \\
 & \romline{prakāśayati tatparam ||}
\end{tabular}
\end{table}

\begin{table}[H]
\begin{tabular}{cl}
\textbf{5.17} & \romline{tadbuddhayastadātmānaḥ} \\
 & \romline{tanniṣṭhāstatparāyaṇāḥ |} \\
 & \romline{gacchantyapunarāvṛttiṃ} \\
 & \romline{jñānanirdhūtakalmaṣāḥ ||}
\end{tabular}
\end{table}

\begin{table}[H]
\begin{tabular}{cl}
\textbf{5.18} & \romline{vidyāvinayasampanne} \\
 & \romline{brāhmaṇe gavi hastini |} \\
 & \romline{śuni caiva śvapāke ca} \\
 & \romline{paṇḍitāḥ samadarśinaḥ ||}
\end{tabular}
\end{table}

\begin{table}[H]
\begin{tabular}{cl}
\textbf{5.19} & \romline{ihaiva tairjitaḥ sargaḥ} \\
 & \romline{yeṣāṃ sāmye sthitaṃ manaḥ |} \\
 & \romline{nirdoṣaṃ hi samaṃ brahma} \\
 & \romline{tasmāt brahmaṇi te sthitāḥ ||}
\end{tabular}
\end{table}

\begin{table}[H]
\begin{tabular}{cl}
\textbf{5.20} & \romline{na prahṛṣyetpriyaṃ prāpya} \\
 & \romline{nodvijetprāpya cāpriyam |} \\
 & \romline{sthirabuddhirasammūḍhaḥ} \\
 & \romline{brahmavit brahmaṇi sthitaḥ ||}
\end{tabular}
\end{table}

\begin{table}[H]
\begin{tabular}{cl}
\textbf{5.21} & \romline{bāhyasparśeṣvasaktātmā} \\
 & \romline{vindatyātmani yat sukham |} \\
 & \romline{sa brahmayogayuktātmā} \\
 & \romline{sukhamakṣayamaśnute ||}
\end{tabular}
\end{table}

\begin{table}[H]
\begin{tabular}{cl}
\textbf{5.22} & \romline{ye hi saṃsparśajā bhogāḥ} \\
 & \romline{duḥkhayonaya eva te |} \\
 & \romline{ādyantavantaḥ kaunteya} \\
 & \romline{na teṣu ramate budhaḥ ||}
\end{tabular}
\end{table}

\begin{table}[H]
\begin{tabular}{cl}
\textbf{5.23} & \romline{śaknotīhaiva yaḥ soḍhuṃ} \\
 & \romline{prāk śarīravimokṣaṇāt |} \\
 & \romline{kāmakrodhodbhavaṃ vegaṃ} \\
 & \romline{sa yuktaḥ sa sukhī naraḥ ||}
\end{tabular}
\end{table}

\begin{table}[H]
\begin{tabular}{cl}
\textbf{5.24} & \romline{yo'ntaḥsukho'ntarārāmaḥ} \\
 & \romline{tathāntarjyotireva yaḥ |} \\
 & \romline{sa yogī brahmanirvāṇaṃ} \\
 & \romline{brahmabhūto'dhigacchati ||}
\end{tabular}
\end{table}

\begin{table}[H]
\begin{tabular}{cl}
\textbf{5.25} & \romline{labhante brahmanirvāṇam} \\
 & \romline{ṛṣayaḥ kṣīṇakalmaṣāḥ |} \\
 & \romline{chinnadvaidhā yatātmānaḥ} \\
 & \romline{sarvabhūtahite ratāḥ ||}
\end{tabular}
\end{table}

\begin{table}[H]
\begin{tabular}{cl}
\textbf{5.26} & \romline{kāmakrodhaviyuktānāṃ} \\
 & \romline{yatīnāṃ yatacetasām |} \\
 & \romline{abhito brahmanirvāṇaṃ} \\
 & \romline{vartate viditātmanām ||}
\end{tabular}
\end{table}

\begin{table}[H]
\begin{tabular}{cl}
\textbf{5.27} & \romline{sparśān kṛtvā bahirbāhyān} \\
 & \romline{cakṣuścaivāntare bhruvoḥ |} \\
 & \romline{prāṇāpānau samau kṛtvā} \\
 & \romline{nāsābhyantaracāriṇau ||}
\end{tabular}
\end{table}

\begin{table}[H]
\begin{tabular}{cl}
\textbf{5.28} & \romline{yatendriyamanobuddhiḥ} \\
 & \romline{munirmokṣaparāyaṇaḥ |} \\
 & \romline{vigatecchābhayakrodhaḥ} \\
 & \romline{yaḥ sadā mukta eva saḥ ||}
\end{tabular}
\end{table}

\begin{table}[H]
\begin{tabular}{cl}
\textbf{5.29} & \romline{bhoktāraṃ yajñatapasāṃ} \\
 & \romline{sarvalokamaheśvaram |} \\
 & \romline{suhṛdaṃ sarvabhūtānāṃ} \\
 & \romline{jñātvā māṃ śāntimṛcchati| ||}
\end{tabular}
\end{table}

\begin{table}[H]
\begin{tabular}{cl}
 & \romline{śrīmadbhagavadgītāsu upaniṣatsu} \\
 & \romline{brahmavidyāyāṃ yogaśāstre} \\
 & \romline{śrīkṛṣṇārjuna saṃvāde} \\
 & \romline{karmasannyāsayogo nāma} \\
 & \romline{pañcamodhyāyaḥ}
\end{tabular}
\end{table}

