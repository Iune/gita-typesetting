\begin{table}[H]
\begin{tabular}{cl}
 & \romline{śrī paramātmane namaḥ} \\
 & \romline{atha pañcamo'dhyāyaḥ} \\
 & \romline{karma-sannyāsa-yogaḥ}
\end{tabular}
\end{table}

\begin{table}[H]
\begin{tabular}{cl}
\textbf{5.1} & \romline{arjuna uvāca} \\
 & \romline{sannyāsaṃ karmaṇāṃ kṛṣṇa} \\
 & \romline{punaryogaṃ ca śaṃsasi |} \\
 & \romline{yacchreya etayorekaṃ} \\
 & \romline{tanme brūhi suniścitam ||}
\end{tabular}
\end{table}

\begin{table}[H]
\begin{tabular}{cl}
\textbf{5.2} & \romline{śrī bhagavānuvāca} \\
 & \romline{sannyāsaḥ karmayogaśca} \\
 & \romline{niśśreyasa-karā-vubhau |} \\
 & \romline{tayostu karma-sannyāsāt} \\
 & \romline{karmayogo viśiṣyate ||}
\end{tabular}
\end{table}

\begin{table}[H]
\begin{tabular}{cl}
\textbf{5.3} & \romline{jñeyaḥ sa nitya-sannyāsī} \\
 & \romline{yo na·dveṣṭi na kāṅkṣati |} \\
 & \romline{nirdvandvo hi mahābāho} \\
 & \romline{sukhaṃ bandhāt-pramucyate ||}
\end{tabular}
\end{table}

\begin{table}[H]
\begin{tabular}{cl}
\textbf{5.4} & \romline{sāṅkhyayogau pṛthag-bālāḥ} \\
 & \romline{pravadanti na paṇḍitāḥ |} \\
 & \romline{eka-mapyā-sthitaḥ samyak} \\
 & \romline{ubhayor-vindate phalam ||}
\end{tabular}
\end{table}

\begin{table}[H]
\begin{tabular}{cl}
\textbf{5.5} & \romline{yat-sāṅkhyaiḥ prāpyate sthānaṃ} \\
 & \romline{tadyogairapi gamyate |} \\
 & \romline{ekaṃ sāṅkhyaṃ ca yogaṃ ca} \\
 & \romline{yaḥ paśyati sa paśyati ||}
\end{tabular}
\end{table}

\begin{table}[H]
\begin{tabular}{cl}
\textbf{5.6} & \romline{sannyāsastu mahābāho} \\
 & \romline{duḥkha-māptuma-yogataḥ |} \\
 & \romline{yogayukto munir-brahma} \\
 & \romline{nacire-ṇādhi-gacchati ||}
\end{tabular}
\end{table}

\begin{table}[H]
\begin{tabular}{cl}
\textbf{5.7} & \romline{yogayukto viśuddhātmā} \\
 & \romline{vijitātmā jitendriyaḥ |} \\
 & \romline{sarva-bhūtātma-bhūtātmā} \\
 & \romline{kurvannapi na lipyate ||}
\end{tabular}
\end{table}

\begin{table}[H]
\begin{tabular}{cl}
\textbf{5.8} & \romline{naiva kiñcit-karomīti} \\
 & \romline{yukto manyeta tattvavit |} \\
 & \romline{paśyan-śṛṇvan spṛśañ-jighran} \\
 & \romline{aśnan-gacchan-svapan-śvasan ||}
\end{tabular}
\end{table}

\begin{table}[H]
\begin{tabular}{cl}
\textbf{5.9} & \romline{pralapan visṛjan gṛhṇan} \\
 & \romline{unmiṣan-nimiṣannapi |} \\
 & \romline{indriyāṇīndriyārtheṣu} \\
 & \romline{vartanta iti dhārayan ||}
\end{tabular}
\end{table}

\begin{table}[H]
\begin{tabular}{cl}
\textbf{5.10} & \romline{brahmaṇyādhāya karmāṇi} \\
 & \romline{saṅgaṃ tyaktvā karoti yaḥ |} \\
 & \romline{lipyate na sa pāpena} \\
 & \romline{padmapatra-mivāmbhasā ||}
\end{tabular}
\end{table}

\begin{table}[H]
\begin{tabular}{cl}
\textbf{5.11} & \romline{kāyena manasā buddhyā} \\
 & \romline{kevalair-indriyairapi |} \\
 & \romline{yoginaḥ karma kurvanti} \\
 & \romline{saṅgaṃ tyaktvāt-maśuddhaye ||}
\end{tabular}
\end{table}

\begin{table}[H]
\begin{tabular}{cl}
\textbf{5.12} & \romline{yuktaḥ karma-phalaṃ tyaktvā} \\
 & \romline{śāntimāpnoti naiṣṭhikīm |} \\
 & \romline{ayuktaḥ kāmakāreṇa} \\
 & \romline{phale sakto nibadhyate ||}
\end{tabular}
\end{table}

\begin{table}[H]
\begin{tabular}{cl}
\textbf{5.13} & \romline{sarva-karmāṇi manasā} \\
 & \romline{sannyasyāste sukhaṃ vaśī |} \\
 & \romline{navadvāre pure dehī} \\
 & \romline{naiva kurvanna kārayan ||}
\end{tabular}
\end{table}

\begin{table}[H]
\begin{tabular}{cl}
\textbf{5.14} & \romline{na kartṛtvaṃ na karmāṇi} \\
 & \romline{lokasya sṛjati·prabhuḥ |} \\
 & \romline{na karma-phala-saṃyogaṃ} \\
 & \romline{svabhāvastu·pravartate ||}
\end{tabular}
\end{table}

\begin{table}[H]
\begin{tabular}{cl}
\textbf{5.15} & \romline{nādatte kasyacit-pāpaṃ} \\
 & \romline{na caiva sukṛtaṃ vibhuḥ |} \\
 & \romline{ajñā-nenāvṛtaṃ jñānaṃ} \\
 & \romline{tena muhyanti jantavaḥ ||}
\end{tabular}
\end{table}

\begin{table}[H]
\begin{tabular}{cl}
\textbf{5.16} & \romline{jñānena tu tada-jñānaṃ} \\
 & \romline{yeṣāṃ nāśita-mātmanaḥ |} \\
 & \romline{teṣām-ādityavaj-jñānaṃ} \\
 & \romline{prakāśayati tatparam ||}
\end{tabular}
\end{table}

\begin{table}[H]
\begin{tabular}{cl}
\textbf{5.17} & \romline{tadbuddhayas-tadātmānaḥ} \\
 & \romline{tanniṣṭhā-statparāyaṇāḥ |} \\
 & \romline{gacchantyapuna-rāvṛttiṃ} \\
 & \romline{jñāna-nirdhūta-kalmaṣāḥ ||}
\end{tabular}
\end{table}

\begin{table}[H]
\begin{tabular}{cl}
\textbf{5.18} & \romline{vidyā-vinaya-sampanne} \\
 & \romline{brāhmaṇe gavi hastini |} \\
 & \romline{śuni caiva śvapāke ca} \\
 & \romline{paṇḍitāḥ samadarśinaḥ ||}
\end{tabular}
\end{table}

\begin{table}[H]
\begin{tabular}{cl}
\textbf{5.19} & \romline{ihaiva tairjitaḥ sargaḥ} \\
 & \romline{yeṣāṃ sāmye sthitaṃ manaḥ |} \\
 & \romline{nirdoṣaṃ hi samaṃ brahma} \\
 & \romline{tasmāt brahmaṇi te sthitāḥ ||}
\end{tabular}
\end{table}

\begin{table}[H]
\begin{tabular}{cl}
\textbf{5.20} & \romline{na·pra-hṛṣyet-priyaṃ prāpya} \\
 & \romline{nod-vijet-prāpya cāpriyam |} \\
 & \romline{sthira-buddhira-sammūḍhaḥ} \\
 & \romline{brahmavit brahmaṇi·sthitaḥ ||}
\end{tabular}
\end{table}

\begin{table}[H]
\begin{tabular}{cl}
\textbf{5.21} & \romline{bāhyas-parśe-ṣvasaktātmā} \\
 & \romline{vindatyātmani yat sukham |} \\
 & \romline{sa brahma-yoga-yuktātmā} \\
 & \romline{sukha-makṣaya-maśnute ||}
\end{tabular}
\end{table}

\begin{table}[H]
\begin{tabular}{cl}
\textbf{5.22} & \romline{ye hi saṃ-sparśajā bhogāḥ} \\
 & \romline{duḥkha-yonaya eva te |} \\
 & \romline{ād-yantavantaḥ kaunteya} \\
 & \romline{na teṣu ramate budhaḥ ||}
\end{tabular}
\end{table}

\begin{table}[H]
\begin{tabular}{cl}
\textbf{5.23} & \romline{śaknotī-haiva yaḥ soḍhuṃ} \\
 & \romline{prāk śarīra-vimokṣaṇāt |} \\
 & \romline{kāma-krodhod-bhavaṃ vegaṃ} \\
 & \romline{sa yuktaḥ sa sukhī naraḥ ||}
\end{tabular}
\end{table}

\begin{table}[H]
\begin{tabular}{cl}
\textbf{5.24} & \romline{yo'ntaḥ-sukho'ntarā-rāmaḥ} \\
 & \romline{tathān-tarjyotireva yaḥ |} \\
 & \romline{sa yogī brahma-nirvāṇaṃ} \\
 & \romline{brahma-bhūto'dhi-gacchati ||}
\end{tabular}
\end{table}

\begin{table}[H]
\begin{tabular}{cl}
\textbf{5.25} & \romline{labhante brahma-nirvāṇam} \\
 & \romline{ṛṣayaḥ·kṣīṇa-kalmaṣāḥ |} \\
 & \romline{chinna-dvaidhā yatātmānaḥ} \\
 & \romline{sarva-bhūtahite ratāḥ ||}
\end{tabular}
\end{table}

\begin{table}[H]
\begin{tabular}{cl}
\textbf{5.26} & \romline{kāma-krodha-viyuktānāṃ} \\
 & \romline{yatīnāṃ yatacetasām |} \\
 & \romline{abhito brahma-nirvāṇaṃ} \\
 & \romline{vartate viditātmanām ||}
\end{tabular}
\end{table}

\begin{table}[H]
\begin{tabular}{cl}
\textbf{5.27} & \romline{sparśān kṛtvā bahir-bāhyān} \\
 & \romline{cakṣuś-caivāntare bhruvoḥ |} \\
 & \romline{prāṇāpānau samau kṛtvā} \\
 & \romline{nāsābhyan-taracāriṇau ||}
\end{tabular}
\end{table}

\begin{table}[H]
\begin{tabular}{cl}
\textbf{5.28} & \romline{yatendriya-manobuddhiḥ} \\
 & \romline{munir-mokṣa-parāyaṇaḥ |} \\
 & \romline{vigatecchā-bhaya-krodhaḥ} \\
 & \romline{yaḥ sadā mukta eva saḥ ||}
\end{tabular}
\end{table}

\begin{table}[H]
\begin{tabular}{cl}
\textbf{5.29} & \romline{bhoktāraṃ yajña-tapasāṃ} \\
 & \romline{sarva-loka-maheśvaram |} \\
 & \romline{suhṛdaṃ sarva-bhūtānāṃ} \\
 & \romline{jñātvā māṃ śānti-mṛcchati| ||}
\end{tabular}
\end{table}

\begin{table}[H]
\begin{tabular}{cl}
 & \romline{śrīmad-bhagavad-gītāsu upaniṣatsu} \\
 & \romline{brahma-vidyāyāṃ yogaśāstre} \\
 & \romline{śrīkṛṣṇārjuna saṃvāde} \\
 & \romline{karma-sannyāsa-yogonāma} \\
 & \romline{pañcamo-dhyāyaḥ}
\end{tabular}
\end{table}

