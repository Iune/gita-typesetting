\begin{table}[H]
\begin{tabular}{cl}
\textbf{17.0} & \romline{oṃ śrī paramātmane namaḥ} \\
 & \romline{atha saptadaśo'dhyāyaḥ} \\
 & \romline{śraddhātraya-vibhāga yogaḥ}
\end{tabular}
\end{table}

\begin{table}[H]
\begin{tabular}{cl}
\textbf{17.1} & \romline{arjuna uvāca} \\
 & \romline{ye śāstra-vidhimut-sṛjya} \\
 & \romline{yajante śraddha-yānvitāḥ |} \\
 & \romline{teṣāṃ niṣṭhā tu kā kṛṣṇa} \\
 & \romline{sattvamāho rajastamaḥ ||}
\end{tabular}
\end{table}

\begin{table}[H]
\begin{tabular}{cl}
\textbf{17.2} & \romline{śrī bhagavānuvāca} \\
 & \romline{trividhā bhavati śraddhā} \\
 & \romline{dehināṃ sā svabhāvajā |} \\
 & \romline{sāttvikī rājasī caiva} \\
 & \romline{tāmasī ceti tāṃ śṛṇu ||}
\end{tabular}
\end{table}

\begin{table}[H]
\begin{tabular}{cl}
\textbf{17.3} & \romline{sattvānurūpā sarvasya} \\
 & \romline{śraddhā bhavati bhārata |} \\
 & \romline{śrraddhā-mayo'yaṃ puruṣaḥ} \\
 & \romline{yo yaccraddhaḥ sa eva saḥ ||}
\end{tabular}
\end{table}

\begin{table}[H]
\begin{tabular}{cl}
\textbf{17.4} & \romline{yajante sāttvikā devān} \\
 & \romline{yakṣa-rakṣāṃsi rājasāḥ |} \\
 & \romline{pretān bhūta-gaṇāṃścānye} \\
 & \romline{yajante tāmasā janāḥ ||}
\end{tabular}
\end{table}

\begin{table}[H]
\begin{tabular}{cl}
\textbf{17.5} & \romline{aśāstra-vihitaṃ ghoraṃ} \\
 & \romline{tapyate ye tapo janāḥ |} \\
 & \romline{dambhā-hankāra-samyutāḥ} \\
 & \romline{kāma-rāga-balān-vitāḥ ||}
\end{tabular}
\end{table}

\begin{table}[H]
\begin{tabular}{cl}
\textbf{17.6} & \romline{karśayantaḥ śarīrasthaṃ} \\
 & \romline{bhūta-grāmama-cetasaḥ |} \\
 & \romline{māṃ caivāntaḥ śarīrasthaṃ} \\
 & \romline{tān viddhyāsura-niścayān ||}
\end{tabular}
\end{table}

\begin{table}[H]
\begin{tabular}{cl}
\textbf{17.7} & \romline{āhārastvapi sarvasya} \\
 & \romline{trividho bhavati·priyaḥ |} \\
 & \romline{yajñas-tapas-tathā dānaṃ} \\
 & \romline{teṣāṃ bhedamimam śṛṇu ||}
\end{tabular}
\end{table}

