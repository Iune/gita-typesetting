\begin{table}[H]
\begin{tabular}{cl}
\textbf{17.0} & \romline{oṃ śrī paramātmane namaḥ} \\
 & \romline{atha saptadaśo'dhyāyaḥ} \\
 & \romline{śraddhātraya-vibhāga yogaḥ}
\end{tabular}
\end{table}

\begin{table}[H]
\begin{tabular}{cl}
\textbf{17.1} & \romline{arjuna uvāca} \\
 & \romline{ye śāstra-vidhimut-sṛjya} \\
 & \romline{yajante śraddha-yānvitāḥ |} \\
 & \romline{teṣāṃ niṣṭhā tu kā kṛṣṇa} \\
 & \romline{sattvamāho rajastamaḥ ||}
\end{tabular}
\end{table}

\begin{table}[H]
\begin{tabular}{cl}
\textbf{17.2} & \romline{śrī bhagavānuvāca} \\
 & \romline{trividhā bhavati śraddhā} \\
 & \romline{dehināṃ sā svabhāvajā |} \\
 & \romline{sāttvikī rājasī caiva} \\
 & \romline{tāmasī ceti tāṃ śṛṇu ||}
\end{tabular}
\end{table}

\begin{table}[H]
\begin{tabular}{cl}
\textbf{17.3} & \romline{sattvānurūpā sarvasya} \\
 & \romline{śraddhā bhavati bhārata |} \\
 & \romline{śrraddhā-mayo'yaṃ puruṣaḥ} \\
 & \romline{yo yaccraddhaḥ sa eva saḥ ||}
\end{tabular}
\end{table}

\begin{table}[H]
\begin{tabular}{cl}
\textbf{17.4} & \romline{yajante sāttvikā devān} \\
 & \romline{yakṣa-rakṣāṃsi rājasāḥ |} \\
 & \romline{pretān bhūta-gaṇāṃścānye} \\
 & \romline{yajante tāmasā janāḥ ||}
\end{tabular}
\end{table}

\begin{table}[H]
\begin{tabular}{cl}
\textbf{17.5} & \romline{aśāstra-vihitaṃ ghoraṃ} \\
 & \romline{tapyate ye tapo janāḥ |} \\
 & \romline{dambhā-hankāra-samyutāḥ} \\
 & \romline{kāma-rāga-balān-vitāḥ ||}
\end{tabular}
\end{table}

\begin{table}[H]
\begin{tabular}{cl}
\textbf{17.6} & \romline{karśayantaḥ śarīrasthaṃ} \\
 & \romline{bhūta-grāmama-cetasaḥ |} \\
 & \romline{māṃ caivāntaḥ śarīrasthaṃ} \\
 & \romline{tān viddhyāsura-niścayān ||}
\end{tabular}
\end{table}

\begin{table}[H]
\begin{tabular}{cl}
\textbf{17.7} & \romline{āhārastvapi sarvasya} \\
 & \romline{trividho bhavati·priyaḥ |} \\
 & \romline{yajñas-tapas-tathā dānaṃ} \\
 & \romline{teṣāṃ bhedamimam śṛṇu ||}
\end{tabular}
\end{table}

\begin{table}[H]
\begin{tabular}{cl}
\textbf{17.8} & \romline{āyu-ssattva-balā-rogya-} \\
 & \romline{sukha-prīti-vivardhanāḥ |} \\
 & \romline{rasyāḥ snigdhāḥ sthirā hṛdyāḥ} \\
 & \romline{āhārāḥ sāttvika-priyāḥ ||}
\end{tabular}
\end{table}

\begin{table}[H]
\begin{tabular}{cl}
\textbf{17.9} & \romline{kaṭvam-lala-vaṇā-tyuṣṇa-} \\
 & \romline{tīkṣṇa-rūkṣa-vidāhinaḥ |} \\
 & \romline{ahārā rājasas-yeṣṭāḥ} \\
 & \romline{duḥkha-śokā-maya-pradāḥ ||}
\end{tabular}
\end{table}

\begin{table}[H]
\begin{tabular}{cl}
\textbf{17.10} & \romline{yātayāmaṃ gatarasaṃ} \\
 & \romline{pūti paryu-ṣitaṃ ca yat |} \\
 & \romline{ucchiṣṭamapi cāmedhyaṃ} \\
 & \romline{bhojanaṃ tāmasa-priyam ||}
\end{tabular}
\end{table}

\begin{table}[H]
\begin{tabular}{cl}
\textbf{17.11} & \romline{aphalā-kāṅkṣibhir-yajñaḥ} \\
 & \romline{vidhi-dṛṣṭo ya ijyate |} \\
 & \romline{yaṣṭavya-meveti manaḥ} \\
 & \romline{samādhāya sa sāttvikaḥ ||}
\end{tabular}
\end{table}

\begin{table}[H]
\begin{tabular}{cl}
\textbf{17.12} & \romline{abhisandhāya tu phalaṃ} \\
 & \romline{dambhārtha-mapi caiva yat |} \\
 & \romline{ijyate bharata-śreṣṭha} \\
 & \romline{taṃ yajñaṃ viddhi rājasam ||}
\end{tabular}
\end{table}

\begin{table}[H]
\begin{tabular}{cl}
\textbf{17.13} & \romline{vidhi-hīna-masṛṣṭānnaṃ} \\
 & \romline{mantra-hīna-madakṣiṇam |} \\
 & \romline{śraddhā-virahitaṃ yajñaṃ} \\
 & \romline{tāmasaṃ paricakṣate ||}
\end{tabular}
\end{table}

\begin{table}[H]
\begin{tabular}{cl}
\textbf{17.14} & \romline{devadvijaguruprājña-} \\
 & \romline{pū̀janaṃ śauca-mārjavam |} \\
 & \romline{brahmacarya-mahiṃsā ca} \\
 & \romline{śārīraṃ tapa ucyate ||}
\end{tabular}
\end{table}

\begin{table}[H]
\begin{tabular}{cl}
\textbf{17.15} & \romline{anudvegakaraṃ vākyaṃ} \\
 & \romline{satyaṃ priyahitaṃ ca yat |} \\
 & \romline{svādhyāyābhyasanaṃ caiva} \\
 & \romline{vāṅmayaṃ tapa ucyate ||}
\end{tabular}
\end{table}

\begin{table}[H]
\begin{tabular}{cl}
\textbf{17.16} & \romline{manaḥ prasādaḥ saumyatvaṃ} \\
 & \romline{maunamātma-vinigrahaḥ |} \\
 & \romline{bhāva-saṃśuddhi-rityetat} \\
 & \romline{tapo mānasamucyate ||}
\end{tabular}
\end{table}

\begin{table}[H]
\begin{tabular}{cl}
\textbf{17.17} & \romline{śraddhayā parayā taptaṃ} \\
 & \romline{tapastat trividhaṃ naraiḥ |} \\
 & \romline{aphalā-kāṅkṣi-bhir-yuktaiḥ} \\
 & \romline{sattvikaṃ paricakṣate ||}
\end{tabular}
\end{table}

\begin{table}[H]
\begin{tabular}{cl}
\textbf{17.18} & \romline{satkāramāna-pūjārthaṃ} \\
 & \romline{tapo dambhena caiva yat |} \\
 & \romline{kriyate tadiha·proktaṃ} \\
 & \romline{rājasaṃ calamadhruvam ||}
\end{tabular}
\end{table}

\begin{table}[H]
\begin{tabular}{cl}
\textbf{17.19} & \romline{mūḍha-grāheṇā́tmano yat} \\
 & \romline{pīḍayā kriyate tapaḥ |} \\
 & \romline{parrasyot-sādanārthaṃ vā} \\
 & \romline{tattā-masamudāhṟṛtam ||}
\end{tabular}
\end{table}

\begin{table}[H]
\begin{tabular}{cl}
\textbf{17.20} & \romline{dātavyamiti yaddānaṃ} \\
 & \romline{dīyate'nupakāriṇe |} \\
 & \romline{deśe kāle ca pātre ca} \\
 & \romline{taddānaṃ sāttvikaṃ smṛtam ||}
\end{tabular}
\end{table}

\begin{table}[H]
\begin{tabular}{cl}
\textbf{17.21} & \romline{yattu·pratyupa-kārārthaṃ} \\
 & \romline{phala-muddiśya vā punaḥ |} \\
 & \romline{dīyate ca parikliṣṭaṃ} \\
 & \romline{taddānaṃ rājasaṃ smṛtam ||}
\end{tabular}
\end{table}

\begin{table}[H]
\begin{tabular}{cl}
\textbf{17.22} & \romline{adeśakāle yaddānam} \\
 & \romline{apātrebhyaśca dīyate |} \\
 & \romline{asatkṛtamavajñātaṃ} \\
 & \romline{tattā-masamudāhṛtam ||}
\end{tabular}
\end{table}

