\begin{table}[H]
\begin{tabular}{cl}
\textbf{1.0} & \romline{oṃ śrī paramātmane namaḥ} \\
 & \romline{atha prathamo'dhyāyaḥ} \\
 & \romline{arjuna-viṣāda-yogaḥ}
\end{tabular}
\end{table}

\begin{table}[H]
\begin{tabular}{cl}
\textbf{1.1} & \romline{dhṛtarāṣṭra uvāca} \\
 & \romline{dharmakṣetre kurukṣetre} \\
 & \romline{samavetā yuyutsavaḥ |} \\
 & \romline{māmakāḥ pāṇḍavāścaiva} \\
 & \romline{kimakurvata sañjaya ||}
\end{tabular}
\end{table}

\begin{table}[H]
\begin{tabular}{cl}
\textbf{1.2} & \romline{sañjaya uvāca} \\
 & \romline{dṛṣṭvā tu pāṇḍavānīkaṃ} \\
 & \romline{vyūḍhaṃ duryodhanastadā |} \\
 & \romline{ācāryamupasaṅgamya} \\
 & \romline{rājā vacanama-bravīt ||}
\end{tabular}
\end{table}

\begin{table}[H]
\begin{tabular}{cl}
\textbf{1.3} & \romline{paśyaitāṃ pāṇḍu-putrāṇām} \\
 & \romline{ācārya mahatīṃ camūm |} \\
 & \romline{vyūḍhāṃ drupada-putreṇa} \\
 & \romline{tava śiṣyeṇa dhīmatā ||}
\end{tabular}
\end{table}

\begin{table}[H]
\begin{tabular}{cl}
\textbf{1.4} & \romline{atra śūrā maheṣvāsāḥ} \\
 & \romline{bhīmārjuna-samā yudhi |} \\
 & \romline{yuyudhāno virāṭaśca} \\
 & \romline{drupadaśca mahārathaḥ ||}
\end{tabular}
\end{table}

\begin{table}[H]
\begin{tabular}{cl}
\textbf{1.5} & \romline{dhṛṣṭa-ketuścekitānaḥ} \\
 & \romline{kāśirājaśca vīryavān |} \\
 & \romline{purujit-kunti-bhojaśca} \\
 & \romline{śaibyaśca narapuṅgavaḥ ||}
\end{tabular}
\end{table}

\begin{table}[H]
\begin{tabular}{cl}
\textbf{1.6} & \romline{yudhā-manyuśca vikrāntaḥ} \\
 & \romline{uttamaujāśca vīryavān |} \\
 & \romline{saubhadro draupadeyāśca} \\
 & \romline{sarva eva mahārathāḥ ||}
\end{tabular}
\end{table}

\begin{table}[H]
\begin{tabular}{cl}
\textbf{1.7} & \romline{asmākaṃ tu viśiṣṭā ye} \\
 & \romline{tānni-bodha·dvijottama |} \\
 & \romline{nāyakā mama sainyasya} \\
 & \romline{sañjñārthaṃ tānbravīmi te ||}
\end{tabular}
\end{table}

\begin{table}[H]
\begin{tabular}{cl}
\textbf{1.8} & \romline{bhavān-bhīṣmaśca karṇaśca} \\
 & \romline{kṛpaśca samitiñjayaḥ |} \\
 & \romline{aśvatthāmā vikarṇaśca} \\
 & \romline{saumadattista-thaiva ca ||}
\end{tabular}
\end{table}

\begin{table}[H]
\begin{tabular}{cl}
\textbf{1.9} & \romline{anye ca bahavaḥ śūrāḥ} \\
 & \romline{madarthe tyakta-jīvitāḥ |} \\
 & \romline{nānāśastra-praharaṇāḥ} \\
 & \romline{sarve yuddha-viśāradāḥ ||}
\end{tabular}
\end{table}

\begin{table}[H]
\begin{tabular}{cl}
\textbf{1.10} & \romline{aparyāptaṃ tadasmākaṃ} \\
 & \romline{balaṃ bhīṣmā-bhirakṣitam |} \\
 & \romline{paryāptaṃ tvidameteṣāṃ} \\
 & \romline{balaṃ bhīmā-bhirakṣitam ||}
\end{tabular}
\end{table}

\begin{table}[H]
\begin{tabular}{cl}
\textbf{1.11} & \romline{ayaneṣu ca sarveṣu} \\
 & \romline{yathā-bhāgama-vasthitāḥ |} \\
 & \romline{bhīṣmamevā-bhirakṣantu} \\
 & \romline{bhavantaḥ sarva eva hi ||}
\end{tabular}
\end{table}

\begin{table}[H]
\begin{tabular}{cl}
\textbf{1.12} & \romline{tasya sañjana-yanharṣaṃ} \\
 & \romline{kuruvṛddhaḥ pitāmahaḥ |} \\
 & \romline{siṃhanādaṃ vina-dyoccaiḥ} \\
 & \romline{śaṅkhaṃ dadhmau pratāpavān ||}
\end{tabular}
\end{table}

\begin{table}[H]
\begin{tabular}{cl}
\textbf{1.13} & \romline{tataḥ śaṅkhāśca bheryaśca} \\
 & \romline{paṇavānaka-gomukhāḥ |} \\
 & \romline{saha-saivābhya-hanyanta} \\
 & \romline{sa śabda-stumulo'bhavat ||}
\end{tabular}
\end{table}

\begin{table}[H]
\begin{tabular}{cl}
\textbf{1.14} & \romline{tataḥ śvetair-hayair-yukte} \\
 & \romline{mahati syandane sthitau |} \\
 & \romline{mādhavaḥ pāṇḍavaścaiva} \\
 & \romline{divyau śaṅkhau pradadhmatuḥ ||}
\end{tabular}
\end{table}

\begin{table}[H]
\begin{tabular}{cl}
\textbf{1.15} & \romline{pāñca-janyaṃ hṛṣīkeśaḥ} \\
 & \romline{devadattaṃ dhanañjayaḥ |} \\
 & \romline{pauṇḍraṃ dadhmau mahā-śaṅkhaṃ} \\
 & \romline{bhīma-karmā vṛkodaraḥ ||}
\end{tabular}
\end{table}

\begin{table}[H]
\begin{tabular}{cl}
\textbf{1.16} & \romline{ananta-vijayaṃ rājā} \\
 & \romline{kuntīputro yudhiṣṭhiraḥ |} \\
 & \romline{nakulaḥ saha-devaśca} \\
 & \romline{sugho-ṣamaṇi-puṣpakau ||}
\end{tabular}
\end{table}

\begin{table}[H]
\begin{tabular}{cl}
\textbf{1.17} & \romline{kāśyaśca parameṣvāsaḥ} \\
 & \romline{śikhaṇḍī ca mahārathaḥ |} \\
 & \romline{dhṛṣṭadyumno virāṭaśca} \\
 & \romline{sātyakiścāparājitaḥ ||}
\end{tabular}
\end{table}

\begin{table}[H]
\begin{tabular}{cl}
\textbf{1.18} & \romline{drupado draupadeyāśca} \\
 & \romline{sarvaśaḥ pṛthivīpate |} \\
 & \romline{soubhadraśca mahābāhuḥ} \\
 & \romline{śaṅkhān-dadhmuḥ pṛthak-pṛthak ||}
\end{tabular}
\end{table}

\begin{table}[H]
\begin{tabular}{cl}
\textbf{1.19} & \romline{sa ghoṣo dhārta-rāṣṭrāṇāṃ} \\
 & \romline{hṛdayāni·vyadārayat |} \\
 & \romline{nabhaśca pṛthivīṃ caiva} \\
 & \romline{tumulo vyanunādayan ||}
\end{tabular}
\end{table}

\begin{table}[H]
\begin{tabular}{cl}
\textbf{1.20} & \romline{atha·vyavasthitān-dṛṣṭvā} \\
 & \romline{dhārta-rāṣṭrān kapidhvajaḥ |} \\
 & \romline{pravṛtte śastra-sampāte} \\
 & \romline{dhanurudyamya pāṇḍavaḥ ||}
\end{tabular}
\end{table}

\begin{table}[H]
\begin{tabular}{cl}
\textbf{1.21} & \romline{hṛṣīkeśaṃ tadā vākyam} \\
 & \romline{idamāha mahīpate} \\
 & \romline{arjuna uvāca |} \\
 & \romline{senayor-ubhayor-madhye} \\
 & \romline{rathaṃ sthāpaya me'cyuta ||}
\end{tabular}
\end{table}

\begin{table}[H]
\begin{tabular}{cl}
\textbf{1.22} & \romline{yāvadetān-nirīkṣe'haṃ} \\
 & \romline{yoddhu-kāmāna-vasthitān |} \\
 & \romline{kairmayā saha yoddhavyam} \\
 & \romline{asmin raṇa-samudyame ||}
\end{tabular}
\end{table}

\begin{table}[H]
\begin{tabular}{cl}
\textbf{1.23} & \romline{yotsya-mānā-navekṣe'haṃ} \\
 & \romline{ya ete'tra samāgatāḥ |} \\
 & \romline{dhārta-rāṣṭrasya-durbuddheḥ} \\
 & \romline{yuddhe priya-cikīrṣavaḥ ||}
\end{tabular}
\end{table}

\begin{table}[H]
\begin{tabular}{cl}
\textbf{1.24} & \romline{sañjaya uvāca} \\
 & \romline{evamukto hṛṣīkeśaḥ} \\
 & \romline{guḍākeśena bhārata |} \\
 & \romline{senayor-ubhayor-madhye} \\
 & \romline{sthāpayitvā rathottamam ||}
\end{tabular}
\end{table}

\begin{table}[H]
\begin{tabular}{cl}
\textbf{1.25} & \romline{bhīṣma-droṇa-pramukhataḥ} \\
 & \romline{sarveṣāṃ ca mahīkṣitām |} \\
 & \romline{uvāca pārtha paśyaitān} \\
 & \romline{sama-vetān-kurūniti ||}
\end{tabular}
\end{table}

\begin{table}[H]
\begin{tabular}{cl}
\textbf{1.26} & \romline{tatrā-paśyat-sthitān-pārthaḥ} \\
 & \romline{pitṝnatha pitāmahān |} \\
 & \romline{ācāryān-mātulān-bhrātṝn} \\
 & \romline{putrān-pautrān-sakhīṃ-stathā ||}
\end{tabular}
\end{table}

\begin{table}[H]
\begin{tabular}{cl}
\textbf{1.27} & \romline{śvaśurān-suhṛdaścaiva} \\
 & \romline{senayor-ubhayorapi |} \\
 & \romline{tānsamīkṣya sa kaunteyaḥ} \\
 & \romline{sarvān-bandhūna-vasthitān ||}
\end{tabular}
\end{table}

\begin{table}[H]
\begin{tabular}{cl}
\textbf{1.28} & \romline{kṛpayā parayā''viṣṭaḥ} \\
 & \romline{viṣīdannidama-bravīt} \\
 & \romline{arjuna uvāca |} \\
 & \romline{dṛṣṭvemaṃ svajanaṃ kṛṣṇa} \\
 & \romline{yuyutsuṃ samupasthitam ||}
\end{tabular}
\end{table}

\begin{table}[H]
\begin{tabular}{cl}
\textbf{1.29} & \romline{sīdanti mama gātrāṇi} \\
 & \romline{mukhaṃ ca pariśuṣyati |} \\
 & \romline{vepathuśca śarīre me} \\
 & \romline{romaharṣaśca jāyate ||}
\end{tabular}
\end{table}

\begin{table}[H]
\begin{tabular}{cl}
\textbf{1.30} & \romline{gāṇḍīvaṃ sraṃsate hastāt} \\
 & \romline{tvakcaiva paridahyate |} \\
 & \romline{na ca śaknomyavasthātuṃ} \\
 & \romline{bhramatīva ca me manaḥ ||}
\end{tabular}
\end{table}

\begin{table}[H]
\begin{tabular}{cl}
\textbf{1.31} & \romline{nimittāni ca paśyāmi} \\
 & \romline{viparītāni keśava |} \\
 & \romline{na ca śreyo'nupaśyāmi} \\
 & \romline{hatvā svajanamāhave ||}
\end{tabular}
\end{table}

\begin{table}[H]
\begin{tabular}{cl}
\textbf{1.32} & \romline{na kāṅkṣe vijayaṃ kṛṣṇa} \\
 & \romline{na ca rājyaṃ sukhāni ca |} \\
 & \romline{kiṃ no rājyena govinda} \\
 & \romline{kiṃ bhogairjīvitena vā ||}
\end{tabular}
\end{table}

\begin{table}[H]
\begin{tabular}{cl}
\textbf{1.33} & \romline{yeṣāmarthe kāṅkṣitaṃ naḥ} \\
 & \romline{rājyaṃ bhogāḥ sukhāni ca |} \\
 & \romline{ta ime'vasthitā yuddhe} \\
 & \romline{prāṇāṃstyaktvā dhanāni ca ||}
\end{tabular}
\end{table}

\begin{table}[H]
\begin{tabular}{cl}
\textbf{1.34} & \romline{ācāryāḥ pitaraḥ putrāḥ} \\
 & \romline{tathaiva ca pitāmahāḥ |} \\
 & \romline{mātulāḥ śvaśurāḥ pautrāḥ} \\
 & \romline{śyālāḥ sambandhinastathā ||}
\end{tabular}
\end{table}

\begin{table}[H]
\begin{tabular}{cl}
\textbf{1.35} & \romline{etānna hantumiccāmi} \\
 & \romline{ghnato'pi madhusūdana |} \\
 & \romline{api trailokyarājyasya} \\
 & \romline{hetoḥ kiṃ nu mahīkṛte ||}
\end{tabular}
\end{table}

\begin{table}[H]
\begin{tabular}{cl}
\textbf{1.36} & \romline{nihatya dhārtarāṣṭrānnaḥ} \\
 & \romline{kā prītiḥ syājjanārdana |} \\
 & \romline{pāpamevāśrayedasmān} \\
 & \romline{hatvaitānātatāyinaḥ ||}
\end{tabular}
\end{table}

\begin{table}[H]
\begin{tabular}{cl}
\textbf{1.37} & \romline{tasmānnārhā vayaṃ hantuṃ} \\
 & \romline{dhārtarāṣṭrānsvabāndhavān |} \\
 & \romline{svajanaṃ hi kathaṃ hatvā} \\
 & \romline{sukhinaḥ syāma mādhava ||}
\end{tabular}
\end{table}

\begin{table}[H]
\begin{tabular}{cl}
\textbf{1.38} & \romline{yadyapyete na paśyanti} \\
 & \romline{obhopahatacetasaḥ |} \\
 & \romline{kulakṣayakṛtaṃ doṣaṃ} \\
 & \romline{mitradrohe ca pātakam ||}
\end{tabular}
\end{table}

\begin{table}[H]
\begin{tabular}{cl}
\textbf{1.39} & \romline{kathaṃ na jñeyamasmābhiḥ} \\
 & \romline{pāpādasmānnivartitum |} \\
 & \romline{kulakṣayakṛtaṃ doṣaṃ} \\
 & \romline{prapaśyadbhirjanārdana ||}
\end{tabular}
\end{table}

\begin{table}[H]
\begin{tabular}{cl}
\textbf{1.40} & \romline{kulakṣaye praṇaśyanti} \\
 & \romline{kuladharmāḥ sanātanāḥ |} \\
 & \romline{dharme naṣṭe kulaṃ kṛtsnam} \\
 & \romline{adharmo'bhibhavatyuta ||}
\end{tabular}
\end{table}

\begin{table}[H]
\begin{tabular}{cl}
\textbf{1.41} & \romline{adharmābhibhavātkṛṣṇa} \\
 & \romline{praduṣyanti kulastriyaḥ |} \\
 & \romline{strīṣu duṣṭāsu vārṣṇeya} \\
 & \romline{jāyate varṇasaṅkaraḥ ||}
\end{tabular}
\end{table}

\begin{table}[H]
\begin{tabular}{cl}
\textbf{1.42} & \romline{saṅkaro narakāyaiva} \\
 & \romline{kulaghnānāṃ kulasya ca |} \\
 & \romline{patanti pitaro hyeṣāṃ} \\
 & \romline{luptapiṇḍodakakriyāḥ ||}
\end{tabular}
\end{table}

\begin{table}[H]
\begin{tabular}{cl}
\textbf{1.43} & \romline{doṣairetaiḥ kulaghnānāṃ} \\
 & \romline{varṇasaṅkarakārakaiḥ |} \\
 & \romline{utsādyante jātidharmāḥ} \\
 & \romline{kuladharmāśca śāśvatāḥ ||}
\end{tabular}
\end{table}

\begin{table}[H]
\begin{tabular}{cl}
\textbf{1.44} & \romline{utsannakuladharmāṇāṃ} \\
 & \romline{manuṣyāṇāṃ janārdana |} \\
 & \romline{narake'niyataṃ vāsaḥ} \\
 & \romline{bhavatītyanuśuśruma ||}
\end{tabular}
\end{table}

\begin{table}[H]
\begin{tabular}{cl}
\textbf{1.45} & \romline{aho bata mahatpāpaṃ} \\
 & \romline{kartuṃ vyavasitā vayam |} \\
 & \romline{yadrājyasukhalobhena} \\
 & \romline{hantuṃ svajanamudyatāḥ ||}
\end{tabular}
\end{table}

\begin{table}[H]
\begin{tabular}{cl}
\textbf{1.46} & \romline{yadi māmapratīkāram} \\
 & \romline{aśastraṃ śastrapāṇayaḥ |} \\
 & \romline{dhārtarāṣṭrā raṇe hanyuḥ} \\
 & \romline{tanme kṣemataraṃ bhavet ||}
\end{tabular}
\end{table}

\begin{table}[H]
\begin{tabular}{cl}
\textbf{1.47} & \romline{sañjaya uvāca} \\
 & \romline{evamuktvā'rjunaḥ saṅkhye} \\
 & \romline{rathopastha upāviśat |} \\
 & \romline{visṛjya saśaraṃ cāpaṃ} \\
 & \romline{śokasaṃvignamānasaḥ ||}
\end{tabular}
\end{table}

