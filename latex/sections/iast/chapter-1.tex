\begin{table}[H]
\begin{tabular}{cl}
\textbf{1.0} & \romline{oṃ śrī paramātmane namaḥ} \\
 & \romline{atha prathamo'dhyāyaḥ} \\
 & \romline{arjuna-viṣāda-yogaḥ}
\end{tabular}
\end{table}

\begin{table}[H]
\begin{tabular}{cl}
\textbf{1.1} & \romline{dhṛtarāṣṭra uvāca} \\
 & \romline{dharmakṣetre kurukṣetre} \\
 & \romline{samavetā yuyutsavaḥ |} \\
 & \romline{māmakāḥ pāṇḍavāścaiva} \\
 & \romline{kimakurvata sañjaya ||}
\end{tabular}
\end{table}

\begin{table}[H]
\begin{tabular}{cl}
\textbf{1.2} & \romline{sañjaya uvāca} \\
 & \romline{dṛṣṭvā tu pāṇḍavānīkaṃ} \\
 & \romline{vyūḍhaṃ duryodhanastadā |} \\
 & \romline{ācāryamupasaṅgamya} \\
 & \romline{rājā vacanama-bravīt ||}
\end{tabular}
\end{table}

\begin{table}[H]
\begin{tabular}{cl}
\textbf{1.3} & \romline{paśyaitāṃ pāṇḍu-putrāṇām} \\
 & \romline{ācārya mahatīṃ camūm |} \\
 & \romline{vyūḍhāṃ drupada-putreṇa} \\
 & \romline{tava śiṣyeṇa dhīmatā ||}
\end{tabular}
\end{table}

\begin{table}[H]
\begin{tabular}{cl}
\textbf{1.4} & \romline{atra śūrā maheṣvāsāḥ} \\
 & \romline{bhīmārjuna-samā yudhi |} \\
 & \romline{yuyudhāno virāṭaśca} \\
 & \romline{drupadaśca mahārathaḥ ||}
\end{tabular}
\end{table}

\begin{table}[H]
\begin{tabular}{cl}
\textbf{1.5} & \romline{dhṛṣṭa-ketuścekitānaḥ} \\
 & \romline{kāśirājaśca vīryavān |} \\
 & \romline{purujit-kunti-bhojaśca} \\
 & \romline{śaibyaśca narapuṅgavaḥ ||}
\end{tabular}
\end{table}

\begin{table}[H]
\begin{tabular}{cl}
\textbf{1.6} & \romline{yudhā-manyuśca vikrāntaḥ} \\
 & \romline{uttamaujāśca vīryavān |} \\
 & \romline{saubhadro draupadeyāśca} \\
 & \romline{sarva eva mahārathāḥ ||}
\end{tabular}
\end{table}

\begin{table}[H]
\begin{tabular}{cl}
\textbf{1.7} & \romline{asmākaṃ tu viśiṣṭā ye} \\
 & \romline{tānni-bodha·dvijottama |} \\
 & \romline{nāyakā mama sainyasya} \\
 & \romline{sañjñārthaṃ tānbravīmi te ||}
\end{tabular}
\end{table}

\begin{table}[H]
\begin{tabular}{cl}
\textbf{1.8} & \romline{bhavān-bhīṣmaśca karṇaśca} \\
 & \romline{kṛpaśca samitiñjayaḥ |} \\
 & \romline{aśvatthāmā vikarṇaśca} \\
 & \romline{saumadattista-thaiva ca ||}
\end{tabular}
\end{table}

\begin{table}[H]
\begin{tabular}{cl}
\textbf{1.9} & \romline{anye ca bahavaḥ śūrāḥ} \\
 & \romline{madarthe tyakta-jīvitāḥ |} \\
 & \romline{nānāśastra-praharaṇāḥ} \\
 & \romline{sarve yuddha-viśāradāḥ ||}
\end{tabular}
\end{table}

\begin{table}[H]
\begin{tabular}{cl}
\textbf{1.10} & \romline{aparyāptaṃ tadasmākaṃ} \\
 & \romline{balaṃ bhīṣmā-bhirakṣitam |} \\
 & \romline{paryāptaṃ tvidameteṣāṃ} \\
 & \romline{balaṃ bhīmā-bhirakṣitam ||}
\end{tabular}
\end{table}

\begin{table}[H]
\begin{tabular}{cl}
\textbf{1.11} & \romline{ayaneṣu ca sarveṣu} \\
 & \romline{yathā-bhāgama-vasthitāḥ |} \\
 & \romline{bhīṣmamevā-bhirakṣantu} \\
 & \romline{bhavantaḥ sarva eva hi ||}
\end{tabular}
\end{table}

\begin{table}[H]
\begin{tabular}{cl}
\textbf{1.12} & \romline{tasya sañjana-yanharṣaṃ} \\
 & \romline{kuruvṛddhaḥ pitāmahaḥ |} \\
 & \romline{siṃhanādaṃ vina-dyoccaiḥ} \\
 & \romline{śaṅkhaṃ dadhmau pratāpavān ||}
\end{tabular}
\end{table}

\begin{table}[H]
\begin{tabular}{cl}
\textbf{1.13} & \romline{tataḥ śaṅkhāśca bheryaśca} \\
 & \romline{paṇavānaka-gomukhāḥ |} \\
 & \romline{saha-saivābhya-hanyanta} \\
 & \romline{sa śabda-stumulo'bhavat ||}
\end{tabular}
\end{table}

\begin{table}[H]
\begin{tabular}{cl}
\textbf{1.14} & \romline{tataḥ śvetair-hayair-yukte} \\
 & \romline{mahati syandane sthitau |} \\
 & \romline{mādhavaḥ pāṇḍavaścaiva} \\
 & \romline{divyau śaṅkhau pradadhmatuḥ ||}
\end{tabular}
\end{table}

\begin{table}[H]
\begin{tabular}{cl}
\textbf{1.15} & \romline{pāñca-janyaṃ hṛṣīkeśaḥ} \\
 & \romline{devadattaṃ dhanañjayaḥ |} \\
 & \romline{pauṇḍraṃ dadhmau mahā-śaṅkhaṃ} \\
 & \romline{bhīma-karmā vṛkodaraḥ ||}
\end{tabular}
\end{table}

\begin{table}[H]
\begin{tabular}{cl}
\textbf{1.16} & \romline{ananta-vijayaṃ rājā} \\
 & \romline{kuntīputro yudhiṣṭhiraḥ |} \\
 & \romline{nakulaḥ saha-devaśca} \\
 & \romline{sugho-ṣamaṇi-puṣpakau ||}
\end{tabular}
\end{table}

\begin{table}[H]
\begin{tabular}{cl}
\textbf{1.17} & \romline{kāśyaśca parameṣvāsaḥ} \\
 & \romline{śikhaṇḍī ca mahārathaḥ |} \\
 & \romline{dhṛṣṭadyumno virāṭaśca} \\
 & \romline{sātyakiścāparājitaḥ ||}
\end{tabular}
\end{table}

\begin{table}[H]
\begin{tabular}{cl}
\textbf{1.18} & \romline{drupado draupadeyāśca} \\
 & \romline{sarvaśaḥ pṛthivīpate |} \\
 & \romline{soubhadraśca mahābāhuḥ} \\
 & \romline{śaṅkhān-dadhmuḥ pṛthak-pṛthak ||}
\end{tabular}
\end{table}

\begin{table}[H]
\begin{tabular}{cl}
\textbf{1.19} & \romline{sa ghoṣo dhārta-rāṣṭrāṇāṃ} \\
 & \romline{hṛdayāni·vyadārayat |} \\
 & \romline{nabhaśca pṛthivīṃ caiva} \\
 & \romline{tumulo vyanunādayan ||}
\end{tabular}
\end{table}

\begin{table}[H]
\begin{tabular}{cl}
\textbf{1.20} & \romline{atha·vyavasthitāndṛṣṭvā} \\
 & \romline{dhārta-rāṣṭrān kapidhvajaḥ |} \\
 & \romline{pravṛtte śastra-sampāte} \\
 & \romline{dhanurudyamya pāṇḍavaḥ ||}
\end{tabular}
\end{table}

\begin{table}[H]
\begin{tabular}{cl}
\textbf{1.21} & \romline{hṛṣīkeśaṃ tadā vākyam} \\
 & \romline{idamāha mahīpate} \\
 & \romline{arjuna uvāca |} \\
 & \romline{senayor-ubhayor-madhye} \\
 & \romline{rathaṃ sthāpaya me'cyuta ||}
\end{tabular}
\end{table}

\begin{table}[H]
\begin{tabular}{cl}
\textbf{1.22} & \romline{yāvadetānnirīkṣe'haṃ} \\
 & \romline{yoddhukāmānavasthitān |} \\
 & \romline{kairmayā saha yoddhavyam} \\
 & \romline{asmin raṇasamudyame ||}
\end{tabular}
\end{table}

\begin{table}[H]
\begin{tabular}{cl}
\textbf{1.23} & \romline{yotsyamānānavekṣe'haṃ} \\
 & \romline{ya ete'tra samāgatāḥ |} \\
 & \romline{dhārtarāṣṭrasyadurbuddheḥ} \\
 & \romline{yuddhe priyacikīrṣavaḥ ||}
\end{tabular}
\end{table}

