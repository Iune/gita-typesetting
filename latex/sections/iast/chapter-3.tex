\begin{table}[H]
\begin{tabular}{cl}
\textbf{3.0} & \romline{oṃ śrī paramātmane namaḥ} \\
 & \romline{atha tṛtīyo'dhyāyaḥ} \\
 & \romline{karma-yogaḥ}
\end{tabular}
\end{table}

\begin{table}[H]
\begin{tabular}{cl}
\textbf{3.1} & \romline{arjuna uvāca} \\
 & \romline{jyāyasī cetkarmaṇaste} \\
 & \romline{matā buddhirjanārdana |} \\
 & \romline{tatkiṃ karmaṇi ghore māṃ} \\
 & \romline{niyojayasi keśava ||}
\end{tabular}
\end{table}

\begin{table}[H]
\begin{tabular}{cl}
\textbf{3.2} & \romline{vyāmiśreṇeva vākyena} \\
 & \romline{buddhiṃ mohayasīva me |} \\
 & \romline{tadekaṃ vada niścitya} \\
 & \romline{yena śreyo'hamāpnuyām ||}
\end{tabular}
\end{table}

\begin{table}[H]
\begin{tabular}{cl}
\textbf{3.3} & \romline{srī bhagavānuvāca} \\
 & \romline{loke'smin-dvividhā niṣṭhā} \\
 & \romline{purā proktā mayā'nagha |} \\
 & \romline{jñānayogena sāṅkhyānāṃ} \\
 & \romline{karmayogena yoginām ||}
\end{tabular}
\end{table}

\begin{table}[H]
\begin{tabular}{cl}
\textbf{3.4} & \romline{na karmaṇāmanārambhāt} \\
 & \romline{naiṣkarmyaṃ puruṣo'śnute |} \\
 & \romline{na ca sannyasanādeva} \\
 & \romline{siddhiṃ samadhigacchati ||}
\end{tabular}
\end{table}

\begin{table}[H]
\begin{tabular}{cl}
\textbf{3.5} & \romline{na hi kaścit-kṣaṇamapi} \\
 & \romline{jātu tiṣṭhatya-karmakṛt |} \\
 & \romline{kāryate hyavaśaḥ karma} \\
 & \romline{sarvaḥ prakṛti-jairguṇaiḥ ||}
\end{tabular}
\end{table}

\begin{table}[H]
\begin{tabular}{cl}
\textbf{3.6} & \romline{karmendriyāṇi saṃyamya} \\
 & \romline{ya āste manasā smaran |} \\
 & \romline{indriyārthān-vimūḍhātmā} \\
 & \romline{mithyācāraḥ sa ucyate ||}
\end{tabular}
\end{table}

\begin{table}[H]
\begin{tabular}{cl}
\textbf{3.7} & \romline{yastvindriyāṇi manasā} \\
 & \romline{niyamyārabhate'rjuna |} \\
 & \romline{karmendriyaiḥ karmayogam} \\
 & \romline{asaktaḥ sa viśiṣyate ||}
\end{tabular}
\end{table}

\begin{table}[H]
\begin{tabular}{cl}
\textbf{3.8} & \romline{niyataṃ kuru karma tvaṃ} \\
 & \romline{karma jyāyo hyakarmaṇaḥ |} \\
 & \romline{śarīra-yātrā'pi ca te} \\
 & \romline{na prasiddhyeda-karmaṇaḥ ||}
\end{tabular}
\end{table}

\begin{table}[H]
\begin{tabular}{cl}
\textbf{3.9} & \romline{yajñārthāt-karmaṇo'nyatra} \\
 & \romline{loko'yaṃ karmabandhanaḥ |} \\
 & \romline{tadarthaṃ karma kaunteya} \\
 & \romline{muktasaṅgaḥ samācara ||}
\end{tabular}
\end{table}

\begin{table}[H]
\begin{tabular}{cl}
\textbf{3.10} & \romline{sahayajñāḥ prajāḥ sṛṣṭvā} \\
 & \romline{purovāca prajāpatiḥ |} \\
 & \romline{anena prasaviṣyadhvaṃ} \\
 & \romline{eṣa vo'stviṣṭakāmadhuk ||}
\end{tabular}
\end{table}

\begin{table}[H]
\begin{tabular}{cl}
\textbf{3.11} & \romline{devānbhāvayatā'nena} \\
 & \romline{te devā bhāvayantu vaḥ |} \\
 & \romline{parasparaṃ bhāvayantaḥ} \\
 & \romline{śreyaḥ paramavāpsyatha ||}
\end{tabular}
\end{table}

\begin{table}[H]
\begin{tabular}{cl}
\textbf{3.12} & \romline{iṣṭānbhogānhi vo devāḥ} \\
 & \romline{dāsyante yajñabhāvitāḥ |} \\
 & \romline{tairdattānapradāyaibhyaḥ} \\
 & \romline{yo bhuṅkte stena eva saḥ ||}
\end{tabular}
\end{table}

\begin{table}[H]
\begin{tabular}{cl}
\textbf{3.13} & \romline{yajñaśiṣṭāśinaḥ santaḥ} \\
 & \romline{mucyante sarvakilbiṣaiḥ |} \\
 & \romline{bhuñjate te tvaghaṃ pāpāḥ} \\
 & \romline{ye pacantyātmakāraṇāt ||}
\end{tabular}
\end{table}

\begin{table}[H]
\begin{tabular}{cl}
\textbf{3.14} & \romline{annādbhavanti bhūtāni} \\
 & \romline{parjanyādannasambhavaḥ |} \\
 & \romline{yajñādbhavati parjanyaḥ} \\
 & \romline{yajñaḥ karmasamudbhavaḥ ||}
\end{tabular}
\end{table}

\begin{table}[H]
\begin{tabular}{cl}
\textbf{3.15} & \romline{karma brahmodbhavaṃ viddhi} \\
 & \romline{brahmākṣarasamudbhavam |} \\
 & \romline{tasmātsarvagataṃ brahma} \\
 & \romline{nityaṃ yajñe pratiṣṭhitam ||}
\end{tabular}
\end{table}

\begin{table}[H]
\begin{tabular}{cl}
\textbf{3.16} & \romline{evaṃ pravartitaṃ cakraṃ} \\
 & \romline{nānuvartayatīha yaḥ |} \\
 & \romline{aghāyurindriyārāmaḥ} \\
 & \romline{moghaṃ pārtha sa jīvati ||}
\end{tabular}
\end{table}

\begin{table}[H]
\begin{tabular}{cl}
\textbf{3.17} & \romline{yastvātmaratireva syāt} \\
 & \romline{ātmatṛptaśca mānavaḥ |} \\
 & \romline{ātmanyeva ca santuṣṭaḥ} \\
 & \romline{tasya kāryaṃ na vidyate ||}
\end{tabular}
\end{table}

\begin{table}[H]
\begin{tabular}{cl}
\textbf{3.18} & \romline{naiva tasya kṛtenārthaḥ} \\
 & \romline{nākṛteneha kaścana |} \\
 & \romline{na cāsya sarvabhūteṣu} \\
 & \romline{kaścidarthavyapāśrayaḥ ||}
\end{tabular}
\end{table}

\begin{table}[H]
\begin{tabular}{cl}
\textbf{3.19} & \romline{tasmādasaktaḥ satataṃ} \\
 & \romline{kāryaṃ karma samācara |} \\
 & \romline{asakto hyācarankarma} \\
 & \romline{paramāpnoti pūruṣaḥ ||}
\end{tabular}
\end{table}

\begin{table}[H]
\begin{tabular}{cl}
\textbf{3.20} & \romline{karmaṇaiva hi saṃsiddhiṃ} \\
 & \romline{āsthitā janakādayaḥ |} \\
 & \romline{lokasaṅgrahamevāpi} \\
 & \romline{sampaśyankartumarhasi ||}
\end{tabular}
\end{table}

