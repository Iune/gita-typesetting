\begin{table}[H]
\begin{tabular}{cl}
 & \romline{śrī paramātmane namaḥ} \\
 & \romline{atha tṛtīyo'dhyāyaḥ} \\
 & \romline{karmayogaḥ}
\end{tabular}
\end{table}

\begin{table}[H]
\begin{tabular}{cl}
\textbf{3.1} & \romline{arjuna uvāca} \\
 & \romline{jyāyasī cetkarmaṇaste} \\
 & \romline{matā buddhirjanārdana |} \\
 & \romline{tatkiṃ karmaṇi ghore māṃ} \\
 & \romline{niyojayasi keśava ||}
\end{tabular}
\end{table}

\begin{table}[H]
\begin{tabular}{cl}
\textbf{3.2} & \romline{vyāmiśreṇeva vākyena} \\
 & \romline{buddhiṃ mohayasīva me |} \\
 & \romline{tadekaṃ vada niścitya} \\
 & \romline{yena śreyo'hamāpnuyām ||}
\end{tabular}
\end{table}

\begin{table}[H]
\begin{tabular}{cl}
\textbf{3.3} & \romline{srī bhagavānuvāca} \\
 & \romline{loke'smindvividhā niṣṭhā} \\
 & \romline{purā proktā mayā'nagha |} \\
 & \romline{jñānayogena sāṅkhyānāṃ} \\
 & \romline{karmayogena yoginām ||}
\end{tabular}
\end{table}

\begin{table}[H]
\begin{tabular}{cl}
\textbf{3.4} & \romline{na karmaṇāmanārambhāt} \\
 & \romline{naiṣkarmyaṃ puruṣo'śnute |} \\
 & \romline{na ca sannyasanādeva} \\
 & \romline{siddhiṃ samadhigacchati ||}
\end{tabular}
\end{table}

\begin{table}[H]
\begin{tabular}{cl}
\textbf{3.5} & \romline{na hi kaścitkṣaṇamapi} \\
 & \romline{jātu tiṣṭhatyakarmakṛt |} \\
 & \romline{kāryate hyavaśaḥ karma} \\
 & \romline{sarvaḥ prakṛtijairguṇaiḥ ||}
\end{tabular}
\end{table}

\begin{table}[H]
\begin{tabular}{cl}
\textbf{3.6} & \romline{karmendriyāṇi saṃyamya} \\
 & \romline{ya āste manasā smaran |} \\
 & \romline{indriyārthānvimūḍhātmā} \\
 & \romline{mithyācāraḥ sa ucyate ||}
\end{tabular}
\end{table}

\begin{table}[H]
\begin{tabular}{cl}
\textbf{3.7} & \romline{yastvindriyāṇi manasā} \\
 & \romline{niyamyārabhate'rjuna |} \\
 & \romline{karmendriyaiḥ karmayogam} \\
 & \romline{asaktaḥ sa viśiṣyate ||}
\end{tabular}
\end{table}

\begin{table}[H]
\begin{tabular}{cl}
\textbf{3.8} & \romline{niyataṃ kuru karma tvaṃ} \\
 & \romline{karma jyāyo hyakarmaṇaḥ |} \\
 & \romline{śarīrayātrā'pi ca te} \\
 & \romline{na prasiddhyedakarmaṇaḥ ||}
\end{tabular}
\end{table}

\begin{table}[H]
\begin{tabular}{cl}
\textbf{3.9} & \romline{yajñārthātkarmaṇo'nyatra} \\
 & \romline{loko'yaṃ karmabandhanaḥ |} \\
 & \romline{tadarthaṃ karma kaunteya} \\
 & \romline{muktasaṅgaḥ samācara ||}
\end{tabular}
\end{table}

\begin{table}[H]
\begin{tabular}{cl}
\textbf{3.10} & \romline{sahayajñāḥ prajāḥ sṛṣṭvā} \\
 & \romline{purovāca prajāpatiḥ |} \\
 & \romline{anena prasaviṣyadhvaṃ} \\
 & \romline{eṣa vo'stviṣṭakāmadhuk ||}
\end{tabular}
\end{table}

\begin{table}[H]
\begin{tabular}{cl}
\textbf{3.11} & \romline{devānbhāvayatā'nena} \\
 & \romline{te devā bhāvayantu vaḥ |} \\
 & \romline{parasparaṃ bhāvayantaḥ} \\
 & \romline{śreyaḥ paramavāpsyatha ||}
\end{tabular}
\end{table}

\begin{table}[H]
\begin{tabular}{cl}
\textbf{3.12} & \romline{iṣṭānbhogānhi vo devāḥ} \\
 & \romline{dāsyante yajñabhāvitāḥ |} \\
 & \romline{tairdattānapradāyaibhyaḥ} \\
 & \romline{yo bhuṅkte stena eva saḥ ||}
\end{tabular}
\end{table}

\begin{table}[H]
\begin{tabular}{cl}
\textbf{3.13} & \romline{yajñaśiṣṭāśinaḥ santaḥ} \\
 & \romline{mucyante sarvakilbiṣaiḥ |} \\
 & \romline{bhuñjate te tvaghaṃ pāpāḥ} \\
 & \romline{ye pacantyātmakāraṇāt ||}
\end{tabular}
\end{table}

\begin{table}[H]
\begin{tabular}{cl}
\textbf{3.14} & \romline{annādbhavanti bhūtāni} \\
 & \romline{parjanyādannasambhavaḥ |} \\
 & \romline{yajñādbhavati parjanyaḥ} \\
 & \romline{yajñaḥ karmasamudbhavaḥ ||}
\end{tabular}
\end{table}

\begin{table}[H]
\begin{tabular}{cl}
\textbf{3.15} & \romline{karma brahmodbhavaṃ viddhi} \\
 & \romline{brahmākṣarasamudbhavam |} \\
 & \romline{tasmātsarvagataṃ brahma} \\
 & \romline{nityaṃ yajñe pratiṣṭhitam ||}
\end{tabular}
\end{table}

\begin{table}[H]
\begin{tabular}{cl}
\textbf{3.16} & \romline{evaṃ pravartitaṃ cakraṃ} \\
 & \romline{nānuvartayatīha yaḥ |} \\
 & \romline{aghāyurindriyārāmaḥ} \\
 & \romline{moghaṃ pārtha sa jīvati ||}
\end{tabular}
\end{table}

\begin{table}[H]
\begin{tabular}{cl}
\textbf{3.17} & \romline{yastvātmaratireva syāt} \\
 & \romline{ātmatṛptaśca mānavaḥ |} \\
 & \romline{ātmanyeva ca santuṣṭaḥ} \\
 & \romline{tasya kāryaṃ na vidyate ||}
\end{tabular}
\end{table}

\begin{table}[H]
\begin{tabular}{cl}
\textbf{3.18} & \romline{naiva tasya kṛtenārthaḥ} \\
 & \romline{nākṛteneha kaścana |} \\
 & \romline{na cāsya sarvabhūteṣu} \\
 & \romline{kaścidarthavyapāśrayaḥ ||}
\end{tabular}
\end{table}

\begin{table}[H]
\begin{tabular}{cl}
\textbf{3.19} & \romline{tasmādasaktaḥ satataṃ} \\
 & \romline{kāryaṃ karma samācara |} \\
 & \romline{asakto hyācarankarma} \\
 & \romline{paramāpnoti pūruṣaḥ ||}
\end{tabular}
\end{table}

\begin{table}[H]
\begin{tabular}{cl}
\textbf{3.20} & \romline{karmaṇaiva hi saṃsiddhiṃ} \\
 & \romline{āsthitā janakādayaḥ |} \\
 & \romline{lokasaṅgrahamevāpi} \\
 & \romline{sampaśyankartumarhasi ||}
\end{tabular}
\end{table}

\begin{table}[H]
\begin{tabular}{cl}
\textbf{3.21} & \romline{yadyadācarati śreṣṭhaḥ} \\
 & \romline{tattadevetaro janaḥ |} \\
 & \romline{sa yatpramāṇaṃ kurute} \\
 & \romline{lokastadanuvartate ||}
\end{tabular}
\end{table}

\begin{table}[H]
\begin{tabular}{cl}
\textbf{3.22} & \romline{na me pārthāsti kartavyaṃ} \\
 & \romline{triṣu lokeṣu kiñcana |} \\
 & \romline{nānavāptamavāptavyaṃ} \\
 & \romline{varta eva ca karmaṇi ||}
\end{tabular}
\end{table}

\begin{table}[H]
\begin{tabular}{cl}
\textbf{3.23} & \romline{yadi hyahaṃ na varteya} \\
 & \romline{jātu karmaṇyatandritaḥ |} \\
 & \romline{mama vartmānuvartante} \\
 & \romline{manuṣyāḥ pārtha sarvaśaḥ ||}
\end{tabular}
\end{table}

\begin{table}[H]
\begin{tabular}{cl}
\textbf{3.24} & \romline{utsīdeyurime lokāḥ} \\
 & \romline{na kuryāṃ karma cedaham |} \\
 & \romline{saṅkarasya ca kartā syām} \\
 & \romline{upahanyāmimāḥ prajāḥ ||}
\end{tabular}
\end{table}

\begin{table}[H]
\begin{tabular}{cl}
\textbf{3.25} & \romline{saktāḥ karmaṇyavidvāṃsaḥ} \\
 & \romline{yathā kurvanti bhārata |} \\
 & \romline{kuryādvidvāṃstathā'saktaḥ} \\
 & \romline{cikīrṣurlokasaṅgraham ||}
\end{tabular}
\end{table}

\begin{table}[H]
\begin{tabular}{cl}
\textbf{3.26} & \romline{na buddhibhedaṃ janayet} \\
 & \romline{ajñānāṃ karmasaṅginām |} \\
 & \romline{joṣayetsarvakarmāṇi} \\
 & \romline{vidvānyuktaḥ samācaran ||}
\end{tabular}
\end{table}

\begin{table}[H]
\begin{tabular}{cl}
\textbf{3.27} & \romline{prakṛteḥ kriyamāṇāni} \\
 & \romline{guṇaiḥ karmāṇi sarvaśaḥ |} \\
 & \romline{ahaṅkāravimūḍhātmā} \\
 & \romline{kartā'hamiti manyate ||}
\end{tabular}
\end{table}

\begin{table}[H]
\begin{tabular}{cl}
\textbf{3.28} & \romline{tattvavittu mahābāho} \\
 & \romline{guṇakarmavibhāgayoḥ |} \\
 & \romline{guṇā guṇeṣu vartante} \\
 & \romline{iti matvā na sajjate ||}
\end{tabular}
\end{table}

\begin{table}[H]
\begin{tabular}{cl}
\textbf{3.29} & \romline{prakṛterguṇasammūḍhāḥ} \\
 & \romline{sajjante guṇakarmasu |} \\
 & \romline{tānakṛtsnavido mandān} \\
 & \romline{kṛtsnavinna vicālayet ||}
\end{tabular}
\end{table}

\begin{table}[H]
\begin{tabular}{cl}
\textbf{3.30} & \romline{mayi sarvāṇi karmāṇi} \\
 & \romline{sannyasyādhyātmacetasā |} \\
 & \romline{nirāśīrnirmamo bhūtvā} \\
 & \romline{yudhyasva vigatajvaraḥ ||}
\end{tabular}
\end{table}

\begin{table}[H]
\begin{tabular}{cl}
\textbf{3.31} & \romline{ye me matamidaṃ nityam} \\
 & \romline{anutiṣṭhanti mānavāḥ |} \\
 & \romline{śraddhāvanto'nasūyantaḥ} \\
 & \romline{mucyante te'pi karmabhiḥ ||}
\end{tabular}
\end{table}

\begin{table}[H]
\begin{tabular}{cl}
\textbf{3.32} & \romline{ye tvetadabhyasūyantaḥ} \\
 & \romline{nānutiṣṭhanti me matam |} \\
 & \romline{sarvajñānavimūḍhāṃstān} \\
 & \romline{viddhi naṣṭānacetasaḥ ||}
\end{tabular}
\end{table}

\begin{table}[H]
\begin{tabular}{cl}
\textbf{3.33} & \romline{sadṛśaṃ ceṣṭate svasyāḥ} \\
 & \romline{prakṛterjñānavānapi |} \\
 & \romline{prakṛtiṃ yānti bhūtāni} \\
 & \romline{nigrahaḥ kiṃ kariṣyati ||}
\end{tabular}
\end{table}

\begin{table}[H]
\begin{tabular}{cl}
\textbf{3.34} & \romline{indriyasyendriyasyārthe} \\
 & \romline{rāgadveṣau vyavasthitau |} \\
 & \romline{tayorna vaśamāgacchet} \\
 & \romline{tau hyasya paripanthinau ||}
\end{tabular}
\end{table}

\begin{table}[H]
\begin{tabular}{cl}
\textbf{3.35} & \romline{śreyānsvadharmo viguṇaḥ} \\
 & \romline{paradharmātsvanuṣṭhitāt |} \\
 & \romline{svadharme nidhanaṃ śreyaḥ} \\
 & \romline{paradharmo bhayāvahaḥ ||}
\end{tabular}
\end{table}

\begin{table}[H]
\begin{tabular}{cl}
\textbf{3.36} & \romline{arjuna uvāca} \\
 & \romline{atha kena prayukto'yaṃ} \\
 & \romline{pāpaṃ carati pūruṣaḥ |} \\
 & \romline{anicchannapi vārṣṇeya} \\
 & \romline{balādiva niyojitaḥ ||}
\end{tabular}
\end{table}

\begin{table}[H]
\begin{tabular}{cl}
\textbf{3.37} & \romline{śrī bhagavānuvāca} \\
 & \romline{kāma eṣa krodha eṣaḥ} \\
 & \romline{rajoguṇasamudbhavaḥ |} \\
 & \romline{mahāśano mahāpāpmā} \\
 & \romline{viddhyenamiha vairiṇam ||}
\end{tabular}
\end{table}

\begin{table}[H]
\begin{tabular}{cl}
\textbf{3.38} & \romline{dhūmenāvriyate vahniḥ} \\
 & \romline{yathā'darśo malena ca |} \\
 & \romline{yatholbenāvṛto garbhaḥ} \\
 & \romline{tathā tenedamāvṛtam ||}
\end{tabular}
\end{table}

\begin{table}[H]
\begin{tabular}{cl}
\textbf{3.39} & \romline{āvṛtaṃ jñānametena} \\
 & \romline{jñānino nityavairiṇā |} \\
 & \romline{kāmarūpeṇa kaunteya} \\
 & \romline{duṣpūreṇānalena ca ||}
\end{tabular}
\end{table}

\begin{table}[H]
\begin{tabular}{cl}
\textbf{3.40} & \romline{indriyāṇi mano buddhiḥ} \\
 & \romline{asyādhiṣṭhānamucyate |} \\
 & \romline{etairvimohayatyeṣaḥ} \\
 & \romline{jñānamāvṛtya dehinam ||}
\end{tabular}
\end{table}

\begin{table}[H]
\begin{tabular}{cl}
\textbf{3.41} & \romline{tasmāttvamindriyāṇyādau} \\
 & \romline{niyamya bharatarṣabha |} \\
 & \romline{pāpmānaṃ prajahi hyenaṃ} \\
 & \romline{jñānavijñānanāśanam ||}
\end{tabular}
\end{table}

\begin{table}[H]
\begin{tabular}{cl}
\textbf{3.42} & \romline{indriyāṇi parāṇyāhuḥ} \\
 & \romline{indriyebhyaḥ paraṃ manaḥ |} \\
 & \romline{manasastu parābuddhiḥ} \\
 & \romline{yo buddheḥ paratastu saḥ ||}
\end{tabular}
\end{table}

\begin{table}[H]
\begin{tabular}{cl}
\textbf{3.43} & \romline{evaṃ buddheḥ paraṃ buddhvā} \\
 & \romline{saṃstabhyātmānamātmanā |} \\
 & \romline{jahi śatruṃ mahābāho} \\
 & \romline{kāmarūpaṃ durāsadam ||}
\end{tabular}
\end{table}

\begin{table}[H]
\begin{tabular}{cl}
 & \romline{śrīmadbhagavadgītāsu upaniṣatsu} \\
 & \romline{brahmavidyāyāṃ yogaśāstre} \\
 & \romline{śrīkṛṣṇārjuna saṃvāde} \\
 & \romline{karmayogo nāma} \\
 & \romline{tṛtīyodhyāyaḥ}
\end{tabular}
\end{table}

