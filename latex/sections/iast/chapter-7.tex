\begin{table}[H]
\begin{tabular}{cl}
\textbf{7.0} & \romline{oṃ śrī paramātmane namaḥ} \\
 & \romline{atha saptamo'dhyāyaḥ} \\
 & \romline{jñāna-vijñāna-yogaḥ}
\end{tabular}
\end{table}

\begin{table}[H]
\begin{tabular}{cl}
\textbf{7.1} & \romline{śrī bhagavānuvāca} \\
 & \romline{mayyā-saktamanāḥ pārtha} \\
 & \romline{yogaṃ yuñjanma-dāśrayaḥ |} \\
 & \romline{asaṃśayaṃ samagraṃ māṃ} \\
 & \romline{yathā jñāsyasi tacchṛṇu ||}
\end{tabular}
\end{table}

\begin{table}[H]
\begin{tabular}{cl}
\textbf{7.2} & \romline{jñānaṃ te'haṃ savijñānam} \\
 & \romline{idaṃ vakṣyāmya-śeṣataḥ |} \\
 & \romline{yaj-jñātvā neha bhūyo'nyat} \\
 & \romline{jñātavya-mavaśiṣyate ||}
\end{tabular}
\end{table}

\begin{table}[H]
\begin{tabular}{cl}
\textbf{7.3} & \romline{manuṣyāṇāṃ sahasreṣu} \\
 & \romline{kaścidyatati siddhaye |} \\
 & \romline{yatatāmapi siddhānāṃ} \\
 & \romline{kascinmāṃ vetti tattvatah ||}
\end{tabular}
\end{table}

\begin{table}[H]
\begin{tabular}{cl}
\textbf{7.4} & \romline{bhūmirāpo'nalo vāyuḥ} \\
 & \romline{khaṃ mano buddhireva ca |} \\
 & \romline{ahaṅkāra itīyaṃ me} \\
 & \romline{bhinnā prakṛti-raṣṭadhā ||}
\end{tabular}
\end{table}

\begin{table}[H]
\begin{tabular}{cl}
\textbf{7.5} & \romline{apareya-mitastvanyāṃ} \\
 & \romline{prakṛtiṃ viddhi me parām |} \\
 & \romline{jīvabhūtāṃ mahābāho} \\
 & \romline{yayedaṃ dhāryate jagat ||}
\end{tabular}
\end{table}

\begin{table}[H]
\begin{tabular}{cl}
\textbf{7.6} & \romline{etad-yonīni bhūtāni} \\
 & \romline{sarvāṇītyu-padhāraya |} \\
 & \romline{ahaṃ kṛtsnasya jagataḥ} \\
 & \romline{prabhavaḥ pralayastathā ||}
\end{tabular}
\end{table}

\begin{table}[H]
\begin{tabular}{cl}
\textbf{7.7} & \romline{mattaḥ parataraṃ nānyat} \\
 & \romline{kiñcidasti dhanañjaya |} \\
 & \romline{mayi sarva-midaṃ protaṃ} \\
 & \romline{sūtre maṇigaṇā iva ||}
\end{tabular}
\end{table}

\begin{table}[H]
\begin{tabular}{cl}
\textbf{7.8} & \romline{raso'hamapsu kaunteya} \\
 & \romline{prabhā'smi śaśi-sūryayoḥ |} \\
 & \romline{praṇavaḥ sarva-vedeṣu} \\
 & \romline{śabdaḥ khe pauruṣaṃ nṛṣu ||}
\end{tabular}
\end{table}

\begin{table}[H]
\begin{tabular}{cl}
\textbf{7.9} & \romline{puṇyo gandhaḥ pṛthivyāṃ ca} \\
 & \romline{tejaścāsmi vibhāvasau |} \\
 & \romline{jīvanaṃ sarvabhuteṣu} \\
 & \romline{tapaścāsmi tapasviṣu ||}
\end{tabular}
\end{table}

\begin{table}[H]
\begin{tabular}{cl}
\textbf{7.10} & \romline{bījaṃ māṃ sarva-bhūtānāṃ} \\
 & \romline{viddhi pārtha sanātanam |} \\
 & \romline{buddhir-buddhi-matāmasmi} \\
 & \romline{tejastejas-vināmaham ||}
\end{tabular}
\end{table}

