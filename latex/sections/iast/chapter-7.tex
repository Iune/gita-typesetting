\begin{table}[H]
\begin{tabular}{cl}
\textbf{7.0} & \romline{oṃ śrī paramātmane namaḥ} \\
 & \romline{atha saptamo'dhyāyaḥ} \\
 & \romline{jñāna-vijñāna-yogaḥ}
\end{tabular}
\end{table}

\begin{table}[H]
\begin{tabular}{cl}
\textbf{7.1} & \romline{śrī bhagavānuvāca} \\
 & \romline{mayyā-saktamanāḥ pārtha} \\
 & \romline{yogaṃ yuñjanma-dāśrayaḥ |} \\
 & \romline{asaṃśayaṃ samagraṃ māṃ} \\
 & \romline{yathā jñāsyasi tacchṛṇu ||}
\end{tabular}
\end{table}

\begin{table}[H]
\begin{tabular}{cl}
\textbf{7.2} & \romline{jñānaṃ te'haṃ savijñānam} \\
 & \romline{idaṃ vakṣyāmya-śeṣataḥ |} \\
 & \romline{yaj-jñātvā neha bhūyo'nyat} \\
 & \romline{jñātavya-mavaśiṣyate ||}
\end{tabular}
\end{table}

\begin{table}[H]
\begin{tabular}{cl}
\textbf{7.3} & \romline{manuṣyāṇāṃ sahasreṣu} \\
 & \romline{kaścidyatati siddhaye |} \\
 & \romline{yatatāmapi siddhānāṃ} \\
 & \romline{kascinmāṃ vetti tattvatah ||}
\end{tabular}
\end{table}

\begin{table}[H]
\begin{tabular}{cl}
\textbf{7.4} & \romline{bhūmirāpo'nalo vāyuḥ} \\
 & \romline{khaṃ mano buddhireva ca |} \\
 & \romline{ahaṅkāra itīyaṃ me} \\
 & \romline{bhinnā prakṛti-raṣṭadhā ||}
\end{tabular}
\end{table}

\begin{table}[H]
\begin{tabular}{cl}
\textbf{7.5} & \romline{apareya-mitastvanyāṃ} \\
 & \romline{prakṛtiṃ viddhi me parām |} \\
 & \romline{jīvabhūtāṃ mahābāho} \\
 & \romline{yayedaṃ dhāryate jagat ||}
\end{tabular}
\end{table}

\begin{table}[H]
\begin{tabular}{cl}
\textbf{7.6} & \romline{etad-yonīni bhūtāni} \\
 & \romline{sarvāṇītyu-padhāraya |} \\
 & \romline{ahaṃ kṛtsnasya jagataḥ} \\
 & \romline{prabhavaḥ pralayastathā ||}
\end{tabular}
\end{table}

\begin{table}[H]
\begin{tabular}{cl}
\textbf{7.7} & \romline{mattaḥ parataraṃ nānyat} \\
 & \romline{kiñcidasti dhanañjaya |} \\
 & \romline{mayi sarva-midaṃ protaṃ} \\
 & \romline{sūtre maṇigaṇā iva ||}
\end{tabular}
\end{table}

\begin{table}[H]
\begin{tabular}{cl}
\textbf{7.8} & \romline{raso'hamapsu kaunteya} \\
 & \romline{prabhā'smi śaśi-sūryayoḥ |} \\
 & \romline{praṇavaḥ sarva-vedeṣu} \\
 & \romline{śabdaḥ khe pauruṣaṃ nṛṣu ||}
\end{tabular}
\end{table}

\begin{table}[H]
\begin{tabular}{cl}
\textbf{7.9} & \romline{puṇyo gandhaḥ pṛthivyāṃ ca} \\
 & \romline{tejaścāsmi vibhāvasau |} \\
 & \romline{jīvanaṃ sarvabhuteṣu} \\
 & \romline{tapaścāsmi tapasviṣu ||}
\end{tabular}
\end{table}

\begin{table}[H]
\begin{tabular}{cl}
\textbf{7.10} & \romline{bījaṃ māṃ sarva-bhūtānāṃ} \\
 & \romline{viddhi pārtha sanātanam |} \\
 & \romline{buddhir-buddhi-matāmasmi} \\
 & \romline{tejastejas-vināmaham ||}
\end{tabular}
\end{table}

\begin{table}[H]
\begin{tabular}{cl}
\textbf{7.11} & \romline{balaṃ balavatāṃ cāhaṃ} \\
 & \romline{kāma-rāga-vivarjitam |} \\
 & \romline{dharmā-viruddho bhūteṣu} \\
 & \romline{kāmo'smi bhara-tarṣabha ||}
\end{tabular}
\end{table}

\begin{table}[H]
\begin{tabular}{cl}
\textbf{7.12} & \romline{ye caiva sāttvikā bhāvāḥ} \\
 & \romline{rājasāstāmasāśca ye |} \\
 & \romline{matta eveti tānviddhi} \\
 & \romline{na tvahaṃ teṣu te mayi ||}
\end{tabular}
\end{table}

\begin{table}[H]
\begin{tabular}{cl}
\textbf{7.13} & \romline{tribhirguṇamayairbhāvaiḥ} \\
 & \romline{ebhiḥ sarvamidaṃ jagat |} \\
 & \romline{mohitaṃ nābhijānāti} \\
 & \romline{māmebhyaḥ paramavyayam ||}
\end{tabular}
\end{table}

\begin{table}[H]
\begin{tabular}{cl}
\textbf{7.14} & \romline{daivī hyeṣā gunāmayī} \\
 & \romline{mama māya duratyayā |} \\
 & \romline{māmeva ye prapadyante} \\
 & \romline{māyāmetāṃ taranti te ||}
\end{tabular}
\end{table}

\begin{table}[H]
\begin{tabular}{cl}
\textbf{7.15} & \romline{na māṃ duṣkṛtino mūḍhāḥ} \\
 & \romline{prapadyante narādhamāḥ |} \\
 & \romline{māyayā'pahṛtajñānāḥ} \\
 & \romline{āsuraṃ bhāvamāśritāḥ ||}
\end{tabular}
\end{table}

\begin{table}[H]
\begin{tabular}{cl}
\textbf{7.16} & \romline{caturvidhā bhajante māṃ} \\
 & \romline{janāḥ sukṛtino'rjuna |} \\
 & \romline{ārto jijñā-surarthārthī} \\
 & \romline{jñānī ca bhara-tarṣabha ||}
\end{tabular}
\end{table}

\begin{table}[H]
\begin{tabular}{cl}
\textbf{7.17} & \romline{teṣāṃ jñānī nityayuktaḥ} \\
 & \romline{ekabhaktir-viśiṣyate |} \\
 & \romline{priyo hi·jñānino'tyartham} \\
 & \romline{ahaṃ sa ca mama·priyaḥ ||}
\end{tabular}
\end{table}

\begin{table}[H]
\begin{tabular}{cl}
\textbf{7.18} & \romline{udārāḥ sarva evaite} \\
 & \romline{jñānī tvātmaiva me matam |} \\
 & \romline{āsthitaḥ sa hi yuktātmā} \\
 & \romline{māmevā-nuttamāṃ gatim ||}
\end{tabular}
\end{table}

\begin{table}[H]
\begin{tabular}{cl}
\textbf{7.19} & \romline{bahūnāṃ janma-nāmante} \\
 & \romline{jñānavān-māṃ prapadyate |} \\
 & \romline{vāsudevaḥ sarvamiti} \\
 & \romline{sa mahātmā sudurlabhaḥ ||}
\end{tabular}
\end{table}

\begin{table}[H]
\begin{tabular}{cl}
\textbf{7.20} & \romline{kāmaistai-stair-hṛta-jñānāḥ} \\
 & \romline{prapadyante'nyadevatāḥ |} \\
 & \romline{taṃ taṃ niyama-māsthāya} \\
 & \romline{prakṛtyā niyatāḥ svayā ||}
\end{tabular}
\end{table}

\begin{table}[H]
\begin{tabular}{cl}
\textbf{7.21} & \romline{yo yo yāṃ yāṃ tanuṃ bhaktaḥ} \\
 & \romline{śraddhayārcitu-micchati |} \\
 & \romline{tasya tasyācalāṃ śraddhāṃ} \\
 & \romline{tāmeva vida-dhāmyaham ||}
\end{tabular}
\end{table}

\begin{table}[H]
\begin{tabular}{cl}
\textbf{7.22} & \romline{sa tayā śraddhayā yuktaḥ} \\
 & \romline{tasyā-rādhana-mīhate |} \\
 & \romline{labhate ca tataḥ kāmān} \\
 & \romline{mayaiva vihitānhi tān ||}
\end{tabular}
\end{table}

\begin{table}[H]
\begin{tabular}{cl}
\textbf{7.23} & \romline{antavattu phalaṃ teṣāṃ} \\
 & \romline{tadbhava-tyalpa-medhasāṃ |} \\
 & \romline{devān-devayajo yānti} \\
 & \romline{madbhaktā yānti māmapi ||}
\end{tabular}
\end{table}

\begin{table}[H]
\begin{tabular}{cl}
\textbf{7.24} & \romline{avyaktaṃ vyakti-māpannaṃ} \\
 & \romline{manyante māma-buddhayaḥ |} \\
 & \romline{paraṃ bhāva-majānantaḥ} \\
 & \romline{mamāvya-yama-nuttaman ||}
\end{tabular}
\end{table}

\begin{table}[H]
\begin{tabular}{cl}
\textbf{7.25} & \romline{nāhāṃ prakāśaḥ sarvasya} \\
 & \romline{yoga-māyā-samāvṛtaḥ |} \\
 & \romline{mūḍho'yaṃ nābhi-jānāti} \\
 & \romline{loko māmaja-mavyayam ||}
\end{tabular}
\end{table}

\begin{table}[H]
\begin{tabular}{cl}
\textbf{7.26} & \romline{vedāhaṃ samatītāni} \\
 & \romline{vartamānāni cārjuna |} \\
 & \romline{bhaviṣyāṇi ca bhūtāni} \\
 & \romline{māṃ tu veda na kaścana ||}
\end{tabular}
\end{table}

\begin{table}[H]
\begin{tabular}{cl}
\textbf{7.27} & \romline{icchādveṣa-samutthena} \\
 & \romline{dvandva-mohena bhārata |} \\
 & \romline{sarva-bhūtāni sammohaṃ} \\
 & \romline{sarge yānti parantapa ||}
\end{tabular}
\end{table}

\begin{table}[H]
\begin{tabular}{cl}
\textbf{7.28} & \romline{yeṣāṃ tvanta-gataṃ pāpam} \\
 & \romline{janānāṃ puṇya-karmaṇām |} \\
 & \romline{te dvandva-mohanirmuktāḥ} \\
 & \romline{bhajante māṃ dṛḍha-vratāḥ ||}
\end{tabular}
\end{table}

\begin{table}[H]
\begin{tabular}{cl}
\textbf{7.29} & \romline{jarā-maraṇa-mokṣāya} \\
 & \romline{māmāśritya yatanti ye |} \\
 & \romline{te brahma tadviduḥ kṛtsnam} \\
 & \romline{adhyātmaṃ karma cākhilam ||}
\end{tabular}
\end{table}

\begin{table}[H]
\begin{tabular}{cl}
\textbf{7.30} & \romline{sādhi-bhūtādhi-daivaṃ māṃ} \\
 & \romline{sādhiyajñaṃ ce ye viduḥ |} \\
 & \romline{prayāṇakāle'pi ca māṃ} \\
 & \romline{te viduryukta-cetasaḥ ||}
\end{tabular}
\end{table}

