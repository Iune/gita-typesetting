\begin{table}[H]
\begin{tabular}{cl}
 & \romline{śrī paramātmane namaḥ} \\
 & \romline{atha daśamo'dhyāyaḥ} \\
 & \romline{vibhutiyogaḥ}
\end{tabular}
\end{table}

\begin{table}[H]
\begin{tabular}{cl}
\textbf{10.1} & \romline{śrī bhagavānuvāca} \\
 & \romline{bhūya eva mahābāho} \\
 & \romline{śṛṇu me paramaṃ vacaḥ |} \\
 & \romline{yatte'haṃ prīyamāṇāya} \\
 & \romline{vakṣyāmi hitakāmyayā ||}
\end{tabular}
\end{table}

\begin{table}[H]
\begin{tabular}{cl}
\textbf{10.2} & \romline{na me viduḥ suragaṇāḥ} \\
 & \romline{prabhavaṃ na maharṣayaḥ |} \\
 & \romline{ahamādirhi devānāṃ} \\
 & \romline{maharṣīṇāṃ ca sarvaśaḥ ||}
\end{tabular}
\end{table}

\begin{table}[H]
\begin{tabular}{cl}
\textbf{10.3} & \romline{yo māmajamanādiṃ ca} \\
 & \romline{vetti lokamaheśvaram |} \\
 & \romline{asammūḍhaḥ sa martyeṣu} \\
 & \romline{sarvapāpaiḥ pramucyate ||}
\end{tabular}
\end{table}

\begin{table}[H]
\begin{tabular}{cl}
\textbf{10.4} & \romline{buddhirjñānamasammohaḥ} \\
 & \romline{kṣamā satyaṃ damaḥ śamaḥ |} \\
 & \romline{sukhaṃ duḥkhaṃ bhavo'bhāvaḥ} \\
 & \romline{bhayaṃ cābhayameva ca ||}
\end{tabular}
\end{table}

\begin{table}[H]
\begin{tabular}{cl}
\textbf{10.5} & \romline{ahiṃsā samatā tuṣṭiḥ} \\
 & \romline{tapo dānaṃ yaśo'yaśaḥ |} \\
 & \romline{bhavanti bhāvā bhūtānāṃ} \\
 & \romline{matta eva pṛthagvidhāḥ ||}
\end{tabular}
\end{table}

\begin{table}[H]
\begin{tabular}{cl}
\textbf{10.6} & \romline{maharṣayaḥ sapta pūrve} \\
 & \romline{catvāro manavastathā |} \\
 & \romline{madbhāvā mānasā jātāḥ} \\
 & \romline{yeṣāṃ loka imāḥ prajāḥ ||}
\end{tabular}
\end{table}

\begin{table}[H]
\begin{tabular}{cl}
\textbf{10.7} & \romline{etāṃ vibhūtiṃ yogaṃ ca} \\
 & \romline{mama yo vetti tattvataḥ |} \\
 & \romline{so'vikampena yogena} \\
 & \romline{yujyate nātra saṃśayaḥ ||}
\end{tabular}
\end{table}

\begin{table}[H]
\begin{tabular}{cl}
\textbf{10.8} & \romline{ahaṃ sarvasya prabhavaḥ} \\
 & \romline{mattaḥ sarvaṃ pravartate |} \\
 & \romline{iti matvā bhajante māṃ} \\
 & \romline{budhā bhāvasamanvitāḥ ||}
\end{tabular}
\end{table}

\begin{table}[H]
\begin{tabular}{cl}
\textbf{10.9} & \romline{maccittā madgataprāṇāḥ} \\
 & \romline{bodhayantaḥ parasparam |} \\
 & \romline{kathayantaśca māṃ nityaṃ} \\
 & \romline{tuṣyanti ca ramanti ca ||}
\end{tabular}
\end{table}

\begin{table}[H]
\begin{tabular}{cl}
\textbf{10.10} & \romline{teṣāṃ satatayuktānāṃ} \\
 & \romline{bhajatāṃ prītipūrvakam |} \\
 & \romline{dadāmi buddhiyogaṃ taṃ} \\
 & \romline{yena māmupayānti te ||}
\end{tabular}
\end{table}

\begin{table}[H]
\begin{tabular}{cl}
\textbf{10.11} & \romline{teṣāmevānukampārtham} \\
 & \romline{ahamajñānajaṃ tamaḥ |} \\
 & \romline{nāśayāmyātmabhāvasthaḥ} \\
 & \romline{jñānadīpena bhāsvatā ||}
\end{tabular}
\end{table}

\begin{table}[H]
\begin{tabular}{cl}
\textbf{10.12} & \romline{arjuna uvāca} \\
 & \romline{paraṃ brahma paraṃ dhāma} \\
 & \romline{pavitraṃ paramaṃ bhavān |} \\
 & \romline{puruṣaṃ śāśvataṃ divyam} \\
 & \romline{ādidevamajaṃ vibhum ||}
\end{tabular}
\end{table}

\begin{table}[H]
\begin{tabular}{cl}
\textbf{10.13} & \romline{āhustvāmṛṣayaḥ sarve} \\
 & \romline{devarṣirnāradastathā |} \\
 & \romline{asito devalo vyāsaḥ} \\
 & \romline{svayaṃ caiva bravīṣi me ||}
\end{tabular}
\end{table}

\begin{table}[H]
\begin{tabular}{cl}
\textbf{10.14} & \romline{sarvametadṛtaṃ manye} \\
 & \romline{yanmāṃ vadasi keśava |} \\
 & \romline{na hi te bhagavanvyaktiṃ} \\
 & \romline{vidurdevā na dānavāḥ ||}
\end{tabular}
\end{table}

\begin{table}[H]
\begin{tabular}{cl}
\textbf{10.15} & \romline{svayamevātmanā''tmānaṃ} \\
 & \romline{vettha tvaṃ puruṣottama |} \\
 & \romline{bhūtabhāvana bhūteśa} \\
 & \romline{devadeva jagatpate ||}
\end{tabular}
\end{table}

\begin{table}[H]
\begin{tabular}{cl}
\textbf{10.16} & \romline{vaktumarhasyaśeṣeṇa} \\
 & \romline{divyā hyātmavibhūtayaḥ |} \\
 & \romline{yābhirvibhūtibhirlokān} \\
 & \romline{imāṃstvaṃ vyāpya tiṣṭhasi ||}
\end{tabular}
\end{table}

\begin{table}[H]
\begin{tabular}{cl}
\textbf{10.17} & \romline{kathaṃ vidyāmahaṃ yogin} \\
 & \romline{tvāṃ sadā paricintayan |} \\
 & \romline{keṣu keṣu ca bhāveṣu} \\
 & \romline{cintyo'si bhagavanmayā ||}
\end{tabular}
\end{table}

\begin{table}[H]
\begin{tabular}{cl}
\textbf{10.18} & \romline{vistareṇātmano yogaṃ} \\
 & \romline{vibhūtiṃ ca janārdana |} \\
 & \romline{bhūyaḥ kathaya tṛptirhi} \\
 & \romline{śṛṇvato nāsti me'mṛtam ||}
\end{tabular}
\end{table}

\begin{table}[H]
\begin{tabular}{cl}
\textbf{10.19} & \romline{śrī bhagavānuvāca} \\
 & \romline{hanta te kathayiṣyāmi} \\
 & \romline{divyā hyātmavibhūtayaḥ |} \\
 & \romline{prādhānyataḥ kuruśreṣṭha} \\
 & \romline{nāstyanto vistarasya me ||}
\end{tabular}
\end{table}

\begin{table}[H]
\begin{tabular}{cl}
\textbf{10.20} & \romline{ahamātmā guḍākeśa} \\
 & \romline{sarvabhūtāśayasthitaḥ |} \\
 & \romline{ahamādiśca madhyaṃ ca} \\
 & \romline{bhūtānāmanta eva ca ||}
\end{tabular}
\end{table}

\begin{table}[H]
\begin{tabular}{cl}
\textbf{10.21} & \romline{ādityānāmaham viṣṇuḥ} \\
 & \romline{jyotiṣāṃ raviraṃśumān |} \\
 & \romline{marīcirmarutāmasmi} \\
 & \romline{nakṣatrāṇāmahaṃ śaśī ||}
\end{tabular}
\end{table}

\begin{table}[H]
\begin{tabular}{cl}
\textbf{10.22} & \romline{vedānāṃ sāmavedo'smi} \\
 & \romline{devānāmasmi vāsavaḥ |} \\
 & \romline{indriyāṇāṃ manaścāsmi} \\
 & \romline{bhūtānāmasmi cetanā ||}
\end{tabular}
\end{table}

\begin{table}[H]
\begin{tabular}{cl}
\textbf{10.23} & \romline{rudrāṇāṃ śaṅkaraścāsmi} \\
 & \romline{vitteśo yakṣarakṣasām |} \\
 & \romline{vasūnāṃ pāvakaścāsmi} \\
 & \romline{meruḥ śikhariṇāmaham ||}
\end{tabular}
\end{table}

\begin{table}[H]
\begin{tabular}{cl}
\textbf{10.24} & \romline{purodhasāṃ ca mukhyaṃ māṃ} \\
 & \romline{viddhi pārtha bṛhaspatim |} \\
 & \romline{senānīnāmahaṃ skandaḥ} \\
 & \romline{sarasāmasmi sāgaraḥ ||}
\end{tabular}
\end{table}

\begin{table}[H]
\begin{tabular}{cl}
\textbf{10.25} & \romline{maharṣīṇāṃ bhṛgurahaṃ} \\
 & \romline{girāmasmyekamakṣaram |} \\
 & \romline{yajñānāṃ japayajño'smi} \\
 & \romline{sthāvarāṇāṃ himālayaḥ ||}
\end{tabular}
\end{table}

\begin{table}[H]
\begin{tabular}{cl}
\textbf{10.26} & \romline{aśvatthaḥ sarvavṛkṣāṇāṃ} \\
 & \romline{devarṣīṇāṃ ca nāradaḥ |} \\
 & \romline{gandharvāṇāṃ citrarathaḥ} \\
 & \romline{siddhānāṃ kapilo muniḥ ||}
\end{tabular}
\end{table}

\begin{table}[H]
\begin{tabular}{cl}
\textbf{10.27} & \romline{uccaiḥ śravasamaśvānāṃ} \\
 & \romline{viddhi māmamṛtodbhavam |} \\
 & \romline{airāvataṃ gajendrāṇāṃ} \\
 & \romline{narāṇāṃ ca narādhipam ||}
\end{tabular}
\end{table}

\begin{table}[H]
\begin{tabular}{cl}
\textbf{10.28} & \romline{āyudhānāmahaṃ vajraṃ} \\
 & \romline{dhenūnāmasmi kāmadhuk |} \\
 & \romline{prajanaścāsmi kandarpaḥ} \\
 & \romline{sarpāṇāmasmi vāsukiḥ ||}
\end{tabular}
\end{table}

\begin{table}[H]
\begin{tabular}{cl}
\textbf{10.29} & \romline{anantaścāsmi nāgānāṃ} \\
 & \romline{varuṇo yādasāmaham |} \\
 & \romline{pitṝṇāmaryamā cāsmi} \\
 & \romline{yamaḥ saṃyamatāmaham ||}
\end{tabular}
\end{table}

\begin{table}[H]
\begin{tabular}{cl}
\textbf{10.30} & \romline{prahlādaścāsmi daityānāṃ} \\
 & \romline{kālaḥ kalayatāmaham |} \\
 & \romline{mṛgāṇāṃ ca mṛgendro'haṃ} \\
 & \romline{vainateyaśca pakṣiṇām ||}
\end{tabular}
\end{table}

\begin{table}[H]
\begin{tabular}{cl}
\textbf{10.31} & \romline{pavanaḥ pavatāmasmi} \\
 & \romline{rāmaḥ śastrabhṛtāmaham |} \\
 & \romline{jhaṣāṇāṃ makaraścāsmi} \\
 & \romline{srotasāmasmi jāhnavī ||}
\end{tabular}
\end{table}

\begin{table}[H]
\begin{tabular}{cl}
\textbf{10.32} & \romline{sargāṇāmādirantaśca} \\
 & \romline{madhyaṃ caivāhamarjuna |} \\
 & \romline{adhyātmavidyā vidyānāṃ} \\
 & \romline{vādaḥ pravadatāmaham ||}
\end{tabular}
\end{table}

\begin{table}[H]
\begin{tabular}{cl}
\textbf{10.33} & \romline{akṣarāṇāmakāro'smi} \\
 & \romline{dvandvaḥ sāmāsikasya ca |} \\
 & \romline{ahamevākṣayaḥ kālaḥ} \\
 & \romline{dhātā'haṃ viśvatomukhaḥ ||}
\end{tabular}
\end{table}

\begin{table}[H]
\begin{tabular}{cl}
\textbf{10.34} & \romline{mṛtyuḥ sarvaharaścāham} \\
 & \romline{udbhavaśca bhaviṣyatām |} \\
 & \romline{kīrtiḥ śrīrvākca nārīṇāṃ} \\
 & \romline{smṛtirmedhā dhṛtiḥ kṣamā ||}
\end{tabular}
\end{table}

\begin{table}[H]
\begin{tabular}{cl}
\textbf{10.35} & \romline{bṛhatsāma tathā sāmnāṃ} \\
 & \romline{gāyatrī chandasāmaham |} \\
 & \romline{māsānāṃ mārgaśīrṣo'ham} \\
 & \romline{ṛtūnāṃ kusumākaraḥ ||}
\end{tabular}
\end{table}

\begin{table}[H]
\begin{tabular}{cl}
\textbf{10.36} & \romline{dyūtaṃ chalayatāmasmi} \\
 & \romline{tejastejasvināmaham |} \\
 & \romline{jayo'smi vyavasāyo'smi} \\
 & \romline{sattvaṃ sattvavatāmaham ||}
\end{tabular}
\end{table}

\begin{table}[H]
\begin{tabular}{cl}
\textbf{10.37} & \romline{vṛṣṇīnāṃ vāsudevo'smi} \\
 & \romline{pāṇḍavānāṃ dhanañjayaḥ |} \\
 & \romline{munīnāmapyahaṃ vyāsaḥ} \\
 & \romline{kavīnāmuśanā kaviḥ ||}
\end{tabular}
\end{table}

\begin{table}[H]
\begin{tabular}{cl}
\textbf{10.38} & \romline{daṇḍo damayatāmasmi} \\
 & \romline{nītirasmi jigīṣatām |} \\
 & \romline{maunaṃ caivāsmi guhyānāṃ} \\
 & \romline{jñānaṃ jñānavatāmaham ||}
\end{tabular}
\end{table}

\begin{table}[H]
\begin{tabular}{cl}
\textbf{10.39} & \romline{yaccāpi sarvabhūtānāṃ} \\
 & \romline{bījaṃ tadahamarjuna |} \\
 & \romline{na tadasti vinā yatsyāt} \\
 & \romline{mayā bhūtaṃ carācaram ||}
\end{tabular}
\end{table}

\begin{table}[H]
\begin{tabular}{cl}
\textbf{10.40} & \romline{nānto'sti mama divyānāṃ} \\
 & \romline{vibhūtīnāṃ parantapa |} \\
 & \romline{eṣa tūddeśataḥ proktaḥ} \\
 & \romline{vibhūtervistaro mayā ||}
\end{tabular}
\end{table}

\begin{table}[H]
\begin{tabular}{cl}
\textbf{10.41} & \romline{yadyadvibhūtimatsattvaṃ} \\
 & \romline{śrīmadūrjitameva vā |} \\
 & \romline{tattadevāvagaccha tvaṃ} \\
 & \romline{mama tejo'mśasambhavam ||}
\end{tabular}
\end{table}

\begin{table}[H]
\begin{tabular}{cl}
\textbf{10.42} & \romline{athavā bahunaitena} \\
 & \romline{kiṃ jñātena tavārjuna |} \\
 & \romline{viṣṭabhyāhamidaṃ kṛtsnam} \\
 & \romline{ekāṃśena sthito jagat ||}
\end{tabular}
\end{table}

\begin{table}[H]
\begin{tabular}{cl}
 & \romline{śrīmadbhagavadgītāsu upaniṣatsu} \\
 & \romline{brahmavidyāyāṃ yogaśāstre} \\
 & \romline{śrīkṛṣṇārjuna saṃvāde} \\
 & \romline{vibhutiyogo nāma} \\
 & \romline{daśamodhyāyaḥ}
\end{tabular}
\end{table}

