\begin{table}[H]
\begin{tabular}{cl}
 & \romline{śrī paramātmane namaḥ} \\
 & \romline{atha daśamo'dhyāyaḥ} \\
 & \romline{vibhuti-yogaḥ}
\end{tabular}
\end{table}

\begin{table}[H]
\begin{tabular}{cl}
\textbf{10.1} & \romline{śrī bhagavānuvāca} \\
 & \romline{bhūya eva mahābāho} \\
 & \romline{śṛṇu me paramaṃ vacaḥ |} \\
 & \romline{yatte'haṃ prīya-māṇāya} \\
 & \romline{vakṣyāmi hitakāmyayā ||}
\end{tabular}
\end{table}

\begin{table}[H]
\begin{tabular}{cl}
\textbf{10.2} & \romline{na me viduḥ suragaṇāḥ} \\
 & \romline{prabhavaṃ na maharṣayaḥ |} \\
 & \romline{ahamā-dirhi devānāṃ} \\
 & \romline{maharṣīṇāṃ ca sarvaśaḥ ||}
\end{tabular}
\end{table}

\begin{table}[H]
\begin{tabular}{cl}
\textbf{10.3} & \romline{yo mā-maja-manādiṃ ca} \\
 & \romline{vetti loka-maheśvaram |} \\
 & \romline{asammūḍhaḥ sa martyeṣu} \\
 & \romline{sarvapāpaiḥ pramucyate ||}
\end{tabular}
\end{table}

\begin{table}[H]
\begin{tabular}{cl}
\textbf{10.4} & \romline{buddhir-jñāna-masammohaḥ} \\
 & \romline{kṣamā satyaṃ damaḥ śamaḥ |} \\
 & \romline{sukhaṃ duḥkhaṃ bhavo'bhāvaḥ} \\
 & \romline{bhayaṃ cābhaya-meva ca ||}
\end{tabular}
\end{table}

\begin{table}[H]
\begin{tabular}{cl}
\textbf{10.5} & \romline{ahiṃsā samatā tuṣṭiḥ} \\
 & \romline{tapo dānaṃ yaśo'yaśaḥ |} \\
 & \romline{bhavanti bhāvā bhūtānāṃ} \\
 & \romline{matta eva pṛthag-vidhāḥ ||}
\end{tabular}
\end{table}

\begin{table}[H]
\begin{tabular}{cl}
\textbf{10.6} & \romline{maharṣayaḥ sapta pūrve} \\
 & \romline{catvāro mana-vastathā |} \\
 & \romline{madbhāvā mānasā jātāḥ} \\
 & \romline{yeṣāṃ loka imāḥ prajāḥ ||}
\end{tabular}
\end{table}

\begin{table}[H]
\begin{tabular}{cl}
\textbf{10.7} & \romline{etāṃ vibhūtiṃ yogaṃ ca} \\
 & \romline{mama yo vetti tattvataḥ |} \\
 & \romline{so'vikampena yogena} \\
 & \romline{yujyate nātra saṃśayaḥ ||}
\end{tabular}
\end{table}

\begin{table}[H]
\begin{tabular}{cl}
\textbf{10.8} & \romline{ahaṃ sarvasya prabhavaḥ} \\
 & \romline{mattaḥ sarvaṃ pravartate |} \\
 & \romline{iti matvā bhajante māṃ} \\
 & \romline{budhā bhāva-samanvitāḥ ||}
\end{tabular}
\end{table}

\begin{table}[H]
\begin{tabular}{cl}
\textbf{10.9} & \romline{maccittā madgata-prāṇāḥ} \\
 & \romline{bodha-yantaḥ parasparam |} \\
 & \romline{katha-yantaśca māṃ nityaṃ} \\
 & \romline{tuṣyanti ca ramanti ca ||}
\end{tabular}
\end{table}

\begin{table}[H]
\begin{tabular}{cl}
\textbf{10.10} & \romline{teṣāṃ satata-yuktānāṃ} \\
 & \romline{bhajatāṃ prīti-pūrvakam |} \\
 & \romline{dadāmi buddhi-yogaṃ taṃ} \\
 & \romline{yena māmu-payānti te ||}
\end{tabular}
\end{table}

\begin{table}[H]
\begin{tabular}{cl}
\textbf{10.11} & \romline{teṣām-evānu-kampārtham} \\
 & \romline{ahamajñāna-jaṃ tamaḥ |} \\
 & \romline{nāśayāmyātma-bhāvasthaḥ} \\
 & \romline{jñāna-dīpena bhāsvatā ||}
\end{tabular}
\end{table}

\begin{table}[H]
\begin{tabular}{cl}
\textbf{10.12} & \romline{arjuna uvāca} \\
 & \romline{paraṃ brahma paraṃ dhāma} \\
 & \romline{pavitraṃ paramaṃ bhavān |} \\
 & \romline{puruṣaṃ śāśvataṃ divyam} \\
 & \romline{ādideva-majaṃ vibhum ||}
\end{tabular}
\end{table}

\begin{table}[H]
\begin{tabular}{cl}
\textbf{10.13} & \romline{āhustvā-mṛṣayaḥ sarve} \\
 & \romline{devarṣir-nārada-stathā |} \\
 & \romline{asito devalo vyāsaḥ} \\
 & \romline{svayaṃ caiva bravīṣi me ||}
\end{tabular}
\end{table}

\begin{table}[H]
\begin{tabular}{cl}
\textbf{10.14} & \romline{sarva-meta-dṛtaṃ manye} \\
 & \romline{yanmāṃ vadasi keśava |} \\
 & \romline{na hi te bhagavan-vyaktiṃ} \\
 & \romline{vidurdevā na dānavāḥ ||}
\end{tabular}
\end{table}

\begin{table}[H]
\begin{tabular}{cl}
\textbf{10.15} & \romline{svaya-mevātmanā''tmānaṃ} \\
 & \romline{vettha·tvaṃ puruṣottama |} \\
 & \romline{bhūta-bhāvana bhūteśa} \\
 & \romline{devadeva jagatpate ||}
\end{tabular}
\end{table}

\begin{table}[H]
\begin{tabular}{cl}
\textbf{10.16} & \romline{vaktu-marhasya-śeṣeṇa} \\
 & \romline{divyā hyāt-mavi-bhūtayaḥ |} \\
 & \romline{yābhir-vibhūti-bhir-lokān} \\
 & \romline{imāṃ-stvaṃ vyāpya tiṣṭhasi ||}
\end{tabular}
\end{table}

\begin{table}[H]
\begin{tabular}{cl}
\textbf{10.17} & \romline{kathaṃ vidyā-mahaṃ yogin} \\
 & \romline{tvāṃ sadā paricintayan |} \\
 & \romline{keṣu keṣu ca bhāveṣu} \\
 & \romline{cintyo'si bhagavan-mayā ||}
\end{tabular}
\end{table}

\begin{table}[H]
\begin{tabular}{cl}
\textbf{10.18} & \romline{vistareṇātmano yogaṃ} \\
 & \romline{vibhūtiṃ ca janārdana |} \\
 & \romline{bhūyaḥ kathaya tṛptirhi} \\
 & \romline{śṛṇvato nāsti me'mṛtam ||}
\end{tabular}
\end{table}

\begin{table}[H]
\begin{tabular}{cl}
\textbf{10.19} & \romline{śrī bhagavānuvāca} \\
 & \romline{hanta te katha-yiṣyāmi} \\
 & \romline{divyā hyāt-mavi-bhūtayaḥ |} \\
 & \romline{prādhānya-taḥ kuruśreṣṭha} \\
 & \romline{nāstyanto vistarasya me ||}
\end{tabular}
\end{table}

\begin{table}[H]
\begin{tabular}{cl}
\textbf{10.20} & \romline{ahamātmā guḍākeśa} \\
 & \romline{sarva-bhūtā-śaya-sthitaḥ |} \\
 & \romline{ahamādiśca madhyaṃ ca} \\
 & \romline{bhūtānā-manta eva ca ||}
\end{tabular}
\end{table}

\begin{table}[H]
\begin{tabular}{cl}
\textbf{10.21} & \romline{ādityānā-maham viṣṇuḥ} \\
 & \romline{jyotiṣāṃ raviraṃ-śumān |} \\
 & \romline{marīcir-marutā-masmi} \\
 & \romline{nakṣatrāṇā-mahaṃ śaśī ||}
\end{tabular}
\end{table}

\begin{table}[H]
\begin{tabular}{cl}
\textbf{10.22} & \romline{vedānāṃ sāmavedo'smi} \\
 & \romline{devānā-masmi vāsavaḥ |} \\
 & \romline{indriyāṇāṃ manaścāsmi} \\
 & \romline{bhūtānā-masmi cetanā ||}
\end{tabular}
\end{table}

\begin{table}[H]
\begin{tabular}{cl}
\textbf{10.23} & \romline{rudrāṇāṃ śaṅkaraścāsmi} \\
 & \romline{vitteśo yakṣa-rakṣasām |} \\
 & \romline{vasūnāṃ pāvakaścāsmi} \\
 & \romline{meruḥ śikhari-ṇāmaham ||}
\end{tabular}
\end{table}

\begin{table}[H]
\begin{tabular}{cl}
\textbf{10.24} & \romline{purodhasāṃ ca mukhyaṃ māṃ} \\
 & \romline{viddhi pārtha bṛhaspatim |} \\
 & \romline{senānīnā-mahaṃ skandaḥ} \\
 & \romline{sarasā-masmi sāgaraḥ ||}
\end{tabular}
\end{table}

\begin{table}[H]
\begin{tabular}{cl}
\textbf{10.25} & \romline{maharṣīṇāṃ bhṛgurahaṃ} \\
 & \romline{girā-masmye-kamakṣaram |} \\
 & \romline{yajñānāṃ japa-yajño'smi} \\
 & \romline{sthā-varāṇāṃ himālayaḥ ||}
\end{tabular}
\end{table}

\begin{table}[H]
\begin{tabular}{cl}
\textbf{10.26} & \romline{aśvatthaḥ sarva-vṛkṣāṇāṃ} \\
 & \romline{devarṣīṇāṃ ca nāradaḥ |} \\
 & \romline{gandharvāṇāṃ citra-rathaḥ} \\
 & \romline{siddhānāṃ kapilo muniḥ ||}
\end{tabular}
\end{table}

\begin{table}[H]
\begin{tabular}{cl}
\textbf{10.27} & \romline{uccaiḥ śrava-samaśvānāṃ} \\
 & \romline{viddhi māma-mṛtod-bhavam |} \\
 & \romline{airāvataṃ gajendrāṇāṃ} \\
 & \romline{narāṇāṃ ca narādhipam ||}
\end{tabular}
\end{table}

\begin{table}[H]
\begin{tabular}{cl}
\textbf{10.28} & \romline{āyudhānā-mahaṃ vajraṃ} \\
 & \romline{dhenūnā-masmi kāmadhuk |} \\
 & \romline{prajanaścāsmi kandarpaḥ} \\
 & \romline{sarpāṇā-masmi vāsukiḥ ||}
\end{tabular}
\end{table}

\begin{table}[H]
\begin{tabular}{cl}
\textbf{10.29} & \romline{anantaścāsmi nāgānāṃ} \\
 & \romline{varuṇo yādasā-maham |} \\
 & \romline{pitṝṇā-maryamā cāsmi} \\
 & \romline{yamaḥ saṃyama-tāmaham ||}
\end{tabular}
\end{table}

\begin{table}[H]
\begin{tabular}{cl}
\textbf{10.30} & \romline{prahlādaścāsmi daityānāṃ} \\
 & \romline{kālaḥ kalaya-tāmaham |} \\
 & \romline{mṛgāṇāṃ ca mṛgendro'haṃ} \\
 & \romline{vainateyaśca pakṣiṇām ||}
\end{tabular}
\end{table}

\begin{table}[H]
\begin{tabular}{cl}
\textbf{10.31} & \romline{pavanaḥ pavatā-masmi} \\
 & \romline{rāmaḥ śastra-bhṛtāmaham |} \\
 & \romline{jhaṣāṇāṃ makaraścāsmi} \\
 & \romline{srota-sāmasmi jāhnavī ||}
\end{tabular}
\end{table}

\begin{table}[H]
\begin{tabular}{cl}
\textbf{10.32} & \romline{sargāṇā-mādi-rantaśca} \\
 & \romline{madhyaṃ caivā-hamarjuna |} \\
 & \romline{adhyāt-mavidyā vidyānāṃ} \\
 & \romline{vādaḥ prava-datāmaham ||}
\end{tabular}
\end{table}

\begin{table}[H]
\begin{tabular}{cl}
\textbf{10.33} & \romline{akṣarāṇā-makāro'smi} \\
 & \romline{dvandvaḥ sāmā-sikasya ca |} \\
 & \romline{ahamevā-kṣayaḥ kālaḥ} \\
 & \romline{dhātā'haṃ viśvato-mukhaḥ ||}
\end{tabular}
\end{table}

\begin{table}[H]
\begin{tabular}{cl}
\textbf{10.34} & \romline{mṛtyuḥ sarva-haraścāham} \\
 & \romline{udbhavaśca bhaviṣyatām |} \\
 & \romline{kīrtiḥ śrīr-vākca nārīṇāṃ} \\
 & \romline{smṛtir-medhā dhṛtiḥ kṣamā ||}
\end{tabular}
\end{table}

\begin{table}[H]
\begin{tabular}{cl}
\textbf{10.35} & \romline{bṛhat-sāma tathā sāmnāṃ} \\
 & \romline{gāyatrī chanda-sāmaham |} \\
 & \romline{māsānāṃ mārga-śīrṣo'ham} \\
 & \romline{ṛtūnāṃ kusumākaraḥ ||}
\end{tabular}
\end{table}

\begin{table}[H]
\begin{tabular}{cl}
\textbf{10.36} & \romline{dyūtaṃ chalaya-tāmasmi} \\
 & \romline{tejas-tejas-vināmaham |} \\
 & \romline{jayo'smi vyava-sāyo'smi} \\
 & \romline{sattvaṃ sattva-vatāmaham ||}
\end{tabular}
\end{table}

\begin{table}[H]
\begin{tabular}{cl}
\textbf{10.37} & \romline{vṛṣṇī-nāṃ vāsudevo'smi} \\
 & \romline{pāṇḍavānāṃ dhanañjayaḥ |} \\
 & \romline{munīnā-mapyahaṃ vyāsaḥ} \\
 & \romline{kavīnā-muśanā kaviḥ ||}
\end{tabular}
\end{table}

\begin{table}[H]
\begin{tabular}{cl}
\textbf{10.38} & \romline{daṇḍo damaya-tāmasmi} \\
 & \romline{nītirasmi jigī-ṣatām |} \\
 & \romline{maunaṃ caivāsmi guhyānāṃ} \\
 & \romline{jñānaṃ jñāna-vatāmaham ||}
\end{tabular}
\end{table}

\begin{table}[H]
\begin{tabular}{cl}
\textbf{10.39} & \romline{yaccāpi sarva-bhūtānāṃ} \\
 & \romline{bījaṃ tada-hamarjuna |} \\
 & \romline{na tadasti vinā yatsyāt} \\
 & \romline{mayā bhūtaṃ carācaram ||}
\end{tabular}
\end{table}

\begin{table}[H]
\begin{tabular}{cl}
\textbf{10.40} & \romline{nānto'sti mama divyānāṃ} \\
 & \romline{vibhūtīnāṃ parantapa |} \\
 & \romline{eṣa tūddeśataḥ proktaḥ} \\
 & \romline{vibhūter-vistaro mayā ||}
\end{tabular}
\end{table}

\begin{table}[H]
\begin{tabular}{cl}
\textbf{10.41} & \romline{yadya-dvi-bhūtimat-sattvaṃ} \\
 & \romline{śrīmadūr-jitameva vā |} \\
 & \romline{tatta-devāva-gaccha·tvaṃ} \\
 & \romline{mama tejo'mśa-sambhavam ||}
\end{tabular}
\end{table}

\begin{table}[H]
\begin{tabular}{cl}
\textbf{10.42} & \romline{athavā bahunaitena} \\
 & \romline{kiṃ jñātena tavārjuna |} \\
 & \romline{viṣṭabhyā-hamidaṃ kṛtsnam} \\
 & \romline{ekāṃśena·sthito jagat ||}
\end{tabular}
\end{table}

\begin{table}[H]
\begin{tabular}{cl}
 & \romline{śrīmad-bhagavad-gītāsu upaniṣatsu} \\
 & \romline{brahma-vidyāyāṃ yogaśāstre} \\
 & \romline{śrīkṛṣṇārjuna saṃvāde} \\
 & \romline{vibhuti-yogonāma} \\
 & \romline{daśamo-dhyāyaḥ}
\end{tabular}
\end{table}

