\begin{table}[H]
\begin{tabular}{cl}
\textbf{4.0} & \romline{oṃ śrī paramātmane namaḥ} \\
 & \romline{atha caturtho'dhyāyaḥ} \\
 & \romline{jñāna-yogaḥ}
\end{tabular}
\end{table}

\begin{table}[H]
\begin{tabular}{cl}
\textbf{4.1} & \romline{śrī bhagavānuvāca} \\
 & \romline{imaṃ vivasvate yogaṃ} \\
 & \romline{proktavān-ahamavyayam |} \\
 & \romline{vivasvān-manave prāha} \\
 & \romline{manurikṣvākave'bravīt ||}
\end{tabular}
\end{table}

\begin{table}[H]
\begin{tabular}{cl}
\textbf{4.2} & \romline{evaṃ paramparā-prāptam} \\
 & \romline{imaṃ rājarṣayo viduḥ |} \\
 & \romline{sa kāleneha mahatā} \\
 & \romline{yogo naṣṭaḥ parantapa ||}
\end{tabular}
\end{table}

\begin{table}[H]
\begin{tabular}{cl}
\textbf{4.3} & \romline{sa evāyaṃ mayā te'dya} \\
 & \romline{yogaḥ proktaḥ purātanaḥ |} \\
 & \romline{bhakto'si me sakhā ceti} \\
 & \romline{rahasyaṃ hyetaduttamam ||}
\end{tabular}
\end{table}

\begin{table}[H]
\begin{tabular}{cl}
\textbf{4.4} & \romline{arjuna uvāca} \\
 & \romline{aparaṃ bhavato janma} \\
 & \romline{paraṃ janma vivasvataḥ |} \\
 & \romline{kathameta-dvijānīyāṃ} \\
 & \romline{tvamādau proktavāniti ||}
\end{tabular}
\end{table}

\begin{table}[H]
\begin{tabular}{cl}
\textbf{4.5} & \romline{śrī bhagavānuvāca} \\
 & \romline{bahūni me vyatītāni} \\
 & \romline{janmāni tava cārjuna |} \\
 & \romline{tānyahaṃ veda sarvāṇi} \\
 & \romline{na tvaṃ vettha parantapa ||}
\end{tabular}
\end{table}

\begin{table}[H]
\begin{tabular}{cl}
\textbf{4.6} & \romline{ajo'pi sannavya-yātmā} \\
 & \romline{bhūtānā-mīśvaro'pi san |} \\
 & \romline{prakṛtiṃ svāmadhiṣṭhāya} \\
 & \romline{sambhavā-myātmamāyayā ||}
\end{tabular}
\end{table}

\begin{table}[H]
\begin{tabular}{cl}
\textbf{4.7} & \romline{yadā yadā hi dharmasya} \\
 & \romline{glānirbhavati bhārata |} \\
 & \romline{abhyutthāna-madharmasya} \\
 & \romline{tadā''tmānaṃ sṛjāmyaham ||}
\end{tabular}
\end{table}

\begin{table}[H]
\begin{tabular}{cl}
\textbf{4.8} & \romline{paritrāṇāya sādhūnāṃ} \\
 & \romline{vināśāya ca duṣkṛtām |} \\
 & \romline{dharmasaṃsthāpanārthāya} \\
 & \romline{sambhavāmi yuge yuge ||}
\end{tabular}
\end{table}

\begin{table}[H]
\begin{tabular}{cl}
\textbf{4.9} & \romline{janma karma ca me divyam} \\
 & \romline{evaṃ yo vetti tattvataḥ |} \\
 & \romline{tyaktvā dehaṃ punarjanma} \\
 & \romline{naiti māmeti so'rjuna ||}
\end{tabular}
\end{table}

\begin{table}[H]
\begin{tabular}{cl}
\textbf{4.10} & \romline{vītarāga-bhaya-krodhāḥ} \\
 & \romline{manmayā māmu-pāśritāḥ |} \\
 & \romline{bahavo jñāna-tapasā} \\
 & \romline{pūtā madbhāva-māgatāḥ ||}
\end{tabular}
\end{table}

\begin{table}[H]
\begin{tabular}{cl}
\textbf{4.11} & \romline{ye yathā māṃ prapadyante} \\
 & \romline{tāṃstathaiva bhajāmyaham |} \\
 & \romline{mama vartmānuvartante} \\
 & \romline{manuṣyāḥ pārtha sarvaśaḥ ||}
\end{tabular}
\end{table}

\begin{table}[H]
\begin{tabular}{cl}
\textbf{4.12} & \romline{kāṅkṣantaḥ karmaṇāṃ siddhiṃ} \\
 & \romline{yajanta iha devatāḥ |} \\
 & \romline{kṣipraṃ hi mānuṣe loke} \\
 & \romline{siddhirbhavati karmajā ||}
\end{tabular}
\end{table}

\begin{table}[H]
\begin{tabular}{cl}
\textbf{4.13} & \romline{cāturvarṇyaṃ mayā sṛṣṭaṃ} \\
 & \romline{guṇa-karma-vibhāgaśaḥ |} \\
 & \romline{tasya kartāramapi māṃ} \\
 & \romline{viddhyakartāra-mavyayam ||}
\end{tabular}
\end{table}

\begin{table}[H]
\begin{tabular}{cl}
\textbf{4.14} & \romline{na māṃ karmāṇi limpanti} \\
 & \romline{na me karmaphale spṛhā |} \\
 & \romline{iti māṃ yo'bhijānāti} \\
 & \romline{karmabhirna sa badhyate ||}
\end{tabular}
\end{table}

\begin{table}[H]
\begin{tabular}{cl}
\textbf{4.15} & \romline{evaṃ jñātvā kṛtaṃ karma} \\
 & \romline{pūrvairapi mumukṣubhiḥ |} \\
 & \romline{kuru karmaiva tasmāttvaṃ} \\
 & \romline{pūrvaiḥ pūrvataraṃ kṛtam ||}
\end{tabular}
\end{table}

\begin{table}[H]
\begin{tabular}{cl}
\textbf{4.16} & \romline{kiṃ karma kimakarmeti} \\
 & \romline{kavayo'pyatra mohitāḥ |} \\
 & \romline{tatte karma pravakṣyāmi} \\
 & \romline{yajjñātvā mokṣyase'śubhāt ||}
\end{tabular}
\end{table}

\begin{table}[H]
\begin{tabular}{cl}
\textbf{4.17} & \romline{karmaṇo hyapi boddhavyaṃ} \\
 & \romline{boddhavyaṃ ca vikarmaṇaḥ |} \\
 & \romline{akarmaṇaśca boddhavyaṃ} \\
 & \romline{gahanā karmaṇo gatiḥ ||}
\end{tabular}
\end{table}

\begin{table}[H]
\begin{tabular}{cl}
\textbf{4.18} & \romline{karmaṇyakarma yaḥ paśyet} \\
 & \romline{akarmaṇica karma yaḥ |} \\
 & \romline{sa buddhimān-manuṣyeṣu} \\
 & \romline{sa yuktaḥ kṛtsna-karmakṛt ||}
\end{tabular}
\end{table}

\begin{table}[H]
\begin{tabular}{cl}
\textbf{4.19} & \romline{yasya sarve samārambhāḥ} \\
 & \romline{kāmasaṅkalpa-varjitāḥ |} \\
 & \romline{jñānāgni-dagdhakarmāṇaṃ} \\
 & \romline{tamāhuḥ paṇḍitaṃ budhāḥ ||}
\end{tabular}
\end{table}

\begin{table}[H]
\begin{tabular}{cl}
\textbf{4.20} & \romline{tyaktvā karma-phalāsaṅgaṃ} \\
 & \romline{nityatṛpto nirāśrayaḥ |} \\
 & \romline{karmaṇyabhi-pravṛtto'pi} \\
 & \romline{naiva kiñcitkaroti saḥ ||}
\end{tabular}
\end{table}

\begin{table}[H]
\begin{tabular}{cl}
\textbf{4.21} & \romline{nirāśīr-yatacittātmā} \\
 & \romline{tyaktasarva parigrahaḥ |} \\
 & \romline{śārīraṃ kevalaṃ karma} \\
 & \romline{kurvannāpnoti kilbiṣam ||}
\end{tabular}
\end{table}

\begin{table}[H]
\begin{tabular}{cl}
\textbf{4.22} & \romline{yadṛcchālā-bhasantuṣṭaḥ} \\
 & \romline{dvandvātīto vimatsaraḥ |} \\
 & \romline{samaḥ siddhāva-siddhau ca} \\
 & \romline{kṛtvāpi na nibadhyate ||}
\end{tabular}
\end{table}

\begin{table}[H]
\begin{tabular}{cl}
\textbf{4.23} & \romline{gatasaṅgasya muktasya} \\
 & \romline{jñānā-vasthita-cetasaḥ |} \\
 & \romline{yajñā-yācarataḥ karma} \\
 & \romline{samagraṃ pravilīyate ||}
\end{tabular}
\end{table}

\begin{table}[H]
\begin{tabular}{cl}
\textbf{4.24} & \romline{brahmārpaṇaṃ brahma haviḥ} \\
 & \romline{brahmāgnau brahmaṇā hutam |} \\
 & \romline{brahmaiva tena gantavyaṃ} \\
 & \romline{brahma-karma-samādhinā ||}
\end{tabular}
\end{table}

\begin{table}[H]
\begin{tabular}{cl}
\textbf{4.25} & \romline{daivamevāpare yajñaṃ} \\
 & \romline{yoginaḥ paryupāsate |} \\
 & \romline{brahmāgnā-vapare yajñaṃ} \\
 & \romline{yajñe-naivopa-juhvati ||}
\end{tabular}
\end{table}

\begin{table}[H]
\begin{tabular}{cl}
\textbf{4.26} & \romline{śrotrā-dīnīndriyāṇyanye} \\
 & \romline{saṃyamāgniṣu juhvati |} \\
 & \romline{śabdādīn-viṣayānanye} \\
 & \romline{indriyāgniṣu juhvati ||}
\end{tabular}
\end{table}

\begin{table}[H]
\begin{tabular}{cl}
\textbf{4.27} & \romline{sarvā-ṇīndriya-karmāṇi} \\
 & \romline{prāṇakarmāṇi cāpare |} \\
 & \romline{ātma-saṃyamayogāgnau} \\
 & \romline{juhvati jñānadīpite ||}
\end{tabular}
\end{table}

\begin{table}[H]
\begin{tabular}{cl}
\textbf{4.28} & \romline{dravya-yajñās-tapoyajñāḥ} \\
 & \romline{yogayajñās-tathā'pare |} \\
 & \romline{svādhyāyajñā-nayajñāśca} \\
 & \romline{yatayaḥ saṃśitavratāḥ ||}
\end{tabular}
\end{table}

\begin{table}[H]
\begin{tabular}{cl}
\textbf{4.29} & \romline{apāne juhvati prāṇaṃ} \\
 & \romline{prāṇe'pānaṃ tathāpare |} \\
 & \romline{prāṇāpāna-gatī ruddhvā} \\
 & \romline{prāṇāyāma-parāyaṇāḥ ||}
\end{tabular}
\end{table}

\begin{table}[H]
\begin{tabular}{cl}
\textbf{4.30} & \romline{apare niyatāhārāḥ} \\
 & \romline{prāṇān-prāṇeṣu juhvati |} \\
 & \romline{sarve'pyete yajñavidaḥ} \\
 & \romline{yajñakṣapita-kalmaṣāḥ ||}
\end{tabular}
\end{table}

\begin{table}[H]
\begin{tabular}{cl}
\textbf{4.31} & \romline{yajña-śiṣṭāmṛta-bhujaḥ} \\
 & \romline{yānti brahma sanātanam |} \\
 & \romline{nāyaṃ loko'stya-yajñasya} \\
 & \romline{kuto'nyaḥ kurusattama ||}
\end{tabular}
\end{table}

\begin{table}[H]
\begin{tabular}{cl}
\textbf{4.32} & \romline{evaṃ bahuvidhā yajñāḥ} \\
 & \romline{vitatā brahmaṇo mukhe |} \\
 & \romline{karma-jānviddhi tānsarvān} \\
 & \romline{evaṃ jñātvā vimokṣyase ||}
\end{tabular}
\end{table}

\begin{table}[H]
\begin{tabular}{cl}
\textbf{4.33} & \romline{śreyān-dravya-mayādyajñāt} \\
 & \romline{jñāna-yajñaḥ parantapa |} \\
 & \romline{sarvaṃ karmākhilaṃ pārtha} \\
 & \romline{jñāne parisamāpyate ||}
\end{tabular}
\end{table}

\begin{table}[H]
\begin{tabular}{cl}
\textbf{4.34} & \romline{tadviddhi praṇipātena} \\
 & \romline{paripraśnena sevayā |} \\
 & \romline{upadekṣyanti te jñānaṃ} \\
 & \romline{jñāninastattvadarśinaḥ ||}
\end{tabular}
\end{table}

\begin{table}[H]
\begin{tabular}{cl}
\textbf{4.35} & \romline{yajjñātvā na punarmoham} \\
 & \romline{evaṃ yāsyasi pāṇḍava |} \\
 & \romline{yena bhūtānyaśeṣeṇa} \\
 & \romline{drakṣyasyātmanyatho mayi ||}
\end{tabular}
\end{table}

\begin{table}[H]
\begin{tabular}{cl}
\textbf{4.36} & \romline{api cedasi pāpebhyaḥ} \\
 & \romline{sarvebhyaḥ pāpakṛttamaḥ |} \\
 & \romline{sarvaṃ jñānaplavenaiva} \\
 & \romline{vṛjinaṃ santariṣyasi ||}
\end{tabular}
\end{table}

\begin{table}[H]
\begin{tabular}{cl}
\textbf{4.37} & \romline{yathaidhāṃsi samiddho'gniḥ} \\
 & \romline{bhasmasātkurute'rjuna |} \\
 & \romline{jñānāgniḥ sarvakarmāṇi} \\
 & \romline{bhasmasātkurute tathā ||}
\end{tabular}
\end{table}

\begin{table}[H]
\begin{tabular}{cl}
\textbf{4.38} & \romline{na hi jñānena sadṛśaṃ} \\
 & \romline{pavitramiha vidyate |} \\
 & \romline{tatsvayaṃ yogasaṃsiddhaḥ} \\
 & \romline{kālenātmani vindati ||}
\end{tabular}
\end{table}

\begin{table}[H]
\begin{tabular}{cl}
\textbf{4.39} & \romline{śraddhāvān labhate jñānaṃ} \\
 & \romline{tatparaḥ saṃyatendriyaḥ |} \\
 & \romline{jñānaṃ labdhvā parāṃ śāntim} \\
 & \romline{acireṇādhigacchati ||}
\end{tabular}
\end{table}

\begin{table}[H]
\begin{tabular}{cl}
\textbf{4.40} & \romline{ajñaścāśraddadhānaśca} \\
 & \romline{saṃśayātmā vinaśyati |} \\
 & \romline{nāyaṃ loko'sti na paraḥ} \\
 & \romline{na sukhaṃ saṃśayātmanaḥ ||}
\end{tabular}
\end{table}

\begin{table}[H]
\begin{tabular}{cl}
\textbf{4.41} & \romline{yogasannyastakarmāṇaṃ} \\
 & \romline{jñānasañchinnasaṃśayam |} \\
 & \romline{ātmavantaṃ na karmāṇi} \\
 & \romline{nibadhnanti dhanañjaya ||}
\end{tabular}
\end{table}

\begin{table}[H]
\begin{tabular}{cl}
\textbf{4.42} & \romline{tasmādajñānasambhūtaṃ} \\
 & \romline{hṛtsthaṃ jñānāsinātmanaḥ |} \\
 & \romline{chittvainaṃ saṃśayaṃ yogam} \\
 & \romline{ātiṣṭhottiṣṭha bhārata ||}
\end{tabular}
\end{table}

