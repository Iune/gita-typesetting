\begin{table}[H]
\begin{tabular}{cl}
 & \romline{śrī paramātmane namaḥ} \\
 & \romline{atha ekādaśo'dhyāyaḥ} \\
 & \romline{viśvarūpasandarśanayogaḥ}
\end{tabular}
\end{table}

\begin{table}[H]
\begin{tabular}{cl}
\textbf{11.1} & \romline{arjuna uvāca} \\
 & \romline{madanugrahāya paramaṃ} \\
 & \romline{guhyamadhyātmasañjñitam |} \\
 & \romline{yattvayoktaṃ vacastena} \\
 & \romline{moho'yaṃ vigato mama ||}
\end{tabular}
\end{table}

\begin{table}[H]
\begin{tabular}{cl}
\textbf{11.2} & \romline{bhavāpyayau hi bhūtānāṃ} \\
 & \romline{śrutau vistaraśo mayā |} \\
 & \romline{tvattaḥ kamalapatrākṣa} \\
 & \romline{māhātmyamapi cāvyayam ||}
\end{tabular}
\end{table}

\begin{table}[H]
\begin{tabular}{cl}
\textbf{11.3} & \romline{evametadyathā''ttha tvam} \\
 & \romline{ātmānaṃ parameśvara |} \\
 & \romline{draṣṭumicchāmi te rūpam} \\
 & \romline{aiśvaram puruṣottama ||}
\end{tabular}
\end{table}

\begin{table}[H]
\begin{tabular}{cl}
\textbf{11.4} & \romline{manyase yadi tacchakyaṃ} \\
 & \romline{mayā draṣṭumiti prabho |} \\
 & \romline{yogeśvara tato me tvaṃ} \\
 & \romline{darśayātmānamavyayam ||}
\end{tabular}
\end{table}

\begin{table}[H]
\begin{tabular}{cl}
\textbf{11.5} & \romline{śrī bhagavānuvāca} \\
 & \romline{paśya me pārtha rūpāṇi} \\
 & \romline{śataśo'tha sahasraśaḥ |} \\
 & \romline{nānāvidhāni divyāni} \\
 & \romline{nānāvarṇākṛtīni ca ||}
\end{tabular}
\end{table}

\begin{table}[H]
\begin{tabular}{cl}
\textbf{11.6} & \romline{paśyādityānvasūnrudrān} \\
 & \romline{aśvinau marutastathā |} \\
 & \romline{bahūnyadṛṣṭapūrvāṇi} \\
 & \romline{paśyāścaryāṇi bhārata ||}
\end{tabular}
\end{table}

\begin{table}[H]
\begin{tabular}{cl}
\textbf{11.7} & \romline{ihaikasthaṃ jagatkṛtsnaṃ} \\
 & \romline{paśyādya sacarācaram |} \\
 & \romline{mama dehe guḍākeśa} \\
 & \romline{yaccānyat draṣṭumicchasi ||}
\end{tabular}
\end{table}

\begin{table}[H]
\begin{tabular}{cl}
\textbf{11.8} & \romline{na tu māṃ śakyase draṣṭum} \\
 & \romline{anenaiva svacakṣuṣā |} \\
 & \romline{divyaṃ dadāmi te cakṣuḥ} \\
 & \romline{paśya me yogamaiśvaram ||}
\end{tabular}
\end{table}

\begin{table}[H]
\begin{tabular}{cl}
\textbf{11.9} & \romline{sañjaya uvāca} \\
 & \romline{evamuktvā tato rājan} \\
 & \romline{mahāyogeśvaro hariḥ |} \\
 & \romline{darśayāmāsa pārthāya} \\
 & \romline{paramaṃ rūpamaiśvaram ||}
\end{tabular}
\end{table}

\begin{table}[H]
\begin{tabular}{cl}
\textbf{11.10} & \romline{anekavaktranayanam} \\
 & \romline{anekādbhutadarśanam |} \\
 & \romline{anekadivyābharaṇaṃ} \\
 & \romline{divyānekodyatāyudham ||}
\end{tabular}
\end{table}

\begin{table}[H]
\begin{tabular}{cl}
\textbf{11.11} & \romline{divyamālyāmbaradharaṃ} \\
 & \romline{divyagandhānulepanam |} \\
 & \romline{sarvāścaryamayaṃ devam} \\
 & \romline{anantaṃ viśvatomukham ||}
\end{tabular}
\end{table}

\begin{table}[H]
\begin{tabular}{cl}
\textbf{11.12} & \romline{divi sūryasahasrasya} \\
 & \romline{bhavedyugapadutthitā |} \\
 & \romline{yadi bhāḥ sadṛśī sā syāt} \\
 & \romline{bhāsastasya mahātmanaḥ ||}
\end{tabular}
\end{table}

\begin{table}[H]
\begin{tabular}{cl}
\textbf{11.13} & \romline{tatraikasthaṃ jagatkṛtsnaṃ} \\
 & \romline{pravibhaktamanekadhā |} \\
 & \romline{apaśyaddevadevasya} \\
 & \romline{śarīre pāṇḍavastadā ||}
\end{tabular}
\end{table}

\begin{table}[H]
\begin{tabular}{cl}
\textbf{11.14} & \romline{tataḥ sa vismayāviṣṭaḥ} \\
 & \romline{hṛṣṭaromā dhanañjayaḥ |} \\
 & \romline{praṇamya śirasā devaṃ} \\
 & \romline{kṛtāñjalirabhāṣata ||}
\end{tabular}
\end{table}

\begin{table}[H]
\begin{tabular}{cl}
\textbf{11.15} & \romline{arjuna uvāca} \\
 & \romline{paśyāmi devāṃstava deva dehe} \\
 & \romline{sarvāṃstathā bhūtaviśeṣasaṅghān |} \\
 & \romline{brahmāṇamīśaṃ kamalāsanastham} \\
 & \romline{ṛṣīṃśca sarvānuragāṃśca divyān ||}
\end{tabular}
\end{table}

\begin{table}[H]
\begin{tabular}{cl}
\textbf{11.16} & \romline{anekabāhūdaravaktranetraṃ} \\
 & \romline{paśyāmi tvā sarvato'nantarūpam |} \\
 & \romline{nāntaṃ na madhyaṃ na punastavādiṃ} \\
 & \romline{paśyāmi viśveśvara viśvarūpa ||}
\end{tabular}
\end{table}

\begin{table}[H]
\begin{tabular}{cl}
\textbf{11.17} & \romline{kirīṭinaṃ gadinaṃ cakriṇaṃ ca} \\
 & \romline{tejorāśiṃ sarvato dīptimantam |} \\
 & \romline{paśyāmi tvāṃ durnirīkṣyaṃ samantāt} \\
 & \romline{dīptānalārkadyutimaprameyam ||}
\end{tabular}
\end{table}

\begin{table}[H]
\begin{tabular}{cl}
\textbf{11.18} & \romline{tvamakṣaraṃ paramaṃ veditavyaṃ} \\
 & \romline{tvamasya viśvasya paraṃ nidhānam |} \\
 & \romline{tvamavyayaḥ śāśvatadharmagoptā} \\
 & \romline{sanātanastvaṃ puruṣo mato me ||}
\end{tabular}
\end{table}

\begin{table}[H]
\begin{tabular}{cl}
\textbf{11.19} & \romline{anādimadhyāntamanantavīryam} \\
 & \romline{anantabāhuṃ śaśisūryanetram |} \\
 & \romline{paśyāmi tvāṃ dīptahutāśavaktraṃ} \\
 & \romline{svatejasā viśvamidaṃ tapantam ||}
\end{tabular}
\end{table}

\begin{table}[H]
\begin{tabular}{cl}
\textbf{11.20} & \romline{dyāvāpṛthivyoridamantaraṃ hi} \\
 & \romline{vyāptaṃ tvayaikena diśaśca sarvāḥ |} \\
 & \romline{dṛṣṭvādbhutaṃ rūpamidaṃ tavograṃ} \\
 & \romline{lokatrayaṃ pravyathitaṃ mahātman ||}
\end{tabular}
\end{table}

\begin{table}[H]
\begin{tabular}{cl}
\textbf{11.21} & \romline{amī hi tvā surasaṅghā viśanti} \\
 & \romline{kecidbhītāḥ prāñjalayo gṛṇanti |} \\
 & \romline{svastītyuktvā maharṣisiddhasaṅghāḥ} \\
 & \romline{stuvanti tvāṃ stutibhiḥ puṣkalābhiḥ ||}
\end{tabular}
\end{table}

\begin{table}[H]
\begin{tabular}{cl}
\textbf{11.22} & \romline{rudrādityā vasavo ye ca sādhyāḥ} \\
 & \romline{viśve'śvinau marutaścoṣmapāśca |} \\
 & \romline{gandharvayakṣāsurasiddhasaṅghāḥ} \\
 & \romline{vīkṣante tvāṃ vismitāścaiva sarve ||}
\end{tabular}
\end{table}

\begin{table}[H]
\begin{tabular}{cl}
\textbf{11.23} & \romline{rūpaṃ mahatte bahuvaktra netraṃ} \\
 & \romline{mahābāho bahubāhūrupādam |} \\
 & \romline{bahūdaraṃ bahudaṃṣṭrākarālaṃ} \\
 & \romline{dṛṣṭvā lokāḥ pravyathitāstathā'ham ||}
\end{tabular}
\end{table}

\begin{table}[H]
\begin{tabular}{cl}
\textbf{11.24} & \romline{nabhaḥ spṛśaṃ dīptamanekavarṇaṃ} \\
 & \romline{vyāttānanaṃ dīptaviśālanetram |} \\
 & \romline{dṛṣṭvā hi tvāṃ pravyathitāntarātmā} \\
 & \romline{dhṛtiṃ na vindāmi śamaṃ ca viṣṇo ||}
\end{tabular}
\end{table}

\begin{table}[H]
\begin{tabular}{cl}
\textbf{11.25} & \romline{daṃṣṭrākarālāni ca te mukhāni} \\
 & \romline{dṛṣṭvaiva kālānalasannibhāni |} \\
 & \romline{diśo na jāne na labhe ca śarma} \\
 & \romline{prasīda deveśa jagannivāsa ||}
\end{tabular}
\end{table}

\begin{table}[H]
\begin{tabular}{cl}
\textbf{11.26} & \romline{amī ca tvāṃ dhṛtarāṣṭrasya putrāḥ} \\
 & \romline{sarve sahaivāvanipālasaṅghaiḥ |} \\
 & \romline{bhīṣmo droṇaḥ sūtaputrastathā'sau} \\
 & \romline{sahāsmadīyairapi yodhamukhyaiḥ ||}
\end{tabular}
\end{table}

\begin{table}[H]
\begin{tabular}{cl}
\textbf{11.27} & \romline{vaktrāṇi te tvaramāṇā viśanti} \\
 & \romline{daṃṣṭrākarālāni bhayānakāni |} \\
 & \romline{kecidvilagnā daśanāntareṣu} \\
 & \romline{sandṛśyante cūrṇitairuttamāṅgaiḥ ||}
\end{tabular}
\end{table}

\begin{table}[H]
\begin{tabular}{cl}
\textbf{11.28} & \romline{yathā nadīnāṃ bahavo'mbuvegāḥ} \\
 & \romline{samudramevābhimukhā dravanti |} \\
 & \romline{tathā tavāmī naralokavīrāḥ} \\
 & \romline{viśanti vaktrāṇyabhivijvalanti ||}
\end{tabular}
\end{table}

\begin{table}[H]
\begin{tabular}{cl}
\textbf{11.29} & \romline{yathā pradīptaṃ jvalanaṃ pataṅgāḥ} \\
 & \romline{viśanti nāśāya samṛddhavegāḥ |} \\
 & \romline{tathaiva nāśāya viśanti lokāḥ} \\
 & \romline{tavāpi vaktrāṇi samṛddhavegāḥ ||}
\end{tabular}
\end{table}

\begin{table}[H]
\begin{tabular}{cl}
\textbf{11.30} & \romline{lelihyase grasamānaḥ samantāt} \\
 & \romline{lokānsamagrānvadanairjvaladbhiḥ |} \\
 & \romline{tejobhirāpūrya jagatsamagraṃ} \\
 & \romline{bhāsastavogrāḥ pratapanti viṣṇo ||}
\end{tabular}
\end{table}

\begin{table}[H]
\begin{tabular}{cl}
\textbf{11.31} & \romline{ākhyāhi me ko bhavānugrarūpaḥ} \\
 & \romline{namo'stu te devavara prasīda |} \\
 & \romline{vijñātumicchāmi bhavantamādyaṃ} \\
 & \romline{na hi prajānāmi tava pravṛttim ||}
\end{tabular}
\end{table}

\begin{table}[H]
\begin{tabular}{cl}
\textbf{11.32} & \romline{śrī bhagavānuvāca} \\
 & \romline{kālo'smi lokakṣayakṛtpravṛddhaḥ} \\
 & \romline{lokānsamāhartumiha pravṛttaḥ |} \\
 & \romline{ṛte'pi tvā na bhaviṣyanti sarve} \\
 & \romline{ye'vasthitāḥ pratyanīkeṣu yodhāḥ ||}
\end{tabular}
\end{table}

\begin{table}[H]
\begin{tabular}{cl}
\textbf{11.33} & \romline{tasmāttvamuttiṣṭha yaśo labhasva} \\
 & \romline{jitvā śatrūnbhuṅkṣva rājyaṃ samṛddham |} \\
 & \romline{mayaivaite nihatāḥ pūrvameva} \\
 & \romline{nimittamātraṃ bhava savyasācin ||}
\end{tabular}
\end{table}

\begin{table}[H]
\begin{tabular}{cl}
\textbf{11.34} & \romline{droṇaṃ ca bhīṣmaṃ ca jayadrathaṃ ca} \\
 & \romline{karṇaṃ tathānyānapi yodhavīrān |} \\
 & \romline{mayā hatāṃstvaṃ jahi mā vyathiṣṭhāḥ} \\
 & \romline{yudhyasva jetāsi raṇe sapatnān ||}
\end{tabular}
\end{table}

\begin{table}[H]
\begin{tabular}{cl}
\textbf{11.35} & \romline{sañjaya uvāca} \\
 & \romline{etacchrutvā vacanaṃ keśavasya} \\
 & \romline{kṛtāñjalirvepamānaḥ kirīṭī |} \\
 & \romline{namaskṛtvā bhūya evāha kṛṣṇaṃ} \\
 & \romline{sagadgadaṃ bhītabhītaḥ praṇamya ||}
\end{tabular}
\end{table}

\begin{table}[H]
\begin{tabular}{cl}
\textbf{11.36} & \romline{arjuna uvāca} \\
 & \romline{sthāne hṛṣīkeśa tava prakīrtyā} \\
 & \romline{jagatprahṛṣyatyanurajyate ca |} \\
 & \romline{rakṣāṃsi bhītāni diśo dravanti} \\
 & \romline{sarve namasyanti ca siddhasaṅghāḥ ||}
\end{tabular}
\end{table}

\begin{table}[H]
\begin{tabular}{cl}
\textbf{11.37} & \romline{kasmācca te na nameranmahātman} \\
 & \romline{garīyase brahmaṇo'pyādikartre |} \\
 & \romline{ananta deveśa jagannivāsa} \\
 & \romline{tvamakṣaraṃ sadasattatparaṃ yat ||}
\end{tabular}
\end{table}

\begin{table}[H]
\begin{tabular}{cl}
\textbf{11.38} & \romline{tvamādidevaḥ puruṣaḥ purāṇaḥ} \\
 & \romline{tvamasya viśvasya paraṃ nidhānam |} \\
 & \romline{vettā'si vedyaṃ ca paraṃ ca dhāma} \\
 & \romline{tvayā tataṃ viśvamanantarūpa ||}
\end{tabular}
\end{table}

\begin{table}[H]
\begin{tabular}{cl}
\textbf{11.39} & \romline{vāyuryamo'gnirvaruṇaḥ śaśāṅkaḥ} \\
 & \romline{prajāpatistvaṃ prapitāmahaśca |} \\
 & \romline{namo namaste'stu sahasrakṛtvaḥ} \\
 & \romline{punaśca bhūyo'pi namo namaste ||}
\end{tabular}
\end{table}

\begin{table}[H]
\begin{tabular}{cl}
\textbf{11.40} & \romline{namaḥ purastādatha pṛṣṭhataste} \\
 & \romline{namo'stu te sarvata eva sarva |} \\
 & \romline{anantavīryāmitavikramastvaṃ} \\
 & \romline{sarvaṃ samāpnoṣi tato'si sarvaḥ ||}
\end{tabular}
\end{table}

\begin{table}[H]
\begin{tabular}{cl}
\textbf{11.41} & \romline{sakheti matvā prasabhaṃ yaduktaṃ} \\
 & \romline{he kṛṣṇa he yādava he sakheti |} \\
 & \romline{ajānatā mahimānaṃ tavedaṃ} \\
 & \romline{mayā pramādātpraṇayena vā'pi ||}
\end{tabular}
\end{table}

\begin{table}[H]
\begin{tabular}{cl}
\textbf{11.42} & \romline{yaccāpahāsārthamasatkṛto'si} \\
 & \romline{vihāraśayyāsanabhojaneṣu |} \\
 & \romline{eko'thavāpyacyuta tatsamakṣaṃ} \\
 & \romline{tatkṣāmaye tvāmahamaprameyam ||}
\end{tabular}
\end{table}

\begin{table}[H]
\begin{tabular}{cl}
\textbf{11.43} & \romline{pitāsi lokasya carācarasya} \\
 & \romline{tvamasya pūjyaśca gururgarīyān |} \\
 & \romline{na tvatsamo'styabhyadhikaḥ kuto'nyaḥ} \\
 & \romline{lokatraye'pyapratimaprabhāva ||}
\end{tabular}
\end{table}

\begin{table}[H]
\begin{tabular}{cl}
\textbf{11.44} & \romline{tasmātpraṇamya praṇidhāya kāyaṃ} \\
 & \romline{prasādaye tvāmahamīśamīḍyam |} \\
 & \romline{piteva putrasya sakheva sakhyuḥ} \\
 & \romline{priyaḥ priyāyārhasi deva soḍhum ||}
\end{tabular}
\end{table}

\begin{table}[H]
\begin{tabular}{cl}
\textbf{11.45} & \romline{adṛṣṭapūrvaṃ hṛṣito'smi dṛṣṭvā} \\
 & \romline{bhayena ca pravyathitaṃ mano me |} \\
 & \romline{tadeva me darśaya devarūpaṃ} \\
 & \romline{prasīda deveśa jagannivāsa ||}
\end{tabular}
\end{table}

\begin{table}[H]
\begin{tabular}{cl}
\textbf{11.46} & \romline{kirīṭinaṃ gadinaṃ cakrahastam} \\
 & \romline{icchāmi tvāṃ draṣṭumahaṃ tathaiva |} \\
 & \romline{tenaiva rūpeṇa caturbhujena} \\
 & \romline{sahasrabāho bhava viśvamūrte ||}
\end{tabular}
\end{table}

\begin{table}[H]
\begin{tabular}{cl}
\textbf{11.47} & \romline{śrī bhagavānuvāca} \\
 & \romline{mayā prasannena tavārjunedaṃ} \\
 & \romline{rūpaṃ paraṃ darśitamātmayogāt |} \\
 & \romline{tejomayaṃ viśvamanantamādyaṃ} \\
 & \romline{yanme tvadanyena na dṛṣṭapūrvam ||}
\end{tabular}
\end{table}

\begin{table}[H]
\begin{tabular}{cl}
\textbf{11.48} & \romline{na vedayajñādhyayanairna dānaiḥ} \\
 & \romline{na ca kriyābhirna tapobhirugraiḥ |} \\
 & \romline{evaṃrūpaḥ śakya ahaṃ nṛloke} \\
 & \romline{draṣṭuṃ tvadanyena kurupravīra ||}
\end{tabular}
\end{table}

\begin{table}[H]
\begin{tabular}{cl}
\textbf{11.49} & \romline{mā te vyathā mā ca vimūḍhabhāvaḥ} \\
 & \romline{dṛṣṭvā rūpaṃ ghoramīdṛṅmamedam |} \\
 & \romline{vyapetabhīḥ prītamanāḥ punastvaṃ} \\
 & \romline{tadeva me rūpamidaṃ prapaśya ||}
\end{tabular}
\end{table}

\begin{table}[H]
\begin{tabular}{cl}
\textbf{11.50} & \romline{sañjaya uvāca} \\
 & \romline{ityarjunaṃ vāsudevastathoktvā} \\
 & \romline{svakaṃ rūpaṃ darśayāmāsa bhūyaḥ |} \\
 & \romline{āśvāsayāmāsa ca bhītamenaṃ} \\
 & \romline{bhūtvā punaḥ saumyavapurmahātmā ||}
\end{tabular}
\end{table}

\begin{table}[H]
\begin{tabular}{cl}
\textbf{11.51} & \romline{arjuna uvāca} \\
 & \romline{dṛṣṭvedaṃ mānuṣaṃ rūpaṃ} \\
 & \romline{tava saumyaṃ janārdana |} \\
 & \romline{idānīmasmi saṃvṛttaḥ} \\
 & \romline{sacetāḥ prakṛtiṃ gataḥ ||}
\end{tabular}
\end{table}

\begin{table}[H]
\begin{tabular}{cl}
\textbf{11.52} & \romline{śrī bhagavānuvāca} \\
 & \romline{sudurdarśamidaṃ rūpaṃ} \\
 & \romline{dṛṣṭavānasi yanmama |} \\
 & \romline{devā apyasya rūpasya} \\
 & \romline{nityaṃ darśanakāṅkṣiṇaḥ ||}
\end{tabular}
\end{table}

\begin{table}[H]
\begin{tabular}{cl}
\textbf{11.53} & \romline{nāhaṃ vedairna tapasā} \\
 & \romline{na dānena na cejyayā |} \\
 & \romline{śakya evaṃvidho draṣṭuṃ} \\
 & \romline{dṛṣṭavānasi māṃ yathā ||}
\end{tabular}
\end{table}

\begin{table}[H]
\begin{tabular}{cl}
\textbf{11.54} & \romline{bhaktyā tvananyayā śakyaḥ} \\
 & \romline{ahamevaṃvidho'rjuna |} \\
 & \romline{jñātuṃ draṣṭuṃ ca tattvena} \\
 & \romline{praveṣṭuṃ ca parantapa ||}
\end{tabular}
\end{table}

\begin{table}[H]
\begin{tabular}{cl}
\textbf{11.55} & \romline{matkarmakṛnmatparamaḥ} \\
 & \romline{madbhaktaḥ saṅgavarjitaḥ |} \\
 & \romline{nirvairaḥ sarvabhūteṣu} \\
 & \romline{yaḥ sa māmeti pāṇḍava ||}
\end{tabular}
\end{table}

\begin{table}[H]
\begin{tabular}{cl}
 & \romline{śrīmadbhagavadgītāsu upaniṣatsu} \\
 & \romline{brahmavidyāyāṃ yogaśāstre} \\
 & \romline{śrīkṛṣṇārjuna saṃvāde} \\
 & \romline{viśvarūpasandarśanayogo nāma} \\
 & \romline{ekādaśodhyāyaḥ}
\end{tabular}
\end{table}

