\begin{table}[H]
\begin{tabular}{cl}
\textbf{11.0} & \romline{oṃ śrī paramātmane namaḥ} \\
 & \romline{atha ekādaśo'dhyāyaḥ} \\
 & \romline{viśvarūpa sandarśana yogaḥ}
\end{tabular}
\end{table}

\begin{table}[H]
\begin{tabular}{cl}
\textbf{11.1} & \romline{arjuna uvāca} \\
 & \romline{madanugrahāya paramaṃ} \\
 & \romline{guhya-madhyāt-masañjñitam |} \\
 & \romline{yattvayoktaṃ vacastena} \\
 & \romline{moho'yaṃ vigato mama ||}
\end{tabular}
\end{table}

\begin{table}[H]
\begin{tabular}{cl}
\textbf{11.2} & \romline{bhavāpyayau hi bhūtānāṃ} \\
 & \romline{śrutau vistaraśo mayā |} \\
 & \romline{tvattaḥ kamala-patrākṣa} \\
 & \romline{māhātmyamapi cāvyayam ||}
\end{tabular}
\end{table}

\begin{table}[H]
\begin{tabular}{cl}
\textbf{11.3} & \romline{evametad-yathā''ttha tvam} \\
 & \romline{ātmānaṃ parameśvara |} \\
 & \romline{draṣṭumicchāmi te rūpam} \\
 & \romline{aiśvaram puruṣottama ||}
\end{tabular}
\end{table}

\begin{table}[H]
\begin{tabular}{cl}
\textbf{11.4} & \romline{manyase yadi tacchakyaṃ} \\
 & \romline{mayā draṣṭumiti·prabho |} \\
 & \romline{yogeśvara tato me tvaṃ} \\
 & \romline{darśayātmāna-mavyayam ||}
\end{tabular}
\end{table}

\begin{table}[H]
\begin{tabular}{cl}
\textbf{11.5} & \romline{śrī bhagavānuvāca} \\
 & \romline{paśya me pārtha rūpāṇi} \\
 & \romline{śataśo'tha sahasraśaḥ |} \\
 & \romline{nānāvidhāni divyāni} \\
 & \romline{nānāvarṇākṛtīni ca ||}
\end{tabular}
\end{table}

\begin{table}[H]
\begin{tabular}{cl}
\textbf{11.6} & \romline{paśyā-dityān-vasūnrudrān} \\
 & \romline{aśvinau marutastathā |} \\
 & \romline{bahūnyadṛṣṭa-pūrvāṇi} \\
 & \romline{paśyāścaryāṇi bhārata ||}
\end{tabular}
\end{table}

\begin{table}[H]
\begin{tabular}{cl}
\textbf{11.7} & \romline{ihaikasthaṃ jagatkṛtsnaṃ} \\
 & \romline{paśyādya sacarācaram |} \\
 & \romline{mama dehe guḍākeśa} \\
 & \romline{yaccānyat draṣṭumicchasi ||}
\end{tabular}
\end{table}

\begin{table}[H]
\begin{tabular}{cl}
\textbf{11.8} & \romline{na tu māṃ śakyase draṣṭum} \\
 & \romline{anenaiva·svacakṣuṣā |} \\
 & \romline{divyaṃ dadāmi te cakṣuḥ} \\
 & \romline{paśya me yogamaiśvaram ||}
\end{tabular}
\end{table}

\begin{table}[H]
\begin{tabular}{cl}
\textbf{11.9} & \romline{sañjaya uvāca} \\
 & \romline{evamuktvā tato rājan} \\
 & \romline{mahā-yogeśvaro hariḥ |} \\
 & \romline{darśayā-māsa pārthāya} \\
 & \romline{paramaṃ rūpa-maiśvaram ||}
\end{tabular}
\end{table}

\begin{table}[H]
\begin{tabular}{cl}
\textbf{11.10} & \romline{aneka-vaktra-nayanam} \\
 & \romline{anekād-bhuta-darśanam |} \\
 & \romline{aneka-divyā-bharaṇaṃ} \\
 & \romline{divyā-nekodya-tāyudham ||}
\end{tabular}
\end{table}

\begin{table}[H]
\begin{tabular}{cl}
\textbf{11.11} & \romline{divya-mālyāmbara-dharaṃ} \\
 & \romline{divya-gandhānulepanam |} \\
 & \romline{sarvāścarya-mayaṃ devam} \\
 & \romline{anantaṃ viśvatomukham ||}
\end{tabular}
\end{table}

\begin{table}[H]
\begin{tabular}{cl}
\textbf{11.12} & \romline{divi sūrya-sahasrasya} \\
 & \romline{bhaved-yuga-padutthitā |} \\
 & \romline{yadi bhāḥ sadṛśī sā syāt} \\
 & \romline{bhāsastasya mahātmanaḥ ||}
\end{tabular}
\end{table}

\begin{table}[H]
\begin{tabular}{cl}
\textbf{11.13} & \romline{tatrai-kasthaṃ jagat-kṛtsnaṃ} \\
 & \romline{pravibhakta-manekadhā |} \\
 & \romline{apaśyad-devadevasya} \\
 & \romline{śarīre pāṇḍavastadā ||}
\end{tabular}
\end{table}

\begin{table}[H]
\begin{tabular}{cl}
\textbf{11.14} & \romline{tataḥ sa vismayāviṣṭaḥ} \\
 & \romline{hṛṣṭaromā dhanañjayaḥ |} \\
 & \romline{praṇamya śirasā devaṃ} \\
 & \romline{kṛtāñ-jalira-bhāṣata ||}
\end{tabular}
\end{table}

