\begin{table}[H]
\begin{tabular}{cl}
\textbf{6.0} & \romline{oṃ śrī paramātmane namaḥ} \\
 & \romline{atha ṣaṣṭho'dhyāyaḥ} \\
 & \romline{ātma-saṃyama-yogaḥ}
\end{tabular}
\end{table}

\begin{table}[H]
\begin{tabular}{cl}
\textbf{6.1} & \romline{śrī bhagavānuvāca} \\
 & \romline{anāśritaḥ karma-phalaṃ} \\
 & \romline{kāryaṃ karma karoti yaḥ |} \\
 & \romline{sa sannyāsī ca yogī ca} \\
 & \romline{na niragnirna cākriyaḥ ||}
\end{tabular}
\end{table}

\begin{table}[H]
\begin{tabular}{cl}
\textbf{6.2} & \romline{yaṃ sannyā-samiti·prāhuḥ} \\
 & \romline{yogaṃ taṃ viddhi pāṇḍava |} \\
 & \romline{na hya-sannyasta-saṅkalpaḥ} \\
 & \romline{yogī bhavati kaścana ||}
\end{tabular}
\end{table}

\begin{table}[H]
\begin{tabular}{cl}
\textbf{6.3} & \romline{āru-rukṣor-muneryogaṃ} \\
 & \romline{karma kāraṇa-mucyate |} \\
 & \romline{yogā-rūḍhasya tasyaiva} \\
 & \romline{śamaḥ kāraṇa-mucyate ||}
\end{tabular}
\end{table}

\begin{table}[H]
\begin{tabular}{cl}
\textbf{6.4} & \romline{yadā hi nendriyārtheṣu} \\
 & \romline{na karma-svanuṣajjate |} \\
 & \romline{sarva-saṅkalpa-sannyāsī} \\
 & \romline{yogā-rūḍhasta-docyate ||}
\end{tabular}
\end{table}

\begin{table}[H]
\begin{tabular}{cl}
\textbf{6.5} & \romline{uddhare-dātmanā''tmānaṃ} \\
 & \romline{nātmānama-vasādayet |} \\
 & \romline{ātmaiva hyātmano bandhuḥ} \\
 & \romline{ātmaiva ripurātmanaḥ ||}
\end{tabular}
\end{table}

\begin{table}[H]
\begin{tabular}{cl}
\textbf{6.6} & \romline{bandhurātmā''tmanastasya} \\
 & \romline{yenāt-maivāt-manā jitaḥ |} \\
 & \romline{anāt-manastu śatrutve} \\
 & \romline{varte-tātmaiva śatruvat ||}
\end{tabular}
\end{table}

\begin{table}[H]
\begin{tabular}{cl}
\textbf{6.7} & \romline{jitāt-manaḥ praśāntasya} \\
 & \romline{para-mātmā samāhitaḥ |} \\
 & \romline{śītoṣṇa-sukha-duḥkheṣu} \\
 & \romline{tathā mānāpa-mānayoḥ ||}
\end{tabular}
\end{table}

\begin{table}[H]
\begin{tabular}{cl}
\textbf{6.8} & \romline{jñāna-vijñāna-tṛptātmā} \\
 & \romline{kūṭastho vijitendriyaḥ |} \\
 & \romline{yukta ityucyate yogī} \\
 & \romline{samaloṣṭāś-makāñcanaḥ ||}
\end{tabular}
\end{table}

\begin{table}[H]
\begin{tabular}{cl}
\textbf{6.9} & \romline{suhṛn-mitrāryu-dāsīna-} \\
 & \romline{madhya-stha-dveṣya-bandhuṣu |} \\
 & \romline{sādhuṣvapi ca pāpeṣu} \\
 & \romline{samabuddhir-viśiṣyate ||}
\end{tabular}
\end{table}

\begin{table}[H]
\begin{tabular}{cl}
\textbf{6.10} & \romline{yogī yuñjīta satatam} \\
 & \romline{ātmānaṃ rahasi sthitaḥ |} \\
 & \romline{ekākī yatacittātmā} \\
 & \romline{nirāśīra-parigrahaḥ ||}
\end{tabular}
\end{table}

\begin{table}[H]
\begin{tabular}{cl}
\textbf{6.11} & \romline{śucau deśe pratiṣṭhāpya} \\
 & \romline{sthiramāsanamātmanaḥ |} \\
 & \romline{nātyucchritaṃ nātinīcaṃ} \\
 & \romline{cailājinakuśottaram ||}
\end{tabular}
\end{table}

\begin{table}[H]
\begin{tabular}{cl}
\textbf{6.12} & \romline{tatrikāgraṃ manaḥ kṛtvā} \\
 & \romline{yatacittendriyakriyaḥ |} \\
 & \romline{upaviśyāsane yuñjyāt} \\
 & \romline{yogamātmaviśuddhaye ||}
\end{tabular}
\end{table}

\begin{table}[H]
\begin{tabular}{cl}
\textbf{6.13} & \romline{samaṃ kāyaśirogrīvaṃ} \\
 & \romline{dhārayannacalaṃ sthiraḥ |} \\
 & \romline{samprekṣya nāsikāgraṃ svaṃ} \\
 & \romline{diśaścānavalokayan ||}
\end{tabular}
\end{table}

\begin{table}[H]
\begin{tabular}{cl}
\textbf{6.14} & \romline{praśāntātmā vigatabhīḥ} \\
 & \romline{brahmacārivrate sthitaḥ |} \\
 & \romline{manaḥ saṃyamya maccittaḥ} \\
 & \romline{yukta āsīta matparaḥ ||}
\end{tabular}
\end{table}

\begin{table}[H]
\begin{tabular}{cl}
\textbf{6.15} & \romline{yuñjannevaṃ sadā''tmānaṃ} \\
 & \romline{yogī niyatamānasaḥ |} \\
 & \romline{śāntiṃ nirvāṇaparamāṃ} \\
 & \romline{matsaṃsthāmadhigacchati ||}
\end{tabular}
\end{table}

\begin{table}[H]
\begin{tabular}{cl}
\textbf{6.16} & \romline{nātya-śnatastu yogo'sti} \\
 & \romline{na caikānta-manaśnataḥ |} \\
 & \romline{na cāti·svapna-śīlasya} \\
 & \romline{jāgrato naiva cārjuna ||}
\end{tabular}
\end{table}

\begin{table}[H]
\begin{tabular}{cl}
\textbf{6.17} & \romline{yuktā-hāra-vihārasya} \\
 & \romline{yukta-cEṣṭasya karmasu |} \\
 & \romline{yukta-svapnāva-bOdhasya} \\
 & \romline{yogo bhavati duḥkhahā ||}
\end{tabular}
\end{table}

\begin{table}[H]
\begin{tabular}{cl}
\textbf{6.18} & \romline{yadā viniyataṃ cittam} \\
 & \romline{ātmanyevā-vatiṣṭhate |} \\
 & \romline{nisspṛhaḥ sarva-kāmebhyaḥ} \\
 & \romline{yukta ityucyate tadā ||}
\end{tabular}
\end{table}

\begin{table}[H]
\begin{tabular}{cl}
\textbf{6.19} & \romline{yathā dīpo nivā-tasthaḥ} \\
 & \romline{neṅgate sopamā smṛtā |} \\
 & \romline{yogino yatacittasya} \\
 & \romline{yuñjato yogomātmanaḥ ||}
\end{tabular}
\end{table}

\begin{table}[H]
\begin{tabular}{cl}
\textbf{6.20} & \romline{yatro-paramate cittaṃ} \\
 & \romline{niruddhaṃ yoga-sevayā |} \\
 & \romline{yatra caivāt-manā''t-mānaṃ} \\
 & \romline{pasyan-nātmani tuṣyati ||}
\end{tabular}
\end{table}

\begin{table}[H]
\begin{tabular}{cl}
\textbf{6.21} & \romline{sukhamātyantikaṃ yattat} \\
 & \romline{buddhi-grāhyamatīndriyam |} \\
 & \romline{vetti yatra na caivāyaṃ} \\
 & \romline{sthitaścalati tattvataḥ ||}
\end{tabular}
\end{table}

\begin{table}[H]
\begin{tabular}{cl}
\textbf{6.22} & \romline{yaṃ labdhvā cāparaṃ lābhaṃ} \\
 & \romline{manyate nādhikaṃ tataḥ |} \\
 & \romline{yasmin sthito na duḥkhena} \\
 & \romline{guruṇāpi vicālyate ||}
\end{tabular}
\end{table}

\begin{table}[H]
\begin{tabular}{cl}
\textbf{6.23} & \romline{taṃ vidyāt duḥkha-saṃyoga-} \\
 & \romline{viyogaṃ yogasañjñitam |} \\
 & \romline{sa niścayena yoktavyaḥ} \\
 & \romline{yogo'nirviṇṇa-cetasā ||}
\end{tabular}
\end{table}

\begin{table}[H]
\begin{tabular}{cl}
\textbf{6.24} & \romline{saṅkalpa-prabhavān-kāmān} \\
 & \romline{tyaktvā sarvāna-śeṣataḥ |} \\
 & \romline{manasaivendriya-grāmaṃ} \\
 & \romline{viniyamya samantataḥ ||}
\end{tabular}
\end{table}

\begin{table}[H]
\begin{tabular}{cl}
\textbf{6.25} & \romline{śanaiḥ śanai-ruparamet} \\
 & \romline{buddhyā dhṛti-gṛhītayā |} \\
 & \romline{ātmasaṃsthaṃ manaḥ kṛtvā} \\
 & \romline{na kiñcidapi cintayet ||}
\end{tabular}
\end{table}

\begin{table}[H]
\begin{tabular}{cl}
\textbf{6.26} & \romline{yato yato niścarati} \\
 & \romline{manaścañcala-masthiram |} \\
 & \romline{tatastato niyam-yaitat} \\
 & \romline{ātmanyeva vaśaṃ nayet ||}
\end{tabular}
\end{table}

\begin{table}[H]
\begin{tabular}{cl}
\textbf{6.27} & \romline{praśānta-manasaṃ hyenaṃ} \\
 & \romline{yoginaṃ sukha-muttamam |} \\
 & \romline{upaiti śānta-rajasaṃ} \\
 & \romline{brahma-bhūta-makalmaṣam ||}
\end{tabular}
\end{table}

\begin{table}[H]
\begin{tabular}{cl}
\textbf{6.28} & \romline{yuñjannevaṃ sadā''t-mānaṃ} \\
 & \romline{yogī vigata-kalmaṣaḥ |} \\
 & \romline{sukhena brahma-saṃsparśaṃ} \\
 & \romline{atyantaṃ sukkha-maśnute ||}
\end{tabular}
\end{table}

\begin{table}[H]
\begin{tabular}{cl}
\textbf{6.29} & \romline{sarva-bhūtastha-mātmānaṃ} \\
 & \romline{sarva-bhūtāni cātmani |} \\
 & \romline{īkṣate yoga-yuktātmā} \\
 & \romline{sarvatra samadarśanaḥ ||}
\end{tabular}
\end{table}

\begin{table}[H]
\begin{tabular}{cl}
\textbf{6.30} & \romline{yo māṃ paśyati sarvatra} \\
 & \romline{sarvaṃ ca mayi paśyati |} \\
 & \romline{tasyāhaṃ na·praṇaśyāmi} \\
 & \romline{sa ca me na·praṇaśyati ||}
\end{tabular}
\end{table}

\begin{table}[H]
\begin{tabular}{cl}
\textbf{6.31} & \romline{sarva-bhūta-sthitaṃ yo māṃ} \\
 & \romline{bhajatye-katva-māsthitaḥ |} \\
 & \romline{sarvathā vartamāno'pi} \\
 & \romline{sa yogī mayi vartate ||}
\end{tabular}
\end{table}

\begin{table}[H]
\begin{tabular}{cl}
\textbf{6.32} & \romline{ātmau-pamyena sarvatra} \\
 & \romline{samaṃ paśyati yo'rjuna |} \\
 & \romline{sukhaṃ vā yadi vā duḥkhaṃ} \\
 & \romline{sa yogī paramo matatḥ ||}
\end{tabular}
\end{table}

\begin{table}[H]
\begin{tabular}{cl}
\textbf{6.33} & \romline{arjuna uvāca} \\
 & \romline{yo'yaṃ yogastvayā proktaḥ} \\
 & \romline{sāmyena madhu-sūdana |} \\
 & \romline{etas-yāhaṃ na paśyāmi} \\
 & \romline{cañcalatvāt sthitiṃ sthirām ||}
\end{tabular}
\end{table}

\begin{table}[H]
\begin{tabular}{cl}
\textbf{6.34} & \romline{cañcalaṃ hi manaḥ kṛṣṇa} \\
 & \romline{pramāthi balavad-dṛḍham |} \\
 & \romline{tasyāhaṃ nigrahaṃ manye} \\
 & \romline{vāyoriva suduṣkaram ||}
\end{tabular}
\end{table}

\begin{table}[H]
\begin{tabular}{cl}
\textbf{6.35} & \romline{śrī bhagavānuvāca} \\
 & \romline{asaṃśayaṃ mahābāho} \\
 & \romline{mano durnigrahaṃ calam |} \\
 & \romline{abhyāsena tu kaunteya} \\
 & \romline{vairāgyeṇa ca gṛhyate ||}
\end{tabular}
\end{table}

\begin{table}[H]
\begin{tabular}{cl}
\textbf{6.36} & \romline{asaṃ-yatāt-manā yogaḥ} \\
 & \romline{duṣprāpa iti me matiḥ |} \\
 & \romline{vaśyāt-manā tu yatatā} \\
 & \romline{śakyo'vāptu-mupāyataḥ ||}
\end{tabular}
\end{table}

\begin{table}[H]
\begin{tabular}{cl}
\textbf{6.37} & \romline{arjuna uvāca} \\
 & \romline{ayatiḥ śraddha-yopetaḥ} \\
 & \romline{yogāccalita-mānasaḥ |} \\
 & \romline{aprāpya yoga-saṃsiddhiṃ} \\
 & \romline{kāṃ gatiṃ kṛṣṇa gacchati ||}
\end{tabular}
\end{table}

\begin{table}[H]
\begin{tabular}{cl}
\textbf{6.38} & \romline{kaccinno-bhaya-vibhraṣṭaḥ} \\
 & \romline{cinnā-bhra-miva naśyati |} \\
 & \romline{apratiṣṭho mahābāho} \\
 & \romline{vimūḍho brahmaṇaḥ pathi ||}
\end{tabular}
\end{table}

\begin{table}[H]
\begin{tabular}{cl}
\textbf{6.39} & \romline{etanme saṃśayaṃ kṛṣṇa} \\
 & \romline{chettu-marhasya-śeṣataḥ |} \\
 & \romline{tvadanyaḥ saṃśa-yasyāsya} \\
 & \romline{chettā na hyupa-padyate ||}
\end{tabular}
\end{table}

\begin{table}[H]
\begin{tabular}{cl}
\textbf{6.40} & \romline{śrī bhagavānuvāca} \\
 & \romline{pārtha naiveha nāmutra} \\
 & \romline{vināśastasya vidyate |} \\
 & \romline{na hi kalyāṇa-kṛt-kaścit} \\
 & \romline{durgatiṃ tāta gacchati ||}
\end{tabular}
\end{table}

