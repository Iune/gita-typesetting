\begin{table}[H]
\begin{tabular}{cl}
 & \romline{śrī paramātmane namaḥ} \\
 & \romline{atha ṣaṣṭho'dhyāyaḥ} \\
 & \romline{ātmasaṃyamayogaḥ}
\end{tabular}
\end{table}

\begin{table}[H]
\begin{tabular}{cl}
\textbf{6.1} & \romline{śrī bhagavānuvāca} \\
 & \romline{anāśritaḥ karmaphalaṃ} \\
 & \romline{kāryaṃ karma karoti yaḥ |} \\
 & \romline{sa sannyāsī ca yogī ca} \\
 & \romline{na niragnirna cākriyaḥ ||}
\end{tabular}
\end{table}

\begin{table}[H]
\begin{tabular}{cl}
\textbf{6.2} & \romline{yaṃ sannyāsamiti prāhuḥ} \\
 & \romline{yogaṃ taṃ viddhi pāṇḍava |} \\
 & \romline{na hyasannyastasaṅkalpaḥ} \\
 & \romline{yogī bhavati kaścana ||}
\end{tabular}
\end{table}

\begin{table}[H]
\begin{tabular}{cl}
\textbf{6.3} & \romline{ārurukṣormuneryogaṃ} \\
 & \romline{karma kāraṇamucyate |} \\
 & \romline{yogārūḍhasya tasyaiva} \\
 & \romline{śamaḥ kāraṇamucyate ||}
\end{tabular}
\end{table}

\begin{table}[H]
\begin{tabular}{cl}
\textbf{6.4} & \romline{yadā hi nendriyārtheṣu} \\
 & \romline{na karmasvanuṣajjate |} \\
 & \romline{sarvasaṅkalpasannyāsī} \\
 & \romline{yogārūḍhastadocyate ||}
\end{tabular}
\end{table}

\begin{table}[H]
\begin{tabular}{cl}
\textbf{6.5} & \romline{uddharedātmanā''tmānaṃ} \\
 & \romline{nātmānamavasādayet |} \\
 & \romline{ātmaiva hyātmano bandhuḥ} \\
 & \romline{ātmaiva ripurātmanaḥ ||}
\end{tabular}
\end{table}

\begin{table}[H]
\begin{tabular}{cl}
\textbf{6.6} & \romline{bandhurātmā''tmanastasya} \\
 & \romline{yenātmaivātmanā jitaḥ |} \\
 & \romline{anātmanastu śatrutve} \\
 & \romline{vartetātmaiva śatruvat ||}
\end{tabular}
\end{table}

\begin{table}[H]
\begin{tabular}{cl}
\textbf{6.7} & \romline{jitātmanaḥ praśāntasya} \\
 & \romline{paramātmā samāhitaḥ |} \\
 & \romline{śītoṣṇasukhaduḥkheṣu} \\
 & \romline{tathā mānāpamānayoḥ ||}
\end{tabular}
\end{table}

\begin{table}[H]
\begin{tabular}{cl}
\textbf{6.8} & \romline{jñānavijñānatṛptātmā} \\
 & \romline{kūṭastho vijitendriyaḥ |} \\
 & \romline{yukta ityucyate yogī} \\
 & \romline{samaloṣṭāśmakāñcanaḥ ||}
\end{tabular}
\end{table}

\begin{table}[H]
\begin{tabular}{cl}
\textbf{6.9} & \romline{suhṛnmitrāryudāsīna} \\
 & \romline{madhyasthadveṣyabandhuṣu |} \\
 & \romline{sādhuṣvapi ca pāpeṣu} \\
 & \romline{samabuddhirviśiṣyate ||}
\end{tabular}
\end{table}

\begin{table}[H]
\begin{tabular}{cl}
\textbf{6.10} & \romline{yogī yuñjīta satatam} \\
 & \romline{ātmānaṃ rahasi sthitaḥ |} \\
 & \romline{ekākī yatacittātmā} \\
 & \romline{nirāśīraparigrahaḥ ||}
\end{tabular}
\end{table}

\begin{table}[H]
\begin{tabular}{cl}
\textbf{6.11} & \romline{śucau deśe pratiṣṭhāpya} \\
 & \romline{sthiramāsanamātmanaḥ |} \\
 & \romline{nātyucchritaṃ nātinīcaṃ} \\
 & \romline{cailājinakuśottaram ||}
\end{tabular}
\end{table}

\begin{table}[H]
\begin{tabular}{cl}
\textbf{6.12} & \romline{tatraikāgraṃ manaḥ kṛtvā} \\
 & \romline{yatacittendriyakriyaḥ |} \\
 & \romline{upaviśyāsane yuñjyāt} \\
 & \romline{yogamātmaviśuddhaye ||}
\end{tabular}
\end{table}

\begin{table}[H]
\begin{tabular}{cl}
\textbf{6.13} & \romline{samaṃ kāyaśirogrīvaṃ} \\
 & \romline{dhārayannacalaṃ sthiraḥ |} \\
 & \romline{samprekṣya nāsikāgraṃ svaṃ} \\
 & \romline{diśaścānavalokayan ||}
\end{tabular}
\end{table}

\begin{table}[H]
\begin{tabular}{cl}
\textbf{6.14} & \romline{praśāntātmā vigatabhīḥ} \\
 & \romline{brahmacārivrate sthitaḥ |} \\
 & \romline{manaḥ saṃyamya maccittaḥ} \\
 & \romline{yukta āsīta matparaḥ ||}
\end{tabular}
\end{table}

\begin{table}[H]
\begin{tabular}{cl}
\textbf{6.15} & \romline{yuñjannevaṃ sadā''tmānaṃ} \\
 & \romline{yogī niyatamānasaḥ |} \\
 & \romline{śāntiṃ nirvāṇaparamāṃ} \\
 & \romline{matsaṃsthāmadhigacchati ||}
\end{tabular}
\end{table}

\begin{table}[H]
\begin{tabular}{cl}
\textbf{6.16} & \romline{nātyaśnatastu yogo'sti} \\
 & \romline{na caikāntamanaśnataḥ |} \\
 & \romline{na cāti svapnaśīlasya} \\
 & \romline{jāgrato naiva cārjuna ||}
\end{tabular}
\end{table}

\begin{table}[H]
\begin{tabular}{cl}
\textbf{6.17} & \romline{yuktāhāravihārasya} \\
 & \romline{yuktaceṣṭasya karmasu |} \\
 & \romline{yuktasvapnāvabodhasya} \\
 & \romline{yogo bhavati duḥkhahā ||}
\end{tabular}
\end{table}

\begin{table}[H]
\begin{tabular}{cl}
\textbf{6.18} & \romline{yadā viniyataṃ cittam} \\
 & \romline{ātmanyevāvatiṣṭhate |} \\
 & \romline{nisspṛhaḥ sarvakāmebhyaḥ} \\
 & \romline{yukta ityucyate tadā ||}
\end{tabular}
\end{table}

\begin{table}[H]
\begin{tabular}{cl}
\textbf{6.19} & \romline{yathā dīpo nivātasthaḥ} \\
 & \romline{neṅgate sopamā smṛtā |} \\
 & \romline{yogino yatacittasya} \\
 & \romline{yuñjato yogomātmanaḥ ||}
\end{tabular}
\end{table}

\begin{table}[H]
\begin{tabular}{cl}
\textbf{6.20} & \romline{yatroparamate cittaṃ} \\
 & \romline{niruddhaṃ yogasevayā |} \\
 & \romline{yatra caivātmanā''tmānaṃ} \\
 & \romline{pasyannātmani tuṣyati ||}
\end{tabular}
\end{table}

\begin{table}[H]
\begin{tabular}{cl}
\textbf{6.21} & \romline{sukhamātyantikaṃ yattat} \\
 & \romline{buddhigrāhyamatīndriyam |} \\
 & \romline{vetti yatra na caivāyaṃ} \\
 & \romline{sthitaścalati tattvataḥ ||}
\end{tabular}
\end{table}

\begin{table}[H]
\begin{tabular}{cl}
\textbf{6.22} & \romline{yaṃ labdhvā cāparaṃ lābhaṃ} \\
 & \romline{manyate nādhikaṃ tataḥ |} \\
 & \romline{yasmin sthito na duḥkhena} \\
 & \romline{guruṇāpi vicālyate ||}
\end{tabular}
\end{table}

\begin{table}[H]
\begin{tabular}{cl}
\textbf{6.23} & \romline{taṃ vidyāt duḥkhasaṃyoga} \\
 & \romline{viyogaṃ yogasañjñitam |} \\
 & \romline{sa niścayena yoktavyaḥ} \\
 & \romline{yogo'nirviṇṇacetasā ||}
\end{tabular}
\end{table}

\begin{table}[H]
\begin{tabular}{cl}
\textbf{6.24} & \romline{saṅkalpaprabhavānkāmān} \\
 & \romline{tyaktvā sarvānaśeṣataḥ |} \\
 & \romline{manasaivendriyagrāmaṃ} \\
 & \romline{viniyamya samantataḥ ||}
\end{tabular}
\end{table}

\begin{table}[H]
\begin{tabular}{cl}
\textbf{6.25} & \romline{śanaiḥ śanairuparamet} \\
 & \romline{buddhyā dhṛtigṛhītayā |} \\
 & \romline{ātmasaṃsthaṃ manaḥ kṛtvā} \\
 & \romline{na kiñcidapi cintayet ||}
\end{tabular}
\end{table}

\begin{table}[H]
\begin{tabular}{cl}
\textbf{6.26} & \romline{yato yato niścarati} \\
 & \romline{manaścañcalamasthiram |} \\
 & \romline{tatastato niyamyaitat} \\
 & \romline{ātmanyeva vaśaṃ nayet ||}
\end{tabular}
\end{table}

\begin{table}[H]
\begin{tabular}{cl}
\textbf{6.27} & \romline{praśāntamanasaṃ hyenaṃ} \\
 & \romline{yoginaṃ sukhamuttamam |} \\
 & \romline{upaiti śāntarajasaṃ} \\
 & \romline{brahmabhūtamakalmaṣam ||}
\end{tabular}
\end{table}

\begin{table}[H]
\begin{tabular}{cl}
\textbf{6.28} & \romline{yuñjannevaṃ sadā''tmānaṃ} \\
 & \romline{yogī vigatakalmaṣaḥ |} \\
 & \romline{sukhena brahmasaṃsparśaṃ} \\
 & \romline{atyantaṃ sukhamaśnute ||}
\end{tabular}
\end{table}

\begin{table}[H]
\begin{tabular}{cl}
\textbf{6.29} & \romline{sarvabhūtasthamātmānaṃ} \\
 & \romline{sarvabhūtāni cātmani |} \\
 & \romline{īkṣate yogayuktātmā} \\
 & \romline{sarvatra samadarśanaḥ ||}
\end{tabular}
\end{table}

\begin{table}[H]
\begin{tabular}{cl}
\textbf{6.30} & \romline{yo māṃ paśyati sarvatra} \\
 & \romline{sarvaṃ ca mayi paśyati |} \\
 & \romline{tasyāhaṃ na praṇaśyāmi} \\
 & \romline{sa ca me na praṇaśyati ||}
\end{tabular}
\end{table}

\begin{table}[H]
\begin{tabular}{cl}
\textbf{6.31} & \romline{sarvabhūtasthitaṃ yo māṃ} \\
 & \romline{bhajatyekatvamāsthitaḥ |} \\
 & \romline{sarvathā vartamāno'pi} \\
 & \romline{sa yogī mayi vartate ||}
\end{tabular}
\end{table}

\begin{table}[H]
\begin{tabular}{cl}
\textbf{6.32} & \romline{ātmaupamyena sarvatra} \\
 & \romline{samaṃ paśyati yo'rjuna |} \\
 & \romline{sukhaṃ vā yadi vā duḥkhaṃ} \\
 & \romline{sa yogī paramo mataḥ ||}
\end{tabular}
\end{table}

\begin{table}[H]
\begin{tabular}{cl}
\textbf{6.33} & \romline{arjuna uvāca} \\
 & \romline{yo'yaṃ yogastvayā proktaḥ} \\
 & \romline{sāmyena madhusūdana |} \\
 & \romline{etasyāhaṃ na paśyāmi} \\
 & \romline{cañcalatvāt sthitiṃ sthirām ||}
\end{tabular}
\end{table}

\begin{table}[H]
\begin{tabular}{cl}
\textbf{6.34} & \romline{cañcalaṃ hi manaḥ kṛṣṇa} \\
 & \romline{pramāthi balavaddṛḍham |} \\
 & \romline{tasyāhaṃ nigrahaṃ manye} \\
 & \romline{vāyoriva suduṣkaram ||}
\end{tabular}
\end{table}

\begin{table}[H]
\begin{tabular}{cl}
\textbf{6.35} & \romline{śrī bhagavānuvāca} \\
 & \romline{asaṃśayaṃ mahābāho} \\
 & \romline{mano durnigrahaṃ calam |} \\
 & \romline{abhyāsena tu kaunteya} \\
 & \romline{vairāgyeṇa ca gṛhyate ||}
\end{tabular}
\end{table}

\begin{table}[H]
\begin{tabular}{cl}
\textbf{6.36} & \romline{asaṃyatātmanā yogaḥ} \\
 & \romline{duṣprāpa iti me matiḥ |} \\
 & \romline{vaśyātmanā tu yatatā} \\
 & \romline{śakyo'vāptumupāyataḥ ||}
\end{tabular}
\end{table}

\begin{table}[H]
\begin{tabular}{cl}
\textbf{6.37} & \romline{arjuna uvāca} \\
 & \romline{ayatiḥ śraddhayopetaḥ} \\
 & \romline{yogāccalitamānasaḥ |} \\
 & \romline{aprāpya yogasaṃsiddhiṃ} \\
 & \romline{kāṃ gatiṃ kṛṣṇa gacchati ||}
\end{tabular}
\end{table}

\begin{table}[H]
\begin{tabular}{cl}
\textbf{6.38} & \romline{kaccinnobhayavibhraṣṭaḥ} \\
 & \romline{chinnābhramiva naśyati |} \\
 & \romline{apratiṣṭho mahābāho} \\
 & \romline{vimūḍho brahmaṇaḥ pathi ||}
\end{tabular}
\end{table}

\begin{table}[H]
\begin{tabular}{cl}
\textbf{6.39} & \romline{etanme saṃśayaṃ kṛṣṇa} \\
 & \romline{chettumarhasyaśeṣataḥ |} \\
 & \romline{tvadanyaḥ saṃśayasyāsya} \\
 & \romline{chettā na hyupapadyate ||}
\end{tabular}
\end{table}

\begin{table}[H]
\begin{tabular}{cl}
\textbf{6.40} & \romline{śrī bhagavānuvāca} \\
 & \romline{pārtha naiveha nāmutra} \\
 & \romline{vināśastasya vidyate |} \\
 & \romline{na hi kalyāṇakṛtkaścit} \\
 & \romline{durgatiṃ tāta gacchati ||}
\end{tabular}
\end{table}

\begin{table}[H]
\begin{tabular}{cl}
\textbf{6.41} & \romline{prāpya puṇyakṛtāṃ lokāṇ} \\
 & \romline{uṣitvā śāśvatīḥ samāḥ |} \\
 & \romline{śucīnāṃ śrīmatāṃ gehe} \\
 & \romline{yogabhraṣṭo'bhijāyate ||}
\end{tabular}
\end{table}

\begin{table}[H]
\begin{tabular}{cl}
\textbf{6.42} & \romline{athavā yogināmeva} \\
 & \romline{kule bhavati dhīmatām |} \\
 & \romline{etaddhi durlabhataraṃ} \\
 & \romline{loke janma yadīdṛśam ||}
\end{tabular}
\end{table}

\begin{table}[H]
\begin{tabular}{cl}
\textbf{6.43} & \romline{tatra taṃ buddhisaṃyogaṃ} \\
 & \romline{labhate paurvadehikam |} \\
 & \romline{yatate ca tato bhūyaḥ} \\
 & \romline{saṃsiddhau kurunandana ||}
\end{tabular}
\end{table}

\begin{table}[H]
\begin{tabular}{cl}
\textbf{6.44} & \romline{purvābhyāsena tenaiva} \\
 & \romline{hriyate hyavaśo'pi saḥ |} \\
 & \romline{jijñāsurapi yogasya} \\
 & \romline{śabdabrahmātivartate ||}
\end{tabular}
\end{table}

\begin{table}[H]
\begin{tabular}{cl}
\textbf{6.45} & \romline{prayatnādyatamānastu} \\
 & \romline{yogī saṃśuddhakilbiṣaḥ |} \\
 & \romline{anekajanmasaṃsiddhaḥ} \\
 & \romline{tato yāti parāṃ gatim ||}
\end{tabular}
\end{table}

\begin{table}[H]
\begin{tabular}{cl}
\textbf{6.46} & \romline{tapasvibhyo'dhiko yogī} \\
 & \romline{jñānibhyo'pi mato'dhikaḥ |} \\
 & \romline{karmibhyaścādhiko yogī} \\
 & \romline{tasmādyogī bhavārjuna ||}
\end{tabular}
\end{table}

\begin{table}[H]
\begin{tabular}{cl}
\textbf{6.47} & \romline{yogināmapi sarveṣāṃ} \\
 & \romline{madgatenāntarātmanā |} \\
 & \romline{śraddhāvānbhajate yo māṃ} \\
 & \romline{sa me yuktatamo mataḥ ||}
\end{tabular}
\end{table}

\begin{table}[H]
\begin{tabular}{cl}
 & \romline{śrīmadbhagavadgītāsu upaniṣatsu} \\
 & \romline{brahmavidyāyāṃ yogaśāstre} \\
 & \romline{śrīkṛṣṇārjuna saṃvāde} \\
 & \romline{ātmasaṃyamayogo nāma} \\
 & \romline{ṣaṣṭhodhyāyaḥ}
\end{tabular}
\end{table}

