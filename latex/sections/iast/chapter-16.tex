\begin{table}[H]
\begin{tabular}{cl}
 & \romline{śrī paramātmane namaḥ} \\
 & \romline{atha ṣodaśo'dhyāyaḥ} \\
 & \romline{daivāsurasampadvibhāgayogaḥ}
\end{tabular}
\end{table}

\begin{table}[H]
\begin{tabular}{cl}
\textbf{16.1} & \romline{śrī bhagavānuvāca} \\
 & \romline{abhayaṃ sattvasaṃśuddhiḥ} \\
 & \romline{jñānayogavyavasthitiḥ |} \\
 & \romline{dānaṃ damaśca yajñaśca} \\
 & \romline{svādhyāyastapa ārjavam ||}
\end{tabular}
\end{table}

\begin{table}[H]
\begin{tabular}{cl}
\textbf{16.2} & \romline{ahiṃsā satyamakrodhaḥ} \\
 & \romline{tyāgaḥ śāntirapaiśunam |} \\
 & \romline{dayā bhūteṣvaloluptvaṃ} \\
 & \romline{mārdavaṃ hrīracāpalam ||}
\end{tabular}
\end{table}

\begin{table}[H]
\begin{tabular}{cl}
\textbf{16.3} & \romline{tejaḥ kṣamā dhṛtiḥ śaucam} \\
 & \romline{adroho nātimānitā |} \\
 & \romline{bhavanti sampadaṃ daivīm} \\
 & \romline{abhijātasya bhārata ||}
\end{tabular}
\end{table}

\begin{table}[H]
\begin{tabular}{cl}
\textbf{16.4} & \romline{dambho darpo'bhimānaśca} \\
 & \romline{krodhaḥ pāruṣyameva ca |} \\
 & \romline{ajñānaṃ cābhijātasya} \\
 & \romline{pārtha sampadamāsurīm ||}
\end{tabular}
\end{table}

\begin{table}[H]
\begin{tabular}{cl}
\textbf{16.5} & \romline{daivī sampadvimokṣāya} \\
 & \romline{nibandhāyāsurī matā |} \\
 & \romline{mā śucaḥ sampadaṃ daivīm} \\
 & \romline{abhijāto'si pāṇḍava ||}
\end{tabular}
\end{table}

\begin{table}[H]
\begin{tabular}{cl}
\textbf{16.6} & \romline{dvau bhūtasargau loke'smin} \\
 & \romline{daiva āsura eva ca |} \\
 & \romline{daivo vistaraśaḥ proktaḥ} \\
 & \romline{āsuraṃ pārtha me śṛṇu ||}
\end{tabular}
\end{table}

\begin{table}[H]
\begin{tabular}{cl}
\textbf{16.7} & \romline{pravṛttiṃ ca nivṛttiṃ ca} \\
 & \romline{janā na vidurāsurāḥ |} \\
 & \romline{na śaucaṃ nāpi cācāraḥ} \\
 & \romline{na satyaṃ teṣu vidyate ||}
\end{tabular}
\end{table}

\begin{table}[H]
\begin{tabular}{cl}
\textbf{16.8} & \romline{asatyamapratiṣṭhaṃ te} \\
 & \romline{jagadāhuranīśvaram |} \\
 & \romline{aparasparasambhūtaṃ} \\
 & \romline{kimanyatkāmahaitukam ||}
\end{tabular}
\end{table}

\begin{table}[H]
\begin{tabular}{cl}
\textbf{16.9} & \romline{etāṃ dṛṣṭimavaṣṭabhya} \\
 & \romline{naṣṭātmāno'lpabuddhayaḥ |} \\
 & \romline{prabhavantyugrakarmāṇaḥ} \\
 & \romline{kṣayāya jagato'hitāḥ ||}
\end{tabular}
\end{table}

\begin{table}[H]
\begin{tabular}{cl}
\textbf{16.10} & \romline{kāmamāśritya duṣpūraṃ} \\
 & \romline{dambhamānamadānvitāḥ |} \\
 & \romline{mohādgṛhītvāsadgrāhān} \\
 & \romline{pravartante'śucivratāḥ ||}
\end{tabular}
\end{table}

\begin{table}[H]
\begin{tabular}{cl}
\textbf{16.11} & \romline{cintāmaparimeyāṃ ca} \\
 & \romline{pralayāntāmupāśritāḥ |} \\
 & \romline{kāmopabhogaparamāḥ} \\
 & \romline{etāvaditi niścitāḥ ||}
\end{tabular}
\end{table}

\begin{table}[H]
\begin{tabular}{cl}
\textbf{16.12} & \romline{āśāpāśaśatairbaddhāḥ} \\
 & \romline{kāmakrodhaparāyaṇāḥ |} \\
 & \romline{īhante kāmabhogārtham} \\
 & \romline{anyāyenārthasañcayān ||}
\end{tabular}
\end{table}

\begin{table}[H]
\begin{tabular}{cl}
\textbf{16.13} & \romline{idamadya mayā labdham} \\
 & \romline{imaṃ prāpsye manoratham |} \\
 & \romline{idamastīdamapi me} \\
 & \romline{bhaviṣyati punardhanam ||}
\end{tabular}
\end{table}

\begin{table}[H]
\begin{tabular}{cl}
\textbf{16.14} & \romline{asau mayā hataḥ śatṛḥ} \\
 & \romline{haniṣye cāparānapi |} \\
 & \romline{īśvaro'hamahaṃ bhogī} \\
 & \romline{siddho'haṃ balavānsukhī ||}
\end{tabular}
\end{table}

\begin{table}[H]
\begin{tabular}{cl}
\textbf{16.15} & \romline{āḍhyo'bhijanavānasmi} \\
 & \romline{ko'nyo'sti sadṛśo mayā |} \\
 & \romline{yakṣye dāsyāmi modiṣye} \\
 & \romline{ityajñānavimohitāḥ ||}
\end{tabular}
\end{table}

\begin{table}[H]
\begin{tabular}{cl}
\textbf{16.16} & \romline{anekacittavibhrāntāḥ} \\
 & \romline{mohajālasamāvṛtāḥ |} \\
 & \romline{prasaktāḥ kāmabhogeṣu} \\
 & \romline{patanti narake'śucau ||}
\end{tabular}
\end{table}

\begin{table}[H]
\begin{tabular}{cl}
\textbf{16.17} & \romline{ātmasambhāvitāḥ stabdhāḥ} \\
 & \romline{dhanamānamadānvitāḥ |} \\
 & \romline{yajante nāmayajñaiste} \\
 & \romline{dambhenāvidhipūrvakam ||}
\end{tabular}
\end{table}

\begin{table}[H]
\begin{tabular}{cl}
\textbf{16.18} & \romline{ahaṅkāraṃ balaṃ darpaṃ} \\
 & \romline{kāmaṃ krodhaṃ ca saṃśritāḥ |} \\
 & \romline{māmātmaparadeheṣu} \\
 & \romline{pradviṣanto'bhyasūyakāḥ ||}
\end{tabular}
\end{table}

\begin{table}[H]
\begin{tabular}{cl}
\textbf{16.19} & \romline{tānahaṃ dviṣataḥ krūrān} \\
 & \romline{saṃsāreṣu narādhamān |} \\
 & \romline{kṣipāmyajasramaśubhān} \\
 & \romline{āsurīṣveva yoniṣu ||}
\end{tabular}
\end{table}

\begin{table}[H]
\begin{tabular}{cl}
\textbf{16.20} & \romline{āsurīṃ yonimāpannāḥ} \\
 & \romline{mūḍhā janmani janmani |} \\
 & \romline{māmaprāpyaiva kaunteya} \\
 & \romline{tato yāntyadhamāṃ gatim ||}
\end{tabular}
\end{table}

\begin{table}[H]
\begin{tabular}{cl}
\textbf{16.21} & \romline{trividhaṃ narakasyedaṃ} \\
 & \romline{dvāraṃ nāśanamātmanaḥ |} \\
 & \romline{kāmaḥ krodhastathā lobhaḥ} \\
 & \romline{tasmādetattrayaṃ tyajet ||}
\end{tabular}
\end{table}

\begin{table}[H]
\begin{tabular}{cl}
\textbf{16.22} & \romline{etairvimuktaḥ kaunteya} \\
 & \romline{tamodvāraistribhirnaraḥ |} \\
 & \romline{ācaratyātmanaḥ śreyaḥ} \\
 & \romline{tato yāti parāṃ gatim ||}
\end{tabular}
\end{table}

\begin{table}[H]
\begin{tabular}{cl}
\textbf{16.23} & \romline{yaḥ śāstravidhimutsṛjya} \\
 & \romline{vartate kāmakārataḥ |} \\
 & \romline{na sa siddhimavāpnoti} \\
 & \romline{na sukhaṃ na parāṃ gatim ||}
\end{tabular}
\end{table}

\begin{table}[H]
\begin{tabular}{cl}
\textbf{16.24} & \romline{tasmācchāstraṃ pramāṇaṃ te} \\
 & \romline{kāryākāryavyavasthitau |} \\
 & \romline{jñātvā śāstravidhānoktaṃ} \\
 & \romline{karma kartumihārhasi ||}
\end{tabular}
\end{table}

\begin{table}[H]
\begin{tabular}{cl}
 & \romline{śrīmadbhagavadgītāsu upaniṣatsu} \\
 & \romline{brahmavidyāyāṃ yogaśāstre} \\
 & \romline{śrīkṛṣṇārjuna saṃvāde} \\
 & \romline{daivāsurasampadvibhāgayogo nāma} \\
 & \romline{ṣodaśodhyāyaḥ}
\end{tabular}
\end{table}

