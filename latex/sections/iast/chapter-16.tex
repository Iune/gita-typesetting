\subsection*{16.0}
\begin{table}[H]
\begin{tabular}{l}
\romline{oṃ śrī paramātmane namaḥ} \\
\romline{atha ṣodaśo'dhyāyaḥ} \\
\romline{daivāsura-sampad-vibhāga-yogaḥ}
\end{tabular}
\end{table}

\subsection*{16.1}
\begin{table}[H]
\begin{tabular}{l}
\romline{śrī bhagavānuvāca} \\
\romline{abhayaṃ sattvasaṃśuddhiḥ} \\
\romline{jñānayogavyavasthitiḥ} \\
\romline{dānaṃ damaśca yajñaśca} \\
\romline{svādhyāyastapa ārjavam}
\end{tabular}
\end{table}

\subsection*{16.2}
\begin{table}[H]
\begin{tabular}{l}
\romline{ahiṃsā satyamakrodhaḥ} \\
\romline{tyāgaḥ śāntirapaiśunam} \\
\romline{dayā bhūteṣvaloluptvaṃ} \\
\romline{mārdavaṃ hrīracāpalam}
\end{tabular}
\end{table}

\subsection*{16.3}
\begin{table}[H]
\begin{tabular}{l}
\romline{tejaḥ kṣamā dhṛtiḥ śaucam} \\
\romline{adroho nātimānitā} \\
\romline{bhavanti sampadaṃ daivīm} \\
\romline{abhijātasya bhārata}
\end{tabular}
\end{table}

\subsection*{16.4}
\begin{table}[H]
\begin{tabular}{l}
\romline{dambho darpo'bhimānaśca} \\
\romline{krodhaḥ pāruṣyameva ca} \\
\romline{ajñānaṃ cābhijātasya} \\
\romline{pārtha sampadamāsurīm}
\end{tabular}
\end{table}

\subsection*{16.5}
\begin{table}[H]
\begin{tabular}{l}
\romline{daivī sampadvimokṣāya} \\
\romline{nibandhāyāsurī matā} \\
\romline{mā śucaḥ sampadaṃ daivīm} \\
\romline{abhijāto'si pāṇḍava}
\end{tabular}
\end{table}

\subsection*{16.6}
\begin{table}[H]
\begin{tabular}{l}
\romline{dvau bhūtasargau loke'smin} \\
\romline{daiva āsura eva ca} \\
\romline{daivo vistaraśaḥ proktaḥ} \\
\romline{āsuraṃ pārtha me śṛṇu}
\end{tabular}
\end{table}

\subsection*{16.7}
\begin{table}[H]
\begin{tabular}{l}
\romline{pravṛttiṃ ca nivṛttiṃ ca} \\
\romline{janā na vidurāsurāḥ} \\
\romline{na śaucaṃ nāpi cācāraḥ} \\
\romline{na satyaṃ teṣu vidyate}
\end{tabular}
\end{table}

\subsection*{16.8}
\begin{table}[H]
\begin{tabular}{l}
\romline{asatyamapratiṣṭhaṃ te} \\
\romline{jagadāhuranīśvaram} \\
\romline{aparasparasambhūtaṃ} \\
\romline{kimanyatkāmahaitukam}
\end{tabular}
\end{table}

\subsection*{16.9}
\begin{table}[H]
\begin{tabular}{l}
\romline{etāṃ dṛṣṭimavaṣṭabhya} \\
\romline{naṣṭātmāno'lpa-buddhayaḥ} \\
\romline{prabhavantyugra-karmāṇaḥ} \\
\romline{kṣayāya jagato'hitāḥ}
\end{tabular}
\end{table}

\subsection*{16.10}
\begin{table}[H]
\begin{tabular}{l}
\romline{kāma-māśritya duṣpūraṃ} \\
\romline{dambha-māna-madānvitāḥ} \\
\romline{mohād-gṛhītvā-sadgrāhān} \\
\romline{pravartante'śucivratāḥ}
\end{tabular}
\end{table}

\subsection*{16.11}
\begin{table}[H]
\begin{tabular}{l}
\romline{cintām-aparimeyāṃ ca} \\
\romline{pralayāntām-upāśritāḥ} \\
\romline{kāmopabhoga-paramāḥ} \\
\romline{etāvaditi niścitāḥ}
\end{tabular}
\end{table}

\subsection*{16.12}
\begin{table}[H]
\begin{tabular}{l}
\romline{āśā-pāśa-śatairbaddhāḥ} \\
\romline{kāma-krodha-parāyaṇāḥ} \\
\romline{īhante kāma-bhogārtham} \\
\romline{anyāyenārtha-sañcayān}
\end{tabular}
\end{table}

\subsection*{16.13}
\begin{table}[H]
\begin{tabular}{l}
\romline{idamadya mayā labdham} \\
\romline{imaṃ prāpsye manoratham} \\
\romline{idamastīdamapi me} \\
\romline{bhaviṣyati punardhanam}
\end{tabular}
\end{table}

\subsection*{16.14}
\begin{table}[H]
\begin{tabular}{l}
\romline{asau mayā hataḥ śatṛḥ} \\
\romline{haniṣye cāparānapi} \\
\romline{īśvaro'hamahaṃ bhogī} \\
\romline{siddho'haṃ balavānsukhī}
\end{tabular}
\end{table}

\subsection*{16.15}
\begin{table}[H]
\begin{tabular}{l}
\romline{āḍhyo'bhijanavānasmi} \\
\romline{ko'nyo'sti sadṛśo mayā} \\
\romline{yakṣye dāsyāmi modiṣye} \\
\romline{ityajñānavimohitāḥ}
\end{tabular}
\end{table}

\subsection*{16.16}
\begin{table}[H]
\begin{tabular}{l}
\romline{aneka-citta-vibhrāntāḥ} \\
\romline{moha-jāla-samāvṛtāḥ} \\
\romline{prasaktāḥ kāma-bhogeṣu} \\
\romline{patanti narake'śucau}
\end{tabular}
\end{table}

\subsection*{16.17}
\begin{table}[H]
\begin{tabular}{l}
\romline{ātmasambhāvitāḥ stabdhāḥ} \\
\romline{dhanamānamadānvitāḥ} \\
\romline{yajante nāmayajñaiste} \\
\romline{dambhenāvidhipūrvakam}
\end{tabular}
\end{table}

\subsection*{16.18}
\begin{table}[H]
\begin{tabular}{l}
\romline{ahaṅkāraṃ balaṃ darpaṃ} \\
\romline{kāmaṃ krodhaṃ ca saṃśritāḥ} \\
\romline{māmātmaparadeheṣu} \\
\romline{pradviṣanto'bhyasūyakāḥ}
\end{tabular}
\end{table}

\subsection*{16.19}
\begin{table}[H]
\begin{tabular}{l}
\romline{tānahaṃ dviṣataḥ krūrān} \\
\romline{saṃsāreṣu narādhamān} \\
\romline{kṣipā-myajasra-maśubhān} \\
\romline{āsurīṣveva yoniṣu}
\end{tabular}
\end{table}

\subsection*{16.20}
\begin{table}[H]
\begin{tabular}{l}
\romline{āsurīṃ yonimāpannāḥ} \\
\romline{mūḍhā janmani janmani} \\
\romline{māmaprāpyaiva kaunteya} \\
\romline{tato yāntyadhamāṃ gatim}
\end{tabular}
\end{table}

\subsection*{16.21}
\begin{table}[H]
\begin{tabular}{l}
\romline{trividhaṃ narakasyedaṃ} \\
\romline{dvāraṃ nāśanamātmanaḥ} \\
\romline{kāmaḥ krodhastathā lobhaḥ} \\
\romline{tasmādetattrayaṃ tyajet}
\end{tabular}
\end{table}

\subsection*{16.22}
\begin{table}[H]
\begin{tabular}{l}
\romline{etairvimuktaḥ kaunteya} \\
\romline{tamodvāraistribhirnaraḥ} \\
\romline{ācaratyātmanaḥ śreyaḥ} \\
\romline{tato yāti parāṃ gatim}
\end{tabular}
\end{table}

\subsection*{16.23}
\begin{table}[H]
\begin{tabular}{l}
\romline{yaḥ śāstra-vidhimutsṛjya} \\
\romline{vartate kāmakārataḥ} \\
\romline{na sa siddhimavāpnoti} \\
\romline{na sukhaṃ na parāṃ gatim}
\end{tabular}
\end{table}

\subsection*{16.24}
\begin{table}[H]
\begin{tabular}{l}
\romline{tasmācchāstraṃ pramāṇaṃ te} \\
\romline{kāryākārya-vyavasthitau} \\
\romline{jñātvā śāstra-vidhānoktaṃ} \\
\romline{karma kartum-ihārhasi}
\end{tabular}
\end{table}

