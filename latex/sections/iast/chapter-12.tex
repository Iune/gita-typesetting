\subsection*{12.0}
\begin{table}[H]
\begin{tabular}{l}
\romline{oṃ śrī paramātmane namaḥ} \\
\romline{atha dvādaśo'dhyāyaḥ} \\
\romline{bhakti-yogaḥ}
\end{tabular}
\end{table}

\subsection*{12.1}
\begin{table}[H]
\begin{tabular}{l}
\romline{arjuna uvāca} \\
\romline{evaṃ satatayuktā ye} \\
\romline{bhaktāstvāṃ paryupāsate} \\
\romline{ye cāpyakṣaram-avyaktaṃ} \\
\romline{teṣāṃ ke yogavittamāḥ}
\end{tabular}
\end{table}

\subsection*{12.2}
\begin{table}[H]
\begin{tabular}{l}
\romline{śrī bhagavān uvāca} \\
\romline{mayyāveśya mano ye māṃ} \\
\romline{nityayuktā upāsate} \\
\romline{śraddhayā parayopetāḥ} \\
\romline{te me yuktatamā matāḥ}
\end{tabular}
\end{table}

\subsection*{12.3}
\begin{table}[H]
\begin{tabular}{l}
\romline{ye tvakṣaram-anirdeśyam} \\
\romline{avyaktaṃ paryupāsate} \\
\romline{sarvatra-gam-acintyam ca} \\
\romline{kūṭastham-acalam dhruvaṃ}
\end{tabular}
\end{table}

\subsection*{12.4}
\begin{table}[H]
\begin{tabular}{l}
\romline{sanniyamyendriya-grāmaṃ} \\
\romline{sarvatra samabuddhayaḥ} \\
\romline{te prāpnuvanti māmeva} \\
\romline{sarvabhūtahite ratāḥ}
\end{tabular}
\end{table}

\subsection*{12.5}
\begin{table}[H]
\begin{tabular}{l}
\romline{kleśo'dhika-tarasteṣām} \\
\romline{avyaktāsakta-cetasām} \\
\romline{avyaktā hi gatirduḥkhaṃ} \\
\romline{deha-vadbhiravāpyate}
\end{tabular}
\end{table}

\subsection*{12.6}
\begin{table}[H]
\begin{tabular}{l}
\romline{ye tu sarvāṇi karmāṇi} \\
\romline{mayi sannyasya matparaḥ} \\
\romline{ananyenaiva yogena} \\
\romline{māṃ dhyāyanta upāsate}
\end{tabular}
\end{table}

\subsection*{12.7}
\begin{table}[H]
\begin{tabular}{l}
\romline{teṣāmahaṃ samuddhartā} \\
\romline{mṛtyu-saṃsāra-sāgarāt} \\
\romline{bhavāmi nacirāt-pārtha} \\
\romline{mayyāveśita-cetasām}
\end{tabular}
\end{table}

\subsection*{12.8}
\begin{table}[H]
\begin{tabular}{l}
\romline{mayyeva mana ādhatsva} \\
\romline{mayi buddhiṃ niveśaya} \\
\romline{nivasiṣyasi mayyeva} \\
\romline{ata ūrdhvaṃ na samśayaḥ}
\end{tabular}
\end{table}

\subsection*{12.9}
\begin{table}[H]
\begin{tabular}{l}
\romline{atha cittaṃ samādhātuṃ} \\
\romline{na śaknoṣi mayi sthiram} \\
\romline{abhyāsayogena tataḥ} \\
\romline{māmicchāptuṃ dhanañjaya}
\end{tabular}
\end{table}

\subsection*{12.10}
\begin{table}[H]
\begin{tabular}{l}
\romline{abhyāse'pyasamartho'si} \\
\romline{mat-karma-paramo bhava} \\
\romline{madarthamapi karmāṇi} \\
\romline{kurvan-siddhim-avāpsyasi}
\end{tabular}
\end{table}

