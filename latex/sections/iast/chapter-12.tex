\subsection*{12.0}
\begin{table}[H]
\begin{tabular}{l}
\romline{oṃ śrī paramātmane namaḥ} \\
\romline{atha dvādaśo'dhyāyaḥ} \\
\romline{bhakti-yogaḥ}
\end{tabular}
\end{table}

\subsection*{12.1}
\begin{table}[H]
\begin{tabular}{l}
\romline{arjuna uvāca} \\
\romline{evaṃ satatayuktā ye} \\
\romline{bhaktāstvāṃ paryupāsate} \\
\romline{ye cāpyakṣaram-avyaktaṃ} \\
\romline{teṣāṃ ke yogavittamāḥ}
\end{tabular}
\end{table}

\subsection*{12.2}
\begin{table}[H]
\begin{tabular}{l}
\romline{śrī bhagavān uvāca} \\
\romline{mayyāveśya mano ye māṃ} \\
\romline{nityayuktā upāsate} \\
\romline{śraddhayā parayopetāḥ} \\
\romline{te me yuktatamā matāḥ}
\end{tabular}
\end{table}

\subsection*{12.3}
\begin{table}[H]
\begin{tabular}{l}
\romline{ye tvakṣaram-anirdeśyam} \\
\romline{avyaktaṃ paryupāsate} \\
\romline{sarvatra-gam-acintyam ca} \\
\romline{kūṭastham-acalam dhruvaṃ}
\end{tabular}
\end{table}

\subsection*{12.4}
\begin{table}[H]
\begin{tabular}{l}
\romline{sanniyamyendriya-grāmaṃ} \\
\romline{sarvatra samabuddhayaḥ} \\
\romline{te prāpnuvanti māmeva} \\
\romline{sarvabhūtahite ratāḥ}
\end{tabular}
\end{table}

\subsection*{12.5}
\begin{table}[H]
\begin{tabular}{l}
\romline{kleśo'dhika-tarasteṣām} \\
\romline{avyaktāsakta-cetasām} \\
\romline{avyaktā hi gatirduḥkhaṃ} \\
\romline{dehavad-bhiravāpyate}
\end{tabular}
\end{table}

\subsection*{12.6}
\begin{table}[H]
\begin{tabular}{l}
\romline{ye tu sarvāṇi karmāṇi} \\
\romline{mayi sannyasya matparāḥ} \\
\romline{ananyenaiva yogena} \\
\romline{māṃ dhyāyanta upāsate}
\end{tabular}
\end{table}

\subsection*{12.7}
\begin{table}[H]
\begin{tabular}{l}
\romline{teṣāmahaṃ samuddhartā} \\
\romline{mṛtyu-saṃsāra-sāgarāt} \\
\romline{bhavāmi nacirāt-pārtha} \\
\romline{mayyāveśita-cetasām}
\end{tabular}
\end{table}

\subsection*{12.8}
\begin{table}[H]
\begin{tabular}{l}
\romline{mayyeva mana ādhatsva} \\
\romline{mayi buddhiṃ niveśaya} \\
\romline{nivasiṣyasi mayyeva} \\
\romline{ata ūrdhvaṃ na samśayaḥ}
\end{tabular}
\end{table}

\subsection*{12.9}
\begin{table}[H]
\begin{tabular}{l}
\romline{atha cittaṃ samādhātuṃ} \\
\romline{na śaknoṣi mayi sthiram} \\
\romline{abhyāsayogena tataḥ} \\
\romline{māmicchāptuṃ dhanañjaya}
\end{tabular}
\end{table}

\subsection*{12.10}
\begin{table}[H]
\begin{tabular}{l}
\romline{abhyāse'pyasamartho'si} \\
\romline{mat-karma-paramo bhava} \\
\romline{madarthamapi karmāṇi} \\
\romline{kurvan-siddhi-mavāpsyasi}
\end{tabular}
\end{table}

\subsection*{12.11}
\begin{table}[H]
\begin{tabular}{l}
\romline{athaitadapyaśakto'si} \\
\romline{kartuṃ mad-yogam-āśritaḥ} \\
\romline{sarva-karma-phala-tyāgaṃ} \\
\romline{tataḥ kuru yatātmavān}
\end{tabular}
\end{table}

\subsection*{12.12}
\begin{table}[H]
\begin{tabular}{l}
\romline{śreyo hi·jñānam-abhyāsāt} \\
\romline{jñānāddhyānaṃ viśiṣyate} \\
\romline{dhyānāt-karma-phala-tyāgaḥ} \\
\romline{tyāgācchāntiranantaram}
\end{tabular}
\end{table}

\subsection*{12.13}
\begin{table}[H]
\begin{tabular}{l}
\romline{adveṣṭā sarvabhūtānāṃ} \\
\romline{maitraḥ karuṇa eva ca} \\
\romline{nirmamo nirahankāraḥ} \\
\romline{sama-duḥkha-sukhaḥ·kṣamī}
\end{tabular}
\end{table}

\subsection*{12.14}
\begin{table}[H]
\begin{tabular}{l}
\romline{santuṣṭaḥ satataṃ yogī} \\
\romline{yatātmā dṛḍha-niścayaḥ} \\
\romline{mayyarpitamanobuddhiḥ} \\
\romline{yo madbhaktaḥ sa me priyaḥ}
\end{tabular}
\end{table}

\subsection*{12.15}
\begin{table}[H]
\begin{tabular}{l}
\romline{yasmānnodvijate lokaḥ} \\
\romline{lokānnodvijate ca yaḥ} \\
\romline{harṣāmarṣa-bhayodvegaiḥ} \\
\romline{mukto yaḥ sa ca me priyaḥ}
\end{tabular}
\end{table}

\subsection*{12.16}
\begin{table}[H]
\begin{tabular}{l}
\romline{anapekṣaḥ śucirdakṣaḥ} \\
\romline{udāsīno gatavyathaḥ} \\
\romline{sarvārambha-parityāgī} \\
\romline{yo madbhaktaḥ sa me priyaḥ}
\end{tabular}
\end{table}

\subsection*{12.17}
\begin{table}[H]
\begin{tabular}{l}
\romline{yo na hṛṣyati na·dveṣṭi} \\
\romline{na śocati na kāñkṣati} \\
\romline{śubhāśubha-parityāgī} \\
\romline{bhaktimānyaḥ sa me priyaḥ}
\end{tabular}
\end{table}

\subsection*{12.18}
\begin{table}[H]
\begin{tabular}{l}
\romline{samaḥ śatrau ca mitre ca} \\
\romline{tathā mānāpamānayoḥ} \\
\romline{śītoṣṇa-sukha-duḥkheṣu} \\
\romline{samaḥ sañga-vivarjitaḥ}
\end{tabular}
\end{table}

\subsection*{12.19}
\begin{table}[H]
\begin{tabular}{l}
\romline{tulya-nindā-stutirmaunī} \\
\romline{santuṣṭo yena kenacit} \\
\romline{aniketaḥ sthiramatiḥ} \\
\romline{bhaktimānme priyo naraḥ}
\end{tabular}
\end{table}

\subsection*{12.20}
\begin{table}[H]
\begin{tabular}{l}
\romline{ye tu dharmyāmṛtamidaṃ} \\
\romline{yathoktaṃ paryupāsate} \\
\romline{śraddhadhānā matparamāḥ} \\
\romline{bhaktāste'tīva me priyāḥ}
\end{tabular}
\end{table}

