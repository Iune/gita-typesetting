\begin{table}[H]
\begin{tabular}{cl}
 & \romline{śrī paramātmane namaḥ} \\
 & \romline{atha dvādaśo'dhyāyaḥ} \\
 & \romline{bhaktiyogaḥ}
\end{tabular}
\end{table}

\begin{table}[H]
\begin{tabular}{cl}
\textbf{12.1} & \romline{arjuna uvāca} \\
 & \romline{evaṃ satatayuktā ye} \\
 & \romline{bhaktāstvāṃ paryupāsate |} \\
 & \romline{ye cāpyakṣaramavyaktaṃ} \\
 & \romline{teṣāṃ ke yogavittamāḥ ||}
\end{tabular}
\end{table}

\begin{table}[H]
\begin{tabular}{cl}
\textbf{12.2} & \romline{śrī bhagavān uvāca} \\
 & \romline{mayyāveśya mano ye māṃ} \\
 & \romline{nityayuktā upāsate |} \\
 & \romline{śraddhayā parayopetāḥ} \\
 & \romline{te me yuktatamā matāḥ ||}
\end{tabular}
\end{table}

\begin{table}[H]
\begin{tabular}{cl}
\textbf{12.3} & \romline{ye tvakṣaramanirdeśyam} \\
 & \romline{avyaktaṃ paryupāsate |} \\
 & \romline{sarvatragamacintyam ca} \\
 & \romline{kūṭasthamacalam dhruvaṃ ||}
\end{tabular}
\end{table}

\begin{table}[H]
\begin{tabular}{cl}
\textbf{12.4} & \romline{sanniyamyendriyagrāmaṃ} \\
 & \romline{sarvatra samabuddhayaḥ |} \\
 & \romline{te prāpnuvanti māmeva} \\
 & \romline{sarvabhūtahite ratāḥ ||}
\end{tabular}
\end{table}

\begin{table}[H]
\begin{tabular}{cl}
\textbf{12.5} & \romline{kleśo'dhikatarasteṣām} \\
 & \romline{avyaktāsaktacetasām |} \\
 & \romline{avyaktā hi gatirduḥkhaṃ} \\
 & \romline{dehavadbhiravāpyate ||}
\end{tabular}
\end{table}

\begin{table}[H]
\begin{tabular}{cl}
\textbf{12.6} & \romline{ye tu sarvāṇi karmāṇi} \\
 & \romline{mayi sannyasya matparāḥ |} \\
 & \romline{ananyenaiva yogena} \\
 & \romline{māṃ dhyāyanta upāsate ||}
\end{tabular}
\end{table}

\begin{table}[H]
\begin{tabular}{cl}
\textbf{12.7} & \romline{teṣāmahaṃ samuddhartā} \\
 & \romline{mṛtyusaṃsārasāgarāt |} \\
 & \romline{bhavāmi nacirātpārtha} \\
 & \romline{mayyāveśitacetasām ||}
\end{tabular}
\end{table}

\begin{table}[H]
\begin{tabular}{cl}
\textbf{12.8} & \romline{mayyeva mana ādhatsva} \\
 & \romline{mayi buddhiṃ niveśaya |} \\
 & \romline{nivasiṣyasi mayyeva} \\
 & \romline{ata ūrdhvaṃ na samśayaḥ ||}
\end{tabular}
\end{table}

\begin{table}[H]
\begin{tabular}{cl}
\textbf{12.9} & \romline{atha cittaṃ samādhātuṃ} \\
 & \romline{na śaknoṣi mayi sthiram |} \\
 & \romline{abhyāsayogena tataḥ} \\
 & \romline{māmicchāptuṃ dhanañjaya ||}
\end{tabular}
\end{table}

\begin{table}[H]
\begin{tabular}{cl}
\textbf{12.10} & \romline{abhyāse'pyasamartho'si} \\
 & \romline{matkarmaparamo bhava |} \\
 & \romline{madarthamapi karmāṇi} \\
 & \romline{kurvansiddhimavāpsyasi ||}
\end{tabular}
\end{table}

\begin{table}[H]
\begin{tabular}{cl}
\textbf{12.11} & \romline{athaitadapyaśakto'si} \\
 & \romline{kartuṃ madyogamāśritaḥ |} \\
 & \romline{sarvakarmaphalatyāgaṃ} \\
 & \romline{tataḥ kuru yatātmavān ||}
\end{tabular}
\end{table}

\begin{table}[H]
\begin{tabular}{cl}
\textbf{12.12} & \romline{śreyo hi jñānamabhyāsāt} \\
 & \romline{jñānāddhyānaṃ viśiṣyate |} \\
 & \romline{dhyānātkarmaphalatyāgaḥ} \\
 & \romline{tyāgācchāntiranantaram ||}
\end{tabular}
\end{table}

\begin{table}[H]
\begin{tabular}{cl}
\textbf{12.13} & \romline{adveṣṭā sarvabhūtānāṃ} \\
 & \romline{maitraḥ karuṇa eva ca |} \\
 & \romline{nirmamo nirahankāraḥ} \\
 & \romline{samaduḥkhasukhaḥ kṣamī ||}
\end{tabular}
\end{table}

\begin{table}[H]
\begin{tabular}{cl}
\textbf{12.14} & \romline{santuṣṭaḥ satataṃ yogī} \\
 & \romline{yatātmā dṛḍhaniścayaḥ |} \\
 & \romline{mayyarpitamanobuddhiḥ} \\
 & \romline{yo madbhaktaḥ sa me priyaḥ ||}
\end{tabular}
\end{table}

\begin{table}[H]
\begin{tabular}{cl}
\textbf{12.15} & \romline{yasmānnodvijate lokaḥ} \\
 & \romline{lokānnodvijate ca yaḥ |} \\
 & \romline{harṣāmarṣabhayodvegaiḥ} \\
 & \romline{mukto yaḥ sa ca me priyaḥ ||}
\end{tabular}
\end{table}

\begin{table}[H]
\begin{tabular}{cl}
\textbf{12.16} & \romline{anapekṣaḥ śucirdakṣaḥ} \\
 & \romline{udāsīno gatavyathaḥ |} \\
 & \romline{sarvārambhaparityāgī} \\
 & \romline{yo madbhaktaḥ sa me priyaḥ ||}
\end{tabular}
\end{table}

\begin{table}[H]
\begin{tabular}{cl}
\textbf{12.17} & \romline{yo na hṛṣyati na dveṣṭi} \\
 & \romline{na śocati na kāñkṣati |} \\
 & \romline{śubhāśubhaparityāgī} \\
 & \romline{bhaktimānyaḥ sa me priyaḥ ||}
\end{tabular}
\end{table}

\begin{table}[H]
\begin{tabular}{cl}
\textbf{12.18} & \romline{samaḥ śatrau ca mitre ca} \\
 & \romline{tathā mānāpamānayoḥ |} \\
 & \romline{śītoṣṇasukhaduḥkheṣu} \\
 & \romline{samaḥ sañgavivarjitaḥ ||}
\end{tabular}
\end{table}

\begin{table}[H]
\begin{tabular}{cl}
\textbf{12.19} & \romline{tulyanindāstutirmaunī} \\
 & \romline{santuṣṭo yena kenacit |} \\
 & \romline{aniketaḥ sthiramatiḥ} \\
 & \romline{bhaktimānme priyo naraḥ ||}
\end{tabular}
\end{table}

\begin{table}[H]
\begin{tabular}{cl}
\textbf{12.20} & \romline{ye tu dharmyāmṛtamidaṃ} \\
 & \romline{yathoktaṃ paryupāsate |} \\
 & \romline{śraddadhānā matparamāḥ} \\
 & \romline{bhaktāste'tīva me priyāḥ ||}
\end{tabular}
\end{table}

\begin{table}[H]
\begin{tabular}{cl}
 & \romline{śrīmadbhagavadgītāsu upaniṣatsu} \\
 & \romline{brahmavidyāyāṃ yogaśāstre} \\
 & \romline{śrīkṛṣṇārjuna saṃvāde} \\
 & \romline{bhaktiyogo nāma} \\
 & \romline{dvādaśodhyāyaḥ}
\end{tabular}
\end{table}

