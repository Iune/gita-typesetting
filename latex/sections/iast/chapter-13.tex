\begin{table}[H]
\begin{tabular}{cl}
 & \romline{śrī paramātmane namaḥ} \\
 & \romline{atha trayodaśo'dhyāyaḥ} \\
 & \romline{kṣetrakṣetrajñavibhāgayogaḥ}
\end{tabular}
\end{table}

\begin{table}[H]
\begin{tabular}{cl}
\textbf{13.1} & \romline{arjuna uvāca} \\
 & \romline{prakṛtiṃ puruṣaṃ caiva} \\
 & \romline{kṣetraṃ kṣetrajñameva ca |} \\
 & \romline{etat veditumicchāmi} \\
 & \romline{jñānaṃ jñeyaṃ ca keśava ||}
\end{tabular}
\end{table}

\begin{table}[H]
\begin{tabular}{cl}
\textbf{13.2} & \romline{śrī bhagavānuvāca} \\
 & \romline{idaṃ śarīraṃ kaunteya} \\
 & \romline{kṣetramityabhidhīyate |} \\
 & \romline{etadyo vetti taṃ prāhuḥ} \\
 & \romline{kṣetrajña iti tadvidaḥ ||}
\end{tabular}
\end{table}

\begin{table}[H]
\begin{tabular}{cl}
\textbf{13.3} & \romline{kṣetrajñaṃ cāpi māṃ viddhi} \\
 & \romline{sarvakṣetreṣu bhārata |} \\
 & \romline{kṣetrakṣetrajñayorjñānaṃ} \\
 & \romline{yattajjñānaṃ mataṃ mama ||}
\end{tabular}
\end{table}

\begin{table}[H]
\begin{tabular}{cl}
\textbf{13.4} & \romline{tatkṣetraṃ yacca yādṛkca} \\
 & \romline{yadvikāri yataśca yat |} \\
 & \romline{sa ca yo yatprabhāvaśca} \\
 & \romline{tatsamāsena me śṛṇu ||}
\end{tabular}
\end{table}

\begin{table}[H]
\begin{tabular}{cl}
\textbf{13.5} & \romline{ṛṣibhirbahudhā gītaṃ} \\
 & \romline{chandobhirvividhaiḥ pṛthak |} \\
 & \romline{brahmasūtrapadaiścaiva} \\
 & \romline{hetumadbhirviniścitaiḥ ||}
\end{tabular}
\end{table}

\begin{table}[H]
\begin{tabular}{cl}
\textbf{13.6} & \romline{mahābhūtānyahaṅkāraḥ} \\
 & \romline{buddhiravyaktameva ca |} \\
 & \romline{indriyāṇi daśaikaṃ ca} \\
 & \romline{pañca cendriyagocarāḥ ||}
\end{tabular}
\end{table}

\begin{table}[H]
\begin{tabular}{cl}
\textbf{13.7} & \romline{icchā dveṣaḥ sukhaṃ duḥkhaṃ} \\
 & \romline{saṅghātaścetanā dhṛtiḥ |} \\
 & \romline{etatkṣetraṃ samāsena} \\
 & \romline{savikāramudāhṛtam ||}
\end{tabular}
\end{table}

\begin{table}[H]
\begin{tabular}{cl}
\textbf{13.8} & \romline{amānitvamadaṃbhitvam} \\
 & \romline{ahiṃsā kṣāntirārjavam |} \\
 & \romline{ācāryopāsanaṃ śaucaṃ} \\
 & \romline{sthairyamātmavinigrahaḥ ||}
\end{tabular}
\end{table}

\begin{table}[H]
\begin{tabular}{cl}
\textbf{13.9} & \romline{indriyārtheṣu vairāgyam} \\
 & \romline{anahaṅkāra eva ca |} \\
 & \romline{janmamṛtyujarāvyādhi} \\
 & \romline{duḥkhadoṣānudarśanam ||}
\end{tabular}
\end{table}

\begin{table}[H]
\begin{tabular}{cl}
\textbf{13.10} & \romline{asaktiranabhiṣvaṅgaḥ} \\
 & \romline{putradāragṛhādiṣu |} \\
 & \romline{nityaṃ ca samacittatvam} \\
 & \romline{iṣṭāniṣṭopapattiṣu ||}
\end{tabular}
\end{table}

\begin{table}[H]
\begin{tabular}{cl}
\textbf{13.11} & \romline{mayi cānanyayogena} \\
 & \romline{bhaktiravyabhicāriṇī |} \\
 & \romline{viviktadeśasevitvam} \\
 & \romline{aratirjanasaṃsadi ||}
\end{tabular}
\end{table}

\begin{table}[H]
\begin{tabular}{cl}
\textbf{13.12} & \romline{adhyātmajñānanityatvaṃ} \\
 & \romline{tattvajñānārthadarśanam |} \\
 & \romline{etajjñānamiti proktam} \\
 & \romline{ajñānaṃ yadato'nyathā ||}
\end{tabular}
\end{table}

\begin{table}[H]
\begin{tabular}{cl}
\textbf{13.13} & \romline{jñeyaṃ yattatpravakṣyāmi} \\
 & \romline{yajjñātvā'mṛtamaśnute |} \\
 & \romline{anādimatparaṃ brahma} \\
 & \romline{na sattannāsaducyate ||}
\end{tabular}
\end{table}

\begin{table}[H]
\begin{tabular}{cl}
\textbf{13.14} & \romline{sarvataḥ pāṇipādaṃ tat} \\
 & \romline{sarvato'kṣiśiromukham |} \\
 & \romline{sarvataḥ śrutimalloke} \\
 & \romline{sarvamāvṛtya tiṣṭhati ||}
\end{tabular}
\end{table}

\begin{table}[H]
\begin{tabular}{cl}
\textbf{13.15} & \romline{sarvendriyaguṇābhāsaṃ} \\
 & \romline{sarvendriyavivarjitam |} \\
 & \romline{asaktaṃ sarvabhṛccaiva} \\
 & \romline{nirguṇaṃ guṇabhoktṛ ca ||}
\end{tabular}
\end{table}

\begin{table}[H]
\begin{tabular}{cl}
\textbf{13.16} & \romline{bahirantaśca bhūtānām} \\
 & \romline{acaraṃ carameva ca |} \\
 & \romline{sūkṣmatvāttadavijñeyaṃ} \\
 & \romline{dūrasthaṃ cāntike ca tat ||}
\end{tabular}
\end{table}

\begin{table}[H]
\begin{tabular}{cl}
\textbf{13.17} & \romline{avibhaktaṃ ca bhūteṣu} \\
 & \romline{vibhaktamiva ca sthitam |} \\
 & \romline{bhūtabhartṛ ca tajjñeyaṃ} \\
 & \romline{grasiṣṇu prabhaviṣṇu ca ||}
\end{tabular}
\end{table}

\begin{table}[H]
\begin{tabular}{cl}
\textbf{13.18} & \romline{jyotiṣāmapi tajjyotiḥ} \\
 & \romline{tamasaḥ paramucyate |} \\
 & \romline{jñānaṃ jñeyaṃ jñānagamyaṃ} \\
 & \romline{hṛdi sarvasya viṣṭhitam ||}
\end{tabular}
\end{table}

\begin{table}[H]
\begin{tabular}{cl}
\textbf{13.19} & \romline{iti kṣetraṃ tathā jñānaṃ} \\
 & \romline{jñeyaṃ coktaṃ samāsataḥ |} \\
 & \romline{madbhakta etadvijñāya} \\
 & \romline{madbhāvāyopapadyate ||}
\end{tabular}
\end{table}

\begin{table}[H]
\begin{tabular}{cl}
\textbf{13.20} & \romline{prakṛtiṃ puruṣaṃ caiva} \\
 & \romline{viddhyanādī ubhāvapi |} \\
 & \romline{vikārāṃśca guṇāṃścaiva} \\
 & \romline{viddhi prakṛtisambhavān ||}
\end{tabular}
\end{table}

\begin{table}[H]
\begin{tabular}{cl}
\textbf{13.21} & \romline{kāryakaraṇakartṛtve} \\
 & \romline{hetuḥ prakṛtirucyate |} \\
 & \romline{puruṣaḥ sukhaduḥkhānāṃ} \\
 & \romline{bhoktṛtve heturucyate ||}
\end{tabular}
\end{table}

\begin{table}[H]
\begin{tabular}{cl}
\textbf{13.22} & \romline{puruṣaḥ prakṛtistho hi} \\
 & \romline{bhuṅkte prakṛtijānguṇān |} \\
 & \romline{kāraṇaṃ guṇasaṅgo'sya} \\
 & \romline{sadasadyonijanmasu ||}
\end{tabular}
\end{table}

\begin{table}[H]
\begin{tabular}{cl}
\textbf{13.23} & \romline{upadraṣṭā'numantā ca} \\
 & \romline{bhartā bhoktā maheśvaraḥ |} \\
 & \romline{paramātmeti cāpyuktaḥ} \\
 & \romline{dehe'sminpuruṣaḥ paraḥ ||}
\end{tabular}
\end{table}

\begin{table}[H]
\begin{tabular}{cl}
\textbf{13.24} & \romline{ya evaṃ vetti puruṣaṃ} \\
 & \romline{prakṛtiṃ ca guṇaiḥ saha |} \\
 & \romline{sarvathā vartamāno'pi} \\
 & \romline{na sa bhūyo'bhijāyate ||}
\end{tabular}
\end{table}

\begin{table}[H]
\begin{tabular}{cl}
\textbf{13.25} & \romline{dhyānenātmani paśyanti} \\
 & \romline{kecidātmānamātmanā |} \\
 & \romline{anye sāṅkhyena yogena} \\
 & \romline{karmayogena cāpare ||}
\end{tabular}
\end{table}

\begin{table}[H]
\begin{tabular}{cl}
\textbf{13.26} & \romline{anye tvevamajānantaḥ} \\
 & \romline{śrutvā'nyebhya upāsate |} \\
 & \romline{te'pi cātitarantyeva} \\
 & \romline{mṛtyuṃ śrutiparāyaṇāḥ ||}
\end{tabular}
\end{table}

\begin{table}[H]
\begin{tabular}{cl}
\textbf{13.27} & \romline{yāvatsañjāyate kiñcit} \\
 & \romline{sattvaṃ sthāvarajaṅgamam |} \\
 & \romline{kṣetrakṣetrajñasaṃyogāt} \\
 & \romline{tadviddhi bharatarṣabha ||}
\end{tabular}
\end{table}

\begin{table}[H]
\begin{tabular}{cl}
\textbf{13.28} & \romline{samaṃ sarveṣu bhūteṣu} \\
 & \romline{tiṣṭhantaṃ parameśvaram |} \\
 & \romline{vinaśyatsvavinaśyantaṃ} \\
 & \romline{yaḥ paśyati sa paśyati ||}
\end{tabular}
\end{table}

\begin{table}[H]
\begin{tabular}{cl}
\textbf{13.29} & \romline{samaṃ paśyanhi sarvatra} \\
 & \romline{samavasthitamīśvaram |} \\
 & \romline{na hinastyātmanā''tmānaṃ} \\
 & \romline{tato yāti parāṃ gatim ||}
\end{tabular}
\end{table}

\begin{table}[H]
\begin{tabular}{cl}
\textbf{13.30} & \romline{prakṛtyaiva ca karmāṇi} \\
 & \romline{kriyamāṇāni sarvaśaḥ |} \\
 & \romline{yaḥ paśyati tathā''tmānam} \\
 & \romline{akartāraṃ sa paśyati ||}
\end{tabular}
\end{table}

\begin{table}[H]
\begin{tabular}{cl}
\textbf{13.31} & \romline{yadā bhūtapṛthagbhāvam} \\
 & \romline{ekasthamanupaśyati |} \\
 & \romline{tata eva ca vistāraṃ} \\
 & \romline{brahma sampadyate tadā ||}
\end{tabular}
\end{table}

\begin{table}[H]
\begin{tabular}{cl}
\textbf{13.32} & \romline{anāditvānnirguṇatvāt} \\
 & \romline{paramātmāyamavyayaḥ |} \\
 & \romline{śarīrastho'pi kaunteya} \\
 & \romline{na karoti na lipyate ||}
\end{tabular}
\end{table}

\begin{table}[H]
\begin{tabular}{cl}
\textbf{13.33} & \romline{yathā sarvagataṃ saukṣmyāt} \\
 & \romline{ākāśaṃ nopalipyate |} \\
 & \romline{sarvatrāvasthito dehe} \\
 & \romline{tathā''tmā nopalipyate ||}
\end{tabular}
\end{table}

\begin{table}[H]
\begin{tabular}{cl}
\textbf{13.34} & \romline{yathā prakāśayatyekaḥ} \\
 & \romline{kṛtsnaṃ lokamimaṃ raviḥ |} \\
 & \romline{kṣetraṃ kṣetrī tathā kṛtsnaṃ} \\
 & \romline{prakāśayati bhārata ||}
\end{tabular}
\end{table}

\begin{table}[H]
\begin{tabular}{cl}
\textbf{13.35} & \romline{kṣetrakṣetrajñayorevam} \\
 & \romline{antaraṃ jñānacakṣuṣā |} \\
 & \romline{bhūtaprakṛtimokṣaṃ ca} \\
 & \romline{ye viduryānti te param ||}
\end{tabular}
\end{table}

\begin{table}[H]
\begin{tabular}{cl}
 & \romline{śrīmadbhagavadgītāsu upaniṣatsu} \\
 & \romline{brahmavidyāyāṃ yogaśāstre} \\
 & \romline{śrīkṛṣṇārjuna saṃvāde} \\
 & \romline{kṣetrakṣetrajñavibhāgayogo nāma} \\
 & \romline{trayodaśodhyāyaḥ}
\end{tabular}
\end{table}

