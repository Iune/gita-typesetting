\begin{table}[H]
\begin{tabular}{cl}
\textbf{13.0} & \romline{oṃ śrī paramātmane namaḥ} \\
 & \romline{atha trayodaśo'dhyāyaḥ} \\
 & \romline{kṣetra-kṣetrajñavibhāga-yogaḥ}
\end{tabular}
\end{table}

\begin{table}[H]
\begin{tabular}{cl}
\textbf{13.1} & \romline{arjuna uvāca} \\
 & \romline{prakṛtiṃ puruṣaṃ caiva} \\
 & \romline{kṣetraṃ kṣetrajñameva ca |} \\
 & \romline{etat veditu-micchāmi} \\
 & \romline{jñānaṃ jñeyaṃ ca keśava ||}
\end{tabular}
\end{table}

\begin{table}[H]
\begin{tabular}{cl}
\textbf{13.2} & \romline{śrī bhagavānuvāca} \\
 & \romline{idaṃ śarīraṃ kaunteya} \\
 & \romline{kṣetra-mitya-bhidhīyate |} \\
 & \romline{etadyo vetti taṃ prāhuḥ} \\
 & \romline{kṣetrajña iti tadvidaḥ ||}
\end{tabular}
\end{table}

\begin{table}[H]
\begin{tabular}{cl}
\textbf{13.3} & \romline{kṣetrajñaṃ cāpi māṃ viddhi} \\
 & \romline{sarvakṣetreṣu bhārata |} \\
 & \romline{kṣetra-kṣetra-jñayor-jñānaṃ} \\
 & \romline{yattaj-jñānaṃ mataṃ mama ||}
\end{tabular}
\end{table}

\begin{table}[H]
\begin{tabular}{cl}
\textbf{13.4} & \romline{tatkṣetraṃ yacca yādṛkca} \\
 & \romline{yadvikāri yataśca yat |} \\
 & \romline{sa ca yo yatprabhāvaśca} \\
 & \romline{tatsamāsena me śṛṇu ||}
\end{tabular}
\end{table}

\begin{table}[H]
\begin{tabular}{cl}
\textbf{13.5} & \romline{ṛṣibhir-bahudhā gītaṃ} \\
 & \romline{chandobhir-vividhaiḥ pṛthak |} \\
 & \romline{brahmasūtra-padaiścaiva} \\
 & \romline{hetumadbhir-viniścitaiḥ ||}
\end{tabular}
\end{table}

