\begin{table}[H]
\begin{tabular}{cl}
\textbf{13.0} & \romline{oṃ śrī paramātmane namaḥ} \\
 & \romline{atha trayodaśo'dhyāyaḥ} \\
 & \romline{kṣetra-kṣetrajñavibhāga-yogaḥ}
\end{tabular}
\end{table}

\begin{table}[H]
\begin{tabular}{cl}
\textbf{13.1} & \romline{arjuna uvāca} \\
 & \romline{prakṛtiṃ puruṣaṃ caiva} \\
 & \romline{kṣetraṃ kṣetrajñameva ca |} \\
 & \romline{etat veditu-micchāmi} \\
 & \romline{jñānaṃ jñeyaṃ ca keśava ||}
\end{tabular}
\end{table}

\begin{table}[H]
\begin{tabular}{cl}
\textbf{13.2} & \romline{śrī bhagavānuvāca} \\
 & \romline{idaṃ śarīraṃ kaunteya} \\
 & \romline{kṣetramitya-bhidhīyate |} \\
 & \romline{etadyo vetti taṃ prāhuḥ} \\
 & \romline{kṣetrajña iti tadvidaḥ ||}
\end{tabular}
\end{table}

\begin{table}[H]
\begin{tabular}{cl}
\textbf{13.3} & \romline{kṣetrajñaṃ cāpi māṃ viddhi} \\
 & \romline{sarvakṣetreṣu bhārata |} \\
 & \romline{kṣetra-kṣetra-jñayor-jñānaṃ} \\
 & \romline{yattaj-jñānaṃ mataṃ mama ||}
\end{tabular}
\end{table}

\begin{table}[H]
\begin{tabular}{cl}
\textbf{13.4} & \romline{tatkṣetraṃ yacca yādṛkca} \\
 & \romline{yadvikāri yataśca yat |} \\
 & \romline{sa ca yo yatprabhāvaśca} \\
 & \romline{tatsamāsena me śṛṇu ||}
\end{tabular}
\end{table}

\begin{table}[H]
\begin{tabular}{cl}
\textbf{13.5} & \romline{ṛṣibhir-bahudhā gītaṃ} \\
 & \romline{chandobhir-vividhaiḥ pṛthak |} \\
 & \romline{brahmasūtra-padaiścaiva} \\
 & \romline{hetumadbhir-viniścitaiḥ ||}
\end{tabular}
\end{table}

\begin{table}[H]
\begin{tabular}{cl}
\textbf{13.6} & \romline{mahā-bhūtānya-haṅkāraḥ} \\
 & \romline{buddhir-avyaktameva ca |} \\
 & \romline{indriyāṇi daśaikaṃ ca} \\
 & \romline{pañca cendriya-gocarāḥ ||}
\end{tabular}
\end{table}

\begin{table}[H]
\begin{tabular}{cl}
\textbf{13.7} & \romline{icchā dveṣaḥ sukhaṃ duḥkhaṃ} \\
 & \romline{saṅghātaś-cetanā dhṛtiḥ |} \\
 & \romline{etatkṣetraṃ samāsena} \\
 & \romline{savikāra-mudāhṛtam ||}
\end{tabular}
\end{table}

\begin{table}[H]
\begin{tabular}{cl}
\textbf{13.8} & \romline{amānitva-madaṃbhitvam} \\
 & \romline{ahiṃsā kṣānti-rārjavam |} \\
 & \romline{ācāryo-pāsanaṃ śaucaṃ} \\
 & \romline{sthair-yamātma-vinigrahaḥ ||}
\end{tabular}
\end{table}

\begin{table}[H]
\begin{tabular}{cl}
\textbf{13.9} & \romline{indriyārtheṣu vairāgyam} \\
 & \romline{anahaṅkāra eva ca |} \\
 & \romline{janma-mṛtyu-jarā-vyādhi-} \\
 & \romline{duḥkha-doṣānu-darśanam ||}
\end{tabular}
\end{table}

\begin{table}[H]
\begin{tabular}{cl}
\textbf{13.10} & \romline{asakti-ranabhiṣvaṅgaḥ} \\
 & \romline{putradāra-gṛhādiṣu |} \\
 & \romline{nityaṃ ca sama-cittatvam} \\
 & \romline{iṣṭāniṣṭo-papattiṣu ||}
\end{tabular}
\end{table}

\begin{table}[H]
\begin{tabular}{cl}
\textbf{13.11} & \romline{mayi cānanya-yogena} \\
 & \romline{bhaktiravya-bhicāriṇī |} \\
 & \romline{vivikta-deśa-sevitvam} \\
 & \romline{aratir-janasaṃsadi ||}
\end{tabular}
\end{table}

\begin{table}[H]
\begin{tabular}{cl}
\textbf{13.12} & \romline{adhyātma-jñāna-nityatvaṃ} \\
 & \romline{tattva-jñānārtha-darśanam |} \\
 & \romline{etaj-jñānamiti·proktam} \\
 & \romline{ajñānaṃ yadato'nyathā ||}
\end{tabular}
\end{table}

\begin{table}[H]
\begin{tabular}{cl}
\textbf{13.13} & \romline{jñeyaṃ yattat-pravakṣyāmi} \\
 & \romline{yaj-jñātvā'mṛta-maśnute |} \\
 & \romline{anādi-matparaṃ brahma} \\
 & \romline{na sattannā-saducyate ||}
\end{tabular}
\end{table}

\begin{table}[H]
\begin{tabular}{cl}
\textbf{13.14} & \romline{sarvataḥ pāṇi-pādaṃ tat} \\
 & \romline{sarvato'kṣiśiro-mukham |} \\
 & \romline{sarvataḥ śruti-malloke} \\
 & \romline{sarvamāvṛtya tiṣṭhati ||}
\end{tabular}
\end{table}

\begin{table}[H]
\begin{tabular}{cl}
\textbf{13.15} & \romline{sarvendriya-guṇābhāsaṃ} \\
 & \romline{sarvendriya-vivarjitam |} \\
 & \romline{asaktaṃ sarva-bhṛccaiva} \\
 & \romline{nirguṇaṃ guṇa-bhoktṛ ca ||}
\end{tabular}
\end{table}

\begin{table}[H]
\begin{tabular}{cl}
\textbf{13.16} & \romline{bahirantaśca bhūtānām} \\
 & \romline{acaraṃ carameva ca |} \\
 & \romline{sūkṣmatvāttada-vijñeyaṃ} \\
 & \romline{dūrasthaṃ cāntike ca tat ||}
\end{tabular}
\end{table}

\begin{table}[H]
\begin{tabular}{cl}
\textbf{13.17} & \romline{avibhaktaṃ ca bhūteṣu} \\
 & \romline{vibhakta-miva ca sthitam |} \\
 & \romline{bhūta-bhartṛ ca taj-jñeyaṃ} \\
 & \romline{grasiṣṇu·prabha-viṣṇu ca ||}
\end{tabular}
\end{table}

\begin{table}[H]
\begin{tabular}{cl}
\textbf{13.18} & \romline{jyotiṣā-mapi taj-jyotiḥ} \\
 & \romline{tamasaḥ paramucyate |} \\
 & \romline{jñānaṃ jñeyaṃ jñāna-gamyaṃ} \\
 & \romline{hṛdi sarvasya viṣṭhitam ||}
\end{tabular}
\end{table}

\begin{table}[H]
\begin{tabular}{cl}
\textbf{13.19} & \romline{iti·kṣetraṃ tathā jñānaṃ} \\
 & \romline{jñeyaṃ coktaṃ samāsataḥ |} \\
 & \romline{madbhakta etadvijñāya} \\
 & \romline{madbhāvāyo-papadyate ||}
\end{tabular}
\end{table}

\begin{table}[H]
\begin{tabular}{cl}
\textbf{13.20} & \romline{prakṛtiṃ puruṣaṃ caiva} \\
 & \romline{viddhyanādī ubhāvapi |} \\
 & \romline{vikārāṃśca guṇāṃścaiva} \\
 & \romline{viddhi·prakṛti-sambhavān ||}
\end{tabular}
\end{table}

\begin{table}[H]
\begin{tabular}{cl}
\textbf{13.21} & \romline{kāryakaraṇa-kartṛtve} \\
 & \romline{hetuḥ prakṛti-rucyate |} \\
 & \romline{puruṣaḥ sukha-duḥkhānāṃ} \\
 & \romline{bhok-tṛtve heturucyate ||}
\end{tabular}
\end{table}

\begin{table}[H]
\begin{tabular}{cl}
\textbf{13.22} & \romline{puruṣaḥ prakṛtistho hi} \\
 & \romline{bhuṅkte prakṛti-jānguṇān |} \\
 & \romline{kāraṇaṃ guṇasaṅgo'sya} \\
 & \romline{sadasadyo-nijanmasu ||}
\end{tabular}
\end{table}

\begin{table}[H]
\begin{tabular}{cl}
\textbf{13.23} & \romline{upadraṣṭā'numantā ca} \\
 & \romline{bhartā bhoktā maheśvaraḥ |} \\
 & \romline{paramātmeti cāpyuktaḥ} \\
 & \romline{dehe'smin-puruṣaḥ paraḥ ||}
\end{tabular}
\end{table}

\begin{table}[H]
\begin{tabular}{cl}
\textbf{13.24} & \romline{ya evaṃ vetti puruṣaṃ} \\
 & \romline{prakṛtiṃ ca guṇaiḥ saha |} \\
 & \romline{sarvathā vartamāno'pi} \\
 & \romline{na sa bhūyo'bhijāyate ||}
\end{tabular}
\end{table}

\begin{table}[H]
\begin{tabular}{cl}
\textbf{13.25} & \romline{dhyānenātmani paśyanti} \\
 & \romline{kecidātmānamātmanā |} \\
 & \romline{anye sāṅkhyena yogena} \\
 & \romline{karmayogena cāpare ||}
\end{tabular}
\end{table}

\begin{table}[H]
\begin{tabular}{cl}
\textbf{13.26} & \romline{anye tvevama-jānantaḥ} \\
 & \romline{śrutvā'nyebhya upāsate |} \\
 & \romline{te'pi cāti-tarantyeva} \\
 & \romline{mṛtyuṃ śruti-parāyaṇāḥ ||}
\end{tabular}
\end{table}

\begin{table}[H]
\begin{tabular}{cl}
\textbf{13.27} & \romline{yāvat-sañjāyate kiñcit} \\
 & \romline{sattvaṃ sthāvara-jaṅgamam |} \\
 & \romline{kṣetra-kṣetrajña-saṃyogāt} \\
 & \romline{tadviddhi bhara-tarṣabha ||}
\end{tabular}
\end{table}

\begin{table}[H]
\begin{tabular}{cl}
\textbf{13.28} & \romline{samaṃ sarveṣu bhūteṣu} \\
 & \romline{tiṣṭhantaṃ parameśvaram |} \\
 & \romline{vinaśyat-svavinaśyantaṃ} \\
 & \romline{yaḥ paśyati sa paśyati ||}
\end{tabular}
\end{table}

\begin{table}[H]
\begin{tabular}{cl}
\textbf{13.29} & \romline{samaṃ paśyanhi sarvatra} \\
 & \romline{samavasthita-mīśvaram |} \\
 & \romline{na hi-nastyāt-manā''tmānaṃ} \\
 & \romline{tato yāti parāṃ gatim ||}
\end{tabular}
\end{table}

\begin{table}[H]
\begin{tabular}{cl}
\textbf{13.30} & \romline{prakṛtyaiva ca karmāṇi} \\
 & \romline{kriyamāṇāni sarvaśaḥ |} \\
 & \romline{yaḥ paśyati tathā''tmānam} \\
 & \romline{akartāraṃ sa paśyati ||}
\end{tabular}
\end{table}

\begin{table}[H]
\begin{tabular}{cl}
\textbf{13.31} & \romline{yadā bhūta-pṛthag-bhāvam} \\
 & \romline{ekastha-manu-paśyati |} \\
 & \romline{tata eva ca vistāraṃ} \\
 & \romline{brahma sampadyate tadā ||}
\end{tabular}
\end{table}

\begin{table}[H]
\begin{tabular}{cl}
\textbf{13.32} & \romline{anāditvānnir-guṇatvāt} \\
 & \romline{paramātmāya-mavyayaḥ |} \\
 & \romline{śarīrastho'pi kaunteya} \\
 & \romline{na karoti na lipyate ||}
\end{tabular}
\end{table}

\begin{table}[H]
\begin{tabular}{cl}
\textbf{13.33} & \romline{yathā sarvagataṃ saukṣmyāt} \\
 & \romline{ākāśaṃ nopalipyate |} \\
 & \romline{sarvatrā-vasthito dehe} \\
 & \romline{tathā''tmā no-palipyate ||}
\end{tabular}
\end{table}

\begin{table}[H]
\begin{tabular}{cl}
\textbf{13.34} & \romline{yathā prakāśa-yatyekaḥ} \\
 & \romline{kṛtsnaṃ lokamimaṃ raviḥ |} \\
 & \romline{kṣetraṃ kṣetrī tathā kṛtsnaṃ} \\
 & \romline{prakāśayati bhārata ||}
\end{tabular}
\end{table}

\begin{table}[H]
\begin{tabular}{cl}
\textbf{13.35} & \romline{kṣetra-kṣetrajña-yorevam} \\
 & \romline{antaraṃ jñāna-cakṣuṣā |} \\
 & \romline{bhūta-prakṛti-mokṣaṃ ca} \\
 & \romline{ye viduryānti te param ||}
\end{tabular}
\end{table}

