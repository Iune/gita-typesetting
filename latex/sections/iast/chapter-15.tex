\subsection*{15.0}
\begin{table}[H]
\begin{tabular}{l}
\romline{oṃ śrī paramātmane namaḥ} \\
\romline{atha pañcadaśo'dhyāyaḥ} \\
\romline{puruṣottama-prapti-yogaḥ}
\end{tabular}
\end{table}

\subsection*{15.1}
\begin{table}[H]
\begin{tabular}{l}
\romline{śrī-bhagavān uvāca} \\
\romline{ūrdhva-mūlamadhaḥ* śākham} \\
\romline{aśvatthaṃ prāhuravyayam} \\
\romline{chandāṃsi yasya parṇāni} \\
\romline{yastaṃ veda sa vedavit}
\end{tabular}
\end{table}

\subsection*{15.2}
\begin{table}[H]
\begin{tabular}{l}
\romline{adhaścordhvaṃ prasṛtāstasya śākhāḥ} \\
\romline{guṇa-pravṛddhā viṣaya-pravālāḥ} \\
\romline{adhaśca mūlān-yanusantatāni} \\
\romline{karmānubandhīni manuṣyaloke}
\end{tabular}
\end{table}

\subsection*{15.3}
\begin{table}[H]
\begin{tabular}{l}
\romline{na rūpamasyeha tathopalabhyate} \\
\romline{nānto na cādirna ca saṃpratiṣṭhā} \\
\romline{aśvatthamenaṃ suvirūḍhamūlam} \\
\romline{asaṅgaśastreṇa dṛḍhena chittvā}
\end{tabular}
\end{table}

\subsection*{15.4}
\begin{table}[H]
\begin{tabular}{l}
\romline{tataḥ padaṃ tatparimārgitavyaṃ} \\
\romline{yasmingatā na nivartanti bhūyaḥ} \\
\romline{tameva cādyaṃ puruṣaṃ prapadye} \\
\romline{yataḥ pravṛttiḥ prasṛtā purāṇī}
\end{tabular}
\end{table}

\subsection*{15.5}
\begin{table}[H]
\begin{tabular}{l}
\romline{nirmānamohā jitasaṅgadoṣāḥ} \\
\romline{adhyātmanityā vinivṛttakāmāḥ} \\
\romline{dvandvairvimuktāḥ sukhaduḥkha sañjñaiḥ} \\
\romline{gacchantyamūḍhāḥ padam-avyayaṃ tat}
\end{tabular}
\end{table}

\subsection*{15.6}
\begin{table}[H]
\begin{tabular}{l}
\romline{na tadbhāsayate sūryaḥ} \\
\romline{na śaśāṅko na pāvakaḥ} \\
\romline{yadgatvā na nivartante} \\
\romline{taddhāma paramaṃ mama}
\end{tabular}
\end{table}

\subsection*{15.7}
\begin{table}[H]
\begin{tabular}{l}
\romline{mamaivāṃśo jīvaloke} \\
\romline{jīvabhūtaḥ sanātanaḥ} \\
\romline{manaḥ ṣaṣṭhānīndriyāṇi} \\
\romline{prakṛtisthāni karṣati}
\end{tabular}
\end{table}

\subsection*{15.8}
\begin{table}[H]
\begin{tabular}{l}
\romline{śarīraṃ yadavāpnoti} \\
\romline{yaccāpyut-krāmatīśvaraḥ} \\
\romline{gṛhītvaitāni saṃyāti} \\
\romline{vāyurgandhā-nivāśayāt}
\end{tabular}
\end{table}

\subsection*{15.9}
\begin{table}[H]
\begin{tabular}{l}
\romline{śrotraṃ cakṣuḥ sparśanaṃ ca} \\
\romline{rasanaṃ ghrāṇameva ca} \\
\romline{adhiṣṭhāya manaścāyaṃ} \\
\romline{viṣayānupasevate}
\end{tabular}
\end{table}

\subsection*{15.10}
\begin{table}[H]
\begin{tabular}{l}
\romline{utkrāmantaṃ sthitaṃ vā'pi} \\
\romline{bhuñjānaṃ vā guṇānvitam} \\
\romline{vimūḍhā nānupaśyanti} \\
\romline{paśyanti·jñānacakṣuṣaḥ}
\end{tabular}
\end{table}

