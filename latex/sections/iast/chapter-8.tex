\begin{table}[H]
\begin{tabular}{cl}
\textbf{8.0} & \romline{oṃ śrī paramātmane namaḥ} \\
 & \romline{atha aṣṭamo'dhyāyaḥ} \\
 & \romline{akṣara-parabrahma-yogaḥ}
\end{tabular}
\end{table}

\begin{table}[H]
\begin{tabular}{cl}
\textbf{8.1} & \romline{arjuna uvāca} \\
 & \romline{kiṃ tadbrahma kimadhyātmaṃ} \\
 & \romline{kiṃ karma puruṣottama |} \\
 & \romline{adhi-bhūtaṃ ca kiṃ proktam} \\
 & \romline{adhi-daivaṃ kimucyate ||}
\end{tabular}
\end{table}

\begin{table}[H]
\begin{tabular}{cl}
\textbf{8.2} & \romline{adhi-yajñaḥ kathaṃ ko'tra} \\
 & \romline{dehe'smin-madhusūdana |} \\
 & \romline{prayāṇa-kāle ca kathaṃ} \\
 & \romline{jñeyo'si niyatātma-bhiḥ ||}
\end{tabular}
\end{table}

\begin{table}[H]
\begin{tabular}{cl}
\textbf{8.3} & \romline{śrī bhagavānuvāca} \\
 & \romline{akṣaram brahma paramaṃ} \\
 & \romline{svabhāvo'dhyātma-mucyate |} \\
 & \romline{bhūta-bhāvod-bhavakaraḥ} \\
 & \romline{visargaḥ karma-sañjñitaḥ ||}
\end{tabular}
\end{table}

\begin{table}[H]
\begin{tabular}{cl}
\textbf{8.4} & \romline{adhi-bhūtaṃ kṣaro bhāvaḥ} \\
 & \romline{puruṣaścādhi-daivatam |} \\
 & \romline{adhi-yajño'hamevātra} \\
 & \romline{dehe dehabhṛtāṃ vara ||}
\end{tabular}
\end{table}

\begin{table}[H]
\begin{tabular}{cl}
\textbf{8.5} & \romline{antakāle ca māmeva} \\
 & \romline{smaran-muktvā kalevaram |} \\
 & \romline{yaḥ prayāti sa madbhāvaṃ} \\
 & \romline{yāti nāstyatra saṃśayaḥ ||}
\end{tabular}
\end{table}

\begin{table}[H]
\begin{tabular}{cl}
\textbf{8.6} & \romline{yaṃ yaṃ vāpi smaran-bhāvaṃ} \\
 & \romline{tyaja-tyante kalevaram |} \\
 & \romline{taṃ tamevaiti kaunteya} \\
 & \romline{sadā tad-bhāva-bhāvitaḥ ||}
\end{tabular}
\end{table}

\begin{table}[H]
\begin{tabular}{cl}
\textbf{8.7} & \romline{tasmāt-sarveṣu kāleṣu} \\
 & \romline{māmanusmara yudhya ca |} \\
 & \romline{mayyar-pita-manobuddhiḥ} \\
 & \romline{māme-vaiṣyasya-saṃśayam ||}
\end{tabular}
\end{table}

\begin{table}[H]
\begin{tabular}{cl}
\textbf{8.8} & \romline{abhyāsa-yogayuktena} \\
 & \romline{cetasā nānya-gāminā |} \\
 & \romline{paramaṃ puruṣaṃ divyaṃ} \\
 & \romline{yāti pārthā-nucintayan ||}
\end{tabular}
\end{table}

\begin{table}[H]
\begin{tabular}{cl}
\textbf{8.9} & \romline{kaviṃ purāṇa-manuśāsitāram} \\
 & \romline{aṇoraṇīyāṃ-samanu-smaredyaḥ |} \\
 & \romline{sarvasya dhātāra-macintyarūpam} \\
 & \romline{āditya-varṇaṃ tamasaḥ parastāt ||}
\end{tabular}
\end{table}

\begin{table}[H]
\begin{tabular}{cl}
\textbf{8.10} & \romline{prayāṇakāle manasā'calena} \\
 & \romline{bhaktyā yukto yogabalena caiva |} \\
 & \romline{bhruvor-madhye prāṇa-māveśya samyak} \\
 & \romline{sa taṃ paraṃ puruṣamupaiti divyaṃ ||}
\end{tabular}
\end{table}

\begin{table}[H]
\begin{tabular}{cl}
\textbf{8.11} & \romline{yadakṣaraṃ vedavido vadanti} \\
 & \romline{viśanti yadyatayo vītarāgāḥ |} \\
 & \romline{yadicchanto brahma-caryaṃ caranti} \\
 & \romline{tatte padaṃ saṅgraheṇa·pravakṣye ||}
\end{tabular}
\end{table}

\begin{table}[H]
\begin{tabular}{cl}
\textbf{8.12} & \romline{sarva-dvārāṇi saṃyamya} \\
 & \romline{mano hṛdi nirudhya ca |} \\
 & \romline{mūrdhnyā-dhāyāt-manaḥ prāṇam} \\
 & \romline{āsthito yoga-dhāraṇāṃ ||}
\end{tabular}
\end{table}

\begin{table}[H]
\begin{tabular}{cl}
\textbf{8.13} & \romline{omityekā-kṣaraṃ brahma} \\
 & \romline{vyāharan-māma-nusmaran |} \\
 & \romline{yaḥ prayāti tyajan-dehaṃ} \\
 & \romline{sa yāti paramāṃ gatim ||}
\end{tabular}
\end{table}

\begin{table}[H]
\begin{tabular}{cl}
\textbf{8.14} & \romline{ananyacetāḥ satataṃ} \\
 & \romline{yo māṃ smarati nityaśaḥ |} \\
 & \romline{tasyāhaṃ sulabhaḥ pārtha} \\
 & \romline{nityayuktasya yoginaḥ ||}
\end{tabular}
\end{table}

\begin{table}[H]
\begin{tabular}{cl}
\textbf{8.15} & \romline{māmupetya punarjanma} \\
 & \romline{duḥkhālaya-maśāśvatam |} \\
 & \romline{nāpnuvanti mahātmānaḥ} \\
 & \romline{saṃsiddhiṃ paramāṃ gatāḥ ||}
\end{tabular}
\end{table}

\begin{table}[H]
\begin{tabular}{cl}
\textbf{8.16} & \romline{ābrahma-bhuvanā-llokāḥ} \\
 & \romline{punarā-vartino'rjuna |} \\
 & \romline{māmupetya tu kaunteya} \\
 & \romline{punarjanma na vidyate ||}
\end{tabular}
\end{table}

\begin{table}[H]
\begin{tabular}{cl}
\textbf{8.17} & \romline{sahasra-yuga-paryantam} \\
 & \romline{aharyad-brahmaṇo viduḥ |} \\
 & \romline{rātriṃ yuga-sahasrāntāṃ} \\
 & \romline{te'ho-rātravido janāḥ ||}
\end{tabular}
\end{table}

\begin{table}[H]
\begin{tabular}{cl}
\textbf{8.18} & \romline{avyaktā-dvyakta-yaḥ sarvāḥ} \\
 & \romline{prabha-vantya-harāgame |} \\
 & \romline{rātryā-game pralīyante} \\
 & \romline{tatraivā-vyakta-sañjñake ||}
\end{tabular}
\end{table}

\begin{table}[H]
\begin{tabular}{cl}
\textbf{8.19} & \romline{bhūta-grāmaḥ sa evāyaṃ} \\
 & \romline{bhūtvā bhūtvā pralīyate |} \\
 & \romline{rātryāgame'vaśaḥ pārtha} \\
 & \romline{prabha-vatya-harāgame ||}
\end{tabular}
\end{table}

\begin{table}[H]
\begin{tabular}{cl}
\textbf{8.20} & \romline{parastas-māttu bhāvo'nyaḥ} \\
 & \romline{avyakto'vyaktāt-sanātanaḥ |} \\
 & \romline{yaḥ sa sarveṣu bhūteṣu} \\
 & \romline{naśyatsu na vinaśyati ||}
\end{tabular}
\end{table}

\begin{table}[H]
\begin{tabular}{cl}
\textbf{8.21} & \romline{avyakto'kṣara ityuktaḥ} \\
 & \romline{tamāhuḥ paramāṃ gatim |} \\
 & \romline{yaṃ prāpya na nivartante} \\
 & \romline{taddhāma paramaṃ mama ||}
\end{tabular}
\end{table}

\begin{table}[H]
\begin{tabular}{cl}
\textbf{8.22} & \romline{puruṣaḥ sa paraḥ pārtha} \\
 & \romline{bhaktyā labhyastva-nanyayā |} \\
 & \romline{yasyāntaḥ sthāni bhūtāni} \\
 & \romline{yena sarvamidaṃ tatam ||}
\end{tabular}
\end{table}

\begin{table}[H]
\begin{tabular}{cl}
\textbf{8.23} & \romline{yatra kāle tvanāvṛtim} \\
 & \romline{āvṛtiṃ caiva yoginaḥ |} \\
 & \romline{prayātā yānti taṃ kālaṃ} \\
 & \romline{vakṣyāmi bharatarṣabha ||}
\end{tabular}
\end{table}

\begin{table}[H]
\begin{tabular}{cl}
\textbf{8.24} & \romline{agnir-jyotirahaḥ śuklaḥ} \\
 & \romline{ṣaṇmāsā uttarāyaṇam |} \\
 & \romline{tatra·prayātā gacchanti} \\
 & \romline{brahma·brahmavido janāḥ ||}
\end{tabular}
\end{table}

\begin{table}[H]
\begin{tabular}{cl}
\textbf{8.25} & \romline{dhūmo rātri-stathā kṛṣṇaḥ} \\
 & \romline{ṣaṇmāsā dakṣiṇāyanam |} \\
 & \romline{tatra cāndra-masaṃ jyotiḥ} \\
 & \romline{yogī prāpya nivartate ||}
\end{tabular}
\end{table}

\begin{table}[H]
\begin{tabular}{cl}
\textbf{8.26} & \romline{śuklakṛṣṇe gatī hyete} \\
 & \romline{jagataḥ śāśvate mate |} \\
 & \romline{ekayā yātya-nāvṛtim} \\
 & \romline{anyayā''vartate punaḥ ||}
\end{tabular}
\end{table}

\begin{table}[H]
\begin{tabular}{cl}
\textbf{8.27} & \romline{naite sṛtī pārtha jānan} \\
 & \romline{yogī muhyati kaścana |} \\
 & \romline{tasmāt-sarveṣu kāleṣu} \\
 & \romline{yogayukto bhavārjuna ||}
\end{tabular}
\end{table}

\begin{table}[H]
\begin{tabular}{cl}
\textbf{8.28} & \romline{vedeṣu yajñeṣu tapassu caiva} \\
 & \romline{dāneṣu yat puṇya-phalaṃ pradiṣṭam |} \\
 & \romline{atyeti tat-sarvamidaṃ viditvā} \\
 & \romline{yogī paraṃ sthāna-mupaiti cādyam ||}
\end{tabular}
\end{table}

