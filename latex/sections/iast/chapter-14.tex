\begin{table}[H]
\begin{tabular}{cl}
 & \romline{śrī paramātmane namaḥ} \\
 & \romline{atha caturdaśo'dhyāyaḥ} \\
 & \romline{guṇatrayavibhāgayogaḥ}
\end{tabular}
\end{table}

\begin{table}[H]
\begin{tabular}{cl}
\textbf{14.1} & \romline{śrī bhagavānuvāca} \\
 & \romline{paraṃ bhūyaḥ pravakṣyāmi} \\
 & \romline{jñānānāṃ jñānamuttamam |} \\
 & \romline{yajjñātvā munayaḥ sarve} \\
 & \romline{parāṃ siddhimito gatāḥ ||}
\end{tabular}
\end{table}

\begin{table}[H]
\begin{tabular}{cl}
\textbf{14.2} & \romline{idaṃ jñānamupāśritya} \\
 & \romline{mama sādharmyamāgatāḥ |} \\
 & \romline{sarge'pi nopajāyante} \\
 & \romline{pralaye na vyathanti ca ||}
\end{tabular}
\end{table}

\begin{table}[H]
\begin{tabular}{cl}
\textbf{14.3} & \romline{mama yonirmahadbrahma} \\
 & \romline{tasmingarbhaṃ dadhāmyaham |} \\
 & \romline{sambhavaḥ sarvabhūtānāṃ} \\
 & \romline{tato bhavati bhārata ||}
\end{tabular}
\end{table}

\begin{table}[H]
\begin{tabular}{cl}
\textbf{14.4} & \romline{sarvayoniṣu kaunteya} \\
 & \romline{mūrtayaḥ sambhavanti yāḥ |} \\
 & \romline{tāsāṃ brahma mahadyoniḥ} \\
 & \romline{ahaṃ bījapradaḥ pitā ||}
\end{tabular}
\end{table}

\begin{table}[H]
\begin{tabular}{cl}
\textbf{14.5} & \romline{sattvaṃ rajastama iti} \\
 & \romline{guṇāḥ prakṛtisambhavāḥ |} \\
 & \romline{nibadhnanti mahābāho} \\
 & \romline{dehe dehinamavyayam ||}
\end{tabular}
\end{table}

\begin{table}[H]
\begin{tabular}{cl}
\textbf{14.6} & \romline{tatra sattvaṃ nirmalatvāt} \\
 & \romline{prakāśakamanāmayam |} \\
 & \romline{sukhasaṅgena badhnāti} \\
 & \romline{jñānasaṅgena cānagha ||}
\end{tabular}
\end{table}

\begin{table}[H]
\begin{tabular}{cl}
\textbf{14.7} & \romline{rajo rāgātmakaṃ viddhi} \\
 & \romline{tṛṣṇāsaṅgasamudbhavam |} \\
 & \romline{tannibadhnāti kaunteya} \\
 & \romline{karmasaṅgena dehinam ||}
\end{tabular}
\end{table}

\begin{table}[H]
\begin{tabular}{cl}
\textbf{14.8} & \romline{tamastvajñānajaṃ viddhi} \\
 & \romline{mohanaṃ sarvadehinām |} \\
 & \romline{pramādālasyanidrābhiḥ} \\
 & \romline{tannibadhnāti bhārata ||}
\end{tabular}
\end{table}

\begin{table}[H]
\begin{tabular}{cl}
\textbf{14.9} & \romline{sattvaṃ sukhe sañjayati} \\
 & \romline{rajaḥ karmaṇi bhārata |} \\
 & \romline{jñānamāvṛtya tu tamaḥ} \\
 & \romline{pramāde sañjayatyuta ||}
\end{tabular}
\end{table}

\begin{table}[H]
\begin{tabular}{cl}
\textbf{14.10} & \romline{rajastamaścābhibhūya} \\
 & \romline{sattvaṃ bhavati bhārata |} \\
 & \romline{rajaḥ sattvaṃ tamaścaiva} \\
 & \romline{tamaḥ sattvaṃ rajastathā ||}
\end{tabular}
\end{table}

\begin{table}[H]
\begin{tabular}{cl}
\textbf{14.11} & \romline{sarvadvāreṣu dehe'smin} \\
 & \romline{prakāśa upajāyate |} \\
 & \romline{jñānaṃ yadā tadā vidyāt} \\
 & \romline{vivṛddhaṃ sattvamityuta ||}
\end{tabular}
\end{table}

\begin{table}[H]
\begin{tabular}{cl}
\textbf{14.12} & \romline{lobhaḥ pravṛttirārambhaḥ} \\
 & \romline{karmaṇāmaśamaḥ spṛhā |} \\
 & \romline{rajasyetāni jāyante} \\
 & \romline{vivṛddhe bharatarṣabha ||}
\end{tabular}
\end{table}

\begin{table}[H]
\begin{tabular}{cl}
\textbf{14.13} & \romline{aprakāśo'pravṛttiśca} \\
 & \romline{pramādo moha eva ca |} \\
 & \romline{tamasyetāni jāyante} \\
 & \romline{vivṛddhe kurunandana ||}
\end{tabular}
\end{table}

\begin{table}[H]
\begin{tabular}{cl}
\textbf{14.14} & \romline{yadā sattve pravṛddhe tu} \\
 & \romline{pralayaṃ yāti dehabhṛt |} \\
 & \romline{tadottamavidāṃ lokān} \\
 & \romline{amalānpratipadyate ||}
\end{tabular}
\end{table}

\begin{table}[H]
\begin{tabular}{cl}
\textbf{14.15} & \romline{rajasi pralayaṃ gatvā} \\
 & \romline{karmasaṅgiṣu jāyate |} \\
 & \romline{tathā pralīnastamasi} \\
 & \romline{mūḍhayoniṣu jāyate ||}
\end{tabular}
\end{table}

\begin{table}[H]
\begin{tabular}{cl}
\textbf{14.16} & \romline{karmaṇaḥ sukṛtasyāhuḥ} \\
 & \romline{sāttvikaṃ nirmalaṃ phalam |} \\
 & \romline{rajasastu phalaṃ duḥkham} \\
 & \romline{ajñānaṃ tamasaḥ phalam ||}
\end{tabular}
\end{table}

\begin{table}[H]
\begin{tabular}{cl}
\textbf{14.17} & \romline{sattvātsañjāyate jñānaṃ} \\
 & \romline{rajaso lobha eva ca |} \\
 & \romline{pramādamohau tamasaḥ} \\
 & \romline{bhavato'jñānameva ca ||}
\end{tabular}
\end{table}

\begin{table}[H]
\begin{tabular}{cl}
\textbf{14.18} & \romline{ūrdhvaṃ gacchanti sattvasthāḥ} \\
 & \romline{madhye tiṣṭhanti rājasāḥ |} \\
 & \romline{jaghanyaguṇavṛttisthāḥ} \\
 & \romline{adho gacchanti tāmasāḥ ||}
\end{tabular}
\end{table}

\begin{table}[H]
\begin{tabular}{cl}
\textbf{14.19} & \romline{nānyaṃ guṇebhyaḥ kartāraṃ} \\
 & \romline{yadā draṣṭā'nupaśyati |} \\
 & \romline{guṇebhyaśca paraṃ vetti} \\
 & \romline{madbhāvaṃ so'dhigacchati ||}
\end{tabular}
\end{table}

\begin{table}[H]
\begin{tabular}{cl}
\textbf{14.20} & \romline{guṇānetānatītya trīn} \\
 & \romline{dehī dehasamudbhavān |} \\
 & \romline{janmamṛtyujarāduḥkhaiḥ} \\
 & \romline{vimukto'mṛtamaśnute ||}
\end{tabular}
\end{table}

\begin{table}[H]
\begin{tabular}{cl}
\textbf{14.21} & \romline{arjuna uvāca} \\
 & \romline{kairliṅgaistrīnguṇānetān} \\
 & \romline{atīto bhavati prabho |} \\
 & \romline{kimācāraḥ kathaṃ caitān} \\
 & \romline{trīnguṇānativartate ||}
\end{tabular}
\end{table}

\begin{table}[H]
\begin{tabular}{cl}
\textbf{14.22} & \romline{śrī bhagavānuvāca} \\
 & \romline{prakāśaṃ ca pravṛttiṃ ca} \\
 & \romline{mohameva ca pāṇḍava |} \\
 & \romline{na dveṣṭi sampravṛttāni} \\
 & \romline{na nivṛttāni kāṅkṣati ||}
\end{tabular}
\end{table}

\begin{table}[H]
\begin{tabular}{cl}
\textbf{14.23} & \romline{udāsīnavadāsīnaḥ} \\
 & \romline{guṇairyo na vicālyate |} \\
 & \romline{guṇā vartanta ityeva} \\
 & \romline{yo'vatiṣṭhati neṅgate ||}
\end{tabular}
\end{table}

\begin{table}[H]
\begin{tabular}{cl}
\textbf{14.24} & \romline{samaduḥkhasukhaḥ svasthaḥ} \\
 & \romline{samaloṣṭāśmakāñcanaḥ |} \\
 & \romline{tulyapriyāpriyo dhīraḥ} \\
 & \romline{tulyanindātmasaṃstutiḥ ||}
\end{tabular}
\end{table}

\begin{table}[H]
\begin{tabular}{cl}
\textbf{14.25} & \romline{mānāpamānayostulyaḥ} \\
 & \romline{tulyo mitrāripakṣayoḥ |} \\
 & \romline{sarvārambhaparityāgī} \\
 & \romline{guṇātītaḥ sa ucyate ||}
\end{tabular}
\end{table}

\begin{table}[H]
\begin{tabular}{cl}
\textbf{14.26} & \romline{māṃ ca yo'vyabhicāreṇa} \\
 & \romline{bhaktiyogena sevate |} \\
 & \romline{sa guṇānsamatītyaitān} \\
 & \romline{brahmabhūyāya kalpate ||}
\end{tabular}
\end{table}

\begin{table}[H]
\begin{tabular}{cl}
\textbf{14.27} & \romline{brahmaṇo hi pratiṣṭhā'ham} \\
 & \romline{amṛtasyāvyayasya ca |} \\
 & \romline{śāśvatasya ca dharmasya} \\
 & \romline{sukhasyaikāntikasya ca ||}
\end{tabular}
\end{table}

\begin{table}[H]
\begin{tabular}{cl}
 & \romline{śrīmadbhagavadgītāsu upaniṣatsu} \\
 & \romline{brahmavidyāyāṃ yogaśāstre} \\
 & \romline{śrīkṛṣṇārjuna saṃvāde} \\
 & \romline{guṇatrayavibhāgayogo nāma} \\
 & \romline{caturdaśodhyāyaḥ}
\end{tabular}
\end{table}

