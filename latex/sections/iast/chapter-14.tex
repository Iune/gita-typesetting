\begin{table}[H]
\begin{tabular}{cl}
 & \romline{śrī paramātmane namaḥ} \\
 & \romline{atha caturdaśo'dhyāyaḥ} \\
 & \romline{guṇa-traya-vibhāga-yogaḥ}
\end{tabular}
\end{table}

\begin{table}[H]
\begin{tabular}{cl}
\textbf{14.1} & \romline{śrī bhagavānuvāca} \\
 & \romline{paraṃ bhūyaḥ pravakṣyāmi} \\
 & \romline{jñānānāṃ jñāna-muttamam |} \\
 & \romline{yaj-jñātvā munayaḥ sarve} \\
 & \romline{parāṃ siddhi-mito gatāḥ ||}
\end{tabular}
\end{table}

\begin{table}[H]
\begin{tabular}{cl}
\textbf{14.2} & \romline{idaṃ jñāna-mupāśritya} \\
 & \romline{mama sādharmya-māgatāḥ |} \\
 & \romline{sarge'pi nopajāyante} \\
 & \romline{pralaye na·vyathanti ca ||}
\end{tabular}
\end{table}

\begin{table}[H]
\begin{tabular}{cl}
\textbf{14.3} & \romline{mama yonir-mahad-brahma} \\
 & \romline{tasmin-garbhaṃ dadhāmyaham |} \\
 & \romline{sambhavaḥ sarva-bhūtānāṃ} \\
 & \romline{tato bhavati bhārata ||}
\end{tabular}
\end{table}

\begin{table}[H]
\begin{tabular}{cl}
\textbf{14.4} & \romline{sarva-yoniṣu kaunteya} \\
 & \romline{mūrtayaḥ sambha-vanti yāḥ |} \\
 & \romline{tāsāṃ brahma mahad-yoniḥ} \\
 & \romline{ahaṃ bīja-pradaḥ pitā ||}
\end{tabular}
\end{table}

\begin{table}[H]
\begin{tabular}{cl}
\textbf{14.5} & \romline{sattvaṃ rajas-tama iti} \\
 & \romline{guṇāḥ prakṛti-sambhavāḥ |} \\
 & \romline{nibadhnanti mahābāho} \\
 & \romline{dehe dehina-mavyayam ||}
\end{tabular}
\end{table}

\begin{table}[H]
\begin{tabular}{cl}
\textbf{14.6} & \romline{tatra sattvaṃ nirmala-tvāt} \\
 & \romline{prakāśa-kama-nāmayam |} \\
 & \romline{sukha-saṅgena badhnāti} \\
 & \romline{jñāna-saṅgena cānagha ||}
\end{tabular}
\end{table}

\begin{table}[H]
\begin{tabular}{cl}
\textbf{14.7} & \romline{rajo rāgātmakaṃ viddhi} \\
 & \romline{tṛṣṇā-saṅga-samudbhavam |} \\
 & \romline{tanni-badhnāti kaunteya} \\
 & \romline{karma-saṅgena dehinam ||}
\end{tabular}
\end{table}

\begin{table}[H]
\begin{tabular}{cl}
\textbf{14.8} & \romline{tamastva-jñāna-jaṃ viddhi} \\
 & \romline{mohanaṃ sarva-dehinām |} \\
 & \romline{pramādālasya-nidrābhiḥ} \\
 & \romline{tanni-badhnāti bhārata ||}
\end{tabular}
\end{table}

\begin{table}[H]
\begin{tabular}{cl}
\textbf{14.9} & \romline{sattvaṃ sukhe sañjayati} \\
 & \romline{rajaḥ karmaṇi bhārata |} \\
 & \romline{jñāna-māvṛtya tu tamaḥ} \\
 & \romline{pramāde sañjaya-tyuta ||}
\end{tabular}
\end{table}

\begin{table}[H]
\begin{tabular}{cl}
\textbf{14.10} & \romline{rajas-tamaścā-bhibhūya} \\
 & \romline{sattvaṃ bhavati bhārata |} \\
 & \romline{rajaḥ sattvaṃ tamaścaiva} \\
 & \romline{tamaḥ sattvaṃ rajas-tathā ||}
\end{tabular}
\end{table}

\begin{table}[H]
\begin{tabular}{cl}
\textbf{14.11} & \romline{sarva-dvāreṣu dehe'smin} \\
 & \romline{prakāśa upajāyate |} \\
 & \romline{jñānaṃ yadā tadā vidyāt} \\
 & \romline{vivṛddhaṃ sattvami-tyuta ||}
\end{tabular}
\end{table}

\begin{table}[H]
\begin{tabular}{cl}
\textbf{14.12} & \romline{lobhaḥ pravṛtti-rārambhaḥ} \\
 & \romline{karmaṇā-maśamaḥ spṛhā |} \\
 & \romline{rajas-yetāni jāyante} \\
 & \romline{vivṛddhe bhara-tarṣabha ||}
\end{tabular}
\end{table}

\begin{table}[H]
\begin{tabular}{cl}
\textbf{14.13} & \romline{aprakāśo'pravṛttiśca} \\
 & \romline{pramādo moha eva ca |} \\
 & \romline{tamasyetāni jāyante} \\
 & \romline{vivṛddhe kuru-nandana ||}
\end{tabular}
\end{table}

\begin{table}[H]
\begin{tabular}{cl}
\textbf{14.14} & \romline{yadā sattve pravṛddhe tu} \\
 & \romline{pralayaṃ yāti dehabhṛt |} \\
 & \romline{tadotta-mavidāṃ lokān} \\
 & \romline{amalān-pratipadyate ||}
\end{tabular}
\end{table}

\begin{table}[H]
\begin{tabular}{cl}
\textbf{14.15} & \romline{rajasi·pralayaṃ gatvā} \\
 & \romline{karma-saṅgiṣu jāyate |} \\
 & \romline{tathā pralīna-stamasi} \\
 & \romline{mūḍha-yoniṣu jāyate ||}
\end{tabular}
\end{table}

\begin{table}[H]
\begin{tabular}{cl}
\textbf{14.16} & \romline{karmaṇaḥ sukṛta-syāhuḥ} \\
 & \romline{sāttvikaṃ nirmalaṃ phalam |} \\
 & \romline{raja-sastu phalaṃ duḥkham} \\
 & \romline{ajñānaṃ tamasaḥ phalam ||}
\end{tabular}
\end{table}

\begin{table}[H]
\begin{tabular}{cl}
\textbf{14.17} & \romline{sattvāt-sañjāyate jñānaṃ} \\
 & \romline{rajaso lobha eva ca |} \\
 & \romline{pramāda-mohau tamasaḥ} \\
 & \romline{bhavato'jñāna-meva ca ||}
\end{tabular}
\end{table}

\begin{table}[H]
\begin{tabular}{cl}
\textbf{14.18} & \romline{ūrdhvaṃ gacchanti sattva-sthāḥ} \\
 & \romline{madhye tiṣṭhanti rājasāḥ |} \\
 & \romline{jaghanya-guṇa-vṛtti-sthāḥ} \\
 & \romline{adho gacchanti tāmasāḥ ||}
\end{tabular}
\end{table}

\begin{table}[H]
\begin{tabular}{cl}
\textbf{14.19} & \romline{nānyaṃ guṇebhyaḥ kartāraṃ} \\
 & \romline{yadā draṣṭā'nupaśyati |} \\
 & \romline{guṇebhyaśca paraṃ vetti} \\
 & \romline{madbhāvaṃ so'dhi-gacchati ||}
\end{tabular}
\end{table}

\begin{table}[H]
\begin{tabular}{cl}
\textbf{14.20} & \romline{guṇā-netā-natītya·trīn} \\
 & \romline{dehī deha-samud-bhavān |} \\
 & \romline{janma-mṛtyu-jarā-duḥkhaiḥ} \\
 & \romline{vimukto'mṛta-maśnute ||}
\end{tabular}
\end{table}

\begin{table}[H]
\begin{tabular}{cl}
\textbf{14.21} & \romline{arjuna uvāca} \\
 & \romline{kairliṅgai-strīn-guṇā-netān} \\
 & \romline{atīto bhavati·prabho |} \\
 & \romline{kimācāraḥ kathaṃ caitān} \\
 & \romline{trīn-guṇā-nati-vartate ||}
\end{tabular}
\end{table}

\begin{table}[H]
\begin{tabular}{cl}
\textbf{14.22} & \romline{śrī bhagavānuvāca} \\
 & \romline{prakāśaṃ ca·pravṛttiṃ ca} \\
 & \romline{mohameva ca pāṇḍava |} \\
 & \romline{na·dveṣṭi sam-pravṛttāni} \\
 & \romline{na nivṛttāni kāṅkṣati ||}
\end{tabular}
\end{table}

\begin{table}[H]
\begin{tabular}{cl}
\textbf{14.23} & \romline{udāsīnava-dāsīnaḥ} \\
 & \romline{guṇairyo na vicālyate |} \\
 & \romline{guṇā vartanta ityeva} \\
 & \romline{yo'va-tiṣṭhati neṅgate ||}
\end{tabular}
\end{table}

\begin{table}[H]
\begin{tabular}{cl}
\textbf{14.24} & \romline{sama-duḥkha-sukhaḥ svasthaḥ} \\
 & \romline{samaloṣṭāś-makāñcanaḥ |} \\
 & \romline{tulya-priyāpriyo dhīraḥ} \\
 & \romline{tulya-nindātma-saṃstutiḥ ||}
\end{tabular}
\end{table}

\begin{table}[H]
\begin{tabular}{cl}
\textbf{14.25} & \romline{mānāpa-mānayo-stulyaḥ} \\
 & \romline{tulyo mitrā-ripakṣayoḥ |} \\
 & \romline{sarvā-rambha-pari-tyāgī} \\
 & \romline{guṇā-tītaḥ sa ucyate ||}
\end{tabular}
\end{table}

\begin{table}[H]
\begin{tabular}{cl}
\textbf{14.26} & \romline{māṃ ca yo'vya-bhicāreṇa} \\
 & \romline{bhakti-yogena sevate |} \\
 & \romline{sa guṇān-samatī-tyaitān} \\
 & \romline{brahma-bhūyāya kalpate ||}
\end{tabular}
\end{table}

\begin{table}[H]
\begin{tabular}{cl}
\textbf{14.27} & \romline{brahmaṇo hi·pratiṣṭhā'ham} \\
 & \romline{amṛtas-yāvya-yasya ca |} \\
 & \romline{śāśva-tasya ca dharmasya} \\
 & \romline{sukhasyai-kāntikasya ca ||}
\end{tabular}
\end{table}

\begin{table}[H]
\begin{tabular}{cl}
 & \romline{śrīmad-bhagavad-gītāsu upaniṣatsu} \\
 & \romline{brahma-vidyāyāṃ yogaśāstre} \\
 & \romline{śrīkṛṣṇārjuna saṃvāde} \\
 & \romline{guṇa-trayavibhāga-yogonāma} \\
 & \romline{caturdaśo-dhyāyaḥ}
\end{tabular}
\end{table}

