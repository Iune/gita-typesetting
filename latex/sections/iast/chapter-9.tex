\begin{table}[H]
\begin{tabular}{cl}
 & \romline{śrī paramātmane namaḥ} \\
 & \romline{atha navamo'dhyāyaḥ} \\
 & \romline{rājavidyā-rājaguhya-yogaḥ}
\end{tabular}
\end{table}

\begin{table}[H]
\begin{tabular}{cl}
\textbf{9.1} & \romline{śrī bhagavānuvāca} \\
 & \romline{idaṃ tu te guhyatamam} \\
 & \romline{pravakṣyāmya-nasūyave |} \\
 & \romline{jñānaṃ vijñāna-sahitaṃ} \\
 & \romline{yaj-jñātvā mokṣyase'śubhāt ||}
\end{tabular}
\end{table}

\begin{table}[H]
\begin{tabular}{cl}
\textbf{9.2} & \romline{rāja-vidyā rāja-guhyaṃ} \\
 & \romline{pavitra-midamuttamam |} \\
 & \romline{pratya-kṣāvagamaṃ dharmyaṃ} \\
 & \romline{susukhaṃ kartu-mavyayam ||}
\end{tabular}
\end{table}

\begin{table}[H]
\begin{tabular}{cl}
\textbf{9.3} & \romline{aśradda-dhānāḥ puruṣāḥ} \\
 & \romline{dharma-syāsya parantapa |} \\
 & \romline{aprāpya māṃ nivartante} \\
 & \romline{mṛtyu-saṃsāra-vartmani ||}
\end{tabular}
\end{table}

\begin{table}[H]
\begin{tabular}{cl}
\textbf{9.4} & \romline{mayā tatamidaṃ sarvaṃ} \\
 & \romline{jagadavyakta-mūrtinā |} \\
 & \romline{matsthāni sarva-bhūtāni} \\
 & \romline{na cāhaṃ teṣva-vasthitaḥ ||}
\end{tabular}
\end{table}

\begin{table}[H]
\begin{tabular}{cl}
\textbf{9.5} & \romline{na ca matsthāni bhūtāni} \\
 & \romline{paśya me yogamaiśvaram |} \\
 & \romline{bhūta-bhṛnna ca bhūtasthaḥ} \\
 & \romline{mamātmā bhūta-bhāvanaḥ ||}
\end{tabular}
\end{table}

\begin{table}[H]
\begin{tabular}{cl}
\textbf{9.6} & \romline{yathā''kāśa-sthito nityaṃ} \\
 & \romline{vāyuḥ sarva-trago mahān |} \\
 & \romline{tathā sarvāṇi bhūtāni} \\
 & \romline{matsthānītyu-padhāraya ||}
\end{tabular}
\end{table}

\begin{table}[H]
\begin{tabular}{cl}
\textbf{9.7} & \romline{sarva-bhūtāni kaunteya} \\
 & \romline{prakṛtiṃ yānti māmikām |} \\
 & \romline{kalpa-kṣaye punastāni} \\
 & \romline{kalpādau visṛjāmyaham ||}
\end{tabular}
\end{table}

\begin{table}[H]
\begin{tabular}{cl}
\textbf{9.8} & \romline{prakṛtiṃ svāma-vaṣṭabhya} \\
 & \romline{visṛjāmi punaḥ punaḥ |} \\
 & \romline{bhūta-grāmamimaṃ kṛtsnam} \\
 & \romline{avaśaṃ prakṛtervaśāt ||}
\end{tabular}
\end{table}

\begin{table}[H]
\begin{tabular}{cl}
\textbf{9.9} & \romline{na ca māṃ tāni karmāṇi} \\
 & \romline{nibadhnanti dhanañjaya |} \\
 & \romline{udāsīna-vadāsīnam} \\
 & \romline{asaktaṃ teṣu karmasu ||}
\end{tabular}
\end{table}

\begin{table}[H]
\begin{tabular}{cl}
\textbf{9.10} & \romline{mayā-dhyakṣeṇa prakṛtiḥ} \\
 & \romline{sūyate sacarācaram |} \\
 & \romline{hetunā'nena kaunteya} \\
 & \romline{jagadvipari-vartate ||}
\end{tabular}
\end{table}

\begin{table}[H]
\begin{tabular}{cl}
\textbf{9.11} & \romline{avajānanti māṃ mūḍhāḥ} \\
 & \romline{mānuṣīṃ tanu-māśritam |} \\
 & \romline{paraṃ bhāva-majānantaḥ} \\
 & \romline{mama bhūta-maheśvaram ||}
\end{tabular}
\end{table}

\begin{table}[H]
\begin{tabular}{cl}
\textbf{9.12} & \romline{moghāśā mogha-karmāṇaḥ} \\
 & \romline{mogha-jñānā vicetasaḥ |} \\
 & \romline{rākṣasī-māsurīṃ caiva} \\
 & \romline{prakṛtiṃ mohinīṃ śritāḥ ||}
\end{tabular}
\end{table}

\begin{table}[H]
\begin{tabular}{cl}
\textbf{9.13} & \romline{mahātmā-nastu māṃ pārtha} \\
 & \romline{daivīṃ prakṛti-māśritāḥ |} \\
 & \romline{bhajantya-nanya-manasaḥ} \\
 & \romline{jñātvā bhūtādi-mavyayam ||}
\end{tabular}
\end{table}

\begin{table}[H]
\begin{tabular}{cl}
\textbf{9.14} & \romline{satataṃ kīrtayanto māṃ} \\
 & \romline{yatantaśca dṛḍha-vratāḥ |} \\
 & \romline{namasyantaśca mām bhaktyā} \\
 & \romline{nityayuktā upāsate ||}
\end{tabular}
\end{table}

\begin{table}[H]
\begin{tabular}{cl}
\textbf{9.15} & \romline{jñāna-yajñena cāpyanye} \\
 & \romline{yajanto māmupāsate |} \\
 & \romline{ekatvena pṛthak-tvena} \\
 & \romline{bahudhā viśvato-mukham ||}
\end{tabular}
\end{table}

\begin{table}[H]
\begin{tabular}{cl}
\textbf{9.16} & \romline{ahaṃ kraturahaṃ yajñaḥ} \\
 & \romline{svadhā-hama-hamauṣadham |} \\
 & \romline{mantro'hama-hame-vājyam} \\
 & \romline{aha-magni-rahaṃ hutam ||}
\end{tabular}
\end{table}

\begin{table}[H]
\begin{tabular}{cl}
\textbf{9.17} & \romline{pitā'hamasya jagataḥ} \\
 & \romline{mātā dhātā pitāmahaḥ |} \\
 & \romline{vedyaṃ pavitra-moṅkāraḥ} \\
 & \romline{ṛksāma yajureva ca ||}
\end{tabular}
\end{table}

\begin{table}[H]
\begin{tabular}{cl}
\textbf{9.18} & \romline{gatir-bhartā prabhuḥ sākṣī} \\
 & \romline{nivāsaḥ śaraṇaṃ suhṛt |} \\
 & \romline{prabhavaḥ pralayaḥ sthānaṃ} \\
 & \romline{nidhānaṃ bīja-mavyayam ||}
\end{tabular}
\end{table}

\begin{table}[H]
\begin{tabular}{cl}
\textbf{9.19} & \romline{tapāmya-ham-ahaṃ varṣaṃ} \\
 & \romline{nigṛhṇāmyut-sṛjāmi ca |} \\
 & \romline{amṛtaṃ caiva mṛtyuśca} \\
 & \romline{sadasaccāha-marjuna ||}
\end{tabular}
\end{table}

\begin{table}[H]
\begin{tabular}{cl}
\textbf{9.20} & \romline{traividyā māṃ somapāḥ pūtapāpāḥ} \\
 & \romline{yajñai-riṣṭvā svargatiṃ prārtha-yante |} \\
 & \romline{te puṇyamāsādya surendra-lokam} \\
 & \romline{aśnanti divyāndivi devabhogān ||}
\end{tabular}
\end{table}

\begin{table}[H]
\begin{tabular}{cl}
\textbf{9.21} & \romline{te taṃ bhuktvā svargalokaṃ viśālaṃ} \\
 & \romline{kṣīṇe puṇye martya-lokaṃ viśanti |} \\
 & \romline{evaṃ trayī-dharma-manu-prapannāḥ} \\
 & \romline{gatāgataṃ kāma-kāmā labhante ||}
\end{tabular}
\end{table}

\begin{table}[H]
\begin{tabular}{cl}
\textbf{9.22} & \romline{ananyāścinta-yanto māṃ} \\
 & \romline{ye janāḥ paryupāsate |} \\
 & \romline{teṣāṃ nityābhi-yuktānāṃ} \\
 & \romline{yogakṣemaṃ vahāmyaham ||}
\end{tabular}
\end{table}

\begin{table}[H]
\begin{tabular}{cl}
\textbf{9.23} & \romline{ye'pyanya-devatā bhaktāḥ} \\
 & \romline{yajante śraddha-yānvitāḥ |} \\
 & \romline{te'pi māmeva kaunteya} \\
 & \romline{yajantya-vidhi-pūrvakam ||}
\end{tabular}
\end{table}

\begin{table}[H]
\begin{tabular}{cl}
\textbf{9.24} & \romline{ahaṃ hi sarva-yajñānāṃ} \\
 & \romline{bhoktā ca prabhureva ca |} \\
 & \romline{na tu māma-bhijānanti} \\
 & \romline{tattvenā-taścya-vanti te ||}
\end{tabular}
\end{table}

\begin{table}[H]
\begin{tabular}{cl}
\textbf{9.25} & \romline{yānti deva-vratā devāṇ} \\
 & \romline{pitṝn yānti pitṛ-vratāḥ |} \\
 & \romline{bhūtāni yānti bhūtejyāḥ} \\
 & \romline{yānti madyājino'pi mām ||}
\end{tabular}
\end{table}

\begin{table}[H]
\begin{tabular}{cl}
\textbf{9.26} & \romline{patraṃ puṣpaṃ phalaṃ toyaṃ} \\
 & \romline{yo me bhaktyā prayacchati |} \\
 & \romline{tadahaṃ bhaktyu-pahṛtaṃ} \\
 & \romline{aśnāmi·praya-tāt-manaḥ ||}
\end{tabular}
\end{table}

\begin{table}[H]
\begin{tabular}{cl}
\textbf{9.27} & \romline{yatkaroṣi yadaśnāsi} \\
 & \romline{yajjuhoṣi dadāsi yat |} \\
 & \romline{yattapasyasi kaunteya} \\
 & \romline{tatkuruṣva madarpaṇam ||}
\end{tabular}
\end{table}

\begin{table}[H]
\begin{tabular}{cl}
\textbf{9.28} & \romline{śubhā-śubha-phalairevaṃ} \\
 & \romline{mokṣyase karma-bandhanaiḥ |} \\
 & \romline{sannyāsa-yoga-yuktātmā} \\
 & \romline{vimukto māmupaiṣyasi ||}
\end{tabular}
\end{table}

\begin{table}[H]
\begin{tabular}{cl}
\textbf{9.29} & \romline{samo'haṃ sarva-bhūteṣu} \\
 & \romline{na me dveṣyo'sti na·priyaḥ |} \\
 & \romline{ye bhajanti tu māṃ bhaktyā} \\
 & \romline{mayi te teṣu cāpyaham ||}
\end{tabular}
\end{table}

\begin{table}[H]
\begin{tabular}{cl}
\textbf{9.30} & \romline{api cet-sudurācāraḥ} \\
 & \romline{bhajate māma-nanyabhāk |} \\
 & \romline{sādhureva sa mantavyaḥ} \\
 & \romline{samyag-vyavasito hi saḥ ||}
\end{tabular}
\end{table}

\begin{table}[H]
\begin{tabular}{cl}
\textbf{9.31} & \romline{kṣipram bhavati dharmātmā} \\
 & \romline{śaśvacchāntiṃ nigacchati |} \\
 & \romline{kaunteya·prati-jānīhi} \\
 & \romline{na me bhaktaḥ praṇaśyati ||}
\end{tabular}
\end{table}

\begin{table}[H]
\begin{tabular}{cl}
\textbf{9.32} & \romline{mām hi pārtha·vyapāśritya} \\
 & \romline{ye'pi syuḥ pāpa-yonayaḥ |} \\
 & \romline{striyo vaiśyā-stathā śūdrāḥ} \\
 & \romline{te'pi yānti parāṃ gatim ||}
\end{tabular}
\end{table}

\begin{table}[H]
\begin{tabular}{cl}
\textbf{9.33} & \romline{kiṃ punar-brāhmaṇāḥ puṇyāḥ} \\
 & \romline{bhaktā rājarṣaya-stathā |} \\
 & \romline{anitya-masukhaṃ lokam} \\
 & \romline{imaṃ prāpya bhajasva mām ||}
\end{tabular}
\end{table}

\begin{table}[H]
\begin{tabular}{cl}
\textbf{9.34} & \romline{manmanā bhava madbhaktaḥ} \\
 & \romline{madyājī māṃ namaskuru |} \\
 & \romline{māmevaiṣyasi yuktvaivam} \\
 & \romline{ātmānaṃ matparāyaṇaḥ ||}
\end{tabular}
\end{table}

\begin{table}[H]
\begin{tabular}{cl}
 & \romline{śrīmad-bhagavad-gītāsu upaniṣatsu} \\
 & \romline{brahma-vidyāyāṃ yogaśāstre} \\
 & \romline{śrīkṛṣṇārjuna saṃvāde} \\
 & \romline{rājavidyā-rājaguhya-yogonāma} \\
 & \romline{navamo-dhyāyaḥ}
\end{tabular}
\end{table}

