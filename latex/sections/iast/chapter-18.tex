\begin{table}[H]
\begin{tabular}{cl}
\textbf{18.0} & \romline{oṃ śrī paramātmane namaḥ} \\
 & \romline{atha aṣṭādaśo'dhyāyaḥ} \\
 & \romline{mokṣa-sannyāsa yogaḥ}
\end{tabular}
\end{table}

\begin{table}[H]
\begin{tabular}{cl}
\textbf{18.1} & \romline{arjuna uvāca} \\
 & \romline{sannyāsasya mahābāho} \\
 & \romline{tattva-micchāmi veditum |} \\
 & \romline{tyāgasya ca hṛṣikeśa} \\
 & \romline{pṛthak-keśini-ṣūdana ||}
\end{tabular}
\end{table}

\begin{table}[H]
\begin{tabular}{cl}
\textbf{18.2} & \romline{śrī bhagavanuvāca} \\
 & \romline{kāmyānāṃ karmaṇāṃ nyāsaṃ} \\
 & \romline{sannyāsaṃ kavayo viduḥ |} \\
 & \romline{sarva-karma-phala-tyāgaṃ} \\
 & \romline{prāhus-tyāgaṃ vicakṣaṇāḥ ||}
\end{tabular}
\end{table}

\begin{table}[H]
\begin{tabular}{cl}
\textbf{18.3} & \romline{tyājyaṃ doṣa-vadityeke} \\
 & \romline{karma·prāhur-manīṣiṇaḥ |} \\
 & \romline{yajña-dāna-tapaḥ karma} \\
 & \romline{na·tyājya-miti cāpare ||}
\end{tabular}
\end{table}

\begin{table}[H]
\begin{tabular}{cl}
\textbf{18.4} & \romline{niścayaṃ śṛṇu me tatra} \\
 & \romline{tyāge bharata-sattama |} \\
 & \romline{tyāgo hi puruṣa-vyāghra} \\
 & \romline{trividhaḥ samprakīrtitaḥ ||}
\end{tabular}
\end{table}

\begin{table}[H]
\begin{tabular}{cl}
\textbf{18.5} & \romline{yajña-dāna-tapaḥ-karma} \\
 & \romline{na·tyājyaṃ kāryameva tat |} \\
 & \romline{yajño dānaṃ tapaścaiva} \\
 & \romline{pāvanāni manīṣiṇām ||}
\end{tabular}
\end{table}

\begin{table}[H]
\begin{tabular}{cl}
\textbf{18.6} & \romline{etānyapi tu karmāṇi} \\
 & \romline{saṅgaṃ tyaktvā phalāni ca |} \\
 & \romline{karta-vyānīti me pārtha} \\
 & \romline{niścitaṃ mata-muttamam ||}
\end{tabular}
\end{table}

\begin{table}[H]
\begin{tabular}{cl}
\textbf{18.7} & \romline{niyatasya tu sannyāsaḥ} \\
 & \romline{karmaṇo nopa-padyate |} \\
 & \romline{mohā-ttasya parityāgaḥ} \\
 & \romline{tāmasaḥ parikīrtitaḥ ||}
\end{tabular}
\end{table}

\begin{table}[H]
\begin{tabular}{cl}
\textbf{18.8} & \romline{duḥkha-mityeva yatkarma} \\
 & \romline{kāya-kleśa-bhayāttyajet |} \\
 & \romline{sa kṛtvā rājasaṃ tyāgaṃ} \\
 & \romline{naiva·tyāga-phalaṃ labhet ||}
\end{tabular}
\end{table}

\begin{table}[H]
\begin{tabular}{cl}
\textbf{18.9} & \romline{kārya-mityeva yatkarma} \\
 & \romline{niyataṃ kriyate'rjuna |} \\
 & \romline{saṅgaṃ tyaktvā phalaṃ caiva} \\
 & \romline{sa·tyāgaḥ sāttviko mataḥ ||}
\end{tabular}
\end{table}

\begin{table}[H]
\begin{tabular}{cl}
\textbf{18.10} & \romline{na dveṣṭya-kuśalaṃ karma} \\
 & \romline{kuśale nānuṣajjate |} \\
 & \romline{tyāgī sattva-samāviṣṭaḥ} \\
 & \romline{medhāvī chinna-saṃśayaḥ ||}
\end{tabular}
\end{table}

\begin{table}[H]
\begin{tabular}{cl}
\textbf{18.11} & \romline{na hi deha-bhṛtā śakyaṃ} \\
 & \romline{tyaktuṃ karmāṇya-śeṣataḥ |} \\
 & \romline{yastu karma-phala-tyāgī} \\
 & \romline{sa tyāgītyabhi-dhīyate ||}
\end{tabular}
\end{table}

\begin{table}[H]
\begin{tabular}{cl}
\textbf{18.12} & \romline{aniṣṭa-miṣṭaṃ miśraṃ ca} \\
 & \romline{trividhaṃ karmaṇaḥ phalam |} \\
 & \romline{bhavatya-tyāgināṃ pretya} \\
 & \romline{na tu sannyāsināṃ kvacit ||}
\end{tabular}
\end{table}

\begin{table}[H]
\begin{tabular}{cl}
\textbf{18.13} & \romline{pañcaitāni mahābāho} \\
 & \romline{kāraṇāni nibodha me |} \\
 & \romline{sāṅkhye kṛtānte proktāni} \\
 & \romline{siddhaye sarva-karmaṇām ||}
\end{tabular}
\end{table}

\begin{table}[H]
\begin{tabular}{cl}
\textbf{18.14} & \romline{adhiṣṭhānaṃ tathā kartā} \\
 & \romline{karaṇaṃ ca pṛthag-vidham |} \\
 & \romline{vividhāśca pṛthak-ceṣṭāḥ} \\
 & \romline{daivaṃ caivātra pañcamam ||}
\end{tabular}
\end{table}

\begin{table}[H]
\begin{tabular}{cl}
\textbf{18.15} & \romline{śarīravāṅ-manobhiryat} \\
 & \romline{karma·prāra-bhate naraḥ |} \\
 & \romline{nyāyyaṃ vā viparītaṃ vā} \\
 & \romline{pañcaite tasya hetavaḥ ||}
\end{tabular}
\end{table}

\begin{table}[H]
\begin{tabular}{cl}
\textbf{18.16} & \romline{tatraivaṃ sati kartāram} \\
 & \romline{ātmānaṃ kevalaṃ tu yaḥ |} \\
 & \romline{paśyatya-kṛta-buddhi-tvāt} \\
 & \romline{na sa paśyati durmatiḥ ||}
\end{tabular}
\end{table}

\begin{table}[H]
\begin{tabular}{cl}
\textbf{18.17} & \romline{yasya nāhaṅkṛto bhāvaḥ} \\
 & \romline{buddhir-yasya na lipyate |} \\
 & \romline{hatvā'pi sa imāllokān} \\
 & \romline{na hanti na nibadhyate ||}
\end{tabular}
\end{table}

\begin{table}[H]
\begin{tabular}{cl}
\textbf{18.18} & \romline{jñānaṃ jñeyaṃ parijñātā} \\
 & \romline{trividhā karma-codanā |} \\
 & \romline{karaṇaṃ karma karteti} \\
 & \romline{trividhaḥ karma-saṅgrahaḥ ||}
\end{tabular}
\end{table}

\begin{table}[H]
\begin{tabular}{cl}
\textbf{18.19} & \romline{jñānaṃ karma ca kartā ca} \\
 & \romline{tridhaiva guṇa-bhedataḥ |} \\
 & \romline{procyate guṇa-saṅkhyāne} \\
 & \romline{yathā-vacchṛṇu tānyapi ||}
\end{tabular}
\end{table}

\begin{table}[H]
\begin{tabular}{cl}
\textbf{18.20} & \romline{sarva-bhūteṣu yenaikaṃ} \\
 & \romline{bhāva-mavyaya-mīkṣate |} \\
 & \romline{avibhaktaṃ vibhakteṣu} \\
 & \romline{taj-jñānaṃ viddhi sāttvikam ||}
\end{tabular}
\end{table}

\begin{table}[H]
\begin{tabular}{cl}
\textbf{18.21} & \romline{pṛthak-tvena tu yaj-jñānaṃ} \\
 & \romline{nānābhāvān pṛthag-vidhān |} \\
 & \romline{vetti sarveṣu bhūteṣu} \\
 & \romline{taj-jñānaṃ viddhi rājasam ||}
\end{tabular}
\end{table}

\begin{table}[H]
\begin{tabular}{cl}
\textbf{18.22} & \romline{yattu kṛtsna-vadekasmin} \\
 & \romline{kārye sakta-mahaitukam |} \\
 & \romline{atattvārtha-vadalpaṃ ca} \\
 & \romline{tattāma-samudāhṛtam ||}
\end{tabular}
\end{table}

\begin{table}[H]
\begin{tabular}{cl}
\textbf{18.23} & \romline{niyataṃ saṅgarahitam} \\
 & \romline{arāga-dveṣataḥ kṛtam |} \\
 & \romline{aphala-prepsunā karma} \\
 & \romline{yattat-sāttvika-mucyate ||}
\end{tabular}
\end{table}

\begin{table}[H]
\begin{tabular}{cl}
\textbf{18.24} & \romline{yattu kāmepsunā karma} \\
 & \romline{sāhaṅkāreṇa vā punaḥ |} \\
 & \romline{kriyate bahulāyāsaṃ} \\
 & \romline{tadrāja-samudāhṛtam ||}
\end{tabular}
\end{table}

\begin{table}[H]
\begin{tabular}{cl}
\textbf{18.25} & \romline{anubandhaṃ kṣayaṃ hiṃsām} \\
 & \romline{anapekṣya ca pauruṣam |} \\
 & \romline{mohādā-rabhyate karma} \\
 & \romline{yattattāma-samucyate ||}
\end{tabular}
\end{table}

\begin{table}[H]
\begin{tabular}{cl}
\textbf{18.26} & \romline{mukta-saṅgo'na-haṃvādī} \\
 & \romline{dhṛtyut-sāhasa-manvitaḥ |} \\
 & \romline{siddhya-siddhyor-nirvikāraḥ} \\
 & \romline{kartā sāttvika ucyate ||}
\end{tabular}
\end{table}

\begin{table}[H]
\begin{tabular}{cl}
\textbf{18.27} & \romline{rāgī karma-phala-prepsuḥ} \\
 & \romline{lubdho hiṃsāt-mako'śuciḥ |} \\
 & \romline{harṣa-śokān-vitaḥ kartā} \\
 & \romline{rājasaḥ parikīrtitaḥ ||}
\end{tabular}
\end{table}

\begin{table}[H]
\begin{tabular}{cl}
\textbf{18.28} & \romline{ayuktaḥ prākṛtaḥ stabdhaḥ} \\
 & \romline{śaṭho naiṣkṛti-ko'lasaḥ |} \\
 & \romline{viṣādī dīrgha-sūtrī ca} \\
 & \romline{kartā tāmasa ucyate ||}
\end{tabular}
\end{table}

\begin{table}[H]
\begin{tabular}{cl}
\textbf{18.29} & \romline{buddher-bhedaṃ dhṛteścaiva} \\
 & \romline{guṇata-strividhaṃ śṛṇu |} \\
 & \romline{procya-māna-maśeṣeṇa} \\
 & \romline{pṛthak-tvena dhanañjaya ||}
\end{tabular}
\end{table}

\begin{table}[H]
\begin{tabular}{cl}
\textbf{18.30} & \romline{pravṛttiṃ ca nivṛttiṃ ca} \\
 & \romline{kāryākārye bhayābhaye |} \\
 & \romline{bandhaṃ mokṣaṃ ca yā vetti} \\
 & \romline{buddhiḥ sā pārtha sāttvikī ||}
\end{tabular}
\end{table}

\begin{table}[H]
\begin{tabular}{cl}
\textbf{18.31} & \romline{yayā dharmama-dharmaṃ ca} \\
 & \romline{kāryaṃ cākāryameva ca |} \\
 & \romline{ayathāvat-prajānāti} \\
 & \romline{buddhiḥ sā pārtha rājasī ||}
\end{tabular}
\end{table}

\begin{table}[H]
\begin{tabular}{cl}
\textbf{18.32} & \romline{adharmaṃ dharmamiti yā} \\
 & \romline{manyate tamasā''vṛtā |} \\
 & \romline{sarvārthān-viparītāṃśca} \\
 & \romline{buddhiḥ sā pārtha tāmasī ||}
\end{tabular}
\end{table}

\begin{table}[H]
\begin{tabular}{cl}
\textbf{18.33} & \romline{dhṛtyā yayā dhārayate} \\
 & \romline{manaḥ prāṇendriya-kriyāḥ |} \\
 & \romline{yogenāvya-bhicāriṇyā} \\
 & \romline{dhṛtiḥ sā pārtha sāttvikī ||}
\end{tabular}
\end{table}

\begin{table}[H]
\begin{tabular}{cl}
\textbf{18.34} & \romline{yayā tu dharma-kāmārthān} \\
 & \romline{dhṛtyā dhārayate'rjuna |} \\
 & \romline{prasaṅgena phalākāṅkṣī} \\
 & \romline{dhṛtiḥ sā pārtha rājasī ||}
\end{tabular}
\end{table}

\begin{table}[H]
\begin{tabular}{cl}
\textbf{18.35} & \romline{yayā svapnaṃ bhayaṃ śokaṃ} \\
 & \romline{viṣādaṃ madameva ca |} \\
 & \romline{na vimuñcati durmedhāḥ} \\
 & \romline{dhṛtiḥ sā tāmasī matā ||}
\end{tabular}
\end{table}

\begin{table}[H]
\begin{tabular}{cl}
\textbf{18.36} & \romline{sukhaṃ tvidānīṃ trividhaṃ} \\
 & \romline{śṛṇu me bharatarṣabha |} \\
 & \romline{abhyāsā-dramate yatra} \\
 & \romline{duḥkhāntaṃ ca nigacchati ||}
\end{tabular}
\end{table}

\begin{table}[H]
\begin{tabular}{cl}
\textbf{18.37} & \romline{yattadagre viṣamiva} \\
 & \romline{pariṇāme'mṛtopamam |} \\
 & \romline{tatsukhaṃ sāttvikaṃ proktam} \\
 & \romline{ātma-buddhi-prasādajam ||}
\end{tabular}
\end{table}

\begin{table}[H]
\begin{tabular}{cl}
\textbf{18.38} & \romline{viṣayendriya-saṃyogāt} \\
 & \romline{yattadagre'mṛtopamam |} \\
 & \romline{pariṇāme viṣamiva} \\
 & \romline{tatsukhaṃ rājasaṃ smṛtam ||}
\end{tabular}
\end{table}

\begin{table}[H]
\begin{tabular}{cl}
\textbf{18.39} & \romline{yadagre cānubandhe ca} \\
 & \romline{sukhaṃ mohana-mātmanaḥ |} \\
 & \romline{nidrālasya-pramādotthaṃ} \\
 & \romline{tattāma-samudāhṛtam ||}
\end{tabular}
\end{table}

\begin{table}[H]
\begin{tabular}{cl}
\textbf{18.40} & \romline{na tadasti pṛthivyāṃ vā} \\
 & \romline{divi deveṣu vā punaḥ |} \\
 & \romline{sattvaṃ prakṛti-jairmuktaṃ} \\
 & \romline{yadebhiḥ syāt-tribhir-guṇaiḥ ||}
\end{tabular}
\end{table}

\begin{table}[H]
\begin{tabular}{cl}
\textbf{18.41} & \romline{brāhmaṇa-kṣatriya-viśāṃ} \\
 & \romline{śūdrāṇāṃ ca parantapa |} \\
 & \romline{karmāṇi·pravibhaktāni} \\
 & \romline{svabhāva-prabhavair-guṇaiḥ ||}
\end{tabular}
\end{table}

\begin{table}[H]
\begin{tabular}{cl}
\textbf{18.42} & \romline{śamo damastapaḥ śaucaṃ} \\
 & \romline{ṣānti-rārjavameva ca |} \\
 & \romline{jñānaṃ vijñāna-māstikyaṃ} \\
 & \romline{brahmakarma·svabhāvajam ||}
\end{tabular}
\end{table}

\begin{table}[H]
\begin{tabular}{cl}
\textbf{18.43} & \romline{śauryaṃ tejo dhṛtir-dākṣyaṃ} \\
 & \romline{yuddhe cāpya-palāyanam |} \\
 & \romline{dānamīśvara-bhāvaśca} \\
 & \romline{kṣātraṃ karma·svabhāvajam ||}
\end{tabular}
\end{table}

\begin{table}[H]
\begin{tabular}{cl}
\textbf{18.44} & \romline{kṛṣi-gaura-kṣyavāṇijyaṃ} \\
 & \romline{vaiśyakarma·svabhāvajam |} \\
 & \romline{paricaryāt-makaṃ karma} \\
 & \romline{śūdrasyāpi·svabhāvajam ||}
\end{tabular}
\end{table}

\begin{table}[H]
\begin{tabular}{cl}
\textbf{18.45} & \romline{sve sve karmaṇya-bhirataḥ} \\
 & \romline{saṃsiddhiṃ labhate naraḥ |} \\
 & \romline{svakarma-nirataḥ siddhiṃ} \\
 & \romline{yathā vindati tacchṛṇu ||}
\end{tabular}
\end{table}

\begin{table}[H]
\begin{tabular}{cl}
\textbf{18.46} & \romline{yataḥ pravṛttir-bhūtānāṃ} \\
 & \romline{yena sarvamidaṃ tatam |} \\
 & \romline{svakarmaṇā tamabhyarcya} \\
 & \romline{siddhiṃ vindati mānavaḥ ||}
\end{tabular}
\end{table}

\begin{table}[H]
\begin{tabular}{cl}
\textbf{18.47} & \romline{śreyān-svadharmo viguṇaḥ} \\
 & \romline{paradharmāt-svanuṣṭhitāt |} \\
 & \romline{svabhāva-niyataṃ karma} \\
 & \romline{kurvannāpnoti kilbiṣam ||}
\end{tabular}
\end{table}

\begin{table}[H]
\begin{tabular}{cl}
\textbf{18.48} & \romline{sahajaṃ karma kaunteya} \\
 & \romline{sadoṣamapi na·tyajet |} \\
 & \romline{sarvā-rambhā hi doṣeṇa} \\
 & \romline{dhūmenāgniri-vāvṛtāḥ ||}
\end{tabular}
\end{table}

\begin{table}[H]
\begin{tabular}{cl}
\textbf{18.49} & \romline{asakta-buddhiḥ sarvatra} \\
 & \romline{jitātmā vigata-spṛhaḥ |} \\
 & \romline{naiṣkarmya-siddhiṃ paramāṃ} \\
 & \romline{sannyāsenādhi-gacchati ||}
\end{tabular}
\end{table}

\begin{table}[H]
\begin{tabular}{cl}
\textbf{18.50} & \romline{siddhiṃ prāpto yathā brahma} \\
 & \romline{tathā''pnoti nibodha me |} \\
 & \romline{samāsenaiva kaunteya} \\
 & \romline{niṣṭhā jñānasya yā parā ||}
\end{tabular}
\end{table}

\begin{table}[H]
\begin{tabular}{cl}
\textbf{18.51} & \romline{buddhyā viśuddhayā yuktaḥ} \\
 & \romline{dhṛtyā''tmānaṃ niyamya ca |} \\
 & \romline{śabdādīn-viṣayāṃ-styaktvā} \\
 & \romline{rāga-dveṣau vyudasya ca ||}
\end{tabular}
\end{table}

\begin{table}[H]
\begin{tabular}{cl}
\textbf{18.52} & \romline{vivikta-sevī laghvāśī} \\
 & \romline{yata-vākkāya-mānasaḥ |} \\
 & \romline{dhyāna-yoga-paro nityaṃ} \\
 & \romline{vairāgyaṃ samupāśritaḥ ||}
\end{tabular}
\end{table}

\begin{table}[H]
\begin{tabular}{cl}
\textbf{18.53} & \romline{ahaṅkāraṃ balaṃ darpaṃ} \\
 & \romline{kāmaṃ krodhaṃ pari-graham |} \\
 & \romline{vimucya nirmamaḥ śāntaḥ} \\
 & \romline{brahma-bhūyāya kalpate ||}
\end{tabular}
\end{table}

\begin{table}[H]
\begin{tabular}{cl}
\textbf{18.54} & \romline{brahma-bhūtaḥ prasannātmā} \\
 & \romline{na śocati na kāṅkṣati |} \\
 & \romline{samaḥ sarveṣu bhūteṣu} \\
 & \romline{madbhaktiṃ labhate parām ||}
\end{tabular}
\end{table}

\begin{table}[H]
\begin{tabular}{cl}
\textbf{18.55} & \romline{bhaktyā māma-bhi-jānāti} \\
 & \romline{yāvānyaścāsmi tattvataḥ |} \\
 & \romline{tato māṃ tattvato jñātvā} \\
 & \romline{viśate tada-nantaram ||}
\end{tabular}
\end{table}

\begin{table}[H]
\begin{tabular}{cl}
\textbf{18.56} & \romline{sarva-karmāṇyapi sadā} \\
 & \romline{kurvāṇo madvya-pāśrayaḥ |} \\
 & \romline{mat-prasādā-davāpnoti} \\
 & \romline{śāśvataṃ pada-mavyayam ||}
\end{tabular}
\end{table}

\begin{table}[H]
\begin{tabular}{cl}
\textbf{18.57} & \romline{cetasā sarva-karmāṇi} \\
 & \romline{mayi sannyasya matparaḥ |} \\
 & \romline{buddhi-yogamu-pāśritya} \\
 & \romline{maccittaḥ satataṃ bhava ||}
\end{tabular}
\end{table}

\begin{table}[H]
\begin{tabular}{cl}
\textbf{18.58} & \romline{maccittaḥ sarva-durgāṇi} \\
 & \romline{matprasādāt-tariṣyasi |} \\
 & \romline{atha cettva-mahaṅkārāt} \\
 & \romline{na·śroṣyasi vinaṅkṣyasi ||}
\end{tabular}
\end{table}

\begin{table}[H]
\begin{tabular}{cl}
\textbf{18.59} & \romline{yadahaṅkāra-māśritya} \\
 & \romline{na yotsya iti manyase |} \\
 & \romline{mithyaiṣa vyava-sāyaste} \\
 & \romline{prakṛtistvāṃ niyokṣyati ||}
\end{tabular}
\end{table}

\begin{table}[H]
\begin{tabular}{cl}
\textbf{18.60} & \romline{svabhāva-jena kaunteya} \\
 & \romline{nibaddhaḥ svena karmaṇā |} \\
 & \romline{kartuṃ necchasi yanmohāt} \\
 & \romline{kariṣya-syavaśo'pi tat ||}
\end{tabular}
\end{table}

\begin{table}[H]
\begin{tabular}{cl}
\textbf{18.61} & \romline{īśvaraḥ sarva-bhūtānāṃ} \\
 & \romline{hṛdde-śe'rjuna tiṣṭhati |} \\
 & \romline{bhrāmayan-sarva-bhūtāni} \\
 & \romline{yantrā-rūḍhāni māyayā ||}
\end{tabular}
\end{table}

\begin{table}[H]
\begin{tabular}{cl}
\textbf{18.62} & \romline{tameva śaraṇaṃ gaccha} \\
 & \romline{sarva-bhāvena bhārata |} \\
 & \romline{tatprasādāt-parāṃ śāntiṃ} \\
 & \romline{sthānaṃ prāpsyasi śāśvatam ||}
\end{tabular}
\end{table}

\begin{table}[H]
\begin{tabular}{cl}
\textbf{18.63} & \romline{iti te jñāna-mākhyātaṃ} \\
 & \romline{guhyād-guhya-taraṃ mayā |} \\
 & \romline{vimṛśyai-tadaśeṣeṇa} \\
 & \romline{yathecchasi tathā kuru ||}
\end{tabular}
\end{table}

\begin{table}[H]
\begin{tabular}{cl}
\textbf{18.64} & \romline{sarva-guhya-tamaṃ bhūyaḥ} \\
 & \romline{śṛṇu me paramaṃ vacaḥ |} \\
 & \romline{iṣṭo'si me dṛḍhamiti} \\
 & \romline{tato vakṣyāmi te hitam ||}
\end{tabular}
\end{table}

\begin{table}[H]
\begin{tabular}{cl}
\textbf{18.65} & \romline{manmanā bhava madbhaktaḥ} \\
 & \romline{madyājī māṃ namaskuru |} \\
 & \romline{māmevaiṣyasi satyaṃ te} \\
 & \romline{pratijāne priyo'si me ||}
\end{tabular}
\end{table}

\begin{table}[H]
\begin{tabular}{cl}
\textbf{18.66} & \romline{sarva-dharmān-parityajya} \\
 & \romline{māmekaṃ śaraṇaṃ vraja |} \\
 & \romline{ahaṃ tvā sarva-pāpebhyaḥ} \\
 & \romline{mokṣa-yiṣyāmi mā śucaḥ ||}
\end{tabular}
\end{table}

\begin{table}[H]
\begin{tabular}{cl}
\textbf{18.67} & \romline{idaṃ te nāta-paskāya} \\
 & \romline{nābhaktāya kadācana |} \\
 & \romline{na cāśuś-rūṣave vācyaṃ} \\
 & \romline{na ca māṃ yo'bhya-sūyati ||}
\end{tabular}
\end{table}

\begin{table}[H]
\begin{tabular}{cl}
\textbf{18.68} & \romline{ya imaṃ paramaṃ guhyaṃ} \\
 & \romline{madbhak-teṣva-bhidhāsyati |} \\
 & \romline{bhaktiṃ mayi parāṃ kṛtvā} \\
 & \romline{māmevaiṣya-tya-saṃśayaḥ ||}
\end{tabular}
\end{table}

\begin{table}[H]
\begin{tabular}{cl}
\textbf{18.69} & \romline{na ca tasmān-manuṣyeṣu} \\
 & \romline{kaścinme priya-kṛttamaḥ |} \\
 & \romline{bhavitā na ca me tasmāt} \\
 & \romline{anyaḥ priya-taro bhuvi ||}
\end{tabular}
\end{table}

\begin{table}[H]
\begin{tabular}{cl}
\textbf{18.70} & \romline{adhyeṣyate ca ya imaṃ} \\
 & \romline{dharmyaṃ saṃvāda-māvayoḥ |} \\
 & \romline{jñāna-yajñena tenāham} \\
 & \romline{iṣṭaḥ syāmiti me matiḥ ||}
\end{tabular}
\end{table}

\begin{table}[H]
\begin{tabular}{cl}
\textbf{18.71} & \romline{śraddhāvā-nana-sūyaśca} \\
 & \romline{śṛṇu-yādapi yo naraḥ |} \\
 & \romline{so'pi muktaḥ śubhāllokān} \\
 & \romline{prāpnuyāt-puṇya-karmaṇām ||}
\end{tabular}
\end{table}

\begin{table}[H]
\begin{tabular}{cl}
\textbf{18.72} & \romline{kaccideta-cchrutaṃ pārtha} \\
 & \romline{tvayai-kāgreṇa cetasā |} \\
 & \romline{kaccida-jñāna-sammohaḥ} \\
 & \romline{pranaṣṭaste dhanañjaya ||}
\end{tabular}
\end{table}

\begin{table}[H]
\begin{tabular}{cl}
\textbf{18.73} & \romline{arjuna uvāca} \\
 & \romline{naṣṭo mohaḥ smṛtir-labdhā} \\
 & \romline{tvat-prasādān-mayā'cyuta |} \\
 & \romline{sthito'smi gata-sandehaḥ} \\
 & \romline{kariṣye vacanaṃ tava ||}
\end{tabular}
\end{table}

\begin{table}[H]
\begin{tabular}{cl}
\textbf{18.74} & \romline{sañjaya uvāca} \\
 & \romline{ityahaṃ vāsudevasya} \\
 & \romline{pārthasya ca mahātmanaḥ |} \\
 & \romline{saṃvāda-mima-maśrauṣam} \\
 & \romline{adbhutaṃ roma-harṣaṇam ||}
\end{tabular}
\end{table}

\begin{table}[H]
\begin{tabular}{cl}
\textbf{18.75} & \romline{vyāsa-prasādācchru-tavān} \\
 & \romline{imaṃ guhya-tamaṃ param |} \\
 & \romline{yogaṃ yogeśvarāt-kṛṣṇāt} \\
 & \romline{sākṣāt-kathayataḥ svayam ||}
\end{tabular}
\end{table}

\begin{table}[H]
\begin{tabular}{cl}
\textbf{18.76} & \romline{rājan saṃsmṛtya saṃsmṛtya} \\
 & \romline{saṃvāda-mima-madbhutam |} \\
 & \romline{keśavārjuna-yoḥ puṇyaṃ} \\
 & \romline{hṛṣyāmi ca muhur-muhuḥ ||}
\end{tabular}
\end{table}

\begin{table}[H]
\begin{tabular}{cl}
\textbf{18.77} & \romline{tacca saṃsmṛtya saṃsmṛtya} \\
 & \romline{rūpa-matyad-bhutaṃ hareḥ |} \\
 & \romline{vismayo me mahān-rājan} \\
 & \romline{hṛṣyāmi ca punaḥ punaḥ ||}
\end{tabular}
\end{table}

\begin{table}[H]
\begin{tabular}{cl}
\textbf{18.78} & \romline{yatra yogeśvaraḥ kṛṣṇaḥ} \\
 & \romline{yatra pārtho dhanur-dharaḥ |} \\
 & \romline{tatra śrīrvijayo bhūtiḥ} \\
 & \romline{dhruvā nītir-matirmama ||}
\end{tabular}
\end{table}

