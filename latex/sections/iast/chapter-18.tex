\begin{table}[H]
\begin{tabular}{cl}
\textbf{18.0} & \romline{oṃ śrī paramātmane namaḥ} \\
 & \romline{atha aṣṭādaśo'dhyāyaḥ} \\
 & \romline{mokṣa-sannyāsa yogaḥ}
\end{tabular}
\end{table}

\begin{table}[H]
\begin{tabular}{cl}
\textbf{18.1} & \romline{arjuna uvāca} \\
 & \romline{sannyāsasya mahābāho} \\
 & \romline{tattva-micchāmi veditum |} \\
 & \romline{tyāgasya ca hṛṣikeśa} \\
 & \romline{pṛthak-keśiniṣūdana ||}
\end{tabular}
\end{table}

\begin{table}[H]
\begin{tabular}{cl}
\textbf{18.2} & \romline{śrī bhagavanuvāca} \\
 & \romline{kāmyānāṃ karmaṇāṃ nyāsaṃ} \\
 & \romline{sannyāsaṃ kavayo viduḥ |} \\
 & \romline{sarva-karma-phala-tyāgaṃ} \\
 & \romline{prāhus-tyāgaṃ vicakṣaṇāḥ ||}
\end{tabular}
\end{table}

\begin{table}[H]
\begin{tabular}{cl}
\textbf{18.3} & \romline{tyājyaṃ doṣa-vadityeke} \\
 & \romline{karma·prāhur-manīṣiṇaḥ |} \\
 & \romline{yajña-dāna-tapaḥ karma} \\
 & \romline{na·tyājyamiti cāpare ||}
\end{tabular}
\end{table}

\begin{table}[H]
\begin{tabular}{cl}
\textbf{18.4} & \romline{niścayaṃ śṛṇu me tatra} \\
 & \romline{tyāge bharatasattama |} \\
 & \romline{tyāgo hi puruṣa-vyāghra} \\
 & \romline{trividhaḥ samprakīrtitaḥ ||}
\end{tabular}
\end{table}

\begin{table}[H]
\begin{tabular}{cl}
\textbf{18.5} & \romline{yajña-dāna-tapaḥ-karma} \\
 & \romline{na tyājyaṃ kāryameva tat |} \\
 & \romline{yajño dānaṃ tapaścaiva} \\
 & \romline{pāvanāni manīṣiṇām ||}
\end{tabular}
\end{table}

\begin{table}[H]
\begin{tabular}{cl}
\textbf{18.6} & \romline{etānyapi tu karmāṇi} \\
 & \romline{saṅgaṃ tyaktvā phalāni ca |} \\
 & \romline{karta-vyānīti me pārtha} \\
 & \romline{niścitaṃ matamuttamam ||}
\end{tabular}
\end{table}

\begin{table}[H]
\begin{tabular}{cl}
\textbf{18.7} & \romline{niyatasya tu sannyāsaḥ} \\
 & \romline{karmaṇo nopa-padyate |} \\
 & \romline{mohā-ttasya parityāgaḥ} \\
 & \romline{tāmasaḥ parikīrtitaḥ ||}
\end{tabular}
\end{table}

\begin{table}[H]
\begin{tabular}{cl}
\textbf{18.8} & \romline{duḥkha-mityeva yatkarma} \\
 & \romline{kāya-kleśa-bhayāttyajet |} \\
 & \romline{sa kṛtvā rājasaṃ tyāgaṃ} \\
 & \romline{naiva·tyāga-phalaṃ labhet ||}
\end{tabular}
\end{table}

\begin{table}[H]
\begin{tabular}{cl}
\textbf{18.9} & \romline{kāryamityeva yatkarma} \\
 & \romline{niyataṃ kriyate'rjuna |} \\
 & \romline{saṅgaṃ tyaktvā phalaṃ caiva} \\
 & \romline{sa tyāgaḥ sāttviko mataḥ ||}
\end{tabular}
\end{table}

\begin{table}[H]
\begin{tabular}{cl}
\textbf{18.10} & \romline{na dveṣṭya-kuśalaṃ karma} \\
 & \romline{kuśale nānuṣajjate |} \\
 & \romline{tyāgī sattva-samāviṣṭaḥ} \\
 & \romline{medhāvī chinna-saṃśayaḥ ||}
\end{tabular}
\end{table}

\begin{table}[H]
\begin{tabular}{cl}
\textbf{18.11} & \romline{na hi deha-bhṛtā śakyaṃ} \\
 & \romline{tyaktuṃ karmāṇya-śeṣataḥ |} \\
 & \romline{yastu karma-phala-tyāgī} \\
 & \romline{sa tyāgītyabhi-dhīyate ||}
\end{tabular}
\end{table}

\begin{table}[H]
\begin{tabular}{cl}
\textbf{18.12} & \romline{aniṣṭa-miṣṭaṃ miśraṃ ca} \\
 & \romline{trividhaṃ karmaṇaḥ phalam |} \\
 & \romline{bhavatya-tyāgināṃ pretya} \\
 & \romline{na tu sannyāsināṃ kvacit ||}
\end{tabular}
\end{table}

\begin{table}[H]
\begin{tabular}{cl}
\textbf{18.13} & \romline{pañcaitāni mahābāho} \\
 & \romline{kāraṇāni nibodha me |} \\
 & \romline{sāṅkhye kṛtānte proktāni} \\
 & \romline{siddhaye sarva-karmaṇām ||}
\end{tabular}
\end{table}

\begin{table}[H]
\begin{tabular}{cl}
\textbf{18.14} & \romline{adhiṣṭhānaṃ tathā kartā} \\
 & \romline{karaṇaṃ ca pṛthag-vidham |} \\
 & \romline{vividhāśca pṛthak-ceṣṭāḥ} \\
 & \romline{daivaṃ caivātra pañcamam ||}
\end{tabular}
\end{table}

\begin{table}[H]
\begin{tabular}{cl}
\textbf{18.15} & \romline{śarīravāṅ-manobhiryat} \\
 & \romline{karma·prārabhate naraḥ |} \\
 & \romline{nyāyyaṃ vā viparītaṃ vā} \\
 & \romline{pañcaite tasya hetavaḥ ||}
\end{tabular}
\end{table}

\begin{table}[H]
\begin{tabular}{cl}
\textbf{18.16} & \romline{tatraivaṃ sati kartāram} \\
 & \romline{ātmānaṃ kevalaṃ tu yaḥ |} \\
 & \romline{paśyatya-kṛta-buddhi-tvāt} \\
 & \romline{na sa paśyati durmatiḥ ||}
\end{tabular}
\end{table}

\begin{table}[H]
\begin{tabular}{cl}
\textbf{18.17} & \romline{yasya nāhaṅkṛto bhāvaḥ} \\
 & \romline{buddhir-yasya na lipyate |} \\
 & \romline{hatvā'pi sa imāllokān} \\
 & \romline{na hanti na nibadhyate ||}
\end{tabular}
\end{table}

\begin{table}[H]
\begin{tabular}{cl}
\textbf{18.18} & \romline{jñānaṃ jñeyaṃ parijñātā} \\
 & \romline{trividhā karmacodanā |} \\
 & \romline{karaṇaṃ karma karteti} \\
 & \romline{trividhaḥ karma-saṅgrahaḥ ||}
\end{tabular}
\end{table}

\begin{table}[H]
\begin{tabular}{cl}
\textbf{18.19} & \romline{jñānaṃ karma ca kartā ca} \\
 & \romline{tridhaiva guṇa-bhedataḥ |} \\
 & \romline{procyate guṇa-saṅkhyāne} \\
 & \romline{yathā-vacchṛṇu tānyapi ||}
\end{tabular}
\end{table}

\begin{table}[H]
\begin{tabular}{cl}
\textbf{18.20} & \romline{sarvabhūteṣu yenaikaṃ} \\
 & \romline{bhāvamavyayamīkṣate |} \\
 & \romline{avibhaktaṃ vibhakteṣu} \\
 & \romline{taj-jñānaṃ viddhi sāttvikam ||}
\end{tabular}
\end{table}

\begin{table}[H]
\begin{tabular}{cl}
\textbf{18.21} & \romline{pṛthaktvena tu yaj-jñānaṃ} \\
 & \romline{nānābhāvān pṛthagvidhān |} \\
 & \romline{vetti sarveṣu bhūteṣu} \\
 & \romline{taj-jñānaṃ viddhi rājasam ||}
\end{tabular}
\end{table}

\begin{table}[H]
\begin{tabular}{cl}
\textbf{18.22} & \romline{yattu kṛtsnavadekasmin} \\
 & \romline{kārye saktamahaitukam |} \\
 & \romline{atattvārthavadalpaṃ ca} \\
 & \romline{tattāmasamudāhṛtam ||}
\end{tabular}
\end{table}

\begin{table}[H]
\begin{tabular}{cl}
\textbf{18.23} & \romline{niyataṃ saṅgarahitam} \\
 & \romline{arāgadveṣataḥ kṛtam |} \\
 & \romline{aphalaprepsunā karma} \\
 & \romline{yattatsāttvikamucyate ||}
\end{tabular}
\end{table}

\begin{table}[H]
\begin{tabular}{cl}
\textbf{18.24} & \romline{yattu kāmepsunā karma} \\
 & \romline{sāhaṅkāreṇa vā punaḥ |} \\
 & \romline{kriyate bahulāyāsaṃ} \\
 & \romline{tadrājasamudāhṛtam ||}
\end{tabular}
\end{table}

\begin{table}[H]
\begin{tabular}{cl}
\textbf{18.25} & \romline{anubandhaṃ kṣayaṃ hiṃsām} \\
 & \romline{anapekṣya ca pauruṣam |} \\
 & \romline{mohādārabhyate karma} \\
 & \romline{yattattāmasamucyate ||}
\end{tabular}
\end{table}

\begin{table}[H]
\begin{tabular}{cl}
\textbf{18.26} & \romline{muktasaṅgo'nahaṃvādī} \\
 & \romline{dhṛtyutsāhasamanvitaḥ |} \\
 & \romline{siddhyasiddhyornirvikāraḥ} \\
 & \romline{kartā sāttvika ucyate ||}
\end{tabular}
\end{table}

\begin{table}[H]
\begin{tabular}{cl}
\textbf{18.27} & \romline{rāgī karmaphalaprepsuḥ} \\
 & \romline{lubdho hiṃsātmako'śuciḥ |} \\
 & \romline{harṣaśokānvitaḥ kartā} \\
 & \romline{rājasaḥ parikīrtitaḥ ||}
\end{tabular}
\end{table}

\begin{table}[H]
\begin{tabular}{cl}
\textbf{18.28} & \romline{ayuktaḥ prākṛtaḥ stabdhaḥ} \\
 & \romline{śaṭho naiṣkṛtiko'lasaḥ |} \\
 & \romline{viṣādī dīrghasūtrī ca} \\
 & \romline{kartā tāmasa ucyate ||}
\end{tabular}
\end{table}

\begin{table}[H]
\begin{tabular}{cl}
\textbf{18.29} & \romline{buddherbhedaṃ dhṛteścaiva} \\
 & \romline{guṇatastrividhaṃ śṛṇu |} \\
 & \romline{procyamānamaśeṣeṇa} \\
 & \romline{pṛthaktvena dhanañjaya ||}
\end{tabular}
\end{table}

\begin{table}[H]
\begin{tabular}{cl}
\textbf{18.30} & \romline{pravṛttiṃ ca nivṛttiṃ ca} \\
 & \romline{kāryākārye bhayābhaye |} \\
 & \romline{bandhaṃ mokṣaṃ ca yā vetti} \\
 & \romline{buddhiḥ sā pārtha sāttvikī ||}
\end{tabular}
\end{table}

\begin{table}[H]
\begin{tabular}{cl}
\textbf{18.31} & \romline{yayā dharmamadharmaṃ ca} \\
 & \romline{kāryaṃ cākāryameva ca |} \\
 & \romline{ayathāvatprajānāti} \\
 & \romline{buddhiḥ sā pārtha rājasī ||}
\end{tabular}
\end{table}

\begin{table}[H]
\begin{tabular}{cl}
\textbf{18.32} & \romline{adharmaṃ dharmamiti yā} \\
 & \romline{manyate tamasā''vṛtā |} \\
 & \romline{sarvārthānviparītāṃśca} \\
 & \romline{buddhiḥ sā pārtha tāmasī ||}
\end{tabular}
\end{table}

\begin{table}[H]
\begin{tabular}{cl}
\textbf{18.33} & \romline{dhṛtyā yayā dhārayate} \\
 & \romline{manaḥ prāṇendriyakriyāḥ |} \\
 & \romline{yogenāvyabhicāriṇyā} \\
 & \romline{dhṛtiḥ sā pārtha sāttvikī ||}
\end{tabular}
\end{table}

\begin{table}[H]
\begin{tabular}{cl}
\textbf{18.34} & \romline{yayā tu dharmakāmārthān} \\
 & \romline{dhṛtyā dhārayate'rjuna |} \\
 & \romline{prasaṅgena phalākāṅkṣī} \\
 & \romline{dhṛtiḥ sā pārtha rājasī ||}
\end{tabular}
\end{table}

\begin{table}[H]
\begin{tabular}{cl}
\textbf{18.35} & \romline{yayā svapnaṃ bhayaṃ śokaṃ} \\
 & \romline{viṣādaṃ madameva ca |} \\
 & \romline{na vimuñcati durmedhāḥ} \\
 & \romline{dhṛtiḥ sā tāmasī matā ||}
\end{tabular}
\end{table}

\begin{table}[H]
\begin{tabular}{cl}
\textbf{18.36} & \romline{sukhaṃ tvidānīṃ trividhaṃ} \\
 & \romline{śṛṇu me bharatarṣabha |} \\
 & \romline{abhyāsādramate yatra} \\
 & \romline{duḥkhāntaṃ ca nigacchati ||}
\end{tabular}
\end{table}

\begin{table}[H]
\begin{tabular}{cl}
\textbf{18.37} & \romline{yattadagre viṣamiva} \\
 & \romline{pariṇāme'mṛtopamam |} \\
 & \romline{tatsukhaṃ sāttvikaṃ proktam} \\
 & \romline{ātma-buddhi-prasādajam ||}
\end{tabular}
\end{table}

\begin{table}[H]
\begin{tabular}{cl}
\textbf{18.38} & \romline{viṣayendriyasaṃyogāt} \\
 & \romline{yattadagre'mṛtopamam |} \\
 & \romline{pariṇāme viṣamiva} \\
 & \romline{tatsukhaṃ rājasaṃ smṛtam ||}
\end{tabular}
\end{table}

\begin{table}[H]
\begin{tabular}{cl}
\textbf{18.39} & \romline{yadagre cānubandhe ca} \\
 & \romline{sukhaṃ mohana-mātmanaḥ |} \\
 & \romline{nidrālasya-pramādotthaṃ} \\
 & \romline{tattāma-samudāhṛtam ||}
\end{tabular}
\end{table}

\begin{table}[H]
\begin{tabular}{cl}
\textbf{18.40} & \romline{na tadasti pṛthivyāṃ vā} \\
 & \romline{divi deveṣu vā punaḥ |} \\
 & \romline{sattvaṃ prakṛti-jairmuktaṃ} \\
 & \romline{yadebhiḥ syāttri-bhir-guṇaiḥ ||}
\end{tabular}
\end{table}

\begin{table}[H]
\begin{tabular}{cl}
\textbf{18.41} & \romline{brāhmaṇa-kṣatriya-viśāṃ} \\
 & \romline{śūdrāṇāṃ ca parantapa |} \\
 & \romline{karmāṇi·pravibhaktāni} \\
 & \romline{svabhāva-prabhavair-guṇaiḥ ||}
\end{tabular}
\end{table}

\begin{table}[H]
\begin{tabular}{cl}
\textbf{18.42} & \romline{śamo damastapaḥ śaucaṃ} \\
 & \romline{ṣānti-rārjavameva ca |} \\
 & \romline{jñānaṃ vijñāna-māstikyaṃ} \\
 & \romline{brahmakarma·svabhāvajam ||}
\end{tabular}
\end{table}

\begin{table}[H]
\begin{tabular}{cl}
\textbf{18.43} & \romline{śauryaṃ tejo dhṛtir-dākṣyaṃ} \\
 & \romline{yuddhe cāpya-palāyanam |} \\
 & \romline{dānamīśvara-bhāvaśca} \\
 & \romline{kṣātraṃ karma·svabhāvajam ||}
\end{tabular}
\end{table}

