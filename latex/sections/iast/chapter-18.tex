\begin{table}[H]
\begin{tabular}{cl}
 & \romline{śrī paramātmane namaḥ} \\
 & \romline{atha aṣṭādaśo'dhyāyaḥ} \\
 & \romline{mokṣasannyāsayogaḥ}
\end{tabular}
\end{table}

\begin{table}[H]
\begin{tabular}{cl}
\textbf{18.1} & \romline{arjuna uvāca} \\
 & \romline{sannyāsasya mahābāho} \\
 & \romline{tattvamicchāmi veditum |} \\
 & \romline{tyāgasya ca hṛṣikeśa} \\
 & \romline{pṛthakkeśiniṣūdana ||}
\end{tabular}
\end{table}

\begin{table}[H]
\begin{tabular}{cl}
\textbf{18.2} & \romline{śrī bhagavanuvāca} \\
 & \romline{kāmyānāṃ karmaṇāṃ nyāsaṃ} \\
 & \romline{sannyāsaṃ kavayo viduḥ |} \\
 & \romline{sarvakarmaphalatyāgaṃ} \\
 & \romline{prāhustyāgaṃ vicakṣaṇāḥ ||}
\end{tabular}
\end{table}

\begin{table}[H]
\begin{tabular}{cl}
\textbf{18.3} & \romline{tyājyaṃ doṣavadityeke} \\
 & \romline{karma prāhurmanīṣiṇaḥ |} \\
 & \romline{yajñadānatapaḥ karma} \\
 & \romline{na tyājyamiti cāpare ||}
\end{tabular}
\end{table}

\begin{table}[H]
\begin{tabular}{cl}
\textbf{18.4} & \romline{niścayaṃ śṛṇu me tatra} \\
 & \romline{tyāge bharatasattama |} \\
 & \romline{tyāgo hi puruṣavyāghra} \\
 & \romline{trividhaḥ samprakīrtitaḥ ||}
\end{tabular}
\end{table}

\begin{table}[H]
\begin{tabular}{cl}
\textbf{18.5} & \romline{yajñadānatapaḥkarma} \\
 & \romline{na tyājyaṃ kāryameva tat |} \\
 & \romline{yajño dānaṃ tapaścaiva} \\
 & \romline{pāvanāni manīṣiṇām ||}
\end{tabular}
\end{table}

\begin{table}[H]
\begin{tabular}{cl}
\textbf{18.6} & \romline{etānyapi tu karmāṇi} \\
 & \romline{saṅgaṃ tyaktvā phalāni ca |} \\
 & \romline{kartavyānīti me pārtha} \\
 & \romline{niścitaṃ matamuttamam ||}
\end{tabular}
\end{table}

\begin{table}[H]
\begin{tabular}{cl}
\textbf{18.7} & \romline{niyatasya tu sannyāsaḥ} \\
 & \romline{karmaṇo nopapadyate |} \\
 & \romline{mohāttasya parityāgaḥ} \\
 & \romline{tāmasaḥ parikīrtitaḥ ||}
\end{tabular}
\end{table}

\begin{table}[H]
\begin{tabular}{cl}
\textbf{18.8} & \romline{duḥkhamityeva yatkarma} \\
 & \romline{kāyakleśabhayāttyajet |} \\
 & \romline{sa kṛtvā rājasaṃ tyāgaṃ} \\
 & \romline{naiva tyāgaphalaṃ labhet ||}
\end{tabular}
\end{table}

\begin{table}[H]
\begin{tabular}{cl}
\textbf{18.9} & \romline{kāryamityeva yatkarma} \\
 & \romline{niyataṃ kriyate'rjuna |} \\
 & \romline{saṅgaṃ tyaktvā phalaṃ caiva} \\
 & \romline{sa tyāgaḥ sāttviko mataḥ ||}
\end{tabular}
\end{table}

\begin{table}[H]
\begin{tabular}{cl}
\textbf{18.10} & \romline{na dveṣṭyakuśalaṃ karma} \\
 & \romline{kuśale nānuṣajjate |} \\
 & \romline{tyāgī sattvasamāviṣṭaḥ} \\
 & \romline{medhāvī chinnasaṃśayaḥ ||}
\end{tabular}
\end{table}

\begin{table}[H]
\begin{tabular}{cl}
\textbf{18.11} & \romline{na hi dehabhṛtā śakyaṃ} \\
 & \romline{tyaktuṃ karmāṇyaśeṣataḥ |} \\
 & \romline{yastu karmaphalatyāgī} \\
 & \romline{sa tyāgītyabhidhīyate ||}
\end{tabular}
\end{table}

\begin{table}[H]
\begin{tabular}{cl}
\textbf{18.12} & \romline{aniṣṭamiṣṭaṃ miśraṃ ca} \\
 & \romline{trividhaṃ karmaṇaḥ phalam |} \\
 & \romline{bhavatyatyāgināṃ pretya} \\
 & \romline{na tu sannyāsināṃ kvacit ||}
\end{tabular}
\end{table}

\begin{table}[H]
\begin{tabular}{cl}
\textbf{18.13} & \romline{pañcaitāni mahābāho} \\
 & \romline{kāraṇāni nibodha me |} \\
 & \romline{sāṅkhye kṛtānte proktāni} \\
 & \romline{siddhaye sarvakarmaṇām ||}
\end{tabular}
\end{table}

\begin{table}[H]
\begin{tabular}{cl}
\textbf{18.14} & \romline{adhiṣṭhānaṃ tathā kartā} \\
 & \romline{karaṇaṃ ca pṛthagvidham |} \\
 & \romline{vividhāśca pṛthakceṣṭāḥ} \\
 & \romline{daivaṃ caivātra pañcamam ||}
\end{tabular}
\end{table}

\begin{table}[H]
\begin{tabular}{cl}
\textbf{18.15} & \romline{śarīravāṅmanobhiryat} \\
 & \romline{karma prārabhate naraḥ |} \\
 & \romline{nyāyyaṃ vā viparītaṃ vā} \\
 & \romline{pañcaite tasya hetavaḥ ||}
\end{tabular}
\end{table}

\begin{table}[H]
\begin{tabular}{cl}
\textbf{18.16} & \romline{tatraivaṃ sati kartāram} \\
 & \romline{ātmānaṃ kevalaṃ tu yaḥ |} \\
 & \romline{paśyatyakṛtabuddhitvāt} \\
 & \romline{na sa paśyati durmatiḥ ||}
\end{tabular}
\end{table}

\begin{table}[H]
\begin{tabular}{cl}
\textbf{18.17} & \romline{yasya nāhaṅkṛto bhāvaḥ} \\
 & \romline{buddhiryasya na lipyate |} \\
 & \romline{hatvā'pi sa imāllokān} \\
 & \romline{na hanti na nibadhyate ||}
\end{tabular}
\end{table}

\begin{table}[H]
\begin{tabular}{cl}
\textbf{18.18} & \romline{jñānaṃ jñeyaṃ parijñātā} \\
 & \romline{trividhā karmacodanā |} \\
 & \romline{karaṇaṃ karma karteti} \\
 & \romline{trividhaḥ karmasaṅgrahaḥ ||}
\end{tabular}
\end{table}

\begin{table}[H]
\begin{tabular}{cl}
\textbf{18.19} & \romline{jñānaṃ karma ca kartā ca} \\
 & \romline{tridhaiva guṇabhedataḥ |} \\
 & \romline{procyate guṇasaṅkhyāne} \\
 & \romline{yathāvacchṛṇu tānyapi ||}
\end{tabular}
\end{table}

\begin{table}[H]
\begin{tabular}{cl}
\textbf{18.20} & \romline{sarvabhūteṣu yenaikaṃ} \\
 & \romline{bhāvamavyayamīkṣate |} \\
 & \romline{avibhaktaṃ vibhakteṣu} \\
 & \romline{tajjñānaṃ viddhi sāttvikam ||}
\end{tabular}
\end{table}

\begin{table}[H]
\begin{tabular}{cl}
\textbf{18.21} & \romline{pṛthaktvena tu yajjñānaṃ} \\
 & \romline{nānābhāvān pṛthagvidhān |} \\
 & \romline{vetti sarveṣu bhūteṣu} \\
 & \romline{tajjñānaṃ viddhi rājasam ||}
\end{tabular}
\end{table}

\begin{table}[H]
\begin{tabular}{cl}
\textbf{18.22} & \romline{yattu kṛtsnavadekasmin} \\
 & \romline{kārye saktamahaitukam |} \\
 & \romline{atattvārthavadalpaṃ ca} \\
 & \romline{tattāmasamudāhṛtam ||}
\end{tabular}
\end{table}

\begin{table}[H]
\begin{tabular}{cl}
\textbf{18.23} & \romline{niyataṃ saṅgarahitam} \\
 & \romline{arāgadveṣataḥ kṛtam |} \\
 & \romline{aphalaprepsunā karma} \\
 & \romline{yattatsāttvikamucyate ||}
\end{tabular}
\end{table}

\begin{table}[H]
\begin{tabular}{cl}
\textbf{18.24} & \romline{yattu kāmepsunā karma} \\
 & \romline{sāhaṅkāreṇa vā punaḥ |} \\
 & \romline{kriyate bahulāyāsaṃ} \\
 & \romline{tadrājasamudāhṛtam ||}
\end{tabular}
\end{table}

\begin{table}[H]
\begin{tabular}{cl}
\textbf{18.25} & \romline{anubandhaṃ kṣayaṃ hiṃsām} \\
 & \romline{anapekṣya ca pauruṣam |} \\
 & \romline{mohādārabhyate karma} \\
 & \romline{yattattāmasamucyate ||}
\end{tabular}
\end{table}

\begin{table}[H]
\begin{tabular}{cl}
\textbf{18.26} & \romline{muktasaṅgo'nahaṃvādī} \\
 & \romline{dhṛtyutsāhasamanvitaḥ |} \\
 & \romline{siddhyasiddhyornirvikāraḥ} \\
 & \romline{kartā sāttvika ucyate ||}
\end{tabular}
\end{table}

\begin{table}[H]
\begin{tabular}{cl}
\textbf{18.27} & \romline{rāgī karmaphalaprepsuḥ} \\
 & \romline{lubdho hiṃsātmako'śuciḥ |} \\
 & \romline{harṣaśokānvitaḥ kartā} \\
 & \romline{rājasaḥ parikīrtitaḥ ||}
\end{tabular}
\end{table}

\begin{table}[H]
\begin{tabular}{cl}
\textbf{18.28} & \romline{ayuktaḥ prākṛtaḥ stabdhaḥ} \\
 & \romline{śaṭho naiṣkṛtiko'lasaḥ |} \\
 & \romline{viṣādī dīrghasūtrī ca} \\
 & \romline{kartā tāmasa ucyate ||}
\end{tabular}
\end{table}

\begin{table}[H]
\begin{tabular}{cl}
\textbf{18.29} & \romline{buddherbhedaṃ dhṛteścaiva} \\
 & \romline{guṇatastrividhaṃ śṛṇu |} \\
 & \romline{procyamānamaśeṣeṇa} \\
 & \romline{pṛthaktvena dhanañjaya ||}
\end{tabular}
\end{table}

\begin{table}[H]
\begin{tabular}{cl}
\textbf{18.30} & \romline{pravṛttiṃ ca nivṛttiṃ ca} \\
 & \romline{kāryākārye bhayābhaye |} \\
 & \romline{bandhaṃ mokṣaṃ ca yā vetti} \\
 & \romline{buddhiḥ sā pārtha sāttvikī ||}
\end{tabular}
\end{table}

\begin{table}[H]
\begin{tabular}{cl}
\textbf{18.31} & \romline{yayā dharmamadharmaṃ ca} \\
 & \romline{kāryaṃ cākāryameva ca |} \\
 & \romline{ayathāvatprajānāti} \\
 & \romline{buddhiḥ sā pārtha rājasī ||}
\end{tabular}
\end{table}

\begin{table}[H]
\begin{tabular}{cl}
\textbf{18.32} & \romline{adharmaṃ dharmamiti yā} \\
 & \romline{manyate tamasā''vṛtā |} \\
 & \romline{sarvārthānviparītāṃśca} \\
 & \romline{buddhiḥ sā pārtha tāmasī ||}
\end{tabular}
\end{table}

\begin{table}[H]
\begin{tabular}{cl}
\textbf{18.33} & \romline{dhṛtyā yayā dhārayate} \\
 & \romline{manaḥ prāṇendriyakriyāḥ |} \\
 & \romline{yogenāvyabhicāriṇyā} \\
 & \romline{dhṛtiḥ sā pārtha sāttvikī ||}
\end{tabular}
\end{table}

\begin{table}[H]
\begin{tabular}{cl}
\textbf{18.34} & \romline{yayā tu dharmakāmārthān} \\
 & \romline{dhṛtyā dhārayate'rjuna |} \\
 & \romline{prasaṅgena phalākāṅkṣī} \\
 & \romline{dhṛtiḥ sā pārtha rājasī ||}
\end{tabular}
\end{table}

\begin{table}[H]
\begin{tabular}{cl}
\textbf{18.35} & \romline{yayā svapnaṃ bhayaṃ śokaṃ} \\
 & \romline{viṣādaṃ madameva ca |} \\
 & \romline{na vimuñcati durmedhāḥ} \\
 & \romline{dhṛtiḥ sā tāmasī matā ||}
\end{tabular}
\end{table}

\begin{table}[H]
\begin{tabular}{cl}
\textbf{18.36} & \romline{sukhaṃ tvidānīṃ trividhaṃ} \\
 & \romline{śṛṇu me bharatarṣabha |} \\
 & \romline{abhyāsādramate yatra} \\
 & \romline{duḥkhāntaṃ ca nigacchati ||}
\end{tabular}
\end{table}

\begin{table}[H]
\begin{tabular}{cl}
\textbf{18.37} & \romline{yattadagre viṣamiva} \\
 & \romline{pariṇāme'mṛtopamam |} \\
 & \romline{tatsukhaṃ sāttvikaṃ proktam} \\
 & \romline{ātmabuddhiprasādajam ||}
\end{tabular}
\end{table}

\begin{table}[H]
\begin{tabular}{cl}
\textbf{18.38} & \romline{viṣayendriyasaṃyogāt} \\
 & \romline{yattadagre'mṛtopamam |} \\
 & \romline{pariṇāme viṣamiva} \\
 & \romline{tatsukhaṃ rājasaṃ smṛtam ||}
\end{tabular}
\end{table}

\begin{table}[H]
\begin{tabular}{cl}
\textbf{18.39} & \romline{yadagre cānubandhe ca} \\
 & \romline{sukhaṃ mohanamātmanaḥ |} \\
 & \romline{nidrālasyapramādotthaṃ} \\
 & \romline{tattāmasamudāhṛtam ||}
\end{tabular}
\end{table}

\begin{table}[H]
\begin{tabular}{cl}
\textbf{18.40} & \romline{na tadasti pṛthivyāṃ vā} \\
 & \romline{divi deveṣu vā punaḥ |} \\
 & \romline{sattvaṃ prakṛtijairmuktaṃ} \\
 & \romline{yadebhiḥ syāttribhirguṇaiḥ ||}
\end{tabular}
\end{table}

\begin{table}[H]
\begin{tabular}{cl}
\textbf{18.41} & \romline{brāhmaṇakṣatriyaviśāṃ} \\
 & \romline{śūdrāṇāṃ ca parantapa |} \\
 & \romline{karmāṇi pravibhaktāni} \\
 & \romline{svabhāvaprabhavairguṇaiḥ ||}
\end{tabular}
\end{table}

\begin{table}[H]
\begin{tabular}{cl}
\textbf{18.42} & \romline{śamo damastapaḥ śaucaṃ} \\
 & \romline{kṣāntirārjavameva ca |} \\
 & \romline{jñānaṃ vijñānamāstikyaṃ} \\
 & \romline{brahmakarma svabhāvajam ||}
\end{tabular}
\end{table}

\begin{table}[H]
\begin{tabular}{cl}
\textbf{18.43} & \romline{śauryaṃ tejo dhṛtirdākṣyaṃ} \\
 & \romline{yuddhe cāpyapalāyanam |} \\
 & \romline{dānamīśvarabhāvaśca} \\
 & \romline{kṣātraṃ karma svabhāvajam ||}
\end{tabular}
\end{table}

\begin{table}[H]
\begin{tabular}{cl}
\textbf{18.44} & \romline{kṛṣigaurakṣyavāṇijyaṃ} \\
 & \romline{vaiśyakarma svabhāvajam |} \\
 & \romline{paricaryātmakaṃ karma} \\
 & \romline{śūdrasyāpi svabhāvajam ||}
\end{tabular}
\end{table}

\begin{table}[H]
\begin{tabular}{cl}
\textbf{18.45} & \romline{sve sve karmaṇyabhirataḥ} \\
 & \romline{saṃsiddhiṃ labhate naraḥ |} \\
 & \romline{svakarmanirataḥ siddhiṃ} \\
 & \romline{yathā vindati tacchṛṇu ||}
\end{tabular}
\end{table}

\begin{table}[H]
\begin{tabular}{cl}
\textbf{18.46} & \romline{yataḥ pravṛttirbhūtānāṃ} \\
 & \romline{yena sarvamidaṃ tatam |} \\
 & \romline{svakarmaṇā tamabhyarcya} \\
 & \romline{siddhiṃ vindati mānavaḥ ||}
\end{tabular}
\end{table}

\begin{table}[H]
\begin{tabular}{cl}
\textbf{18.47} & \romline{śreyānsvadharmo viguṇaḥ} \\
 & \romline{paradharmātsvanuṣṭhitāt |} \\
 & \romline{svabhāvaniyataṃ karma} \\
 & \romline{kurvannāpnoti kilbiṣam ||}
\end{tabular}
\end{table}

\begin{table}[H]
\begin{tabular}{cl}
\textbf{18.48} & \romline{sahajaṃ karma kaunteya} \\
 & \romline{sadoṣamapi na tyajet |} \\
 & \romline{sarvārambhā hi doṣeṇa} \\
 & \romline{dhūmenāgnirivāvṛtāḥ ||}
\end{tabular}
\end{table}

\begin{table}[H]
\begin{tabular}{cl}
\textbf{18.49} & \romline{asaktabuddhiḥ sarvatra} \\
 & \romline{jitātmā vigataspṛhaḥ |} \\
 & \romline{naiṣkarmyasiddhiṃ paramāṃ} \\
 & \romline{sannyāsenādhigacchati ||}
\end{tabular}
\end{table}

\begin{table}[H]
\begin{tabular}{cl}
\textbf{18.50} & \romline{siddhiṃ prāpto yathā brahma} \\
 & \romline{tathā''pnoti nibodha me |} \\
 & \romline{samāsenaiva kaunteya} \\
 & \romline{niṣṭhā jñānasya yā parā ||}
\end{tabular}
\end{table}

\begin{table}[H]
\begin{tabular}{cl}
\textbf{18.51} & \romline{buddhyā viśuddhayā yuktaḥ} \\
 & \romline{dhṛtyā''tmānaṃ niyamya ca |} \\
 & \romline{śabdādīnviṣayāṃstyaktvā} \\
 & \romline{rāgadveṣau vyudasya ca ||}
\end{tabular}
\end{table}

\begin{table}[H]
\begin{tabular}{cl}
\textbf{18.52} & \romline{viviktasevī laghvāśī} \\
 & \romline{yatavākkāyamānasaḥ |} \\
 & \romline{dhyānayogaparo nityaṃ} \\
 & \romline{vairāgyaṃ samupāśritaḥ ||}
\end{tabular}
\end{table}

\begin{table}[H]
\begin{tabular}{cl}
\textbf{18.53} & \romline{ahaṅkāraṃ balaṃ darpaṃ} \\
 & \romline{kāmaṃ krodhaṃ parigraham |} \\
 & \romline{vimucya nirmamaḥ śāntaḥ} \\
 & \romline{brahmabhūyāya kalpate ||}
\end{tabular}
\end{table}

\begin{table}[H]
\begin{tabular}{cl}
\textbf{18.54} & \romline{brahmabhūtaḥ prasannātmā} \\
 & \romline{na śocati na kāṅkṣati |} \\
 & \romline{samaḥ sarveṣu bhūteṣu} \\
 & \romline{madbhaktiṃ labhate parām ||}
\end{tabular}
\end{table}

\begin{table}[H]
\begin{tabular}{cl}
\textbf{18.55} & \romline{bhaktyā māmabhijānāti} \\
 & \romline{yāvānyaścāsmi tattvataḥ |} \\
 & \romline{tato māṃ tattvato jñātvā} \\
 & \romline{viśate tadanantaram ||}
\end{tabular}
\end{table}

\begin{table}[H]
\begin{tabular}{cl}
\textbf{18.56} & \romline{sarvakarmāṇyapi sadā} \\
 & \romline{kurvāṇo madvyapāśrayaḥ |} \\
 & \romline{matprasādādavāpnoti} \\
 & \romline{śāśvataṃ padamavyayam ||}
\end{tabular}
\end{table}

\begin{table}[H]
\begin{tabular}{cl}
\textbf{18.57} & \romline{cetasā sarvakarmāṇi} \\
 & \romline{mayi sannyasya matparaḥ |} \\
 & \romline{buddhiyogamupāśritya} \\
 & \romline{maccittaḥ satataṃ bhava ||}
\end{tabular}
\end{table}

\begin{table}[H]
\begin{tabular}{cl}
\textbf{18.58} & \romline{maccittaḥ sarvadurgāṇi} \\
 & \romline{matprasādāttariṣyasi |} \\
 & \romline{atha cettvamahaṅkārāt} \\
 & \romline{na śroṣyasi vinaṅkṣyasi ||}
\end{tabular}
\end{table}

\begin{table}[H]
\begin{tabular}{cl}
\textbf{18.59} & \romline{yadahaṅkāramāśritya} \\
 & \romline{na yotsya iti manyase |} \\
 & \romline{mithyaiṣa vyavasāyaste} \\
 & \romline{prakṛtistvāṃ niyokṣyati ||}
\end{tabular}
\end{table}

\begin{table}[H]
\begin{tabular}{cl}
\textbf{18.60} & \romline{svabhāvajena kaunteya} \\
 & \romline{nibaddhaḥ svena karmaṇā |} \\
 & \romline{kartuṃ necchasi yanmohāt} \\
 & \romline{kariṣyasyavaśo'pi tat ||}
\end{tabular}
\end{table}

\begin{table}[H]
\begin{tabular}{cl}
\textbf{18.61} & \romline{īśvaraḥ sarvabhūtānāṃ} \\
 & \romline{hṛddeśe'rjuna tiṣṭhati |} \\
 & \romline{bhrāmayansarvabhūtāni} \\
 & \romline{yantrārūḍhāni māyayā ||}
\end{tabular}
\end{table}

\begin{table}[H]
\begin{tabular}{cl}
\textbf{18.62} & \romline{tameva śaraṇaṃ gaccha} \\
 & \romline{sarvabhāvena bhārata |} \\
 & \romline{tatprasādātparāṃ śāntiṃ} \\
 & \romline{sthānaṃ prāpsyasi śāśvatam ||}
\end{tabular}
\end{table}

\begin{table}[H]
\begin{tabular}{cl}
\textbf{18.63} & \romline{iti te jñānamākhyātaṃ} \\
 & \romline{guhyādguhyataraṃ mayā |} \\
 & \romline{vimṛśyaitadaśeṣeṇa} \\
 & \romline{yathecchasi tathā kuru ||}
\end{tabular}
\end{table}

\begin{table}[H]
\begin{tabular}{cl}
\textbf{18.64} & \romline{sarvaguhyatamaṃ bhūyaḥ} \\
 & \romline{śṛṇu me paramaṃ vacaḥ |} \\
 & \romline{iṣṭo'si me dṛḍhamiti} \\
 & \romline{tato vakṣyāmi te hitam ||}
\end{tabular}
\end{table}

\begin{table}[H]
\begin{tabular}{cl}
\textbf{18.65} & \romline{manmanā bhava madbhaktaḥ} \\
 & \romline{madyājī māṃ namaskuru |} \\
 & \romline{māmevaiṣyasi satyaṃ te} \\
 & \romline{pratijāne priyo'si me ||}
\end{tabular}
\end{table}

\begin{table}[H]
\begin{tabular}{cl}
\textbf{18.66} & \romline{sarvadharmānparityajya} \\
 & \romline{māmekaṃ śaraṇaṃ vraja |} \\
 & \romline{ahaṃ tvā sarvapāpebhyaḥ} \\
 & \romline{mokṣayiṣyāmi mā śucaḥ ||}
\end{tabular}
\end{table}

\begin{table}[H]
\begin{tabular}{cl}
\textbf{18.67} & \romline{idaṃ te nātapaskāya} \\
 & \romline{nābhaktāya kadācana |} \\
 & \romline{na cāśuśrūṣave vācyaṃ} \\
 & \romline{na ca māṃ yo'bhyasūyati ||}
\end{tabular}
\end{table}

\begin{table}[H]
\begin{tabular}{cl}
\textbf{18.68} & \romline{ya imaṃ paramaṃ guhyaṃ} \\
 & \romline{madbhakteṣvabhidhāsyati |} \\
 & \romline{bhaktiṃ mayi parāṃ kṛtvā} \\
 & \romline{māmevaiṣyatyasaṃśayaḥ ||}
\end{tabular}
\end{table}

\begin{table}[H]
\begin{tabular}{cl}
\textbf{18.69} & \romline{na ca tasmānmanuṣyeṣu} \\
 & \romline{kaścinme priyakṛttamaḥ |} \\
 & \romline{bhavitā na ca me tasmāt} \\
 & \romline{anyaḥ priyataro bhuvi ||}
\end{tabular}
\end{table}

\begin{table}[H]
\begin{tabular}{cl}
\textbf{18.70} & \romline{adhyeṣyate ca ya imaṃ} \\
 & \romline{dharmyaṃ saṃvādamāvayoḥ |} \\
 & \romline{jñānayajñena tenāham} \\
 & \romline{iṣṭaḥ syāmiti me matiḥ ||}
\end{tabular}
\end{table}

\begin{table}[H]
\begin{tabular}{cl}
\textbf{18.71} & \romline{śraddhāvānanasūyaśca} \\
 & \romline{śṛṇuyādapi yo naraḥ |} \\
 & \romline{so'pi muktaḥ śubhāllokān} \\
 & \romline{prāpnuyātpuṇyakarmaṇām ||}
\end{tabular}
\end{table}

\begin{table}[H]
\begin{tabular}{cl}
\textbf{18.72} & \romline{kaccidetacchrutaṃ pārtha} \\
 & \romline{tvayaikāgreṇa cetasā |} \\
 & \romline{kaccidajñānasammohaḥ} \\
 & \romline{pranaṣṭaste dhanañjaya ||}
\end{tabular}
\end{table}

\begin{table}[H]
\begin{tabular}{cl}
\textbf{18.73} & \romline{arjuna uvāca} \\
 & \romline{naṣṭo mohaḥ smṛtirlabdhā} \\
 & \romline{tvatprasādānmayā'cyuta |} \\
 & \romline{sthito'smi gatasandehaḥ} \\
 & \romline{kariṣye vacanaṃ tava ||}
\end{tabular}
\end{table}

\begin{table}[H]
\begin{tabular}{cl}
\textbf{18.74} & \romline{sañjaya uvāca} \\
 & \romline{ityahaṃ vāsudevasya} \\
 & \romline{pārthasya ca mahātmanaḥ |} \\
 & \romline{saṃvādamimamaśrauṣam} \\
 & \romline{adbhutaṃ romaharṣaṇam ||}
\end{tabular}
\end{table}

\begin{table}[H]
\begin{tabular}{cl}
\textbf{18.75} & \romline{vyāsaprasādācchrutavān} \\
 & \romline{imaṃ guhyatamaṃ param |} \\
 & \romline{yogaṃ yogeśvarātkṛṣṇāt} \\
 & \romline{sākṣātkathayataḥ svayam ||}
\end{tabular}
\end{table}

\begin{table}[H]
\begin{tabular}{cl}
\textbf{18.76} & \romline{rājan saṃsmṛtya saṃsmṛtya} \\
 & \romline{saṃvādamimamadbhutam |} \\
 & \romline{keśavārjunayoḥ puṇyaṃ} \\
 & \romline{hṛṣyāmi ca muhurmuhuḥ ||}
\end{tabular}
\end{table}

\begin{table}[H]
\begin{tabular}{cl}
\textbf{18.77} & \romline{tacca saṃsmṛtya saṃsmṛtya} \\
 & \romline{rūpamatyadbhutaṃ hareḥ |} \\
 & \romline{vismayo me mahānrājan} \\
 & \romline{hṛṣyāmi ca punaḥ punaḥ ||}
\end{tabular}
\end{table}

\begin{table}[H]
\begin{tabular}{cl}
\textbf{18.78} & \romline{yatra yogeśvaraḥ kṛṣṇaḥ} \\
 & \romline{yatra pārtho dhanurdharaḥ |} \\
 & \romline{tatra śrīrvijayo bhūtiḥ} \\
 & \romline{dhruvā nītirmatirmama ||}
\end{tabular}
\end{table}

\begin{table}[H]
\begin{tabular}{cl}
 & \romline{śrīmadbhagavadgītāsu upaniṣatsu} \\
 & \romline{brahmavidyāyāṃ yogaśāstre} \\
 & \romline{śrīkṛṣṇārjuna saṃvāde} \\
 & \romline{mokṣasannyāsayogo nāma} \\
 & \romline{aṣṭādaśodhyāyaḥ}
\end{tabular}
\end{table}

