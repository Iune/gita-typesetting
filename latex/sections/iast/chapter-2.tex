\begin{table}[H]
\begin{tabular}{cl}
\textbf{2.0} & \romline{oṃ śrī paramātmane namaḥ} \\
 & \romline{atha dvitīyo'dhyāyaḥ} \\
 & \romline{sāṅkhya-yogaḥ}
\end{tabular}
\end{table}

\begin{table}[H]
\begin{tabular}{cl}
\textbf{2.1} & \romline{saṃjaya uvāca} \\
 & \romline{taṃ tathā kṛpayāviṣṭam} \\
 & \romline{aśru-pūrṇākulekṣaṇam |} \\
 & \romline{viṣīdaṃtamidaṃ vākyam} \\
 & \romline{uvāca madhusūdanaḥ ||}
\end{tabular}
\end{table}

\begin{table}[H]
\begin{tabular}{cl}
\textbf{2.2} & \romline{śrī bhagavān-uvāca} \\
 & \romline{kutastvā kaśmalamidaṃ} \\
 & \romline{viṣame samupasthitam |} \\
 & \romline{anārya-juṣṭamasvargyam} \\
 & \romline{akīrti-karam-arjuna ||}
\end{tabular}
\end{table}

\begin{table}[H]
\begin{tabular}{cl}
\textbf{2.3} & \romline{klaibyaṃ mā sma gamaḥ pārtha} \\
 & \romline{naitat-tvayyupapadyate |} \\
 & \romline{kṣudraṃ hṛdaya-daurbalyaṃ} \\
 & \romline{tyaktvottiṣṭha paraṃtapa ||}
\end{tabular}
\end{table}

\begin{table}[H]
\begin{tabular}{cl}
\textbf{2.4} & \romline{arjuna uvāca} \\
 & \romline{kathaṃ bhīśmamahaṃ saṃkhye} \\
 & \romline{droṇaṃ ca madhusūdana |} \\
 & \romline{iśubhiḥ pratiyotsyāmi} \\
 & \romline{pūjārhāvarisūdana ||}
\end{tabular}
\end{table}

\begin{table}[H]
\begin{tabular}{cl}
\textbf{2.5} & \romline{gurūnahatvā hi mahānubhāvān} \\
 & \romline{śreyo bhoktuṃ bhaikṣyamapīha loke |} \\
 & \romline{hatvārtha-kāmaṃstu gurūnihaiva} \\
 & \romline{bhuṃjīya bhogān rudhira-pradigdhān ||}
\end{tabular}
\end{table}

\begin{table}[H]
\begin{tabular}{cl}
\textbf{2.6} & \romline{na caitadvidmaḥ kataranno garīyaḥ} \\
 & \romline{yadvā jayema yadi vā no jayeyuḥ |} \\
 & \romline{yāneva hatvā na jijīviṣāmaḥ} \\
 & \romline{te'vasthitāḥ pramukhe dhārtarāṣṭrāḥ ||}
\end{tabular}
\end{table}

\begin{table}[H]
\begin{tabular}{cl}
\textbf{2.7} & \romline{kārpanya-doṣopahata-svabhāvaḥ} \\
 & \romline{pṛcchāmi tvāṃ dharma-sammūḍha-cetāḥ |} \\
 & \romline{yacchreyaḥ syānniścitaṃ brūhi tanme} \\
 & \romline{śiṣyaste'haṃ śādhi māṃ tvāṃ prapannam ||}
\end{tabular}
\end{table}

\begin{table}[H]
\begin{tabular}{cl}
\textbf{2.8} & \romline{na hi·prapaśyāmi mamāpanudyād} \\
 & \romline{yacchokam-ucchoṣaṇam-indriyāṇām |} \\
 & \romline{avāpya bhūmāv-asapatnamṛddhaṃ} \\
 & \romline{rājyaṃ surāṇāmapi cādhipatyam ||}
\end{tabular}
\end{table}

\begin{table}[H]
\begin{tabular}{cl}
\textbf{2.9} & \romline{saṃjaya uvāca} \\
 & \romline{evam-uktvā hṛṣīkeśaṃ} \\
 & \romline{guḍākeśaḥ parantapaḥ |} \\
 & \romline{na yotsya iti goviṃdam} \\
 & \romline{uktvā tūṣṇīm babhūva ha ||}
\end{tabular}
\end{table}

\begin{table}[H]
\begin{tabular}{cl}
\textbf{2.10} & \romline{tam-uvāca hṛṣīkeśaḥ} \\
 & \romline{prahasanniva bhārata |} \\
 & \romline{senayorubhayor-madhye} \\
 & \romline{viṣīdaṃtam-idaṃ vacaḥ ||}
\end{tabular}
\end{table}

\begin{table}[H]
\begin{tabular}{cl}
\textbf{2.11} & \romline{śrī bhagavān-uvāca} \\
 & \romline{aśocyān-anvaśocas-tvaṃ} \\
 & \romline{prajñā-vādāṃśca bhāṣase |} \\
 & \romline{gatāsūn-agatāsūṃś-ca} \\
 & \romline{nānuśocanti paṇḍitāḥ ||}
\end{tabular}
\end{table}

\begin{table}[H]
\begin{tabular}{cl}
\textbf{2.12} & \romline{na tvevāhaṃ jātu nāsaṃ} \\
 & \romline{na tvaṃ neme janādhipāḥ |} \\
 & \romline{na caiva na bhaviṣyāmaḥ} \\
 & \romline{sarve vayamataḥ param ||}
\end{tabular}
\end{table}

\begin{table}[H]
\begin{tabular}{cl}
\textbf{2.13} & \romline{dehino'smin yathā dehe} \\
 & \romline{kaumāraṃ yauvanaṃ jarā |} \\
 & \romline{tathā dehāṃtara-prāptiḥ} \\
 & \romline{dhīras-tatra na muhyati ||}
\end{tabular}
\end{table}

\begin{table}[H]
\begin{tabular}{cl}
\textbf{2.14} & \romline{mātrā-sparśās-tu kauṃteya} \\
 & \romline{śītoṣṇa-sukha-duḥkha-dāḥ |} \\
 & \romline{āgamāpāyino'nityāḥ} \\
 & \romline{tāṃs-titikṣasva bhārata ||}
\end{tabular}
\end{table}

\begin{table}[H]
\begin{tabular}{cl}
\textbf{2.15} & \romline{yaṃ hi na vyathayaṃtyete} \\
 & \romline{puruṣaṃ puruṣarṣabha |} \\
 & \romline{sama-duḥkha-sukhaṃ dhīraṃ} \\
 & \romline{so'mṛtatvāya kalpate ||}
\end{tabular}
\end{table}

\begin{table}[H]
\begin{tabular}{cl}
\textbf{2.16} & \romline{nāsato vidyate bhāvaḥ} \\
 & \romline{nābhāvo vidyate sataḥ |} \\
 & \romline{ubhayorapi dṛṣto'ntaḥ} \\
 & \romline{tvanayos-tattva-darśibhiḥ ||}
\end{tabular}
\end{table}

\begin{table}[H]
\begin{tabular}{cl}
\textbf{2.17} & \romline{avināśi tu tadviddhi} \\
 & \romline{yena sarvamidaṃ tatam |} \\
 & \romline{vināśam-avyayasyāsya} \\
 & \romline{na kaścit-kartum-arhati ||}
\end{tabular}
\end{table}

\begin{table}[H]
\begin{tabular}{cl}
\textbf{2.18} & \romline{aṃtavanta ime dehāḥ} \\
 & \romline{nityasyoktāḥ śarīriṇaḥ |} \\
 & \romline{anāśino'prameyasya} \\
 & \romline{tasmād-yudhyasva bhārata ||}
\end{tabular}
\end{table}

\begin{table}[H]
\begin{tabular}{cl}
\textbf{2.19} & \romline{ya enaṃ vetti hantāraṃ} \\
 & \romline{yaścainaṃ manyate hataṃ |} \\
 & \romline{ubhau tau na vijānītaḥ} \\
 & \romline{nāyaṃ hanti na hanyate ||}
\end{tabular}
\end{table}

\begin{table}[H]
\begin{tabular}{cl}
\textbf{2.20} & \romline{na jāyate mriyate vā kadācit} \\
 & \romline{nāyaṃ bhūtvā bhavitā vā na bhūyaḥ |} \\
 & \romline{ajo nityaḥ śāśvato'yaṃ purāṇaḥ} \\
 & \romline{na hanyate hanyamāne śarīre ||}
\end{tabular}
\end{table}

\begin{table}[H]
\begin{tabular}{cl}
\textbf{2.21} & \romline{vedāvināśinaṃ nityaṃ} \\
 & \romline{ya enamajam-avyayam |} \\
 & \romline{kathaṃ sa puruṣaḥ pārtha} \\
 & \romline{kaṃ ghātayati haṃti kam ||}
\end{tabular}
\end{table}

\begin{table}[H]
\begin{tabular}{cl}
\textbf{2.22} & \romline{vāsāṃsi jīrṇāni yathā vihāya} \\
 & \romline{navāni gṛhṇāti naro'parāṇi |} \\
 & \romline{tathā śarīrāṇi vihāya jīrṇāni} \\
 & \romline{anyāni saṃyāti navāni dehī ||}
\end{tabular}
\end{table}

\begin{table}[H]
\begin{tabular}{cl}
\textbf{2.23} & \romline{nainaṃ chindanti śastrāṇi} \\
 & \romline{nainaṃ dahati pāvakaḥ |} \\
 & \romline{na cainaṃ kledayantyāpaḥ} \\
 & \romline{na śoṣayati mārutaḥ ||}
\end{tabular}
\end{table}

\begin{table}[H]
\begin{tabular}{cl}
\textbf{2.24} & \romline{acchedyo'yam adāhyo'yam} \\
 & \romline{akledyo'śoṣya eva ca |} \\
 & \romline{nityaḥ sarva-gataḥ sthāṇuḥ} \\
 & \romline{acalo'yaṃ sanātanaḥ ||}
\end{tabular}
\end{table}

\begin{table}[H]
\begin{tabular}{cl}
\textbf{2.25} & \romline{avyakto'yam acintyo'yam} \\
 & \romline{avikāryo'yamucyate |} \\
 & \romline{tasmādevaṃ viditvainaṃ} \\
 & \romline{nānuśocitumarhasi ||}
\end{tabular}
\end{table}

\begin{table}[H]
\begin{tabular}{cl}
\textbf{2.26} & \romline{atha cainaṃ nityajātaṃ} \\
 & \romline{nityaṃ vā manyase mṛtam |} \\
 & \romline{tathā'pi tvaṃ mahābāho} \\
 & \romline{naivaṃ śocitumarhasi ||}
\end{tabular}
\end{table}

\begin{table}[H]
\begin{tabular}{cl}
\textbf{2.27} & \romline{jātasya hi·dhruvo mṛtyuḥ} \\
 & \romline{dhruvaṃ janma mṛtasya ca |} \\
 & \romline{tasmādaparihārye'rthe} \\
 & \romline{na tvaṃ śocitumarhasi ||}
\end{tabular}
\end{table}

\begin{table}[H]
\begin{tabular}{cl}
\textbf{2.28} & \romline{avyaktādīni bhūtāni} \\
 & \romline{vyaktamadhyāni bhārata |} \\
 & \romline{avyakta-nidhanānyeva} \\
 & \romline{tatra kā paridevanā ||}
\end{tabular}
\end{table}

\begin{table}[H]
\begin{tabular}{cl}
\textbf{2.29} & \romline{āścaryavat paśyati kaścidenam} \\
 & \romline{āścaryavad vadati tathaiva cānyaḥ |} \\
 & \romline{āścaryavaccainamanyaḥ śṛṇoti} \\
 & \romline{śrutvāpyenaṃ veda na caiva kaścit ||}
\end{tabular}
\end{table}

\begin{table}[H]
\begin{tabular}{cl}
\textbf{2.30} & \romline{dehī nityamavadhyo'yaṃ} \\
 & \romline{dehe sarvasya bhārata |} \\
 & \romline{tasmātsarvāṇi bhūtāni} \\
 & \romline{na tvaṃ śocitumarhasi ||}
\end{tabular}
\end{table}

\begin{table}[H]
\begin{tabular}{cl}
\textbf{2.31} & \romline{svadharmamapi cāvekṣya} \\
 & \romline{na vikampitumarhasi |} \\
 & \romline{dharmyāddhi yuddhācchreyo'nyat} \\
 & \romline{kṣatriyasya-na-vidyate ||}
\end{tabular}
\end{table}

\begin{table}[H]
\begin{tabular}{cl}
\textbf{2.32} & \romline{yadṛcchayā copapannaṃ} \\
 & \romline{svarga-dvāra-mapāvṛtam |} \\
 & \romline{sukhinaḥ kṣatriyāḥ pārtha} \\
 & \romline{labhante yuddhamīdṛśam ||}
\end{tabular}
\end{table}

\begin{table}[H]
\begin{tabular}{cl}
\textbf{2.33} & \romline{atha cettvamimaṃ dharmyaṃ} \\
 & \romline{saṅgrāmaṃ na kariṣyasi |} \\
 & \romline{tataḥ svadharmaṃ kīrtiṃ ca} \\
 & \romline{hitvā pāpamavāpsyasi ||}
\end{tabular}
\end{table}

\begin{table}[H]
\begin{tabular}{cl}
\textbf{2.34} & \romline{akīrtiṃ cāpi bhūtāni} \\
 & \romline{kathayiṣyanti te'vyayām |} \\
 & \romline{sambhāvitasya cākīrtiḥ} \\
 & \romline{maraṇādatiricyate ||}
\end{tabular}
\end{table}

\begin{table}[H]
\begin{tabular}{cl}
\textbf{2.35} & \romline{bhayādraṇāduparataṃ} \\
 & \romline{maṃsyante tvāṃ mahārathāḥ |} \\
 & \romline{yeṣāṃ ca tvaṃ bahumataḥ} \\
 & \romline{bhūtvā yāsyasi lāghavam ||}
\end{tabular}
\end{table}

\begin{table}[H]
\begin{tabular}{cl}
\textbf{2.36} & \romline{avācyavādāṃśca bahūn} \\
 & \romline{vadiṣyanti tavāhitāḥ |} \\
 & \romline{nindantastava sāmarthyaṃ} \\
 & \romline{tato duḥkhataraṃ nu kim ||}
\end{tabular}
\end{table}

\begin{table}[H]
\begin{tabular}{cl}
\textbf{2.37} & \romline{hato vā prāpsyasi·svargaṃ} \\
 & \romline{jitvā vā bhokṣyase mahīm |} \\
 & \romline{tasmāduttiṣṭha kaunteya} \\
 & \romline{yuddhāya kṛtaniścayaḥ ||}
\end{tabular}
\end{table}

\begin{table}[H]
\begin{tabular}{cl}
\textbf{2.38} & \romline{sukha-duḥkhe same kṛtvā} \\
 & \romline{lābhālābhau jayājayau |} \\
 & \romline{tato yuddhāya yujyasva} \\
 & \romline{naivaṃ pāpamavāpsyasi ||}
\end{tabular}
\end{table}

\begin{table}[H]
\begin{tabular}{cl}
\textbf{2.39} & \romline{eṣā te'bhihitā sāṅkhye} \\
 & \romline{buddhiryoge tvimāṃ śṛṇu |} \\
 & \romline{buddhyā yukto yayā pārtha} \\
 & \romline{karmabandhaṃ prahāsyasi ||}
\end{tabular}
\end{table}

\begin{table}[H]
\begin{tabular}{cl}
\textbf{2.40} & \romline{nehābhikrama-nāśo'sti} \\
 & \romline{pratyavāyo na vidyate |} \\
 & \romline{svalpamapyasya dharmasya} \\
 & \romline{trāyate mahato bhayāt ||}
\end{tabular}
\end{table}

\begin{table}[H]
\begin{tabular}{cl}
\textbf{2.41} & \romline{vyavasāyātmikā buddhiḥ} \\
 & \romline{ekeha kurunandana |} \\
 & \romline{bahuśākhā hyanantāśca} \\
 & \romline{buddhayo'vyavasāyinām ||}
\end{tabular}
\end{table}

\begin{table}[H]
\begin{tabular}{cl}
\textbf{2.42} & \romline{yāmimāṃ puṣpitāṃ vācaṃ} \\
 & \romline{pravadantyavipaścitaḥ |} \\
 & \romline{vedavādaratāḥ pārtha} \\
 & \romline{nānyadastīti vādinaḥ ||}
\end{tabular}
\end{table}

\begin{table}[H]
\begin{tabular}{cl}
\textbf{2.43} & \romline{kāmātmānaḥ svargaparāḥ} \\
 & \romline{janma-karma-phala-pradām |} \\
 & \romline{kriyā-viśeṣa-bahulāṃ} \\
 & \romline{bhogaiśvaryagatiṃ prati ||}
\end{tabular}
\end{table}

\begin{table}[H]
\begin{tabular}{cl}
\textbf{2.44} & \romline{bhogaiśvarya-prasaktānāṃ} \\
 & \romline{tayā'pahṛta-cetasām |} \\
 & \romline{vyavasāyātmikā buddhiḥ} \\
 & \romline{samādhau na vidhīyate ||}
\end{tabular}
\end{table}

\begin{table}[H]
\begin{tabular}{cl}
\textbf{2.45} & \romline{traiguṇyaviṣayā vedāḥ} \\
 & \romline{nistraiguṇyo bhavārjuna |} \\
 & \romline{nirdvandvo nityasattvasthaḥ} \\
 & \romline{niryogakṣema ātmavān ||}
\end{tabular}
\end{table}

\begin{table}[H]
\begin{tabular}{cl}
\textbf{2.46} & \romline{yāvānartha udapāne} \\
 & \romline{sarvataḥ samplutodake |} \\
 & \romline{tāvānsarveṣu vedeṣu} \\
 & \romline{brāhmaṇasya vijānataḥ ||}
\end{tabular}
\end{table}

\begin{table}[H]
\begin{tabular}{cl}
\textbf{2.47} & \romline{karmaṇyevādhikāraste} \\
 & \romline{mā phaleṣu kadācana |} \\
 & \romline{mā karma-phala-heturbhūḥ} \\
 & \romline{mā te saṅgo'stvakarmaṇi ||}
\end{tabular}
\end{table}

\begin{table}[H]
\begin{tabular}{cl}
\textbf{2.48} & \romline{yogasthaḥ kuru karmāṇi} \\
 & \romline{saṅgaṃ tyaktvā dhanañjaya |} \\
 & \romline{siddhyasiddhyoḥ samo bhūtvā} \\
 & \romline{samatvaṃ yoga ucyate ||}
\end{tabular}
\end{table}

\begin{table}[H]
\begin{tabular}{cl}
\textbf{2.49} & \romline{dūreṇa hyavaraṃ karma} \\
 & \romline{buddhi-yogāddhanañjaya |} \\
 & \romline{buddhau śaraṇam-anviccha} \\
 & \romline{kṛpaṇāḥ phalahetavaḥ ||}
\end{tabular}
\end{table}

\begin{table}[H]
\begin{tabular}{cl}
\textbf{2.50} & \romline{buddhi-yukto jahātīha} \\
 & \romline{ubhe sukṛtaduṣkṛte |} \\
 & \romline{tasmādyogāya yujyasva} \\
 & \romline{yogaḥ karmasu kauśalam ||}
\end{tabular}
\end{table}

\begin{table}[H]
\begin{tabular}{cl}
\textbf{2.51} & \romline{karmajaṃ buddhi-yuktā hi} \\
 & \romline{phalaṃ tyaktvā manīṣiṇaḥ |} \\
 & \romline{janma-bandha-vinirmuktāḥ} \\
 & \romline{padaṃ gacchantyanāmayam ||}
\end{tabular}
\end{table}

\begin{table}[H]
\begin{tabular}{cl}
\textbf{2.52} & \romline{yadā te mohakalilaṃ} \\
 & \romline{buddhir vyatitariṣyati |} \\
 & \romline{tadā gantāsi nirvedaṃ} \\
 & \romline{śrotavyasya śrutasya ca ||}
\end{tabular}
\end{table}

\begin{table}[H]
\begin{tabular}{cl}
\textbf{2.53} & \romline{śruti-vipratipannā te} \\
 & \romline{yadā sthāsyati niścalā |} \\
 & \romline{samādhāvacalā buddhiḥ} \\
 & \romline{tadā yogam-avāpsyasi ||}
\end{tabular}
\end{table}

\begin{table}[H]
\begin{tabular}{cl}
\textbf{2.54} & \romline{arjuna uvāca} \\
 & \romline{sthita-prajñasya kā bhāṣā} \\
 & \romline{samādhisthasya keśava |} \\
 & \romline{sthitadhīḥ kiṃ prabhāṣeta} \\
 & \romline{kimāsīta·vrajeta kim ||}
\end{tabular}
\end{table}

\begin{table}[H]
\begin{tabular}{cl}
\textbf{2.55} & \romline{śrī bhagavān uvāca} \\
 & \romline{prajahāti yadā kāmān} \\
 & \romline{sarvān-pārtha manogatān |} \\
 & \romline{ātmanyevātmanā tuṣṭaḥ} \\
 & \romline{sthita-prajñas tadocyate ||}
\end{tabular}
\end{table}

\begin{table}[H]
\begin{tabular}{cl}
\textbf{2.56} & \romline{duḥkheṣvanudvigna-manāḥ} \\
 & \romline{sukheṣu vigata-spṛhaḥ |} \\
 & \romline{vīta-rāga-bhaya-krodhaḥ} \\
 & \romline{sthita-dhīrmunirucyate ||}
\end{tabular}
\end{table}

\begin{table}[H]
\begin{tabular}{cl}
\textbf{2.57} & \romline{yaḥ sarvatrānabhi-snehaḥ} \\
 & \romline{tattatprāpya śubhāśubham |} \\
 & \romline{nābhinandati na·dveṣṭi} \\
 & \romline{tasya·prajñā·pratiṣṭhitā ||}
\end{tabular}
\end{table}

\begin{table}[H]
\begin{tabular}{cl}
\textbf{2.58} & \romline{yadā samharate cāyaṃ} \\
 & \romline{kūrmo'ṅgānīva sarvaśaḥ |} \\
 & \romline{indriyāṇīndriyārthebhyaḥ} \\
 & \romline{tasya·prajñā·pratiṣṭhitā ||}
\end{tabular}
\end{table}

\begin{table}[H]
\begin{tabular}{cl}
\textbf{2.59} & \romline{viṣayā vinivartante} \\
 & \romline{nirāhārasya dehinaḥ |} \\
 & \romline{rasa-varjaṃ raso'pyasya} \\
 & \romline{paraṃ dṛṣṭvā nivartate ||}
\end{tabular}
\end{table}

\begin{table}[H]
\begin{tabular}{cl}
\textbf{2.60} & \romline{yatato hyapi kaunteya} \\
 & \romline{puruṣasya vipaścitaḥ |} \\
 & \romline{indriyāṇi·pramāthīni} \\
 & \romline{haranti·prasabhaṃ manaḥ ||}
\end{tabular}
\end{table}

\begin{table}[H]
\begin{tabular}{cl}
\textbf{2.61} & \romline{tāni sarvāṇi samyamya} \\
 & \romline{yukta āsīta matparaḥ |} \\
 & \romline{vaśe hi yasyendriyāṇi} \\
 & \romline{tasya·prajñā·pratiṣṭhitā ||}
\end{tabular}
\end{table}

\begin{table}[H]
\begin{tabular}{cl}
\textbf{2.62} & \romline{dhyāyato viṣayānpuṃsaḥ} \\
 & \romline{saṅgasteṣūpajāyate |} \\
 & \romline{saṅgāt-sañjāyate kāmaḥ} \\
 & \romline{kāmāt-krodho'bhijāyate ||}
\end{tabular}
\end{table}

\begin{table}[H]
\begin{tabular}{cl}
\textbf{2.63} & \romline{krodhād bhavati sammohaḥ} \\
 & \romline{sammohāt smṛti-vibhramaḥ |} \\
 & \romline{smṛti-bhramśāt buddhi-nāśaḥ} \\
 & \romline{buddhi-nāśāt-praṇaśyati ||}
\end{tabular}
\end{table}

\begin{table}[H]
\begin{tabular}{cl}
\textbf{2.64} & \romline{rāga-dveṣa-viyuktaistu} \\
 & \romline{viṣayā-nindriyaiścaran |} \\
 & \romline{ātma-vaśyairvidheyātmā} \\
 & \romline{prasāda-madhigacchati ||}
\end{tabular}
\end{table}

\begin{table}[H]
\begin{tabular}{cl}
\textbf{2.65} & \romline{prasāde sarvaduḥkhānāṃ} \\
 & \romline{hānirasyopajāyate |} \\
 & \romline{prasanna-cetaso hyāśu} \\
 & \romline{buddhiḥ paryavatiṣṭhate ||}
\end{tabular}
\end{table}

\begin{table}[H]
\begin{tabular}{cl}
\textbf{2.66} & \romline{nāsti buddhirayuktasya} \\
 & \romline{na cāyuktasya bhāvanā |} \\
 & \romline{na cābhāvayataḥ śāntiḥ} \\
 & \romline{aśāntasya kutaḥ sukham ||}
\end{tabular}
\end{table}

\begin{table}[H]
\begin{tabular}{cl}
\textbf{2.67} & \romline{indriyāṇāṃ hi caratāṃ} \\
 & \romline{yanmano'nuvidhīyate |} \\
 & \romline{tadasya harati·prajñāṃ} \\
 & \romline{vāyurnāvamivāmbhasi ||}
\end{tabular}
\end{table}

\begin{table}[H]
\begin{tabular}{cl}
\textbf{2.68} & \romline{tasmād yasya mahābāho} \\
 & \romline{nigṛhītāni sarvaśaḥ |} \\
 & \romline{indriyāṇīndriyārthebhyaḥ} \\
 & \romline{tasya·prajñā·pratiṣṭhitā ||}
\end{tabular}
\end{table}

\begin{table}[H]
\begin{tabular}{cl}
\textbf{2.69} & \romline{yā niśā sarvabhūtānāṃ} \\
 & \romline{tasyāṃ jāgarti sanyamī |} \\
 & \romline{yasyāṃ jāgrati bhūtāni} \\
 & \romline{sā niśā paśyato muneḥ ||}
\end{tabular}
\end{table}

\begin{table}[H]
\begin{tabular}{cl}
\textbf{2.70} & \romline{āpuryamāṇam-acala-pratiṣṭhaṃ} \\
 & \romline{samudram-āpaḥ praviśanti yadvat |} \\
 & \romline{tadvat-kāmā yaṃ praviśanti sarve} \\
 & \romline{sa śāntim-āpnoti na kāma-kāmī ||}
\end{tabular}
\end{table}

\begin{table}[H]
\begin{tabular}{cl}
\textbf{2.71} & \romline{vihāya kāmān yaḥ sarvān} \\
 & \romline{pumāmścarati nisspṛhaḥ |} \\
 & \romline{nirmamo nirahankāraḥ} \\
 & \romline{sa śāntim-adhigacchati ||}
\end{tabular}
\end{table}

\begin{table}[H]
\begin{tabular}{cl}
\textbf{2.72} & \romline{eṣā brāhmī sthitiḥ pārtha} \\
 & \romline{naināṃ prāpya vimuhyati |} \\
 & \romline{sthitvāsyā-mantakāle'pi} \\
 & \romline{brahmanirvāṇam-ṛcchati ||}
\end{tabular}
\end{table}

