\subsection*{2.0}
\begin{table}[H]
\begin{tabular}{l}
\romline{oṃ śrī paramātmane namaḥ} \\
\romline{atha dvitīyo'dhyāyaḥ} \\
\romline{sāṅkhya-yogaḥ}
\end{tabular}
\end{table}

\subsection*{2.1}
\begin{table}[H]
\begin{tabular}{l}
\romline{saṃjaya uvāca} \\
\romline{taṃ tathā kṛpayāviṣṭam} \\
\romline{aśru-pūrṇākulekṣaṇam} \\
\romline{viṣīdaṃtamidaṃ vākyam} \\
\romline{uvāca madhusūdanaḥ}
\end{tabular}
\end{table}

\subsection*{2.2}
\begin{table}[H]
\begin{tabular}{l}
\romline{śrī bhagavān-uvāca} \\
\romline{kutastvā kaśmalamidaṃ} \\
\romline{viṣame samupasthitam} \\
\romline{anārya-juṣṭamasvargyam} \\
\romline{akīrti-karam-arjuna}
\end{tabular}
\end{table}

\subsection*{2.3}
\begin{table}[H]
\begin{tabular}{l}
\romline{klaibyaṃ mā sma gamaḥ pārtha} \\
\romline{naitat-tvayyupapadyate} \\
\romline{kṣudraṃ hṛdaya-daurbalyaṃ} \\
\romline{tyaktvottiṣṭha paraṃtapa}
\end{tabular}
\end{table}

\subsection*{2.4}
\begin{table}[H]
\begin{tabular}{l}
\romline{arjuna uvāca} \\
\romline{kathaṃ bhīśmamahaṃ saṃkhye} \\
\romline{droṇaṃ ca madhusūdana} \\
\romline{iśubhiḥ pratiyotsyāmi} \\
\romline{pūjārhāvarisūdana}
\end{tabular}
\end{table}

\subsection*{2.5}
\begin{table}[H]
\begin{tabular}{l}
\romline{gurūnahatvā hi mahānubhāvān} \\
\romline{śreyo bhoktuṃ bhaikṣyamapīha loke} \\
\romline{hatvārtha-kāmaṃstu gurūnihaiva} \\
\romline{bhuṃjīya bhogān rudhira-pradigdhān}
\end{tabular}
\end{table}

\subsection*{2.6}
\begin{table}[H]
\begin{tabular}{l}
\romline{na caitadvidmaḥ kataranno garīyaḥ} \\
\romline{yadvā jayema yadi vā no jayeyuḥ} \\
\romline{yāneva hatvā na jijīviṣāmaḥ} \\
\romline{te'vasthitāḥ pramukhe dhārtarāṣṭrāḥ}
\end{tabular}
\end{table}

\subsection*{2.7}
\begin{table}[H]
\begin{tabular}{l}
\romline{kārpanya-doṣopahata-svabhāvaḥ} \\
\romline{pṛcchāmi tvāṃ dharma-sammūḍha-cetāḥ} \\
\romline{yacchreyaḥ syānniścitaṃ brūhi tanme} \\
\romline{śiṣyaste'haṃ śādhi māṃ tvāṃ prapannam}
\end{tabular}
\end{table}

\subsection*{2.8}
\begin{table}[H]
\begin{tabular}{l}
\romline{na hi·prapaśyāmi mamāpanudyād} \\
\romline{yacchokam-ucchoṣaṇam-indriyāṇām} \\
\romline{avāpya bhūmāv-asapatnamṛddhaṃ} \\
\romline{rājyaṃ surāṇāmapi cādhipatyam}
\end{tabular}
\end{table}

\subsection*{2.9}
\begin{table}[H]
\begin{tabular}{l}
\romline{saṃjaya uvāca} \\
\romline{evam-uktvā hṛṣīkeśaṃ} \\
\romline{guḍākeśaḥ parantapaḥ} \\
\romline{na yotsya iti goviṃdam} \\
\romline{uktvā tūṣṇīm babhūva ha}
\end{tabular}
\end{table}

\subsection*{2.10}
\begin{table}[H]
\begin{tabular}{l}
\romline{tam-uvāca hṛṣīkeśaḥ} \\
\romline{prahasanniva bhārata} \\
\romline{senayorubhayor-madhye} \\
\romline{viṣīdaṃtam-idaṃ vacaḥ}
\end{tabular}
\end{table}

\subsection*{2.11}
\begin{table}[H]
\begin{tabular}{l}
\romline{śrī bhagavān-uvāca} \\
\romline{aśocyān-anvaśocas-tvaṃ} \\
\romline{prajñā-vādāṃśca bhāṣase} \\
\romline{gatāsūn-agatāsūṃś-ca} \\
\romline{nānuśocanti paṇḍitāḥ}
\end{tabular}
\end{table}

\subsection*{2.12}
\begin{table}[H]
\begin{tabular}{l}
\romline{na tvevāhaṃ jātu nāsaṃ} \\
\romline{na tvaṃ neme janādhipāḥ} \\
\romline{na caiva na bhaviṣyāmaḥ} \\
\romline{sarve vayamataḥ param}
\end{tabular}
\end{table}

\subsection*{2.13}
\begin{table}[H]
\begin{tabular}{l}
\romline{dehino'smin yathā dehe} \\
\romline{kaumāraṃ yauvanaṃ jarā} \\
\romline{tathā dehāṃtara-prāptiḥ} \\
\romline{dhīras-tatra na muhyati}
\end{tabular}
\end{table}

\subsection*{2.14}
\begin{table}[H]
\begin{tabular}{l}
\romline{mātrā-sparśās-tu kauṃteya} \\
\romline{śītoṣṇa-sukha-duḥkha-dāḥ} \\
\romline{āgamāpāyino'nityāḥ} \\
\romline{tāṃs-titikṣasva bhārata}
\end{tabular}
\end{table}

\subsection*{2.15}
\begin{table}[H]
\begin{tabular}{l}
\romline{yaṃ hi na vyathayaṃtyete} \\
\romline{puruṣaṃ puruṣarṣabha} \\
\romline{sama-duḥkha-sukhaṃ dhīraṃ} \\
\romline{so'mṛtatvāya kalpate}
\end{tabular}
\end{table}

\subsection*{2.16}
\begin{table}[H]
\begin{tabular}{l}
\romline{nāsato vidyate bhāvaḥ} \\
\romline{nābhāvo vidyate sataḥ} \\
\romline{ubhayorapi dṛṣto'ntaḥ} \\
\romline{tvanayos-tattva-darśibhiḥ}
\end{tabular}
\end{table}

\subsection*{2.17}
\begin{table}[H]
\begin{tabular}{l}
\romline{avināśi tu tadviddhi} \\
\romline{yena sarvamidaṃ tatam} \\
\romline{vināśam-avyayasyāsya} \\
\romline{na kaścit-kartum-arhati}
\end{tabular}
\end{table}

\subsection*{2.18}
\begin{table}[H]
\begin{tabular}{l}
\romline{aṃtavanta ime dehāḥ} \\
\romline{nityasyoktāḥ śarīriṇaḥ} \\
\romline{anāśino'prameyasya} \\
\romline{tasmād-yudhyasva bhārata}
\end{tabular}
\end{table}

\subsection*{2.19}
\begin{table}[H]
\begin{tabular}{l}
\romline{ya enaṃ vetti hantāraṃ} \\
\romline{yaścainaṃ manyate hataṃ} \\
\romline{ubhau tau na vijānītaḥ} \\
\romline{nāyaṃ hanti na hanyate}
\end{tabular}
\end{table}

\subsection*{2.20}
\begin{table}[H]
\begin{tabular}{l}
\romline{na jāyate mriyate vā kadācit} \\
\romline{nāyaṃ bhūtvā bhavitā vā na bhūyaḥ} \\
\romline{ajo nityaḥ śāśvato'yaṃ purāṇaḥ} \\
\romline{na hanyate hanyamāne śarīre}
\end{tabular}
\end{table}

\subsection*{2.21}
\begin{table}[H]
\begin{tabular}{l}
\romline{vedāvināśinaṃ nityaṃ} \\
\romline{ya enamajam-avyayam} \\
\romline{kathaṃ sa puruṣaḥ pārtha} \\
\romline{kaṃ ghātayati haṃti kam}
\end{tabular}
\end{table}

\subsection*{2.22}
\begin{table}[H]
\begin{tabular}{l}
\romline{vāsāṃsi jīrṇāni yathā vihāya} \\
\romline{navāni gṛhṇāti naro'parāṇi} \\
\romline{tathā śarīrāṇi vihāya jīrṇāni} \\
\romline{anyāni saṃyāti navāni dehī}
\end{tabular}
\end{table}

\subsection*{2.23}
\begin{table}[H]
\begin{tabular}{l}
\romline{nainaṃ chindanti śastrāṇi} \\
\romline{nainaṃ dahati pāvakaḥ} \\
\romline{na cainaṃ kledayantyāpaḥ} \\
\romline{na śoṣayati mārutaḥ}
\end{tabular}
\end{table}

\subsection*{2.24}
\begin{table}[H]
\begin{tabular}{l}
\romline{acchedyo'yam adāhyo'yam} \\
\romline{akledyo'śoṣya eva ca} \\
\romline{nityaḥ sarva-gataḥ sthāṇuḥ} \\
\romline{acalo'yaṃ sanātanaḥ}
\end{tabular}
\end{table}

\subsection*{2.25}
\begin{table}[H]
\begin{tabular}{l}
\romline{avyakto'yam acintyo'yam} \\
\romline{avikāryo'yamucyate} \\
\romline{tasmādevaṃ viditvainaṃ} \\
\romline{nānuśocitumarhasi}
\end{tabular}
\end{table}

\subsection*{2.26}
\begin{table}[H]
\begin{tabular}{l}
\romline{atha cainaṃ nityajātaṃ} \\
\romline{nityaṃ vā manyase mṛtam} \\
\romline{tathā'pi tvaṃ mahābāho} \\
\romline{naivaṃ śocitumarhasi}
\end{tabular}
\end{table}

\subsection*{2.27}
\begin{table}[H]
\begin{tabular}{l}
\romline{jātasya hi·dhruvo mṛtyuḥ} \\
\romline{dhruvaṃ janma mṛtasya ca} \\
\romline{tasmādaparihārye'rthe} \\
\romline{na tvaṃ śocitumarhasi}
\end{tabular}
\end{table}

\subsection*{2.28}
\begin{table}[H]
\begin{tabular}{l}
\romline{avyaktādīni bhūtāni} \\
\romline{vyaktamadhyāni bhārata} \\
\romline{avyakta-nidhanānyeva} \\
\romline{tatra kā paridevanā}
\end{tabular}
\end{table}

\subsection*{2.29}
\begin{table}[H]
\begin{tabular}{l}
\romline{āścaryavat paśyati kaścidenam} \\
\romline{āścaryavad vadati tathaiva cānyaḥ} \\
\romline{āścaryavaccainamanyaḥ śṛṇoti} \\
\romline{śrutvāpyenaṃ veda na caiva kaścit}
\end{tabular}
\end{table}

\subsection*{2.30}
\begin{table}[H]
\begin{tabular}{l}
\romline{dehī nityamavadhyo'yaṃ} \\
\romline{dehe sarvasya bhārata} \\
\romline{tasmātsarvāṇi bhūtāni} \\
\romline{na tvaṃ śocitumarhasi}
\end{tabular}
\end{table}

\subsection*{2.31}
\begin{table}[H]
\begin{tabular}{l}
\romline{svadharmamapi cāvekṣya} \\
\romline{na vikampitumarhasi} \\
\romline{dharmyāddhi yuddhācchreyo'nyat} \\
\romline{kṣatriyasya-na-vidyate}
\end{tabular}
\end{table}

\subsection*{2.32}
\begin{table}[H]
\begin{tabular}{l}
\romline{yadṛcchayā copapannaṃ} \\
\romline{svarga-dvāra-mapāvṛtam} \\
\romline{sukhinaḥ kṣatriyāḥ pārtha} \\
\romline{labhante yuddhamīdṛśam}
\end{tabular}
\end{table}

\subsection*{2.33}
\begin{table}[H]
\begin{tabular}{l}
\romline{atha cettvamimaṃ dharmyaṃ} \\
\romline{saṅgrāmaṃ na kariṣyasi} \\
\romline{tataḥ svadharmaṃ kīrtiṃ ca} \\
\romline{hitvā pāpamavāpsyasi}
\end{tabular}
\end{table}

\subsection*{2.34}
\begin{table}[H]
\begin{tabular}{l}
\romline{akīrtiṃ cāpi bhūtāni} \\
\romline{kathayiṣyanti te'vyayām} \\
\romline{sambhāvitasya cākīrtiḥ} \\
\romline{maraṇādatiricyate}
\end{tabular}
\end{table}

\subsection*{2.35}
\begin{table}[H]
\begin{tabular}{l}
\romline{bhayādraṇāduparataṃ} \\
\romline{maṃsyante tvāṃ mahārathāḥ} \\
\romline{yeṣāṃ ca tvaṃ bahumataḥ} \\
\romline{bhūtvā yāsyasi lāghavam}
\end{tabular}
\end{table}

\subsection*{2.36}
\begin{table}[H]
\begin{tabular}{l}
\romline{avācyavādāṃśca bahūn} \\
\romline{vadiṣyanti tavāhitāḥ} \\
\romline{nindantastava sāmarthyaṃ} \\
\romline{tato duḥkhataraṃ nu kim}
\end{tabular}
\end{table}

\subsection*{2.37}
\begin{table}[H]
\begin{tabular}{l}
\romline{hato vā prāpsyasi svargaṃ} \\
\romline{jitvā vā bhokṣyase mahīm} \\
\romline{tasmāduttiṣṭha kaunteya} \\
\romline{yuddhāya kṛtaniścayaḥ}
\end{tabular}
\end{table}

\subsection*{2.38}
\begin{table}[H]
\begin{tabular}{l}
\romline{sukha-duḥkhe same kṛtvā} \\
\romline{lābhālābhau jayājayau} \\
\romline{tato yuddhāya yujyasva} \\
\romline{naivaṃ pāpamavāpsyasi}
\end{tabular}
\end{table}

\subsection*{2.39}
\begin{table}[H]
\begin{tabular}{l}
\romline{eṣā te'bhihitā sāṅkhye} \\
\romline{buddhiryoge tvimāṃ śṛṇu} \\
\romline{buddhyā yukto yayā pārtha} \\
\romline{karmabandhaṃ prahāsyasi}
\end{tabular}
\end{table}

\subsection*{2.40}
\begin{table}[H]
\begin{tabular}{l}
\romline{nehābhikrama-nāśo'sti} \\
\romline{pratyavāyo na vidyate} \\
\romline{svalpamapyasya dharmasya} \\
\romline{trāyate mahato bhayāt}
\end{tabular}
\end{table}

\subsection*{2.41}
\begin{table}[H]
\begin{tabular}{l}
\romline{vyavasāyātmikā buddhiḥ} \\
\romline{ekeha kurunandana} \\
\romline{bahuśākhā hyanantāśca} \\
\romline{buddhayo'vyasāyinām}
\end{tabular}
\end{table}

\subsection*{2.42}
\begin{table}[H]
\begin{tabular}{l}
\romline{yāmimāṃ puṣpitāṃ vācaṃ} \\
\romline{pravadantyavipaścitaḥ} \\
\romline{vedavādaratāḥ pārtha} \\
\romline{nānyadastīti vādinaḥ}
\end{tabular}
\end{table}

\subsection*{2.43}
\begin{table}[H]
\begin{tabular}{l}
\romline{kāmātmānaḥ svargaparāḥ} \\
\romline{janma-karma-phala-pradām} \\
\romline{kriyā-viśeṣa-bahulāṃ} \\
\romline{bhogaiśvaryagatiṃ prati}
\end{tabular}
\end{table}

\subsection*{2.44}
\begin{table}[H]
\begin{tabular}{l}
\romline{bhogaiśvarya-prasaktānāṃ} \\
\romline{tayā'pahṛta-cetasām} \\
\romline{vyavasāyātmikā buddhiḥ} \\
\romline{samādhau na vidhīyate}
\end{tabular}
\end{table}

\subsection*{2.45}
\begin{table}[H]
\begin{tabular}{l}
\romline{traiguṇyaviṣayā vedāḥ} \\
\romline{nistraiguṇyo bhavārjuna} \\
\romline{nirdvandvo nityasattvasthaḥ} \\
\romline{niryogakṣema ātmavān}
\end{tabular}
\end{table}

\subsection*{2.46}
\begin{table}[H]
\begin{tabular}{l}
\romline{yāvānartha udapāne} \\
\romline{sarvataḥ samplutodake} \\
\romline{tāvānsarveṣu vedeṣu} \\
\romline{brāhmaṇasya vijānataḥ}
\end{tabular}
\end{table}

\subsection*{2.47}
\begin{table}[H]
\begin{tabular}{l}
\romline{karmaṇyevādhikāraste} \\
\romline{mā phaleṣu kadācana} \\
\romline{mā karma-phala-heturbhūḥ} \\
\romline{mā te saṅgo'stvakarmaṇi}
\end{tabular}
\end{table}

\subsection*{2.48}
\begin{table}[H]
\begin{tabular}{l}
\romline{yogasthaḥ kuru karmāṇi} \\
\romline{saṅgaṃ tyaktvā dhanañjaya} \\
\romline{siddhyasiddhyoḥ samo bhūtvā} \\
\romline{samatvaṃ yoga ucyate}
\end{tabular}
\end{table}

\subsection*{2.49}
\begin{table}[H]
\begin{tabular}{l}
\romline{dūreṇa hyavaraṃ karma} \\
\romline{buddhi-yogāddhanañjaya} \\
\romline{buddhau śaraṇam-anviccha} \\
\romline{kṛpaṇāḥ phalahetavaḥ}
\end{tabular}
\end{table}

\subsection*{2.50}
\begin{table}[H]
\begin{tabular}{l}
\romline{buddhi-yukto jahātīha} \\
\romline{ubhe sukṛtaduṣkṛte} \\
\romline{tasmādyogāya yujyasva} \\
\romline{yogaḥ karmasu kauśalam}
\end{tabular}
\end{table}

\subsection*{2.51}
\begin{table}[H]
\begin{tabular}{l}
\romline{karmajaṃ buddhi-yuktā hi} \\
\romline{phalaṃ tyaktvā manīṣiṇaḥ} \\
\romline{janma-bandha-vinirmuktāḥ} \\
\romline{padaṃ gacchantyanāmayam}
\end{tabular}
\end{table}

\subsection*{2.52}
\begin{table}[H]
\begin{tabular}{l}
\romline{yadā te mohakalilaṃ} \\
\romline{buddhir vyatitariṣyati} \\
\romline{tadā gantāsi nirvedaṃ} \\
\romline{śrotavyasya śrutasya ca}
\end{tabular}
\end{table}

\subsection*{2.53}
\begin{table}[H]
\begin{tabular}{l}
\romline{śruti-vipratipannā te} \\
\romline{yadā sthāsyati niścalā} \\
\romline{samādhāvacalā buddhiḥ} \\
\romline{tadā yogam-avāpsyasi}
\end{tabular}
\end{table}

\subsection*{2.54}
\begin{table}[H]
\begin{tabular}{l}
\romline{arjuna uvāca} \\
\romline{sthita-prajñasya kā bhāṣā} \\
\romline{samādhisthasya keśava} \\
\romline{sthitadhīḥ kiṃ prabhāṣeta} \\
\romline{kimāsīta vrajeta kim}
\end{tabular}
\end{table}

\subsection*{2.55}
\begin{table}[H]
\begin{tabular}{l}
\romline{śrī bhagavān uvāca} \\
\romline{prajahāti yadā kāmān} \\
\romline{sarvān-pārtha manogatān} \\
\romline{ātmanyevātmanā tuṣṭaḥ} \\
\romline{sthita-prajñas tadocyate}
\end{tabular}
\end{table}

\subsection*{2.56}
\begin{table}[H]
\begin{tabular}{l}
\romline{duḥkheṣvanudvigna-manāḥ} \\
\romline{sukheṣu vigata-spṛhaḥ} \\
\romline{vīta-rāga-bhaya-krodhaḥ} \\
\romline{sthita-dhīrmunirucyate}
\end{tabular}
\end{table}

\subsection*{2.57}
\begin{table}[H]
\begin{tabular}{l}
\romline{yaḥ sarvatrānabhi-snehaḥ} \\
\romline{tattatprāpya śubhāśubham} \\
\romline{nābhinandati na·dveṣṭi} \\
\romline{tasya prajñā pratiṣṭhitā}
\end{tabular}
\end{table}

\subsection*{2.58}
\begin{table}[H]
\begin{tabular}{l}
\romline{yadā samharate cāyaṃ} \\
\romline{kūrmo'ṅgānīva sarvaśaḥ} \\
\romline{indriyāṇīndriyārthebhyaḥ} \\
\romline{tasya prajñā pratiṣṭhitā}
\end{tabular}
\end{table}

\subsection*{2.59}
\begin{table}[H]
\begin{tabular}{l}
\romline{viṣayā vinivartante} \\
\romline{nirāhārasya dehinaḥ} \\
\romline{rasa-varjaṃ raso'pyasya} \\
\romline{paraṃ dṛṣṭvā nivartate}
\end{tabular}
\end{table}

\subsection*{2.60}
\begin{table}[H]
\begin{tabular}{l}
\romline{yatato hyapi kaunteya} \\
\romline{puruṣasya vipaścitaḥ} \\
\romline{indriyāṇi·pramāthīni} \\
\romline{haranti·prasabhaṃ manaḥ}
\end{tabular}
\end{table}

