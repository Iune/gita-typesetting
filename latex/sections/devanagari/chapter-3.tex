\subsection*{3.0}
\begin{table}[H]
\begin{tabular}{l}
\natline{ओं श्री परमात्मने नमः} \\
\natline{अथ तृतीयोऽध्यायः} \\
\natline{कर्मयोगः}
\end{tabular}
\end{table}

\subsection*{3.1}
\begin{table}[H]
\begin{tabular}{l}
\natline{अर्जुन उवाच} \\
\natline{ज्यायसी चेत्कर्मणस्ते} \\
\natline{मता बुद्धिर्जनार्दन} \\
\natline{तत्किं कर्मणि घोरे मां} \\
\natline{नियोजयसि केशव}
\end{tabular}
\end{table}

\subsection*{3.2}
\begin{table}[H]
\begin{tabular}{l}
\natline{व्यामिश्रेणेव वाक्येन} \\
\natline{बुद्धिं मोहयसीव मे} \\
\natline{तदेकं वद निश्चित्य} \\
\natline{येन श्रेयोऽहमाप्नुयाम्}
\end{tabular}
\end{table}

\subsection*{3.3}
\begin{table}[H]
\begin{tabular}{l}
\natline{स्री भगवानुवाच} \\
\natline{लोकेऽस्मिन्द्विविधा निष्ठा} \\
\natline{पुरा प्रोक्ता मयाऽनघ} \\
\natline{ज्ञानयोगेन साङ्ख्यानां} \\
\natline{कर्मयोगेन योगिनाम्}
\end{tabular}
\end{table}

\subsection*{3.4}
\begin{table}[H]
\begin{tabular}{l}
\natline{न कर्मणामनारम्भात्} \\
\natline{नैष्कर्म्यं पुरुषोऽश्नुते} \\
\natline{न च सन्न्यसनादेव} \\
\natline{सिद्धिं समधिगच्छति}
\end{tabular}
\end{table}

\subsection*{3.5}
\begin{table}[H]
\begin{tabular}{l}
\natline{न हि कश्चित्क्षणमपि} \\
\natline{जातु तिष्ठत्यकर्मकृत्} \\
\natline{कार्यते ह्यवशः कर्म} \\
\natline{सर्वः प्रकृतिजैर्गुणैः}
\end{tabular}
\end{table}

\subsection*{3.6}
\begin{table}[H]
\begin{tabular}{l}
\natline{कर्मेन्द्रियाणि संयम्य} \\
\natline{य आस्ते मनसा स्मरन्} \\
\natline{इन्द्रियार्थान्विमूढात्मा} \\
\natline{मिथ्याचारः स उच्यते}
\end{tabular}
\end{table}

\subsection*{3.7}
\begin{table}[H]
\begin{tabular}{l}
\natline{यस्त्विन्द्रियाणि मनसा} \\
\natline{नियम्यारभतेऽर्जुन} \\
\natline{कर्मेन्द्रियैः कर्मयोगम्} \\
\natline{असक्तः स विशिष्यते}
\end{tabular}
\end{table}

\subsection*{3.8}
\begin{table}[H]
\begin{tabular}{l}
\natline{नियतं कुरु कर्म त्वं} \\
\natline{कर्म ज्यायो ह्यकर्मणः} \\
\natline{शरीरयात्राऽपि च ते} \\
\natline{न प्रसिद्ध्येदकर्मणः}
\end{tabular}
\end{table}

\subsection*{3.9}
\begin{table}[H]
\begin{tabular}{l}
\natline{यज्ञार्थात्कर्मणोऽन्यत्र} \\
\natline{लोकोऽयं कर्मबन्धनः} \\
\natline{तदर्थं कर्म कौन्तेय} \\
\natline{मुक्तसङ्गः समाचर}
\end{tabular}
\end{table}

\subsection*{3.10}
\begin{table}[H]
\begin{tabular}{l}
\natline{सहयज्ञाः प्रजाः सृष्ट्वा} \\
\natline{पुरोवाच प्रजापतिः} \\
\natline{अनेन प्रसविष्यध्वं} \\
\natline{एष वोऽस्त्विष्टकामधुक्}
\end{tabular}
\end{table}

\subsection*{3.11}
\begin{table}[H]
\begin{tabular}{l}
\natline{देवान्भावयताऽनेन} \\
\natline{ते देवा भावयन्तु वः} \\
\natline{परस्परं भावयन्तः} \\
\natline{श्रेयः परमवाप्स्यथ}
\end{tabular}
\end{table}

\subsection*{3.12}
\begin{table}[H]
\begin{tabular}{l}
\natline{इष्टान्भोगान्हि वो देवाः} \\
\natline{दास्यन्ते यज्ञभाविताः} \\
\natline{तैर्दत्तानप्रदायैभ्यः} \\
\natline{यो भुङ्क्ते स्तेन एव सः}
\end{tabular}
\end{table}

\subsection*{3.13}
\begin{table}[H]
\begin{tabular}{l}
\natline{यज्ञशिष्टाशिनः सन्तः} \\
\natline{मुच्यन्ते सर्वकिल्बिषैः} \\
\natline{भुञ्जते ते त्वघं पापाः} \\
\natline{ये पचन्त्यात्मकारणात्}
\end{tabular}
\end{table}

\subsection*{3.14}
\begin{table}[H]
\begin{tabular}{l}
\natline{अन्नाद्भवन्ति भूतानि} \\
\natline{पर्जन्यादन्नसम्भवः} \\
\natline{यज्ञाद्भवति पर्जन्यः} \\
\natline{यज्ञः कर्मसमुद्भवः}
\end{tabular}
\end{table}

\subsection*{3.15}
\begin{table}[H]
\begin{tabular}{l}
\natline{कर्म ब्रह्मोद्भवं विद्धि} \\
\natline{ब्रह्माक्षरसमुद्भवम्} \\
\natline{तस्मात्सर्वगतं ब्रह्म} \\
\natline{नित्यं यज्ञे प्रतिष्ठितम्}
\end{tabular}
\end{table}

\subsection*{3.16}
\begin{table}[H]
\begin{tabular}{l}
\natline{एवं प्रवर्तितं चक्रं} \\
\natline{नानुवर्तयतीह यः} \\
\natline{अघायुरिन्द्रियारामः} \\
\natline{मोघं पार्थ स जीवति}
\end{tabular}
\end{table}

\subsection*{3.17}
\begin{table}[H]
\begin{tabular}{l}
\natline{यस्त्वात्मरतिरेव स्यात्} \\
\natline{आत्मतृप्तश्च मानवः} \\
\natline{आत्मन्येव च सन्तुष्टः} \\
\natline{तस्य कार्यं न विद्यते}
\end{tabular}
\end{table}

\subsection*{3.18}
\begin{table}[H]
\begin{tabular}{l}
\natline{नैव तस्य कृतेनार्थः} \\
\natline{नाकृतेनेह कश्चन} \\
\natline{न चास्य सर्वभूतेषु} \\
\natline{कश्चिदर्थव्यपाश्रयः}
\end{tabular}
\end{table}

\subsection*{3.19}
\begin{table}[H]
\begin{tabular}{l}
\natline{तस्मादसक्तः सततं} \\
\natline{कार्यं कर्म समाचर} \\
\natline{असक्तो ह्याचरन्कर्म} \\
\natline{परमाप्नोति पूरुषः}
\end{tabular}
\end{table}

\subsection*{3.20}
\begin{table}[H]
\begin{tabular}{l}
\natline{कर्मणैव हि संसिद्धिं} \\
\natline{आस्थिता जनकादयः} \\
\natline{लोकसङ्ग्रहमेवापि} \\
\natline{सम्पश्यन्कर्तुमर्हसि}
\end{tabular}
\end{table}

