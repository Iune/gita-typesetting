\begin{table}[H]
\begin{tabular}{cl}
 & \natline{श्री परमात्मने नमः} \\
 & \natline{अथ पञ्चदशोऽध्यायः} \\
 & \natline{पुरुषोत्तमप्रप्तियोगः}
\end{tabular}
\end{table}

\begin{table}[H]
\begin{tabular}{cl}
\textbf{15.1} & \natline{श्रीभगवान् उवाच} \\
 & \natline{ऊर्ध्वमूलमधः* शाखम्} \\
 & \natline{अश्वत्थं प्राहुरव्ययम् |} \\
 & \natline{छन्दांसि यस्य पर्णानि} \\
 & \natline{यस्तं वेद स वेदवित् ||}
\end{tabular}
\end{table}

\begin{table}[H]
\begin{tabular}{cl}
\textbf{15.2} & \natline{अधश्चोर्ध्वं प्रसृतास्तस्य शाखाः} \\
 & \natline{गुणप्रवृद्धा विषयप्रवालाः |} \\
 & \natline{अधश्च मूलान्यनुसन्ततानि} \\
 & \natline{कर्मानुबन्धीनि मनुष्यलोके ||}
\end{tabular}
\end{table}

\begin{table}[H]
\begin{tabular}{cl}
\textbf{15.3} & \natline{न रूपमस्येह तथोपलभ्यते} \\
 & \natline{नान्तो न चादिर्न च संप्रतिष्ठा |} \\
 & \natline{अश्वत्थमेनं सुविरूढमूलम्} \\
 & \natline{असङ्गशस्त्रेण दृढेन छित्त्वा ||}
\end{tabular}
\end{table}

\begin{table}[H]
\begin{tabular}{cl}
\textbf{15.4} & \natline{ततः पदं तत्परिमार्गितव्यं} \\
 & \natline{यस्मिन्गता न निवर्तन्ति भूयः |} \\
 & \natline{तमेव चाद्यं पुरुषं प्रपद्ये} \\
 & \natline{यतः प्रवृत्तिः प्रसृता पुराणी ||}
\end{tabular}
\end{table}

\begin{table}[H]
\begin{tabular}{cl}
\textbf{15.5} & \natline{निर्मानमोहा जितसङ्गदोषाः} \\
 & \natline{अध्यात्मनित्या विनिवृत्तकामाः |} \\
 & \natline{द्वन्द्वैर्विमुक्ताः सुखदुःख सञ्ज्ञैः} \\
 & \natline{गच्छन्त्यमूढाः पदमव्ययं तत् ||}
\end{tabular}
\end{table}

\begin{table}[H]
\begin{tabular}{cl}
\textbf{15.6} & \natline{न तद्भासयते सूर्यः} \\
 & \natline{न शशाङ्को न पावकः |} \\
 & \natline{यद्गत्वा न निवर्तन्ते} \\
 & \natline{तद्धाम परमं मम ||}
\end{tabular}
\end{table}

\begin{table}[H]
\begin{tabular}{cl}
\textbf{15.7} & \natline{ममैवांशो जीवलोके} \\
 & \natline{जीवभूतः सनातनः |} \\
 & \natline{मनः षष्ठानीन्द्रियाणि} \\
 & \natline{प्रकृतिस्थानि कर्षति ||}
\end{tabular}
\end{table}

\begin{table}[H]
\begin{tabular}{cl}
\textbf{15.8} & \natline{शरीरं यदवाप्नोति} \\
 & \natline{यच्चाप्युत्क्रामतीश्वरः |} \\
 & \natline{गृहीत्वैतानि संयाति} \\
 & \natline{वायुर्गन्धानिवाशयात् ||}
\end{tabular}
\end{table}

\begin{table}[H]
\begin{tabular}{cl}
\textbf{15.9} & \natline{श्रोत्रं चक्षुः स्पर्शनं च} \\
 & \natline{रसनं घ्राणमेव च |} \\
 & \natline{अधिष्ठाय मनश्चायं} \\
 & \natline{विषयानुपसेवते ||}
\end{tabular}
\end{table}

\begin{table}[H]
\begin{tabular}{cl}
\textbf{15.10} & \natline{उत्क्रामन्तं स्थितं वाऽपि} \\
 & \natline{भुञ्जानं वा गुणान्वितम् |} \\
 & \natline{विमूढा नानुपश्यन्ति} \\
 & \natline{पश्यन्ति ज्ञानचक्षुषः ||}
\end{tabular}
\end{table}

\begin{table}[H]
\begin{tabular}{cl}
\textbf{15.11} & \natline{यतन्तो योगिनश्चैनं} \\
 & \natline{पश्यन्त्यात्मन्यवस्थितम् |} \\
 & \natline{यतन्तोऽप्यकृतात्मानः} \\
 & \natline{नैनं पश्यन्त्यचेतसः ||}
\end{tabular}
\end{table}

\begin{table}[H]
\begin{tabular}{cl}
\textbf{15.12} & \natline{यदादित्यगतं तेजः} \\
 & \natline{जगद्भासयतेऽखिलम् |} \\
 & \natline{यच्चन्द्रमसि यच्चाग्नौ} \\
 & \natline{तत्तेजो विद्धि मामकम् ||}
\end{tabular}
\end{table}

\begin{table}[H]
\begin{tabular}{cl}
\textbf{15.13} & \natline{गामाविश्य च भूतानि} \\
 & \natline{धारयाम्यहमोजसा |} \\
 & \natline{पुष्णामि चौषधीः सर्वाः} \\
 & \natline{सोमो भूत्वा रसात्मकः ||}
\end{tabular}
\end{table}

\begin{table}[H]
\begin{tabular}{cl}
\textbf{15.14} & \natline{अहं वैश्वानरो भूत्वा} \\
 & \natline{प्राणिनां देहमाश्रितः |} \\
 & \natline{प्राणापानसमायुक्तः} \\
 & \natline{पचाम्यन्नं चतुर्विधम् ||}
\end{tabular}
\end{table}

\begin{table}[H]
\begin{tabular}{cl}
\textbf{15.15} & \natline{सर्वस्य चाहं हृदि सन्निविष्टः} \\
 & \natline{मत्तः स्मृतिर्ज्ञानमपोहनं च |} \\
 & \natline{वेदैश्च सर्वैरहमेव वेद्यः} \\
 & \natline{वेदान्तकृद्वेदविदेव चाहम् ||}
\end{tabular}
\end{table}

\begin{table}[H]
\begin{tabular}{cl}
\textbf{15.16} & \natline{द्वाविमौ पुरुषौ लोके} \\
 & \natline{क्षरश्चाक्षर एव च |} \\
 & \natline{क्षरः सर्वाणि भूतानि} \\
 & \natline{कूटस्थोऽक्षर उच्यते ||}
\end{tabular}
\end{table}

\begin{table}[H]
\begin{tabular}{cl}
\textbf{15.17} & \natline{उत्तमः पुरुषस्त्वन्यः} \\
 & \natline{परमात्मेत्युदाहृतः |} \\
 & \natline{यो लोकत्रयमाविश्य} \\
 & \natline{बिभर्त्यव्यय ईश्वरः ||}
\end{tabular}
\end{table}

\begin{table}[H]
\begin{tabular}{cl}
\textbf{15.18} & \natline{यस्मात्क्षरमतीतोऽहम्} \\
 & \natline{अक्षरादपि चोत्तमः |} \\
 & \natline{अतोऽस्मि लोके वेदे च} \\
 & \natline{प्रथितः पुरुषोत्तमः ||}
\end{tabular}
\end{table}

\begin{table}[H]
\begin{tabular}{cl}
\textbf{15.19} & \natline{यो मामेवमसम्मूढः} \\
 & \natline{जानाति पुरुषोत्तमम् |} \\
 & \natline{स सर्वविद्भजति मां} \\
 & \natline{सर्वभावेन भारत ||}
\end{tabular}
\end{table}

\begin{table}[H]
\begin{tabular}{cl}
\textbf{15.20} & \natline{इति गुह्यतमं शास्त्रम्} \\
 & \natline{इदमुक्तं मयाऽनघ |} \\
 & \natline{एतद्बुद्ध्वा बुद्धिमान्स्यात्} \\
 & \natline{कृतकृत्यश्च भारत ||}
\end{tabular}
\end{table}

\begin{table}[H]
\begin{tabular}{cl}
 & \natline{श्रीमद्भगवद्गीतासु उपनिषत्सु} \\
 & \natline{ब्रह्मविद्यायां योगशास्त्रे} \\
 & \natline{श्रीकृष्णार्जुन संवादे} \\
 & \natline{पुरुषोत्तमप्रप्तियोगो नाम} \\
 & \natline{पञ्चदशोध्यायः}
\end{tabular}
\end{table}

