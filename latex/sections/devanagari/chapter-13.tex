\begin{table}[H]
\begin{tabular}{cl}
 & \natline{श्री परमात्मने नमः} \\
 & \natline{अथ त्रयोदशोऽध्यायः} \\
 & \natline{क्षेत्रक्षेत्रज्ञविभागयोगः}
\end{tabular}
\end{table}

\begin{table}[H]
\begin{tabular}{cl}
\textbf{13.1} & \natline{अर्जुन उवाच} \\
 & \natline{प्रकृतिं पुरुषं चैव} \\
 & \natline{क्षेत्रं क्षेत्रज्ञमेव च |} \\
 & \natline{एतत् वेदितुमिच्छामि} \\
 & \natline{ज्ञानं ज्ञेयं च केशव ||}
\end{tabular}
\end{table}

\begin{table}[H]
\begin{tabular}{cl}
\textbf{13.2} & \natline{श्री भगवानुवाच} \\
 & \natline{इदं शरीरं कौन्तेय} \\
 & \natline{क्षेत्रमित्यभिधीयते |} \\
 & \natline{एतद्यो वेत्ति तं प्राहुः} \\
 & \natline{क्षेत्रज्ञ इति तद्विदः ||}
\end{tabular}
\end{table}

\begin{table}[H]
\begin{tabular}{cl}
\textbf{13.3} & \natline{क्षेत्रज्ञं चापि मां विद्धि} \\
 & \natline{सर्वक्षेत्रेषु भारत |} \\
 & \natline{क्षेत्रक्षेत्रज्ञयोर्ज्ञानं} \\
 & \natline{यत्तज्ज्ञानं मतं मम ||}
\end{tabular}
\end{table}

\begin{table}[H]
\begin{tabular}{cl}
\textbf{13.4} & \natline{तत्क्षेत्रं यच्च यादृक्च} \\
 & \natline{यद्विकारि यतश्च यत् |} \\
 & \natline{स च यो यत्प्रभावश्च} \\
 & \natline{तत्समासेन मे शृणु ||}
\end{tabular}
\end{table}

\begin{table}[H]
\begin{tabular}{cl}
\textbf{13.5} & \natline{ऋषिभिर्बहुधा गीतं} \\
 & \natline{छन्दोभिर्विविधैः पृथक् |} \\
 & \natline{ब्रह्मसूत्रपदैश्चैव} \\
 & \natline{हेतुमद्भिर्विनिश्चितैः ||}
\end{tabular}
\end{table}

\begin{table}[H]
\begin{tabular}{cl}
\textbf{13.6} & \natline{महाभूतान्यहङ्कारः} \\
 & \natline{बुद्धिरव्यक्तमेव च |} \\
 & \natline{इन्द्रियाणि दशैकं च} \\
 & \natline{पञ्च चेन्द्रियगोचराः ||}
\end{tabular}
\end{table}

\begin{table}[H]
\begin{tabular}{cl}
\textbf{13.7} & \natline{इच्छा द्वेषः सुखं दुःखं} \\
 & \natline{सङ्घातश्चेतना धृतिः |} \\
 & \natline{एतत्क्षेत्रं समासेन} \\
 & \natline{सविकारमुदाहृतम् ||}
\end{tabular}
\end{table}

\begin{table}[H]
\begin{tabular}{cl}
\textbf{13.8} & \natline{अमानित्वमदंभित्वम्} \\
 & \natline{अहिंसा क्षान्तिरार्जवम् |} \\
 & \natline{आचार्योपासनं शौचं} \\
 & \natline{स्थैर्यमात्मविनिग्रहः ||}
\end{tabular}
\end{table}

\begin{table}[H]
\begin{tabular}{cl}
\textbf{13.9} & \natline{इन्द्रियार्थेषु वैराग्यम्} \\
 & \natline{अनहङ्कार एव च |} \\
 & \natline{जन्ममृत्युजराव्याधि} \\
 & \natline{दुःखदोषानुदर्शनम् ||}
\end{tabular}
\end{table}

\begin{table}[H]
\begin{tabular}{cl}
\textbf{13.10} & \natline{असक्तिरनभिष्वङ्गः} \\
 & \natline{पुत्रदारगृहादिषु |} \\
 & \natline{नित्यं च समचित्तत्वम्} \\
 & \natline{इष्टानिष्टोपपत्तिषु ||}
\end{tabular}
\end{table}

\begin{table}[H]
\begin{tabular}{cl}
\textbf{13.11} & \natline{मयि चानन्ययोगेन} \\
 & \natline{भक्तिरव्यभिचारिणी |} \\
 & \natline{विविक्तदेशसेवित्वम्} \\
 & \natline{अरतिर्जनसंसदि ||}
\end{tabular}
\end{table}

\begin{table}[H]
\begin{tabular}{cl}
\textbf{13.12} & \natline{अध्यात्मज्ञाननित्यत्वं} \\
 & \natline{तत्त्वज्ञानार्थदर्शनम् |} \\
 & \natline{एतज्ज्ञानमिति प्रोक्तम्} \\
 & \natline{अज्ञानं यदतोऽन्यथा ||}
\end{tabular}
\end{table}

\begin{table}[H]
\begin{tabular}{cl}
\textbf{13.13} & \natline{ज्ञेयं यत्तत्प्रवक्ष्यामि} \\
 & \natline{यज्ज्ञात्वाऽमृतमश्नुते |} \\
 & \natline{अनादिमत्परं ब्रह्म} \\
 & \natline{न सत्तन्नासदुच्यते ||}
\end{tabular}
\end{table}

\begin{table}[H]
\begin{tabular}{cl}
\textbf{13.14} & \natline{सर्वतः पाणिपादं तत्} \\
 & \natline{सर्वतोऽक्षिशिरोमुखम् |} \\
 & \natline{सर्वतः श्रुतिमल्लोके} \\
 & \natline{सर्वमावृत्य तिष्ठति ||}
\end{tabular}
\end{table}

\begin{table}[H]
\begin{tabular}{cl}
\textbf{13.15} & \natline{सर्वेन्द्रियगुणाभासं} \\
 & \natline{सर्वेन्द्रियविवर्जितम् |} \\
 & \natline{असक्तं सर्वभृच्चैव} \\
 & \natline{निर्गुणं गुणभोक्तृ च ||}
\end{tabular}
\end{table}

\begin{table}[H]
\begin{tabular}{cl}
\textbf{13.16} & \natline{बहिरन्तश्च भूतानाम्} \\
 & \natline{अचरं चरमेव च |} \\
 & \natline{सूक्ष्मत्वात्तदविज्ञेयं} \\
 & \natline{दूरस्थं चान्तिके च तत् ||}
\end{tabular}
\end{table}

\begin{table}[H]
\begin{tabular}{cl}
\textbf{13.17} & \natline{अविभक्तं च भूतेषु} \\
 & \natline{विभक्तमिव च स्थितम् |} \\
 & \natline{भूतभर्तृ च तज्ज्ञेयं} \\
 & \natline{ग्रसिष्णु प्रभविष्णु च ||}
\end{tabular}
\end{table}

\begin{table}[H]
\begin{tabular}{cl}
\textbf{13.18} & \natline{ज्योतिषामपि तज्ज्योतिः} \\
 & \natline{तमसः परमुच्यते |} \\
 & \natline{ज्ञानं ज्ञेयं ज्ञानगम्यं} \\
 & \natline{हृदि सर्वस्य विष्ठितम् ||}
\end{tabular}
\end{table}

\begin{table}[H]
\begin{tabular}{cl}
\textbf{13.19} & \natline{इति क्षेत्रं तथा ज्ञानं} \\
 & \natline{ज्ञेयं चोक्तं समासतः |} \\
 & \natline{मद्भक्त एतद्विज्ञाय} \\
 & \natline{मद्भावायोपपद्यते ||}
\end{tabular}
\end{table}

\begin{table}[H]
\begin{tabular}{cl}
\textbf{13.20} & \natline{प्रकृतिं पुरुषं चैव} \\
 & \natline{विद्ध्यनादी उभावपि |} \\
 & \natline{विकारांश्च गुणांश्चैव} \\
 & \natline{विद्धि प्रकृतिसम्भवान् ||}
\end{tabular}
\end{table}

\begin{table}[H]
\begin{tabular}{cl}
\textbf{13.21} & \natline{कार्यकरणकर्तृत्वे} \\
 & \natline{हेतुः प्रकृतिरुच्यते |} \\
 & \natline{पुरुषः सुखदुःखानां} \\
 & \natline{भोक्तृत्वे हेतुरुच्यते ||}
\end{tabular}
\end{table}

\begin{table}[H]
\begin{tabular}{cl}
\textbf{13.22} & \natline{पुरुषः प्रकृतिस्थो हि} \\
 & \natline{भुङ्क्ते प्रकृतिजान्गुणान् |} \\
 & \natline{कारणं गुणसङ्गोऽस्य} \\
 & \natline{सदसद्योनिजन्मसु ||}
\end{tabular}
\end{table}

\begin{table}[H]
\begin{tabular}{cl}
\textbf{13.23} & \natline{उपद्रष्टाऽनुमन्ता च} \\
 & \natline{भर्ता भोक्ता महेश्वरः |} \\
 & \natline{परमात्मेति चाप्युक्तः} \\
 & \natline{देहेऽस्मिन्पुरुषः परः ||}
\end{tabular}
\end{table}

\begin{table}[H]
\begin{tabular}{cl}
\textbf{13.24} & \natline{य एवं वेत्ति पुरुषं} \\
 & \natline{प्रकृतिं च गुणैः सह |} \\
 & \natline{सर्वथा वर्तमानोऽपि} \\
 & \natline{न स भूयोऽभिजायते ||}
\end{tabular}
\end{table}

\begin{table}[H]
\begin{tabular}{cl}
\textbf{13.25} & \natline{ध्यानेनात्मनि पश्यन्ति} \\
 & \natline{केचिदात्मानमात्मना |} \\
 & \natline{अन्ये साङ्ख्येन योगेन} \\
 & \natline{कर्मयोगेन चापरे ||}
\end{tabular}
\end{table}

\begin{table}[H]
\begin{tabular}{cl}
\textbf{13.26} & \natline{अन्ये त्वेवमजानन्तः} \\
 & \natline{श्रुत्वाऽन्येभ्य उपासते |} \\
 & \natline{तेऽपि चातितरन्त्येव} \\
 & \natline{मृत्युं श्रुतिपरायणाः ||}
\end{tabular}
\end{table}

\begin{table}[H]
\begin{tabular}{cl}
\textbf{13.27} & \natline{यावत्सञ्जायते किञ्चित्} \\
 & \natline{सत्त्वं स्थावरजङ्गमम् |} \\
 & \natline{क्षेत्रक्षेत्रज्ञसंयोगात्} \\
 & \natline{तद्विद्धि भरतर्षभ ||}
\end{tabular}
\end{table}

\begin{table}[H]
\begin{tabular}{cl}
\textbf{13.28} & \natline{समं सर्वेषु भूतेषु} \\
 & \natline{तिष्ठन्तं परमेश्वरम् |} \\
 & \natline{विनश्यत्स्वविनश्यन्तं} \\
 & \natline{यः पश्यति स पश्यति ||}
\end{tabular}
\end{table}

\begin{table}[H]
\begin{tabular}{cl}
\textbf{13.29} & \natline{समं पश्यन्हि सर्वत्र} \\
 & \natline{समवस्थितमीश्वरम् |} \\
 & \natline{न हिनस्त्यात्मनाऽऽत्मानं} \\
 & \natline{ततो याति परां गतिम् ||}
\end{tabular}
\end{table}

\begin{table}[H]
\begin{tabular}{cl}
\textbf{13.30} & \natline{प्रकृत्यैव च कर्माणि} \\
 & \natline{क्रियमाणानि सर्वशः |} \\
 & \natline{यः पश्यति तथाऽऽत्मानम्} \\
 & \natline{अकर्तारं स पश्यति ||}
\end{tabular}
\end{table}

\begin{table}[H]
\begin{tabular}{cl}
\textbf{13.31} & \natline{यदा भूतपृथग्भावम्} \\
 & \natline{एकस्थमनुपश्यति |} \\
 & \natline{तत एव च विस्तारं} \\
 & \natline{ब्रह्म सम्पद्यते तदा ||}
\end{tabular}
\end{table}

\begin{table}[H]
\begin{tabular}{cl}
\textbf{13.32} & \natline{अनादित्वान्निर्गुणत्वात्} \\
 & \natline{परमात्मायमव्ययः |} \\
 & \natline{शरीरस्थोऽपि कौन्तेय} \\
 & \natline{न करोति न लिप्यते ||}
\end{tabular}
\end{table}

\begin{table}[H]
\begin{tabular}{cl}
\textbf{13.33} & \natline{यथा सर्वगतं सौक्ष्म्यात्} \\
 & \natline{आकाशं नोपलिप्यते |} \\
 & \natline{सर्वत्रावस्थितो देहे} \\
 & \natline{तथाऽऽत्मा नोपलिप्यते ||}
\end{tabular}
\end{table}

\begin{table}[H]
\begin{tabular}{cl}
\textbf{13.34} & \natline{यथा प्रकाशयत्येकः} \\
 & \natline{कृत्स्नं लोकमिमं रविः |} \\
 & \natline{क्षेत्रं क्षेत्री तथा कृत्स्नं} \\
 & \natline{प्रकाशयति भारत ||}
\end{tabular}
\end{table}

\begin{table}[H]
\begin{tabular}{cl}
\textbf{13.35} & \natline{क्षेत्रक्षेत्रज्ञयोरेवम्} \\
 & \natline{अन्तरं ज्ञानचक्षुषा |} \\
 & \natline{भूतप्रकृतिमोक्षं च} \\
 & \natline{ये विदुर्यान्ति ते परम् ||}
\end{tabular}
\end{table}

\begin{table}[H]
\begin{tabular}{cl}
 & \natline{श्रीमद्भगवद्गीतासु उपनिषत्सु} \\
 & \natline{ब्रह्मविद्यायां योगशास्त्रे} \\
 & \natline{श्रीकृष्णार्जुन संवादे} \\
 & \natline{क्षेत्रक्षेत्रज्ञविभागयोगो नाम} \\
 & \natline{त्रयोदशोध्यायः}
\end{tabular}
\end{table}

