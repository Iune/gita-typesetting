\begin{table}[H]
\begin{tabular}{cl}
\textbf{13.0} & \natline{ओं श्री परमात्मने नमः} \\
 & \natline{अथ त्रयोदशोऽध्यायः} \\
 & \natline{क्षेत्रक्षेत्रज्ञविभागयोगः}
\end{tabular}
\end{table}

\begin{table}[H]
\begin{tabular}{cl}
\textbf{13.1} & \natline{अर्जुन उवाच} \\
 & \natline{प्रकृतिं पुरुषं चैव} \\
 & \natline{क्षेत्रं क्षेत्रज्ञमेव च |} \\
 & \natline{एतत् वेदितुमिच्छामि} \\
 & \natline{ज्ञानं ज्ञेयं च केशव ||}
\end{tabular}
\end{table}

\begin{table}[H]
\begin{tabular}{cl}
\textbf{13.2} & \natline{श्री भगवानुवाच} \\
 & \natline{इदं शरीरं कौन्तेय} \\
 & \natline{क्षेत्रमित्यभिधीयते |} \\
 & \natline{एतद्यो वेत्ति तं प्राहुः} \\
 & \natline{क्षेत्रज्ञ इति तद्विदः ||}
\end{tabular}
\end{table}

\begin{table}[H]
\begin{tabular}{cl}
\textbf{13.3} & \natline{क्षेत्रज्ञं चापि मां विद्धि} \\
 & \natline{सर्वक्षेत्रेषु भारत |} \\
 & \natline{क्षेत्रक्षेत्रज्ञयोर्ज्ञानं} \\
 & \natline{यत्तज्ज्ञानं मतं मम ||}
\end{tabular}
\end{table}

\begin{table}[H]
\begin{tabular}{cl}
\textbf{13.4} & \natline{तत्क्षेत्रं यच्च यादृक्च} \\
 & \natline{यद्विकारि यतश्च यत् |} \\
 & \natline{स च यो यत्प्रभावश्च} \\
 & \natline{तत्समासेन मे शृणु ||}
\end{tabular}
\end{table}

\begin{table}[H]
\begin{tabular}{cl}
\textbf{13.5} & \natline{ऋषिभिर्बहुधा गीतं} \\
 & \natline{छन्दोभिर्विविधैः पृथक् |} \\
 & \natline{ब्रह्मसूत्रपदैश्चैव} \\
 & \natline{हेतुमद्भिर्विनिश्चितैः ||}
\end{tabular}
\end{table}

