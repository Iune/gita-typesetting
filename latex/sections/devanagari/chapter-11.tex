\begin{table}[H]
\begin{tabular}{cl}
\textbf{11.0} & \natline{ओं श्री परमात्मने नमः} \\
 & \natline{अथ एकादशोऽध्यायः} \\
 & \natline{विश्वरूप सन्दर्शन योगः}
\end{tabular}
\end{table}

\begin{table}[H]
\begin{tabular}{cl}
\textbf{11.1} & \natline{अर्जुन उवाच} \\
 & \natline{मदनुग्रहाय परमं} \\
 & \natline{गुह्यमध्यात्मसञ्ज्ञितम् |} \\
 & \natline{यत्त्वयोक्तं वचस्तेन} \\
 & \natline{मोहोऽयं विगतो मम ||}
\end{tabular}
\end{table}

\begin{table}[H]
\begin{tabular}{cl}
\textbf{11.2} & \natline{भवाप्ययौ हि भूतानां} \\
 & \natline{श्रुतौ विस्तरशो मया |} \\
 & \natline{त्वत्तः कमलपत्राक्ष} \\
 & \natline{माहात्म्यमपि चाव्ययम् ||}
\end{tabular}
\end{table}

\begin{table}[H]
\begin{tabular}{cl}
\textbf{11.3} & \natline{एवमेतद्यथाऽऽत्थ त्वम्} \\
 & \natline{आत्मानं परमेश्वर |} \\
 & \natline{द्रष्टुमिच्छामि ते रूपम्} \\
 & \natline{ऐश्वरम् पुरुषोत्तम ||}
\end{tabular}
\end{table}

\begin{table}[H]
\begin{tabular}{cl}
\textbf{11.4} & \natline{मन्यसे यदि तच्छक्यं} \\
 & \natline{मया द्रष्टुमिति प्रभो |} \\
 & \natline{योगेश्वर ततो मे त्वं} \\
 & \natline{दर्शयात्मानमव्ययम् ||}
\end{tabular}
\end{table}

\begin{table}[H]
\begin{tabular}{cl}
\textbf{11.5} & \natline{श्री भगवानुवाच} \\
 & \natline{पश्य मे पार्थ रूपाणि} \\
 & \natline{शतशोऽथ सहस्रशः |} \\
 & \natline{नानाविधानि दिव्यानि} \\
 & \natline{नानावर्णाकृतीनि च ||}
\end{tabular}
\end{table}

\begin{table}[H]
\begin{tabular}{cl}
\textbf{11.6} & \natline{पश्यादित्यान्वसून्रुद्रान्} \\
 & \natline{अश्विनौ मरुतस्तथा |} \\
 & \natline{बहून्यदृष्टपूर्वाणि} \\
 & \natline{पश्याश्चर्याणि भारत ||}
\end{tabular}
\end{table}

\begin{table}[H]
\begin{tabular}{cl}
\textbf{11.7} & \natline{इहैकस्थं जगत्कृत्स्नं} \\
 & \natline{पश्याद्य सचराचरम् |} \\
 & \natline{मम देहे गुडाकेश} \\
 & \natline{यच्चान्यत् द्रष्टुमिच्छसि ||}
\end{tabular}
\end{table}

\begin{table}[H]
\begin{tabular}{cl}
\textbf{11.8} & \natline{न तु मां शक्यसे द्रष्टुम्} \\
 & \natline{अनेनैव स्वचक्षुषा |} \\
 & \natline{दिव्यं ददामि ते चक्षुः} \\
 & \natline{पश्य मे योगमैश्वरम् ||}
\end{tabular}
\end{table}

\begin{table}[H]
\begin{tabular}{cl}
\textbf{11.9} & \natline{सञ्जय उवाच} \\
 & \natline{एवमुक्त्वा ततो राजन्} \\
 & \natline{महायोगेश्वरो हरिः |} \\
 & \natline{दर्शयामास पार्थाय} \\
 & \natline{परमं रूपमैश्वरम् ||}
\end{tabular}
\end{table}

\begin{table}[H]
\begin{tabular}{cl}
\textbf{11.10} & \natline{अनेकवक्त्रनयनम्} \\
 & \natline{अनेकाद्भुतदर्शनम् |} \\
 & \natline{अनेकदिव्याभरणं} \\
 & \natline{दिव्यानेकोद्यतायुधम् ||}
\end{tabular}
\end{table}

\begin{table}[H]
\begin{tabular}{cl}
\textbf{11.11} & \natline{दिव्यमाल्याम्बरधरं} \\
 & \natline{दिव्यगन्धानुलेपनम् |} \\
 & \natline{सर्वाश्चर्यमयं देवम्} \\
 & \natline{अनन्तं विश्वतोमुखम् ||}
\end{tabular}
\end{table}

\begin{table}[H]
\begin{tabular}{cl}
\textbf{11.12} & \natline{दिवि सूर्यसहस्रस्य} \\
 & \natline{भवेद्युगपदुत्थिता |} \\
 & \natline{यदि भाः सदृशी सा स्यात्} \\
 & \natline{भासस्तस्य महात्मनः ||}
\end{tabular}
\end{table}

\begin{table}[H]
\begin{tabular}{cl}
\textbf{11.13} & \natline{तत्रैकस्थं जगत्कृत्स्नं} \\
 & \natline{प्रविभक्तमनेकधा |} \\
 & \natline{अपश्यद्देवदेवस्य} \\
 & \natline{शरीरे पाण्डवस्तदा ||}
\end{tabular}
\end{table}

\begin{table}[H]
\begin{tabular}{cl}
\textbf{11.14} & \natline{ततः स विस्मयाविष्टः} \\
 & \natline{हृष्टरोमा धनञ्जयः |} \\
 & \natline{प्रणम्य शिरसा देवं} \\
 & \natline{कृताञ्जलिरभाषत ||}
\end{tabular}
\end{table}

\begin{table}[H]
\begin{tabular}{cl}
\textbf{11.15} & \natline{अर्जुन उवाच} \\
 & \natline{पश्यामि देवांस्तव देव देहे} \\
 & \natline{सर्वांस्तथा भूतविशेषसङ्घान् |} \\
 & \natline{ब्रह्माणमीशं कमलासनस्थम्} \\
 & \natline{ऋषींश्च सर्वानुरगांश्च दिव्यान् ||}
\end{tabular}
\end{table}

\begin{table}[H]
\begin{tabular}{cl}
\textbf{11.16} & \natline{अनेकबाहूदरवक्त्रनेत्रं} \\
 & \natline{पश्यामि त्वा सर्वतोऽनन्तरूपम् |} \\
 & \natline{नान्तं न मध्यं न पुनस्तवादिं} \\
 & \natline{पश्यामि विश्वेश्वर विश्वरूप ||}
\end{tabular}
\end{table}

\begin{table}[H]
\begin{tabular}{cl}
\textbf{11.17} & \natline{किरीटिनं गदिनं चक्रिणं च} \\
 & \natline{तेजोराशिं सर्वतो दीप्तिमन्तम् |} \\
 & \natline{पश्यामि त्वां दुर्निरीक्ष्यं समन्तात्} \\
 & \natline{दीप्तानलार्कद्युतिमप्रमेयम् ||}
\end{tabular}
\end{table}

\begin{table}[H]
\begin{tabular}{cl}
\textbf{11.18} & \natline{त्वमक्षरं परमं वेदितव्यं} \\
 & \natline{त्वमस्य विश्वस्य परं निधानम् |} \\
 & \natline{त्वमव्ययः शाश्वतधर्मगोप्ता} \\
 & \natline{सनातनस्त्वं पुरुषो मतो मे ||}
\end{tabular}
\end{table}

\begin{table}[H]
\begin{tabular}{cl}
\textbf{11.19} & \natline{अनादिमध्यान्तमनन्तवीर्यम्} \\
 & \natline{अनन्तबाहुं शशिसूर्यनेत्रम् |} \\
 & \natline{पश्यामि त्वां दीप्तहुताशवक्त्रं} \\
 & \natline{स्वतेजसा विश्वमिदं तपन्तम् ||}
\end{tabular}
\end{table}

\begin{table}[H]
\begin{tabular}{cl}
\textbf{11.20} & \natline{द्यावापृथिव्योरिदमन्तरं हि} \\
 & \natline{व्याप्तं त्वयैकेन दिशश्च सर्वाः |} \\
 & \natline{दृष्ट्वाद्भुतं रूपमिदं तवोग्रं} \\
 & \natline{लोकत्रयं प्रव्यथितं महात्मन् ||}
\end{tabular}
\end{table}

\begin{table}[H]
\begin{tabular}{cl}
\textbf{11.21} & \natline{अमी हि त्वा सुरसङ्घा विशन्ति} \\
 & \natline{केचिद्भीताः प्राञ्जलयो गृणन्ति |} \\
 & \natline{स्वस्तीत्युक्त्वा महर्षिसिद्धसङ्घाः} \\
 & \natline{स्तुवन्ति त्वां स्तुतिभिः पुष्कलाभिः ||}
\end{tabular}
\end{table}

\begin{table}[H]
\begin{tabular}{cl}
\textbf{11.22} & \natline{रुद्रादित्या वसवो ये च साध्याः} \\
 & \natline{विश्वेऽश्विनौ मरुतश्चोष्मपाश्च |} \\
 & \natline{गन्धर्वयक्षासुरसिद्धसङ्घाः} \\
 & \natline{वीक्षन्ते त्वां विस्मिताश्चैव सर्वे ||}
\end{tabular}
\end{table}

\begin{table}[H]
\begin{tabular}{cl}
\textbf{11.23} & \natline{रूपं महत्ते बहुवक्त्र नेत्रं} \\
 & \natline{महाबाहो बहुबाहूरुपादम् |} \\
 & \natline{बहूदरं बहुदंष्ट्राकरालं} \\
 & \natline{दृष्ट्वा लोकाः प्रव्यथितास्तथाऽहम् ||}
\end{tabular}
\end{table}

\begin{table}[H]
\begin{tabular}{cl}
\textbf{11.24} & \natline{नभः स्पृशं दीप्तमनेकवर्णं} \\
 & \natline{व्यात्ताननं दीप्तविशालनेत्रम् |} \\
 & \natline{दृष्ट्वा हि त्वां प्रव्यथितान्तरात्मा} \\
 & \natline{धृतिं न विन्दामि शमं च विष्णो ||}
\end{tabular}
\end{table}

\begin{table}[H]
\begin{tabular}{cl}
\textbf{11.25} & \natline{दंष्ट्राकरालानि च ते मुखानि} \\
 & \natline{दृष्ट्वैव कालानलसन्निभानि |} \\
 & \natline{दिशो न जाने न लभे च शर्म} \\
 & \natline{प्रसीद देवेश जगन्निवास ||}
\end{tabular}
\end{table}

\begin{table}[H]
\begin{tabular}{cl}
\textbf{11.26} & \natline{अमी च त्वां धृतराष्ट्रस्य पुत्राः} \\
 & \natline{सर्वे सहैवावनिपालसङ्घैः |} \\
 & \natline{भीष्मो द्रोणः सूतपुत्रस्तथाऽसौ} \\
 & \natline{सहास्मदीयैरपि योधमुख्यैः ||}
\end{tabular}
\end{table}

\begin{table}[H]
\begin{tabular}{cl}
\textbf{11.27} & \natline{वक्त्राणि ते त्वरमाणा विशन्ति} \\
 & \natline{दंष्ट्राकरालानि भयानकानि |} \\
 & \natline{केचिद्विलग्ना दशनान्तरेषु} \\
 & \natline{सन्दृश्यन्ते चूर्णितैरुत्तमाङ्गैः ||}
\end{tabular}
\end{table}

\begin{table}[H]
\begin{tabular}{cl}
\textbf{11.28} & \natline{यथा नदीनां बहवोऽम्बुवेगाः} \\
 & \natline{समुद्रमेवाभिमुखा द्रवन्ति |} \\
 & \natline{तथा तवामी नरलोकवीराः} \\
 & \natline{विशन्ति वक्त्राण्यभिविज्वलन्ति ||}
\end{tabular}
\end{table}

\begin{table}[H]
\begin{tabular}{cl}
\textbf{11.29} & \natline{यथा प्रदीप्तं ज्वलनं पतङ्गाः} \\
 & \natline{विशन्ति नाशाय समृद्धवेगाः |} \\
 & \natline{तथैव नाशाय विशन्ति लोकाः} \\
 & \natline{तवापि वक्त्राणि समृद्धवेगाः ||}
\end{tabular}
\end{table}

\begin{table}[H]
\begin{tabular}{cl}
\textbf{11.30} & \natline{लेलिह्यसे ग्रसमानः समन्तात्} \\
 & \natline{लोकान्समग्रान्वदनैर्ज्वलद्भिः |} \\
 & \natline{तेजोभिरापूर्य जगत्समग्रं} \\
 & \natline{भासस्तवोग्राः प्रतपन्ति विष्णो ||}
\end{tabular}
\end{table}

\begin{table}[H]
\begin{tabular}{cl}
\textbf{11.31} & \natline{आख्याहि मे को भवानुग्ररूपः} \\
 & \natline{नमोऽस्तु ते देववर प्रसीद |} \\
 & \natline{विज्ञातुमिच्छामि भवन्तमाद्यं} \\
 & \natline{न हि प्रजानामि तव प्रवृत्तिम् ||}
\end{tabular}
\end{table}

\begin{table}[H]
\begin{tabular}{cl}
\textbf{11.32} & \natline{श्री भगवानुवाच} \\
 & \natline{कालोऽस्मि लोकक्षयकृत्प्रवृद्धः} \\
 & \natline{लोकान्समाहर्तुमिह प्रवृत्तः |} \\
 & \natline{ऋतेऽपि त्वा न भविष्यन्ति सर्वे} \\
 & \natline{येऽवस्थिताः प्रत्यनीकेषु योधाः ||}
\end{tabular}
\end{table}

\begin{table}[H]
\begin{tabular}{cl}
\textbf{11.33} & \natline{तस्मात्त्वमुत्तिष्ठ यशो लभस्व} \\
 & \natline{जित्वा शत्रून्भुङ्क्ष्व राज्यं समृद्धम् |} \\
 & \natline{मयैवैते निहताः पूर्वमेव} \\
 & \natline{निमित्तमात्रं भव सव्यसाचिन् ||}
\end{tabular}
\end{table}

\begin{table}[H]
\begin{tabular}{cl}
\textbf{11.34} & \natline{द्रोणं च भीष्मं च जयद्रथं च} \\
 & \natline{कर्णं तथान्यानपि योधवीरान् |} \\
 & \natline{मया हतांस्त्वं जहि मा व्यथिष्ठाः} \\
 & \natline{युध्यस्व जेतासि रणे सपत्नान् ||}
\end{tabular}
\end{table}

\begin{table}[H]
\begin{tabular}{cl}
\textbf{11.35} & \natline{सञ्जय उवाच} \\
 & \natline{एतच्छ्रुत्वा वचनं केशवस्य} \\
 & \natline{कृताञ्जलिर्वेपमानः किरीटी |} \\
 & \natline{नमस्कृत्वा भूय एवाह कृष्णं} \\
 & \natline{सगद्गदं भीतभीतः प्रणम्य ||}
\end{tabular}
\end{table}

\begin{table}[H]
\begin{tabular}{cl}
\textbf{11.36} & \natline{अर्जुन उवाच} \\
 & \natline{स्थाने हृषीकेश तव प्रकीर्त्या} \\
 & \natline{जगत्प्रहृष्यत्यनुरज्यते च |} \\
 & \natline{रक्षांसि भीतानि दिशो द्रवन्ति} \\
 & \natline{सर्वे नमस्यन्ति च सिद्धसङ्घाः ||}
\end{tabular}
\end{table}

\begin{table}[H]
\begin{tabular}{cl}
\textbf{11.37} & \natline{कस्माच्च ते न नमेरन्महात्मन्} \\
 & \natline{गरीयसे ब्रह्मणोऽप्यादिकर्त्रे |} \\
 & \natline{अनन्त देवेश जगन्निवास} \\
 & \natline{त्वमक्षरं सदसत्तत्परं यत् ||}
\end{tabular}
\end{table}

\begin{table}[H]
\begin{tabular}{cl}
\textbf{11.38} & \natline{त्वमादिदेवः पुरुषः पुराणः} \\
 & \natline{त्वमस्य विश्वस्य परं निधानम् |} \\
 & \natline{वेत्ताऽसि वेद्यं च परं च धाम} \\
 & \natline{त्वया ततं विश्वमनन्तरूप ||}
\end{tabular}
\end{table}

\begin{table}[H]
\begin{tabular}{cl}
\textbf{11.39} & \natline{वायुर्यमोऽग्निर्वरुणः शशाङ्कः} \\
 & \natline{प्रजापतिस्त्वं प्रपितामहश्च |} \\
 & \natline{नमो नमस्तेऽस्तु सहस्रकृत्वः} \\
 & \natline{पुनश्च भूयोऽपि नमो नमस्ते ||}
\end{tabular}
\end{table}

\begin{table}[H]
\begin{tabular}{cl}
\textbf{11.40} & \natline{नमः पुरस्तादथ पृष्ठतस्ते} \\
 & \natline{नमोऽस्तु ते सर्वत एव सर्व |} \\
 & \natline{अनन्तवीर्यामितविक्रमस्त्वं} \\
 & \natline{सर्वं समाप्नोषि ततोऽसि सर्वः ||}
\end{tabular}
\end{table}

\begin{table}[H]
\begin{tabular}{cl}
\textbf{11.41} & \natline{सखेति मत्वा प्रसभं यदुक्तं} \\
 & \natline{हे कृष्ण हे यादव हे सखेति |} \\
 & \natline{अजानता महिमानं तवेदं} \\
 & \natline{मया प्रमादात्प्रणयेन वाऽपि ||}
\end{tabular}
\end{table}

\begin{table}[H]
\begin{tabular}{cl}
\textbf{11.42} & \natline{यच्चापहासार्थमसत्कृतोऽसि} \\
 & \natline{विहारशय्यासनभोजनेषु |} \\
 & \natline{एकोऽथवाप्यच्युत तत्समक्षं} \\
 & \natline{तत्क्षामये त्वामहमप्रमेयम् ||}
\end{tabular}
\end{table}

\begin{table}[H]
\begin{tabular}{cl}
\textbf{11.43} & \natline{पितासि लोकस्य चराचरस्य} \\
 & \natline{त्वमस्य पूज्यश्च गुरुर्गरीयान् |} \\
 & \natline{न त्वत्समोऽस्त्यभ्यधिकः कुतोऽन्यः} \\
 & \natline{लोकत्रयेऽप्यप्रतिमप्रभाव ||}
\end{tabular}
\end{table}

\begin{table}[H]
\begin{tabular}{cl}
\textbf{11.44} & \natline{तस्मात्प्रणम्य प्रणिधाय कायं} \\
 & \natline{प्रसादये त्वामहमीशमीड्यम् |} \\
 & \natline{पितेव पुत्रस्य सखेव सख्युः} \\
 & \natline{प्रियः प्रियायार्हसि देव सोढुम् ||}
\end{tabular}
\end{table}

\begin{table}[H]
\begin{tabular}{cl}
\textbf{11.45} & \natline{अदृष्टपूर्वं हृषितोऽस्मि दृष्ट्वा} \\
 & \natline{भयेन च प्रव्यथितं मनो मे |} \\
 & \natline{तदेव मे दर्शय देवरूपं} \\
 & \natline{प्रसीद देवेश जगन्निवास ||}
\end{tabular}
\end{table}

\begin{table}[H]
\begin{tabular}{cl}
\textbf{11.46} & \natline{किरीटिनं गदिनं चक्रहस्तम्} \\
 & \natline{इच्छामि त्वां द्रष्टुमहं तथैव |} \\
 & \natline{तेनैव रूपेण चतुर्भुजेन} \\
 & \natline{सहस्रबाहो भव विश्वमूर्ते ||}
\end{tabular}
\end{table}

\begin{table}[H]
\begin{tabular}{cl}
\textbf{11.47} & \natline{श्री भगवानुवाच} \\
 & \natline{मया प्रसन्नेन तवार्जुनेदं} \\
 & \natline{रूपं परं दर्शितमात्मयोगात् |} \\
 & \natline{तेजोमयं विश्वमनन्तमाद्यं} \\
 & \natline{यन्मे त्वदन्येन न दृष्टपूर्वम् ||}
\end{tabular}
\end{table}

\begin{table}[H]
\begin{tabular}{cl}
\textbf{11.48} & \natline{न वेदयज्ञाध्ययनैर्न दानैः} \\
 & \natline{न च क्रियाभिर्न तपोभिरुग्रैः |} \\
 & \natline{एवंरूपः शक्य अहं नृलोके} \\
 & \natline{द्रष्टुं त्वदन्येन कुरुप्रवीर ||}
\end{tabular}
\end{table}

\begin{table}[H]
\begin{tabular}{cl}
\textbf{11.49} & \natline{मा ते व्यथा मा च विमूढभावः} \\
 & \natline{दृष्ट्वा रूपं घोरमीदृङ्ममेदम् |} \\
 & \natline{व्यपेतभीः प्रीतमनाः पुनस्त्वं} \\
 & \natline{तदेव मे रूपमिदं प्रपश्य ||}
\end{tabular}
\end{table}

\begin{table}[H]
\begin{tabular}{cl}
\textbf{11.50} & \natline{सञ्जय उवाच} \\
 & \natline{इत्यर्जुनं वासुदेवस्तथोक्त्वा} \\
 & \natline{स्वकं रूपं दर्शयामास भूयः |} \\
 & \natline{आश्वासयामास च भीतमेनं} \\
 & \natline{भूत्वा पुनः सौम्यवपुर्महात्मा ||}
\end{tabular}
\end{table}

\begin{table}[H]
\begin{tabular}{cl}
\textbf{11.51} & \natline{अर्जुन उवाच} \\
 & \natline{दृष्ट्वेदं मानुषं रूपं} \\
 & \natline{तव सौम्यं जनार्दन |} \\
 & \natline{इदानीमस्मि संवृत्तः} \\
 & \natline{सचेताः प्रकृतिं गतः ||}
\end{tabular}
\end{table}

\begin{table}[H]
\begin{tabular}{cl}
\textbf{11.52} & \natline{श्री भगवानुवाच} \\
 & \natline{सुदुर्दर्शमिदं रूपं} \\
 & \natline{दृष्टवानसि यन्मम |} \\
 & \natline{देवा अप्यस्य रूपस्य} \\
 & \natline{नित्यं दर्शनकाङ्क्षिणः ||}
\end{tabular}
\end{table}

\begin{table}[H]
\begin{tabular}{cl}
\textbf{11.53} & \natline{नाहं वेदैर्न तपसा} \\
 & \natline{न दानेन न चेज्यया |} \\
 & \natline{शक्य एवंविधो द्रष्टुं} \\
 & \natline{दृष्टवानसि मां यथा ||}
\end{tabular}
\end{table}

\begin{table}[H]
\begin{tabular}{cl}
\textbf{11.54} & \natline{भक्त्या त्वनन्यया शक्यः} \\
 & \natline{अहमेवंविधोऽर्जुन |} \\
 & \natline{ज्ञातुं द्रष्टुं च तत्त्वेन} \\
 & \natline{प्रवेष्टुं च परन्तप ||}
\end{tabular}
\end{table}

\begin{table}[H]
\begin{tabular}{cl}
\textbf{11.55} & \natline{मत्कर्मकृन्मत्परमः} \\
 & \natline{मद्भक्तः सङ्गवर्जितः |} \\
 & \natline{निर्वैरः सर्वभूतेषु} \\
 & \natline{यः स मामेति पाण्डव ||}
\end{tabular}
\end{table}

