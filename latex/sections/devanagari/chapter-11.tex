\begin{table}[H]
\begin{tabular}{cl}
\textbf{11.0} & \natline{ओं श्री परमात्मने नमः} \\
 & \natline{अथ एकादशोऽध्यायः} \\
 & \natline{विश्वरूप सन्दर्शन योगः}
\end{tabular}
\end{table}

\begin{table}[H]
\begin{tabular}{cl}
\textbf{11.1} & \natline{अर्जुन उवाच} \\
 & \natline{मदनुग्रहाय परमं} \\
 & \natline{गुह्यमध्यात्मसञ्ज्ञितम् |} \\
 & \natline{यत्त्वयोक्तं वचस्तेन} \\
 & \natline{मोहोऽयं विगतो मम ||}
\end{tabular}
\end{table}

\begin{table}[H]
\begin{tabular}{cl}
\textbf{11.2} & \natline{भवाप्ययौ हि भूतानां} \\
 & \natline{श्रुतौ विस्तरशो मया |} \\
 & \natline{त्वत्तः कमलपत्राक्ष} \\
 & \natline{माहात्म्यमपि चाव्ययम् ||}
\end{tabular}
\end{table}

\begin{table}[H]
\begin{tabular}{cl}
\textbf{11.3} & \natline{एवमेतद्यथाऽऽत्थ त्वम्} \\
 & \natline{आत्मानं परमेश्वर |} \\
 & \natline{द्रष्टुमिच्छामि ते रूपम्} \\
 & \natline{ऐश्वरम् पुरुषोत्तम ||}
\end{tabular}
\end{table}

\begin{table}[H]
\begin{tabular}{cl}
\textbf{11.4} & \natline{मन्यसे यदि तच्छक्यं} \\
 & \natline{मया द्रष्टुमिति प्रभो |} \\
 & \natline{योगेश्वर ततो मे त्वं} \\
 & \natline{दर्शयात्मानमव्ययम् ||}
\end{tabular}
\end{table}

\begin{table}[H]
\begin{tabular}{cl}
\textbf{11.5} & \natline{श्री भगवानुवाच} \\
 & \natline{पश्य मे पार्थ रूपाणि} \\
 & \natline{शतशोऽथ सहस्रशः |} \\
 & \natline{नानाविधानि दिव्यानि} \\
 & \natline{नानावर्णाकृतीनि च ||}
\end{tabular}
\end{table}

\begin{table}[H]
\begin{tabular}{cl}
\textbf{11.6} & \natline{पश्यादित्यान्वसून्रुद्रान्} \\
 & \natline{अश्विनौ मरुतस्तथा |} \\
 & \natline{बहून्यदृष्टपूर्वाणि} \\
 & \natline{पश्याश्चर्याणि भारत ||}
\end{tabular}
\end{table}

\begin{table}[H]
\begin{tabular}{cl}
\textbf{11.7} & \natline{इहैकस्थं जगत्कृत्स्नं} \\
 & \natline{पश्याद्य सचराचरम् |} \\
 & \natline{मम देहे गुडाकेश} \\
 & \natline{यच्चान्यत् द्रष्टुमिच्छसि ||}
\end{tabular}
\end{table}

\begin{table}[H]
\begin{tabular}{cl}
\textbf{11.8} & \natline{न तु मां शक्यसे द्रष्टुम्} \\
 & \natline{अनेनैव स्वचक्षुषा |} \\
 & \natline{दिव्यं ददामि ते चक्षुः} \\
 & \natline{पश्य मे योगमैश्वरम् ||}
\end{tabular}
\end{table}

\begin{table}[H]
\begin{tabular}{cl}
\textbf{11.9} & \natline{सञ्जय उवाच} \\
 & \natline{एवमुक्त्वा ततो राजन्} \\
 & \natline{महायोगेश्वरो हरिः |} \\
 & \natline{दर्शयामास पार्थाय} \\
 & \natline{परमं रूपमैश्वरम् ||}
\end{tabular}
\end{table}

\begin{table}[H]
\begin{tabular}{cl}
\textbf{11.10} & \natline{अनेकवक्त्रनयनम्} \\
 & \natline{अनेकाद्भुतदर्शनम् |} \\
 & \natline{अनेकदिव्याभरणं} \\
 & \natline{दिव्यानेकोद्यतायुधम् ||}
\end{tabular}
\end{table}

\begin{table}[H]
\begin{tabular}{cl}
\textbf{11.11} & \natline{दिव्यमाल्याम्बरधरं} \\
 & \natline{दिव्यगन्धानुलेपनम् |} \\
 & \natline{सर्वाश्चर्यमयं देवम्} \\
 & \natline{अनन्तं विश्वतोमुखम् ||}
\end{tabular}
\end{table}

\begin{table}[H]
\begin{tabular}{cl}
\textbf{11.12} & \natline{दिवि सूर्यसहस्रस्य} \\
 & \natline{भवेद्युगपदुत्थिता |} \\
 & \natline{यदि भाः सदृशी सा स्यात्} \\
 & \natline{भासस्तस्य महात्मनः ||}
\end{tabular}
\end{table}

\begin{table}[H]
\begin{tabular}{cl}
\textbf{11.13} & \natline{तत्रैकस्थं जगत्कृत्स्नं} \\
 & \natline{प्रविभक्तमनेकधा |} \\
 & \natline{अपश्यद्देवदेवस्य} \\
 & \natline{शरीरे पाण्डवस्तदा ||}
\end{tabular}
\end{table}

\begin{table}[H]
\begin{tabular}{cl}
\textbf{11.14} & \natline{ततः स विस्मयाविष्टः} \\
 & \natline{हृष्टरोमा धनञ्जयः |} \\
 & \natline{प्रणम्य शिरसा देवं} \\
 & \natline{कृताञ्जलिरभाषत ||}
\end{tabular}
\end{table}

\begin{table}[H]
\begin{tabular}{cl}
\textbf{11.15} & \natline{अर्जुन उवाच} \\
 & \natline{पश्यामि देवांस्तव देव देहे} \\
 & \natline{सर्वांस्तथा भूतविशेषसङ्घान् |} \\
 & \natline{ब्रह्माणमीशं कमलासनस्थम्} \\
 & \natline{ऋषींश्च सर्वानुरगांश्च दिव्यान् ||}
\end{tabular}
\end{table}

\begin{table}[H]
\begin{tabular}{cl}
\textbf{11.16} & \natline{अनेकबाहूदरवक्त्रनेत्रं} \\
 & \natline{पश्यामि त्वा सर्वतोऽनन्तरूपम् |} \\
 & \natline{नान्तं न मध्यं न पुनस्तवादिं} \\
 & \natline{पश्यामि विश्वेश्वर विश्वरूप ||}
\end{tabular}
\end{table}

\begin{table}[H]
\begin{tabular}{cl}
\textbf{11.17} & \natline{किरीटिनं गदिनं चक्रिणं च} \\
 & \natline{तेजोराशिं सर्वतो दीप्तिमन्तम् |} \\
 & \natline{पश्यामि त्वां दुर्निरीक्ष्यं समन्तात्} \\
 & \natline{दीप्तानलार्कद्युतिमप्रमेयम् ||}
\end{tabular}
\end{table}

\begin{table}[H]
\begin{tabular}{cl}
\textbf{11.18} & \natline{त्वमक्षरं परमं वेदितव्यं} \\
 & \natline{त्वमस्य विश्वस्य परं निधानम् |} \\
 & \natline{त्वमव्ययः शाश्वतधर्मगोप्ता} \\
 & \natline{सनातनस्त्वं पुरुषो मतो मे ||}
\end{tabular}
\end{table}

\begin{table}[H]
\begin{tabular}{cl}
\textbf{11.19} & \natline{अनादिमध्यान्तमनन्तवीर्यम्} \\
 & \natline{अनन्तबाहुं शशिसूर्यनेत्रम् |} \\
 & \natline{पश्यामि त्वां दीप्तहुताशवक्त्रं} \\
 & \natline{स्वतेजसा विश्वमिदं तपन्तम् ||}
\end{tabular}
\end{table}

\begin{table}[H]
\begin{tabular}{cl}
\textbf{11.20} & \natline{द्यावापृथिव्योरिदमन्तरं हि} \\
 & \natline{व्याप्तं त्वयैकेन दिशश्च सर्वाः |} \\
 & \natline{दृष्ट्वाद्भुतं रूपमिदं तवोग्रं} \\
 & \natline{लोकत्रयं प्रव्यथितं महात्मन् ||}
\end{tabular}
\end{table}

