\begin{table}[H]
\begin{tabular}{cl}
 & \natline{श्री परमात्मने नमः} \\
 & \natline{अथ द्वादशोऽध्यायः} \\
 & \natline{भक्तियोगः}
\end{tabular}
\end{table}

\begin{table}[H]
\begin{tabular}{cl}
\textbf{12.1} & \natline{अर्जुन उवाच} \\
 & \natline{एवं सततयुक्ता ये} \\
 & \natline{भक्तास्त्वां पर्युपासते |} \\
 & \natline{ये चाप्यक्षरमव्यक्तं} \\
 & \natline{तेषां के योगवित्तमाः ||}
\end{tabular}
\end{table}

\begin{table}[H]
\begin{tabular}{cl}
\textbf{12.2} & \natline{श्री भगवान् उवाच} \\
 & \natline{मय्यावेश्य मनो ये मां} \\
 & \natline{नित्ययुक्ता उपासते |} \\
 & \natline{श्रद्धया परयोपेताः} \\
 & \natline{ते मे युक्ततमा मताः ||}
\end{tabular}
\end{table}

\begin{table}[H]
\begin{tabular}{cl}
\textbf{12.3} & \natline{ये त्वक्षरमनिर्देश्यम्} \\
 & \natline{अव्यक्तं पर्युपासते |} \\
 & \natline{सर्वत्रगमचिन्त्यम् च} \\
 & \natline{कूटस्थमचलम् ध्रुवं ||}
\end{tabular}
\end{table}

\begin{table}[H]
\begin{tabular}{cl}
\textbf{12.4} & \natline{सन्नियम्येन्द्रियग्रामं} \\
 & \natline{सर्वत्र समबुद्धयः |} \\
 & \natline{ते प्राप्नुवन्ति मामेव} \\
 & \natline{सर्वभूतहिते रताः ||}
\end{tabular}
\end{table}

\begin{table}[H]
\begin{tabular}{cl}
\textbf{12.5} & \natline{क्लेशोऽधिकतरस्तेषाम्} \\
 & \natline{अव्यक्तासक्तचेतसाम् |} \\
 & \natline{अव्यक्ता हि गतिर्दुःखं} \\
 & \natline{देहवद्भिरवाप्यते ||}
\end{tabular}
\end{table}

\begin{table}[H]
\begin{tabular}{cl}
\textbf{12.6} & \natline{ये तु सर्वाणि कर्माणि} \\
 & \natline{मयि सन्न्यस्य मत्पराः |} \\
 & \natline{अनन्येनैव योगेन} \\
 & \natline{मां ध्यायन्त उपासते ||}
\end{tabular}
\end{table}

\begin{table}[H]
\begin{tabular}{cl}
\textbf{12.7} & \natline{तेषामहं समुद्धर्ता} \\
 & \natline{मृत्युसंसारसागरात् |} \\
 & \natline{भवामि नचिरात्पार्थ} \\
 & \natline{मय्यावेशितचेतसाम् ||}
\end{tabular}
\end{table}

\begin{table}[H]
\begin{tabular}{cl}
\textbf{12.8} & \natline{मय्येव मन आधत्स्व} \\
 & \natline{मयि बुद्धिं निवेशय |} \\
 & \natline{निवसिष्यसि मय्येव} \\
 & \natline{अत ऊर्ध्वं न सम्शयः ||}
\end{tabular}
\end{table}

\begin{table}[H]
\begin{tabular}{cl}
\textbf{12.9} & \natline{अथ चित्तं समाधातुं} \\
 & \natline{न शक्नोषि मयि स्थिरम् |} \\
 & \natline{अभ्यासयोगेन ततः} \\
 & \natline{मामिच्छाप्तुं धनञ्जय ||}
\end{tabular}
\end{table}

\begin{table}[H]
\begin{tabular}{cl}
\textbf{12.10} & \natline{अभ्यासेऽप्यसमर्थोऽसि} \\
 & \natline{मत्कर्मपरमो भव |} \\
 & \natline{मदर्थमपि कर्माणि} \\
 & \natline{कुर्वन्सिद्धिमवाप्स्यसि ||}
\end{tabular}
\end{table}

\begin{table}[H]
\begin{tabular}{cl}
\textbf{12.11} & \natline{अथैतदप्यशक्तोऽसि} \\
 & \natline{कर्तुं मद्योगमाश्रितः |} \\
 & \natline{सर्वकर्मफलत्यागं} \\
 & \natline{ततः कुरु यतात्मवान् ||}
\end{tabular}
\end{table}

\begin{table}[H]
\begin{tabular}{cl}
\textbf{12.12} & \natline{श्रेयो हि ज्ञानमभ्यासात्} \\
 & \natline{ज्ञानाद्ध्यानं विशिष्यते |} \\
 & \natline{ध्यानात्कर्मफलत्यागः} \\
 & \natline{त्यागाच्छान्तिरनन्तरम् ||}
\end{tabular}
\end{table}

\begin{table}[H]
\begin{tabular}{cl}
\textbf{12.13} & \natline{अद्वेष्टा सर्वभूतानां} \\
 & \natline{मैत्रः करुण एव च |} \\
 & \natline{निर्ममो निरहन्कारः} \\
 & \natline{समदुःखसुखः क्षमी ||}
\end{tabular}
\end{table}

\begin{table}[H]
\begin{tabular}{cl}
\textbf{12.14} & \natline{सन्तुष्टः सततं योगी} \\
 & \natline{यतात्मा दृढनिश्चयः |} \\
 & \natline{मय्यर्पितमनोबुद्धिः} \\
 & \natline{यो मद्भक्तः स मे प्रियः ||}
\end{tabular}
\end{table}

\begin{table}[H]
\begin{tabular}{cl}
\textbf{12.15} & \natline{यस्मान्नोद्विजते लोकः} \\
 & \natline{लोकान्नोद्विजते च यः |} \\
 & \natline{हर्षामर्षभयोद्वेगैः} \\
 & \natline{मुक्तो यः स च मे प्रियः ||}
\end{tabular}
\end{table}

\begin{table}[H]
\begin{tabular}{cl}
\textbf{12.16} & \natline{अनपेक्षः शुचिर्दक्षः} \\
 & \natline{उदासीनो गतव्यथः |} \\
 & \natline{सर्वारम्भपरित्यागी} \\
 & \natline{यो मद्भक्तः स मे प्रियः ||}
\end{tabular}
\end{table}

\begin{table}[H]
\begin{tabular}{cl}
\textbf{12.17} & \natline{यो न हृष्यति न द्वेष्टि} \\
 & \natline{न शोचति न काञ्क्षति |} \\
 & \natline{शुभाशुभपरित्यागी} \\
 & \natline{भक्तिमान्यः स मे प्रियः ||}
\end{tabular}
\end{table}

\begin{table}[H]
\begin{tabular}{cl}
\textbf{12.18} & \natline{समः शत्रौ च मित्रे च} \\
 & \natline{तथा मानापमानयोः |} \\
 & \natline{शीतोष्णसुखदुःखेषु} \\
 & \natline{समः सञ्गविवर्जितः ||}
\end{tabular}
\end{table}

\begin{table}[H]
\begin{tabular}{cl}
\textbf{12.19} & \natline{तुल्यनिन्दास्तुतिर्मौनी} \\
 & \natline{सन्तुष्टो येन केनचित् |} \\
 & \natline{अनिकेतः स्थिरमतिः} \\
 & \natline{भक्तिमान्मे प्रियो नरः ||}
\end{tabular}
\end{table}

\begin{table}[H]
\begin{tabular}{cl}
\textbf{12.20} & \natline{ये तु धर्म्यामृतमिदं} \\
 & \natline{यथोक्तं पर्युपासते |} \\
 & \natline{श्रद्दधाना मत्परमाः} \\
 & \natline{भक्तास्तेऽतीव मे प्रियाः ||}
\end{tabular}
\end{table}

\begin{table}[H]
\begin{tabular}{cl}
 & \natline{श्रीमद्भगवद्गीतासु उपनिषत्सु} \\
 & \natline{ब्रह्मविद्यायां योगशास्त्रे} \\
 & \natline{श्रीकृष्णार्जुन संवादे} \\
 & \natline{भक्तियोगो नाम} \\
 & \natline{द्वादशोध्यायः}
\end{tabular}
\end{table}

