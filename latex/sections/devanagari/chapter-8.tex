\begin{table}[H]
\begin{tabular}{cl}
\textbf{8.0} & \natline{ओं श्री परमात्मने नमः} \\
 & \natline{अथ अष्टमोऽध्यायः} \\
 & \natline{अक्षरपरब्रह्मयोगः}
\end{tabular}
\end{table}

\begin{table}[H]
\begin{tabular}{cl}
\textbf{8.1} & \natline{अर्जुन उवाच} \\
 & \natline{किं तद्ब्रह्म किमध्यात्मं} \\
 & \natline{किं कर्म पुरुषोत्तम |} \\
 & \natline{अधिभूतं च किं प्रोक्तम्} \\
 & \natline{अधिदैवं किमुच्यते ||}
\end{tabular}
\end{table}

\begin{table}[H]
\begin{tabular}{cl}
\textbf{8.2} & \natline{अधियज्ञः कथं कोऽत्र} \\
 & \natline{देहेऽस्मिन्मधुसूदन |} \\
 & \natline{प्रयाणकाले च कथं} \\
 & \natline{ज्ञेयोऽसि नियतात्मभिः ||}
\end{tabular}
\end{table}

\begin{table}[H]
\begin{tabular}{cl}
\textbf{8.3} & \natline{श्री भगवानुवाच} \\
 & \natline{अक्षरम् ब्रह्म परमं} \\
 & \natline{स्वभावोऽध्यात्ममुच्यते |} \\
 & \natline{भूतभावोद्भवकरः} \\
 & \natline{विसर्गः कर्मसञ्ज्ञितः ||}
\end{tabular}
\end{table}

\begin{table}[H]
\begin{tabular}{cl}
\textbf{8.4} & \natline{अधिभूतं क्षरो भावः} \\
 & \natline{पुरुषश्चाधिदैवतम् |} \\
 & \natline{अधियज्ञोऽहमेवात्र} \\
 & \natline{देहे देहभृतां वर ||}
\end{tabular}
\end{table}

\begin{table}[H]
\begin{tabular}{cl}
\textbf{8.5} & \natline{अन्तकाले च मामेव} \\
 & \natline{स्मरन्मुक्त्वा कलेवरम् |} \\
 & \natline{यः प्रयाति स मद्भावं} \\
 & \natline{याति नास्त्यत्र संशयः ||}
\end{tabular}
\end{table}

\begin{table}[H]
\begin{tabular}{cl}
\textbf{8.6} & \natline{यं यं वापि स्मरन्भावं} \\
 & \natline{त्यजत्यन्ते कलेवरम् |} \\
 & \natline{तं तमेवैति कौन्तेय} \\
 & \natline{सदा तद्भावभावितः ||}
\end{tabular}
\end{table}

\begin{table}[H]
\begin{tabular}{cl}
\textbf{8.7} & \natline{तस्मात्सर्वेषु कालेषु} \\
 & \natline{मामनुस्मर युध्य च |} \\
 & \natline{मय्यर्पितमनोबुद्धिः} \\
 & \natline{मामेवैष्यस्यसंशयम् ||}
\end{tabular}
\end{table}

\begin{table}[H]
\begin{tabular}{cl}
\textbf{8.8} & \natline{अभ्यासयोगयुक्तेन} \\
 & \natline{चेतसा नान्यगामिना |} \\
 & \natline{परमं पुरुषं दिव्यं} \\
 & \natline{याति पार्थानुचिन्तयन् ||}
\end{tabular}
\end{table}

\begin{table}[H]
\begin{tabular}{cl}
\textbf{8.9} & \natline{कविं पुराणमनुशासितारम्} \\
 & \natline{अणोरणीयांसमनुस्मरेद्यः |} \\
 & \natline{सर्वस्य धातारमचिन्त्यरूपम्} \\
 & \natline{आदित्यवर्णं तमसः परस्तात् ||}
\end{tabular}
\end{table}

\begin{table}[H]
\begin{tabular}{cl}
\textbf{8.10} & \natline{प्रयाणकाले मनसाऽचलेन} \\
 & \natline{भक्त्या युक्तो योगबलेन चैव |} \\
 & \natline{भ्रुवोर्मध्ये प्राणमावेश्य सम्यक्} \\
 & \natline{स तं परं पुरुषमुपैति दिव्यं ||}
\end{tabular}
\end{table}

\begin{table}[H]
\begin{tabular}{cl}
\textbf{8.11} & \natline{यदक्षरं वेदविदो वदन्ति} \\
 & \natline{विशन्ति यद्यतयो वीतरागाः |} \\
 & \natline{यदिच्छन्तो ब्रह्मचर्यं चरन्ति} \\
 & \natline{तत्ते पदं सङ्ग्रहेण प्रवक्ष्ये ||}
\end{tabular}
\end{table}

\begin{table}[H]
\begin{tabular}{cl}
\textbf{8.12} & \natline{सर्वद्वाराणि संयम्य} \\
 & \natline{मनो हृदि निरुध्य च |} \\
 & \natline{मूर्ध्न्याधायात्मनः प्राणम्} \\
 & \natline{आस्थितो योगधारणां ||}
\end{tabular}
\end{table}

\begin{table}[H]
\begin{tabular}{cl}
\textbf{8.13} & \natline{ओमित्येकाक्षरं ब्रह्म} \\
 & \natline{व्याहरन्मामनुस्मरन् |} \\
 & \natline{यः प्रयाति त्यजन्देहं} \\
 & \natline{स याति परमां गतिम् ||}
\end{tabular}
\end{table}

\begin{table}[H]
\begin{tabular}{cl}
\textbf{8.14} & \natline{अनन्यचेताः सततं} \\
 & \natline{यो मां स्मरति नित्यशः |} \\
 & \natline{तस्याहं सुलभः पार्थ} \\
 & \natline{नित्ययुक्तस्य योगिनः ||}
\end{tabular}
\end{table}

\begin{table}[H]
\begin{tabular}{cl}
\textbf{8.15} & \natline{मामुपेत्य पुनर्जन्म} \\
 & \natline{दुःखालयमशाश्वतम् |} \\
 & \natline{नाप्नुवन्ति महात्मानः} \\
 & \natline{संसिद्धिं परमां गताः ||}
\end{tabular}
\end{table}

\begin{table}[H]
\begin{tabular}{cl}
\textbf{8.16} & \natline{आब्रह्मभुवनाल्लोकाः} \\
 & \natline{पुनरावर्तिनोऽर्जुन |} \\
 & \natline{मामुपेत्य तु कौन्तेय} \\
 & \natline{पुनर्जन्म न विद्यते ||}
\end{tabular}
\end{table}

\begin{table}[H]
\begin{tabular}{cl}
\textbf{8.17} & \natline{सहस्रयुगपर्यन्तम्} \\
 & \natline{अहर्यद्ब्रह्मणो विदुः |} \\
 & \natline{रात्रिं युगसहस्रान्तां} \\
 & \natline{तेऽहोरात्रविदो जनाः ||}
\end{tabular}
\end{table}

\begin{table}[H]
\begin{tabular}{cl}
\textbf{8.18} & \natline{अव्यक्ताद्व्यक्तयः सर्वाः} \\
 & \natline{प्रभवन्त्यहरागमे |} \\
 & \natline{रात्र्यागमे प्रलीयन्ते} \\
 & \natline{तत्रैवाव्यक्तसञ्ज्ञके ||}
\end{tabular}
\end{table}

\begin{table}[H]
\begin{tabular}{cl}
\textbf{8.19} & \natline{भूतग्रामः स एवायं} \\
 & \natline{भूत्वा भूत्वा प्रलीयते |} \\
 & \natline{रात्र्यागमेऽवशः पार्थ} \\
 & \natline{प्रभवत्यहरागमे ||}
\end{tabular}
\end{table}

\begin{table}[H]
\begin{tabular}{cl}
\textbf{8.20} & \natline{परस्तस्मात्तु भावोऽन्यः} \\
 & \natline{अव्यक्तोऽव्यक्तात्सनातनः |} \\
 & \natline{यः स सर्वेषु भूतेषु} \\
 & \natline{नश्यत्सु न विनश्यति ||}
\end{tabular}
\end{table}

\begin{table}[H]
\begin{tabular}{cl}
\textbf{8.21} & \natline{अव्यक्तोऽक्षर इत्युक्तः} \\
 & \natline{तमाहुः परमां गतिम् |} \\
 & \natline{यं प्राप्य न निवर्तन्ते} \\
 & \natline{तद्धाम परमं मम ||}
\end{tabular}
\end{table}

\begin{table}[H]
\begin{tabular}{cl}
\textbf{8.22} & \natline{पुरुषः स परः पार्थ} \\
 & \natline{भक्त्या लभ्यस्त्वनन्यया |} \\
 & \natline{यस्यान्तः स्थानि भूतानि} \\
 & \natline{येन सर्वमिदं ततम् ||}
\end{tabular}
\end{table}

\begin{table}[H]
\begin{tabular}{cl}
\textbf{8.23} & \natline{यत्र काले त्वनावृतिम्} \\
 & \natline{आवृतिं चैव योगिनः |} \\
 & \natline{प्रयाता यान्ति तं कालं} \\
 & \natline{वक्ष्यामि भरतर्षभ ||}
\end{tabular}
\end{table}

\begin{table}[H]
\begin{tabular}{cl}
\textbf{8.24} & \natline{अग्निर्ज्योतिरहः शुक्लः} \\
 & \natline{षण्मासा उत्तरायणम् |} \\
 & \natline{तत्र प्रयाता गच्छन्ति} \\
 & \natline{ब्रह्म ब्रह्मविदो जनाः ||}
\end{tabular}
\end{table}

\begin{table}[H]
\begin{tabular}{cl}
\textbf{8.25} & \natline{धूमो रात्रिस्तथा कृष्णः} \\
 & \natline{षण्मासा दक्षिणायनम् |} \\
 & \natline{तत्र चान्द्रमसं ज्योतिः} \\
 & \natline{योगी प्राप्य निवर्तते ||}
\end{tabular}
\end{table}

\begin{table}[H]
\begin{tabular}{cl}
\textbf{8.26} & \natline{शुक्लकृष्णे गती ह्येते} \\
 & \natline{जगतः शाश्वते मते |} \\
 & \natline{एकया यात्यनावृतिम्} \\
 & \natline{अन्ययाऽऽवर्तते पुनः ||}
\end{tabular}
\end{table}

\begin{table}[H]
\begin{tabular}{cl}
\textbf{8.27} & \natline{नैते सृती पार्थ जानन्} \\
 & \natline{योगी मुह्यति कश्चन |} \\
 & \natline{तस्मात्सर्वेषु कालेषु} \\
 & \natline{योगयुक्तो भवार्जुन ||}
\end{tabular}
\end{table}

\begin{table}[H]
\begin{tabular}{cl}
\textbf{8.28} & \natline{वेदेषु यज्ञेषु तपस्सु चैव} \\
 & \natline{दानेषु यत् पुण्यफलं प्रदिष्टम् |} \\
 & \natline{अत्येति तत्सर्वमिदं विदित्वा} \\
 & \natline{योगी परं स्थानमुपैति चाद्यम् ||}
\end{tabular}
\end{table}

