\subsection*{2.0}
\begin{table}[H]
\centering
\begin{tabular}{ll}
\natline{ओं श्री परमात्मने नमः} & \romline{oṃ śrī paramātmane namaḥ} \\
\natline{अथ द्वितीयोऽध्यायः} & \romline{atha dvitīyo'dhyāyaḥ} \\
\natline{साण्ख्ययोगः} & \romline{sāṅkhya-yogaḥ}
\end{tabular}
\end{table}

\subsection*{2.1}
\begin{table}[H]
\centering
\begin{tabular}{ll}
\natline{संजय उवाच} & \romline{saṃjaya uvāca} \\
\natline{तं तथा कृपयाविष्टम्} & \romline{taṃ tathā kṛpayāviṣṭam} \\
\natline{अश्रुपूर्णाकुलेक्षणम्} & \romline{aśru-pūrṇākulekṣaṇam} \\
\natline{विषीदंतमिदं वाक्यम्} & \romline{viṣīdaṃtamidaṃ vākyam} \\
\natline{उवाच मधुसूदनः} & \romline{uvāca madhusūdanaḥ}
\end{tabular}
\end{table}

\subsection*{2.2}
\begin{table}[H]
\centering
\begin{tabular}{ll}
\natline{श्री भगवानुवाच} & \romline{śrī bhagavān-uvāca} \\
\natline{कुतस्त्वा कश्मलमिदं} & \romline{kutastvā kaśmalamidaṃ} \\
\natline{विषमे समुपस्थितम्} & \romline{viṣame samupasthitam} \\
\natline{अनार्यजुष्टमस्वर्ग्यम्} & \romline{anārya-juṣṭamasvargyam} \\
\natline{अकीर्तिकरमर्जुन} & \romline{akīrti-karam-arjuna}
\end{tabular}
\end{table}

\subsection*{2.3}
\begin{table}[H]
\centering
\begin{tabular}{ll}
\natline{क्लैब्यं मा स्म गमः पार्थ} & \romline{klaibyaṃ mā sma gamaḥ pārtha} \\
\natline{नैतत्त्वय्युपपद्यते} & \romline{naitat-tvayyupapadyate} \\
\natline{क्षुद्रं हृदयदौर्बल्यं} & \romline{kṣudraṃ hṛdaya-daurbalyaṃ} \\
\natline{त्यक्त्वोत्तिष्ठ परंतप} & \romline{tyaktvottiṣṭha paraṃtapa}
\end{tabular}
\end{table}

\subsection*{2.4}
\begin{table}[H]
\centering
\begin{tabular}{ll}
\natline{अर्जुन उवाच} & \romline{arjuna uvāca} \\
\natline{कथं भीश्ममहं संख्ये} & \romline{kathaṃ bhīśmamahaṃ saṃkhye} \\
\natline{द्रोणं च मधुसूदन} & \romline{droṇaṃ ca madhusūdana} \\
\natline{इशुभिः प्रतियोत्स्यामि} & \romline{iśubhiḥ pratiyotsyāmi} \\
\natline{पूजार्हावरिसूदन} & \romline{pūjārhāvarisūdana}
\end{tabular}
\end{table}

\subsection*{2.5}
\begin{table}[H]
\centering
\begin{tabular}{ll}
\natline{गुरूनहत्वा हि महानुभावान्} & \romline{gurūnahatvā hi mahānubhāvān} \\
\natline{श्रेयो भोक्तुं भैक्ष्यमपीह लोके} & \romline{śreyo bhoktuṃ bhaikṣyamapīha loke} \\
\natline{हत्वार्थकामंस्तु गुरूनिहैव} & \romline{hatvārtha-kāmaṃstu gurūnihaiva} \\
\natline{भुंजीय भोगान् रुधिरप्रदिग्धान्} & \romline{bhuṃjīya bhogān rudhira-pradigdhān}
\end{tabular}
\end{table}

\subsection*{2.6}
\begin{table}[H]
\centering
\begin{tabular}{ll}
\natline{न चैतद्विद्मः कतरन्नो गरीयः} & \romline{na caitadvidmaḥ kataranno garīyaḥ} \\
\natline{यद्वा जयेम यदि वा नो जयेयुः} & \romline{yadvā jayema yadi vā no jayeyuḥ} \\
\natline{यानेव हत्वा न जिजीविषामः} & \romline{yāneva hatvā na jijīviṣāmaḥ} \\
\natline{तेऽवस्थिताः प्रमुखे धार्तराष्ट्राः} & \romline{te'vasthitāḥ pramukhe dhārtarāṣṭrāḥ}
\end{tabular}
\end{table}

\subsection*{2.7}
\begin{table}[H]
\centering
\begin{tabular}{ll}
\natline{कार्पन्यदोषोपहतस्वभावः} & \romline{kārpanya-doṣopahata-svabhāvaḥ} \\
\natline{पृच्छामि त्वां धर्मसम्मूढचेताः} & \romline{pṛcchāmi tvāṃ dharma-sammūḍha-cetāḥ} \\
\natline{यच्छ्रेयः स्यान्निश्चितं ब्रूहि तन्मे} & \romline{yacchreyaḥ syānniścitaṃ brūhi tanme} \\
\natline{शिष्यस्तेऽहं शाधि मां त्वां प्रपन्नम्} & \romline{śiṣyaste'haṃ śādhi māṃ tvāṃ prapannam}
\end{tabular}
\end{table}

\subsection*{2.8}
\begin{table}[H]
\centering
\begin{tabular}{ll}
\natline{न हिप्रपश्यामि ममापनुद्याद्} & \romline{na hi\textasciitilde{}prapaśyāmi mamāpanudyād} \\
\natline{यच्छोकमुच्छोषणमिन्द्रियाणाम्} & \romline{yacchokam-ucchoṣaṇam-indriyāṇām} \\
\natline{अवाप्य भूमावसपत्नमृद्धं} & \romline{avāpya bhūmāv-asapatnamṛddhaṃ} \\
\natline{राज्यं सुराणामपि चाधिपत्यम्} & \romline{rājyaṃ surāṇāmapi cādhipatyam}
\end{tabular}
\end{table}

\subsection*{2.9}
\begin{table}[H]
\centering
\begin{tabular}{ll}
\natline{संजय उवाच} & \romline{saṃjaya uvāca} \\
\natline{एवमुक्त्वा हृषीकेशं} & \romline{evam-uktvā hṛṣīkeśaṃ} \\
\natline{गुडाकेशः परन्तपः} & \romline{guḍākeśaḥ parantapaḥ} \\
\natline{न योत्स्य इति गोविंदम्} & \romline{na yotsya iti goviṃdam} \\
\natline{उक्त्वा तूष्णीम् बभूव ह} & \romline{uktvā tūṣṇīm babhūva ha}
\end{tabular}
\end{table}

\subsection*{2.10}
\begin{table}[H]
\centering
\begin{tabular}{ll}
\natline{तमुवाच हृषीकेशः} & \romline{tam-uvāca hṛṣīkeśaḥ} \\
\natline{प्रहसन्निव भारत} & \romline{prahasanniva bhārata} \\
\natline{सेनयोरुभयोर्मध्ये} & \romline{senayorubhayor-madhye} \\
\natline{विषीदंतमिदं वचः} & \romline{viṣīdaṃtam-idaṃ vacaḥ}
\end{tabular}
\end{table}

\subsection*{2.11}
\begin{table}[H]
\centering
\begin{tabular}{ll}
\natline{श्री भगवानुवाच} & \romline{śrī bhagavān-uvāca} \\
\natline{अशोच्यानन्वशोचस्त्वं} & \romline{aśocyān-anvaśocas-tvaṃ} \\
\natline{प्रज्ञावादांश्च भाषसे} & \romline{prajñā-vādāṃśca bhāṣase} \\
\natline{गतासूनगतासूंश्च} & \romline{gatāsūn-agatāsūṃś-ca} \\
\natline{नानुशोचन्ति पण्डिताः} & \romline{nānuśocanti paṇḍitāḥ}
\end{tabular}
\end{table}

\subsection*{2.12}
\begin{table}[H]
\centering
\begin{tabular}{ll}
\natline{न त्वेवाहं जातु नासं} & \romline{na tvevāhaṃ jātu nāsaṃ} \\
\natline{न त्वं नेमे जनाधिपाः} & \romline{na tvaṃ neme janādhipāḥ} \\
\natline{न चैव न भविष्यामः} & \romline{na caiva na bhaviṣyāmaḥ} \\
\natline{सर्वे वयमतः परम्} & \romline{sarve vayamataḥ param}
\end{tabular}
\end{table}

\subsection*{2.13}
\begin{table}[H]
\centering
\begin{tabular}{ll}
\natline{देहिनोऽस्मिन् यथा देहे} & \romline{dehino'smin yathā dehe} \\
\natline{कौमारं यौवनं जरा} & \romline{kaumāraṃ yauvanaṃ jarā} \\
\natline{तथा देहांतरप्राप्तिः} & \romline{tathā dehāṃtara-prāptiḥ} \\
\natline{धीरस्तत्र न मुह्यति} & \romline{dhīras-tatra na muhyati}
\end{tabular}
\end{table}

\subsection*{2.14}
\begin{table}[H]
\centering
\begin{tabular}{ll}
\natline{मात्रास्पर्शास्तु कौंतेय} & \romline{mātrā-sparśās-tu kauṃteya} \\
\natline{शीतोष्णसुखदुःखदाः} & \romline{śītoṣṇa-sukha-duḥkha-dāḥ} \\
\natline{आगमापायिनोऽनित्याः} & \romline{āgamāpāyino'nityāḥ} \\
\natline{तांस्तितिक्षस्व भारत} & \romline{tāṃs-titikṣasva bhārata}
\end{tabular}
\end{table}

\subsection*{2.15}
\begin{table}[H]
\centering
\begin{tabular}{ll}
\natline{यं हि न व्यथयंत्येते} & \romline{yaṃ hi na vyathayaṃtyete} \\
\natline{पुरुषं पुरुषर्षभ} & \romline{puruṣaṃ puruṣarṣabha} \\
\natline{समदुःखसुखं धीरं} & \romline{sama-duḥkha-sukhaṃ dhīraṃ} \\
\natline{सोऽमृतत्वाय कल्पते} & \romline{so'mṛtatvāya kalpate}
\end{tabular}
\end{table}

\subsection*{2.16}
\begin{table}[H]
\centering
\begin{tabular}{ll}
\natline{नासतो विद्यते भावः} & \romline{nāsato vidyate bhāvaḥ} \\
\natline{नाभावो विद्यते सतः} & \romline{nābhāvo vidyate sataḥ} \\
\natline{उभयोरपि दृष्तोऽन्तः} & \romline{ubhayorapi dṛṣto'ntaḥ} \\
\natline{त्वनयोस्तत्त्वदर्शिभिः} & \romline{tvanayos-tattva-darśibhiḥ}
\end{tabular}
\end{table}

\subsection*{2.17}
\begin{table}[H]
\centering
\begin{tabular}{ll}
\natline{अविनाशि तु तद्विद्धि} & \romline{avināśi tu tadviddhi} \\
\natline{येन सर्वमिदं ततम्} & \romline{yena sarvamidaṃ tatam} \\
\natline{विनाशमव्ययस्यास्य} & \romline{vināśam-avyayasyāsya} \\
\natline{न कश्चित्कर्तुमर्हति} & \romline{na kaścit-kartum-arhati}
\end{tabular}
\end{table}

\subsection*{2.18}
\begin{table}[H]
\centering
\begin{tabular}{ll}
\natline{अंतवन्त इमे देहाः} & \romline{aṃtavanta ime dehāḥ} \\
\natline{नित्यस्योक्ताः शरीरिणः} & \romline{nityasyoktāḥ śarīriṇaḥ} \\
\natline{अनाशिनोऽप्रमेयस्य} & \romline{anāśino'prameyasya} \\
\natline{तस्माद्युध्यस्व भारत} & \romline{tasmād-yudhyasva bhārata}
\end{tabular}
\end{table}

\subsection*{2.19}
\begin{table}[H]
\centering
\begin{tabular}{ll}
\natline{य एनं वेत्ति हन्तारं} & \romline{ya enaṃ vetti hantāraṃ} \\
\natline{यश्चैनं मन्यते हतं} & \romline{yaścainaṃ manyate hataṃ} \\
\natline{उभौ तौ न विजानीतः} & \romline{ubhau tau na vijānītaḥ} \\
\natline{नायं हन्ति न हन्यते} & \romline{nāyaṃ hanti na hanyate}
\end{tabular}
\end{table}

\subsection*{2.20}
\begin{table}[H]
\centering
\begin{tabular}{ll}
\natline{न जायते म्रियते वा कदाचित्} & \romline{na jāyate mriyate vā kadācit} \\
\natline{नायं भूत्वा भविता वा न भूयः} & \romline{nāyaṃ bhūtvā bhavitā vā na bhūyaḥ} \\
\natline{अजो नित्यः शाश्वतोऽयं पुराणः} & \romline{ajo nityaḥ śāśvato'yaṃ purāṇaḥ} \\
\natline{न हन्यते हन्यमाने शरीरे} & \romline{na hanyate hanyamāne śarīre}
\end{tabular}
\end{table}

\subsection*{2.21}
\begin{table}[H]
\centering
\begin{tabular}{ll}
\natline{वेदाविनाशिनं नित्यं} & \romline{vedāvināśinaṃ nityaṃ} \\
\natline{य एनमजमव्ययम्} & \romline{ya enamajam-avyayam} \\
\natline{कथं स पुरुषः पार्थ} & \romline{kathaṃ sa puruṣaḥ pārtha} \\
\natline{कं घातयति हंति कम्} & \romline{kaṃ ghātayati haṃti kam}
\end{tabular}
\end{table}

\subsection*{2.22}
\begin{table}[H]
\centering
\begin{tabular}{ll}
\natline{वासांसि जीर्णानि यथा विहाय} & \romline{vāsāṃsi jīrṇāni yathā vihāya} \\
\natline{नवानि गृह्णाति नरोऽपराणि} & \romline{navāni gṛhṇāti naro'parāṇi} \\
\natline{तथा शरीराणि विहाय जीर्णानि} & \romline{tathā śarīrāṇi vihāya jīrṇāni} \\
\natline{अन्यानि संयाति नवानि देही} & \romline{anyāni saṃyāti navāni dehī}
\end{tabular}
\end{table}

\subsection*{2.23}
\begin{table}[H]
\centering
\begin{tabular}{ll}
\natline{नैनं छिन्दन्ति शस्त्राणि} & \romline{nainaṃ chindanti śastrāṇi} \\
\natline{नैनं दहति पावकः} & \romline{nainaṃ dahati pāvakaḥ} \\
\natline{न चैनं क्लेदयन्त्यापः} & \romline{na cainaṃ kledayantyāpaḥ} \\
\natline{न शोषयति मारुतः} & \romline{na śoṣayati mārutaḥ}
\end{tabular}
\end{table}

\subsection*{2.24}
\begin{table}[H]
\centering
\begin{tabular}{ll}
\natline{अच्छेद्योऽयम् अदाह्योऽयम्} & \romline{acchedyo'yam adāhyo'yam} \\
\natline{अक्लेद्योऽशोष्य एव च} & \romline{akledyo'śoṣya eva ca} \\
\natline{नित्यः सर्वगतः स्थाणुः} & \romline{nityaḥ sarva-gataḥ sthāṇuḥ} \\
\natline{अचलोऽयं सनातनः} & \romline{acalo'yaṃ sanātanaḥ}
\end{tabular}
\end{table}

\subsection*{2.25}
\begin{table}[H]
\centering
\begin{tabular}{ll}
\natline{अव्यक्तोऽयम् अचिन्त्योऽयम्} & \romline{avyakto'yam acintyo'yam} \\
\natline{अविकार्योऽयमुच्यते} & \romline{avikāryo'yamucyate} \\
\natline{तस्मादेवं विदित्वैनं} & \romline{tasmādevaṃ viditvainaṃ} \\
\natline{नानुशोचितुमर्हसि} & \romline{nānuśocitumarhasi}
\end{tabular}
\end{table}

\subsection*{2.26}
\begin{table}[H]
\centering
\begin{tabular}{ll}
\natline{अथ चैनं नित्यजातं} & \romline{atha cainaṃ nityajātaṃ} \\
\natline{नित्यं वा मन्यसे मृतम्} & \romline{nityaṃ vā manyase mṛtam} \\
\natline{तथाऽपि त्वं महाबाहो} & \romline{tathā'pi tvaṃ mahābāho} \\
\natline{नैवं शोचितुमर्हसि} & \romline{naivaṃ śocitumarhasi}
\end{tabular}
\end{table}

\subsection*{2.27}
\begin{table}[H]
\centering
\begin{tabular}{ll}
\natline{जातस्य हिध्रुवो मृत्युः} & \romline{jātasya hi\textasciitilde{}dhruvo mṛtyuḥ} \\
\natline{ध्रुवं जन्म मृतस्य च} & \romline{dhruvaṃ janma mṛtasya ca} \\
\natline{तस्मादपरिहार्येऽर्थे} & \romline{tasmādaparihārye'rthe} \\
\natline{न त्वं शोचितुमर्हसि} & \romline{na tvaṃ śocitumarhasi}
\end{tabular}
\end{table}

\subsection*{2.28}
\begin{table}[H]
\centering
\begin{tabular}{ll}
\natline{अव्यक्तादीनि भूतानि} & \romline{avyaktādīni bhūtāni} \\
\natline{व्यक्तमध्यानि भारत} & \romline{vyaktamadhyāni bhārata} \\
\natline{अव्यक्तनिधनान्येव} & \romline{avyakta-nidhanānyeva} \\
\natline{तत्र का परिदेवना} & \romline{tatra kā paridevanā}
\end{tabular}
\end{table}

\subsection*{2.29}
\begin{table}[H]
\centering
\begin{tabular}{ll}
\natline{आश्चर्यवत् पश्यति कश्चिदेनम्} & \romline{āścaryavat paśyati kaścidenam} \\
\natline{आश्चर्यवद् वदति तथैव cआन्यः} & \romline{āścaryavad vadati tathaiva cānyaḥ} \\
\natline{आश्चर्यवच्चैनमन्यः शृणोति} & \romline{āścaryavaccainamanyaḥ śṛṇoti} \\
\natline{श्रुत्वाप्येनं वेद न चैव कश्चित्} & \romline{śrutvāpyenaṃ veda na caiva kaścit}
\end{tabular}
\end{table}

\subsection*{2.30}
\begin{table}[H]
\centering
\begin{tabular}{ll}
\natline{देही नित्यमवध्योऽयं} & \romline{dehī nityamavadhyo'yaṃ} \\
\natline{देहे सर्वस्य भारत} & \romline{dehe sarvasya bhārata} \\
\natline{तस्मात्सर्वाणि भूतानि} & \romline{tasmātsarvāṇi bhūtāni} \\
\natline{न त्वं शोचितुमर्हसि} & \romline{na tvaṃ śocitumarhasi}
\end{tabular}
\end{table}

