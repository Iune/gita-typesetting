\begin{table}[H]
\begin{tabular}{cl}
 & \natline{श्री परमात्मने नमः} \\
 & \natline{अथ द्वितीयोऽध्यायः} \\
 & \natline{साङ्ख्ययोगः}
\end{tabular}
\end{table}

\begin{table}[H]
\begin{tabular}{cl}
\textbf{2.1} & \natline{संजय उवाच} \\
 & \natline{तं तथा कृपयाविष्टम्} \\
 & \natline{अश्रुपूर्णाकुलेक्षणम् |} \\
 & \natline{विषीदंतमिदं वाक्यम्} \\
 & \natline{उवाच मधुसूदनः ||}
\end{tabular}
\end{table}

\begin{table}[H]
\begin{tabular}{cl}
\textbf{2.2} & \natline{श्री भगवानुवाच} \\
 & \natline{कुतस्त्वा कश्मलमिदं} \\
 & \natline{विषमे समुपस्थितम् |} \\
 & \natline{अनार्यजुष्टमस्वर्ग्यम्} \\
 & \natline{अकीर्तिकरमर्जुन ||}
\end{tabular}
\end{table}

\begin{table}[H]
\begin{tabular}{cl}
\textbf{2.3} & \natline{क्लैब्यं मा स्म गमः पार्थ} \\
 & \natline{नैतत्त्वय्युपपद्यते |} \\
 & \natline{क्षुद्रं हृदयदौर्बल्यं} \\
 & \natline{त्यक्त्वोत्तिष्ठ परंतप ||}
\end{tabular}
\end{table}

\begin{table}[H]
\begin{tabular}{cl}
\textbf{2.4} & \natline{अर्जुन उवाच} \\
 & \natline{कथं भीश्ममहं संख्ये} \\
 & \natline{द्रोणं च मधुसूदन |} \\
 & \natline{इशुभिः प्रतियोत्स्यामि} \\
 & \natline{पूजार्हावरिसूदन ||}
\end{tabular}
\end{table}

\begin{table}[H]
\begin{tabular}{cl}
\textbf{2.5} & \natline{गुरूनहत्वा हि महानुभावान्} \\
 & \natline{श्रेयो भोक्तुं भैक्ष्यमपीह लोके |} \\
 & \natline{हत्वार्थकामंस्तु गुरूनिहैव} \\
 & \natline{भुंजीय भोगान् रुधिरप्रदिग्धान् ||}
\end{tabular}
\end{table}

\begin{table}[H]
\begin{tabular}{cl}
\textbf{2.6} & \natline{न चैतद्विद्मः कतरन्नो गरीयः} \\
 & \natline{यद्वा जयेम यदि वा नो जयेयुः |} \\
 & \natline{यानेव हत्वा न जिजीविषामः} \\
 & \natline{तेऽवस्थिताः प्रमुखे धार्तराष्ट्राः ||}
\end{tabular}
\end{table}

\begin{table}[H]
\begin{tabular}{cl}
\textbf{2.7} & \natline{कार्पन्यदोषोपहतस्वभावः} \\
 & \natline{पृच्छामि त्वां धर्मसम्मूढचेताः |} \\
 & \natline{यच्छ्रेयः स्यान्निश्चितं ब्रूहि तन्मे} \\
 & \natline{शिष्यस्तेऽहं शाधि मां त्वां प्रपन्नम् ||}
\end{tabular}
\end{table}

\begin{table}[H]
\begin{tabular}{cl}
\textbf{2.8} & \natline{न हि प्रपश्यामि ममापनुद्याद्} \\
 & \natline{यच्छोकमुच्छोषणमिन्द्रियाणाम् |} \\
 & \natline{अवाप्य भूमावसपत्नमृद्धं} \\
 & \natline{राज्यं सुराणामपि चाधिपत्यम् ||}
\end{tabular}
\end{table}

\begin{table}[H]
\begin{tabular}{cl}
\textbf{2.9} & \natline{संजय उवाच} \\
 & \natline{एवमुक्त्वा हृषीकेशं} \\
 & \natline{गुडाकेशः परन्तपः |} \\
 & \natline{न योत्स्य इति गोविंदम्} \\
 & \natline{उक्त्वा तूष्णीम् बभूव ह ||}
\end{tabular}
\end{table}

\begin{table}[H]
\begin{tabular}{cl}
\textbf{2.10} & \natline{तमुवाच हृषीकेशः} \\
 & \natline{प्रहसन्निव भारत |} \\
 & \natline{सेनयोरुभयोर्मध्ये} \\
 & \natline{विषीदंतमिदं वचः ||}
\end{tabular}
\end{table}

\begin{table}[H]
\begin{tabular}{cl}
\textbf{2.11} & \natline{श्री भगवानुवाच} \\
 & \natline{अशोच्यानन्वशोचस्त्वं} \\
 & \natline{प्रज्ञावादांश्च भाषसे |} \\
 & \natline{गतासूनगतासूंश्च} \\
 & \natline{नानुशोचन्ति पण्डिताः ||}
\end{tabular}
\end{table}

\begin{table}[H]
\begin{tabular}{cl}
\textbf{2.12} & \natline{न त्वेवाहं जातु नासं} \\
 & \natline{न त्वं नेमे जनाधिपाः |} \\
 & \natline{न चैव न भविष्यामः} \\
 & \natline{सर्वे वयमतः परम् ||}
\end{tabular}
\end{table}

\begin{table}[H]
\begin{tabular}{cl}
\textbf{2.13} & \natline{देहिनोऽस्मिन् यथा देहे} \\
 & \natline{कौमारं यौवनं जरा |} \\
 & \natline{तथा देहांतरप्राप्तिः} \\
 & \natline{धीरस्तत्र न मुह्यति ||}
\end{tabular}
\end{table}

\begin{table}[H]
\begin{tabular}{cl}
\textbf{2.14} & \natline{मात्रास्पर्शास्तु कौंतेय} \\
 & \natline{शीतोष्णसुखदुःखदाः |} \\
 & \natline{आगमापायिनोऽनित्याः} \\
 & \natline{तांस्तितिक्षस्व भारत ||}
\end{tabular}
\end{table}

\begin{table}[H]
\begin{tabular}{cl}
\textbf{2.15} & \natline{यं हि न व्यथयंत्येते} \\
 & \natline{पुरुषं पुरुषर्षभ |} \\
 & \natline{समदुःखसुखं धीरं} \\
 & \natline{सोऽमृतत्वाय कल्पते ||}
\end{tabular}
\end{table}

\begin{table}[H]
\begin{tabular}{cl}
\textbf{2.16} & \natline{नासतो विद्यते भावः} \\
 & \natline{नाभावो विद्यते सतः |} \\
 & \natline{उभयोरपि दृष्तोऽन्तः} \\
 & \natline{त्वनयोस्तत्त्वदर्शिभिः ||}
\end{tabular}
\end{table}

\begin{table}[H]
\begin{tabular}{cl}
\textbf{2.17} & \natline{अविनाशि तु तद्विद्धि} \\
 & \natline{येन सर्वमिदं ततम् |} \\
 & \natline{विनाशमव्ययस्यास्य} \\
 & \natline{न कश्चित्कर्तुमर्हति ||}
\end{tabular}
\end{table}

\begin{table}[H]
\begin{tabular}{cl}
\textbf{2.18} & \natline{अंतवन्त इमे देहाः} \\
 & \natline{नित्यस्योक्ताः शरीरिणः |} \\
 & \natline{अनाशिनोऽप्रमेयस्य} \\
 & \natline{तस्माद्युध्यस्व भारत ||}
\end{tabular}
\end{table}

\begin{table}[H]
\begin{tabular}{cl}
\textbf{2.19} & \natline{य एनं वेत्ति हन्तारं} \\
 & \natline{यश्चैनं मन्यते हतं |} \\
 & \natline{उभौ तौ न विजानीतः} \\
 & \natline{नायं हन्ति न हन्यते ||}
\end{tabular}
\end{table}

\begin{table}[H]
\begin{tabular}{cl}
\textbf{2.20} & \natline{न जायते म्रियते वा कदाचित्} \\
 & \natline{नायं भूत्वा भविता वा न भूयः |} \\
 & \natline{अजो नित्यः शाश्वतोऽयं पुराणः} \\
 & \natline{न हन्यते हन्यमाने शरीरे ||}
\end{tabular}
\end{table}

\begin{table}[H]
\begin{tabular}{cl}
\textbf{2.21} & \natline{वेदाविनाशिनं नित्यं} \\
 & \natline{य एनमजमव्ययम् |} \\
 & \natline{कथं स पुरुषः पार्थ} \\
 & \natline{कं घातयति हंति कम् ||}
\end{tabular}
\end{table}

\begin{table}[H]
\begin{tabular}{cl}
\textbf{2.22} & \natline{वासांसि जीर्णानि यथा विहाय} \\
 & \natline{नवानि गृह्णाति नरोऽपराणि |} \\
 & \natline{तथा शरीराणि विहाय जीर्णानि} \\
 & \natline{अन्यानि संयाति नवानि देही ||}
\end{tabular}
\end{table}

\begin{table}[H]
\begin{tabular}{cl}
\textbf{2.23} & \natline{नैनं छिन्दन्ति शस्त्राणि} \\
 & \natline{नैनं दहति पावकः |} \\
 & \natline{न चैनं क्लेदयन्त्यापः} \\
 & \natline{न शोषयति मारुतः ||}
\end{tabular}
\end{table}

\begin{table}[H]
\begin{tabular}{cl}
\textbf{2.24} & \natline{अच्छेद्योऽयम् अदाह्योऽयम्} \\
 & \natline{अक्लेद्योऽशोष्य एव च |} \\
 & \natline{नित्यः सर्वगतः स्थाणुः} \\
 & \natline{अचलोऽयं सनातनः ||}
\end{tabular}
\end{table}

\begin{table}[H]
\begin{tabular}{cl}
\textbf{2.25} & \natline{अव्यक्तोऽयम् अचिन्त्योऽयम्} \\
 & \natline{अविकार्योऽयमुच्यते |} \\
 & \natline{तस्मादेवं विदित्वैनं} \\
 & \natline{नानुशोचितुमर्हसि ||}
\end{tabular}
\end{table}

\begin{table}[H]
\begin{tabular}{cl}
\textbf{2.26} & \natline{अथ चैनं नित्यजातं} \\
 & \natline{नित्यं वा मन्यसे मृतम् |} \\
 & \natline{तथाऽपि त्वं महाबाहो} \\
 & \natline{नैवं शोचितुमर्हसि ||}
\end{tabular}
\end{table}

\begin{table}[H]
\begin{tabular}{cl}
\textbf{2.27} & \natline{जातस्य हि ध्रुवो मृत्युः} \\
 & \natline{ध्रुवं जन्म मृतस्य च |} \\
 & \natline{तस्मादपरिहार्येऽर्थे} \\
 & \natline{न त्वं शोचितुमर्हसि ||}
\end{tabular}
\end{table}

\begin{table}[H]
\begin{tabular}{cl}
\textbf{2.28} & \natline{अव्यक्तादीनि भूतानि} \\
 & \natline{व्यक्तमध्यानि भारत |} \\
 & \natline{अव्यक्तनिधनान्येव} \\
 & \natline{तत्र का परिदेवना ||}
\end{tabular}
\end{table}

\begin{table}[H]
\begin{tabular}{cl}
\textbf{2.29} & \natline{आश्चर्यवत् पश्यति कश्चिदेनम्} \\
 & \natline{आश्चर्यवद् वदति तथैव चान्यः |} \\
 & \natline{आश्चर्यवच्चैनमन्यः शृणोति} \\
 & \natline{श्रुत्वाप्येनं वेद न चैव कश्चित् ||}
\end{tabular}
\end{table}

\begin{table}[H]
\begin{tabular}{cl}
\textbf{2.30} & \natline{देही नित्यमवध्योऽयं} \\
 & \natline{देहे सर्वस्य भारत |} \\
 & \natline{तस्मात्सर्वाणि भूतानि} \\
 & \natline{न त्वं शोचितुमर्हसि ||}
\end{tabular}
\end{table}

\begin{table}[H]
\begin{tabular}{cl}
\textbf{2.31} & \natline{स्वधर्ममपि चावेक्ष्य} \\
 & \natline{न विकम्पितुमर्हसि |} \\
 & \natline{धर्म्याद्धि युद्धाच्छ्रेयोऽन्यत्} \\
 & \natline{क्षत्रियस्यनविद्यते ||}
\end{tabular}
\end{table}

\begin{table}[H]
\begin{tabular}{cl}
\textbf{2.32} & \natline{यदृच्छया चोपपन्नं} \\
 & \natline{स्वर्गद्वारमपावृतम् |} \\
 & \natline{सुखिनः क्षत्रियाः पार्थ} \\
 & \natline{लभन्ते युद्धमीदृशम् ||}
\end{tabular}
\end{table}

\begin{table}[H]
\begin{tabular}{cl}
\textbf{2.33} & \natline{अथ चेत्त्वमिमं धर्म्यं} \\
 & \natline{सङ्ग्रामं न करिष्यसि |} \\
 & \natline{ततः स्वधर्मं कीर्तिं च} \\
 & \natline{हित्वा पापमवाप्स्यसि ||}
\end{tabular}
\end{table}

\begin{table}[H]
\begin{tabular}{cl}
\textbf{2.34} & \natline{अकीर्तिं चापि भूतानि} \\
 & \natline{कथयिष्यन्ति तेऽव्ययाम् |} \\
 & \natline{सम्भावितस्य चाकीर्तिः} \\
 & \natline{मरणादतिरिच्यते ||}
\end{tabular}
\end{table}

\begin{table}[H]
\begin{tabular}{cl}
\textbf{2.35} & \natline{भयाद्रणादुपरतं} \\
 & \natline{मंस्यन्ते त्वां महारथाः |} \\
 & \natline{येषां च त्वं बहुमतः} \\
 & \natline{भूत्वा यास्यसि लाघवम् ||}
\end{tabular}
\end{table}

\begin{table}[H]
\begin{tabular}{cl}
\textbf{2.36} & \natline{अवाच्यवादांश्च बहून्} \\
 & \natline{वदिष्यन्ति तवाहिताः |} \\
 & \natline{निन्दन्तस्तव सामर्थ्यं} \\
 & \natline{ततो दुःखतरं नु किम् ||}
\end{tabular}
\end{table}

\begin{table}[H]
\begin{tabular}{cl}
\textbf{2.37} & \natline{हतो वा प्राप्स्यसि स्वर्गं} \\
 & \natline{जित्वा वा भोक्ष्यसे महीम् |} \\
 & \natline{तस्मादुत्तिष्ठ कौन्तेय} \\
 & \natline{युद्धाय कृतनिश्चयः ||}
\end{tabular}
\end{table}

\begin{table}[H]
\begin{tabular}{cl}
\textbf{2.38} & \natline{सुखदुःखे समे कृत्वा} \\
 & \natline{लाभालाभौ जयाजयौ |} \\
 & \natline{ततो युद्धाय युज्यस्व} \\
 & \natline{नैवं पापमवाप्स्यसि ||}
\end{tabular}
\end{table}

\begin{table}[H]
\begin{tabular}{cl}
\textbf{2.39} & \natline{एषा तेऽभिहिता साङ्ख्ये} \\
 & \natline{बुद्धिर्योगे त्विमां शृणु |} \\
 & \natline{बुद्ध्या युक्तो यया पार्थ} \\
 & \natline{कर्मबन्धं प्रहास्यसि ||}
\end{tabular}
\end{table}

\begin{table}[H]
\begin{tabular}{cl}
\textbf{2.40} & \natline{नेहाभिक्रमनाशोऽस्ति} \\
 & \natline{प्रत्यवायो न विद्यते |} \\
 & \natline{स्वल्पमप्यस्य धर्मस्य} \\
 & \natline{त्रायते महतो भयात् ||}
\end{tabular}
\end{table}

\begin{table}[H]
\begin{tabular}{cl}
\textbf{2.41} & \natline{व्यवसायात्मिका बुद्धिः} \\
 & \natline{एकेह कुरुनन्दन |} \\
 & \natline{बहुशाखा ह्यनन्ताश्च} \\
 & \natline{बुद्धयोऽव्यवसायिनाम् ||}
\end{tabular}
\end{table}

\begin{table}[H]
\begin{tabular}{cl}
\textbf{2.42} & \natline{यामिमां पुष्पितां वाचं} \\
 & \natline{प्रवदन्त्यविपश्चितः |} \\
 & \natline{वेदवादरताः पार्थ} \\
 & \natline{नान्यदस्तीति वादिनः ||}
\end{tabular}
\end{table}

\begin{table}[H]
\begin{tabular}{cl}
\textbf{2.43} & \natline{कामात्मानः स्वर्गपराः} \\
 & \natline{जन्मकर्मफलप्रदाम् |} \\
 & \natline{क्रियाविशेषबहुलां} \\
 & \natline{भोगैश्वर्यगतिं प्रति ||}
\end{tabular}
\end{table}

\begin{table}[H]
\begin{tabular}{cl}
\textbf{2.44} & \natline{भोगैश्वर्यप्रसक्तानां} \\
 & \natline{तयाऽपहृतचेतसाम् |} \\
 & \natline{व्यवसायात्मिका बुद्धिः} \\
 & \natline{समाधौ न विधीयते ||}
\end{tabular}
\end{table}

\begin{table}[H]
\begin{tabular}{cl}
\textbf{2.45} & \natline{त्रैगुण्यविषया वेदाः} \\
 & \natline{निस्त्रैगुण्यो भवार्जुन |} \\
 & \natline{निर्द्वन्द्वो नित्यसत्त्वस्थः} \\
 & \natline{निर्योगक्षेम आत्मवान् ||}
\end{tabular}
\end{table}

\begin{table}[H]
\begin{tabular}{cl}
\textbf{2.46} & \natline{यावानर्थ उदपाने} \\
 & \natline{सर्वतः सम्प्लुतोदके |} \\
 & \natline{तावान्सर्वेषु वेदेषु} \\
 & \natline{ब्राह्मणस्य विजानतः ||}
\end{tabular}
\end{table}

\begin{table}[H]
\begin{tabular}{cl}
\textbf{2.47} & \natline{कर्मण्येवाधिकारस्ते} \\
 & \natline{मा फलेषु कदाचन |} \\
 & \natline{मा कर्मफलहेतुर्भूः} \\
 & \natline{मा ते सङ्गोऽस्त्वकर्मणि ||}
\end{tabular}
\end{table}

\begin{table}[H]
\begin{tabular}{cl}
\textbf{2.48} & \natline{योगस्थः कुरु कर्माणि} \\
 & \natline{सङ्गं त्यक्त्वा धनञ्जय |} \\
 & \natline{सिद्ध्यसिद्ध्योः समो भूत्वा} \\
 & \natline{समत्वं योग उच्यते ||}
\end{tabular}
\end{table}

\begin{table}[H]
\begin{tabular}{cl}
\textbf{2.49} & \natline{दूरेण ह्यवरं कर्म} \\
 & \natline{बुद्धियोगाद्धनञ्जय |} \\
 & \natline{बुद्धौ शरणमन्विच्छ} \\
 & \natline{कृपणाः फलहेतवः ||}
\end{tabular}
\end{table}

\begin{table}[H]
\begin{tabular}{cl}
\textbf{2.50} & \natline{बुद्धियुक्तो जहातीह} \\
 & \natline{उभे सुकृतदुष्कृते |} \\
 & \natline{तस्माद्योगाय युज्यस्व} \\
 & \natline{योगः कर्मसु कौशलम् ||}
\end{tabular}
\end{table}

\begin{table}[H]
\begin{tabular}{cl}
\textbf{2.51} & \natline{कर्मजं बुद्धियुक्ता हि} \\
 & \natline{फलं त्यक्त्वा मनीषिणः |} \\
 & \natline{जन्मबन्धविनिर्मुक्ताः} \\
 & \natline{पदं गच्छन्त्यनामयम् ||}
\end{tabular}
\end{table}

\begin{table}[H]
\begin{tabular}{cl}
\textbf{2.52} & \natline{यदा ते मोहकलिलं} \\
 & \natline{बुद्धिर् व्यतितरिष्यति |} \\
 & \natline{तदा गन्तासि निर्वेदं} \\
 & \natline{श्रोतव्यस्य श्रुतस्य च ||}
\end{tabular}
\end{table}

\begin{table}[H]
\begin{tabular}{cl}
\textbf{2.53} & \natline{श्रुतिविप्रतिपन्ना ते} \\
 & \natline{यदा स्थास्यति निश्चला |} \\
 & \natline{समाधावचला बुद्धिः} \\
 & \natline{तदा योगमवाप्स्यसि ||}
\end{tabular}
\end{table}

\begin{table}[H]
\begin{tabular}{cl}
\textbf{2.54} & \natline{अर्जुन उवाच} \\
 & \natline{स्थितप्रज्ञस्य का भाषा} \\
 & \natline{समाधिस्थस्य केशव |} \\
 & \natline{स्थितधीः किं प्रभाषेत} \\
 & \natline{किमासीत व्रजेत किम् ||}
\end{tabular}
\end{table}

\begin{table}[H]
\begin{tabular}{cl}
\textbf{2.55} & \natline{श्री भगवान् उवाच} \\
 & \natline{प्रजहाति यदा कामान्} \\
 & \natline{सर्वान्पार्थ मनोगतान् |} \\
 & \natline{आत्मन्येवात्मना तुष्टः} \\
 & \natline{स्थितप्रज्ञस् तदोच्यते ||}
\end{tabular}
\end{table}

\begin{table}[H]
\begin{tabular}{cl}
\textbf{2.56} & \natline{दुःखेष्वनुद्विग्नमनाः} \\
 & \natline{सुखेषु विगतस्पृहः |} \\
 & \natline{वीतरागभयक्रोधः} \\
 & \natline{स्थितधीर्मुनिरुच्यते ||}
\end{tabular}
\end{table}

\begin{table}[H]
\begin{tabular}{cl}
\textbf{2.57} & \natline{यः सर्वत्रानभिस्नेहः} \\
 & \natline{तत्तत्प्राप्य शुभाशुभम् |} \\
 & \natline{नाभिनन्दति न द्वेष्टि} \\
 & \natline{तस्य प्रज्ञा प्रतिष्ठिता ||}
\end{tabular}
\end{table}

\begin{table}[H]
\begin{tabular}{cl}
\textbf{2.58} & \natline{यदा सम्हरते चायं} \\
 & \natline{कूर्मोऽङ्गानीव सर्वशः |} \\
 & \natline{इन्द्रियाणीन्द्रियार्थेभ्यः} \\
 & \natline{तस्य प्रज्ञा प्रतिष्ठिता ||}
\end{tabular}
\end{table}

\begin{table}[H]
\begin{tabular}{cl}
\textbf{2.59} & \natline{विषया विनिवर्तन्ते} \\
 & \natline{निराहारस्य देहिनः |} \\
 & \natline{रसवर्जं रसोऽप्यस्य} \\
 & \natline{परं दृष्ट्वा निवर्तते ||}
\end{tabular}
\end{table}

\begin{table}[H]
\begin{tabular}{cl}
\textbf{2.60} & \natline{यततो ह्यपि कौन्तेय} \\
 & \natline{पुरुषस्य विपश्चितः |} \\
 & \natline{इन्द्रियाणि प्रमाथीनि} \\
 & \natline{हरन्ति प्रसभं मनः ||}
\end{tabular}
\end{table}

\begin{table}[H]
\begin{tabular}{cl}
\textbf{2.61} & \natline{तानि सर्वाणि सम्यम्य} \\
 & \natline{युक्त आसीत मत्परः |} \\
 & \natline{वशे हि यस्येन्द्रियाणि} \\
 & \natline{तस्य प्रज्ञा प्रतिष्ठिता ||}
\end{tabular}
\end{table}

\begin{table}[H]
\begin{tabular}{cl}
\textbf{2.62} & \natline{ध्यायतो विषयान्पुंसः} \\
 & \natline{सङ्गस्तेषूपजायते |} \\
 & \natline{सङ्गात्सञ्जायते कामः} \\
 & \natline{कामात्क्रोधोऽभिजायते ||}
\end{tabular}
\end{table}

\begin{table}[H]
\begin{tabular}{cl}
\textbf{2.63} & \natline{क्रोधाद् भवति सम्मोहः} \\
 & \natline{सम्मोहात् स्मृतिविभ्रमः |} \\
 & \natline{स्मृतिभ्रम्शात् बुद्धिनाशः} \\
 & \natline{बुद्धिनाशात्प्रणश्यति ||}
\end{tabular}
\end{table}

\begin{table}[H]
\begin{tabular}{cl}
\textbf{2.64} & \natline{रागद्वेषवियुक्तैस्तु} \\
 & \natline{विषयानिन्द्रियैश्चरन् |} \\
 & \natline{आत्मवश्यैर्विधेयात्मा} \\
 & \natline{प्रसादमधिगच्छति ||}
\end{tabular}
\end{table}

\begin{table}[H]
\begin{tabular}{cl}
\textbf{2.65} & \natline{प्रसादे सर्वदुःखानां} \\
 & \natline{हानिरस्योपजायते |} \\
 & \natline{प्रसन्नचेतसो ह्याशु} \\
 & \natline{बुद्धिः पर्यवतिष्ठते ||}
\end{tabular}
\end{table}

\begin{table}[H]
\begin{tabular}{cl}
\textbf{2.66} & \natline{नास्ति बुद्धिरयुक्तस्य} \\
 & \natline{न चायुक्तस्य भावना |} \\
 & \natline{न चाभावयतः शान्तिः} \\
 & \natline{अशान्तस्य कुतः सुखम् ||}
\end{tabular}
\end{table}

\begin{table}[H]
\begin{tabular}{cl}
\textbf{2.67} & \natline{इन्द्रियाणां हि चरतां} \\
 & \natline{यन्मनोऽनुविधीयते |} \\
 & \natline{तदस्य हरति प्रज्ञां} \\
 & \natline{वायुर्नावमिवाम्भसि ||}
\end{tabular}
\end{table}

\begin{table}[H]
\begin{tabular}{cl}
\textbf{2.68} & \natline{तस्माद् यस्य महाबाहो} \\
 & \natline{निगृहीतानि सर्वशः |} \\
 & \natline{इन्द्रियाणीन्द्रियार्थेभ्यः} \\
 & \natline{तस्य प्रज्ञा प्रतिष्ठिता ||}
\end{tabular}
\end{table}

\begin{table}[H]
\begin{tabular}{cl}
\textbf{2.69} & \natline{या निशा सर्वभूतानां} \\
 & \natline{तस्यां जागर्ति संयमी |} \\
 & \natline{यस्यां जाग्रति भूतानि} \\
 & \natline{सा निशा पश्यतो मुनेः ||}
\end{tabular}
\end{table}

\begin{table}[H]
\begin{tabular}{cl}
\textbf{2.70} & \natline{आपुर्यमाणमचलप्रतिष्ठं} \\
 & \natline{समुद्रमापः प्रविशन्ति यद्वत् |} \\
 & \natline{तद्वत्कामा यं प्रविशन्ति सर्वे} \\
 & \natline{स शान्तिमाप्नोति न कामकामी ||}
\end{tabular}
\end{table}

\begin{table}[H]
\begin{tabular}{cl}
\textbf{2.71} & \natline{विहाय कामान् यः सर्वान्} \\
 & \natline{पुमाम्श्चरति निस्स्पृहः |} \\
 & \natline{निर्ममो निरहन्कारः} \\
 & \natline{स शान्तिमधिगच्छति ||}
\end{tabular}
\end{table}

\begin{table}[H]
\begin{tabular}{cl}
\textbf{2.72} & \natline{एषा ब्राह्मी स्थितिः पार्थ} \\
 & \natline{नैनां प्राप्य विमुह्यति |} \\
 & \natline{स्थित्वास्यामन्तकालेऽपि} \\
 & \natline{ब्रह्मनिर्वाणमृच्छति ||}
\end{tabular}
\end{table}

\begin{table}[H]
\begin{tabular}{cl}
 & \natline{श्रीमद्भगवद्गीतासु उपनिषत्सु} \\
 & \natline{ब्रह्मविद्यायां योगशास्त्रे} \\
 & \natline{श्रीकृष्णार्जुन संवादे} \\
 & \natline{साङ्ख्ययोगोनाम} \\
 & \natline{द्वितीयोध्यायः}
\end{tabular}
\end{table}

