\begin{table}[H]
\begin{tabular}{cl}
 & \natline{श्री परमात्मने नमः} \\
 & \natline{अथ सप्तमोऽध्यायः} \\
 & \natline{ज्ञानविज्ञानयोगः}
\end{tabular}
\end{table}

\begin{table}[H]
\begin{tabular}{cl}
\textbf{7.1} & \natline{श्री भगवानुवाच} \\
 & \natline{मय्यासक्तमनाः पार्थ} \\
 & \natline{योगं युञ्जन्मदाश्रयः |} \\
 & \natline{असंशयं समग्रं मां} \\
 & \natline{यथा ज्ञास्यसि तच्छृणु ||}
\end{tabular}
\end{table}

\begin{table}[H]
\begin{tabular}{cl}
\textbf{7.2} & \natline{ज्ञानं तेऽहं सविज्ञानम्} \\
 & \natline{इदं वक्ष्याम्यशेषतः |} \\
 & \natline{यज्ज्ञात्वा नेह भूयोऽन्यत्} \\
 & \natline{ज्ञातव्यमवशिष्यते ||}
\end{tabular}
\end{table}

\begin{table}[H]
\begin{tabular}{cl}
\textbf{7.3} & \natline{मनुष्याणां सहस्रेषु} \\
 & \natline{कश्चिद्यतति सिद्धये |} \\
 & \natline{यततामपि सिद्धानां} \\
 & \natline{कस्चिन्मां वेत्ति तत्त्वतह् ||}
\end{tabular}
\end{table}

\begin{table}[H]
\begin{tabular}{cl}
\textbf{7.4} & \natline{भूमिरापोऽनलो वायुः} \\
 & \natline{खं मनो बुद्धिरेव च |} \\
 & \natline{अहङ्कार इतीयं मे} \\
 & \natline{भिन्ना प्रकृतिरष्टधा ||}
\end{tabular}
\end{table}

\begin{table}[H]
\begin{tabular}{cl}
\textbf{7.5} & \natline{अपरेयमितस्त्वन्यां} \\
 & \natline{प्रकृतिं विद्धि मे पराम् |} \\
 & \natline{जीवभूतां महाबाहो} \\
 & \natline{ययेदं धार्यते जगत् ||}
\end{tabular}
\end{table}

\begin{table}[H]
\begin{tabular}{cl}
\textbf{7.6} & \natline{एतद्योनीनि भूतानि} \\
 & \natline{सर्वाणीत्युपधारय |} \\
 & \natline{अहं कृत्स्नस्य जगतः} \\
 & \natline{प्रभवः प्रलयस्तथा ||}
\end{tabular}
\end{table}

\begin{table}[H]
\begin{tabular}{cl}
\textbf{7.7} & \natline{मत्तः परतरं नान्यत्} \\
 & \natline{किञ्चिदस्ति धनञ्जय |} \\
 & \natline{मयि सर्वमिदं प्रोतं} \\
 & \natline{सूत्रे मणिगणा इव ||}
\end{tabular}
\end{table}

\begin{table}[H]
\begin{tabular}{cl}
\textbf{7.8} & \natline{रसोऽहमप्सु कौन्तेय} \\
 & \natline{प्रभाऽस्मि शशिसूर्ययोः |} \\
 & \natline{प्रणवः सर्ववेदेषु} \\
 & \natline{शब्दः खे पौरुषं नृषु ||}
\end{tabular}
\end{table}

\begin{table}[H]
\begin{tabular}{cl}
\textbf{7.9} & \natline{पुण्यो गन्धः पृथिव्यां च} \\
 & \natline{तेजश्चास्मि विभावसौ |} \\
 & \natline{जीवनं सर्वभुतेषु} \\
 & \natline{तपश्चास्मि तपस्विषु ||}
\end{tabular}
\end{table}

\begin{table}[H]
\begin{tabular}{cl}
\textbf{7.10} & \natline{बीजं मां सर्वभूतानां} \\
 & \natline{विद्धि पार्थ सनातनम् |} \\
 & \natline{बुद्धिर्बुद्धिमतामस्मि} \\
 & \natline{तेजस्तेजस्विनामहम् ||}
\end{tabular}
\end{table}

\begin{table}[H]
\begin{tabular}{cl}
\textbf{7.11} & \natline{बलं बलवतां चाहं} \\
 & \natline{कामरागविवर्जितम् |} \\
 & \natline{धर्माविरुद्धो भूतेषु} \\
 & \natline{कामोऽस्मि भरतर्षभ ||}
\end{tabular}
\end{table}

\begin{table}[H]
\begin{tabular}{cl}
\textbf{7.12} & \natline{ये चैव सात्त्विका भावाः} \\
 & \natline{राजसास्तामसाश्च ये |} \\
 & \natline{मत्त एवेति तान्विद्धि} \\
 & \natline{न त्वहं तेषु ते मयि ||}
\end{tabular}
\end{table}

\begin{table}[H]
\begin{tabular}{cl}
\textbf{7.13} & \natline{त्रिभिर्गुणमयैर्भावैः} \\
 & \natline{एभिः सर्वमिदं जगत् |} \\
 & \natline{मोहितं नाभिजानाति} \\
 & \natline{मामेभ्यः परमव्ययम् ||}
\end{tabular}
\end{table}

\begin{table}[H]
\begin{tabular}{cl}
\textbf{7.14} & \natline{दैवी ह्येषा गुनामयी} \\
 & \natline{मम माय दुरत्यया |} \\
 & \natline{मामेव ये प्रपद्यन्ते} \\
 & \natline{मायामेतां तरन्ति ते ||}
\end{tabular}
\end{table}

\begin{table}[H]
\begin{tabular}{cl}
\textbf{7.15} & \natline{न मां दुष्कृतिनो मूढाः} \\
 & \natline{प्रपद्यन्ते नराधमाः |} \\
 & \natline{माययाऽपहृतज्ञानाः} \\
 & \natline{आसुरं भावमाश्रिताः ||}
\end{tabular}
\end{table}

\begin{table}[H]
\begin{tabular}{cl}
\textbf{7.16} & \natline{चतुर्विधा भजन्ते मां} \\
 & \natline{जनाः सुकृतिनोऽर्जुन |} \\
 & \natline{आर्तो जिज्ञासुरर्थार्थी} \\
 & \natline{ज्ञानी च भरतर्षभ ||}
\end{tabular}
\end{table}

\begin{table}[H]
\begin{tabular}{cl}
\textbf{7.17} & \natline{तेषां ज्ञानी नित्ययुक्तः} \\
 & \natline{एकभक्तिर्विशिष्यते |} \\
 & \natline{प्रियो हि ज्ञानिनोऽत्यर्थम्} \\
 & \natline{अहं स च मम प्रियः ||}
\end{tabular}
\end{table}

\begin{table}[H]
\begin{tabular}{cl}
\textbf{7.18} & \natline{उदाराः सर्व एवैते} \\
 & \natline{ज्ञानी त्वात्मैव मे मतम् |} \\
 & \natline{आस्थितः स हि युक्तात्मा} \\
 & \natline{मामेवानुत्तमां गतिम् ||}
\end{tabular}
\end{table}

\begin{table}[H]
\begin{tabular}{cl}
\textbf{7.19} & \natline{बहूनां जन्मनामन्ते} \\
 & \natline{ज्ञानवान्मां प्रपद्यते |} \\
 & \natline{वासुदेवः सर्वमिति} \\
 & \natline{स महात्मा सुदुर्लभः ||}
\end{tabular}
\end{table}

\begin{table}[H]
\begin{tabular}{cl}
\textbf{7.20} & \natline{कामैस्तैस्तैर्हृतज्ञानाः} \\
 & \natline{प्रपद्यन्तेऽन्यदेवताः |} \\
 & \natline{तं तं नियममास्थाय} \\
 & \natline{प्रकृत्या नियताः स्वया ||}
\end{tabular}
\end{table}

\begin{table}[H]
\begin{tabular}{cl}
\textbf{7.21} & \natline{यो यो यां यां तनुं भक्तः} \\
 & \natline{श्रद्धयार्चितुमिच्छति |} \\
 & \natline{तस्य तस्याचलां श्रद्धां} \\
 & \natline{तामेव विदधाम्यहम् ||}
\end{tabular}
\end{table}

\begin{table}[H]
\begin{tabular}{cl}
\textbf{7.22} & \natline{स तया श्रद्धया युक्तः} \\
 & \natline{तस्याराधनमीहते |} \\
 & \natline{लभते च ततः कामान्} \\
 & \natline{मयैव विहितान्हि तान् ||}
\end{tabular}
\end{table}

\begin{table}[H]
\begin{tabular}{cl}
\textbf{7.23} & \natline{अन्तवत्तु फलं तेषां} \\
 & \natline{तद्भवत्यल्पमेधसां |} \\
 & \natline{देवान्देवयजो यान्ति} \\
 & \natline{मद्भक्ता यान्ति मामपि ||}
\end{tabular}
\end{table}

\begin{table}[H]
\begin{tabular}{cl}
\textbf{7.24} & \natline{अव्यक्तं व्यक्तिमापन्नं} \\
 & \natline{मन्यन्ते मामबुद्धयः |} \\
 & \natline{परं भावमजानन्तः} \\
 & \natline{ममाव्ययमनुत्तमन् ||}
\end{tabular}
\end{table}

\begin{table}[H]
\begin{tabular}{cl}
\textbf{7.25} & \natline{नाहं प्रकाशः सर्वस्य} \\
 & \natline{योगमायासमावृतः |} \\
 & \natline{मूढोऽयं नाभिजानाति} \\
 & \natline{लोको मामजमव्ययम् ||}
\end{tabular}
\end{table}

\begin{table}[H]
\begin{tabular}{cl}
\textbf{7.26} & \natline{वेदाहं समतीतानि} \\
 & \natline{वर्तमानानि चार्जुन |} \\
 & \natline{भविष्याणि च भूतानि} \\
 & \natline{मां तु वेद न कश्चन ||}
\end{tabular}
\end{table}

\begin{table}[H]
\begin{tabular}{cl}
\textbf{7.27} & \natline{इच्छाद्वेषसमुत्थेन} \\
 & \natline{द्वन्द्वमोहेन भारत |} \\
 & \natline{सर्वभूतानि सम्मोहं} \\
 & \natline{सर्गे यान्ति परन्तप ||}
\end{tabular}
\end{table}

\begin{table}[H]
\begin{tabular}{cl}
\textbf{7.28} & \natline{येषां त्वन्तगतं पापम्} \\
 & \natline{जनानां पुण्यकर्मणाम् |} \\
 & \natline{ते द्वन्द्वमोहनिर्मुक्ताः} \\
 & \natline{भजन्ते मां दृढव्रताः ||}
\end{tabular}
\end{table}

\begin{table}[H]
\begin{tabular}{cl}
\textbf{7.29} & \natline{जरामरणमोक्षाय} \\
 & \natline{मामाश्रित्य यतन्ति ये |} \\
 & \natline{ते ब्रह्म तद्विदुः कृत्स्नम्} \\
 & \natline{अध्यात्मं कर्म चाखिलम् ||}
\end{tabular}
\end{table}

\begin{table}[H]
\begin{tabular}{cl}
\textbf{7.30} & \natline{साधिभूताधिदैवं मां} \\
 & \natline{साधियज्ञं चे ये विदुः |} \\
 & \natline{प्रयाणकालेऽपि च मां} \\
 & \natline{ते विदुर्युक्तचेतसः ||}
\end{tabular}
\end{table}

\begin{table}[H]
\begin{tabular}{cl}
 & \natline{श्रीमद्भगवद्गीतासु उपनिषत्सु} \\
 & \natline{ब्रह्मविद्यायां योगशास्त्रे} \\
 & \natline{श्रीकृष्णार्जुन संवादे} \\
 & \natline{ज्ञानविज्ञानयोगो नाम} \\
 & \natline{सप्तमोध्यायः}
\end{tabular}
\end{table}

