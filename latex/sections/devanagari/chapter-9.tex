\begin{table}[H]
\begin{tabular}{cl}
\textbf{9.0} & \natline{ओं श्री परमात्मने नमः} \\
 & \natline{अथ नवमोऽध्यायः} \\
 & \natline{राजविद्याराजगुह्ययोगः}
\end{tabular}
\end{table}

\begin{table}[H]
\begin{tabular}{cl}
\textbf{9.1} & \natline{श्री भगवानुवाच} \\
 & \natline{इदं तु ते गुह्यतमम्} \\
 & \natline{प्रवक्ष्याम्यनसूयवे |} \\
 & \natline{ज्ञानं विज्ञानसहितं} \\
 & \natline{यज्ज्ञात्वा मोक्ष्यसेऽशुभात् ||}
\end{tabular}
\end{table}

\begin{table}[H]
\begin{tabular}{cl}
\textbf{9.2} & \natline{राजविद्या राजगुह्यं} \\
 & \natline{पवित्रमिदमुत्तमम् |} \\
 & \natline{प्रत्यक्षावगमं धर्म्यं} \\
 & \natline{सुसुखं कर्तुमव्ययम् ||}
\end{tabular}
\end{table}

\begin{table}[H]
\begin{tabular}{cl}
\textbf{9.3} & \natline{अश्रद्दधानाः पुरुषाः} \\
 & \natline{धर्मस्यास्य परन्तप |} \\
 & \natline{अप्राप्य मां निवर्तन्ते} \\
 & \natline{मृत्युसंसारवर्त्मनि ||}
\end{tabular}
\end{table}

\begin{table}[H]
\begin{tabular}{cl}
\textbf{9.4} & \natline{मया ततमिदं सर्वं} \\
 & \natline{जगदव्यक्तमूर्तिना |} \\
 & \natline{मत्स्थानि सर्वभूतानि} \\
 & \natline{न चाहं तेष्ववस्थितः ||}
\end{tabular}
\end{table}

\begin{table}[H]
\begin{tabular}{cl}
\textbf{9.5} & \natline{न च मत्स्थानि भूतानि} \\
 & \natline{पश्य मे योगमैश्वरम् |} \\
 & \natline{भूतभृन्न च भूतस्थः} \\
 & \natline{ममात्मा भूतभावनः ||}
\end{tabular}
\end{table}

\begin{table}[H]
\begin{tabular}{cl}
\textbf{9.6} & \natline{यथाऽऽकाशस्थितो नित्यं} \\
 & \natline{वायुः सर्वत्रगो महान् |} \\
 & \natline{तथा सर्वाणि भूतानि} \\
 & \natline{मत्स्थानीत्युपधारय ||}
\end{tabular}
\end{table}

\begin{table}[H]
\begin{tabular}{cl}
\textbf{9.7} & \natline{सर्वभूतानि कौन्तेय} \\
 & \natline{प्रकृतिं यानि मामिकाम् |} \\
 & \natline{कल्पक्षये पुनस्तानि} \\
 & \natline{कल्पादौ विसृजाम्यहम् ||}
\end{tabular}
\end{table}

\begin{table}[H]
\begin{tabular}{cl}
\textbf{9.8} & \natline{प्रकृतिं स्वामवष्टभ्य} \\
 & \natline{विसृजामि पुनः पुनः |} \\
 & \natline{भूतग्राममिमं कृत्स्नम्} \\
 & \natline{अवशं प्रकृतेर्वशात् ||}
\end{tabular}
\end{table}

\begin{table}[H]
\begin{tabular}{cl}
\textbf{9.9} & \natline{न च मां तानि कर्माणि} \\
 & \natline{निबध्नन्ति धनञ्जय |} \\
 & \natline{उदासीनवदासीनम्} \\
 & \natline{असक्तं तेषु कर्मसु ||}
\end{tabular}
\end{table}

\begin{table}[H]
\begin{tabular}{cl}
\textbf{9.10} & \natline{मयाध्यक्षेण प्रकृतिः} \\
 & \natline{सूयते सचराचरम् |} \\
 & \natline{हेतुनाऽनेन कौन्तेय} \\
 & \natline{जगद्विपरिवर्तते ||}
\end{tabular}
\end{table}

\begin{table}[H]
\begin{tabular}{cl}
\textbf{9.11} & \natline{अवजानन्ति मां मूढाः} \\
 & \natline{मानुषीं तनुमाश्रितम् |} \\
 & \natline{परं भावमजानन्तः} \\
 & \natline{मम भूतमहेश्वरम् ||}
\end{tabular}
\end{table}

\begin{table}[H]
\begin{tabular}{cl}
\textbf{9.12} & \natline{मोघाशा मोघकर्माणः} \\
 & \natline{मोघज्ञाना विचेतसः |} \\
 & \natline{राक्षसीमासुरीं चैव} \\
 & \natline{प्रकृतिं मोहिनीं श्रिताः ||}
\end{tabular}
\end{table}

\begin{table}[H]
\begin{tabular}{cl}
\textbf{9.13} & \natline{महात्मानस्तु मां पार्थ} \\
 & \natline{दैवीं प्रकृतिमाश्रिताः |} \\
 & \natline{भजन्त्यनन्यमनसः} \\
 & \natline{ज्ञात्वा भूतादिमव्ययम् ||}
\end{tabular}
\end{table}

\begin{table}[H]
\begin{tabular}{cl}
\textbf{9.14} & \natline{सततं कीर्तयन्तो मां} \\
 & \natline{यतन्तश्च दृढव्रताः |} \\
 & \natline{नमस्यन्तश्च माम् भक्त्या} \\
 & \natline{नित्ययुक्ता उपासते ||}
\end{tabular}
\end{table}

\begin{table}[H]
\begin{tabular}{cl}
\textbf{9.15} & \natline{ज्ञानयज्ञेन चाप्यन्ये} \\
 & \natline{यजन्तो मामुपासते |} \\
 & \natline{एकत्वेन पृथक्त्वेन} \\
 & \natline{बहुधा विश्वतोमुखम् ||}
\end{tabular}
\end{table}

\begin{table}[H]
\begin{tabular}{cl}
\textbf{9.16} & \natline{अहं क्रतुरहं यज्ञः} \\
 & \natline{स्वधाहमहमौषधम् |} \\
 & \natline{मन्त्रोऽहमहमेवाज्यम्} \\
 & \natline{अहमग्निरहं हुतम् ||}
\end{tabular}
\end{table}

\begin{table}[H]
\begin{tabular}{cl}
\textbf{9.17} & \natline{पिताऽहमस्य जगतः} \\
 & \natline{माता धाता पितामहः |} \\
 & \natline{वेद्यं पवित्रमोङ्कारः} \\
 & \natline{ऋक्साम यजुरेव च ||}
\end{tabular}
\end{table}

\begin{table}[H]
\begin{tabular}{cl}
\textbf{9.18} & \natline{गतिर्भर्ता प्रभुः साक्षी} \\
 & \natline{निवासः शरणं सुहृत् |} \\
 & \natline{प्रभवः प्रलयः स्थानं} \\
 & \natline{निधानं बीजमव्ययम् ||}
\end{tabular}
\end{table}

\begin{table}[H]
\begin{tabular}{cl}
\textbf{9.19} & \natline{तपाम्यहमहं वर्षं} \\
 & \natline{निगृह्णाम्युत्सृजामि च |} \\
 & \natline{अमृतं चैव मृत्युश्च} \\
 & \natline{सदसच्चाहमर्जुन ||}
\end{tabular}
\end{table}

\begin{table}[H]
\begin{tabular}{cl}
\textbf{9.20} & \natline{त्रैविद्या मां सोमपाः पूतपापाः} \\
 & \natline{यज्ञैरिष्ट्वा स्वर्गतिं प्रार्थयन्ते |} \\
 & \natline{ते पुण्यमासाद्य सुरेन्द्रलोकम्} \\
 & \natline{अश्नन्ति दिव्यान्दिवि देवभोगान् ||}
\end{tabular}
\end{table}

\begin{table}[H]
\begin{tabular}{cl}
\textbf{9.21} & \natline{ते तं भुक्त्वा स्वर्गलोकं विशालं} \\
 & \natline{क्षीणे पुण्ये मर्त्यलोकं विशन्ति |} \\
 & \natline{एवं त्रयीधर्ममनुप्रपन्नाः} \\
 & \natline{गतागतं कामकामा लभन्ते ||}
\end{tabular}
\end{table}

\begin{table}[H]
\begin{tabular}{cl}
\textbf{9.22} & \natline{अनन्याश्चिन्तयन्तो मां} \\
 & \natline{ये जनाः पर्युपासते |} \\
 & \natline{तेषां नित्याभियुक्तानां} \\
 & \natline{योगक्षेमं वहाम्यहम् ||}
\end{tabular}
\end{table}

\begin{table}[H]
\begin{tabular}{cl}
\textbf{9.23} & \natline{येऽप्यन्यदेवता भक्ताः} \\
 & \natline{यजन्ते श्रद्धयान्विताः |} \\
 & \natline{तेऽपि मामेव कौन्तेय} \\
 & \natline{यजन्त्यविधिपूर्वकम् ||}
\end{tabular}
\end{table}

\begin{table}[H]
\begin{tabular}{cl}
\textbf{9.24} & \natline{अहं हि सर्वयज्ञानां} \\
 & \natline{भोक्ता च प्रभुरेव च |} \\
 & \natline{न तु मामभिजानन्ति} \\
 & \natline{तत्त्वेनातश्च्यवन्ति ते ||}
\end{tabular}
\end{table}

\begin{table}[H]
\begin{tabular}{cl}
\textbf{9.25} & \natline{यान्ति देवव्रता देवाण्} \\
 & \natline{पितॄन् यान्ति पितृव्रताः |} \\
 & \natline{भूतानि यान्ति भूतेज्याः} \\
 & \natline{यान्ति मद्याजिनोऽपि माम् ||}
\end{tabular}
\end{table}

\begin{table}[H]
\begin{tabular}{cl}
\textbf{9.26} & \natline{पत्रं पुष्पं फलं तोयं} \\
 & \natline{यो मे भक्त्या प्रयच्छति |} \\
 & \natline{तदहं भक्त्युपहृतं} \\
 & \natline{अश्नामि प्रयतात्मनः ||}
\end{tabular}
\end{table}

\begin{table}[H]
\begin{tabular}{cl}
\textbf{9.27} & \natline{यत्करोषि यदश्नासि} \\
 & \natline{यज्जुहोषि ददासि यत् |} \\
 & \natline{यत्तपस्यसि कौन्तेय} \\
 & \natline{तत्कुरुष्व मदर्पणम् ||}
\end{tabular}
\end{table}

\begin{table}[H]
\begin{tabular}{cl}
\textbf{9.28} & \natline{शुभाशुभफलैरेवं} \\
 & \natline{मोक्ष्यसे कर्मबन्धनैः |} \\
 & \natline{सन्न्यासयोगयुक्तात्मा} \\
 & \natline{विमुक्तो मामुपैष्यसि ||}
\end{tabular}
\end{table}

\begin{table}[H]
\begin{tabular}{cl}
\textbf{9.29} & \natline{समोऽहं सर्वभूतेषु} \\
 & \natline{न मे द्वेष्योऽस्ति न प्रियः |} \\
 & \natline{ये भजन्ति तु मां भक्त्या} \\
 & \natline{मयि ते तेषु चाप्यहम् ||}
\end{tabular}
\end{table}

\begin{table}[H]
\begin{tabular}{cl}
\textbf{9.30} & \natline{अपि चेत्सुदुराचारः} \\
 & \natline{भजते मामनन्यभाक् |} \\
 & \natline{साधुरेव स मन्तव्यः} \\
 & \natline{सम्यग्व्यवसितो हि सः ||}
\end{tabular}
\end{table}

\begin{table}[H]
\begin{tabular}{cl}
\textbf{9.31} & \natline{क्षिप्रम् भवति धर्मात्मा} \\
 & \natline{शश्वच्छान्तिं निगच्छति |} \\
 & \natline{कौन्तेय प्रतिजानीहि} \\
 & \natline{न मे भक्तः प्रणश्यति ||}
\end{tabular}
\end{table}

\begin{table}[H]
\begin{tabular}{cl}
\textbf{9.32} & \natline{माम् हि पार्थ व्यपाश्रित्य} \\
 & \natline{येऽपि स्युः पापयोनयः |} \\
 & \natline{स्त्रियो वैश्यास्तथा शूद्राः} \\
 & \natline{तेऽपि यान्ति परां गतिम् ||}
\end{tabular}
\end{table}

\begin{table}[H]
\begin{tabular}{cl}
\textbf{9.33} & \natline{किं पुनर्ब्राह्मणाः पुण्याः} \\
 & \natline{भक्ता राजर्षयस्तथा |} \\
 & \natline{अनित्यमसुखं लोकम्} \\
 & \natline{इमं प्राप्य भजस्व माम् ||}
\end{tabular}
\end{table}

\begin{table}[H]
\begin{tabular}{cl}
\textbf{9.34} & \natline{मन्मना भव मद्भक्तः} \\
 & \natline{मद्याजी मां नमस्कुरु |} \\
 & \natline{मामेवैष्यसि युक्त्वैवम्} \\
 & \natline{आत्मानं मत्परायणः ||}
\end{tabular}
\end{table}

