\begin{table}[H]
\begin{tabular}{cl}
 & \natline{श्री परमात्मने नमः} \\
 & \natline{अथ षष्ठोऽध्यायः} \\
 & \natline{आत्मसंयमयोगः}
\end{tabular}
\end{table}

\begin{table}[H]
\begin{tabular}{cl}
\textbf{6.1} & \natline{श्री भगवानुवाच} \\
 & \natline{अनाश्रितः कर्मफलं} \\
 & \natline{कार्यं कर्म करोति यः |} \\
 & \natline{स सन्न्यासी च योगी च} \\
 & \natline{न निरग्निर्न चाक्रियः ||}
\end{tabular}
\end{table}

\begin{table}[H]
\begin{tabular}{cl}
\textbf{6.2} & \natline{यं सन्न्यासमिति प्राहुः} \\
 & \natline{योगं तं विद्धि पाण्डव |} \\
 & \natline{न ह्यसन्न्यस्तसङ्कल्पः} \\
 & \natline{योगी भवति कश्चन ||}
\end{tabular}
\end{table}

\begin{table}[H]
\begin{tabular}{cl}
\textbf{6.3} & \natline{आरुरुक्षोर्मुनेर्योगं} \\
 & \natline{कर्म कारणमुच्यते |} \\
 & \natline{योगारूढस्य तस्यैव} \\
 & \natline{शमः कारणमुच्यते ||}
\end{tabular}
\end{table}

\begin{table}[H]
\begin{tabular}{cl}
\textbf{6.4} & \natline{यदा हि नेन्द्रियार्थेषु} \\
 & \natline{न कर्मस्वनुषज्जते |} \\
 & \natline{सर्वसङ्कल्पसन्न्यासी} \\
 & \natline{योगारूढस्तदोच्यते ||}
\end{tabular}
\end{table}

\begin{table}[H]
\begin{tabular}{cl}
\textbf{6.5} & \natline{उद्धरेदात्मनाऽऽत्मानं} \\
 & \natline{नात्मानमवसादयेत् |} \\
 & \natline{आत्मैव ह्यात्मनो बन्धुः} \\
 & \natline{आत्मैव रिपुरात्मनः ||}
\end{tabular}
\end{table}

\begin{table}[H]
\begin{tabular}{cl}
\textbf{6.6} & \natline{बन्धुरात्माऽऽत्मनस्तस्य} \\
 & \natline{येनात्मैवात्मना जितः |} \\
 & \natline{अनात्मनस्तु शत्रुत्वे} \\
 & \natline{वर्तेतात्मैव शत्रुवत् ||}
\end{tabular}
\end{table}

\begin{table}[H]
\begin{tabular}{cl}
\textbf{6.7} & \natline{जितात्मनः प्रशान्तस्य} \\
 & \natline{परमात्मा समाहितः |} \\
 & \natline{शीतोष्णसुखदुःखेषु} \\
 & \natline{तथा मानापमानयोः ||}
\end{tabular}
\end{table}

\begin{table}[H]
\begin{tabular}{cl}
\textbf{6.8} & \natline{ज्ञानविज्ञानतृप्तात्मा} \\
 & \natline{कूटस्थो विजितेन्द्रियः |} \\
 & \natline{युक्त इत्युच्यते योगी} \\
 & \natline{समलोष्टाश्मकाञ्चनः ||}
\end{tabular}
\end{table}

\begin{table}[H]
\begin{tabular}{cl}
\textbf{6.9} & \natline{सुहृन्मित्रार्युदासीन} \\
 & \natline{मध्यस्थद्वेष्यबन्धुषु |} \\
 & \natline{साधुष्वपि च पापेषु} \\
 & \natline{समबुद्धिर्विशिष्यते ||}
\end{tabular}
\end{table}

\begin{table}[H]
\begin{tabular}{cl}
\textbf{6.10} & \natline{योगी युञ्जीत सततम्} \\
 & \natline{आत्मानं रहसि स्थितः |} \\
 & \natline{एकाकी यतचित्तात्मा} \\
 & \natline{निराशीरपरिग्रहः ||}
\end{tabular}
\end{table}

\begin{table}[H]
\begin{tabular}{cl}
\textbf{6.11} & \natline{शुचौ देशे प्रतिष्ठाप्य} \\
 & \natline{स्थिरमासनमात्मनः |} \\
 & \natline{नात्युच्छ्रितं नातिनीचं} \\
 & \natline{चैलाजिनकुशोत्तरम् ||}
\end{tabular}
\end{table}

\begin{table}[H]
\begin{tabular}{cl}
\textbf{6.12} & \natline{तत्रैकाग्रं मनः कृत्वा} \\
 & \natline{यतचित्तेन्द्रियक्रियः |} \\
 & \natline{उपविश्यासने युञ्ज्यात्} \\
 & \natline{योगमात्मविशुद्धये ||}
\end{tabular}
\end{table}

\begin{table}[H]
\begin{tabular}{cl}
\textbf{6.13} & \natline{समं कायशिरोग्रीवं} \\
 & \natline{धारयन्नचलं स्थिरः |} \\
 & \natline{सम्प्रेक्ष्य नासिकाग्रं स्वं} \\
 & \natline{दिशश्चानवलोकयन् ||}
\end{tabular}
\end{table}

\begin{table}[H]
\begin{tabular}{cl}
\textbf{6.14} & \natline{प्रशान्तात्मा विगतभीः} \\
 & \natline{ब्रह्मचारिव्रते स्थितः |} \\
 & \natline{मनः संयम्य मच्चित्तः} \\
 & \natline{युक्त आसीत मत्परः ||}
\end{tabular}
\end{table}

\begin{table}[H]
\begin{tabular}{cl}
\textbf{6.15} & \natline{युञ्जन्नेवं सदाऽऽत्मानं} \\
 & \natline{योगी नियतमानसः |} \\
 & \natline{शान्तिं निर्वाणपरमां} \\
 & \natline{मत्संस्थामधिगच्छति ||}
\end{tabular}
\end{table}

\begin{table}[H]
\begin{tabular}{cl}
\textbf{6.16} & \natline{नात्यश्नतस्तु योगोऽस्ति} \\
 & \natline{न चैकान्तमनश्नतः |} \\
 & \natline{न चाति स्वप्नशीलस्य} \\
 & \natline{जाग्रतो नैव चार्जुन ||}
\end{tabular}
\end{table}

\begin{table}[H]
\begin{tabular}{cl}
\textbf{6.17} & \natline{युक्ताहारविहारस्य} \\
 & \natline{युक्तचेष्टस्य कर्मसु |} \\
 & \natline{युक्तस्वप्नावबोधस्य} \\
 & \natline{योगो भवति दुःखहा ||}
\end{tabular}
\end{table}

\begin{table}[H]
\begin{tabular}{cl}
\textbf{6.18} & \natline{यदा विनियतं चित्तम्} \\
 & \natline{आत्मन्येवावतिष्ठते |} \\
 & \natline{निस्स्पृहः सर्वकामेभ्यः} \\
 & \natline{युक्त इत्युच्यते तदा ||}
\end{tabular}
\end{table}

\begin{table}[H]
\begin{tabular}{cl}
\textbf{6.19} & \natline{यथा दीपो निवातस्थः} \\
 & \natline{नेङ्गते सोपमा स्मृता |} \\
 & \natline{योगिनो यतचित्तस्य} \\
 & \natline{युञ्जतो योगोमात्मनः ||}
\end{tabular}
\end{table}

\begin{table}[H]
\begin{tabular}{cl}
\textbf{6.20} & \natline{यत्रोपरमते चित्तं} \\
 & \natline{निरुद्धं योगसेवया |} \\
 & \natline{यत्र चैवात्मनाऽऽत्मानं} \\
 & \natline{पस्यन्नात्मनि तुष्यति ||}
\end{tabular}
\end{table}

\begin{table}[H]
\begin{tabular}{cl}
\textbf{6.21} & \natline{सुखमात्यन्तिकं यत्तत्} \\
 & \natline{बुद्धिग्राह्यमतीन्द्रियम् |} \\
 & \natline{वेत्ति यत्र न चैवायं} \\
 & \natline{स्थितश्चलति तत्त्वतः ||}
\end{tabular}
\end{table}

\begin{table}[H]
\begin{tabular}{cl}
\textbf{6.22} & \natline{यं लब्ध्वा चापरं लाभं} \\
 & \natline{मन्यते नाधिकं ततः |} \\
 & \natline{यस्मिन् स्थितो न दुःखेन} \\
 & \natline{गुरुणापि विचाल्यते ||}
\end{tabular}
\end{table}

\begin{table}[H]
\begin{tabular}{cl}
\textbf{6.23} & \natline{तं विद्यात् दुःखसंयोग} \\
 & \natline{वियोगं योगसञ्ज्ञितम् |} \\
 & \natline{स निश्चयेन योक्तव्यः} \\
 & \natline{योगोऽनिर्विण्णचेतसा ||}
\end{tabular}
\end{table}

\begin{table}[H]
\begin{tabular}{cl}
\textbf{6.24} & \natline{सङ्कल्पप्रभवान्कामान्} \\
 & \natline{त्यक्त्वा सर्वानशेषतः |} \\
 & \natline{मनसैवेन्द्रियग्रामं} \\
 & \natline{विनियम्य समन्ततः ||}
\end{tabular}
\end{table}

\begin{table}[H]
\begin{tabular}{cl}
\textbf{6.25} & \natline{शनैः शनैरुपरमेत्} \\
 & \natline{बुद्ध्या धृतिगृहीतया |} \\
 & \natline{आत्मसंस्थं मनः कृत्वा} \\
 & \natline{न किञ्चिदपि चिन्तयेत् ||}
\end{tabular}
\end{table}

\begin{table}[H]
\begin{tabular}{cl}
\textbf{6.26} & \natline{यतो यतो निश्चरति} \\
 & \natline{मनश्चञ्चलमस्थिरम् |} \\
 & \natline{ततस्ततो नियम्यैतत्} \\
 & \natline{आत्मन्येव वशं नयेत् ||}
\end{tabular}
\end{table}

\begin{table}[H]
\begin{tabular}{cl}
\textbf{6.27} & \natline{प्रशान्तमनसं ह्येनं} \\
 & \natline{योगिनं सुखमुत्तमम् |} \\
 & \natline{उपैति शान्तरजसं} \\
 & \natline{ब्रह्मभूतमकल्मषम् ||}
\end{tabular}
\end{table}

\begin{table}[H]
\begin{tabular}{cl}
\textbf{6.28} & \natline{युञ्जन्नेवं सदाऽऽत्मानं} \\
 & \natline{योगी विगतकल्मषः |} \\
 & \natline{सुखेन ब्रह्मसंस्पर्शं} \\
 & \natline{अत्यन्तं सुखमश्नुते ||}
\end{tabular}
\end{table}

\begin{table}[H]
\begin{tabular}{cl}
\textbf{6.29} & \natline{सर्वभूतस्थमात्मानं} \\
 & \natline{सर्वभूतानि चात्मनि |} \\
 & \natline{ईक्षते योगयुक्तात्मा} \\
 & \natline{सर्वत्र समदर्शनः ||}
\end{tabular}
\end{table}

\begin{table}[H]
\begin{tabular}{cl}
\textbf{6.30} & \natline{यो मां पश्यति सर्वत्र} \\
 & \natline{सर्वं च मयि पश्यति |} \\
 & \natline{तस्याहं न प्रणश्यामि} \\
 & \natline{स च मे न प्रणश्यति ||}
\end{tabular}
\end{table}

\begin{table}[H]
\begin{tabular}{cl}
\textbf{6.31} & \natline{सर्वभूतस्थितं यो मां} \\
 & \natline{भजत्येकत्वमास्थितः |} \\
 & \natline{सर्वथा वर्तमानोऽपि} \\
 & \natline{स योगी मयि वर्तते ||}
\end{tabular}
\end{table}

\begin{table}[H]
\begin{tabular}{cl}
\textbf{6.32} & \natline{आत्मौपम्येन सर्वत्र} \\
 & \natline{समं पश्यति योऽर्जुन |} \\
 & \natline{सुखं वा यदि वा दुःखं} \\
 & \natline{स योगी परमो मतः ||}
\end{tabular}
\end{table}

\begin{table}[H]
\begin{tabular}{cl}
\textbf{6.33} & \natline{अर्जुन उवाच} \\
 & \natline{योऽयं योगस्त्वया प्रोक्तः} \\
 & \natline{साम्येन मधुसूदन |} \\
 & \natline{एतस्याहं न पश्यामि} \\
 & \natline{चञ्चलत्वात् स्थितिं स्थिराम् ||}
\end{tabular}
\end{table}

\begin{table}[H]
\begin{tabular}{cl}
\textbf{6.34} & \natline{चञ्चलं हि मनः कृष्ण} \\
 & \natline{प्रमाथि बलवद्दृढम् |} \\
 & \natline{तस्याहं निग्रहं मन्ये} \\
 & \natline{वायोरिव सुदुष्करम् ||}
\end{tabular}
\end{table}

\begin{table}[H]
\begin{tabular}{cl}
\textbf{6.35} & \natline{श्री भगवानुवाच} \\
 & \natline{असंशयं महाबाहो} \\
 & \natline{मनो दुर्निग्रहं चलम् |} \\
 & \natline{अभ्यासेन तु कौन्तेय} \\
 & \natline{वैराग्येण च गृह्यते ||}
\end{tabular}
\end{table}

\begin{table}[H]
\begin{tabular}{cl}
\textbf{6.36} & \natline{असंयतात्मना योगः} \\
 & \natline{दुष्प्राप इति मे मतिः |} \\
 & \natline{वश्यात्मना तु यतता} \\
 & \natline{शक्योऽवाप्तुमुपायतः ||}
\end{tabular}
\end{table}

\begin{table}[H]
\begin{tabular}{cl}
\textbf{6.37} & \natline{अर्जुन उवाच} \\
 & \natline{अयतिः श्रद्धयोपेतः} \\
 & \natline{योगाच्चलितमानसः |} \\
 & \natline{अप्राप्य योगसंसिद्धिं} \\
 & \natline{कां गतिं कृष्ण गच्छति ||}
\end{tabular}
\end{table}

\begin{table}[H]
\begin{tabular}{cl}
\textbf{6.38} & \natline{कच्चिन्नोभयविभ्रष्टः} \\
 & \natline{छिन्नाभ्रमिव नश्यति |} \\
 & \natline{अप्रतिष्ठो महाबाहो} \\
 & \natline{विमूढो ब्रह्मणः पथि ||}
\end{tabular}
\end{table}

\begin{table}[H]
\begin{tabular}{cl}
\textbf{6.39} & \natline{एतन्मे संशयं कृष्ण} \\
 & \natline{छेत्तुमर्हस्यशेषतः |} \\
 & \natline{त्वदन्यः संशयस्यास्य} \\
 & \natline{छेत्ता न ह्युपपद्यते ||}
\end{tabular}
\end{table}

\begin{table}[H]
\begin{tabular}{cl}
\textbf{6.40} & \natline{श्री भगवानुवाच} \\
 & \natline{पार्थ नैवेह नामुत्र} \\
 & \natline{विनाशस्तस्य विद्यते |} \\
 & \natline{न हि कल्याणकृत्कश्चित्} \\
 & \natline{दुर्गतिं तात गच्छति ||}
\end{tabular}
\end{table}

\begin{table}[H]
\begin{tabular}{cl}
\textbf{6.41} & \natline{प्राप्य पुण्यकृतां लोकाण्} \\
 & \natline{उषित्वा शाश्वतीः समाः |} \\
 & \natline{शुचीनां श्रीमतां गेहे} \\
 & \natline{योगभ्रष्टोऽभिजायते ||}
\end{tabular}
\end{table}

\begin{table}[H]
\begin{tabular}{cl}
\textbf{6.42} & \natline{अथवा योगिनामेव} \\
 & \natline{कुले भवति धीमताम् |} \\
 & \natline{एतद्धि दुर्लभतरं} \\
 & \natline{लोके जन्म यदीदृशम् ||}
\end{tabular}
\end{table}

\begin{table}[H]
\begin{tabular}{cl}
\textbf{6.43} & \natline{तत्र तं बुद्धिसंयोगं} \\
 & \natline{लभते पौर्वदेहिकम् |} \\
 & \natline{यतते च ततो भूयः} \\
 & \natline{संसिद्धौ कुरुनन्दन ||}
\end{tabular}
\end{table}

\begin{table}[H]
\begin{tabular}{cl}
\textbf{6.44} & \natline{पुर्वाभ्यासेन तेनैव} \\
 & \natline{ह्रियते ह्यवशोऽपि सः |} \\
 & \natline{जिज्ञासुरपि योगस्य} \\
 & \natline{शब्दब्रह्मातिवर्तते ||}
\end{tabular}
\end{table}

\begin{table}[H]
\begin{tabular}{cl}
\textbf{6.45} & \natline{प्रयत्नाद्यतमानस्तु} \\
 & \natline{योगी संशुद्धकिल्बिषः |} \\
 & \natline{अनेकजन्मसंसिद्धः} \\
 & \natline{ततो याति परां गतिम् ||}
\end{tabular}
\end{table}

\begin{table}[H]
\begin{tabular}{cl}
\textbf{6.46} & \natline{तपस्विभ्योऽधिको योगी} \\
 & \natline{ज्ञानिभ्योऽपि मतोऽधिकः |} \\
 & \natline{कर्मिभ्यश्चाधिको योगी} \\
 & \natline{तस्माद्योगी भवार्जुन ||}
\end{tabular}
\end{table}

\begin{table}[H]
\begin{tabular}{cl}
\textbf{6.47} & \natline{योगिनामपि सर्वेषां} \\
 & \natline{मद्गतेनान्तरात्मना |} \\
 & \natline{श्रद्धावान्भजते यो मां} \\
 & \natline{स मे युक्ततमो मतः ||}
\end{tabular}
\end{table}

\begin{table}[H]
\begin{tabular}{cl}
 & \natline{श्रीमद्भगवद्गीतासु उपनिषत्सु} \\
 & \natline{ब्रह्मविद्यायां योगशास्त्रे} \\
 & \natline{श्रीकृष्णार्जुन संवादे} \\
 & \natline{आत्मसंयमयोगो नाम} \\
 & \natline{षष्ठोध्यायः}
\end{tabular}
\end{table}

