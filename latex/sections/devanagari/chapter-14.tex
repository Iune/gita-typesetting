\begin{table}[H]
\begin{tabular}{cl}
\textbf{14.0} & \natline{ओं श्री परमात्मने नमः} \\
 & \natline{अथ चतुर्दशोऽध्यायः} \\
 & \natline{गुणत्रयविभागयोगः}
\end{tabular}
\end{table}

\begin{table}[H]
\begin{tabular}{cl}
\textbf{14.1} & \natline{श्री भगवानुवाच} \\
 & \natline{परं भूयः प्रवक्ष्यामि} \\
 & \natline{ज्ञानानां ज्ञानमुत्तमम् |} \\
 & \natline{यज्ज्ञात्वा मुनयः सर्वे} \\
 & \natline{परां सिद्धिमितो गताः ||}
\end{tabular}
\end{table}

\begin{table}[H]
\begin{tabular}{cl}
\textbf{14.2} & \natline{इदं ज्ञानमुपाश्रित्य} \\
 & \natline{मम साधर्म्यमागताः |} \\
 & \natline{सर्गेऽपि नोपजायन्ते} \\
 & \natline{प्रलये न व्यथन्ति च ||}
\end{tabular}
\end{table}

\begin{table}[H]
\begin{tabular}{cl}
\textbf{14.3} & \natline{मम योनिर्महद्ब्रह्म} \\
 & \natline{तस्मिन्गर्भं दधाम्यहम् |} \\
 & \natline{सम्भवः सर्वभूतानां} \\
 & \natline{ततो भवति भारत ||}
\end{tabular}
\end{table}

\begin{table}[H]
\begin{tabular}{cl}
\textbf{14.4} & \natline{सर्वयोनिषु कौन्तेय} \\
 & \natline{मूर्तयः सम्भवन्ति याः |} \\
 & \natline{तासां ब्रह्म महद्योनिः} \\
 & \natline{अहं बीजप्रदः पिता ||}
\end{tabular}
\end{table}

\begin{table}[H]
\begin{tabular}{cl}
\textbf{14.5} & \natline{सत्त्वं रजस्तम इति} \\
 & \natline{गुणाः प्रकृतिसम्भवाः |} \\
 & \natline{निबध्नन्ति महाबाहो} \\
 & \natline{देहे देहिनमव्ययम् ||}
\end{tabular}
\end{table}

\begin{table}[H]
\begin{tabular}{cl}
\textbf{14.6} & \natline{तत्र सत्त्वं निर्मलत्वात्} \\
 & \natline{प्रकाशकमनामयम् |} \\
 & \natline{सुखसङ्गेन बध्नाति} \\
 & \natline{ज्ञानसङ्गेन चानघ ||}
\end{tabular}
\end{table}

\begin{table}[H]
\begin{tabular}{cl}
\textbf{14.7} & \natline{रजो रागात्मकं विद्धि} \\
 & \natline{तृष्णासङ्गसमुद्भवम् |} \\
 & \natline{तन्निबध्नाति कौन्तेय} \\
 & \natline{कर्मसङ्गेन देहिनम् ||}
\end{tabular}
\end{table}

\begin{table}[H]
\begin{tabular}{cl}
\textbf{14.8} & \natline{तमस्त्वज्ञानजं विद्धि} \\
 & \natline{मोहनं सर्वदेहिनाम् |} \\
 & \natline{प्रमादालस्यनिद्राभिः} \\
 & \natline{तन्निबध्नाति भारत ||}
\end{tabular}
\end{table}

\begin{table}[H]
\begin{tabular}{cl}
\textbf{14.9} & \natline{सत्त्वं सुखे सञ्जयति} \\
 & \natline{रजः कर्मणि भारत |} \\
 & \natline{ज्ञानमावृत्य तु तमः} \\
 & \natline{प्रमादे सञ्जयत्युत ||}
\end{tabular}
\end{table}

\begin{table}[H]
\begin{tabular}{cl}
\textbf{14.10} & \natline{रजस्तमश्चाभिभूय} \\
 & \natline{सत्त्वं भवति भारत |} \\
 & \natline{रजः सत्त्वं तमश्चैव} \\
 & \natline{तमः सत्त्वं रजस्तथा ||}
\end{tabular}
\end{table}

\begin{table}[H]
\begin{tabular}{cl}
\textbf{14.11} & \natline{सर्वद्वारेषु देहेऽस्मिन्} \\
 & \natline{प्रकाश उपजायते |} \\
 & \natline{ज्ञानं यदा तदा विद्यात्} \\
 & \natline{विवृद्धं सत्त्वमित्युत ||}
\end{tabular}
\end{table}

\begin{table}[H]
\begin{tabular}{cl}
\textbf{14.12} & \natline{लोभः प्रवृत्तिरारम्भः} \\
 & \natline{कर्मणामशमः स्पृहा |} \\
 & \natline{रजस्येतानि जायन्ते} \\
 & \natline{विवृद्धे भरतर्षभ ||}
\end{tabular}
\end{table}

\begin{table}[H]
\begin{tabular}{cl}
\textbf{14.13} & \natline{अप्रकाशोऽप्रवृत्तिश्च} \\
 & \natline{प्रमादो मोह एव च |} \\
 & \natline{तमस्येतानि जायन्ते} \\
 & \natline{विवृद्धे कुरुनन्दन ||}
\end{tabular}
\end{table}

\begin{table}[H]
\begin{tabular}{cl}
\textbf{14.14} & \natline{यदा सत्त्वे प्रवृद्धे तु} \\
 & \natline{प्रलयं याति देहभृत् |} \\
 & \natline{तदोत्तमविदां लोकान्} \\
 & \natline{अमलान्प्रतिपद्यते ||}
\end{tabular}
\end{table}

\begin{table}[H]
\begin{tabular}{cl}
\textbf{14.15} & \natline{रजसि प्रलयं गत्वा} \\
 & \natline{कर्मसङ्गिषु जायते |} \\
 & \natline{तथा प्रलीनस्तमसि} \\
 & \natline{मूढयोनिषु जायते ||}
\end{tabular}
\end{table}

\begin{table}[H]
\begin{tabular}{cl}
\textbf{14.16} & \natline{कर्मणः सुकृतस्याहुः} \\
 & \natline{सात्त्विकं निर्मलं फलम् |} \\
 & \natline{रजसस्तु फलं दुःखम्} \\
 & \natline{अज्ञानं तमसः फलम् ||}
\end{tabular}
\end{table}

\begin{table}[H]
\begin{tabular}{cl}
\textbf{14.17} & \natline{सत्त्वात्सञ्जायते ज्ञानं} \\
 & \natline{रजसो लोभ एव च |} \\
 & \natline{प्रमादमोहौ तमसः} \\
 & \natline{भवतोऽज्ञानमेव च ||}
\end{tabular}
\end{table}

\begin{table}[H]
\begin{tabular}{cl}
\textbf{14.18} & \natline{ऊर्ध्वं गच्छन्ति सत्त्वस्थाः} \\
 & \natline{मध्ये तिष्ठन्ति राजसाः |} \\
 & \natline{जघन्यगुणवृत्तिस्थाः} \\
 & \natline{अधो गच्छन्ति तामसाः ||}
\end{tabular}
\end{table}

\begin{table}[H]
\begin{tabular}{cl}
\textbf{14.19} & \natline{नान्यं गुणेभ्यः कर्तारं} \\
 & \natline{यदा द्रष्टाऽनुपश्यति |} \\
 & \natline{गुणेभ्यश्च परं वेत्ति} \\
 & \natline{मद्भावं सोऽधिगच्छति ||}
\end{tabular}
\end{table}

\begin{table}[H]
\begin{tabular}{cl}
\textbf{14.20} & \natline{गुणानेतानतीत्य त्रीन्} \\
 & \natline{देही देहसमुद्भवान् |} \\
 & \natline{जन्ममृत्युजरादुःखैः} \\
 & \natline{विमुक्तोऽमृतमश्नुते ||}
\end{tabular}
\end{table}

\begin{table}[H]
\begin{tabular}{cl}
\textbf{14.21} & \natline{अर्जुन उवाच} \\
 & \natline{कैर्लिङ्गैस्त्रीन्गुणानेतान्} \\
 & \natline{अतीतो भवति प्रभो |} \\
 & \natline{किमाचारः कथं चैतान्} \\
 & \natline{त्रीन्गुणानतिवर्तते ||}
\end{tabular}
\end{table}

\begin{table}[H]
\begin{tabular}{cl}
\textbf{14.22} & \natline{श्री भगवानुवाच} \\
 & \natline{प्रकाशं च प्रवृत्तिं च} \\
 & \natline{मोहमेव च पाण्डव |} \\
 & \natline{न द्वेष्टि सम्प्रवृत्तानि} \\
 & \natline{न निवृत्तानि काङ्क्षति ||}
\end{tabular}
\end{table}

\begin{table}[H]
\begin{tabular}{cl}
\textbf{14.23} & \natline{उदासीनवदासीनः} \\
 & \natline{गुणैर्यो न विचाल्यते |} \\
 & \natline{गुणा वर्तन्त इत्येव} \\
 & \natline{योऽवतिष्ठति नेङ्गते ||}
\end{tabular}
\end{table}

\begin{table}[H]
\begin{tabular}{cl}
\textbf{14.24} & \natline{समदुःखसुखः स्वस्थः} \\
 & \natline{समलोष्टाश्मकाञ्चनः |} \\
 & \natline{तुल्यप्रियाप्रियो धीरः} \\
 & \natline{तुल्यनिन्दात्मसंस्तुतिः ||}
\end{tabular}
\end{table}

\begin{table}[H]
\begin{tabular}{cl}
\textbf{14.25} & \natline{मानापमानयोस्तुल्यः} \\
 & \natline{तुल्यो मित्रारिपक्षयोः |} \\
 & \natline{सर्वारम्भपरित्यागी} \\
 & \natline{गुणातीतः स उच्यते ||}
\end{tabular}
\end{table}

\begin{table}[H]
\begin{tabular}{cl}
\textbf{14.26} & \natline{मां च योऽव्यभिचारेण} \\
 & \natline{भक्तियोगेन सेवते |} \\
 & \natline{स गुणान्समतीत्यैतान्} \\
 & \natline{ब्रह्मभूयाय कल्पते ||}
\end{tabular}
\end{table}

\begin{table}[H]
\begin{tabular}{cl}
\textbf{14.27} & \natline{ब्रह्मणो हि प्रतिष्ठाऽहम्} \\
 & \natline{अमृतस्याव्ययस्य च |} \\
 & \natline{शाश्वतस्य च धर्मस्य} \\
 & \natline{सुखस्यैकान्तिकस्य च ||}
\end{tabular}
\end{table}

