\begin{table}[H]
\begin{tabular}{cl}
\textbf{16.0} & \natline{ओं श्री परमात्मने नमः} \\
 & \natline{अथ षोदशोऽध्यायः} \\
 & \natline{दैवासुरसम्पद्विभागयोगः}
\end{tabular}
\end{table}

\begin{table}[H]
\begin{tabular}{cl}
\textbf{16.1} & \natline{श्री भगवानुवाच} \\
 & \natline{अभयं सत्त्वसंशुद्धिः} \\
 & \natline{ज्ञानयोगव्यवस्थितिः |} \\
 & \natline{दानं दमश्च यज्ञश्च} \\
 & \natline{स्वाध्यायस्तप आर्जवम् ||}
\end{tabular}
\end{table}

\begin{table}[H]
\begin{tabular}{cl}
\textbf{16.2} & \natline{अहिंसा सत्यमक्रोधः} \\
 & \natline{त्यागः शान्तिरपैशुनम् |} \\
 & \natline{दया भूतेष्वलोलुप्त्वं} \\
 & \natline{मार्दवं ह्रीरचापलम् ||}
\end{tabular}
\end{table}

\begin{table}[H]
\begin{tabular}{cl}
\textbf{16.3} & \natline{तेजः क्षमा धृतिः शौचम्} \\
 & \natline{अद्रोहो नातिमानिता |} \\
 & \natline{भवन्ति सम्पदं दैवीम्} \\
 & \natline{अभिजातस्य भारत ||}
\end{tabular}
\end{table}

\begin{table}[H]
\begin{tabular}{cl}
\textbf{16.4} & \natline{दम्भो दर्पोऽभिमानश्च} \\
 & \natline{क्रोधः पारुष्यमेव च |} \\
 & \natline{अज्ञानं चाभिजातस्य} \\
 & \natline{पार्थ सम्पदमासुरीम् ||}
\end{tabular}
\end{table}

\begin{table}[H]
\begin{tabular}{cl}
\textbf{16.5} & \natline{दैवी सम्पद्विमोक्षाय} \\
 & \natline{निबन्धायासुरी मता |} \\
 & \natline{मा शुचः सम्पदं दैवीम्} \\
 & \natline{अभिजातोऽसि पाण्डव ||}
\end{tabular}
\end{table}

\begin{table}[H]
\begin{tabular}{cl}
\textbf{16.6} & \natline{द्वौ भूतसर्गौ लोकेऽस्मिन्} \\
 & \natline{दैव आसुर एव च |} \\
 & \natline{दैवो विस्तरशः प्रोक्तः} \\
 & \natline{आसुरं पार्थ मे शृणु ||}
\end{tabular}
\end{table}

\begin{table}[H]
\begin{tabular}{cl}
\textbf{16.7} & \natline{प्रवृत्तिं च निवृत्तिं च} \\
 & \natline{जना न विदुरासुराः |} \\
 & \natline{न शौचं नापि चाचारः} \\
 & \natline{न सत्यं तेषु विद्यते ||}
\end{tabular}
\end{table}

\begin{table}[H]
\begin{tabular}{cl}
\textbf{16.8} & \natline{असत्यमप्रतिष्ठं ते} \\
 & \natline{जगदाहुरनीश्वरम् |} \\
 & \natline{अपरस्परसम्भूतं} \\
 & \natline{किमन्यत्कामहैतुकम् ||}
\end{tabular}
\end{table}

\begin{table}[H]
\begin{tabular}{cl}
\textbf{16.9} & \natline{एतां दृष्टिमवष्टभ्य} \\
 & \natline{नष्टात्मानोऽल्पबुद्धयः |} \\
 & \natline{प्रभवन्त्युग्रकर्माणः} \\
 & \natline{क्षयाय जगतोऽहिताः ||}
\end{tabular}
\end{table}

\begin{table}[H]
\begin{tabular}{cl}
\textbf{16.10} & \natline{काममाश्रित्य दुष्पूरं} \\
 & \natline{दम्भमानमदान्विताः |} \\
 & \natline{मोहाद्गृहीत्वासद्ग्राहान्} \\
 & \natline{प्रवर्तन्तेऽशुचिव्रताः ||}
\end{tabular}
\end{table}

\begin{table}[H]
\begin{tabular}{cl}
\textbf{16.11} & \natline{चिन्तामपरिमेयां च} \\
 & \natline{प्रलयान्तामुपाश्रिताः |} \\
 & \natline{कामोपभोगपरमाः} \\
 & \natline{एतावदिति निश्चिताः ||}
\end{tabular}
\end{table}

\begin{table}[H]
\begin{tabular}{cl}
\textbf{16.12} & \natline{आशापाशशतैर्बद्धाः} \\
 & \natline{कामक्रोधपरायणाः |} \\
 & \natline{ईहन्ते कामभोगार्थम्} \\
 & \natline{अन्यायेनार्थसञ्चयान् ||}
\end{tabular}
\end{table}

\begin{table}[H]
\begin{tabular}{cl}
\textbf{16.13} & \natline{इदमद्य मया लब्धम्} \\
 & \natline{इमं प्राप्स्ये मनोरथम् |} \\
 & \natline{इदमस्तीदमपि मे} \\
 & \natline{भविष्यति पुनर्धनम् ||}
\end{tabular}
\end{table}

\begin{table}[H]
\begin{tabular}{cl}
\textbf{16.14} & \natline{असौ मया हतः शतृः} \\
 & \natline{हनिष्ये चापरानपि |} \\
 & \natline{ईश्वरोऽहमहं भोगी} \\
 & \natline{सिद्धोऽहं बलवान्सुखी ||}
\end{tabular}
\end{table}

\begin{table}[H]
\begin{tabular}{cl}
\textbf{16.15} & \natline{आढ्योऽभिजनवानस्मि} \\
 & \natline{कोऽन्योऽस्ति सदृशो मया |} \\
 & \natline{यक्ष्ये दास्यामि मोदिष्ये} \\
 & \natline{इत्यज्ञानविमोहिताः ||}
\end{tabular}
\end{table}

\begin{table}[H]
\begin{tabular}{cl}
\textbf{16.16} & \natline{अनेकचित्तविभ्रान्ताः} \\
 & \natline{मोहजालसमावृताः |} \\
 & \natline{प्रसक्ताः कामभोगेषु} \\
 & \natline{पतन्ति नरकेऽशुचौ ||}
\end{tabular}
\end{table}

\begin{table}[H]
\begin{tabular}{cl}
\textbf{16.17} & \natline{आत्मसम्भाविताः स्तब्धाः} \\
 & \natline{धनमानमदान्विताः |} \\
 & \natline{यजन्ते नामयज्ञैस्ते} \\
 & \natline{दम्भेनाविधिपूर्वकम् ||}
\end{tabular}
\end{table}

\begin{table}[H]
\begin{tabular}{cl}
\textbf{16.18} & \natline{अहङ्कारं बलं दर्पं} \\
 & \natline{कामं क्रोधं च संश्रिताः |} \\
 & \natline{मामात्मपरदेहेषु} \\
 & \natline{प्रद्विषन्तोऽभ्यसूयकाः ||}
\end{tabular}
\end{table}

\begin{table}[H]
\begin{tabular}{cl}
\textbf{16.19} & \natline{तानहं द्विषतः क्रूरान्} \\
 & \natline{संसारेषु नराधमान् |} \\
 & \natline{क्षिपाम्यजस्रमशुभान्} \\
 & \natline{आसुरीष्वेव योनिषु ||}
\end{tabular}
\end{table}

\begin{table}[H]
\begin{tabular}{cl}
\textbf{16.20} & \natline{आसुरीं योनिमापन्नाः} \\
 & \natline{मूढा जन्मनि जन्मनि |} \\
 & \natline{मामप्राप्यैव कौन्तेय} \\
 & \natline{ततो यान्त्यधमां गतिम् ||}
\end{tabular}
\end{table}

\begin{table}[H]
\begin{tabular}{cl}
\textbf{16.21} & \natline{त्रिविधं नरकस्येदं} \\
 & \natline{द्वारं नाशनमात्मनः |} \\
 & \natline{कामः क्रोधस्तथा लोभः} \\
 & \natline{तस्मादेतत्त्रयं त्यजेत् ||}
\end{tabular}
\end{table}

\begin{table}[H]
\begin{tabular}{cl}
\textbf{16.22} & \natline{एतैर्विमुक्तः कौन्तेय} \\
 & \natline{तमोद्वारैस्त्रिभिर्नरः |} \\
 & \natline{आचरत्यात्मनः श्रेयः} \\
 & \natline{ततो याति परां गतिम् ||}
\end{tabular}
\end{table}

\begin{table}[H]
\begin{tabular}{cl}
\textbf{16.23} & \natline{यः शास्त्रविधिमुत्सृज्य} \\
 & \natline{वर्तते कामकारतः |} \\
 & \natline{न स सिद्धिमवाप्नोति} \\
 & \natline{न सुखं न परां गतिम् ||}
\end{tabular}
\end{table}

\begin{table}[H]
\begin{tabular}{cl}
\textbf{16.24} & \natline{तस्माच्छास्त्रं प्रमाणं ते} \\
 & \natline{कार्याकार्यव्यवस्थितौ |} \\
 & \natline{ज्ञात्वा शास्त्रविधानोक्तं} \\
 & \natline{कर्म कर्तुमिहार्हसि ||}
\end{tabular}
\end{table}

