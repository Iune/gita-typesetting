\begin{table}[H]
\begin{tabular}{cl}
 & \natline{श्री परमात्मने नमः} \\
 & \natline{अथ दशमोऽध्यायः} \\
 & \natline{विभुतियोगः}
\end{tabular}
\end{table}

\begin{table}[H]
\begin{tabular}{cl}
\textbf{10.1} & \natline{श्री भगवानुवाच} \\
 & \natline{भूय एव महाबाहो} \\
 & \natline{शृणु मे परमं वचः |} \\
 & \natline{यत्तेऽहं प्रीयमाणाय} \\
 & \natline{वक्ष्यामि हितकाम्यया ||}
\end{tabular}
\end{table}

\begin{table}[H]
\begin{tabular}{cl}
\textbf{10.2} & \natline{न मे विदुः सुरगणाः} \\
 & \natline{प्रभवं न महर्षयः |} \\
 & \natline{अहमादिर्हि देवानां} \\
 & \natline{महर्षीणां च सर्वशः ||}
\end{tabular}
\end{table}

\begin{table}[H]
\begin{tabular}{cl}
\textbf{10.3} & \natline{यो मामजमनादिं च} \\
 & \natline{वेत्ति लोकमहेश्वरम् |} \\
 & \natline{असम्मूढः स मर्त्येषु} \\
 & \natline{सर्वपापैः प्रमुच्यते ||}
\end{tabular}
\end{table}

\begin{table}[H]
\begin{tabular}{cl}
\textbf{10.4} & \natline{बुद्धिर्ज्ञानमसम्मोहः} \\
 & \natline{क्षमा सत्यं दमः शमः |} \\
 & \natline{सुखं दुःखं भवोऽभावः} \\
 & \natline{भयं चाभयमेव च ||}
\end{tabular}
\end{table}

\begin{table}[H]
\begin{tabular}{cl}
\textbf{10.5} & \natline{अहिंसा समता तुष्टिः} \\
 & \natline{तपो दानं यशोऽयशः |} \\
 & \natline{भवन्ति भावा भूतानां} \\
 & \natline{मत्त एव पृथग्विधाः ||}
\end{tabular}
\end{table}

\begin{table}[H]
\begin{tabular}{cl}
\textbf{10.6} & \natline{महर्षयः सप्त पूर्वे} \\
 & \natline{चत्वारो मनवस्तथा |} \\
 & \natline{मद्भावा मानसा जाताः} \\
 & \natline{येषां लोक इमाः प्रजाः ||}
\end{tabular}
\end{table}

\begin{table}[H]
\begin{tabular}{cl}
\textbf{10.7} & \natline{एतां विभूतिं योगं च} \\
 & \natline{मम यो वेत्ति तत्त्वतः |} \\
 & \natline{सोऽविकम्पेन योगेन} \\
 & \natline{युज्यते नात्र संशयः ||}
\end{tabular}
\end{table}

\begin{table}[H]
\begin{tabular}{cl}
\textbf{10.8} & \natline{अहं सर्वस्य प्रभवः} \\
 & \natline{मत्तः सर्वं प्रवर्तते |} \\
 & \natline{इति मत्वा भजन्ते मां} \\
 & \natline{बुधा भावसमन्विताः ||}
\end{tabular}
\end{table}

\begin{table}[H]
\begin{tabular}{cl}
\textbf{10.9} & \natline{मच्चित्ता मद्गतप्राणाः} \\
 & \natline{बोधयन्तः परस्परम् |} \\
 & \natline{कथयन्तश्च मां नित्यं} \\
 & \natline{तुष्यन्ति च रमन्ति च ||}
\end{tabular}
\end{table}

\begin{table}[H]
\begin{tabular}{cl}
\textbf{10.10} & \natline{तेषां सततयुक्तानां} \\
 & \natline{भजतां प्रीतिपूर्वकम् |} \\
 & \natline{ददामि बुद्धियोगं तं} \\
 & \natline{येन मामुपयान्ति ते ||}
\end{tabular}
\end{table}

\begin{table}[H]
\begin{tabular}{cl}
\textbf{10.11} & \natline{तेषामेवानुकम्पार्थम्} \\
 & \natline{अहमज्ञानजं तमः |} \\
 & \natline{नाशयाम्यात्मभावस्थः} \\
 & \natline{ज्ञानदीपेन भास्वता ||}
\end{tabular}
\end{table}

\begin{table}[H]
\begin{tabular}{cl}
\textbf{10.12} & \natline{अर्जुन उवाच} \\
 & \natline{परं ब्रह्म परं धाम} \\
 & \natline{पवित्रं परमं भवान् |} \\
 & \natline{पुरुषं शाश्वतं दिव्यम्} \\
 & \natline{आदिदेवमजं विभुम् ||}
\end{tabular}
\end{table}

\begin{table}[H]
\begin{tabular}{cl}
\textbf{10.13} & \natline{आहुस्त्वामृषयः सर्वे} \\
 & \natline{देवर्षिर्नारदस्तथा |} \\
 & \natline{असितो देवलो व्यासः} \\
 & \natline{स्वयं चैव ब्रवीषि मे ||}
\end{tabular}
\end{table}

\begin{table}[H]
\begin{tabular}{cl}
\textbf{10.14} & \natline{सर्वमेतदृतं मन्ये} \\
 & \natline{यन्मां वदसि केशव |} \\
 & \natline{न हि ते भगवन्व्यक्तिं} \\
 & \natline{विदुर्देवा न दानवाः ||}
\end{tabular}
\end{table}

\begin{table}[H]
\begin{tabular}{cl}
\textbf{10.15} & \natline{स्वयमेवात्मनाऽऽत्मानं} \\
 & \natline{वेत्थ त्वं पुरुषोत्तम |} \\
 & \natline{भूतभावन भूतेश} \\
 & \natline{देवदेव जगत्पते ||}
\end{tabular}
\end{table}

\begin{table}[H]
\begin{tabular}{cl}
\textbf{10.16} & \natline{वक्तुमर्हस्यशेषेण} \\
 & \natline{दिव्या ह्यात्मविभूतयः |} \\
 & \natline{याभिर्विभूतिभिर्लोकान्} \\
 & \natline{इमांस्त्वं व्याप्य तिष्ठसि ||}
\end{tabular}
\end{table}

\begin{table}[H]
\begin{tabular}{cl}
\textbf{10.17} & \natline{कथं विद्यामहं योगिन्} \\
 & \natline{त्वां सदा परिचिन्तयन् |} \\
 & \natline{केषु केषु च भावेषु} \\
 & \natline{चिन्त्योऽसि भगवन्मया ||}
\end{tabular}
\end{table}

\begin{table}[H]
\begin{tabular}{cl}
\textbf{10.18} & \natline{विस्तरेणात्मनो योगं} \\
 & \natline{विभूतिं च जनार्दन |} \\
 & \natline{भूयः कथय तृप्तिर्हि} \\
 & \natline{शृण्वतो नास्ति मेऽमृतम् ||}
\end{tabular}
\end{table}

\begin{table}[H]
\begin{tabular}{cl}
\textbf{10.19} & \natline{श्री भगवानुवाच} \\
 & \natline{हन्त ते कथयिष्यामि} \\
 & \natline{दिव्या ह्यात्मविभूतयः |} \\
 & \natline{प्राधान्यतः कुरुश्रेष्ठ} \\
 & \natline{नास्त्यन्तो विस्तरस्य मे ||}
\end{tabular}
\end{table}

\begin{table}[H]
\begin{tabular}{cl}
\textbf{10.20} & \natline{अहमात्मा गुडाकेश} \\
 & \natline{सर्वभूताशयस्थितः |} \\
 & \natline{अहमादिश्च मध्यं च} \\
 & \natline{भूतानामन्त एव च ||}
\end{tabular}
\end{table}

\begin{table}[H]
\begin{tabular}{cl}
\textbf{10.21} & \natline{आदित्यानामहम् विष्णुः} \\
 & \natline{ज्योतिषां रविरंशुमान् |} \\
 & \natline{मरीचिर्मरुतामस्मि} \\
 & \natline{नक्षत्राणामहं शशी ||}
\end{tabular}
\end{table}

\begin{table}[H]
\begin{tabular}{cl}
\textbf{10.22} & \natline{वेदानां सामवेदोऽस्मि} \\
 & \natline{देवानामस्मि वासवः |} \\
 & \natline{इन्द्रियाणां मनश्चास्मि} \\
 & \natline{भूतानामस्मि चेतना ||}
\end{tabular}
\end{table}

\begin{table}[H]
\begin{tabular}{cl}
\textbf{10.23} & \natline{रुद्राणां शङ्करश्चास्मि} \\
 & \natline{वित्तेशो यक्षरक्षसाम् |} \\
 & \natline{वसूनां पावकश्चास्मि} \\
 & \natline{मेरुः शिखरिणामहम् ||}
\end{tabular}
\end{table}

\begin{table}[H]
\begin{tabular}{cl}
\textbf{10.24} & \natline{पुरोधसां च मुख्यं मां} \\
 & \natline{विद्धि पार्थ बृहस्पतिम् |} \\
 & \natline{सेनानीनामहं स्कन्दः} \\
 & \natline{सरसामस्मि सागरः ||}
\end{tabular}
\end{table}

\begin{table}[H]
\begin{tabular}{cl}
\textbf{10.25} & \natline{महर्षीणां भृगुरहं} \\
 & \natline{गिरामस्म्येकमक्षरम् |} \\
 & \natline{यज्ञानां जपयज्ञोऽस्मि} \\
 & \natline{स्थावराणां हिमालयः ||}
\end{tabular}
\end{table}

\begin{table}[H]
\begin{tabular}{cl}
\textbf{10.26} & \natline{अश्वत्थः सर्ववृक्षाणां} \\
 & \natline{देवर्षीणां च नारदः |} \\
 & \natline{गन्धर्वाणां चित्ररथः} \\
 & \natline{सिद्धानां कपिलो मुनिः ||}
\end{tabular}
\end{table}

\begin{table}[H]
\begin{tabular}{cl}
\textbf{10.27} & \natline{उच्चैः श्रवसमश्वानां} \\
 & \natline{विद्धि माममृतोद्भवम् |} \\
 & \natline{ऐरावतं गजेन्द्राणां} \\
 & \natline{नराणां च नराधिपम् ||}
\end{tabular}
\end{table}

\begin{table}[H]
\begin{tabular}{cl}
\textbf{10.28} & \natline{आयुधानामहं वज्रं} \\
 & \natline{धेनूनामस्मि कामधुक् |} \\
 & \natline{प्रजनश्चास्मि कन्दर्पः} \\
 & \natline{सर्पाणामस्मि वासुकिः ||}
\end{tabular}
\end{table}

\begin{table}[H]
\begin{tabular}{cl}
\textbf{10.29} & \natline{अनन्तश्चास्मि नागानां} \\
 & \natline{वरुणो यादसामहम् |} \\
 & \natline{पितॄणामर्यमा चास्मि} \\
 & \natline{यमः संयमतामहम् ||}
\end{tabular}
\end{table}

\begin{table}[H]
\begin{tabular}{cl}
\textbf{10.30} & \natline{प्रह्लादश्चास्मि दैत्यानां} \\
 & \natline{कालः कलयतामहम् |} \\
 & \natline{मृगाणां च मृगेन्द्रोऽहं} \\
 & \natline{वैनतेयश्च पक्षिणाम् ||}
\end{tabular}
\end{table}

\begin{table}[H]
\begin{tabular}{cl}
\textbf{10.31} & \natline{पवनः पवतामस्मि} \\
 & \natline{रामः शस्त्रभृतामहम् |} \\
 & \natline{झषाणां मकरश्चास्मि} \\
 & \natline{स्रोतसामस्मि जाह्नवी ||}
\end{tabular}
\end{table}

\begin{table}[H]
\begin{tabular}{cl}
\textbf{10.32} & \natline{सर्गाणामादिरन्तश्च} \\
 & \natline{मध्यं चैवाहमर्जुन |} \\
 & \natline{अध्यात्मविद्या विद्यानां} \\
 & \natline{वादः प्रवदतामहम् ||}
\end{tabular}
\end{table}

\begin{table}[H]
\begin{tabular}{cl}
\textbf{10.33} & \natline{अक्षराणामकारोऽस्मि} \\
 & \natline{द्वन्द्वः सामासिकस्य च |} \\
 & \natline{अहमेवाक्षयः कालः} \\
 & \natline{धाताऽहं विश्वतोमुखः ||}
\end{tabular}
\end{table}

\begin{table}[H]
\begin{tabular}{cl}
\textbf{10.34} & \natline{मृत्युः सर्वहरश्चाहम्} \\
 & \natline{उद्भवश्च भविष्यताम् |} \\
 & \natline{कीर्तिः श्रीर्वाक्च नारीणां} \\
 & \natline{स्मृतिर्मेधा धृतिः क्षमा ||}
\end{tabular}
\end{table}

\begin{table}[H]
\begin{tabular}{cl}
\textbf{10.35} & \natline{बृहत्साम तथा साम्नां} \\
 & \natline{गायत्री छन्दसामहम् |} \\
 & \natline{मासानां मार्गशीर्षोऽहम्} \\
 & \natline{ऋतूनां कुसुमाकरः ||}
\end{tabular}
\end{table}

\begin{table}[H]
\begin{tabular}{cl}
\textbf{10.36} & \natline{द्यूतं छलयतामस्मि} \\
 & \natline{तेजस्तेजस्विनामहम् |} \\
 & \natline{जयोऽस्मि व्यवसायोऽस्मि} \\
 & \natline{सत्त्वं सत्त्ववतामहम् ||}
\end{tabular}
\end{table}

\begin{table}[H]
\begin{tabular}{cl}
\textbf{10.37} & \natline{वृष्णीनां वासुदेवोऽस्मि} \\
 & \natline{पाण्डवानां धनञ्जयः |} \\
 & \natline{मुनीनामप्यहं व्यासः} \\
 & \natline{कवीनामुशना कविः ||}
\end{tabular}
\end{table}

\begin{table}[H]
\begin{tabular}{cl}
\textbf{10.38} & \natline{दण्डो दमयतामस्मि} \\
 & \natline{नीतिरस्मि जिगीषताम् |} \\
 & \natline{मौनं चैवास्मि गुह्यानां} \\
 & \natline{ज्ञानं ज्ञानवतामहम् ||}
\end{tabular}
\end{table}

\begin{table}[H]
\begin{tabular}{cl}
\textbf{10.39} & \natline{यच्चापि सर्वभूतानां} \\
 & \natline{बीजं तदहमर्जुन |} \\
 & \natline{न तदस्ति विना यत्स्यात्} \\
 & \natline{मया भूतं चराचरम् ||}
\end{tabular}
\end{table}

\begin{table}[H]
\begin{tabular}{cl}
\textbf{10.40} & \natline{नान्तोऽस्ति मम दिव्यानां} \\
 & \natline{विभूतीनां परन्तप |} \\
 & \natline{एष तूद्देशतः प्रोक्तः} \\
 & \natline{विभूतेर्विस्तरो मया ||}
\end{tabular}
\end{table}

\begin{table}[H]
\begin{tabular}{cl}
\textbf{10.41} & \natline{यद्यद्विभूतिमत्सत्त्वं} \\
 & \natline{श्रीमदूर्जितमेव वा |} \\
 & \natline{तत्तदेवावगच्छ त्वं} \\
 & \natline{मम तेजोऽम्शसम्भवम् ||}
\end{tabular}
\end{table}

\begin{table}[H]
\begin{tabular}{cl}
\textbf{10.42} & \natline{अथवा बहुनैतेन} \\
 & \natline{किं ज्ञातेन तवार्जुन |} \\
 & \natline{विष्टभ्याहमिदं कृत्स्नम्} \\
 & \natline{एकांशेन स्थितो जगत् ||}
\end{tabular}
\end{table}

\begin{table}[H]
\begin{tabular}{cl}
 & \natline{श्रीमद्भगवद्गीतासु उपनिषत्सु} \\
 & \natline{ब्रह्मविद्यायां योगशास्त्रे} \\
 & \natline{श्रीकृष्णार्जुन संवादे} \\
 & \natline{विभुतियोगोनाम} \\
 & \natline{दशमोध्यायः}
\end{tabular}
\end{table}

