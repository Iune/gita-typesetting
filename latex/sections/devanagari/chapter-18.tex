\begin{table}[H]
\begin{tabular}{cl}
 & \natline{श्री परमात्मने नमः} \\
 & \natline{अथ अष्टादशोऽध्यायः} \\
 & \natline{मोक्षसन्न्यासयोगः}
\end{tabular}
\end{table}

\begin{table}[H]
\begin{tabular}{cl}
\textbf{18.1} & \natline{अर्जुन उवाच} \\
 & \natline{सन्न्यासस्य महाबाहो} \\
 & \natline{तत्त्वमिच्छामि वेदितुम् |} \\
 & \natline{त्यागस्य च हृषिकेश} \\
 & \natline{पृथक्केशिनिषूदन ||}
\end{tabular}
\end{table}

\begin{table}[H]
\begin{tabular}{cl}
\textbf{18.2} & \natline{श्री भगवनुवाच} \\
 & \natline{काम्यानां कर्मणां न्यासं} \\
 & \natline{सन्न्यासं कवयो विदुः |} \\
 & \natline{सर्वकर्मफलत्यागं} \\
 & \natline{प्राहुस्त्यागं विचक्षणाः ||}
\end{tabular}
\end{table}

\begin{table}[H]
\begin{tabular}{cl}
\textbf{18.3} & \natline{त्याज्यं दोषवदित्येके} \\
 & \natline{कर्म प्राहुर्मनीषिणः |} \\
 & \natline{यज्ञदानतपः कर्म} \\
 & \natline{न त्याज्यमिति चापरे ||}
\end{tabular}
\end{table}

\begin{table}[H]
\begin{tabular}{cl}
\textbf{18.4} & \natline{निश्चयं शृणु मे तत्र} \\
 & \natline{त्यागे भरतसत्तम |} \\
 & \natline{त्यागो हि पुरुषव्याघ्र} \\
 & \natline{त्रिविधः सम्प्रकीर्तितः ||}
\end{tabular}
\end{table}

\begin{table}[H]
\begin{tabular}{cl}
\textbf{18.5} & \natline{यज्ञदानतपःकर्म} \\
 & \natline{न त्याज्यं कार्यमेव तत् |} \\
 & \natline{यज्ञो दानं तपश्चैव} \\
 & \natline{पावनानि मनीषिणाम् ||}
\end{tabular}
\end{table}

\begin{table}[H]
\begin{tabular}{cl}
\textbf{18.6} & \natline{एतान्यपि तु कर्माणि} \\
 & \natline{सङ्गं त्यक्त्वा फलानि च |} \\
 & \natline{कर्तव्यानीति मे पार्थ} \\
 & \natline{निश्चितं मतमुत्तमम् ||}
\end{tabular}
\end{table}

\begin{table}[H]
\begin{tabular}{cl}
\textbf{18.7} & \natline{नियतस्य तु सन्न्यासः} \\
 & \natline{कर्मणो नोपपद्यते |} \\
 & \natline{मोहात्तस्य परित्यागः} \\
 & \natline{तामसः परिकीर्तितः ||}
\end{tabular}
\end{table}

\begin{table}[H]
\begin{tabular}{cl}
\textbf{18.8} & \natline{दुःखमित्येव यत्कर्म} \\
 & \natline{कायक्लेशभयात्त्यजेत् |} \\
 & \natline{स कृत्वा राजसं त्यागं} \\
 & \natline{नैव त्यागफलं लभेत् ||}
\end{tabular}
\end{table}

\begin{table}[H]
\begin{tabular}{cl}
\textbf{18.9} & \natline{कार्यमित्येव यत्कर्म} \\
 & \natline{नियतं क्रियतेऽर्जुन |} \\
 & \natline{सङ्गं त्यक्त्वा फलं चैव} \\
 & \natline{स त्यागः सात्त्विको मतः ||}
\end{tabular}
\end{table}

\begin{table}[H]
\begin{tabular}{cl}
\textbf{18.10} & \natline{न द्वेष्ट्यकुशलं कर्म} \\
 & \natline{कुशले नानुषज्जते |} \\
 & \natline{त्यागी सत्त्वसमाविष्टः} \\
 & \natline{मेधावी छिन्नसंशयः ||}
\end{tabular}
\end{table}

\begin{table}[H]
\begin{tabular}{cl}
\textbf{18.11} & \natline{न हि देहभृता शक्यं} \\
 & \natline{त्यक्तुं कर्माण्यशेषतः |} \\
 & \natline{यस्तु कर्मफलत्यागी} \\
 & \natline{स त्यागीत्यभिधीयते ||}
\end{tabular}
\end{table}

\begin{table}[H]
\begin{tabular}{cl}
\textbf{18.12} & \natline{अनिष्टमिष्टं मिश्रं च} \\
 & \natline{त्रिविधं कर्मणः फलम् |} \\
 & \natline{भवत्यत्यागिनां प्रेत्य} \\
 & \natline{न तु सन्न्यासिनां क्वचित् ||}
\end{tabular}
\end{table}

\begin{table}[H]
\begin{tabular}{cl}
\textbf{18.13} & \natline{पञ्चैतानि महाबाहो} \\
 & \natline{कारणानि निबोध मे |} \\
 & \natline{साङ्ख्ये कृतान्ते प्रोक्तानि} \\
 & \natline{सिद्धये सर्वकर्मणाम् ||}
\end{tabular}
\end{table}

\begin{table}[H]
\begin{tabular}{cl}
\textbf{18.14} & \natline{अधिष्ठानं तथा कर्ता} \\
 & \natline{करणं च पृथग्विधम् |} \\
 & \natline{विविधाश्च पृथक्चेष्टाः} \\
 & \natline{दैवं चैवात्र पञ्चमम् ||}
\end{tabular}
\end{table}

\begin{table}[H]
\begin{tabular}{cl}
\textbf{18.15} & \natline{शरीरवाङ्मनोभिर्यत्} \\
 & \natline{कर्म प्रारभते नरः |} \\
 & \natline{न्याय्यं वा विपरीतं वा} \\
 & \natline{पञ्चैते तस्य हेतवः ||}
\end{tabular}
\end{table}

\begin{table}[H]
\begin{tabular}{cl}
\textbf{18.16} & \natline{तत्रैवं सति कर्तारम्} \\
 & \natline{आत्मानं केवलं तु यः |} \\
 & \natline{पश्यत्यकृतबुद्धित्वात्} \\
 & \natline{न स पश्यति दुर्मतिः ||}
\end{tabular}
\end{table}

\begin{table}[H]
\begin{tabular}{cl}
\textbf{18.17} & \natline{यस्य नाहङ्कृतो भावः} \\
 & \natline{बुद्धिर्यस्य न लिप्यते |} \\
 & \natline{हत्वाऽपि स इमाल्लोकान्} \\
 & \natline{न हन्ति न निबध्यते ||}
\end{tabular}
\end{table}

\begin{table}[H]
\begin{tabular}{cl}
\textbf{18.18} & \natline{ज्ञानं ज्ञेयं परिज्ञाता} \\
 & \natline{त्रिविधा कर्मचोदना |} \\
 & \natline{करणं कर्म कर्तेति} \\
 & \natline{त्रिविधः कर्मसङ्ग्रहः ||}
\end{tabular}
\end{table}

\begin{table}[H]
\begin{tabular}{cl}
\textbf{18.19} & \natline{ज्ञानं कर्म च कर्ता च} \\
 & \natline{त्रिधैव गुणभेदतः |} \\
 & \natline{प्रोच्यते गुणसङ्ख्याने} \\
 & \natline{यथावच्छृणु तान्यपि ||}
\end{tabular}
\end{table}

\begin{table}[H]
\begin{tabular}{cl}
\textbf{18.20} & \natline{सर्वभूतेषु येनैकं} \\
 & \natline{भावमव्ययमीक्षते |} \\
 & \natline{अविभक्तं विभक्तेषु} \\
 & \natline{तज्ज्ञानं विद्धि सात्त्विकम् ||}
\end{tabular}
\end{table}

\begin{table}[H]
\begin{tabular}{cl}
\textbf{18.21} & \natline{पृथक्त्वेन तु यज्ज्ञानं} \\
 & \natline{नानाभावान् पृथग्विधान् |} \\
 & \natline{वेत्ति सर्वेषु भूतेषु} \\
 & \natline{तज्ज्ञानं विद्धि राजसम् ||}
\end{tabular}
\end{table}

\begin{table}[H]
\begin{tabular}{cl}
\textbf{18.22} & \natline{यत्तु कृत्स्नवदेकस्मिन्} \\
 & \natline{कार्ये सक्तमहैतुकम् |} \\
 & \natline{अतत्त्वार्थवदल्पं च} \\
 & \natline{तत्तामसमुदाहृतम् ||}
\end{tabular}
\end{table}

\begin{table}[H]
\begin{tabular}{cl}
\textbf{18.23} & \natline{नियतं सङ्गरहितम्} \\
 & \natline{अरागद्वेषतः कृतम् |} \\
 & \natline{अफलप्रेप्सुना कर्म} \\
 & \natline{यत्तत्सात्त्विकमुच्यते ||}
\end{tabular}
\end{table}

\begin{table}[H]
\begin{tabular}{cl}
\textbf{18.24} & \natline{यत्तु कामेप्सुना कर्म} \\
 & \natline{साहङ्कारेण वा पुनः |} \\
 & \natline{क्रियते बहुलायासं} \\
 & \natline{तद्राजसमुदाहृतम् ||}
\end{tabular}
\end{table}

\begin{table}[H]
\begin{tabular}{cl}
\textbf{18.25} & \natline{अनुबन्धं क्षयं हिंसाम्} \\
 & \natline{अनपेक्ष्य च पौरुषम् |} \\
 & \natline{मोहादारभ्यते कर्म} \\
 & \natline{यत्तत्तामसमुच्यते ||}
\end{tabular}
\end{table}

\begin{table}[H]
\begin{tabular}{cl}
\textbf{18.26} & \natline{मुक्तसङ्गोऽनहंवादी} \\
 & \natline{धृत्युत्साहसमन्वितः |} \\
 & \natline{सिद्ध्यसिद्ध्योर्निर्विकारः} \\
 & \natline{कर्ता सात्त्विक उच्यते ||}
\end{tabular}
\end{table}

\begin{table}[H]
\begin{tabular}{cl}
\textbf{18.27} & \natline{रागी कर्मफलप्रेप्सुः} \\
 & \natline{लुब्धो हिंसात्मकोऽशुचिः |} \\
 & \natline{हर्षशोकान्वितः कर्ता} \\
 & \natline{राजसः परिकीर्तितः ||}
\end{tabular}
\end{table}

\begin{table}[H]
\begin{tabular}{cl}
\textbf{18.28} & \natline{अयुक्तः प्राकृतः स्तब्धः} \\
 & \natline{शठो नैष्कृतिकोऽलसः |} \\
 & \natline{विषादी दीर्घसूत्री च} \\
 & \natline{कर्ता तामस उच्यते ||}
\end{tabular}
\end{table}

\begin{table}[H]
\begin{tabular}{cl}
\textbf{18.29} & \natline{बुद्धेर्भेदं धृतेश्चैव} \\
 & \natline{गुणतस्त्रिविधं शृणु |} \\
 & \natline{प्रोच्यमानमशेषेण} \\
 & \natline{पृथक्त्वेन धनञ्जय ||}
\end{tabular}
\end{table}

\begin{table}[H]
\begin{tabular}{cl}
\textbf{18.30} & \natline{प्रवृत्तिं च निवृत्तिं च} \\
 & \natline{कार्याकार्ये भयाभये |} \\
 & \natline{बन्धं मोक्षं च या वेत्ति} \\
 & \natline{बुद्धिः सा पार्थ सात्त्विकी ||}
\end{tabular}
\end{table}

\begin{table}[H]
\begin{tabular}{cl}
\textbf{18.31} & \natline{यया धर्ममधर्मं च} \\
 & \natline{कार्यं चाकार्यमेव च |} \\
 & \natline{अयथावत्प्रजानाति} \\
 & \natline{बुद्धिः सा पार्थ राजसी ||}
\end{tabular}
\end{table}

\begin{table}[H]
\begin{tabular}{cl}
\textbf{18.32} & \natline{अधर्मं धर्ममिति या} \\
 & \natline{मन्यते तमसाऽऽवृता |} \\
 & \natline{सर्वार्थान्विपरीतांश्च} \\
 & \natline{बुद्धिः सा पार्थ तामसी ||}
\end{tabular}
\end{table}

\begin{table}[H]
\begin{tabular}{cl}
\textbf{18.33} & \natline{धृत्या यया धारयते} \\
 & \natline{मनः प्राणेन्द्रियक्रियाः |} \\
 & \natline{योगेनाव्यभिचारिण्या} \\
 & \natline{धृतिः सा पार्थ सात्त्विकी ||}
\end{tabular}
\end{table}

\begin{table}[H]
\begin{tabular}{cl}
\textbf{18.34} & \natline{यया तु धर्मकामार्थान्} \\
 & \natline{धृत्या धारयतेऽर्जुन |} \\
 & \natline{प्रसङ्गेन फलाकाङ्क्षी} \\
 & \natline{धृतिः सा पार्थ राजसी ||}
\end{tabular}
\end{table}

\begin{table}[H]
\begin{tabular}{cl}
\textbf{18.35} & \natline{यया स्वप्नं भयं शोकं} \\
 & \natline{विषादं मदमेव च |} \\
 & \natline{न विमुञ्चति दुर्मेधाः} \\
 & \natline{धृतिः सा तामसी मता ||}
\end{tabular}
\end{table}

\begin{table}[H]
\begin{tabular}{cl}
\textbf{18.36} & \natline{सुखं त्विदानीं त्रिविधं} \\
 & \natline{शृणु मे भरतर्षभ |} \\
 & \natline{अभ्यासाद्रमते यत्र} \\
 & \natline{दुःखान्तं च निगच्छति ||}
\end{tabular}
\end{table}

\begin{table}[H]
\begin{tabular}{cl}
\textbf{18.37} & \natline{यत्तदग्रे विषमिव} \\
 & \natline{परिणामेऽमृतोपमम् |} \\
 & \natline{तत्सुखं सात्त्विकं प्रोक्तम्} \\
 & \natline{आत्मबुद्धिप्रसादजम् ||}
\end{tabular}
\end{table}

\begin{table}[H]
\begin{tabular}{cl}
\textbf{18.38} & \natline{विषयेन्द्रियसंयोगात्} \\
 & \natline{यत्तदग्रेऽमृतोपमम् |} \\
 & \natline{परिणामे विषमिव} \\
 & \natline{तत्सुखं राजसं स्मृतम् ||}
\end{tabular}
\end{table}

\begin{table}[H]
\begin{tabular}{cl}
\textbf{18.39} & \natline{यदग्रे चानुबन्धे च} \\
 & \natline{सुखं मोहनमात्मनः |} \\
 & \natline{निद्रालस्यप्रमादोत्थं} \\
 & \natline{तत्तामसमुदाहृतम् ||}
\end{tabular}
\end{table}

\begin{table}[H]
\begin{tabular}{cl}
\textbf{18.40} & \natline{न तदस्ति पृथिव्यां वा} \\
 & \natline{दिवि देवेषु वा पुनः |} \\
 & \natline{सत्त्वं प्रकृतिजैर्मुक्तं} \\
 & \natline{यदेभिः स्यात्त्रिभिर्गुणैः ||}
\end{tabular}
\end{table}

\begin{table}[H]
\begin{tabular}{cl}
\textbf{18.41} & \natline{ब्राह्मणक्षत्रियविशां} \\
 & \natline{शूद्राणां च परन्तप |} \\
 & \natline{कर्माणि प्रविभक्तानि} \\
 & \natline{स्वभावप्रभवैर्गुणैः ||}
\end{tabular}
\end{table}

\begin{table}[H]
\begin{tabular}{cl}
\textbf{18.42} & \natline{शमो दमस्तपः शौचं} \\
 & \natline{क्षान्तिरार्जवमेव च |} \\
 & \natline{ज्ञानं विज्ञानमास्तिक्यं} \\
 & \natline{ब्रह्मकर्म स्वभावजम् ||}
\end{tabular}
\end{table}

\begin{table}[H]
\begin{tabular}{cl}
\textbf{18.43} & \natline{शौर्यं तेजो धृतिर्दाक्ष्यं} \\
 & \natline{युद्धे चाप्यपलायनम् |} \\
 & \natline{दानमीश्वरभावश्च} \\
 & \natline{क्षात्रं कर्म स्वभावजम् ||}
\end{tabular}
\end{table}

\begin{table}[H]
\begin{tabular}{cl}
\textbf{18.44} & \natline{कृषिगौरक्ष्यवाणिज्यं} \\
 & \natline{वैश्यकर्म स्वभावजम् |} \\
 & \natline{परिचर्यात्मकं कर्म} \\
 & \natline{शूद्रस्यापि स्वभावजम् ||}
\end{tabular}
\end{table}

\begin{table}[H]
\begin{tabular}{cl}
\textbf{18.45} & \natline{स्वे स्वे कर्मण्यभिरतः} \\
 & \natline{संसिद्धिं लभते नरः |} \\
 & \natline{स्वकर्मनिरतः सिद्धिं} \\
 & \natline{यथा विन्दति तच्छृणु ||}
\end{tabular}
\end{table}

\begin{table}[H]
\begin{tabular}{cl}
\textbf{18.46} & \natline{यतः प्रवृत्तिर्भूतानां} \\
 & \natline{येन सर्वमिदं ततम् |} \\
 & \natline{स्वकर्मणा तमभ्यर्च्य} \\
 & \natline{सिद्धिं विन्दति मानवः ||}
\end{tabular}
\end{table}

\begin{table}[H]
\begin{tabular}{cl}
\textbf{18.47} & \natline{श्रेयान्स्वधर्मो विगुणः} \\
 & \natline{परधर्मात्स्वनुष्ठितात् |} \\
 & \natline{स्वभावनियतं कर्म} \\
 & \natline{कुर्वन्नाप्नोति किल्बिषम् ||}
\end{tabular}
\end{table}

\begin{table}[H]
\begin{tabular}{cl}
\textbf{18.48} & \natline{सहजं कर्म कौन्तेय} \\
 & \natline{सदोषमपि न त्यजेत् |} \\
 & \natline{सर्वारम्भा हि दोषेण} \\
 & \natline{धूमेनाग्निरिवावृताः ||}
\end{tabular}
\end{table}

\begin{table}[H]
\begin{tabular}{cl}
\textbf{18.49} & \natline{असक्तबुद्धिः सर्वत्र} \\
 & \natline{जितात्मा विगतस्पृहः |} \\
 & \natline{नैष्कर्म्यसिद्धिं परमां} \\
 & \natline{सन्न्यासेनाधिगच्छति ||}
\end{tabular}
\end{table}

\begin{table}[H]
\begin{tabular}{cl}
\textbf{18.50} & \natline{सिद्धिं प्राप्तो यथा ब्रह्म} \\
 & \natline{तथाऽऽप्नोति निबोध मे |} \\
 & \natline{समासेनैव कौन्तेय} \\
 & \natline{निष्ठा ज्ञानस्य या परा ||}
\end{tabular}
\end{table}

\begin{table}[H]
\begin{tabular}{cl}
\textbf{18.51} & \natline{बुद्ध्या विशुद्धया युक्तः} \\
 & \natline{धृत्याऽऽत्मानं नियम्य च |} \\
 & \natline{शब्दादीन्विषयांस्त्यक्त्वा} \\
 & \natline{रागद्वेषौ व्युदस्य च ||}
\end{tabular}
\end{table}

\begin{table}[H]
\begin{tabular}{cl}
\textbf{18.52} & \natline{विविक्तसेवी लघ्वाशी} \\
 & \natline{यतवाक्कायमानसः |} \\
 & \natline{ध्यानयोगपरो नित्यं} \\
 & \natline{वैराग्यं समुपाश्रितः ||}
\end{tabular}
\end{table}

\begin{table}[H]
\begin{tabular}{cl}
\textbf{18.53} & \natline{अहङ्कारं बलं दर्पं} \\
 & \natline{कामं क्रोधं परिग्रहम् |} \\
 & \natline{विमुच्य निर्ममः शान्तः} \\
 & \natline{ब्रह्मभूयाय कल्पते ||}
\end{tabular}
\end{table}

\begin{table}[H]
\begin{tabular}{cl}
\textbf{18.54} & \natline{ब्रह्मभूतः प्रसन्नात्मा} \\
 & \natline{न शोचति न काङ्क्षति |} \\
 & \natline{समः सर्वेषु भूतेषु} \\
 & \natline{मद्भक्तिं लभते पराम् ||}
\end{tabular}
\end{table}

\begin{table}[H]
\begin{tabular}{cl}
\textbf{18.55} & \natline{भक्त्या मामभिजानाति} \\
 & \natline{यावान्यश्चास्मि तत्त्वतः |} \\
 & \natline{ततो मां तत्त्वतो ज्ञात्वा} \\
 & \natline{विशते तदनन्तरम् ||}
\end{tabular}
\end{table}

\begin{table}[H]
\begin{tabular}{cl}
\textbf{18.56} & \natline{सर्वकर्माण्यपि सदा} \\
 & \natline{कुर्वाणो मद्व्यपाश्रयः |} \\
 & \natline{मत्प्रसादादवाप्नोति} \\
 & \natline{शाश्वतं पदमव्ययम् ||}
\end{tabular}
\end{table}

\begin{table}[H]
\begin{tabular}{cl}
\textbf{18.57} & \natline{चेतसा सर्वकर्माणि} \\
 & \natline{मयि सन्न्यस्य मत्परः |} \\
 & \natline{बुद्धियोगमुपाश्रित्य} \\
 & \natline{मच्चित्तः सततं भव ||}
\end{tabular}
\end{table}

\begin{table}[H]
\begin{tabular}{cl}
\textbf{18.58} & \natline{मच्चित्तः सर्वदुर्गाणि} \\
 & \natline{मत्प्रसादात्तरिष्यसि |} \\
 & \natline{अथ चेत्त्वमहङ्कारात्} \\
 & \natline{न श्रोष्यसि विनङ्क्ष्यसि ||}
\end{tabular}
\end{table}

\begin{table}[H]
\begin{tabular}{cl}
\textbf{18.59} & \natline{यदहङ्कारमाश्रित्य} \\
 & \natline{न योत्स्य इति मन्यसे |} \\
 & \natline{मिथ्यैष व्यवसायस्ते} \\
 & \natline{प्रकृतिस्त्वां नियोक्ष्यति ||}
\end{tabular}
\end{table}

\begin{table}[H]
\begin{tabular}{cl}
\textbf{18.60} & \natline{स्वभावजेन कौन्तेय} \\
 & \natline{निबद्धः स्वेन कर्मणा |} \\
 & \natline{कर्तुं नेच्छसि यन्मोहात्} \\
 & \natline{करिष्यस्यवशोऽपि तत् ||}
\end{tabular}
\end{table}

\begin{table}[H]
\begin{tabular}{cl}
\textbf{18.61} & \natline{ईश्वरः सर्वभूतानां} \\
 & \natline{हृद्देशेऽर्जुन तिष्ठति |} \\
 & \natline{भ्रामयन्सर्वभूतानि} \\
 & \natline{यन्त्रारूढानि मायया ||}
\end{tabular}
\end{table}

\begin{table}[H]
\begin{tabular}{cl}
\textbf{18.62} & \natline{तमेव शरणं गच्छ} \\
 & \natline{सर्वभावेन भारत |} \\
 & \natline{तत्प्रसादात्परां शान्तिं} \\
 & \natline{स्थानं प्राप्स्यसि शाश्वतम् ||}
\end{tabular}
\end{table}

\begin{table}[H]
\begin{tabular}{cl}
\textbf{18.63} & \natline{इति ते ज्ञानमाख्यातं} \\
 & \natline{गुह्याद्गुह्यतरं मया |} \\
 & \natline{विमृश्यैतदशेषेण} \\
 & \natline{यथेच्छसि तथा कुरु ||}
\end{tabular}
\end{table}

\begin{table}[H]
\begin{tabular}{cl}
\textbf{18.64} & \natline{सर्वगुह्यतमं भूयः} \\
 & \natline{शृणु मे परमं वचः |} \\
 & \natline{इष्टोऽसि मे दृढमिति} \\
 & \natline{ततो वक्ष्यामि ते हितम् ||}
\end{tabular}
\end{table}

\begin{table}[H]
\begin{tabular}{cl}
\textbf{18.65} & \natline{मन्मना भव मद्भक्तः} \\
 & \natline{मद्याजी मां नमस्कुरु |} \\
 & \natline{मामेवैष्यसि सत्यं ते} \\
 & \natline{प्रतिजाने प्रियोऽसि मे ||}
\end{tabular}
\end{table}

\begin{table}[H]
\begin{tabular}{cl}
\textbf{18.66} & \natline{सर्वधर्मान्परित्यज्य} \\
 & \natline{मामेकं शरणं व्रज |} \\
 & \natline{अहं त्वा सर्वपापेभ्यः} \\
 & \natline{मोक्षयिष्यामि मा शुचः ||}
\end{tabular}
\end{table}

\begin{table}[H]
\begin{tabular}{cl}
\textbf{18.67} & \natline{इदं ते नातपस्काय} \\
 & \natline{नाभक्ताय कदाचन |} \\
 & \natline{न चाशुश्रूषवे वाच्यं} \\
 & \natline{न च मां योऽभ्यसूयति ||}
\end{tabular}
\end{table}

\begin{table}[H]
\begin{tabular}{cl}
\textbf{18.68} & \natline{य इमं परमं गुह्यं} \\
 & \natline{मद्भक्तेष्वभिधास्यति |} \\
 & \natline{भक्तिं मयि परां कृत्वा} \\
 & \natline{मामेवैष्यत्यसंशयः ||}
\end{tabular}
\end{table}

\begin{table}[H]
\begin{tabular}{cl}
\textbf{18.69} & \natline{न च तस्मान्मनुष्येषु} \\
 & \natline{कश्चिन्मे प्रियकृत्तमः |} \\
 & \natline{भविता न च मे तस्मात्} \\
 & \natline{अन्यः प्रियतरो भुवि ||}
\end{tabular}
\end{table}

\begin{table}[H]
\begin{tabular}{cl}
\textbf{18.70} & \natline{अध्येष्यते च य इमं} \\
 & \natline{धर्म्यं संवादमावयोः |} \\
 & \natline{ज्ञानयज्ञेन तेनाहम्} \\
 & \natline{इष्टः स्यामिति मे मतिः ||}
\end{tabular}
\end{table}

\begin{table}[H]
\begin{tabular}{cl}
\textbf{18.71} & \natline{श्रद्धावाननसूयश्च} \\
 & \natline{शृणुयादपि यो नरः |} \\
 & \natline{सोऽपि मुक्तः शुभाल्लोकान्} \\
 & \natline{प्राप्नुयात्पुण्यकर्मणाम् ||}
\end{tabular}
\end{table}

\begin{table}[H]
\begin{tabular}{cl}
\textbf{18.72} & \natline{कच्चिदेतच्छ्रुतं पार्थ} \\
 & \natline{त्वयैकाग्रेण चेतसा |} \\
 & \natline{कच्चिदज्ञानसम्मोहः} \\
 & \natline{प्रनष्टस्ते धनञ्जय ||}
\end{tabular}
\end{table}

\begin{table}[H]
\begin{tabular}{cl}
\textbf{18.73} & \natline{अर्जुन उवाच} \\
 & \natline{नष्टो मोहः स्मृतिर्लब्धा} \\
 & \natline{त्वत्प्रसादान्मयाऽच्युत |} \\
 & \natline{स्थितोऽस्मि गतसन्देहः} \\
 & \natline{करिष्ये वचनं तव ||}
\end{tabular}
\end{table}

\begin{table}[H]
\begin{tabular}{cl}
\textbf{18.74} & \natline{सञ्जय उवाच} \\
 & \natline{इत्यहं वासुदेवस्य} \\
 & \natline{पार्थस्य च महात्मनः |} \\
 & \natline{संवादमिममश्रौषम्} \\
 & \natline{अद्भुतं रोमहर्षणम् ||}
\end{tabular}
\end{table}

\begin{table}[H]
\begin{tabular}{cl}
\textbf{18.75} & \natline{व्यासप्रसादाच्छ्रुतवान्} \\
 & \natline{इमं गुह्यतमं परम् |} \\
 & \natline{योगं योगेश्वरात्कृष्णात्} \\
 & \natline{साक्षात्कथयतः स्वयम् ||}
\end{tabular}
\end{table}

\begin{table}[H]
\begin{tabular}{cl}
\textbf{18.76} & \natline{राजन् संस्मृत्य संस्मृत्य} \\
 & \natline{संवादमिममद्भुतम् |} \\
 & \natline{केशवार्जुनयोः पुण्यं} \\
 & \natline{हृष्यामि च मुहुर्मुहुः ||}
\end{tabular}
\end{table}

\begin{table}[H]
\begin{tabular}{cl}
\textbf{18.77} & \natline{तच्च संस्मृत्य संस्मृत्य} \\
 & \natline{रूपमत्यद्भुतं हरेः |} \\
 & \natline{विस्मयो मे महान्राजन्} \\
 & \natline{हृष्यामि च पुनः पुनः ||}
\end{tabular}
\end{table}

\begin{table}[H]
\begin{tabular}{cl}
\textbf{18.78} & \natline{यत्र योगेश्वरः कृष्णः} \\
 & \natline{यत्र पार्थो धनुर्धरः |} \\
 & \natline{तत्र श्रीर्विजयो भूतिः} \\
 & \natline{ध्रुवा नीतिर्मतिर्मम ||}
\end{tabular}
\end{table}

\begin{table}[H]
\begin{tabular}{cl}
 & \natline{श्रीमद्भगवद्गीतासु उपनिषत्सु} \\
 & \natline{ब्रह्मविद्यायां योगशास्त्रे} \\
 & \natline{श्रीकृष्णार्जुन संवादे} \\
 & \natline{मोक्षसन्न्यासयोगो नाम} \\
 & \natline{अष्टादशोध्यायः}
\end{tabular}
\end{table}

