\begin{table}[H]
\begin{tabular}{cl}
\textbf{18.0} & \natline{ओं श्री परमात्मने नमः} \\
 & \natline{अथ अष्टादशोऽध्यायः} \\
 & \natline{मोक्षसन्न्यास योगः}
\end{tabular}
\end{table}

\begin{table}[H]
\begin{tabular}{cl}
\textbf{18.1} & \natline{अर्जुन उवाच} \\
 & \natline{सन्न्यासस्य महाबाहो} \\
 & \natline{तत्त्वमिच्छामि वेदितुम् |} \\
 & \natline{त्यागस्य च हृषिकेश} \\
 & \natline{पृथक्केशिनिषूदन ||}
\end{tabular}
\end{table}

\begin{table}[H]
\begin{tabular}{cl}
\textbf{18.2} & \natline{श्री भगवनुवाच} \\
 & \natline{काम्यानां कर्मणां न्यासं} \\
 & \natline{सन्न्यासं कवयो विदुः |} \\
 & \natline{सर्वकर्मफलत्यागं} \\
 & \natline{प्राहुस्त्यागं विचक्षणाः ||}
\end{tabular}
\end{table}

\begin{table}[H]
\begin{tabular}{cl}
\textbf{18.3} & \natline{त्याज्यं दोषवदित्येके} \\
 & \natline{कर्म प्राहुर्मनीषिणः |} \\
 & \natline{यज्ञदानतपः कर्म} \\
 & \natline{न त्याज्यमिति चापरे ||}
\end{tabular}
\end{table}

\begin{table}[H]
\begin{tabular}{cl}
\textbf{18.4} & \natline{निश्चयं शृणु मे तत्र} \\
 & \natline{त्यागे भरतसत्तम |} \\
 & \natline{त्यागो हि पुरुषव्याघ्र} \\
 & \natline{त्रिविधः सम्प्रकीर्तितः ||}
\end{tabular}
\end{table}

\begin{table}[H]
\begin{tabular}{cl}
\textbf{18.5} & \natline{यज्ञदानतपःकर्म} \\
 & \natline{न त्याज्यं कार्यमेव तत् |} \\
 & \natline{यज्ञो दानं तपश्चैव} \\
 & \natline{पावनानि मनीषिणाम् ||}
\end{tabular}
\end{table}

\begin{table}[H]
\begin{tabular}{cl}
\textbf{18.6} & \natline{एतान्यपि तु कर्माणि} \\
 & \natline{सङ्गं त्यक्त्वा फलानि च |} \\
 & \natline{कर्तव्यानीति मे पार्थ} \\
 & \natline{निश्चितं मतमुत्तमम् ||}
\end{tabular}
\end{table}

\begin{table}[H]
\begin{tabular}{cl}
\textbf{18.7} & \natline{नियतस्य तु सन्न्यासः} \\
 & \natline{कर्मणो नोपपद्यते |} \\
 & \natline{मोहात्तस्य परित्यागः} \\
 & \natline{तामसः परिकीर्तितः ||}
\end{tabular}
\end{table}

\begin{table}[H]
\begin{tabular}{cl}
\textbf{18.8} & \natline{दुःखमित्येव यत्कर्म} \\
 & \natline{कायक्लेशभयात्त्यजेत् |} \\
 & \natline{स कृत्वा राजसं त्यागं} \\
 & \natline{नैव त्यागफलं लभेत् ||}
\end{tabular}
\end{table}

\begin{table}[H]
\begin{tabular}{cl}
\textbf{18.9} & \natline{कार्यमित्येव यत्कर्म} \\
 & \natline{नियतं क्रियतेऽर्जुन |} \\
 & \natline{सङ्गं त्यक्त्वा फलं चैव} \\
 & \natline{स त्यागः सात्त्विको मतः ||}
\end{tabular}
\end{table}

\begin{table}[H]
\begin{tabular}{cl}
\textbf{18.10} & \natline{न द्वेष्ट्यकुशलं कर्म} \\
 & \natline{कुशले नानुषज्जते |} \\
 & \natline{त्यागी सत्त्वसमाविष्टः} \\
 & \natline{मेधावी छिन्नसंशयः ||}
\end{tabular}
\end{table}

\begin{table}[H]
\begin{tabular}{cl}
\textbf{18.11} & \natline{न हि देहभृता शक्यं} \\
 & \natline{त्यक्तुं कर्माण्यशेषतः |} \\
 & \natline{यस्तु कर्मफलत्यागी} \\
 & \natline{स त्यागीत्यभिधीयते ||}
\end{tabular}
\end{table}

\begin{table}[H]
\begin{tabular}{cl}
\textbf{18.12} & \natline{अनिष्टमिष्टं मिश्रं च} \\
 & \natline{त्रिविधं कर्मणः फलम् |} \\
 & \natline{भवत्यत्यागिनां प्रेत्य} \\
 & \natline{न तु सन्न्यासिनां क्वचित् ||}
\end{tabular}
\end{table}

\begin{table}[H]
\begin{tabular}{cl}
\textbf{18.13} & \natline{पञ्चैतानि महाबाहो} \\
 & \natline{कारणानि निबोध मे |} \\
 & \natline{साङ्ख्ये कृतान्ते प्रोक्तानि} \\
 & \natline{सिद्धये सर्वकर्मणाम् ||}
\end{tabular}
\end{table}

\begin{table}[H]
\begin{tabular}{cl}
\textbf{18.14} & \natline{अधिष्ठानं तथा कर्ता} \\
 & \natline{करणं च पृथग्विधम् |} \\
 & \natline{विविधाश्च पृथक्चेष्टाः} \\
 & \natline{दैवं चैवात्र पञ्चमम् ||}
\end{tabular}
\end{table}

\begin{table}[H]
\begin{tabular}{cl}
\textbf{18.15} & \natline{शरीरवाङ्मनोभिर्यत्} \\
 & \natline{कर्म प्रारभते नरः |} \\
 & \natline{न्याय्यं वा विपरीतं वा} \\
 & \natline{पञ्चैते तस्य हेतवः ||}
\end{tabular}
\end{table}

\begin{table}[H]
\begin{tabular}{cl}
\textbf{18.16} & \natline{तत्रैवं सति कर्तारम्} \\
 & \natline{आत्मानं केवलं तु यः |} \\
 & \natline{पश्यत्यकृतबुद्धित्वात्} \\
 & \natline{न स पश्यति दुर्मतिः ||}
\end{tabular}
\end{table}

\begin{table}[H]
\begin{tabular}{cl}
\textbf{18.17} & \natline{यस्य नाहङ्कृतो भावः} \\
 & \natline{बुद्धिर्यस्य न लिप्यते |} \\
 & \natline{हत्वाऽपि स इमाल्लोकान्} \\
 & \natline{न हन्ति न निबध्यते ||}
\end{tabular}
\end{table}

\begin{table}[H]
\begin{tabular}{cl}
\textbf{18.18} & \natline{ज्ञानं ज्ञेयं परिज्ञाता} \\
 & \natline{त्रिविधा कर्मचोदना |} \\
 & \natline{करणं कर्म कर्तेति} \\
 & \natline{त्रिविधः कर्मसङ्ग्रहः ||}
\end{tabular}
\end{table}

\begin{table}[H]
\begin{tabular}{cl}
\textbf{18.19} & \natline{ज्ञानं कर्म च कर्ता च} \\
 & \natline{त्रिधैव गुणभेदतः |} \\
 & \natline{प्रोच्यते गुणसङ्ख्याने} \\
 & \natline{यथावच्छृणु तान्यपि ||}
\end{tabular}
\end{table}

\begin{table}[H]
\begin{tabular}{cl}
\textbf{18.20} & \natline{सर्वभूतेषु येनैकं} \\
 & \natline{भावमव्ययमीक्षते |} \\
 & \natline{अविभक्तं विभक्तेषु} \\
 & \natline{तज्ज्ञानं विद्धि सात्त्विकम् ||}
\end{tabular}
\end{table}

\begin{table}[H]
\begin{tabular}{cl}
\textbf{18.21} & \natline{पृथक्त्वेन तु यज्ज्ञानं} \\
 & \natline{नानाभावान् पृथग्विधान् |} \\
 & \natline{वेत्ति सर्वेषु भूतेषु} \\
 & \natline{तज्ज्ञानं विद्धि राजसम् ||}
\end{tabular}
\end{table}

\begin{table}[H]
\begin{tabular}{cl}
\textbf{18.22} & \natline{यत्तु कृत्स्नवदेकस्मिन्} \\
 & \natline{कार्ये सक्तमहैतुकम् |} \\
 & \natline{अतत्त्वार्थवदल्पं च} \\
 & \natline{तत्तामसमुदाहृतम् ||}
\end{tabular}
\end{table}

\begin{table}[H]
\begin{tabular}{cl}
\textbf{18.23} & \natline{नियतं सङ्गरहितम्} \\
 & \natline{अरागद्वेषतः कृतम् |} \\
 & \natline{अफलप्रेप्सुना कर्म} \\
 & \natline{यत्तत्सात्त्विकमुच्यते ||}
\end{tabular}
\end{table}

\begin{table}[H]
\begin{tabular}{cl}
\textbf{18.24} & \natline{यत्तु कामेप्सुना कर्म} \\
 & \natline{साहङ्कारेण वा पुनः |} \\
 & \natline{क्रियते बहुलायासं} \\
 & \natline{तद्राजसमुदाहृतम् ||}
\end{tabular}
\end{table}

\begin{table}[H]
\begin{tabular}{cl}
\textbf{18.25} & \natline{अनुबन्धं क्षयं हिंसाम्} \\
 & \natline{अनपेक्ष्य च पौरुषम् |} \\
 & \natline{मोहादारभ्यते कर्म} \\
 & \natline{यत्तत्तामसमुच्यते ||}
\end{tabular}
\end{table}

\begin{table}[H]
\begin{tabular}{cl}
\textbf{18.26} & \natline{मुक्तसङ्गोऽनहंवादी} \\
 & \natline{धृत्युत्साहसमन्वितः |} \\
 & \natline{सिद्ध्यसिद्ध्योर्निर्विकारः} \\
 & \natline{कर्ता सात्त्विक उच्यते ||}
\end{tabular}
\end{table}

\begin{table}[H]
\begin{tabular}{cl}
\textbf{18.27} & \natline{रागी कर्मफलप्रेप्सुः} \\
 & \natline{लुब्धो हिंसात्मकोऽशुचिः |} \\
 & \natline{हर्षशोकान्वितः कर्ता} \\
 & \natline{राजसः परिकीर्तितः ||}
\end{tabular}
\end{table}

\begin{table}[H]
\begin{tabular}{cl}
\textbf{18.28} & \natline{अयुक्तः प्राकृतः स्तब्धः} \\
 & \natline{शठो नैष्कृतिकोऽलसः |} \\
 & \natline{विषादी दीर्घसूत्री च} \\
 & \natline{कर्ता तामस उच्यते ||}
\end{tabular}
\end{table}

\begin{table}[H]
\begin{tabular}{cl}
\textbf{18.29} & \natline{बुद्धेर्भेदं धृतेश्चैव} \\
 & \natline{गुणतस्त्रिविधं शृणु |} \\
 & \natline{प्रोच्यमानमशेषेण} \\
 & \natline{पृथक्त्वेन धनञ्जय ||}
\end{tabular}
\end{table}

\begin{table}[H]
\begin{tabular}{cl}
\textbf{18.30} & \natline{प्रवृत्तिं च निवृत्तिं च} \\
 & \natline{कार्याकार्ये भयाभये |} \\
 & \natline{बन्धं मोक्षं च या वेत्ति} \\
 & \natline{बुद्धिः सा पार्थ सात्त्विकी ||}
\end{tabular}
\end{table}

\begin{table}[H]
\begin{tabular}{cl}
\textbf{18.31} & \natline{यया धर्ममधर्मं च} \\
 & \natline{कार्यं चाकार्यमेव च |} \\
 & \natline{अयथावत्प्रजानाति} \\
 & \natline{बुद्धिः सा पार्थ राजसी ||}
\end{tabular}
\end{table}

\begin{table}[H]
\begin{tabular}{cl}
\textbf{18.32} & \natline{अधर्मं धर्ममिति या} \\
 & \natline{मन्यते तमसाऽऽवृता |} \\
 & \natline{सर्वार्थान्विपरीतांश्च} \\
 & \natline{बुद्धिः सा पार्थ तामसी ||}
\end{tabular}
\end{table}

\begin{table}[H]
\begin{tabular}{cl}
\textbf{18.33} & \natline{धृत्या यया धारयते} \\
 & \natline{मनः प्राणेन्द्रियक्रियाः |} \\
 & \natline{योगेनाव्यभिचारिण्या} \\
 & \natline{धृतिः सा पार्थ सात्त्विकी ||}
\end{tabular}
\end{table}

\begin{table}[H]
\begin{tabular}{cl}
\textbf{18.34} & \natline{यया तु धर्मकामार्थान्} \\
 & \natline{धृत्या धारयतेऽर्जुन |} \\
 & \natline{प्रसङ्गेन फलाकाङ्क्षी} \\
 & \natline{धृतिः सा पार्थ राजसी ||}
\end{tabular}
\end{table}

\begin{table}[H]
\begin{tabular}{cl}
\textbf{18.35} & \natline{यया स्वप्नं भयं शोकं} \\
 & \natline{विषादं मदमेव च |} \\
 & \natline{न विमुञ्चति दुर्मेधाः} \\
 & \natline{धृतिः सा तामसी मता ||}
\end{tabular}
\end{table}

\begin{table}[H]
\begin{tabular}{cl}
\textbf{18.36} & \natline{सुखं त्विदानीं त्रिविधं} \\
 & \natline{शृणु मे भरतर्षभ |} \\
 & \natline{अभ्यासाद्रमते यत्र} \\
 & \natline{दुःखान्तं च निगच्छति ||}
\end{tabular}
\end{table}

\begin{table}[H]
\begin{tabular}{cl}
\textbf{18.37} & \natline{यत्तदग्रे विषमिव} \\
 & \natline{परिणामेऽमृतोपमम् |} \\
 & \natline{तत्सुखं सात्त्विकं प्रोक्तम्} \\
 & \natline{आत्मबुद्धिप्रसादजम् ||}
\end{tabular}
\end{table}

\begin{table}[H]
\begin{tabular}{cl}
\textbf{18.38} & \natline{विषयेन्द्रियसंयोगात्} \\
 & \natline{यत्तदग्रेऽमृतोपमम् |} \\
 & \natline{परिणामे विषमिव} \\
 & \natline{तत्सुखं राजसं स्मृतम् ||}
\end{tabular}
\end{table}

\begin{table}[H]
\begin{tabular}{cl}
\textbf{18.39} & \natline{यदग्रे चानुबन्धे च} \\
 & \natline{सुखं मोहनमात्मनः |} \\
 & \natline{निद्रालस्यप्रमादोत्थं} \\
 & \natline{तत्तामसमुदाहृतम् ||}
\end{tabular}
\end{table}

\begin{table}[H]
\begin{tabular}{cl}
\textbf{18.40} & \natline{न तदस्ति पृथिव्यां वा} \\
 & \natline{दिवि देवेषु वा पुनः |} \\
 & \natline{सत्त्वं प्रकृतिजैर्मुक्तं} \\
 & \natline{यदेभिः स्यात्त्रिभिर्गुणैः ||}
\end{tabular}
\end{table}

\begin{table}[H]
\begin{tabular}{cl}
\textbf{18.41} & \natline{ब्राह्मणक्षत्रियविशां} \\
 & \natline{शूद्राणां च परन्तप |} \\
 & \natline{कर्माणि प्रविभक्तानि} \\
 & \natline{स्वभावप्रभवैर्गुणैः ||}
\end{tabular}
\end{table}

\begin{table}[H]
\begin{tabular}{cl}
\textbf{18.42} & \natline{शमो दमस्तपः शौचं} \\
 & \natline{षान्तिरार्जवमेव च |} \\
 & \natline{ज्ञानं विज्ञानमास्तिक्यं} \\
 & \natline{ब्रह्मकर्म स्वभावजम् ||}
\end{tabular}
\end{table}

\begin{table}[H]
\begin{tabular}{cl}
\textbf{18.43} & \natline{शौर्यं तेजो धृतिर्दाक्ष्यं} \\
 & \natline{युद्धे चाप्यपलायनम् |} \\
 & \natline{दानमीश्वरभावश्च} \\
 & \natline{क्षात्रं कर्म स्वभावजम् ||}
\end{tabular}
\end{table}

