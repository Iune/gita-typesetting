\begin{table}[H]
\begin{tabular}{cl}
\textbf{17.0} & \natline{ओं श्री परमात्मने नमः} \\
 & \natline{अथ सप्तदशोऽध्यायः} \\
 & \natline{श्रद्धात्रयविभाग योगः}
\end{tabular}
\end{table}

\begin{table}[H]
\begin{tabular}{cl}
\textbf{17.1} & \natline{अर्जुन उवाच} \\
 & \natline{ये शास्त्रविधिमुत्सृज्य} \\
 & \natline{यजन्ते श्रद्धयान्विताः |} \\
 & \natline{तेषां निष्ठा तु का कृष्ण} \\
 & \natline{सत्त्वमाहो रजस्तमः ||}
\end{tabular}
\end{table}

\begin{table}[H]
\begin{tabular}{cl}
\textbf{17.2} & \natline{श्री भगवानुवाच} \\
 & \natline{त्रिविधा भवति श्रद्धा} \\
 & \natline{देहिनां सा स्वभावजा |} \\
 & \natline{सात्त्विकी राजसी चैव} \\
 & \natline{तामसी चेति तां शृणु ||}
\end{tabular}
\end{table}

\begin{table}[H]
\begin{tabular}{cl}
\textbf{17.3} & \natline{सत्त्वानुरूपा सर्वस्य} \\
 & \natline{श्रद्धा भवति भारत |} \\
 & \natline{श्रद्धामयोऽयं पुरुषः} \\
 & \natline{यो यcच्रद्धः स एव सः ||}
\end{tabular}
\end{table}

\begin{table}[H]
\begin{tabular}{cl}
\textbf{17.4} & \natline{यजन्ते सात्त्विका देवान्} \\
 & \natline{यक्षरक्षांसि राजसाः |} \\
 & \natline{प्रेतान् भूतगणांश्चान्ये} \\
 & \natline{यजन्ते तामसा जनाः ||}
\end{tabular}
\end{table}

\begin{table}[H]
\begin{tabular}{cl}
\textbf{17.5} & \natline{अशास्त्रविहितं घोरं} \\
 & \natline{तप्यते ये तपो जनाः |} \\
 & \natline{दम्भाहन्कारसम्युताः} \\
 & \natline{कामरागबलान्विताः ||}
\end{tabular}
\end{table}

\begin{table}[H]
\begin{tabular}{cl}
\textbf{17.6} & \natline{कर्शयन्तः शरीरस्थं} \\
 & \natline{भूतग्राममcएतसः |} \\
 & \natline{मां चैवान्तः शरीरस्थं} \\
 & \natline{तान् विद्ध्यासुरनिश्चयान् ||}
\end{tabular}
\end{table}

\begin{table}[H]
\begin{tabular}{cl}
\textbf{17.7} & \natline{आहारस्त्वपि सर्वस्य} \\
 & \natline{त्रिविधो भवति प्रियः |} \\
 & \natline{यज्ञस्तपस्तथा दानं} \\
 & \natline{तेषां भेदमिमम् शृणु ||}
\end{tabular}
\end{table}

\begin{table}[H]
\begin{tabular}{cl}
\textbf{17.8} & \natline{आयुस्सत्त्वबलारोग्य} \\
 & \natline{सुखप्रीतिविवर्धनाः |} \\
 & \natline{रस्याः स्निग्धाः स्थिरा हृद्याः} \\
 & \natline{आहाराः सात्त्विकप्रियाः ||}
\end{tabular}
\end{table}

\begin{table}[H]
\begin{tabular}{cl}
\textbf{17.9} & \natline{कट्वम्ललवणात्युष्ण} \\
 & \natline{तीक्ष्णरूक्षविदाहिनः |} \\
 & \natline{अहारा राजसस्येष्टाः} \\
 & \natline{दुःखशोकामयप्रदाः ||}
\end{tabular}
\end{table}

\begin{table}[H]
\begin{tabular}{cl}
\textbf{17.10} & \natline{यातयामं गतरसं} \\
 & \natline{पूति पर्युषितं च यत् |} \\
 & \natline{उच्छिष्टमपि चामेध्यं} \\
 & \natline{भोजनं तामसप्रियम् ||}
\end{tabular}
\end{table}

\begin{table}[H]
\begin{tabular}{cl}
\textbf{17.11} & \natline{अफलाकाङ्क्षिभिर्यज्ञः} \\
 & \natline{विधिदृष्टो य इज्यते |} \\
 & \natline{यष्टव्यमेवेति मनः} \\
 & \natline{समाधाय स सात्त्विकः ||}
\end{tabular}
\end{table}

\begin{table}[H]
\begin{tabular}{cl}
\textbf{17.12} & \natline{अभिसन्धाय तु फलं} \\
 & \natline{दम्भार्थमपि चैव यत् |} \\
 & \natline{इज्यते भरतश्रेष्ठ} \\
 & \natline{तं यज्ञं विद्धि राजसम् ||}
\end{tabular}
\end{table}

\begin{table}[H]
\begin{tabular}{cl}
\textbf{17.13} & \natline{विधिहीनमसृष्टान्नं} \\
 & \natline{मन्त्रहीनमदक्षिणम् |} \\
 & \natline{श्रद्धाविरहितं यज्ञं} \\
 & \natline{तामसं परिचक्षते ||}
\end{tabular}
\end{table}

\begin{table}[H]
\begin{tabular}{cl}
\textbf{17.14} & \natline{देवद्विजगुरुप्राज्ञ} \\
 & \natline{प्ū̀जनं शौचमार्जवम् |} \\
 & \natline{ब्रह्मचर्यमहिंसा च} \\
 & \natline{शारीरं तप उच्यते ||}
\end{tabular}
\end{table}

\begin{table}[H]
\begin{tabular}{cl}
\textbf{17.15} & \natline{अनुद्वेगकरं वाक्यं} \\
 & \natline{सत्यं प्रियहितं च यत् |} \\
 & \natline{स्वाध्यायाभ्यसनं चैव} \\
 & \natline{वाङ्मयं तप उच्यते ||}
\end{tabular}
\end{table}

\begin{table}[H]
\begin{tabular}{cl}
\textbf{17.16} & \natline{मनः प्रसादः सौम्यत्वं} \\
 & \natline{मौनमात्मविनिग्रहः |} \\
 & \natline{भावसंशुद्धिरित्येतत्} \\
 & \natline{तपो मानसमुच्यते ||}
\end{tabular}
\end{table}

\begin{table}[H]
\begin{tabular}{cl}
\textbf{17.17} & \natline{श्रद्धया परया तप्तं} \\
 & \natline{तपस्तत् त्रिविधं नरैः |} \\
 & \natline{अफलाकाङ्क्षिभिर्युक्तैः} \\
 & \natline{सत्त्विकं परिचक्षते ||}
\end{tabular}
\end{table}

\begin{table}[H]
\begin{tabular}{cl}
\textbf{17.18} & \natline{सत्कारमानपूजार्थं} \\
 & \natline{तपो दम्भेन चैव यत् |} \\
 & \natline{क्रियते तदिह प्रोक्तं} \\
 & \natline{राजसं चलमध्रुवम् ||}
\end{tabular}
\end{table}

\begin{table}[H]
\begin{tabular}{cl}
\textbf{17.19} & \natline{मूढग्राहेण्ā́त्मनो यत्} \\
 & \natline{पीडया क्रियते तपः |} \\
 & \natline{पर्रस्योत्सादनार्थं वा} \\
 & \natline{तत्तामसमुदाहृतम् ||}
\end{tabular}
\end{table}

\begin{table}[H]
\begin{tabular}{cl}
\textbf{17.20} & \natline{दातव्यमिति यद्दानं} \\
 & \natline{दीयतेऽनुपकारिणे |} \\
 & \natline{देशे काले च पात्रे च} \\
 & \natline{तद्दानं सात्त्विकं स्मृतम् ||}
\end{tabular}
\end{table}

\begin{table}[H]
\begin{tabular}{cl}
\textbf{17.21} & \natline{यत्तु प्रत्युपकारार्थं} \\
 & \natline{फलमुद्दिश्य वा पुनः |} \\
 & \natline{दीयते च परिक्लिष्टं} \\
 & \natline{तद्दानं राजसं स्मृतम् ||}
\end{tabular}
\end{table}

\begin{table}[H]
\begin{tabular}{cl}
\textbf{17.22} & \natline{अदेशकाले यद्दानम्} \\
 & \natline{अपात्रेभ्यश्च दीयते |} \\
 & \natline{असत्कृतमवज्ञातं} \\
 & \natline{तत्तामसमुदाहृतम् ||}
\end{tabular}
\end{table}

\begin{table}[H]
\begin{tabular}{cl}
\textbf{17.23} & \natline{ओं तत्सदिति निर्देशः} \\
 & \natline{ब्रह्मणस्त्रिविधः स्मृतः |} \\
 & \natline{ब्राह्मणास्तेन वेदाश्च} \\
 & \natline{यज्ञाश्च विहिताः पुरा ||}
\end{tabular}
\end{table}

\begin{table}[H]
\begin{tabular}{cl}
\textbf{17.24} & \natline{तस्मादोमित्युदाहृत्य} \\
 & \natline{यज्ञदानतपःक्रियाः |} \\
 & \natline{प्रवर्तन्ते विधानोक्ताः} \\
 & \natline{सततं ब्रह्मवादिनाम् ||}
\end{tabular}
\end{table}

\begin{table}[H]
\begin{tabular}{cl}
\textbf{17.25} & \natline{तदित्यनभिसन्धाय} \\
 & \natline{फलं यज्ञतपःक्रियाः |} \\
 & \natline{दानक्रियाश्च विविधाः} \\
 & \natline{क्रियन्ते मोक्षकाङ्क्षिभिः ||}
\end{tabular}
\end{table}

\begin{table}[H]
\begin{tabular}{cl}
\textbf{17.26} & \natline{सद्भावे साधुभावे च} \\
 & \natline{सदित्येतत्प्रयुज्यते |} \\
 & \natline{प्रशस्ते कर्मणि तथा} \\
 & \natline{सच्च्हब्दः पार्थ युज्यते ||}
\end{tabular}
\end{table}

\begin{table}[H]
\begin{tabular}{cl}
\textbf{17.27} & \natline{यज्ञे तपसि दाने च} \\
 & \natline{स्थितिः सदिति चोच्यते |} \\
 & \natline{कर्म चैव तदर्थीयं} \\
 & \natline{सदित्येवाभिधीयते ||}
\end{tabular}
\end{table}

\begin{table}[H]
\begin{tabular}{cl}
\textbf{17.28} & \natline{अश्रद्धया हुतं दत्तं} \\
 & \natline{तपस्तप्तं कृतं च यत् |} \\
 & \natline{असदित्युच्यते पार्थ} \\
 & \natline{न च तत्प्रेत्य नो इह ||}
\end{tabular}
\end{table}

