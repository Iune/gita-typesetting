\begin{table}[H]
\begin{tabular}{cl}
\textbf{4.0} & \natline{ओं श्री परमात्मने नमः} \\
 & \natline{अथ चतुर्थोऽध्यायः} \\
 & \natline{ज्ञानयोगः}
\end{tabular}
\end{table}

\begin{table}[H]
\begin{tabular}{cl}
\textbf{4.1} & \natline{श्री भगवानुवाच} \\
 & \natline{इमं विवस्वते योगं} \\
 & \natline{प्रोक्तवानहमव्ययम् |} \\
 & \natline{विवस्वान्मनवे प्राह} \\
 & \natline{मनुरिक्ष्वाकवेऽब्रवीत् ||}
\end{tabular}
\end{table}

\begin{table}[H]
\begin{tabular}{cl}
\textbf{4.2} & \natline{एवं परम्पराप्राप्तम्} \\
 & \natline{इमं राजर्षयो विदुः |} \\
 & \natline{स कालेनेह महता} \\
 & \natline{योगो नष्टः परन्तप ||}
\end{tabular}
\end{table}

\begin{table}[H]
\begin{tabular}{cl}
\textbf{4.3} & \natline{स एवायं मया तेऽद्य} \\
 & \natline{योगः प्रोक्तः पुरातनः |} \\
 & \natline{भक्तोऽसि मे सखा cएति} \\
 & \natline{रहस्यं ह्येतदुत्तमम् ||}
\end{tabular}
\end{table}

\begin{table}[H]
\begin{tabular}{cl}
\textbf{4.4} & \natline{अर्जुन उवाच} \\
 & \natline{अपरं भवतो जन्म} \\
 & \natline{परं जन्म विवस्वतः |} \\
 & \natline{कथमेतद्विजानीयां} \\
 & \natline{त्वमादौ प्रोक्तवानिति ||}
\end{tabular}
\end{table}

\begin{table}[H]
\begin{tabular}{cl}
\textbf{4.5} & \natline{श्री भगवानुवाच} \\
 & \natline{बहूनि मे व्यतीतानि} \\
 & \natline{जन्मानि तव चार्जुन |} \\
 & \natline{तान्यहं वेद सर्वाणि} \\
 & \natline{न त्वं वेत्थ परन्तप ||}
\end{tabular}
\end{table}

