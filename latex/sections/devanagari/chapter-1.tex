\begin{table}[H]
\begin{tabular}{cl}
\textbf{1.0} & \natline{ओं श्री परमात्मने नमः} \\
 & \natline{अथ प्रथमोऽध्यायः} \\
 & \natline{अर्जुनविषादयोगः}
\end{tabular}
\end{table}

\begin{table}[H]
\begin{tabular}{cl}
\textbf{1.1} & \natline{धृतराष्ट्र उवाच} \\
 & \natline{धर्मक्षेत्रे कुरुक्षेत्रे} \\
 & \natline{समवेता युयुत्सवः |} \\
 & \natline{मामकाः पाण्डवाश्चैव} \\
 & \natline{किमकुर्वत सञ्जय ||}
\end{tabular}
\end{table}

\begin{table}[H]
\begin{tabular}{cl}
\textbf{1.2} & \natline{सञ्जय उवाच} \\
 & \natline{दृष्ट्वा तु पाण्डवानीकं} \\
 & \natline{व्यूढं दुर्योधनस्तदा |} \\
 & \natline{आचार्यमुपसङ्गम्य} \\
 & \natline{राजा वचनमब्रवीत् ||}
\end{tabular}
\end{table}

\begin{table}[H]
\begin{tabular}{cl}
\textbf{1.3} & \natline{पश्यैतां पाण्डुपुत्राणाम्} \\
 & \natline{आचार्य महतीं चमूम् |} \\
 & \natline{व्यूढां द्रुपदपुत्रेण} \\
 & \natline{तव शिष्येण धीमता ||}
\end{tabular}
\end{table}

\begin{table}[H]
\begin{tabular}{cl}
\textbf{1.4} & \natline{अत्र शूरा महेष्वासाः} \\
 & \natline{भीमार्जुनसमा युधि |} \\
 & \natline{युयुधानो विराटश्च} \\
 & \natline{द्रुपदश्च महारथः ||}
\end{tabular}
\end{table}

\begin{table}[H]
\begin{tabular}{cl}
\textbf{1.5} & \natline{धृष्टकेतुश्चेकितानः} \\
 & \natline{काशिराजश्च वीर्यवान् |} \\
 & \natline{पुरुजित्कुन्तिभोजश्च} \\
 & \natline{शैब्यश्च नरपुङ्गवः ||}
\end{tabular}
\end{table}

\begin{table}[H]
\begin{tabular}{cl}
\textbf{1.6} & \natline{युधामन्युश्च विक्रान्तः} \\
 & \natline{उत्तमौजाश्च वीर्यवान् |} \\
 & \natline{सौभद्रो द्रौपदेयाश्च} \\
 & \natline{सर्व एव महारथाः ||}
\end{tabular}
\end{table}

\begin{table}[H]
\begin{tabular}{cl}
\textbf{1.7} & \natline{अस्माकं तु विशिष्टा ये} \\
 & \natline{तान्निबोध द्विजोत्तम |} \\
 & \natline{नायका मम सैन्यस्य} \\
 & \natline{सञ्ज्ञार्थं तान्ब्रवीमि ते ||}
\end{tabular}
\end{table}

\begin{table}[H]
\begin{tabular}{cl}
\textbf{1.8} & \natline{भवान्भीष्मश्च कर्णश्च} \\
 & \natline{कृपश्च समितिञ्जयः |} \\
 & \natline{अश्वत्थामा विकर्णश्च} \\
 & \natline{सौमदत्तिस्तथैव च ||}
\end{tabular}
\end{table}

\begin{table}[H]
\begin{tabular}{cl}
\textbf{1.9} & \natline{अन्ये च बहवः शूराः} \\
 & \natline{मदर्थे त्यक्तजीविताः |} \\
 & \natline{नानाशस्त्रप्रहरणाः} \\
 & \natline{सर्वे युद्धविशारदाः ||}
\end{tabular}
\end{table}

\begin{table}[H]
\begin{tabular}{cl}
\textbf{1.10} & \natline{अपर्याप्तं तदस्माकं} \\
 & \natline{बलं भीष्माभिरक्षितम् |} \\
 & \natline{पर्याप्तं त्विदमेतेषां} \\
 & \natline{बलं भीमाभिरक्षितम् ||}
\end{tabular}
\end{table}

\begin{table}[H]
\begin{tabular}{cl}
\textbf{1.11} & \natline{अयनेषु च सर्वेषु} \\
 & \natline{यथाभागमवस्थिताः |} \\
 & \natline{भीष्ममेवाभिरक्षन्तु} \\
 & \natline{भवन्तः सर्व एव हि ||}
\end{tabular}
\end{table}

\begin{table}[H]
\begin{tabular}{cl}
\textbf{1.12} & \natline{तस्य सञ्जनयन्हर्षं} \\
 & \natline{कुरुवृद्धः पितामहः |} \\
 & \natline{सिंहनादं विनद्योच्चैः} \\
 & \natline{शङ्खं दध्मौ प्रतापवान् ||}
\end{tabular}
\end{table}

\begin{table}[H]
\begin{tabular}{cl}
\textbf{1.13} & \natline{ततः शङ्खाश्च भेर्यश्च} \\
 & \natline{पणवानकगोमुखाः |} \\
 & \natline{सहसैवाभ्यहन्यन्त} \\
 & \natline{स शब्दस्तुमुलोऽभवत् ||}
\end{tabular}
\end{table}

\begin{table}[H]
\begin{tabular}{cl}
\textbf{1.14} & \natline{ततः श्वेतैर्हयैर्युक्ते} \\
 & \natline{महति स्यन्दने स्थितौ |} \\
 & \natline{माधवः पाण्डवश्चैव} \\
 & \natline{दिव्यौ शङ्खौ प्रदध्मतुः ||}
\end{tabular}
\end{table}

\begin{table}[H]
\begin{tabular}{cl}
\textbf{1.15} & \natline{पाञ्चजन्यं हृषीकेशः} \\
 & \natline{देवदत्तं धनञ्जयः |} \\
 & \natline{पौण्ड्रं दध्मौ महाशङ्खं} \\
 & \natline{भीमकर्मा वृकोदरः ||}
\end{tabular}
\end{table}

\begin{table}[H]
\begin{tabular}{cl}
\textbf{1.16} & \natline{अनन्तविजयं राजा} \\
 & \natline{कुन्तीपुत्रो युधिष्ठिरः |} \\
 & \natline{नकुलः सहदेवश्च} \\
 & \natline{सुघोषमणिपुष्पकौ ||}
\end{tabular}
\end{table}

\begin{table}[H]
\begin{tabular}{cl}
\textbf{1.17} & \natline{काश्यश्च परमेष्वासः} \\
 & \natline{शिखण्डी च महारथः |} \\
 & \natline{धृष्टद्युम्नो विराटश्च} \\
 & \natline{सात्यकिश्चापराजितः ||}
\end{tabular}
\end{table}

\begin{table}[H]
\begin{tabular}{cl}
\textbf{1.18} & \natline{द्रुपदो द्रौपदेयाश्च} \\
 & \natline{सर्वशः पृथिवीपते |} \\
 & \natline{सोउभद्रश्च महाबाहुः} \\
 & \natline{शङ्खान्दध्मुः पृथक्पृथक् ||}
\end{tabular}
\end{table}

\begin{table}[H]
\begin{tabular}{cl}
\textbf{1.19} & \natline{स घोषो धार्तराष्ट्राणां} \\
 & \natline{हृदयानि व्यदारयत् |} \\
 & \natline{नभश्च पृथिवीं चैव} \\
 & \natline{तुमुलो व्यनुनादयन् ||}
\end{tabular}
\end{table}

\begin{table}[H]
\begin{tabular}{cl}
\textbf{1.20} & \natline{अथ व्यवस्थितान्दृष्ट्वा} \\
 & \natline{धार्तराष्ट्रान् कपिध्वजः |} \\
 & \natline{प्रवृत्ते शस्त्रसम्पाते} \\
 & \natline{धनुरुद्यम्य पाण्डवः ||}
\end{tabular}
\end{table}

\begin{table}[H]
\begin{tabular}{cl}
\textbf{1.21} & \natline{हृषीकेशं तदा वाक्यम्} \\
 & \natline{इदमाह महीपते} \\
 & \natline{अर्जुन उवाच |} \\
 & \natline{सेनयोरुभयोर्मध्ये} \\
 & \natline{रथं स्थापय मेऽच्युत ||}
\end{tabular}
\end{table}

\begin{table}[H]
\begin{tabular}{cl}
\textbf{1.22} & \natline{यावदेतान्निरीक्षेऽहं} \\
 & \natline{योद्धुकामानवस्थितान् |} \\
 & \natline{कैर्मया सह योद्धव्यम्} \\
 & \natline{अस्मिन् रणसमुद्यमे ||}
\end{tabular}
\end{table}

\begin{table}[H]
\begin{tabular}{cl}
\textbf{1.23} & \natline{योत्स्यमानानवेक्षेऽहं} \\
 & \natline{य एतेऽत्र समागताः |} \\
 & \natline{धार्तराष्ट्रस्यदुर्बुद्धेः} \\
 & \natline{युद्धे प्रियचिकीर्षवः ||}
\end{tabular}
\end{table}

