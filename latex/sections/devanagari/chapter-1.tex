\begin{table}[H]
\begin{tabular}{cl}
\textbf{1.0} & \natline{ओं श्री परमात्मने नमः} \\
 & \natline{अथ प्रथमोऽध्यायः} \\
 & \natline{अर्जुनविषादयोगः}
\end{tabular}
\end{table}

\begin{table}[H]
\begin{tabular}{cl}
\textbf{1.1} & \natline{धृतराष्ट्र उवाच} \\
 & \natline{धर्मक्षेत्रे कुरुक्षेत्रे} \\
 & \natline{समवेता युयुत्सवः |} \\
 & \natline{मामकाः पाण्डवाश्चैव} \\
 & \natline{किमकुर्वत सञ्जय ||}
\end{tabular}
\end{table}

\begin{table}[H]
\begin{tabular}{cl}
\textbf{1.2} & \natline{सञ्जय उवाच} \\
 & \natline{दृष्ट्वा तु पाण्डवानीकं} \\
 & \natline{व्यूढं दुर्योधनस्तदा |} \\
 & \natline{आचार्यमुपसङ्गम्य} \\
 & \natline{राजा वचनमब्रवीत् ||}
\end{tabular}
\end{table}

\begin{table}[H]
\begin{tabular}{cl}
\textbf{1.3} & \natline{पश्यैतां पाण्डुपुत्राणाम्} \\
 & \natline{आचार्य महतीं चमूम् |} \\
 & \natline{व्यूढां द्रुपदपुत्रेण} \\
 & \natline{तव शिष्येण धीमता ||}
\end{tabular}
\end{table}

\begin{table}[H]
\begin{tabular}{cl}
\textbf{1.4} & \natline{अत्र शूरा महेष्वासाः} \\
 & \natline{भीमार्जुनसमा युधि |} \\
 & \natline{युयुधानो विराटश्च} \\
 & \natline{द्रुपदश्च महारथः ||}
\end{tabular}
\end{table}

\begin{table}[H]
\begin{tabular}{cl}
\textbf{1.5} & \natline{धृष्टकेतुश्चेकितानः} \\
 & \natline{काशिराजश्च वीर्यवान् |} \\
 & \natline{पुरुजित्कुन्तिभोजश्च} \\
 & \natline{शैब्यश्च नरपुङ्गवः ||}
\end{tabular}
\end{table}

\begin{table}[H]
\begin{tabular}{cl}
\textbf{1.6} & \natline{युधामन्युश्च विक्रान्तः} \\
 & \natline{उत्तमौजाश्च वीर्यवान् |} \\
 & \natline{सौभद्रो द्रौपदेयाश्च} \\
 & \natline{सर्व एव महारथाः ||}
\end{tabular}
\end{table}

\begin{table}[H]
\begin{tabular}{cl}
\textbf{1.7} & \natline{अस्माकं तु विशिष्टा ये} \\
 & \natline{तान्निबोध द्विजोत्तम |} \\
 & \natline{नायका मम सैन्यस्य} \\
 & \natline{सञ्ज्ञार्थं तान्ब्रवीमि ते ||}
\end{tabular}
\end{table}

\begin{table}[H]
\begin{tabular}{cl}
\textbf{1.8} & \natline{भवान्भीष्मश्च कर्णश्च} \\
 & \natline{कृपश्च समितिञ्जयः |} \\
 & \natline{अश्वत्थामा विकर्णश्च} \\
 & \natline{सौमदत्तिस्तथैव च ||}
\end{tabular}
\end{table}

\begin{table}[H]
\begin{tabular}{cl}
\textbf{1.9} & \natline{अन्ये च बहवः शूराः} \\
 & \natline{मदर्थे त्यक्तजीविताः |} \\
 & \natline{नानाशस्त्रप्रहरणाः} \\
 & \natline{सर्वे युद्धविशारदाः ||}
\end{tabular}
\end{table}

\begin{table}[H]
\begin{tabular}{cl}
\textbf{1.10} & \natline{अपर्याप्तं तदस्माकं} \\
 & \natline{बलं भीष्माभिरक्षितम् |} \\
 & \natline{पर्याप्तं त्विदमेतेषां} \\
 & \natline{बलं भीमाभिरक्षितम् ||}
\end{tabular}
\end{table}

\begin{table}[H]
\begin{tabular}{cl}
\textbf{1.11} & \natline{अयनेषु च सर्वेषु} \\
 & \natline{यथाभागमवस्थिताः |} \\
 & \natline{भीष्ममेवाभिरक्षन्तु} \\
 & \natline{भवन्तः सर्व एव हि ||}
\end{tabular}
\end{table}

\begin{table}[H]
\begin{tabular}{cl}
\textbf{1.12} & \natline{तस्य सञ्जनयन्हर्षं} \\
 & \natline{कुरुवृद्धः पितामहः |} \\
 & \natline{सिंहनादं विनद्योच्चैः} \\
 & \natline{शङ्खं दध्मौ प्रतापवान् ||}
\end{tabular}
\end{table}

\begin{table}[H]
\begin{tabular}{cl}
\textbf{1.13} & \natline{ततः शङ्खाश्च भेर्यश्च} \\
 & \natline{पणवानकगोमुखाः |} \\
 & \natline{सहसैवाभ्यहन्यन्त} \\
 & \natline{स शब्दस्तुमुलोऽभवत् ||}
\end{tabular}
\end{table}

\begin{table}[H]
\begin{tabular}{cl}
\textbf{1.14} & \natline{ततः श्वेतैर्हयैर्युक्ते} \\
 & \natline{महति स्यन्दने स्थितौ |} \\
 & \natline{माधवः पाण्डवश्चैव} \\
 & \natline{दिव्यौ शङ्खौ प्रदध्मतुः ||}
\end{tabular}
\end{table}

\begin{table}[H]
\begin{tabular}{cl}
\textbf{1.15} & \natline{पाञ्चजन्यं हृषीकेशः} \\
 & \natline{देवदत्तं धनञ्जयः |} \\
 & \natline{पौण्ड्रं दध्मौ महाशङ्खं} \\
 & \natline{भीमकर्मा वृकोदरः ||}
\end{tabular}
\end{table}

\begin{table}[H]
\begin{tabular}{cl}
\textbf{1.16} & \natline{अनन्तविजयं राजा} \\
 & \natline{कुन्तीपुत्रो युधिष्ठिरः |} \\
 & \natline{नकुलः सहदेवश्च} \\
 & \natline{सुघोषमणिपुष्पकौ ||}
\end{tabular}
\end{table}

\begin{table}[H]
\begin{tabular}{cl}
\textbf{1.17} & \natline{काश्यश्च परमेष्वासः} \\
 & \natline{शिखण्डी च महारथः |} \\
 & \natline{धृष्टद्युम्नो विराटश्च} \\
 & \natline{सात्यकिश्चापराजितः ||}
\end{tabular}
\end{table}

\begin{table}[H]
\begin{tabular}{cl}
\textbf{1.18} & \natline{द्रुपदो द्रौपदेयाश्च} \\
 & \natline{सर्वशः पृथिवीपते |} \\
 & \natline{सोउभद्रश्च महाबाहुः} \\
 & \natline{शङ्खान्दध्मुः पृथक्पृथक् ||}
\end{tabular}
\end{table}

\begin{table}[H]
\begin{tabular}{cl}
\textbf{1.19} & \natline{स घोषो धार्तराष्ट्राणां} \\
 & \natline{हृदयानि व्यदारयत् |} \\
 & \natline{नभश्च पृथिवीं चैव} \\
 & \natline{तुमुलो व्यनुनादयन् ||}
\end{tabular}
\end{table}

\begin{table}[H]
\begin{tabular}{cl}
\textbf{1.20} & \natline{अथ व्यवस्थितान्दृष्ट्वा} \\
 & \natline{धार्तराष्ट्रान् कपिध्वजः |} \\
 & \natline{प्रवृत्ते शस्त्रसम्पाते} \\
 & \natline{धनुरुद्यम्य पाण्डवः ||}
\end{tabular}
\end{table}

\begin{table}[H]
\begin{tabular}{cl}
\textbf{1.21} & \natline{हृषीकेशं तदा वाक्यम्} \\
 & \natline{इदमाह महीपते} \\
 & \natline{अर्जुन उवाच |} \\
 & \natline{सेनयोरुभयोर्मध्ये} \\
 & \natline{रथं स्थापय मेऽच्युत ||}
\end{tabular}
\end{table}

\begin{table}[H]
\begin{tabular}{cl}
\textbf{1.22} & \natline{यावदेतान्निरीक्षेऽहं} \\
 & \natline{योद्धुकामानवस्थितान् |} \\
 & \natline{कैर्मया सह योद्धव्यम्} \\
 & \natline{अस्मिन् रणसमुद्यमे ||}
\end{tabular}
\end{table}

\begin{table}[H]
\begin{tabular}{cl}
\textbf{1.23} & \natline{योत्स्यमानानवेक्षेऽहं} \\
 & \natline{य एतेऽत्र समागताः |} \\
 & \natline{धार्तराष्ट्रस्यदुर्बुद्धेः} \\
 & \natline{युद्धे प्रियचिकीर्षवः ||}
\end{tabular}
\end{table}

\begin{table}[H]
\begin{tabular}{cl}
\textbf{1.24} & \natline{सञ्जय उवाच} \\
 & \natline{एवमुक्तो हृषीकेशः} \\
 & \natline{गुडाकेशेन भारत |} \\
 & \natline{सेनयोरुभयोर्मध्ये} \\
 & \natline{स्थापयित्वा रथोत्तमम् ||}
\end{tabular}
\end{table}

\begin{table}[H]
\begin{tabular}{cl}
\textbf{1.25} & \natline{भीष्मद्रोणप्रमुखतः} \\
 & \natline{सर्वेषां च महीक्षिताम् |} \\
 & \natline{उवाच पार्थ पश्यैतान्} \\
 & \natline{समवेतान्कुरूनिति ||}
\end{tabular}
\end{table}

\begin{table}[H]
\begin{tabular}{cl}
\textbf{1.26} & \natline{तत्रापश्यत्स्थितान्पार्थः} \\
 & \natline{पितॄनथ पितामहान् |} \\
 & \natline{आचार्यान्मातुलान्भ्रातॄन्} \\
 & \natline{पुत्रान्पौत्रान्सखींस्तथा ||}
\end{tabular}
\end{table}

\begin{table}[H]
\begin{tabular}{cl}
\textbf{1.27} & \natline{श्वशुरान्सुहृदश्चैव} \\
 & \natline{सेनयोरुभयोरपि |} \\
 & \natline{तान्समीक्ष्य स कौन्तेयः} \\
 & \natline{सर्वान्बन्धूनवस्थितान् ||}
\end{tabular}
\end{table}

\begin{table}[H]
\begin{tabular}{cl}
\textbf{1.28} & \natline{कृपया परयाऽऽविष्टः} \\
 & \natline{विषीदन्निदमब्रवीत्} \\
 & \natline{अर्जुन उवाच |} \\
 & \natline{दृष्ट्वेमं स्वजनं कृष्ण} \\
 & \natline{युयुत्सुं समुपस्थितम् ||}
\end{tabular}
\end{table}

\begin{table}[H]
\begin{tabular}{cl}
\textbf{1.29} & \natline{सीदन्ति मम गात्राणि} \\
 & \natline{मुखं च परिशुष्यति |} \\
 & \natline{वेपथुश्च शरीरे मे} \\
 & \natline{रोमहर्षश्च जायते ||}
\end{tabular}
\end{table}

\begin{table}[H]
\begin{tabular}{cl}
\textbf{1.30} & \natline{गाण्डीवं स्रंसते हस्तात्} \\
 & \natline{त्वक्चैव परिदह्यते |} \\
 & \natline{न च शक्नोम्यवस्थातुं} \\
 & \natline{भ्रमतीव च मे मनः ||}
\end{tabular}
\end{table}

\begin{table}[H]
\begin{tabular}{cl}
\textbf{1.31} & \natline{निमित्तानि च पश्यामि} \\
 & \natline{विपरीतानि केशव |} \\
 & \natline{न च श्रेयोऽनुपश्यामि} \\
 & \natline{हत्वा स्वजनमाहवे ||}
\end{tabular}
\end{table}

\begin{table}[H]
\begin{tabular}{cl}
\textbf{1.32} & \natline{न काङ्क्षे विजयं कृष्ण} \\
 & \natline{न च राज्यं सुखानि च |} \\
 & \natline{किं नो राज्येन गोविन्द} \\
 & \natline{किं भोगैर्जीवितेन वा ||}
\end{tabular}
\end{table}

\begin{table}[H]
\begin{tabular}{cl}
\textbf{1.33} & \natline{येषामर्थे काङ्क्षितं नः} \\
 & \natline{राज्यं भोगाः सुखानि च |} \\
 & \natline{त इमेऽवस्थिता युद्धे} \\
 & \natline{प्राणांस्त्यक्त्वा धनानि च ||}
\end{tabular}
\end{table}

\begin{table}[H]
\begin{tabular}{cl}
\textbf{1.34} & \natline{आचार्याः पितरः पुत्राः} \\
 & \natline{तथैव च पितामहाः |} \\
 & \natline{मातुलाः श्वशुराः पौत्राः} \\
 & \natline{श्यालाः सम्बन्धिनस्तथा ||}
\end{tabular}
\end{table}

\begin{table}[H]
\begin{tabular}{cl}
\textbf{1.35} & \natline{एतान्न हन्तुमिच्चामि} \\
 & \natline{घ्नतोऽपि मधुसूदन |} \\
 & \natline{अपि त्रैलोक्यराज्यस्य} \\
 & \natline{हेतोः किं नु महीकृते ||}
\end{tabular}
\end{table}

\begin{table}[H]
\begin{tabular}{cl}
\textbf{1.36} & \natline{निहत्य धार्तराष्ट्रान्नः} \\
 & \natline{का प्रीतिः स्याज्जनार्दन |} \\
 & \natline{पापमेवाश्रयेदस्मान्} \\
 & \natline{हत्वैतानाततायिनः ||}
\end{tabular}
\end{table}

\begin{table}[H]
\begin{tabular}{cl}
\textbf{1.37} & \natline{तस्मान्नार्हा वयं हन्तुं} \\
 & \natline{धार्तराष्ट्रान्स्वबान्धवान् |} \\
 & \natline{स्वजनं हि कथं हत्वा} \\
 & \natline{सुखिनः स्याम माधव ||}
\end{tabular}
\end{table}

\begin{table}[H]
\begin{tabular}{cl}
\textbf{1.38} & \natline{यद्यप्येते न पश्यन्ति} \\
 & \natline{ओभोपहतचेतसः |} \\
 & \natline{कुलक्षयकृतं दोषं} \\
 & \natline{मित्रद्रोहे च पातकम् ||}
\end{tabular}
\end{table}

\begin{table}[H]
\begin{tabular}{cl}
\textbf{1.39} & \natline{कथं न ज्ञेयमस्माभिः} \\
 & \natline{पापादस्मान्निवर्तितुम् |} \\
 & \natline{कुलक्षयकृतं दोषं} \\
 & \natline{प्रपश्यद्भिर्जनार्दन ||}
\end{tabular}
\end{table}

\begin{table}[H]
\begin{tabular}{cl}
\textbf{1.40} & \natline{कुलक्षये प्रणश्यन्ति} \\
 & \natline{कुलधर्माः सनातनाः |} \\
 & \natline{धर्मे नष्टे कुलं कृत्स्नम्} \\
 & \natline{अधर्मोऽभिभवत्युत ||}
\end{tabular}
\end{table}

\begin{table}[H]
\begin{tabular}{cl}
\textbf{1.41} & \natline{अधर्माभिभवात्कृष्ण} \\
 & \natline{प्रदुष्यन्ति कुलस्त्रियः |} \\
 & \natline{स्त्रीषु दुष्टासु वार्ष्णेय} \\
 & \natline{जायते वर्णसङ्करः ||}
\end{tabular}
\end{table}

\begin{table}[H]
\begin{tabular}{cl}
\textbf{1.42} & \natline{सङ्करो नरकायैव} \\
 & \natline{कुलघ्नानां कुलस्य च |} \\
 & \natline{पतन्ति पितरो ह्येषां} \\
 & \natline{लुप्तपिण्डोदकक्रियाः ||}
\end{tabular}
\end{table}

\begin{table}[H]
\begin{tabular}{cl}
\textbf{1.43} & \natline{दोषैरेतैः कुलघ्नानां} \\
 & \natline{वर्णसङ्करकारकैः |} \\
 & \natline{उत्साद्यन्ते जातिधर्माः} \\
 & \natline{कुलधर्माश्च शाश्वताः ||}
\end{tabular}
\end{table}

\begin{table}[H]
\begin{tabular}{cl}
\textbf{1.44} & \natline{उत्सन्नकुलधर्माणां} \\
 & \natline{मनुष्याणां जनार्दन |} \\
 & \natline{नरकेऽनियतं वासः} \\
 & \natline{भवतीत्यनुशुश्रुम ||}
\end{tabular}
\end{table}

\begin{table}[H]
\begin{tabular}{cl}
\textbf{1.45} & \natline{अहो बत महत्पापं} \\
 & \natline{कर्तुं व्यवसिता वयम् |} \\
 & \natline{यद्राज्यसुखलोभेन} \\
 & \natline{हन्तुं स्वजनमुद्यताः ||}
\end{tabular}
\end{table}

\begin{table}[H]
\begin{tabular}{cl}
\textbf{1.46} & \natline{यदि मामप्रतीकारम्} \\
 & \natline{अशस्त्रं शस्त्रपाणयः |} \\
 & \natline{धार्तराष्ट्रा रणे हन्युः} \\
 & \natline{तन्मे क्षेमतरं भवेत् ||}
\end{tabular}
\end{table}

\begin{table}[H]
\begin{tabular}{cl}
\textbf{1.47} & \natline{सञ्जय उवाच} \\
 & \natline{एवमुक्त्वाऽर्जुनः सङ्ख्ये} \\
 & \natline{रथोपस्थ उपाविशत् |} \\
 & \natline{विसृज्य सशरं चापं} \\
 & \natline{शोकसंविग्नमानसः ||}
\end{tabular}
\end{table}

