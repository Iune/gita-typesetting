\begin{table}[H]
\begin{tabular}{cl}
\textbf{18.0} & \natline{ఓం శ్రీ పరమాత్మనే నమః} \\
 & \natline{అథ అష్టాదశోఽధ్యాయః} \\
 & \natline{మోక్షసన్న్యాస యోగః}
\end{tabular}
\end{table}

\begin{table}[H]
\begin{tabular}{cl}
\textbf{18.1} & \natline{అర్జున ఉవాచ} \\
 & \natline{సన్న్యాసస్య మహాబాహో} \\
 & \natline{తత్త్వమిచ్ఛామి వేదితుమ్ |} \\
 & \natline{త్యాగస్య చ హృషికేశ} \\
 & \natline{పృథక్కేశినిషూదన ||}
\end{tabular}
\end{table}

\begin{table}[H]
\begin{tabular}{cl}
\textbf{18.2} & \natline{శ్రీ భగవనువాచ} \\
 & \natline{కామ్యానాం కర్మణాం న్యాసం} \\
 & \natline{సన్న్యాసం కవయో విదుః |} \\
 & \natline{సర్వకర్మఫలత్యాగం} \\
 & \natline{ప్రాహుస్త్యాగం విచక్షణాః ||}
\end{tabular}
\end{table}

\begin{table}[H]
\begin{tabular}{cl}
\textbf{18.3} & \natline{త్యాజ్యం దోషవదిత్యేకే} \\
 & \natline{కర్మ ప్రాహుర్మనీషిణః |} \\
 & \natline{యజ్ఞదానతపః కర్మ} \\
 & \natline{న త్యాజ్యమితి చాపరే ||}
\end{tabular}
\end{table}

\begin{table}[H]
\begin{tabular}{cl}
\textbf{18.4} & \natline{నిశ్చయం శృణు మే తత్ర} \\
 & \natline{త్యాగే భరతసత్తమ |} \\
 & \natline{త్యాగో హి పురుషవ్యాఘ్ర} \\
 & \natline{త్రివిధః సమ్ప్రకీర్తితః ||}
\end{tabular}
\end{table}

\begin{table}[H]
\begin{tabular}{cl}
\textbf{18.5} & \natline{యజ్ఞదానతపఃకర్మ} \\
 & \natline{న త్యాజ్యం కార్యమేవ తత్ |} \\
 & \natline{యజ్ఞో దానం తపశ్చైవ} \\
 & \natline{పావనాని మనీషిణామ్ ||}
\end{tabular}
\end{table}

\begin{table}[H]
\begin{tabular}{cl}
\textbf{18.6} & \natline{ఏతాన్యపి తు కర్మాణి} \\
 & \natline{సఙ్గం త్యక్త్వా ఫలాని చ |} \\
 & \natline{కర్తవ్యానీతి మే పార్థ} \\
 & \natline{నిశ్చితం మతముత్తమమ్ ||}
\end{tabular}
\end{table}

\begin{table}[H]
\begin{tabular}{cl}
\textbf{18.7} & \natline{నియతస్య తు సన్న్యాసః} \\
 & \natline{కర్మణో నోపపద్యతే |} \\
 & \natline{మోహాత్తస్య పరిత్యాగః} \\
 & \natline{తామసః పరికీర్తితః ||}
\end{tabular}
\end{table}

\begin{table}[H]
\begin{tabular}{cl}
\textbf{18.8} & \natline{దుఃఖమిత్యేవ యత్కర్మ} \\
 & \natline{కాయక్లేశభయాత్త్యజేత్ |} \\
 & \natline{స కృత్వా రాజసం త్యాగం} \\
 & \natline{నైవ త్యాగఫలం లభేత్ ||}
\end{tabular}
\end{table}

\begin{table}[H]
\begin{tabular}{cl}
\textbf{18.9} & \natline{కార్యమిత్యేవ యత్కర్మ} \\
 & \natline{నియతం క్రియతేఽర్జున |} \\
 & \natline{సఙ్గం త్యక్త్వా ఫలం చైవ} \\
 & \natline{స త్యాగః సాత్త్వికో మతః ||}
\end{tabular}
\end{table}

\begin{table}[H]
\begin{tabular}{cl}
\textbf{18.10} & \natline{న ద్వేష్ట్యకుశలం కర్మ} \\
 & \natline{కుశలే నానుషజ్జతే |} \\
 & \natline{త్యాగీ సత్త్వసమావిష్టః} \\
 & \natline{మేధావీ ఛిన్నసంశయః ||}
\end{tabular}
\end{table}

\begin{table}[H]
\begin{tabular}{cl}
\textbf{18.11} & \natline{న హి దేహభృతా శక్యం} \\
 & \natline{త్యక్తుం కర్మాణ్యశేషతః |} \\
 & \natline{యస్తు కర్మఫలత్యాగీ} \\
 & \natline{స త్యాగీత్యభిధీయతే ||}
\end{tabular}
\end{table}

\begin{table}[H]
\begin{tabular}{cl}
\textbf{18.12} & \natline{అనిష్టమిష్టం మిశ్రం చ} \\
 & \natline{త్రివిధం కర్మణః ఫలమ్ |} \\
 & \natline{భవత్యత్యాగినాం ప్రేత్య} \\
 & \natline{న తు సన్న్యాసినాం క్వచిత్ ||}
\end{tabular}
\end{table}

\begin{table}[H]
\begin{tabular}{cl}
\textbf{18.13} & \natline{పఞ్చైతాని మహాబాహో} \\
 & \natline{కారణాని నిబోధ మే |} \\
 & \natline{సాఙ్ఖ్యే కృతాన్తే ప్రోక్తాని} \\
 & \natline{సిద్ధయే సర్వకర్మణామ్ ||}
\end{tabular}
\end{table}

\begin{table}[H]
\begin{tabular}{cl}
\textbf{18.14} & \natline{అధిష్ఠానం తథా కర్తా} \\
 & \natline{కరణం చ పృథగ్విధమ్ |} \\
 & \natline{వివిధాశ్చ పృథక్చేష్టాః} \\
 & \natline{దైవం చైవాత్ర పఞ్చమమ్ ||}
\end{tabular}
\end{table}

\begin{table}[H]
\begin{tabular}{cl}
\textbf{18.15} & \natline{శరీరవాఙ్మనోభిర్యత్} \\
 & \natline{కర్మ ప్రారభతే నరః |} \\
 & \natline{న్యాయ్యం వా విపరీతం వా} \\
 & \natline{పఞ్చైతే తస్య హేతవః ||}
\end{tabular}
\end{table}

\begin{table}[H]
\begin{tabular}{cl}
\textbf{18.16} & \natline{తత్రైవం సతి కర్తారమ్} \\
 & \natline{ఆత్మానం కేవలం తు యః |} \\
 & \natline{పశ్యత్యకృతబుద్ధిత్వాత్} \\
 & \natline{న స పశ్యతి దుర్మతిః ||}
\end{tabular}
\end{table}

\begin{table}[H]
\begin{tabular}{cl}
\textbf{18.17} & \natline{యస్య నాహఙ్కృతో భావః} \\
 & \natline{బుద్ధిర్యస్య న లిప్యతే |} \\
 & \natline{హత్వాఽపి స ఇమాల్లోకాన్} \\
 & \natline{న హన్తి న నిబధ్యతే ||}
\end{tabular}
\end{table}

\begin{table}[H]
\begin{tabular}{cl}
\textbf{18.18} & \natline{జ్ఞానం జ్ఞేయం పరిజ్ఞాతా} \\
 & \natline{త్రివిధా కర్మచోదనా |} \\
 & \natline{కరణం కర్మ కర్తేతి} \\
 & \natline{త్రివిధః కర్మసఙ్గ్రహః ||}
\end{tabular}
\end{table}

\begin{table}[H]
\begin{tabular}{cl}
\textbf{18.19} & \natline{జ్ఞానం కర్మ చ కర్తా చ} \\
 & \natline{త్రిధైవ గుణభేదతః |} \\
 & \natline{ప్రోచ్యతే గుణసఙ్ఖ్యానే} \\
 & \natline{యథావచ్ఛృణు తాన్యపి ||}
\end{tabular}
\end{table}

\begin{table}[H]
\begin{tabular}{cl}
\textbf{18.20} & \natline{సర్వభూతేషు యేనైకం} \\
 & \natline{భావమవ్యయమీక్షతే |} \\
 & \natline{అవిభక్తం విభక్తేషు} \\
 & \natline{తజ్జ్ఞానం విద్ధి సాత్త్వికమ్ ||}
\end{tabular}
\end{table}

\begin{table}[H]
\begin{tabular}{cl}
\textbf{18.21} & \natline{పృథక్త్వేన తు యజ్జ్ఞానం} \\
 & \natline{నానాభావాన్ పృథగ్విధాన్ |} \\
 & \natline{వేత్తి సర్వేషు భూతేషు} \\
 & \natline{తజ్జ్ఞానం విద్ధి రాజసమ్ ||}
\end{tabular}
\end{table}

\begin{table}[H]
\begin{tabular}{cl}
\textbf{18.22} & \natline{యత్తు కృత్స్నవదేకస్మిన్} \\
 & \natline{కార్యే సక్తమహైతుకమ్ |} \\
 & \natline{అతత్త్వార్థవదల్పం చ} \\
 & \natline{తత్తామసముదాహృతమ్ ||}
\end{tabular}
\end{table}

\begin{table}[H]
\begin{tabular}{cl}
\textbf{18.23} & \natline{నియతం సఙ్గరహితమ్} \\
 & \natline{అరాగద్వేషతః కృతమ్ |} \\
 & \natline{అఫలప్రేప్సునా కర్మ} \\
 & \natline{యత్తత్సాత్త్వికముచ్యతే ||}
\end{tabular}
\end{table}

\begin{table}[H]
\begin{tabular}{cl}
\textbf{18.24} & \natline{యత్తు కామేప్సునా కర్మ} \\
 & \natline{సాహఙ్కారేణ వా పునః |} \\
 & \natline{క్రియతే బహులాయాసం} \\
 & \natline{తద్రాజసముదాహృతమ్ ||}
\end{tabular}
\end{table}

\begin{table}[H]
\begin{tabular}{cl}
\textbf{18.25} & \natline{అనుబన్ధం క్షయం హింసామ్} \\
 & \natline{అనపేక్ష్య చ పౌరుషమ్ |} \\
 & \natline{మోహాదారభ్యతే కర్మ} \\
 & \natline{యత్తత్తామసముచ్యతే ||}
\end{tabular}
\end{table}

\begin{table}[H]
\begin{tabular}{cl}
\textbf{18.26} & \natline{ముక్తసఙ్గోఽనహంవాదీ} \\
 & \natline{ధృత్యుత్సాహసమన్వితః |} \\
 & \natline{సిద్ధ్యసిద్ధ్యోర్నిర్వికారః} \\
 & \natline{కర్తా సాత్త్విక ఉచ్యతే ||}
\end{tabular}
\end{table}

\begin{table}[H]
\begin{tabular}{cl}
\textbf{18.27} & \natline{రాగీ కర్మఫలప్రేప్సుః} \\
 & \natline{లుబ్ధో హింసాత్మకోఽశుచిః |} \\
 & \natline{హర్షశోకాన్వితః కర్తా} \\
 & \natline{రాజసః పరికీర్తితః ||}
\end{tabular}
\end{table}

\begin{table}[H]
\begin{tabular}{cl}
\textbf{18.28} & \natline{అయుక్తః ప్రాకృతః స్తబ్ధః} \\
 & \natline{శఠో నైష్కృతికోఽలసః |} \\
 & \natline{విషాదీ దీర్ఘసూత్రీ చ} \\
 & \natline{కర్తా తామస ఉచ్యతే ||}
\end{tabular}
\end{table}

\begin{table}[H]
\begin{tabular}{cl}
\textbf{18.29} & \natline{బుద్ధేర్భేదం ధృతేశ్చైవ} \\
 & \natline{గుణతస్త్రివిధం శృణు |} \\
 & \natline{ప్రోచ్యమానమశేషేణ} \\
 & \natline{పృథక్త్వేన ధనఞ్జయ ||}
\end{tabular}
\end{table}

\begin{table}[H]
\begin{tabular}{cl}
\textbf{18.30} & \natline{ప్రవృత్తిం చ నివృత్తిం చ} \\
 & \natline{కార్యాకార్యే భయాభయే |} \\
 & \natline{బన్ధం మోక్షం చ యా వేత్తి} \\
 & \natline{బుద్ధిః సా పార్థ సాత్త్వికీ ||}
\end{tabular}
\end{table}

\begin{table}[H]
\begin{tabular}{cl}
\textbf{18.31} & \natline{యయా ధర్మమధర్మం చ} \\
 & \natline{కార్యం చాకార్యమేవ చ |} \\
 & \natline{అయథావత్ప్రజానాతి} \\
 & \natline{బుద్ధిః సా పార్థ రాజసీ ||}
\end{tabular}
\end{table}

\begin{table}[H]
\begin{tabular}{cl}
\textbf{18.32} & \natline{అధర్మం ధర్మమితి యా} \\
 & \natline{మన్యతే తమసాఽఽవృతా |} \\
 & \natline{సర్వార్థాన్విపరీతాంశ్చ} \\
 & \natline{బుద్ధిః సా పార్థ తామసీ ||}
\end{tabular}
\end{table}

\begin{table}[H]
\begin{tabular}{cl}
\textbf{18.33} & \natline{ధృత్యా యయా ధారయతే} \\
 & \natline{మనః ప్రాణేన్ద్రియక్రియాః |} \\
 & \natline{యోగేనావ్యభిచారిణ్యా} \\
 & \natline{ధృతిః సా పార్థ సాత్త్వికీ ||}
\end{tabular}
\end{table}

\begin{table}[H]
\begin{tabular}{cl}
\textbf{18.34} & \natline{యయా తు ధర్మకామార్థాన్} \\
 & \natline{ధృత్యా ధారయతేఽర్జున |} \\
 & \natline{ప్రసఙ్గేన ఫలాకాఙ్క్షీ} \\
 & \natline{ధృతిః సా పార్థ రాజసీ ||}
\end{tabular}
\end{table}

\begin{table}[H]
\begin{tabular}{cl}
\textbf{18.35} & \natline{యయా స్వప్నం భయం శోకం} \\
 & \natline{విషాదం మదమేవ చ |} \\
 & \natline{న విముఞ్చతి దుర్మేధాః} \\
 & \natline{ధృతిః సా తామసీ మతా ||}
\end{tabular}
\end{table}

\begin{table}[H]
\begin{tabular}{cl}
\textbf{18.36} & \natline{సుఖం త్విదానీం త్రివిధం} \\
 & \natline{శృణు మే భరతర్షభ |} \\
 & \natline{అభ్యాసాద్రమతే యత్ర} \\
 & \natline{దుఃఖాన్తం చ నిగచ్ఛతి ||}
\end{tabular}
\end{table}

\begin{table}[H]
\begin{tabular}{cl}
\textbf{18.37} & \natline{యత్తదగ్రే విషమివ} \\
 & \natline{పరిణామేఽమృతోపమమ్ |} \\
 & \natline{తత్సుఖం సాత్త్వికం ప్రోక్తమ్} \\
 & \natline{ఆత్మబుద్ధిప్రసాదజమ్ ||}
\end{tabular}
\end{table}

\begin{table}[H]
\begin{tabular}{cl}
\textbf{18.38} & \natline{విషయేన్ద్రియసంయోగాత్} \\
 & \natline{యత్తదగ్రేఽమృతోపమమ్ |} \\
 & \natline{పరిణామే విషమివ} \\
 & \natline{తత్సుఖం రాజసం స్మృతమ్ ||}
\end{tabular}
\end{table}

\begin{table}[H]
\begin{tabular}{cl}
\textbf{18.39} & \natline{యదగ్రే చానుబన్ధే చ} \\
 & \natline{సుఖం మోహనమాత్మనః |} \\
 & \natline{నిద్రాలస్యప్రమాదోత్థం} \\
 & \natline{తత్తామసముదాహృతమ్ ||}
\end{tabular}
\end{table}

\begin{table}[H]
\begin{tabular}{cl}
\textbf{18.40} & \natline{న తదస్తి పృథివ్యాం వా} \\
 & \natline{దివి దేవేషు వా పునః |} \\
 & \natline{సత్త్వం ప్రకృతిజైర్ముక్తం} \\
 & \natline{యదేభిః స్యాత్త్రిభిర్గుణైః ||}
\end{tabular}
\end{table}

\begin{table}[H]
\begin{tabular}{cl}
\textbf{18.41} & \natline{బ్రాహ్మణక్షత్రియవిశాం} \\
 & \natline{శూద్రాణాం చ పరన్తప |} \\
 & \natline{కర్మాణి ప్రవిభక్తాని} \\
 & \natline{స్వభావప్రభవైర్గుణైః ||}
\end{tabular}
\end{table}

\begin{table}[H]
\begin{tabular}{cl}
\textbf{18.42} & \natline{శమో దమస్తపః శౌచం} \\
 & \natline{షాన్తిరార్జవమేవ చ |} \\
 & \natline{జ్ఞానం విజ్ఞానమాస్తిక్యం} \\
 & \natline{బ్రహ్మకర్మ స్వభావజమ్ ||}
\end{tabular}
\end{table}

\begin{table}[H]
\begin{tabular}{cl}
\textbf{18.43} & \natline{శౌర్యం తేజో ధృతిర్దాక్ష్యం} \\
 & \natline{యుద్ధే చాప్యపలాయనమ్ |} \\
 & \natline{దానమీశ్వరభావశ్చ} \\
 & \natline{క్షాత్రం కర్మ స్వభావజమ్ ||}
\end{tabular}
\end{table}

