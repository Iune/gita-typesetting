\begin{table}[H]
\begin{tabular}{cl}
\textbf{18.0} & \natline{ఓం శ్రీ పరమాత్మనే నమః} \\
 & \natline{అథ అష్టాదశోఽధ్యాయః} \\
 & \natline{మోక్షసన్న్యాస యోగః}
\end{tabular}
\end{table}

\begin{table}[H]
\begin{tabular}{cl}
\textbf{18.1} & \natline{అర్జున ఉవాచ} \\
 & \natline{సన్న్యాసస్య మహాబాహో} \\
 & \natline{తత్త్వమిచ్ఛామి వేదితుమ్ |} \\
 & \natline{త్యాగస్య చ హృషికేశ} \\
 & \natline{పృథక్కేశినిషూదన ||}
\end{tabular}
\end{table}

\begin{table}[H]
\begin{tabular}{cl}
\textbf{18.2} & \natline{శ్రీ భగవనువాచ} \\
 & \natline{కామ్యానాం కర్మణాం న్యాసం} \\
 & \natline{సన్న్యాసం కవయో విదుః |} \\
 & \natline{సర్వకర్మఫలత్యాగం} \\
 & \natline{ప్రాహుస్త్యాగం విచక్షణాః ||}
\end{tabular}
\end{table}

\begin{table}[H]
\begin{tabular}{cl}
\textbf{18.3} & \natline{త్యాజ్యం దోషవదిత్యేకే} \\
 & \natline{కర్మ ప్రాహుర్మనీషిణః |} \\
 & \natline{యజ్ఞదానతపః కర్మ} \\
 & \natline{న త్యాజ్యమితి చాపరే ||}
\end{tabular}
\end{table}

\begin{table}[H]
\begin{tabular}{cl}
\textbf{18.4} & \natline{నిశ్చయం శృణు మే తత్ర} \\
 & \natline{త్యాగే భరతసత్తమ |} \\
 & \natline{త్యాగో హి పురుషవ్యాఘ్ర} \\
 & \natline{త్రివిధః సమ్ప్రకీర్తితః ||}
\end{tabular}
\end{table}

\begin{table}[H]
\begin{tabular}{cl}
\textbf{18.5} & \natline{యజ్ఞదానతపఃకర్మ} \\
 & \natline{న త్యాజ్యం కార్యమేవ తత్ |} \\
 & \natline{యజ్ఞో దానం తపశ్చైవ} \\
 & \natline{పావనాని మనీషిణామ్ ||}
\end{tabular}
\end{table}

\begin{table}[H]
\begin{tabular}{cl}
\textbf{18.6} & \natline{ఏతాన్యపి తు కర్మాణి} \\
 & \natline{సఙ్గం త్యక్త్వా ఫలాని చ |} \\
 & \natline{కర్తవ్యానీతి మే పార్థ} \\
 & \natline{నిశ్చితం మతముత్తమమ్ ||}
\end{tabular}
\end{table}

\begin{table}[H]
\begin{tabular}{cl}
\textbf{18.7} & \natline{నియతస్య తు సన్న్యాసః} \\
 & \natline{కర్మణో నోపపద్యతే |} \\
 & \natline{మోహాత్తస్య పరిత్యాగః} \\
 & \natline{తామసః పరికీర్తితః ||}
\end{tabular}
\end{table}

\begin{table}[H]
\begin{tabular}{cl}
\textbf{18.8} & \natline{దుఃఖమిత్యేవ యత్కర్మ} \\
 & \natline{కాయక్లేశభయాత్త్యజేత్ |} \\
 & \natline{స కృత్వా రాజసం త్యాగం} \\
 & \natline{నైవ త్యాగఫలం లభేత్ ||}
\end{tabular}
\end{table}

\begin{table}[H]
\begin{tabular}{cl}
\textbf{18.9} & \natline{కార్యమిత్యేవ యత్కర్మ} \\
 & \natline{నియతం క్రియతేఽర్జున |} \\
 & \natline{సఙ్గం త్యక్త్వా ఫలం చైవ} \\
 & \natline{స త్యాగః సాత్త్వికో మతః ||}
\end{tabular}
\end{table}

\begin{table}[H]
\begin{tabular}{cl}
\textbf{18.10} & \natline{న ద్వేష్ట్యకుశలం కర్మ} \\
 & \natline{కుశలే నానుషజ్జతే |} \\
 & \natline{త్యాగీ సత్త్వసమావిష్టః} \\
 & \natline{మేధావీ ఛిన్నసంశయః ||}
\end{tabular}
\end{table}

\begin{table}[H]
\begin{tabular}{cl}
\textbf{18.11} & \natline{న హి దేహభృతా శక్యం} \\
 & \natline{త్యక్తుం కర్మాణ్యశేషతః |} \\
 & \natline{యస్తు కర్మఫలత్యాగీ} \\
 & \natline{స త్యాగీత్యభిధీయతే ||}
\end{tabular}
\end{table}

\begin{table}[H]
\begin{tabular}{cl}
\textbf{18.12} & \natline{అనిష్టమిష్టం మిశ్రం చ} \\
 & \natline{త్రివిధం కర్మణః ఫలమ్ |} \\
 & \natline{భవత్యత్యాగినాం ప్రేత్య} \\
 & \natline{న తు సన్న్యాసినాం క్వచిత్ ||}
\end{tabular}
\end{table}

\begin{table}[H]
\begin{tabular}{cl}
\textbf{18.13} & \natline{పఞ్చైతాని మహాబాహో} \\
 & \natline{కారణాని నిబోధ మే |} \\
 & \natline{సాఙ్ఖ్యే కృతాన్తే ప్రోక్తాని} \\
 & \natline{సిద్ధయే సర్వకర్మణామ్ ||}
\end{tabular}
\end{table}

\begin{table}[H]
\begin{tabular}{cl}
\textbf{18.14} & \natline{అధిష్ఠానం తథా కర్తా} \\
 & \natline{కరణం చ పృథగ్విధమ్ |} \\
 & \natline{వివిధాశ్చ పృథక్చేష్టాః} \\
 & \natline{దైవం చైవాత్ర పఞ్చమమ్ ||}
\end{tabular}
\end{table}

\begin{table}[H]
\begin{tabular}{cl}
\textbf{18.15} & \natline{శరీరవాఙ్మనోభిర్యత్} \\
 & \natline{కర్మ ప్రారభతే నరః |} \\
 & \natline{న్యాయ్యం వా విపరీతం వా} \\
 & \natline{పఞ్చైతే తస్య హేతవః ||}
\end{tabular}
\end{table}

\begin{table}[H]
\begin{tabular}{cl}
\textbf{18.16} & \natline{తత్రైవం సతి కర్తారమ్} \\
 & \natline{ఆత్మానం కేవలం తు యః |} \\
 & \natline{పశ్యత్యకృతబుద్ధిత్వాత్} \\
 & \natline{న స పశ్యతి దుర్మతిః ||}
\end{tabular}
\end{table}

\begin{table}[H]
\begin{tabular}{cl}
\textbf{18.17} & \natline{యస్య నాహఙ్కృతో భావః} \\
 & \natline{బుద్ధిర్యస్య న లిప్యతే |} \\
 & \natline{హత్వాఽపి స ఇమాల్లోకాన్} \\
 & \natline{న హన్తి న నిబధ్యతే ||}
\end{tabular}
\end{table}

\begin{table}[H]
\begin{tabular}{cl}
\textbf{18.18} & \natline{జ్ఞానం జ్ఞేయం పరిజ్ఞాతా} \\
 & \natline{త్రివిధా కర్మచోదనా |} \\
 & \natline{కరణం కర్మ కర్తేతి} \\
 & \natline{త్రివిధః కర్మసఙ్గ్రహః ||}
\end{tabular}
\end{table}

\begin{table}[H]
\begin{tabular}{cl}
\textbf{18.19} & \natline{జ్ఞానం కర్మ చ కర్తా చ} \\
 & \natline{త్రిధైవ గుణభేదతః |} \\
 & \natline{ప్రోచ్యతే గుణసఙ్ఖ్యానే} \\
 & \natline{యథావచ్ఛృణు తాన్యపి ||}
\end{tabular}
\end{table}

\begin{table}[H]
\begin{tabular}{cl}
\textbf{18.20} & \natline{సర్వభూతేషు యేనైకం} \\
 & \natline{భావమవ్యయమీక్షతే |} \\
 & \natline{అవిభక్తం విభక్తేషు} \\
 & \natline{తజ్జ్ఞానం విద్ధి సాత్త్వికమ్ ||}
\end{tabular}
\end{table}

\begin{table}[H]
\begin{tabular}{cl}
\textbf{18.21} & \natline{పృథక్త్వేన తు యజ్జ్ఞానం} \\
 & \natline{నానాభావాన్ పృథగ్విధాన్ |} \\
 & \natline{వేత్తి సర్వేషు భూతేషు} \\
 & \natline{తజ్జ్ఞానం విద్ధి రాజసమ్ ||}
\end{tabular}
\end{table}

\begin{table}[H]
\begin{tabular}{cl}
\textbf{18.22} & \natline{యత్తు కృత్స్నవదేకస్మిన్} \\
 & \natline{కార్యే సక్తమహైతుకమ్ |} \\
 & \natline{అతత్త్వార్థవదల్పం చ} \\
 & \natline{తత్తామసముదాహృతమ్ ||}
\end{tabular}
\end{table}

\begin{table}[H]
\begin{tabular}{cl}
\textbf{18.23} & \natline{నియతం సఙ్గరహితమ్} \\
 & \natline{అరాగద్వేషతః కృతమ్ |} \\
 & \natline{అఫలప్రేప్సునా కర్మ} \\
 & \natline{యత్తత్సాత్త్వికముచ్యతే ||}
\end{tabular}
\end{table}

\begin{table}[H]
\begin{tabular}{cl}
\textbf{18.24} & \natline{యత్తు కామేప్సునా కర్మ} \\
 & \natline{సాహఙ్కారేణ వా పునః |} \\
 & \natline{క్రియతే బహులాయాసం} \\
 & \natline{తద్రాజసముదాహృతమ్ ||}
\end{tabular}
\end{table}

\begin{table}[H]
\begin{tabular}{cl}
\textbf{18.25} & \natline{అనుబన్ధం క్షయం హింసామ్} \\
 & \natline{అనపేక్ష్య చ పౌరుషమ్ |} \\
 & \natline{మోహాదారభ్యతే కర్మ} \\
 & \natline{యత్తత్తామసముచ్యతే ||}
\end{tabular}
\end{table}

\begin{table}[H]
\begin{tabular}{cl}
\textbf{18.26} & \natline{ముక్తసఙ్గోఽనహంవాదీ} \\
 & \natline{ధృత్యుత్సాహసమన్వితః |} \\
 & \natline{సిద్ధ్యసిద్ధ్యోర్నిర్వికారః} \\
 & \natline{కర్తా సాత్త్విక ఉచ్యతే ||}
\end{tabular}
\end{table}

\begin{table}[H]
\begin{tabular}{cl}
\textbf{18.27} & \natline{రాగీ కర్మఫలప్రేప్సుః} \\
 & \natline{లుబ్ధో హింసాత్మకోఽశుచిః |} \\
 & \natline{హర్షశోకాన్వితః కర్తా} \\
 & \natline{రాజసః పరికీర్తితః ||}
\end{tabular}
\end{table}

\begin{table}[H]
\begin{tabular}{cl}
\textbf{18.28} & \natline{అయుక్తః ప్రాకృతః స్తబ్ధః} \\
 & \natline{శఠో నైష్కృతికోఽలసః |} \\
 & \natline{విషాదీ దీర్ఘసూత్రీ చ} \\
 & \natline{కర్తా తామస ఉచ్యతే ||}
\end{tabular}
\end{table}

\begin{table}[H]
\begin{tabular}{cl}
\textbf{18.29} & \natline{బుద్ధేర్భేదం ధృతేశ్చైవ} \\
 & \natline{గుణతస్త్రివిధం శృణు |} \\
 & \natline{ప్రోచ్యమానమశేషేణ} \\
 & \natline{పృథక్త్వేన ధనఞ్జయ ||}
\end{tabular}
\end{table}

\begin{table}[H]
\begin{tabular}{cl}
\textbf{18.30} & \natline{ప్రవృత్తిం చ నివృత్తిం చ} \\
 & \natline{కార్యాకార్యే భయాభయే |} \\
 & \natline{బన్ధం మోక్షం చ యా వేత్తి} \\
 & \natline{బుద్ధిః సా పార్థ సాత్త్వికీ ||}
\end{tabular}
\end{table}

\begin{table}[H]
\begin{tabular}{cl}
\textbf{18.31} & \natline{యయా ధర్మమధర్మం చ} \\
 & \natline{కార్యం చాకార్యమేవ చ |} \\
 & \natline{అయథావత్ప్రజానాతి} \\
 & \natline{బుద్ధిః సా పార్థ రాజసీ ||}
\end{tabular}
\end{table}

\begin{table}[H]
\begin{tabular}{cl}
\textbf{18.32} & \natline{అధర్మం ధర్మమితి యా} \\
 & \natline{మన్యతే తమసాఽఽవృతా |} \\
 & \natline{సర్వార్థాన్విపరీతాంశ్చ} \\
 & \natline{బుద్ధిః సా పార్థ తామసీ ||}
\end{tabular}
\end{table}

\begin{table}[H]
\begin{tabular}{cl}
\textbf{18.33} & \natline{ధృత్యా యయా ధారయతే} \\
 & \natline{మనః ప్రాణేన్ద్రియక్రియాః |} \\
 & \natline{యోగేనావ్యభిచారిణ్యా} \\
 & \natline{ధృతిః సా పార్థ సాత్త్వికీ ||}
\end{tabular}
\end{table}

\begin{table}[H]
\begin{tabular}{cl}
\textbf{18.34} & \natline{యయా తు ధర్మకామార్థాన్} \\
 & \natline{ధృత్యా ధారయతేఽర్జున |} \\
 & \natline{ప్రసఙ్గేన ఫలాకాఙ్క్షీ} \\
 & \natline{ధృతిః సా పార్థ రాజసీ ||}
\end{tabular}
\end{table}

\begin{table}[H]
\begin{tabular}{cl}
\textbf{18.35} & \natline{యయా స్వప్నం భయం శోకం} \\
 & \natline{విషాదం మదమేవ చ |} \\
 & \natline{న విముఞ్చతి దుర్మేధాః} \\
 & \natline{ధృతిః సా తామసీ మతా ||}
\end{tabular}
\end{table}

\begin{table}[H]
\begin{tabular}{cl}
\textbf{18.36} & \natline{సుఖం త్విదానీం త్రివిధం} \\
 & \natline{శృణు మే భరతర్షభ |} \\
 & \natline{అభ్యాసాద్రమతే యత్ర} \\
 & \natline{దుఃఖాన్తం చ నిగచ్ఛతి ||}
\end{tabular}
\end{table}

\begin{table}[H]
\begin{tabular}{cl}
\textbf{18.37} & \natline{యత్తదగ్రే విషమివ} \\
 & \natline{పరిణామేఽమృతోపమమ్ |} \\
 & \natline{తత్సుఖం సాత్త్వికం ప్రోక్తమ్} \\
 & \natline{ఆత్మబుద్ధిప్రసాదజమ్ ||}
\end{tabular}
\end{table}

\begin{table}[H]
\begin{tabular}{cl}
\textbf{18.38} & \natline{విషయేన్ద్రియసంయోగాత్} \\
 & \natline{యత్తదగ్రేఽమృతోపమమ్ |} \\
 & \natline{పరిణామే విషమివ} \\
 & \natline{తత్సుఖం రాజసం స్మృతమ్ ||}
\end{tabular}
\end{table}

\begin{table}[H]
\begin{tabular}{cl}
\textbf{18.39} & \natline{యదగ్రే చానుబన్ధే చ} \\
 & \natline{సుఖం మోహనమాత్మనః |} \\
 & \natline{నిద్రాలస్యప్రమాదోత్థం} \\
 & \natline{తత్తామసముదాహృతమ్ ||}
\end{tabular}
\end{table}

\begin{table}[H]
\begin{tabular}{cl}
\textbf{18.40} & \natline{న తదస్తి పృథివ్యాం వా} \\
 & \natline{దివి దేవేషు వా పునః |} \\
 & \natline{సత్త్వం ప్రకృతిజైర్ముక్తం} \\
 & \natline{యదేభిః స్యాత్త్రిభిర్గుణైః ||}
\end{tabular}
\end{table}

\begin{table}[H]
\begin{tabular}{cl}
\textbf{18.41} & \natline{బ్రాహ్మణక్షత్రియవిశాం} \\
 & \natline{శూద్రాణాం చ పరన్తప |} \\
 & \natline{కర్మాణి ప్రవిభక్తాని} \\
 & \natline{స్వభావప్రభవైర్గుణైః ||}
\end{tabular}
\end{table}

\begin{table}[H]
\begin{tabular}{cl}
\textbf{18.42} & \natline{శమో దమస్తపః శౌచం} \\
 & \natline{షాన్తిరార్జవమేవ చ |} \\
 & \natline{జ్ఞానం విజ్ఞానమాస్తిక్యం} \\
 & \natline{బ్రహ్మకర్మ స్వభావజమ్ ||}
\end{tabular}
\end{table}

\begin{table}[H]
\begin{tabular}{cl}
\textbf{18.43} & \natline{శౌర్యం తేజో ధృతిర్దాక్ష్యం} \\
 & \natline{యుద్ధే చాప్యపలాయనమ్ |} \\
 & \natline{దానమీశ్వరభావశ్చ} \\
 & \natline{క్షాత్రం కర్మ స్వభావజమ్ ||}
\end{tabular}
\end{table}

\begin{table}[H]
\begin{tabular}{cl}
\textbf{18.44} & \natline{కృషిగౌరక్ష్యవాణిజ్యం} \\
 & \natline{వైశ్యకర్మ స్వభావజమ్ |} \\
 & \natline{పరిచర్యాత్మకం కర్మ} \\
 & \natline{శూద్రస్యాపి స్వభావజమ్ ||}
\end{tabular}
\end{table}

\begin{table}[H]
\begin{tabular}{cl}
\textbf{18.45} & \natline{స్వే స్వే కర్మణ్యభిరతః} \\
 & \natline{సంసిద్ధిం లభతే నరః |} \\
 & \natline{స్వకర్మనిరతః సిద్ధిం} \\
 & \natline{యథా విన్దతి తచ్ఛృణు ||}
\end{tabular}
\end{table}

\begin{table}[H]
\begin{tabular}{cl}
\textbf{18.46} & \natline{యతః ప్రవృత్తిర్భూతానాం} \\
 & \natline{యేన సర్వమిదం తతమ్ |} \\
 & \natline{స్వకర్మణా తమభ్యర్చ్య} \\
 & \natline{సిద్ధిం విన్దతి మానవః ||}
\end{tabular}
\end{table}

\begin{table}[H]
\begin{tabular}{cl}
\textbf{18.47} & \natline{శ్రేయాన్స్వధర్మో విగుణః} \\
 & \natline{పరధర్మాత్స్వనుష్ఠితాత్ |} \\
 & \natline{స్వభావనియతం కర్మ} \\
 & \natline{కుర్వన్నాప్నోతి కిల్బిషమ్ ||}
\end{tabular}
\end{table}

\begin{table}[H]
\begin{tabular}{cl}
\textbf{18.48} & \natline{సహజం కర్మ కౌన్తేయ} \\
 & \natline{సదోషమపి న త్యజేత్ |} \\
 & \natline{సర్వారమ్భా హి దోషేణ} \\
 & \natline{ధూమేనాగ్నిరివావృతాః ||}
\end{tabular}
\end{table}

\begin{table}[H]
\begin{tabular}{cl}
\textbf{18.49} & \natline{అసక్తబుద్ధిః సర్వత్ర} \\
 & \natline{జితాత్మా విగతస్పృహః |} \\
 & \natline{నైష్కర్మ్యసిద్ధిం పరమాం} \\
 & \natline{సన్న్యాసేనాధిగచ్ఛతి ||}
\end{tabular}
\end{table}

\begin{table}[H]
\begin{tabular}{cl}
\textbf{18.50} & \natline{సిద్ధిం ప్రాప్తో యథా బ్రహ్మ} \\
 & \natline{తథాఽఽప్నోతి నిబోధ మే |} \\
 & \natline{సమాసేనైవ కౌన్తేయ} \\
 & \natline{నిష్ఠా జ్ఞానస్య యా పరా ||}
\end{tabular}
\end{table}

\begin{table}[H]
\begin{tabular}{cl}
\textbf{18.51} & \natline{బుద్ధ్యా విశుద్ధయా యుక్తః} \\
 & \natline{ధృత్యాఽఽత్మానం నియమ్య చ |} \\
 & \natline{శబ్దాదీన్విషయాంస్త్యక్త్వా} \\
 & \natline{రాగద్వేషౌ వ్యుదస్య చ ||}
\end{tabular}
\end{table}

\begin{table}[H]
\begin{tabular}{cl}
\textbf{18.52} & \natline{వివిక్తసేవీ లఘ్వాశీ} \\
 & \natline{యతవాక్కాయమానసః |} \\
 & \natline{ధ్యానయోగపరో నిత్యం} \\
 & \natline{వైరాగ్యం సముపాశ్రితః ||}
\end{tabular}
\end{table}

\begin{table}[H]
\begin{tabular}{cl}
\textbf{18.53} & \natline{అహఙ్కారం బలం దర్పం} \\
 & \natline{కామం క్రోధం పరిగ్రహమ్ |} \\
 & \natline{విముచ్య నిర్మమః శాన్తః} \\
 & \natline{బ్రహ్మభూయాయ కల్పతే ||}
\end{tabular}
\end{table}

\begin{table}[H]
\begin{tabular}{cl}
\textbf{18.54} & \natline{బ్రహ్మభూతః ప్రసన్నాత్మా} \\
 & \natline{న శోచతి న కాఙ్క్షతి |} \\
 & \natline{సమః సర్వేషు భూతేషు} \\
 & \natline{మద్భక్తిం లభతే పరామ్ ||}
\end{tabular}
\end{table}

\begin{table}[H]
\begin{tabular}{cl}
\textbf{18.55} & \natline{భక్త్యా మామభిజానాతి} \\
 & \natline{యావాన్యశ్చాస్మి తత్త్వతః |} \\
 & \natline{తతో మాం తత్త్వతో జ్ఞాత్వా} \\
 & \natline{విశతే తదనన్తరమ్ ||}
\end{tabular}
\end{table}

\begin{table}[H]
\begin{tabular}{cl}
\textbf{18.56} & \natline{సర్వకర్మాణ్యపి సదా} \\
 & \natline{కుర్వాణో మద్వ్యపాశ్రయః |} \\
 & \natline{మత్ప్రసాదాదవాప్నోతి} \\
 & \natline{శాశ్వతం పదమవ్యయమ్ ||}
\end{tabular}
\end{table}

\begin{table}[H]
\begin{tabular}{cl}
\textbf{18.57} & \natline{చేతసా సర్వకర్మాణి} \\
 & \natline{మయి సన్న్యస్య మత్పరః |} \\
 & \natline{బుద్ధియోగముపాశ్రిత్య} \\
 & \natline{మచ్చిత్తః సతతం భవ ||}
\end{tabular}
\end{table}

\begin{table}[H]
\begin{tabular}{cl}
\textbf{18.58} & \natline{మచ్చిత్తః సర్వదుర్గాణి} \\
 & \natline{మత్ప్రసాదాత్తరిష్యసి |} \\
 & \natline{అథ చేత్త్వమహఙ్కారాత్} \\
 & \natline{న శ్రోష్యసి వినఙ్క్ష్యసి ||}
\end{tabular}
\end{table}

\begin{table}[H]
\begin{tabular}{cl}
\textbf{18.59} & \natline{యదహఙ్కారమాశ్రిత్య} \\
 & \natline{న యోత్స్య ఇతి మన్యసే |} \\
 & \natline{మిథ్యైష వ్యవసాయస్తే} \\
 & \natline{ప్రకృతిస్త్వాం నియోక్ష్యతి ||}
\end{tabular}
\end{table}

\begin{table}[H]
\begin{tabular}{cl}
\textbf{18.60} & \natline{స్వభావజేన కౌన్తేయ} \\
 & \natline{నిబద్ధః స్వేన కర్మణా |} \\
 & \natline{కర్తుం నేచ్ఛసి యన్మోహాత్} \\
 & \natline{కరిష్యస్యవశోఽపి తత్ ||}
\end{tabular}
\end{table}

\begin{table}[H]
\begin{tabular}{cl}
\textbf{18.61} & \natline{ఈశ్వరః సర్వభూతానాం} \\
 & \natline{హృద్దేశేఽర్జున తిష్ఠతి |} \\
 & \natline{భ్రామయన్సర్వభూతాని} \\
 & \natline{యన్త్రారూఢాని మాయయా ||}
\end{tabular}
\end{table}

\begin{table}[H]
\begin{tabular}{cl}
\textbf{18.62} & \natline{తమేవ శరణం గచ్ఛ} \\
 & \natline{సర్వభావేన భారత |} \\
 & \natline{తత్ప్రసాదాత్పరాం శాన్తిం} \\
 & \natline{స్థానం ప్రాప్స్యసి శాశ్వతమ్ ||}
\end{tabular}
\end{table}

\begin{table}[H]
\begin{tabular}{cl}
\textbf{18.63} & \natline{ఇతి తే జ్ఞానమాఖ్యాతం} \\
 & \natline{గుహ్యాద్గుహ్యతరం మయా |} \\
 & \natline{విమృశ్యైతదశేషేణ} \\
 & \natline{యథేచ్ఛసి తథా కురు ||}
\end{tabular}
\end{table}

\begin{table}[H]
\begin{tabular}{cl}
\textbf{18.64} & \natline{సర్వగుహ్యతమం భూయః} \\
 & \natline{శృణు మే పరమం వచః |} \\
 & \natline{ఇష్టోఽసి మే దృఢమితి} \\
 & \natline{తతో వక్ష్యామి తే హితమ్ ||}
\end{tabular}
\end{table}

\begin{table}[H]
\begin{tabular}{cl}
\textbf{18.65} & \natline{మన్మనా భవ మద్భక్తః} \\
 & \natline{మద్యాజీ మాం నమస్కురు |} \\
 & \natline{మామేవైష్యసి సత్యం తే} \\
 & \natline{ప్రతిజానే ప్రియోఽసి మే ||}
\end{tabular}
\end{table}

\begin{table}[H]
\begin{tabular}{cl}
\textbf{18.66} & \natline{సర్వధర్మాన్పరిత్యజ్య} \\
 & \natline{మామేకం శరణం వ్రజ |} \\
 & \natline{అహం త్వా సర్వపాపేభ్యః} \\
 & \natline{మోక్షయిష్యామి మా శుచః ||}
\end{tabular}
\end{table}

\begin{table}[H]
\begin{tabular}{cl}
\textbf{18.67} & \natline{ఇదం తే నాతపస్కాయ} \\
 & \natline{నాభక్తాయ కదాచన |} \\
 & \natline{న చాశుశ్రూషవే వాచ్యం} \\
 & \natline{న చ మాం యోఽభ్యసూయతి ||}
\end{tabular}
\end{table}

\begin{table}[H]
\begin{tabular}{cl}
\textbf{18.68} & \natline{య ఇమం పరమం గుహ్యం} \\
 & \natline{మద్భక్తేష్వభిధాస్యతి |} \\
 & \natline{భక్తిం మయి పరాం కృత్వా} \\
 & \natline{మామేవైష్యత్యసంశయః ||}
\end{tabular}
\end{table}

\begin{table}[H]
\begin{tabular}{cl}
\textbf{18.69} & \natline{న చ తస్మాన్మనుష్యేషు} \\
 & \natline{కశ్చిన్మే ప్రియకృత్తమః |} \\
 & \natline{భవితా న చ మే తస్మాత్} \\
 & \natline{అన్యః ప్రియతరో భువి ||}
\end{tabular}
\end{table}

\begin{table}[H]
\begin{tabular}{cl}
\textbf{18.70} & \natline{అధ్యేష్యతే చ య ఇమం} \\
 & \natline{ధర్మ్యం సంవాదమావయోః |} \\
 & \natline{జ్ఞానయజ్ఞేన తేనాహమ్} \\
 & \natline{ఇష్టః స్యామితి మే మతిః ||}
\end{tabular}
\end{table}

\begin{table}[H]
\begin{tabular}{cl}
\textbf{18.71} & \natline{శ్రద్ధావాననసూయశ్చ} \\
 & \natline{శృణుయాదపి యో నరః |} \\
 & \natline{సోఽపి ముక్తః శుభాల్లోకాన్} \\
 & \natline{ప్రాప్నుయాత్పుణ్యకర్మణామ్ ||}
\end{tabular}
\end{table}

\begin{table}[H]
\begin{tabular}{cl}
\textbf{18.72} & \natline{కచ్చిదేతచ్ఛ్రుతం పార్థ} \\
 & \natline{త్వయైకాగ్రేణ చేతసా |} \\
 & \natline{కచ్చిదజ్ఞానసమ్మోహః} \\
 & \natline{ప్రనష్టస్తే ధనఞ్జయ ||}
\end{tabular}
\end{table}

\begin{table}[H]
\begin{tabular}{cl}
\textbf{18.73} & \natline{అర్జున ఉవాచ} \\
 & \natline{నష్టో మోహః స్మృతిర్లబ్ధా} \\
 & \natline{త్వత్ప్రసాదాన్మయాఽచ్యుత |} \\
 & \natline{స్థితోఽస్మి గతసన్దేహః} \\
 & \natline{కరిష్యే వచనం తవ ||}
\end{tabular}
\end{table}

\begin{table}[H]
\begin{tabular}{cl}
\textbf{18.74} & \natline{సఞ్జయ ఉవాచ} \\
 & \natline{ఇత్యహం వాసుదేవస్య} \\
 & \natline{పార్థస్య చ మహాత్మనః |} \\
 & \natline{సంవాదమిమమశ్రౌషమ్} \\
 & \natline{అద్భుతం రోమహర్షణమ్ ||}
\end{tabular}
\end{table}

\begin{table}[H]
\begin{tabular}{cl}
\textbf{18.75} & \natline{వ్యాసప్రసాదాచ్ఛ్రుతవాన్} \\
 & \natline{ఇమం గుహ్యతమం పరమ్ |} \\
 & \natline{యోగం యోగేశ్వరాత్కృష్ణాత్} \\
 & \natline{సాక్షాత్కథయతః స్వయమ్ ||}
\end{tabular}
\end{table}

\begin{table}[H]
\begin{tabular}{cl}
\textbf{18.76} & \natline{రాజన్ సంస్మృత్య సంస్మృత్య} \\
 & \natline{సంవాదమిమమద్భుతమ్ |} \\
 & \natline{కేశవార్జునయోః పుణ్యం} \\
 & \natline{హృష్యామి చ ముహుర్ముహుః ||}
\end{tabular}
\end{table}

\begin{table}[H]
\begin{tabular}{cl}
\textbf{18.77} & \natline{తచ్చ సంస్మృత్య సంస్మృత్య} \\
 & \natline{రూపమత్యద్భుతం హరేః |} \\
 & \natline{విస్మయో మే మహాన్రాజన్} \\
 & \natline{హృష్యామి చ పునః పునః ||}
\end{tabular}
\end{table}

\begin{table}[H]
\begin{tabular}{cl}
\textbf{18.78} & \natline{యత్ర యోగేశ్వరః కృష్ణః} \\
 & \natline{యత్ర పార్థో ధనుర్ధరః |} \\
 & \natline{తత్ర శ్రీర్విజయో భూతిః} \\
 & \natline{ధ్రువా నీతిర్మతిర్మమ ||}
\end{tabular}
\end{table}

