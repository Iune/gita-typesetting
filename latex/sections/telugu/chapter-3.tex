\begin{table}[H]
\begin{tabular}{cl}
\textbf{3.0} & \natline{ఓం శ్రీ పరమాత్మనే నమః} \\
 & \natline{అథ తృతీయోఽధ్యాయః} \\
 & \natline{కర్మయోగః}
\end{tabular}
\end{table}

\begin{table}[H]
\begin{tabular}{cl}
\textbf{3.1} & \natline{అర్జున ఉవాచ} \\
 & \natline{జ్యాయసీ చేత్కర్మణస్తే} \\
 & \natline{మతా బుద్ధిర్జనార్దన |} \\
 & \natline{తత్కిం కర్మణి ఘోరే మాం} \\
 & \natline{నియోజయసి కేశవ ||}
\end{tabular}
\end{table}

\begin{table}[H]
\begin{tabular}{cl}
\textbf{3.2} & \natline{వ్యామిశ్రేణేవ వాక్యేన} \\
 & \natline{బుద్ధిం మోహయసీవ మే |} \\
 & \natline{తదేకం వద నిశ్చిత్య} \\
 & \natline{యేన శ్రేయోఽహమాప్నుయామ్ ||}
\end{tabular}
\end{table}

\begin{table}[H]
\begin{tabular}{cl}
\textbf{3.3} & \natline{స్రీ భగవానువాచ} \\
 & \natline{లోకేఽస్మిన్ద్వివిధా నిష్ఠా} \\
 & \natline{పురా ప్రోక్తా మయాఽనఘ |} \\
 & \natline{జ్ఞానయోగేన సాఙ్ఖ్యానాం} \\
 & \natline{కర్మయోగేన యోగినామ్ ||}
\end{tabular}
\end{table}

\begin{table}[H]
\begin{tabular}{cl}
\textbf{3.4} & \natline{న కర్మణామనారమ్భాత్} \\
 & \natline{నైష్కర్మ్యం పురుషోఽశ్నుతే |} \\
 & \natline{న చ సన్న్యసనాదేవ} \\
 & \natline{సిద్ధిం సమధిగచ్ఛతి ||}
\end{tabular}
\end{table}

\begin{table}[H]
\begin{tabular}{cl}
\textbf{3.5} & \natline{న హి కశ్చిత్క్షణమపి} \\
 & \natline{జాతు తిష్ఠత్యకర్మకృత్ |} \\
 & \natline{కార్యతే హ్యవశః కర్మ} \\
 & \natline{సర్వః ప్రకృతిజైర్గుణైః ||}
\end{tabular}
\end{table}

\begin{table}[H]
\begin{tabular}{cl}
\textbf{3.6} & \natline{కర్మేన్ద్రియాణి సంయమ్య} \\
 & \natline{య ఆస్తే మనసా స్మరన్ |} \\
 & \natline{ఇన్ద్రియార్థాన్విమూఢాత్మా} \\
 & \natline{మిథ్యాచారః స ఉచ్యతే ||}
\end{tabular}
\end{table}

\begin{table}[H]
\begin{tabular}{cl}
\textbf{3.7} & \natline{యస్త్విన్ద్రియాణి మనసా} \\
 & \natline{నియమ్యారభతేఽర్జున |} \\
 & \natline{కర్మేన్ద్రియైః కర్మయోగమ్} \\
 & \natline{అసక్తః స విశిష్యతే ||}
\end{tabular}
\end{table}

\begin{table}[H]
\begin{tabular}{cl}
\textbf{3.8} & \natline{నియతం కురు కర్మ త్వం} \\
 & \natline{కర్మ జ్యాయో హ్యకర్మణః |} \\
 & \natline{శరీరయాత్రాఽపి చ తే} \\
 & \natline{న ప్రసిద్ధ్యేదకర్మణః ||}
\end{tabular}
\end{table}

\begin{table}[H]
\begin{tabular}{cl}
\textbf{3.9} & \natline{యజ్ఞార్థాత్కర్మణోఽన్యత్ర} \\
 & \natline{లోకోఽయం కర్మబన్ధనః |} \\
 & \natline{తదర్థం కర్మ కౌన్తేయ} \\
 & \natline{ముక్తసఙ్గః సమాచర ||}
\end{tabular}
\end{table}

\begin{table}[H]
\begin{tabular}{cl}
\textbf{3.10} & \natline{సహయజ్ఞాః ప్రజాః సృష్ట్వా} \\
 & \natline{పురోవాచ ప్రజాపతిః |} \\
 & \natline{అనేన ప్రసవిష్యధ్వం} \\
 & \natline{ఏష వోఽస్త్విష్టకామధుక్ ||}
\end{tabular}
\end{table}

\begin{table}[H]
\begin{tabular}{cl}
\textbf{3.11} & \natline{దేవాన్భావయతాఽనేన} \\
 & \natline{తే దేవా భావయన్తు వః |} \\
 & \natline{పరస్పరం భావయన్తః} \\
 & \natline{శ్రేయః పరమవాప్స్యథ ||}
\end{tabular}
\end{table}

\begin{table}[H]
\begin{tabular}{cl}
\textbf{3.12} & \natline{ఇష్టాన్భోగాన్హి వో దేవాః} \\
 & \natline{దాస్యన్తే యజ్ఞభావితాః |} \\
 & \natline{తైర్దత్తానప్రదాయైభ్యః} \\
 & \natline{యో భుఙ్క్తే స్తేన ఏవ సః ||}
\end{tabular}
\end{table}

\begin{table}[H]
\begin{tabular}{cl}
\textbf{3.13} & \natline{యజ్ఞశిష్టాశినః సన్తః} \\
 & \natline{ముచ్యన్తే సర్వకిల్బిషైః |} \\
 & \natline{భుఞ్జతే తే త్వఘం పాపాః} \\
 & \natline{యే పచన్త్యాత్మకారణాత్ ||}
\end{tabular}
\end{table}

\begin{table}[H]
\begin{tabular}{cl}
\textbf{3.14} & \natline{అన్నాద్భవన్తి భూతాని} \\
 & \natline{పర్జన్యాదన్నసమ్భవః |} \\
 & \natline{యజ్ఞాద్భవతి పర్జన్యః} \\
 & \natline{యజ్ఞః కర్మసముద్భవః ||}
\end{tabular}
\end{table}

\begin{table}[H]
\begin{tabular}{cl}
\textbf{3.15} & \natline{కర్మ బ్రహ్మోద్భవం విద్ధి} \\
 & \natline{బ్రహ్మాక్షరసముద్భవమ్ |} \\
 & \natline{తస్మాత్సర్వగతం బ్రహ్మ} \\
 & \natline{నిత్యం యజ్ఞే ప్రతిష్ఠితమ్ ||}
\end{tabular}
\end{table}

\begin{table}[H]
\begin{tabular}{cl}
\textbf{3.16} & \natline{ఏవం ప్రవర్తితం చక్రం} \\
 & \natline{నానువర్తయతీహ యః |} \\
 & \natline{అఘాయురిన్ద్రియారామః} \\
 & \natline{మోఘం పార్థ స జీవతి ||}
\end{tabular}
\end{table}

\begin{table}[H]
\begin{tabular}{cl}
\textbf{3.17} & \natline{యస్త్వాత్మరతిరేవ స్యాత్} \\
 & \natline{ఆత్మతృప్తశ్చ మానవః |} \\
 & \natline{ఆత్మన్యేవ చ సన్తుష్టః} \\
 & \natline{తస్య కార్యం న విద్యతే ||}
\end{tabular}
\end{table}

\begin{table}[H]
\begin{tabular}{cl}
\textbf{3.18} & \natline{నైవ తస్య కృతేనార్థః} \\
 & \natline{నాకృతేనేహ కశ్చన |} \\
 & \natline{న చాస్య సర్వభూతేషు} \\
 & \natline{కశ్చిదర్థవ్యపాశ్రయః ||}
\end{tabular}
\end{table}

\begin{table}[H]
\begin{tabular}{cl}
\textbf{3.19} & \natline{తస్మాదసక్తః సతతం} \\
 & \natline{కార్యం కర్మ సమాచర |} \\
 & \natline{అసక్తో హ్యాచరన్కర్మ} \\
 & \natline{పరమాప్నోతి పూరుషః ||}
\end{tabular}
\end{table}

\begin{table}[H]
\begin{tabular}{cl}
\textbf{3.20} & \natline{కర్మణైవ హి సంసిద్ధిం} \\
 & \natline{ఆస్థితా జనకాదయః |} \\
 & \natline{లోకసఙ్గ్రహమేవాపి} \\
 & \natline{సమ్పశ్యన్కర్తుమర్హసి ||}
\end{tabular}
\end{table}

