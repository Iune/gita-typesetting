\begin{table}[H]
\begin{tabular}{cl}
 & \natline{శ్రీ పరమాత్మనే నమః} \\
 & \natline{అథ ప్రథమోఽధ్యాయః} \\
 & \natline{అర్జునవిషాదయోగః}
\end{tabular}
\end{table}

\begin{table}[H]
\begin{tabular}{cl}
\textbf{1.1} & \natline{ధృతరాష్ట్ర ఉవాచ} \\
 & \natline{ధర్మక్షేత్రే కురుక్షేత్రే} \\
 & \natline{సమవేతా యుయుత్సవః |} \\
 & \natline{మామకాః పాణ్డవాశ్చైవ} \\
 & \natline{కిమకుర్వత సఞ్జయ ||}
\end{tabular}
\end{table}

\begin{table}[H]
\begin{tabular}{cl}
\textbf{1.2} & \natline{సఞ్జయ ఉవాచ} \\
 & \natline{దృష్ట్వా తు పాణ్డవానీకం} \\
 & \natline{వ్యూఢం దుర్యోధనస్తదా |} \\
 & \natline{ఆచార్యముపసఙ్గమ్య} \\
 & \natline{రాజా వచనమబ్రవీత్ ||}
\end{tabular}
\end{table}

\begin{table}[H]
\begin{tabular}{cl}
\textbf{1.3} & \natline{పశ్యైతాం పాణ్డుపుత్రాణామ్} \\
 & \natline{ఆచార్య మహతీం చమూమ్ |} \\
 & \natline{వ్యూఢాం ద్రుపదపుత్రేణ} \\
 & \natline{తవ శిష్యేణ ధీమతా ||}
\end{tabular}
\end{table}

\begin{table}[H]
\begin{tabular}{cl}
\textbf{1.4} & \natline{అత్ర శూరా మహేష్వాసాః} \\
 & \natline{భీమార్జునసమా యుధి |} \\
 & \natline{యుయుధానో విరాటశ్చ} \\
 & \natline{ద్రుపదశ్చ మహారథః ||}
\end{tabular}
\end{table}

\begin{table}[H]
\begin{tabular}{cl}
\textbf{1.5} & \natline{ధృష్టకేతుశ్చేకితానః} \\
 & \natline{కాశిరాజశ్చ వీర్యవాన్ |} \\
 & \natline{పురుజిత్కున్తిభోజశ్చ} \\
 & \natline{శైబ్యశ్చ నరపుఙ్గవః ||}
\end{tabular}
\end{table}

\begin{table}[H]
\begin{tabular}{cl}
\textbf{1.6} & \natline{యుధామన్యుశ్చ విక్రాన్తః} \\
 & \natline{ఉత్తమౌజాశ్చ వీర్యవాన్ |} \\
 & \natline{సౌభద్రో ద్రౌపదేయాశ్చ} \\
 & \natline{సర్వ ఏవ మహారథాః ||}
\end{tabular}
\end{table}

\begin{table}[H]
\begin{tabular}{cl}
\textbf{1.7} & \natline{అస్మాకం తు విశిష్టా యే} \\
 & \natline{తాన్నిబోధ ద్విజోత్తమ |} \\
 & \natline{నాయకా మమ సైన్యస్య} \\
 & \natline{సఞ్జ్ఞార్థం తాన్బ్రవీమి తే ||}
\end{tabular}
\end{table}

\begin{table}[H]
\begin{tabular}{cl}
\textbf{1.8} & \natline{భవాన్భీష్మశ్చ కర్ణశ్చ} \\
 & \natline{కృపశ్చ సమితిఞ్జయః |} \\
 & \natline{అశ్వత్థామా వికర్ణశ్చ} \\
 & \natline{సౌమదత్తిస్తథైవ చ ||}
\end{tabular}
\end{table}

\begin{table}[H]
\begin{tabular}{cl}
\textbf{1.9} & \natline{అన్యే చ బహవః శూరాః} \\
 & \natline{మదర్థే త్యక్తజీవితాః |} \\
 & \natline{నానాశస్త్రప్రహరణాః} \\
 & \natline{సర్వే యుద్ధవిశారదాః ||}
\end{tabular}
\end{table}

\begin{table}[H]
\begin{tabular}{cl}
\textbf{1.10} & \natline{అపర్యాప్తం తదస్మాకం} \\
 & \natline{బలం భీష్మాభిరక్షితమ్ |} \\
 & \natline{పర్యాప్తం త్విదమేతేషాం} \\
 & \natline{బలం భీమాభిరక్షితమ్ ||}
\end{tabular}
\end{table}

\begin{table}[H]
\begin{tabular}{cl}
\textbf{1.11} & \natline{అయనేషు చ సర్వేషు} \\
 & \natline{యథాభాగమవస్థితాః |} \\
 & \natline{భీష్మమేవాభిరక్షన్తు} \\
 & \natline{భవన్తః సర్వ ఏవ హి ||}
\end{tabular}
\end{table}

\begin{table}[H]
\begin{tabular}{cl}
\textbf{1.12} & \natline{తస్య సఞ్జనయన్హర్షం} \\
 & \natline{కురువృద్ధః పితామహః |} \\
 & \natline{సింహనాదం వినద్యోచ్చైః} \\
 & \natline{శఙ్ఖం దధ్మౌ ప్రతాపవాన్ ||}
\end{tabular}
\end{table}

\begin{table}[H]
\begin{tabular}{cl}
\textbf{1.13} & \natline{తతః శఙ్ఖాశ్చ భేర్యశ్చ} \\
 & \natline{పణవానకగోముఖాః |} \\
 & \natline{సహసైవాభ్యహన్యన్త} \\
 & \natline{స శబ్దస్తుములోఽభవత్ ||}
\end{tabular}
\end{table}

\begin{table}[H]
\begin{tabular}{cl}
\textbf{1.14} & \natline{తతః శ్వేతైర్హయైర్యుక్తే} \\
 & \natline{మహతి స్యన్దనే స్థితౌ |} \\
 & \natline{మాధవః పాణ్డవశ్చైవ} \\
 & \natline{దివ్యౌ శఙ్ఖౌ ప్రదధ్మతుః ||}
\end{tabular}
\end{table}

\begin{table}[H]
\begin{tabular}{cl}
\textbf{1.15} & \natline{పాఞ్చజన్యం హృషీకేశః} \\
 & \natline{దేవదత్తం ధనఞ్జయః |} \\
 & \natline{పౌణ్డ్రం దధ్మౌ మహాశఙ్ఖం} \\
 & \natline{భీమకర్మా వృకోదరః ||}
\end{tabular}
\end{table}

\begin{table}[H]
\begin{tabular}{cl}
\textbf{1.16} & \natline{అనన్తవిజయం రాజా} \\
 & \natline{కున్తీపుత్రో యుధిష్ఠిరః |} \\
 & \natline{నకులః సహదేవశ్చ} \\
 & \natline{సుఘోషమణిపుష్పకౌ ||}
\end{tabular}
\end{table}

\begin{table}[H]
\begin{tabular}{cl}
\textbf{1.17} & \natline{కాశ్యశ్చ పరమేష్వాసః} \\
 & \natline{శిఖణ్డీ చ మహారథః |} \\
 & \natline{ధృష్టద్యుమ్నో విరాటశ్చ} \\
 & \natline{సాత్యకిశ్చాపరాజితః ||}
\end{tabular}
\end{table}

\begin{table}[H]
\begin{tabular}{cl}
\textbf{1.18} & \natline{ద్రుపదో ద్రౌపదేయాశ్చ} \\
 & \natline{సర్వశః పృథివీపతే |} \\
 & \natline{సోఉభద్రశ్చ మహాబాహుః} \\
 & \natline{శఙ్ఖాన్దధ్ముః పృథక్పృథక్ ||}
\end{tabular}
\end{table}

\begin{table}[H]
\begin{tabular}{cl}
\textbf{1.19} & \natline{స ఘోషో ధార్తరాష్ట్రాణాం} \\
 & \natline{హృదయాని వ్యదారయత్ |} \\
 & \natline{నభశ్చ పృథివీం చైవ} \\
 & \natline{తుములో వ్యనునాదయన్ ||}
\end{tabular}
\end{table}

\begin{table}[H]
\begin{tabular}{cl}
\textbf{1.20} & \natline{అథ వ్యవస్థితాన్దృష్ట్వా} \\
 & \natline{ధార్తరాష్ట్రాన్ కపిధ్వజః |} \\
 & \natline{ప్రవృత్తే శస్త్రసమ్పాతే} \\
 & \natline{ధనురుద్యమ్య పాణ్డవః ||}
\end{tabular}
\end{table}

\begin{table}[H]
\begin{tabular}{cl}
\textbf{1.21} & \natline{హృషీకేశం తదా వాక్యమ్} \\
 & \natline{ఇదమాహ మహీపతే} \\
 & \natline{అర్జున ఉవాచ} \\
 & \natline{సేనయోరుభయోర్మధ్యే} \\
 & \natline{రథం స్థాపయ మేఽచ్యుత}
\end{tabular}
\end{table}

\begin{table}[H]
\begin{tabular}{cl}
\textbf{1.22} & \natline{యావదేతాన్నిరీక్షేఽహం} \\
 & \natline{యోద్ధుకామానవస్థితాన్ |} \\
 & \natline{కైర్మయా సహ యోద్ధవ్యమ్} \\
 & \natline{అస్మిన్ రణసముద్యమే ||}
\end{tabular}
\end{table}

\begin{table}[H]
\begin{tabular}{cl}
\textbf{1.23} & \natline{యోత్స్యమానానవేక్షేఽహం} \\
 & \natline{య ఏతేఽత్ర సమాగతాః |} \\
 & \natline{ధార్తరాష్ట్రస్యదుర్బుద్ధేః} \\
 & \natline{యుద్ధే ప్రియచికీర్షవః ||}
\end{tabular}
\end{table}

\begin{table}[H]
\begin{tabular}{cl}
\textbf{1.24} & \natline{సఞ్జయ ఉవాచ} \\
 & \natline{ఏవముక్తో హృషీకేశః} \\
 & \natline{గుడాకేశేన భారత |} \\
 & \natline{సేనయోరుభయోర్మధ్యే} \\
 & \natline{స్థాపయిత్వా రథోత్తమమ్ ||}
\end{tabular}
\end{table}

\begin{table}[H]
\begin{tabular}{cl}
\textbf{1.25} & \natline{భీష్మద్రోణప్రముఖతః} \\
 & \natline{సర్వేషాం చ మహీక్షితామ్ |} \\
 & \natline{ఉవాచ పార్థ పశ్యైతాన్} \\
 & \natline{సమవేతాన్కురూనితి ||}
\end{tabular}
\end{table}

\begin{table}[H]
\begin{tabular}{cl}
\textbf{1.26} & \natline{తత్రాపశ్యత్స్థితాన్పార్థః} \\
 & \natline{పితౄనథ పితామహాన్ |} \\
 & \natline{ఆచార్యాన్మాతులాన్భ్రాతౄన్} \\
 & \natline{పుత్రాన్పౌత్రాన్సఖీంస్తథా ||}
\end{tabular}
\end{table}

\begin{table}[H]
\begin{tabular}{cl}
\textbf{1.27} & \natline{శ్వశురాన్సుహృదశ్చైవ} \\
 & \natline{సేనయోరుభయోరపి |} \\
 & \natline{తాన్సమీక్ష్య స కౌన్తేయః} \\
 & \natline{సర్వాన్బన్ధూనవస్థితాన్ ||}
\end{tabular}
\end{table}

\begin{table}[H]
\begin{tabular}{cl}
\textbf{1.28} & \natline{కృపయా పరయాఽఽవిష్టః} \\
 & \natline{విషీదన్నిదమబ్రవీత్} \\
 & \natline{అర్జున ఉవాచ |} \\
 & \natline{దృష్ట్వేమం స్వజనం కృష్ణ} \\
 & \natline{యుయుత్సుం సముపస్థితమ్ ||}
\end{tabular}
\end{table}

\begin{table}[H]
\begin{tabular}{cl}
\textbf{1.29} & \natline{సీదన్తి మమ గాత్రాణి} \\
 & \natline{ముఖం చ పరిశుష్యతి |} \\
 & \natline{వేపథుశ్చ శరీరే మే} \\
 & \natline{రోమహర్షశ్చ జాయతే ||}
\end{tabular}
\end{table}

\begin{table}[H]
\begin{tabular}{cl}
\textbf{1.30} & \natline{గాణ్డీవం స్రంసతే హస్తాత్} \\
 & \natline{త్వక్చైవ పరిదహ్యతే |} \\
 & \natline{న చ శక్నోమ్యవస్థాతుం} \\
 & \natline{భ్రమతీవ చ మే మనః ||}
\end{tabular}
\end{table}

\begin{table}[H]
\begin{tabular}{cl}
\textbf{1.31} & \natline{నిమిత్తాని చ పశ్యామి} \\
 & \natline{విపరీతాని కేశవ |} \\
 & \natline{న చ శ్రేయోఽనుపశ్యామి} \\
 & \natline{హత్వా స్వజనమాహవే ||}
\end{tabular}
\end{table}

\begin{table}[H]
\begin{tabular}{cl}
\textbf{1.32} & \natline{న కాఙ్క్షే విజయం కృష్ణ} \\
 & \natline{న చ రాజ్యం సుఖాని చ |} \\
 & \natline{కిం నో రాజ్యేన గోవిన్ద} \\
 & \natline{కిం భోగైర్జీవితేన వా ||}
\end{tabular}
\end{table}

\begin{table}[H]
\begin{tabular}{cl}
\textbf{1.33} & \natline{యేషామర్థే కాఙ్క్షితం నః} \\
 & \natline{రాజ్యం భోగాః సుఖాని చ |} \\
 & \natline{త ఇమేఽవస్థితా యుద్ధే} \\
 & \natline{ప్రాణాంస్త్యక్త్వా ధనాని చ ||}
\end{tabular}
\end{table}

\begin{table}[H]
\begin{tabular}{cl}
\textbf{1.34} & \natline{ఆచార్యాః పితరః పుత్రాః} \\
 & \natline{తథైవ చ పితామహాః |} \\
 & \natline{మాతులాః శ్వశురాః పౌత్రాః} \\
 & \natline{శ్యాలాః సమ్బన్ధినస్తథా ||}
\end{tabular}
\end{table}

\begin{table}[H]
\begin{tabular}{cl}
\textbf{1.35} & \natline{ఏతాన్న హన్తుమిచ్చామి} \\
 & \natline{ఘ్నతోఽపి మధుసూదన |} \\
 & \natline{అపి త్రైలోక్యరాజ్యస్య} \\
 & \natline{హేతోః కిం ను మహీకృతే ||}
\end{tabular}
\end{table}

\begin{table}[H]
\begin{tabular}{cl}
\textbf{1.36} & \natline{నిహత్య ధార్తరాష్ట్రాన్నః} \\
 & \natline{కా ప్రీతిః స్యాజ్జనార్దన |} \\
 & \natline{పాపమేవాశ్రయేదస్మాన్} \\
 & \natline{హత్వైతానాతతాయినః ||}
\end{tabular}
\end{table}

\begin{table}[H]
\begin{tabular}{cl}
\textbf{1.37} & \natline{తస్మాన్నార్హా వయం హన్తుం} \\
 & \natline{ధార్తరాష్ట్రాన్స్వబాన్ధవాన్ |} \\
 & \natline{స్వజనం హి కథం హత్వా} \\
 & \natline{సుఖినః స్యామ మాధవ ||}
\end{tabular}
\end{table}

\begin{table}[H]
\begin{tabular}{cl}
\textbf{1.38} & \natline{యద్యప్యేతే న పశ్యన్తి} \\
 & \natline{లోభోపహతచేతసః |} \\
 & \natline{కులక్షయకృతం దోషం} \\
 & \natline{మిత్రద్రోహే చ పాతకమ్ ||}
\end{tabular}
\end{table}

\begin{table}[H]
\begin{tabular}{cl}
\textbf{1.39} & \natline{కథం న జ్ఞేయమస్మాభిః} \\
 & \natline{పాపాదస్మాన్నివర్తితుమ్ |} \\
 & \natline{కులక్షయకృతం దోషం} \\
 & \natline{ప్రపశ్యద్భిర్జనార్దన ||}
\end{tabular}
\end{table}

\begin{table}[H]
\begin{tabular}{cl}
\textbf{1.40} & \natline{కులక్షయే ప్రణశ్యన్తి} \\
 & \natline{కులధర్మాః సనాతనాః |} \\
 & \natline{ధర్మే నష్టే కులం కృత్స్నమ్} \\
 & \natline{అధర్మోఽభిభవత్యుత ||}
\end{tabular}
\end{table}

\begin{table}[H]
\begin{tabular}{cl}
\textbf{1.41} & \natline{అధర్మాభిభవాత్కృష్ణ} \\
 & \natline{ప్రదుష్యన్తి కులస్త్రియః |} \\
 & \natline{స్త్రీషు దుష్టాసు వార్ష్ణేయ} \\
 & \natline{జాయతే వర్ణసఙ్కరః ||}
\end{tabular}
\end{table}

\begin{table}[H]
\begin{tabular}{cl}
\textbf{1.42} & \natline{సఙ్కరో నరకాయైవ} \\
 & \natline{కులఘ్నానాం కులస్య చ |} \\
 & \natline{పతన్తి పితరో హ్యేషాం} \\
 & \natline{లుప్తపిణ్డోదకక్రియాః ||}
\end{tabular}
\end{table}

\begin{table}[H]
\begin{tabular}{cl}
\textbf{1.43} & \natline{దోషైరేతైః కులఘ్నానాం} \\
 & \natline{వర్ణసఙ్కరకారకైః |} \\
 & \natline{ఉత్సాద్యన్తే జాతిధర్మాః} \\
 & \natline{కులధర్మాశ్చ శాశ్వతాః ||}
\end{tabular}
\end{table}

\begin{table}[H]
\begin{tabular}{cl}
\textbf{1.44} & \natline{ఉత్సన్నకులధర్మాణాం} \\
 & \natline{మనుష్యాణాం జనార్దన |} \\
 & \natline{నరకేఽనియతం వాసః} \\
 & \natline{భవతీత్యనుశుశ్రుమ ||}
\end{tabular}
\end{table}

\begin{table}[H]
\begin{tabular}{cl}
\textbf{1.45} & \natline{అహో బత మహత్పాపం} \\
 & \natline{కర్తుం వ్యవసితా వయమ్ |} \\
 & \natline{యద్రాజ్యసుఖలోభేన} \\
 & \natline{హన్తుం స్వజనముద్యతాః ||}
\end{tabular}
\end{table}

\begin{table}[H]
\begin{tabular}{cl}
\textbf{1.46} & \natline{యది మామప్రతీకారమ్} \\
 & \natline{అశస్త్రం శస్త్రపాణయః |} \\
 & \natline{ధార్తరాష్ట్రా రణే హన్యుః} \\
 & \natline{తన్మే క్షేమతరం భవేత్ ||}
\end{tabular}
\end{table}

\begin{table}[H]
\begin{tabular}{cl}
\textbf{1.47} & \natline{సఞ్జయ ఉవాచ} \\
 & \natline{ఏవముక్త్వాఽర్జునః సఙ్ఖ్యే} \\
 & \natline{రథోపస్థ ఉపావిశత్ |} \\
 & \natline{విసృజ్య సశరం చాపం} \\
 & \natline{శోకసంవిగ్నమానసః ||}
\end{tabular}
\end{table}

\begin{table}[H]
\begin{tabular}{cl}
 & \natline{శ్రీమద్భగవద్గీతాసు ఉపనిషత్సు} \\
 & \natline{బ్రహ్మవిద్యాయాం యోగశాస్త్రే} \\
 & \natline{శ్రీకృష్ణార్జున సంవాదే} \\
 & \natline{అర్జునవిషాదయోగోనామ} \\
 & \natline{ప్రథమోధ్యాయః}
\end{tabular}
\end{table}

