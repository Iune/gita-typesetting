\begin{table}[H]
\begin{tabular}{cl}
 & \natline{శ్రీ పరమాత్మనే నమః} \\
 & \natline{అథ చతుర్దశోఽధ్యాయః} \\
 & \natline{గుణత్రయవిభాగయోగః}
\end{tabular}
\end{table}

\begin{table}[H]
\begin{tabular}{cl}
\textbf{14.1} & \natline{శ్రీ భగవానువాచ} \\
 & \natline{పరం భూయః ప్రవక్ష్యామి} \\
 & \natline{జ్ఞానానాం జ్ఞానముత్తమమ్ |} \\
 & \natline{యజ్జ్ఞాత్వా మునయః సర్వే} \\
 & \natline{పరాం సిద్ధిమితో గతాః ||}
\end{tabular}
\end{table}

\begin{table}[H]
\begin{tabular}{cl}
\textbf{14.2} & \natline{ఇదం జ్ఞానముపాశ్రిత్య} \\
 & \natline{మమ సాధర్మ్యమాగతాః |} \\
 & \natline{సర్గేఽపి నోపజాయన్తే} \\
 & \natline{ప్రలయే న వ్యథన్తి చ ||}
\end{tabular}
\end{table}

\begin{table}[H]
\begin{tabular}{cl}
\textbf{14.3} & \natline{మమ యోనిర్మహద్బ్రహ్మ} \\
 & \natline{తస్మిన్గర్భం దధామ్యహమ్ |} \\
 & \natline{సమ్భవః సర్వభూతానాం} \\
 & \natline{తతో భవతి భారత ||}
\end{tabular}
\end{table}

\begin{table}[H]
\begin{tabular}{cl}
\textbf{14.4} & \natline{సర్వయోనిషు కౌన్తేయ} \\
 & \natline{మూర్తయః సమ్భవన్తి యాః |} \\
 & \natline{తాసాం బ్రహ్మ మహద్యోనిః} \\
 & \natline{అహం బీజప్రదః పితా ||}
\end{tabular}
\end{table}

\begin{table}[H]
\begin{tabular}{cl}
\textbf{14.5} & \natline{సత్త్వం రజస్తమ ఇతి} \\
 & \natline{గుణాః ప్రకృతిసమ్భవాః |} \\
 & \natline{నిబధ్నన్తి మహాబాహో} \\
 & \natline{దేహే దేహినమవ్యయమ్ ||}
\end{tabular}
\end{table}

\begin{table}[H]
\begin{tabular}{cl}
\textbf{14.6} & \natline{తత్ర సత్త్వం నిర్మలత్వాత్} \\
 & \natline{ప్రకాశకమనామయమ్ |} \\
 & \natline{సుఖసఙ్గేన బధ్నాతి} \\
 & \natline{జ్ఞానసఙ్గేన చానఘ ||}
\end{tabular}
\end{table}

\begin{table}[H]
\begin{tabular}{cl}
\textbf{14.7} & \natline{రజో రాగాత్మకం విద్ధి} \\
 & \natline{తృష్ణాసఙ్గసముద్భవమ్ |} \\
 & \natline{తన్నిబధ్నాతి కౌన్తేయ} \\
 & \natline{కర్మసఙ్గేన దేహినమ్ ||}
\end{tabular}
\end{table}

\begin{table}[H]
\begin{tabular}{cl}
\textbf{14.8} & \natline{తమస్త్వజ్ఞానజం విద్ధి} \\
 & \natline{మోహనం సర్వదేహినామ్ |} \\
 & \natline{ప్రమాదాలస్యనిద్రాభిః} \\
 & \natline{తన్నిబధ్నాతి భారత ||}
\end{tabular}
\end{table}

\begin{table}[H]
\begin{tabular}{cl}
\textbf{14.9} & \natline{సత్త్వం సుఖే సఞ్జయతి} \\
 & \natline{రజః కర్మణి భారత |} \\
 & \natline{జ్ఞానమావృత్య తు తమః} \\
 & \natline{ప్రమాదే సఞ్జయత్యుత ||}
\end{tabular}
\end{table}

\begin{table}[H]
\begin{tabular}{cl}
\textbf{14.10} & \natline{రజస్తమశ్చాభిభూయ} \\
 & \natline{సత్త్వం భవతి భారత |} \\
 & \natline{రజః సత్త్వం తమశ్చైవ} \\
 & \natline{తమః సత్త్వం రజస్తథా ||}
\end{tabular}
\end{table}

\begin{table}[H]
\begin{tabular}{cl}
\textbf{14.11} & \natline{సర్వద్వారేషు దేహేఽస్మిన్} \\
 & \natline{ప్రకాశ ఉపజాయతే |} \\
 & \natline{జ్ఞానం యదా తదా విద్యాత్} \\
 & \natline{వివృద్ధం సత్త్వమిత్యుత ||}
\end{tabular}
\end{table}

\begin{table}[H]
\begin{tabular}{cl}
\textbf{14.12} & \natline{లోభః ప్రవృత్తిరారమ్భః} \\
 & \natline{కర్మణామశమః స్పృహా |} \\
 & \natline{రజస్యేతాని జాయన్తే} \\
 & \natline{వివృద్ధే భరతర్షభ ||}
\end{tabular}
\end{table}

\begin{table}[H]
\begin{tabular}{cl}
\textbf{14.13} & \natline{అప్రకాశోఽప్రవృత్తిశ్చ} \\
 & \natline{ప్రమాదో మోహ ఏవ చ |} \\
 & \natline{తమస్యేతాని జాయన్తే} \\
 & \natline{వివృద్ధే కురునన్దన ||}
\end{tabular}
\end{table}

\begin{table}[H]
\begin{tabular}{cl}
\textbf{14.14} & \natline{యదా సత్త్వే ప్రవృద్ధే తు} \\
 & \natline{ప్రలయం యాతి దేహభృత్ |} \\
 & \natline{తదోత్తమవిదాం లోకాన్} \\
 & \natline{అమలాన్ప్రతిపద్యతే ||}
\end{tabular}
\end{table}

\begin{table}[H]
\begin{tabular}{cl}
\textbf{14.15} & \natline{రజసి ప్రలయం గత్వా} \\
 & \natline{కర్మసఙ్గిషు జాయతే |} \\
 & \natline{తథా ప్రలీనస్తమసి} \\
 & \natline{మూఢయోనిషు జాయతే ||}
\end{tabular}
\end{table}

\begin{table}[H]
\begin{tabular}{cl}
\textbf{14.16} & \natline{కర్మణః సుకృతస్యాహుః} \\
 & \natline{సాత్త్వికం నిర్మలం ఫలమ్ |} \\
 & \natline{రజసస్తు ఫలం దుఃఖమ్} \\
 & \natline{అజ్ఞానం తమసః ఫలమ్ ||}
\end{tabular}
\end{table}

\begin{table}[H]
\begin{tabular}{cl}
\textbf{14.17} & \natline{సత్త్వాత్సఞ్జాయతే జ్ఞానం} \\
 & \natline{రజసో లోభ ఏవ చ |} \\
 & \natline{ప్రమాదమోహౌ తమసః} \\
 & \natline{భవతోఽజ్ఞానమేవ చ ||}
\end{tabular}
\end{table}

\begin{table}[H]
\begin{tabular}{cl}
\textbf{14.18} & \natline{ఊర్ధ్వం గచ్ఛన్తి సత్త్వస్థాః} \\
 & \natline{మధ్యే తిష్ఠన్తి రాజసాః |} \\
 & \natline{జఘన్యగుణవృత్తిస్థాః} \\
 & \natline{అధో గచ్ఛన్తి తామసాః ||}
\end{tabular}
\end{table}

\begin{table}[H]
\begin{tabular}{cl}
\textbf{14.19} & \natline{నాన్యం గుణేభ్యః కర్తారం} \\
 & \natline{యదా ద్రష్టాఽనుపశ్యతి |} \\
 & \natline{గుణేభ్యశ్చ పరం వేత్తి} \\
 & \natline{మద్భావం సోఽధిగచ్ఛతి ||}
\end{tabular}
\end{table}

\begin{table}[H]
\begin{tabular}{cl}
\textbf{14.20} & \natline{గుణానేతానతీత్య త్రీన్} \\
 & \natline{దేహీ దేహసముద్భవాన్ |} \\
 & \natline{జన్మమృత్యుజరాదుఃఖైః} \\
 & \natline{విముక్తోఽమృతమశ్నుతే ||}
\end{tabular}
\end{table}

\begin{table}[H]
\begin{tabular}{cl}
\textbf{14.21} & \natline{అర్జున ఉవాచ} \\
 & \natline{కైర్లిఙ్గైస్త్రీన్గుణానేతాన్} \\
 & \natline{అతీతో భవతి ప్రభో |} \\
 & \natline{కిమాచారః కథం చైతాన్} \\
 & \natline{త్రీన్గుణానతివర్తతే ||}
\end{tabular}
\end{table}

\begin{table}[H]
\begin{tabular}{cl}
\textbf{14.22} & \natline{శ్రీ భగవానువాచ} \\
 & \natline{ప్రకాశం చ ప్రవృత్తిం చ} \\
 & \natline{మోహమేవ చ పాణ్డవ |} \\
 & \natline{న ద్వేష్టి సమ్ప్రవృత్తాని} \\
 & \natline{న నివృత్తాని కాఙ్క్షతి ||}
\end{tabular}
\end{table}

\begin{table}[H]
\begin{tabular}{cl}
\textbf{14.23} & \natline{ఉదాసీనవదాసీనః} \\
 & \natline{గుణైర్యో న విచాల్యతే |} \\
 & \natline{గుణా వర్తన్త ఇత్యేవ} \\
 & \natline{యోఽవతిష్ఠతి నేఙ్గతే ||}
\end{tabular}
\end{table}

\begin{table}[H]
\begin{tabular}{cl}
\textbf{14.24} & \natline{సమదుఃఖసుఖః స్వస్థః} \\
 & \natline{సమలోష్టాశ్మకాఞ్చనః |} \\
 & \natline{తుల్యప్రియాప్రియో ధీరః} \\
 & \natline{తుల్యనిన్దాత్మసంస్తుతిః ||}
\end{tabular}
\end{table}

\begin{table}[H]
\begin{tabular}{cl}
\textbf{14.25} & \natline{మానాపమానయోస్తుల్యః} \\
 & \natline{తుల్యో మిత్రారిపక్షయోః |} \\
 & \natline{సర్వారమ్భపరిత్యాగీ} \\
 & \natline{గుణాతీతః స ఉచ్యతే ||}
\end{tabular}
\end{table}

\begin{table}[H]
\begin{tabular}{cl}
\textbf{14.26} & \natline{మాం చ యోఽవ్యభిచారేణ} \\
 & \natline{భక్తియోగేన సేవతే |} \\
 & \natline{స గుణాన్సమతీత్యైతాన్} \\
 & \natline{బ్రహ్మభూయాయ కల్పతే ||}
\end{tabular}
\end{table}

\begin{table}[H]
\begin{tabular}{cl}
\textbf{14.27} & \natline{బ్రహ్మణో హి ప్రతిష్ఠాఽహమ్} \\
 & \natline{అమృతస్యావ్యయస్య చ |} \\
 & \natline{శాశ్వతస్య చ ధర్మస్య} \\
 & \natline{సుఖస్యైకాన్తికస్య చ ||}
\end{tabular}
\end{table}

\begin{table}[H]
\begin{tabular}{cl}
 & \natline{శ్రీమద్భగవద్గీతాసు ఉపనిషత్సు} \\
 & \natline{బ్రహ్మవిద్యాయాం యోగశాస్త్రే} \\
 & \natline{శ్రీకృష్ణార్జున సంవాదే} \\
 & \natline{గుణత్రయవిభాగయోగోనామ} \\
 & \natline{చతుర్దశోధ్యాయః}
\end{tabular}
\end{table}

