\begin{table}[H]
\begin{tabular}{cl}
\textbf{13.0} & \natline{ఓం శ్రీ పరమాత్మనే నమః} \\
 & \natline{అథ త్రయోదశోఽధ్యాయః} \\
 & \natline{క్షేత్రక్షేత్రజ్ఞవిభాగయోగః}
\end{tabular}
\end{table}

\begin{table}[H]
\begin{tabular}{cl}
\textbf{13.1} & \natline{అర్జున ఉవాచ} \\
 & \natline{ప్రకృతిం పురుషం చైవ} \\
 & \natline{క్షేత్రం క్షేత్రజ్ఞమేవ చ |} \\
 & \natline{ఏతత్ వేదితుమిచ్ఛామి} \\
 & \natline{జ్ఞానం జ్ఞేయం చ కేశవ ||}
\end{tabular}
\end{table}

\begin{table}[H]
\begin{tabular}{cl}
\textbf{13.2} & \natline{శ్రీ భగవానువాచ} \\
 & \natline{ఇదం శరీరం కౌన్తేయ} \\
 & \natline{క్షేత్రమిత్యభిధీయతే |} \\
 & \natline{ఏతద్యో వేత్తి తం ప్రాహుః} \\
 & \natline{క్షేత్రజ్ఞ ఇతి తద్విదః ||}
\end{tabular}
\end{table}

\begin{table}[H]
\begin{tabular}{cl}
\textbf{13.3} & \natline{క్షేత్రజ్ఞం చాపి మాం విద్ధి} \\
 & \natline{సర్వక్షేత్రేషు భారత |} \\
 & \natline{క్షేత్రక్షేత్రజ్ఞయోర్జ్ఞానం} \\
 & \natline{యత్తజ్జ్ఞానం మతం మమ ||}
\end{tabular}
\end{table}

\begin{table}[H]
\begin{tabular}{cl}
\textbf{13.4} & \natline{తత్క్షేత్రం యచ్చ యాదృక్చ} \\
 & \natline{యద్వికారి యతశ్చ యత్ |} \\
 & \natline{స చ యో యత్ప్రభావశ్చ} \\
 & \natline{తత్సమాసేన మే శృణు ||}
\end{tabular}
\end{table}

\begin{table}[H]
\begin{tabular}{cl}
\textbf{13.5} & \natline{ఋషిభిర్బహుధా గీతం} \\
 & \natline{ఛన్దోభిర్వివిధైః పృథక్ |} \\
 & \natline{బ్రహ్మసూత్రపదైశ్చైవ} \\
 & \natline{హేతుమద్భిర్వినిశ్చితైః ||}
\end{tabular}
\end{table}

