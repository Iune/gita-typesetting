\begin{table}[H]
\begin{tabular}{cl}
 & \natline{శ్రీ పరమాత్మనే నమః} \\
 & \natline{అథ త్రయోదశోఽధ్యాయః} \\
 & \natline{క్షేత్రక్షేత్రజ్ఞవిభాగయోగః}
\end{tabular}
\end{table}

\begin{table}[H]
\begin{tabular}{cl}
\textbf{13.1} & \natline{అర్జున ఉవాచ} \\
 & \natline{ప్రకృతిం పురుషం చైవ} \\
 & \natline{క్షేత్రం క్షేత్రజ్ఞమేవ చ |} \\
 & \natline{ఏతత్ వేదితుమిచ్ఛామి} \\
 & \natline{జ్ఞానం జ్ఞేయం చ కేశవ ||}
\end{tabular}
\end{table}

\begin{table}[H]
\begin{tabular}{cl}
\textbf{13.2} & \natline{శ్రీ భగవానువాచ} \\
 & \natline{ఇదం శరీరం కౌన్తేయ} \\
 & \natline{క్షేత్రమిత్యభిధీయతే |} \\
 & \natline{ఏతద్యో వేత్తి తం ప్రాహుః} \\
 & \natline{క్షేత్రజ్ఞ ఇతి తద్విదః ||}
\end{tabular}
\end{table}

\begin{table}[H]
\begin{tabular}{cl}
\textbf{13.3} & \natline{క్షేత్రజ్ఞం చాపి మాం విద్ధి} \\
 & \natline{సర్వక్షేత్రేషు భారత |} \\
 & \natline{క్షేత్రక్షేత్రజ్ఞయోర్జ్ఞానం} \\
 & \natline{యత్తజ్జ్ఞానం మతం మమ ||}
\end{tabular}
\end{table}

\begin{table}[H]
\begin{tabular}{cl}
\textbf{13.4} & \natline{తత్క్షేత్రం యచ్చ యాదృక్చ} \\
 & \natline{యద్వికారి యతశ్చ యత్ |} \\
 & \natline{స చ యో యత్ప్రభావశ్చ} \\
 & \natline{తత్సమాసేన మే శృణు ||}
\end{tabular}
\end{table}

\begin{table}[H]
\begin{tabular}{cl}
\textbf{13.5} & \natline{ఋషిభిర్బహుధా గీతం} \\
 & \natline{ఛన్దోభిర్వివిధైః పృథక్ |} \\
 & \natline{బ్రహ్మసూత్రపదైశ్చైవ} \\
 & \natline{హేతుమద్భిర్వినిశ్చితైః ||}
\end{tabular}
\end{table}

\begin{table}[H]
\begin{tabular}{cl}
\textbf{13.6} & \natline{మహాభూతాన్యహఙ్కారః} \\
 & \natline{బుద్ధిరవ్యక్తమేవ చ |} \\
 & \natline{ఇన్ద్రియాణి దశైకం చ} \\
 & \natline{పఞ్చ చేన్ద్రియగోచరాః ||}
\end{tabular}
\end{table}

\begin{table}[H]
\begin{tabular}{cl}
\textbf{13.7} & \natline{ఇచ్ఛా ద్వేషః సుఖం దుఃఖం} \\
 & \natline{సఙ్ఘాతశ్చేతనా ధృతిః |} \\
 & \natline{ఏతత్క్షేత్రం సమాసేన} \\
 & \natline{సవికారముదాహృతమ్ ||}
\end{tabular}
\end{table}

\begin{table}[H]
\begin{tabular}{cl}
\textbf{13.8} & \natline{అమానిత్వమదంభిత్వమ్} \\
 & \natline{అహింసా క్షాన్తిరార్జవమ్ |} \\
 & \natline{ఆచార్యోపాసనం శౌచం} \\
 & \natline{స్థైర్యమాత్మవినిగ్రహః ||}
\end{tabular}
\end{table}

\begin{table}[H]
\begin{tabular}{cl}
\textbf{13.9} & \natline{ఇన్ద్రియార్థేషు వైరాగ్యమ్} \\
 & \natline{అనహఙ్కార ఏవ చ |} \\
 & \natline{జన్మమృత్యుజరావ్యాధి} \\
 & \natline{దుఃఖదోషానుదర్శనమ్ ||}
\end{tabular}
\end{table}

\begin{table}[H]
\begin{tabular}{cl}
\textbf{13.10} & \natline{అసక్తిరనభిష్వఙ్గః} \\
 & \natline{పుత్రదారగృహాదిషు |} \\
 & \natline{నిత్యం చ సమచిత్తత్వమ్} \\
 & \natline{ఇష్టానిష్టోపపత్తిషు ||}
\end{tabular}
\end{table}

\begin{table}[H]
\begin{tabular}{cl}
\textbf{13.11} & \natline{మయి చానన్యయోగేన} \\
 & \natline{భక్తిరవ్యభిచారిణీ |} \\
 & \natline{వివిక్తదేశసేవిత్వమ్} \\
 & \natline{అరతిర్జనసంసది ||}
\end{tabular}
\end{table}

\begin{table}[H]
\begin{tabular}{cl}
\textbf{13.12} & \natline{అధ్యాత్మజ్ఞాననిత్యత్వం} \\
 & \natline{తత్త్వజ్ఞానార్థదర్శనమ్ |} \\
 & \natline{ఏతజ్జ్ఞానమితి ప్రోక్తమ్} \\
 & \natline{అజ్ఞానం యదతోఽన్యథా ||}
\end{tabular}
\end{table}

\begin{table}[H]
\begin{tabular}{cl}
\textbf{13.13} & \natline{జ్ఞేయం యత్తత్ప్రవక్ష్యామి} \\
 & \natline{యజ్జ్ఞాత్వాఽమృతమశ్నుతే |} \\
 & \natline{అనాదిమత్పరం బ్రహ్మ} \\
 & \natline{న సత్తన్నాసదుచ్యతే ||}
\end{tabular}
\end{table}

\begin{table}[H]
\begin{tabular}{cl}
\textbf{13.14} & \natline{సర్వతః పాణిపాదం తత్} \\
 & \natline{సర్వతోఽక్షిశిరోముఖమ్ |} \\
 & \natline{సర్వతః శ్రుతిమల్లోకే} \\
 & \natline{సర్వమావృత్య తిష్ఠతి ||}
\end{tabular}
\end{table}

\begin{table}[H]
\begin{tabular}{cl}
\textbf{13.15} & \natline{సర్వేన్ద్రియగుణాభాసం} \\
 & \natline{సర్వేన్ద్రియవివర్జితమ్ |} \\
 & \natline{అసక్తం సర్వభృచ్చైవ} \\
 & \natline{నిర్గుణం గుణభోక్తృ చ ||}
\end{tabular}
\end{table}

\begin{table}[H]
\begin{tabular}{cl}
\textbf{13.16} & \natline{బహిరన్తశ్చ భూతానామ్} \\
 & \natline{అచరం చరమేవ చ |} \\
 & \natline{సూక్ష్మత్వాత్తదవిజ్ఞేయం} \\
 & \natline{దూరస్థం చాన్తికే చ తత్ ||}
\end{tabular}
\end{table}

\begin{table}[H]
\begin{tabular}{cl}
\textbf{13.17} & \natline{అవిభక్తం చ భూతేషు} \\
 & \natline{విభక్తమివ చ స్థితమ్ |} \\
 & \natline{భూతభర్తృ చ తజ్జ్ఞేయం} \\
 & \natline{గ్రసిష్ణు ప్రభవిష్ణు చ ||}
\end{tabular}
\end{table}

\begin{table}[H]
\begin{tabular}{cl}
\textbf{13.18} & \natline{జ్యోతిషామపి తజ్జ్యోతిః} \\
 & \natline{తమసః పరముచ్యతే |} \\
 & \natline{జ్ఞానం జ్ఞేయం జ్ఞానగమ్యం} \\
 & \natline{హృది సర్వస్య విష్ఠితమ్ ||}
\end{tabular}
\end{table}

\begin{table}[H]
\begin{tabular}{cl}
\textbf{13.19} & \natline{ఇతి క్షేత్రం తథా జ్ఞానం} \\
 & \natline{జ్ఞేయం చోక్తం సమాసతః |} \\
 & \natline{మద్భక్త ఏతద్విజ్ఞాయ} \\
 & \natline{మద్భావాయోపపద్యతే ||}
\end{tabular}
\end{table}

\begin{table}[H]
\begin{tabular}{cl}
\textbf{13.20} & \natline{ప్రకృతిం పురుషం చైవ} \\
 & \natline{విద్ధ్యనాదీ ఉభావపి |} \\
 & \natline{వికారాంశ్చ గుణాంశ్చైవ} \\
 & \natline{విద్ధి ప్రకృతిసమ్భవాన్ ||}
\end{tabular}
\end{table}

\begin{table}[H]
\begin{tabular}{cl}
\textbf{13.21} & \natline{కార్యకరణకర్తృత్వే} \\
 & \natline{హేతుః ప్రకృతిరుచ్యతే |} \\
 & \natline{పురుషః సుఖదుఃఖానాం} \\
 & \natline{భోక్తృత్వే హేతురుచ్యతే ||}
\end{tabular}
\end{table}

\begin{table}[H]
\begin{tabular}{cl}
\textbf{13.22} & \natline{పురుషః ప్రకృతిస్థో హి} \\
 & \natline{భుఙ్క్తే ప్రకృతిజాన్గుణాన్ |} \\
 & \natline{కారణం గుణసఙ్గోఽస్య} \\
 & \natline{సదసద్యోనిజన్మసు ||}
\end{tabular}
\end{table}

\begin{table}[H]
\begin{tabular}{cl}
\textbf{13.23} & \natline{ఉపద్రష్టాఽనుమన్తా చ} \\
 & \natline{భర్తా భోక్తా మహేశ్వరః |} \\
 & \natline{పరమాత్మేతి చాప్యుక్తః} \\
 & \natline{దేహేఽస్మిన్పురుషః పరః ||}
\end{tabular}
\end{table}

\begin{table}[H]
\begin{tabular}{cl}
\textbf{13.24} & \natline{య ఏవం వేత్తి పురుషం} \\
 & \natline{ప్రకృతిం చ గుణైః సహ |} \\
 & \natline{సర్వథా వర్తమానోఽపి} \\
 & \natline{న స భూయోఽభిజాయతే ||}
\end{tabular}
\end{table}

\begin{table}[H]
\begin{tabular}{cl}
\textbf{13.25} & \natline{ధ్యానేనాత్మని పశ్యన్తి} \\
 & \natline{కేచిదాత్మానమాత్మనా |} \\
 & \natline{అన్యే సాఙ్ఖ్యేన యోగేన} \\
 & \natline{కర్మయోగేన చాపరే ||}
\end{tabular}
\end{table}

\begin{table}[H]
\begin{tabular}{cl}
\textbf{13.26} & \natline{అన్యే త్వేవమజానన్తః} \\
 & \natline{శ్రుత్వాఽన్యేభ్య ఉపాసతే |} \\
 & \natline{తేఽపి చాతితరన్త్యేవ} \\
 & \natline{మృత్యుం శ్రుతిపరాయణాః ||}
\end{tabular}
\end{table}

\begin{table}[H]
\begin{tabular}{cl}
\textbf{13.27} & \natline{యావత్సఞ్జాయతే కిఞ్చిత్} \\
 & \natline{సత్త్వం స్థావరజఙ్గమమ్ |} \\
 & \natline{క్షేత్రక్షేత్రజ్ఞసంయోగాత్} \\
 & \natline{తద్విద్ధి భరతర్షభ ||}
\end{tabular}
\end{table}

\begin{table}[H]
\begin{tabular}{cl}
\textbf{13.28} & \natline{సమం సర్వేషు భూతేషు} \\
 & \natline{తిష్ఠన్తం పరమేశ్వరమ్ |} \\
 & \natline{వినశ్యత్స్వవినశ్యన్తం} \\
 & \natline{యః పశ్యతి స పశ్యతి ||}
\end{tabular}
\end{table}

\begin{table}[H]
\begin{tabular}{cl}
\textbf{13.29} & \natline{సమం పశ్యన్హి సర్వత్ర} \\
 & \natline{సమవస్థితమీశ్వరమ్ |} \\
 & \natline{న హినస్త్యాత్మనాఽఽత్మానం} \\
 & \natline{తతో యాతి పరాం గతిమ్ ||}
\end{tabular}
\end{table}

\begin{table}[H]
\begin{tabular}{cl}
\textbf{13.30} & \natline{ప్రకృత్యైవ చ కర్మాణి} \\
 & \natline{క్రియమాణాని సర్వశః |} \\
 & \natline{యః పశ్యతి తథాఽఽత్మానమ్} \\
 & \natline{అకర్తారం స పశ్యతి ||}
\end{tabular}
\end{table}

\begin{table}[H]
\begin{tabular}{cl}
\textbf{13.31} & \natline{యదా భూతపృథగ్భావమ్} \\
 & \natline{ఏకస్థమనుపశ్యతి |} \\
 & \natline{తత ఏవ చ విస్తారం} \\
 & \natline{బ్రహ్మ సమ్పద్యతే తదా ||}
\end{tabular}
\end{table}

\begin{table}[H]
\begin{tabular}{cl}
\textbf{13.32} & \natline{అనాదిత్వాన్నిర్గుణత్వాత్} \\
 & \natline{పరమాత్మాయమవ్యయః |} \\
 & \natline{శరీరస్థోఽపి కౌన్తేయ} \\
 & \natline{న కరోతి న లిప్యతే ||}
\end{tabular}
\end{table}

\begin{table}[H]
\begin{tabular}{cl}
\textbf{13.33} & \natline{యథా సర్వగతం సౌక్ష్మ్యాత్} \\
 & \natline{ఆకాశం నోపలిప్యతే |} \\
 & \natline{సర్వత్రావస్థితో దేహే} \\
 & \natline{తథాఽఽత్మా నోపలిప్యతే ||}
\end{tabular}
\end{table}

\begin{table}[H]
\begin{tabular}{cl}
\textbf{13.34} & \natline{యథా ప్రకాశయత్యేకః} \\
 & \natline{కృత్స్నం లోకమిమం రవిః |} \\
 & \natline{క్షేత్రం క్షేత్రీ తథా కృత్స్నం} \\
 & \natline{ప్రకాశయతి భారత ||}
\end{tabular}
\end{table}

\begin{table}[H]
\begin{tabular}{cl}
\textbf{13.35} & \natline{క్షేత్రక్షేత్రజ్ఞయోరేవమ్} \\
 & \natline{అన్తరం జ్ఞానచక్షుషా |} \\
 & \natline{భూతప్రకృతిమోక్షం చ} \\
 & \natline{యే విదుర్యాన్తి తే పరమ్ ||}
\end{tabular}
\end{table}

\begin{table}[H]
\begin{tabular}{cl}
 & \natline{శ్రీమద్భగవద్గీతాసు ఉపనిషత్సు} \\
 & \natline{బ్రహ్మవిద్యాయాం యోగశాస్త్రే} \\
 & \natline{శ్రీకృష్ణార్జున సంవాదే} \\
 & \natline{క్షేత్రక్షేత్రజ్ఞవిభాగయోగోనామ} \\
 & \natline{త్రయోదశోధ్యాయః}
\end{tabular}
\end{table}

