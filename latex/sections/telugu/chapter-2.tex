\subsection*{2.0}
\begin{table}[H]
\centering
\begin{tabular}{ll}
\natline{ఓం శ్రీ పరమాత్మనే నమః} & \romline{oṃ śrī paramātmane namaḥ} \\
\natline{అథ ద్వితీయోఽధ్యాయః} & \romline{atha dvitīyo'dhyāyaḥ} \\
\natline{సాఙ్ఖ్యయోగః} & \romline{sāṅkhya-yogaḥ}
\end{tabular}
\end{table}

\subsection*{2.1}
\begin{table}[H]
\centering
\begin{tabular}{ll}
\natline{సంజయ ఉవాచ} & \romline{saṃjaya uvāca} \\
\natline{తం తథా కృపయావిష్టమ్} & \romline{taṃ tathā kṛpayāviṣṭam} \\
\natline{అశ్రుపూర్ణాకులేక్షణమ్} & \romline{aśru-pūrṇākulekṣaṇam} \\
\natline{విషీదంతమిదం వాక్యమ్} & \romline{viṣīdaṃtamidaṃ vākyam} \\
\natline{ఉవాచ మధుసూదనః} & \romline{uvāca madhusūdanaḥ}
\end{tabular}
\end{table}

\subsection*{2.2}
\begin{table}[H]
\centering
\begin{tabular}{ll}
\natline{శ్రీ భగవానువాచ} & \romline{śrī bhagavān-uvāca} \\
\natline{కుతస్త్వా కశ్మలమిదం} & \romline{kutastvā kaśmalamidaṃ} \\
\natline{విషమే సముపస్థితమ్} & \romline{viṣame samupasthitam} \\
\natline{అనార్యజుష్టమస్వర్గ్యమ్} & \romline{anārya-juṣṭamasvargyam} \\
\natline{అకీర్తికరమర్జున} & \romline{akīrti-karam-arjuna}
\end{tabular}
\end{table}

\subsection*{2.3}
\begin{table}[H]
\centering
\begin{tabular}{ll}
\natline{క్లైబ్యం మా స్మ గమః పార్థ} & \romline{klaibyaṃ mā sma gamaḥ pārtha} \\
\natline{నైతత్త్వయ్యుపపద్యతే} & \romline{naitat-tvayyupapadyate} \\
\natline{క్షుద్రం హృదయదౌర్బల్యం} & \romline{kṣudraṃ hṛdaya-daurbalyaṃ} \\
\natline{త్యక్త్వోత్తిష్ఠ పరంతప} & \romline{tyaktvottiṣṭha paraṃtapa}
\end{tabular}
\end{table}

\subsection*{2.4}
\begin{table}[H]
\centering
\begin{tabular}{ll}
\natline{అర్జున ఉవాచ} & \romline{arjuna uvāca} \\
\natline{కథం భీశ్మమహం సంఖ్యే} & \romline{kathaṃ bhīśmamahaṃ saṃkhye} \\
\natline{ద్రోణం చ మధుసూదన} & \romline{droṇaṃ ca madhusūdana} \\
\natline{ఇశుభిః ప్రతియోత్స్యామి} & \romline{iśubhiḥ pratiyotsyāmi} \\
\natline{పూజార్హావరిసూదన} & \romline{pūjārhāvarisūdana}
\end{tabular}
\end{table}

\subsection*{2.5}
\begin{table}[H]
\centering
\begin{tabular}{ll}
\natline{గురూనహత్వా హి మహానుభావాన్} & \romline{gurūnahatvā hi mahānubhāvān} \\
\natline{శ్రేయో భోక్తుం భైక్ష్యమపీహ లోకే} & \romline{śreyo bhoktuṃ bhaikṣyamapīha loke} \\
\natline{హత్వార్థకామంస్తు గురూనిహైవ} & \romline{hatvārtha-kāmaṃstu gurūnihaiva} \\
\natline{భుంజీయ భోగాన్ రుధిరప్రదిగ్ధాన్} & \romline{bhuṃjīya bhogān rudhira-pradigdhān}
\end{tabular}
\end{table}

\subsection*{2.6}
\begin{table}[H]
\centering
\begin{tabular}{ll}
\natline{న చైతద్విద్మః కతరన్నో గరీయః} & \romline{na caitadvidmaḥ kataranno garīyaḥ} \\
\natline{యద్వా జయేమ యది వా నో జయేయుః} & \romline{yadvā jayema yadi vā no jayeyuḥ} \\
\natline{యానేవ హత్వా న జిజీవిషామః} & \romline{yāneva hatvā na jijīviṣāmaḥ} \\
\natline{తేఽవస్థితాః ప్రముఖే ధార్తరాష్ట్రాః} & \romline{te'vasthitāḥ pramukhe dhārtarāṣṭrāḥ}
\end{tabular}
\end{table}

\subsection*{2.7}
\begin{table}[H]
\centering
\begin{tabular}{ll}
\natline{కార్పన్యదోషోపహతస్వభావః} & \romline{kārpanya-doṣopahata-svabhāvaḥ} \\
\natline{పృచ్ఛామి త్వాం ధర్మసమ్మూఢచేతాః} & \romline{pṛcchāmi tvāṃ dharma-sammūḍha-cetāḥ} \\
\natline{యచ్ఛ్రేయః స్యాన్నిశ్చితం బ్రూహి తన్మే} & \romline{yacchreyaḥ syānniścitaṃ brūhi tanme} \\
\natline{శిష్యస్తేఽహం శాధి మాం త్వాం ప్రపన్నమ్} & \romline{śiṣyaste'haṃ śādhi māṃ tvāṃ prapannam}
\end{tabular}
\end{table}

\subsection*{2.8}
\begin{table}[H]
\centering
\begin{tabular}{ll}
\natline{న హి ప్రపశ్యామి మమాపనుద్యాద్} & \romline{na hi·prapaśyāmi mamāpanudyād} \\
\natline{యచ్ఛోకముచ్ఛోషణమిన్ద్రియాణామ్} & \romline{yacchokam-ucchoṣaṇam-indriyāṇām} \\
\natline{అవాప్య భూమావసపత్నమృద్ధం} & \romline{avāpya bhūmāv-asapatnamṛddhaṃ} \\
\natline{రాజ్యం సురాణామపి చాధిపత్యమ్} & \romline{rājyaṃ surāṇāmapi cādhipatyam}
\end{tabular}
\end{table}

\subsection*{2.9}
\begin{table}[H]
\centering
\begin{tabular}{ll}
\natline{సంజయ ఉవాచ} & \romline{saṃjaya uvāca} \\
\natline{ఏవముక్త్వా హృషీకేశం} & \romline{evam-uktvā hṛṣīkeśaṃ} \\
\natline{గుడాకేశః పరన్తపః} & \romline{guḍākeśaḥ parantapaḥ} \\
\natline{న యోత్స్య ఇతి గోవిందమ్} & \romline{na yotsya iti goviṃdam} \\
\natline{ఉక్త్వా తూష్ణీమ్ బభూవ హ} & \romline{uktvā tūṣṇīm babhūva ha}
\end{tabular}
\end{table}

\subsection*{2.10}
\begin{table}[H]
\centering
\begin{tabular}{ll}
\natline{తమువాచ హృషీకేశః} & \romline{tam-uvāca hṛṣīkeśaḥ} \\
\natline{ప్రహసన్నివ భారత} & \romline{prahasanniva bhārata} \\
\natline{సేనయోరుభయోర్మధ్యే} & \romline{senayorubhayor-madhye} \\
\natline{విషీదంతమిదం వచః} & \romline{viṣīdaṃtam-idaṃ vacaḥ}
\end{tabular}
\end{table}

\subsection*{2.11}
\begin{table}[H]
\centering
\begin{tabular}{ll}
\natline{శ్రీ భగవానువాచ} & \romline{śrī bhagavān-uvāca} \\
\natline{అశోచ్యానన్వశోచస్త్వం} & \romline{aśocyān-anvaśocas-tvaṃ} \\
\natline{ప్రజ్ఞావాదాంశ్చ భాషసే} & \romline{prajñā-vādāṃśca bhāṣase} \\
\natline{గతాసూనగతాసూంశ్చ} & \romline{gatāsūn-agatāsūṃś-ca} \\
\natline{నానుశోచన్తి పణ్డితాః} & \romline{nānuśocanti paṇḍitāḥ}
\end{tabular}
\end{table}

\subsection*{2.12}
\begin{table}[H]
\centering
\begin{tabular}{ll}
\natline{న త్వేవాహం జాతు నాసం} & \romline{na tvevāhaṃ jātu nāsaṃ} \\
\natline{న త్వం నేమే జనాధిపాః} & \romline{na tvaṃ neme janādhipāḥ} \\
\natline{న చైవ న భవిష్యామః} & \romline{na caiva na bhaviṣyāmaḥ} \\
\natline{సర్వే వయమతః పరమ్} & \romline{sarve vayamataḥ param}
\end{tabular}
\end{table}

\subsection*{2.13}
\begin{table}[H]
\centering
\begin{tabular}{ll}
\natline{దేహినోఽస్మిన్ యథా దేహే} & \romline{dehino'smin yathā dehe} \\
\natline{కౌమారం యౌవనం జరా} & \romline{kaumāraṃ yauvanaṃ jarā} \\
\natline{తథా దేహాంతరప్రాప్తిః} & \romline{tathā dehāṃtara-prāptiḥ} \\
\natline{ధీరస్తత్ర న ముహ్యతి} & \romline{dhīras-tatra na muhyati}
\end{tabular}
\end{table}

\subsection*{2.14}
\begin{table}[H]
\centering
\begin{tabular}{ll}
\natline{మాత్రాస్పర్శాస్తు కౌంతేయ} & \romline{mātrā-sparśās-tu kauṃteya} \\
\natline{శీతోష్ణసుఖదుఃఖదాః} & \romline{śītoṣṇa-sukha-duḥkha-dāḥ} \\
\natline{ఆగమాపాయినోఽనిత్యాః} & \romline{āgamāpāyino'nityāḥ} \\
\natline{తాంస్తితిక్షస్వ భారత} & \romline{tāṃs-titikṣasva bhārata}
\end{tabular}
\end{table}

\subsection*{2.15}
\begin{table}[H]
\centering
\begin{tabular}{ll}
\natline{యం హి న వ్యథయంత్యేతే} & \romline{yaṃ hi na vyathayaṃtyete} \\
\natline{పురుషం పురుషర్షభ} & \romline{puruṣaṃ puruṣarṣabha} \\
\natline{సమదుఃఖసుఖం ధీరం} & \romline{sama-duḥkha-sukhaṃ dhīraṃ} \\
\natline{సోఽమృతత్వాయ కల్పతే} & \romline{so'mṛtatvāya kalpate}
\end{tabular}
\end{table}

\subsection*{2.16}
\begin{table}[H]
\centering
\begin{tabular}{ll}
\natline{నాసతో విద్యతే భావః} & \romline{nāsato vidyate bhāvaḥ} \\
\natline{నాభావో విద్యతే సతః} & \romline{nābhāvo vidyate sataḥ} \\
\natline{ఉభయోరపి దృష్తోఽన్తః} & \romline{ubhayorapi dṛṣto'ntaḥ} \\
\natline{త్వనయోస్తత్త్వదర్శిభిః} & \romline{tvanayos-tattva-darśibhiḥ}
\end{tabular}
\end{table}

\subsection*{2.17}
\begin{table}[H]
\centering
\begin{tabular}{ll}
\natline{అవినాశి తు తద్విద్ధి} & \romline{avināśi tu tadviddhi} \\
\natline{యేన సర్వమిదం తతమ్} & \romline{yena sarvamidaṃ tatam} \\
\natline{వినాశమవ్యయస్యాస్య} & \romline{vināśam-avyayasyāsya} \\
\natline{న కశ్చిత్కర్తుమర్హతి} & \romline{na kaścit-kartum-arhati}
\end{tabular}
\end{table}

\subsection*{2.18}
\begin{table}[H]
\centering
\begin{tabular}{ll}
\natline{అంతవన్త ఇమే దేహాః} & \romline{aṃtavanta ime dehāḥ} \\
\natline{నిత్యస్యోక్తాః శరీరిణః} & \romline{nityasyoktāḥ śarīriṇaḥ} \\
\natline{అనాశినోఽప్రమేయస్య} & \romline{anāśino'prameyasya} \\
\natline{తస్మాద్యుధ్యస్వ భారత} & \romline{tasmād-yudhyasva bhārata}
\end{tabular}
\end{table}

\subsection*{2.19}
\begin{table}[H]
\centering
\begin{tabular}{ll}
\natline{య ఏనం వేత్తి హన్తారం} & \romline{ya enaṃ vetti hantāraṃ} \\
\natline{యశ్చైనం మన్యతే హతం} & \romline{yaścainaṃ manyate hataṃ} \\
\natline{ఉభౌ తౌ న విజానీతః} & \romline{ubhau tau na vijānītaḥ} \\
\natline{నాయం హన్తి న హన్యతే} & \romline{nāyaṃ hanti na hanyate}
\end{tabular}
\end{table}

\subsection*{2.20}
\begin{table}[H]
\centering
\begin{tabular}{ll}
\natline{న జాయతే మ్రియతే వా కదాచిత్} & \romline{na jāyate mriyate vā kadācit} \\
\natline{నాయం భూత్వా భవితా వా న భూయః} & \romline{nāyaṃ bhūtvā bhavitā vā na bhūyaḥ} \\
\natline{అజో నిత్యః శాశ్వతోఽయం పురాణః} & \romline{ajo nityaḥ śāśvato'yaṃ purāṇaḥ} \\
\natline{న హన్యతే హన్యమానే శరీరే} & \romline{na hanyate hanyamāne śarīre}
\end{tabular}
\end{table}

\subsection*{2.21}
\begin{table}[H]
\centering
\begin{tabular}{ll}
\natline{వేదావినాశినం నిత్యం} & \romline{vedāvināśinaṃ nityaṃ} \\
\natline{య ఏనమజమవ్యయమ్} & \romline{ya enamajam-avyayam} \\
\natline{కథం స పురుషః పార్థ} & \romline{kathaṃ sa puruṣaḥ pārtha} \\
\natline{కం ఘాతయతి హంతి కమ్} & \romline{kaṃ ghātayati haṃti kam}
\end{tabular}
\end{table}

\subsection*{2.22}
\begin{table}[H]
\centering
\begin{tabular}{ll}
\natline{వాసాంసి జీర్ణాని యథా విహాయ} & \romline{vāsāṃsi jīrṇāni yathā vihāya} \\
\natline{నవాని గృహ్ణాతి నరోఽపరాణి} & \romline{navāni gṛhṇāti naro'parāṇi} \\
\natline{తథా శరీరాణి విహాయ జీర్ణాని} & \romline{tathā śarīrāṇi vihāya jīrṇāni} \\
\natline{అన్యాని సంయాతి నవాని దేహీ} & \romline{anyāni saṃyāti navāni dehī}
\end{tabular}
\end{table}

\subsection*{2.23}
\begin{table}[H]
\centering
\begin{tabular}{ll}
\natline{నైనం ఛిన్దన్తి శస్త్రాణి} & \romline{nainaṃ chindanti śastrāṇi} \\
\natline{నైనం దహతి పావకః} & \romline{nainaṃ dahati pāvakaḥ} \\
\natline{న చైనం క్లేదయన్త్యాపః} & \romline{na cainaṃ kledayantyāpaḥ} \\
\natline{న శోషయతి మారుతః} & \romline{na śoṣayati mārutaḥ}
\end{tabular}
\end{table}

\subsection*{2.24}
\begin{table}[H]
\centering
\begin{tabular}{ll}
\natline{అచ్ఛేద్యోఽయమ్ అదాహ్యోఽయమ్} & \romline{acchedyo'yam adāhyo'yam} \\
\natline{అక్లేద్యోఽశోష్య ఏవ చ} & \romline{akledyo'śoṣya eva ca} \\
\natline{నిత్యః సర్వగతః స్థాణుః} & \romline{nityaḥ sarva-gataḥ sthāṇuḥ} \\
\natline{అచలోఽయం సనాతనః} & \romline{acalo'yaṃ sanātanaḥ}
\end{tabular}
\end{table}

\subsection*{2.25}
\begin{table}[H]
\centering
\begin{tabular}{ll}
\natline{అవ్యక్తోఽయమ్ అచిన్త్యోఽయమ్} & \romline{avyakto'yam acintyo'yam} \\
\natline{అవికార్యోఽయముచ్యతే} & \romline{avikāryo'yamucyate} \\
\natline{తస్మాదేవం విదిత్వైనం} & \romline{tasmādevaṃ viditvainaṃ} \\
\natline{నానుశోచితుమర్హసి} & \romline{nānuśocitumarhasi}
\end{tabular}
\end{table}

\subsection*{2.26}
\begin{table}[H]
\centering
\begin{tabular}{ll}
\natline{అథ చైనం నిత్యజాతం} & \romline{atha cainaṃ nityajātaṃ} \\
\natline{నిత్యం వా మన్యసే మృతమ్} & \romline{nityaṃ vā manyase mṛtam} \\
\natline{తథాఽపి త్వం మహాబాహో} & \romline{tathā'pi tvaṃ mahābāho} \\
\natline{నైవం శోచితుమర్హసి} & \romline{naivaṃ śocitumarhasi}
\end{tabular}
\end{table}

\subsection*{2.27}
\begin{table}[H]
\centering
\begin{tabular}{ll}
\natline{జాతస్య హి ధ్రువో మృత్యుః} & \romline{jātasya hi·dhruvo mṛtyuḥ} \\
\natline{ధ్రువం జన్మ మృతస్య చ} & \romline{dhruvaṃ janma mṛtasya ca} \\
\natline{తస్మాదపరిహార్యేఽర్థే} & \romline{tasmādaparihārye'rthe} \\
\natline{న త్వం శోచితుమర్హసి} & \romline{na tvaṃ śocitumarhasi}
\end{tabular}
\end{table}

\subsection*{2.28}
\begin{table}[H]
\centering
\begin{tabular}{ll}
\natline{అవ్యక్తాదీని భూతాని} & \romline{avyaktādīni bhūtāni} \\
\natline{వ్యక్తమధ్యాని భారత} & \romline{vyaktamadhyāni bhārata} \\
\natline{అవ్యక్తనిధనాన్యేవ} & \romline{avyakta-nidhanānyeva} \\
\natline{తత్ర కా పరిదేవనా} & \romline{tatra kā paridevanā}
\end{tabular}
\end{table}

\subsection*{2.29}
\begin{table}[H]
\centering
\begin{tabular}{ll}
\natline{ఆశ్చర్యవత్ పశ్యతి కశ్చిదేనమ్} & \romline{āścaryavat paśyati kaścidenam} \\
\natline{ఆశ్చర్యవద్ వదతి తథైవ cఆన్యః} & \romline{āścaryavad vadati tathaiva cānyaḥ} \\
\natline{ఆశ్చర్యవచ్చైనమన్యః శృణోతి} & \romline{āścaryavaccainamanyaḥ śṛṇoti} \\
\natline{శ్రుత్వాప్యేనం వేద న చైవ కశ్చిత్} & \romline{śrutvāpyenaṃ veda na caiva kaścit}
\end{tabular}
\end{table}

\subsection*{2.30}
\begin{table}[H]
\centering
\begin{tabular}{ll}
\natline{దేహీ నిత్యమవధ్యోఽయం} & \romline{dehī nityamavadhyo'yaṃ} \\
\natline{దేహే సర్వస్య భారత} & \romline{dehe sarvasya bhārata} \\
\natline{తస్మాత్సర్వాణి భూతాని} & \romline{tasmātsarvāṇi bhūtāni} \\
\natline{న త్వం శోచితుమర్హసి} & \romline{na tvaṃ śocitumarhasi}
\end{tabular}
\end{table}

\subsection*{2.31}
\begin{table}[H]
\centering
\begin{tabular}{ll}
\natline{స్వధర్మమపి చావేక్ష్య} & \romline{svadharmamapi cāvekṣya} \\
\natline{న వికమ్పితుమర్హసి} & \romline{na vikampitumarhasi} \\
\natline{ధర్మ్యాద్ధి యుద్ధాచ్ఛ్రేయోఽన్యత్} & \romline{dharmyāddhi yuddhācchreyo'nyat} \\
\natline{క్షత్రియస్యనవిద్యతే} & \romline{kṣatriyasya-na-vidyate}
\end{tabular}
\end{table}

\subsection*{2.32}
\begin{table}[H]
\centering
\begin{tabular}{ll}
\natline{యదృచ్ఛయా చోపపన్నం} & \romline{yadṛcchayā copapannaṃ} \\
\natline{స్వర్గద్వారమపావృతమ్} & \romline{svarga-dvāra-mapāvṛtam} \\
\natline{సుఖినః క్షత్రియాః పార్థ} & \romline{sukhinaḥ kṣatriyāḥ pārtha} \\
\natline{లభన్తే యుద్ధమీదృశమ్} & \romline{labhante yuddhamīdṛśam}
\end{tabular}
\end{table}

\subsection*{2.33}
\begin{table}[H]
\centering
\begin{tabular}{ll}
\natline{అథ చేత్త్వమిమం ధర్మ్యం} & \romline{atha cettvamimaṃ dharmyaṃ} \\
\natline{సఙ్గ్రామం న కరిష్యసి} & \romline{saṅgrāmaṃ na kariṣyasi} \\
\natline{తతః స్వధర్మం కీర్తిం చ} & \romline{tataḥ svadharmaṃ kīrtiṃ ca} \\
\natline{హిత్వా పాపమవాప్స్యసి} & \romline{hitvā pāpamavāpsyasi}
\end{tabular}
\end{table}

\subsection*{2.34}
\begin{table}[H]
\centering
\begin{tabular}{ll}
\natline{అకీర్తిం చాపి భూతాని} & \romline{akīrtiṃ cāpi bhūtāni} \\
\natline{కథయిష్యన్తి తేఽవ్యయామ్} & \romline{kathayiṣyanti te'vyayām} \\
\natline{సమ్భావితస్య చాకీర్తిః} & \romline{sambhāvitasya cākīrtiḥ} \\
\natline{మరణాదతిరిచ్యతే} & \romline{maraṇādatiricyate}
\end{tabular}
\end{table}

\subsection*{2.35}
\begin{table}[H]
\centering
\begin{tabular}{ll}
\natline{భయాద్రణాదుపరతం} & \romline{bhayādraṇāduparataṃ} \\
\natline{మంస్యన్తే త్వాం మహారథాః} & \romline{maṃsyante tvāṃ mahārathāḥ} \\
\natline{యేషాం చ త్వం బహుమతః} & \romline{yeṣāṃ ca tvaṃ bahumataḥ} \\
\natline{భూత్వా యాస్యసి లాఘవమ్} & \romline{bhūtvā yāsyasi lāghavam}
\end{tabular}
\end{table}

