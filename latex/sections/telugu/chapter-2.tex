\subsection*{2.0}
\begin{table}[H]
\begin{tabular}{l}
\natline{ఓం శ్రీ పరమాత్మనే నమః} \\
\natline{అథ ద్వితీయోఽధ్యాయః} \\
\natline{సాఙ్ఖ్యయోగః}
\end{tabular}
\end{table}

\subsection*{2.1}
\begin{table}[H]
\begin{tabular}{l}
\natline{సంజయ ఉవాచ} \\
\natline{తం తథా కృపయావిష్టమ్} \\
\natline{అశ్రుపూర్ణాకులేక్షణమ్} \\
\natline{విషీదంతమిదం వాక్యమ్} \\
\natline{ఉవాచ మధుసూదనః}
\end{tabular}
\end{table}

\subsection*{2.2}
\begin{table}[H]
\begin{tabular}{l}
\natline{శ్రీ భగవానువాచ} \\
\natline{కుతస్త్వా కశ్మలమిదం} \\
\natline{విషమే సముపస్థితమ్} \\
\natline{అనార్యజుష్టమస్వర్గ్యమ్} \\
\natline{అకీర్తికరమ్-అర్జున}
\end{tabular}
\end{table}

\subsection*{2.3}
\begin{table}[H]
\begin{tabular}{l}
\natline{క్లైబ్యం మా స్మ గమః పార్థ} \\
\natline{నైతత్త్వయ్యుపపద్యతే} \\
\natline{క్షుద్రం హృదయదౌర్బల్యం} \\
\natline{త్యక్త్వోత్తిష్ఠ పరంతప}
\end{tabular}
\end{table}

\subsection*{2.4}
\begin{table}[H]
\begin{tabular}{l}
\natline{అర్జున ఉవాచ} \\
\natline{కథం భీశ్మమహం సంఖ్యే} \\
\natline{ద్రోణం చ మధుసూదన} \\
\natline{ఇశుభిః ప్రతియోత్స్యామి} \\
\natline{పూజార్హావరిసూదన}
\end{tabular}
\end{table}

\subsection*{2.5}
\begin{table}[H]
\begin{tabular}{l}
\natline{గురూనహత్వా హి మహానుభావాన్} \\
\natline{శ్రేయో భోక్తుం భైక్ష్యమపీహ లోకే} \\
\natline{హత్వార్థకామంస్తు గురూనిహైవ} \\
\natline{భుంజీయ భోగాన్ రుధిరప్రదిగ్ధాన్}
\end{tabular}
\end{table}

\subsection*{2.6}
\begin{table}[H]
\begin{tabular}{l}
\natline{న చైతద్విద్మః కతరన్నో గరీయః} \\
\natline{యద్వా జయేమ యది వా నో జయేయుః} \\
\natline{యానేవ హత్వా న జిజీవిషామః} \\
\natline{తేఽవస్థితాః ప్రముఖే ధార్తరాష్ట్రాః}
\end{tabular}
\end{table}

\subsection*{2.7}
\begin{table}[H]
\begin{tabular}{l}
\natline{కార్పన్యదోషోపహత-స్వభావః} \\
\natline{పృచ్ఛామి త్వాం ధర్మసమ్మూఢ-చేతాః} \\
\natline{యచ్ఛ్రేయః స్యాన్నిశ్చితం బ్రూహి తన్మే} \\
\natline{శిష్యస్తేఽహం శాధి మాం త్వాం ప్రపన్నమ్}
\end{tabular}
\end{table}

\subsection*{2.8}
\begin{table}[H]
\begin{tabular}{l}
\natline{న హి ప్రపశ్యామి మమాపనుద్యాద్} \\
\natline{యచ్ఛోకముచ్ఛోషణమ్-ఇన్ద్రియాణామ్} \\
\natline{అవాప్య భూమావసపత్నమృద్ధం} \\
\natline{రాజ్యం సురాణామపి చాధిపత్యమ్}
\end{tabular}
\end{table}

\subsection*{2.9}
\begin{table}[H]
\begin{tabular}{l}
\natline{సంజయ ఉవాచ} \\
\natline{ఏవముక్త్వా హృషీకేశం} \\
\natline{గుడాకేశః పరన్తపః} \\
\natline{న యోత్స్య ఇతి గోవిందమ్} \\
\natline{ఉక్త్వా తూష్ణీమ్ బభూవ హ}
\end{tabular}
\end{table}

\subsection*{2.10}
\begin{table}[H]
\begin{tabular}{l}
\natline{తమువాచ హృషీకేశః} \\
\natline{ప్రహసన్నివ భారత} \\
\natline{సేనయోరుభయోర్మధ్యే} \\
\natline{విషీదంతమిదం వచః}
\end{tabular}
\end{table}

\subsection*{2.11}
\begin{table}[H]
\begin{tabular}{l}
\natline{శ్రీ భగవానువాచ} \\
\natline{అశోచ్యానన్వశోచస్-త్వం} \\
\natline{ప్రజ్ఞావాదాంశ్చ భాషసే} \\
\natline{గతాసూనగతాసూంశ్-చ} \\
\natline{నానుశోచన్తి పణ్డితాః}
\end{tabular}
\end{table}

\subsection*{2.12}
\begin{table}[H]
\begin{tabular}{l}
\natline{న త్వేవాహం జాతు నాసం} \\
\natline{న త్వం నేమే జనాధిపాః} \\
\natline{న చైవ న భవిష్యామః} \\
\natline{సర్వే వయమతః పరమ్}
\end{tabular}
\end{table}

\subsection*{2.13}
\begin{table}[H]
\begin{tabular}{l}
\natline{దేహినోఽస్మిన్ యథా దేహే} \\
\natline{కౌమారం యౌవనం జరా} \\
\natline{తథా దేహాంతరప్రాప్తిః} \\
\natline{ధీరస్తత్ర న ముహ్యతి}
\end{tabular}
\end{table}

\subsection*{2.14}
\begin{table}[H]
\begin{tabular}{l}
\natline{మాత్రాస్పర్శాస్-తు కౌంతేయ} \\
\natline{శీతోష్ణసుఖ-దుఃఖ-దాః} \\
\natline{ఆగమాపాయినోఽనిత్యాః} \\
\natline{తాంస్తితిక్షస్వ భారత}
\end{tabular}
\end{table}

\subsection*{2.15}
\begin{table}[H]
\begin{tabular}{l}
\natline{యం హి న వ్యథయంత్యేతే} \\
\natline{పురుషం పురుషర్షభ} \\
\natline{సమదుఃఖ-సుఖం ధీరం} \\
\natline{సోఽమృతత్వాయ కల్పతే}
\end{tabular}
\end{table}

\subsection*{2.16}
\begin{table}[H]
\begin{tabular}{l}
\natline{నాసతో విద్యతే భావః} \\
\natline{నాభావో విద్యతే సతః} \\
\natline{ఉభయోరపి దృష్తోఽన్తః} \\
\natline{త్వనయోస్తత్త్వ-దర్శిభిః}
\end{tabular}
\end{table}

\subsection*{2.17}
\begin{table}[H]
\begin{tabular}{l}
\natline{అవినాశి తు తద్విద్ధి} \\
\natline{యేన సర్వమిదం తతమ్} \\
\natline{వినాశమవ్యయస్యాస్య} \\
\natline{న కశ్చిత్కర్తుమ్-అర్హతి}
\end{tabular}
\end{table}

\subsection*{2.18}
\begin{table}[H]
\begin{tabular}{l}
\natline{అంతవన్త ఇమే దేహాః} \\
\natline{నిత్యస్యోక్తాః శరీరిణః} \\
\natline{అనాశినోఽప్రమేయస్య} \\
\natline{తస్మాద్యుధ్యస్వ భారత}
\end{tabular}
\end{table}

\subsection*{2.19}
\begin{table}[H]
\begin{tabular}{l}
\natline{య ఏనం వేత్తి హన్తారం} \\
\natline{యశ్చైనం మన్యతే హతం} \\
\natline{ఉభౌ తౌ న విజానీతః} \\
\natline{నాయం హన్తి న హన్యతే}
\end{tabular}
\end{table}

\subsection*{2.20}
\begin{table}[H]
\begin{tabular}{l}
\natline{న జాయతే మ్రియతే వా కదాచిత్} \\
\natline{నాయం భూత్వా భవితా వా న భూయః} \\
\natline{అజో నిత్యః శాశ్వతోఽయం పురాణః} \\
\natline{న హన్యతే హన్యమానే శరీరే}
\end{tabular}
\end{table}

\subsection*{2.21}
\begin{table}[H]
\begin{tabular}{l}
\natline{వేదావినాశినం నిత్యం} \\
\natline{య ఏనమజమవ్యయమ్} \\
\natline{కథం స పురుషః పార్థ} \\
\natline{కం ఘాతయతి హంతి కమ్}
\end{tabular}
\end{table}

\subsection*{2.22}
\begin{table}[H]
\begin{tabular}{l}
\natline{వాసాంసి జీర్ణాని యథా విహాయ} \\
\natline{నవాని గృహ్ణాతి నరోఽపరాణి} \\
\natline{తథా శరీరాణి విహాయ జీర్ణాని} \\
\natline{అన్యాని సంయాతి నవాని దేహీ}
\end{tabular}
\end{table}

\subsection*{2.23}
\begin{table}[H]
\begin{tabular}{l}
\natline{నైనం ఛిన్దన్తి శస్త్రాణి} \\
\natline{నైనం దహతి పావకః} \\
\natline{న చైనం క్లేదయన్త్యాపః} \\
\natline{న శోషయతి మారుతః}
\end{tabular}
\end{table}

\subsection*{2.24}
\begin{table}[H]
\begin{tabular}{l}
\natline{అచ్ఛేద్యోఽయమ్ అదాహ్యోఽయమ్} \\
\natline{అక్లేద్యోఽశోష్య ఏవ చ} \\
\natline{నిత్యః సర్వగతః స్థాణుః} \\
\natline{అచలోఽయం సనాతనః}
\end{tabular}
\end{table}

\subsection*{2.25}
\begin{table}[H]
\begin{tabular}{l}
\natline{అవ్యక్తోఽయమ్ అచిన్త్యోఽయమ్} \\
\natline{అవికార్యోఽయముచ్యతే} \\
\natline{తస్మాదేవం విదిత్వైనం} \\
\natline{నానుశోచితుమర్హసి}
\end{tabular}
\end{table}

\subsection*{2.26}
\begin{table}[H]
\begin{tabular}{l}
\natline{అథ చైనం నిత్యజాతం} \\
\natline{నిత్యం వా మన్యసే మృతమ్} \\
\natline{తథాఽపి త్వం మహాబాహో} \\
\natline{నైవం శోచితుమర్హసి}
\end{tabular}
\end{table}

\subsection*{2.27}
\begin{table}[H]
\begin{tabular}{l}
\natline{జాతస్య హి ధ్రువో మృత్యుః} \\
\natline{ధ్రువం జన్మ మృతస్య చ} \\
\natline{తస్మాదపరిహార్యేఽర్థే} \\
\natline{న త్వం శోచితుమర్హసి}
\end{tabular}
\end{table}

\subsection*{2.28}
\begin{table}[H]
\begin{tabular}{l}
\natline{అవ్యక్తాదీని భూతాని} \\
\natline{వ్యక్తమధ్యాని భారత} \\
\natline{అవ్యక్తనిధనాన్యేవ} \\
\natline{తత్ర కా పరిదేవనా}
\end{tabular}
\end{table}

\subsection*{2.29}
\begin{table}[H]
\begin{tabular}{l}
\natline{ఆశ్చర్యవత్ పశ్యతి కశ్చిదేనమ్} \\
\natline{ఆశ్చర్యవద్ వదతి తథైవ cఆన్యః} \\
\natline{ఆశ్చర్యవచ్చైనమన్యః శృణోతి} \\
\natline{శ్రుత్వాప్యేనం వేద న చైవ కశ్చిత్}
\end{tabular}
\end{table}

\subsection*{2.30}
\begin{table}[H]
\begin{tabular}{l}
\natline{దేహీ నిత్యమవధ్యోఽయం} \\
\natline{దేహే సర్వస్య భారత} \\
\natline{తస్మాత్సర్వాణి భూతాని} \\
\natline{న త్వం శోచితుమర్హసి}
\end{tabular}
\end{table}

\subsection*{2.31}
\begin{table}[H]
\begin{tabular}{l}
\natline{స్వధర్మమపి చావేక్ష్య} \\
\natline{న వికమ్పితుమర్హసి} \\
\natline{ధర్మ్యాద్ధి యుద్ధాచ్ఛ్రేయోఽన్యత్} \\
\natline{క్షత్రియస్యన-విద్యతే}
\end{tabular}
\end{table}

\subsection*{2.32}
\begin{table}[H]
\begin{tabular}{l}
\natline{యదృచ్ఛయా చోపపన్నం} \\
\natline{స్వర్గద్వార-మపావృతమ్} \\
\natline{సుఖినః క్షత్రియాః పార్థ} \\
\natline{లభన్తే యుద్ధమీదృశమ్}
\end{tabular}
\end{table}

\subsection*{2.33}
\begin{table}[H]
\begin{tabular}{l}
\natline{అథ చేత్త్వమిమం ధర్మ్యం} \\
\natline{సఙ్గ్రామం న కరిష్యసి} \\
\natline{తతః స్వధర్మం కీర్తిం చ} \\
\natline{హిత్వా పాపమవాప్స్యసి}
\end{tabular}
\end{table}

\subsection*{2.34}
\begin{table}[H]
\begin{tabular}{l}
\natline{అకీర్తిం చాపి భూతాని} \\
\natline{కథయిష్యన్తి తేఽవ్యయామ్} \\
\natline{సమ్భావితస్య చాకీర్తిః} \\
\natline{మరణాదతిరిచ్యతే}
\end{tabular}
\end{table}

\subsection*{2.35}
\begin{table}[H]
\begin{tabular}{l}
\natline{భయాద్రణాదుపరతం} \\
\natline{మంస్యన్తే త్వాం మహారథాః} \\
\natline{యేషాం చ త్వం బహుమతః} \\
\natline{భూత్వా యాస్యసి లాఘవమ్}
\end{tabular}
\end{table}

\subsection*{2.36}
\begin{table}[H]
\begin{tabular}{l}
\natline{అవాచ్యవాదాంశ్చ బహూన్} \\
\natline{వదిష్యన్తి తవాహితాః} \\
\natline{నిన్దన్తస్తవ సామర్థ్యం} \\
\natline{తతో దుఃఖతరం ను కిమ్}
\end{tabular}
\end{table}

\subsection*{2.37}
\begin{table}[H]
\begin{tabular}{l}
\natline{హతో వా ప్రాప్స్యసి స్వర్గం} \\
\natline{జిత్వా వా భోక్ష్యసే మహీమ్} \\
\natline{తస్మాదుత్తిష్ఠ కౌన్తేయ} \\
\natline{యుద్ధాయ కృతనిశ్చయః}
\end{tabular}
\end{table}

\subsection*{2.38}
\begin{table}[H]
\begin{tabular}{l}
\natline{సుఖదుఃఖే సమే కృత్వా} \\
\natline{లాభాలాభౌ జయాజయౌ} \\
\natline{తతో యుద్ధాయ యుజ్యస్వ} \\
\natline{నైవం పాపమవాప్స్యసి}
\end{tabular}
\end{table}

\subsection*{2.39}
\begin{table}[H]
\begin{tabular}{l}
\natline{ఏషా తేఽభిహితా సాఙ్ఖ్యే} \\
\natline{బుద్ధిర్యోగే త్విమాం శృణు} \\
\natline{బుద్ధ్యా యుక్తో యయా పార్థ} \\
\natline{కర్మబన్ధం ప్రహాస్యసి}
\end{tabular}
\end{table}

\subsection*{2.40}
\begin{table}[H]
\begin{tabular}{l}
\natline{నేహాభిక్రమనాశోఽస్తి} \\
\natline{ప్రత్యవాయో న విద్యతే} \\
\natline{స్వల్పమప్యస్య ధర్మస్య} \\
\natline{త్రాయతే మహతో భయాత్}
\end{tabular}
\end{table}

\subsection*{2.41}
\begin{table}[H]
\begin{tabular}{l}
\natline{వ్యవసాయాత్మికా బుద్ధిః} \\
\natline{ఏకేహ కురునన్దన} \\
\natline{బహుశాఖా హ్యనన్తాశ్చ} \\
\natline{బుద్ధయోఽవ్యసాయినామ్}
\end{tabular}
\end{table}

\subsection*{2.42}
\begin{table}[H]
\begin{tabular}{l}
\natline{యామిమాం పుష్పితాం వాcఅం} \\
\natline{ప్రవదన్త్యవిపశ్చితః} \\
\natline{వేదవాదరతాః పార్థ} \\
\natline{నాన్యదస్తీతి వాదినః}
\end{tabular}
\end{table}

\subsection*{2.43}
\begin{table}[H]
\begin{tabular}{l}
\natline{కామాత్మానః స్వర్గపరాః} \\
\natline{జన్మకర్మ-ఫల-ప్రదామ్} \\
\natline{క్రియావిశేష-బహులాం} \\
\natline{భోగైశ్వర్యగతిం ప్రతి}
\end{tabular}
\end{table}

\subsection*{2.44}
\begin{table}[H]
\begin{tabular}{l}
\natline{భోగైశ్వర్యప్రసక్తానాం} \\
\natline{తయాఽపహృతచేతసామ్} \\
\natline{వ్యవసాయాత్మికా బుద్ధిః} \\
\natline{సమాధౌ న విధీయతే}
\end{tabular}
\end{table}

\subsection*{2.45}
\begin{table}[H]
\begin{tabular}{l}
\natline{త్రైగుణ్యవిషయా వేదాః} \\
\natline{నిస్త్రైగుణ్యో భవార్జున} \\
\natline{నిర్ద్వన్ద్వో నిత్యసత్త్వస్థః} \\
\natline{నిర్యోగక్షేమ ఆత్మవాన్}
\end{tabular}
\end{table}

\subsection*{2.46}
\begin{table}[H]
\begin{tabular}{l}
\natline{యావానర్థ ఉదపానే} \\
\natline{సర్వతః సమ్ప్లుతోదకే} \\
\natline{తావాన్సర్వేషు వేదేషు} \\
\natline{బ్రాహ్మణస్య విజానతః}
\end{tabular}
\end{table}

\subsection*{2.47}
\begin{table}[H]
\begin{tabular}{l}
\natline{కర్మణ్యేవాధికారస్తే} \\
\natline{మా ఫలేషు కదాచన} \\
\natline{మా కర్మఫల-హేతుర్భూః} \\
\natline{మా తే సఙ్గోఽస్త్వకర్మణి}
\end{tabular}
\end{table}

\subsection*{2.48}
\begin{table}[H]
\begin{tabular}{l}
\natline{యోగస్థః కురు కర్మాణి} \\
\natline{సఙ్గం త్యక్త్వా ధనఞ్జయ} \\
\natline{సిద్ధ్యసిద్ధ్యోః సమో భూత్వా} \\
\natline{సమత్వం యోగ ఉచ్యతే}
\end{tabular}
\end{table}

\subsection*{2.49}
\begin{table}[H]
\begin{tabular}{l}
\natline{దూరేణ హ్యవరం కర్మ} \\
\natline{బుద్ధియోగాద్ధనఞ్జయ} \\
\natline{బుద్ధౌ శరణమన్విచ్ఛ} \\
\natline{కృపణాః ఫలహేతవః}
\end{tabular}
\end{table}

\subsection*{2.50}
\begin{table}[H]
\begin{tabular}{l}
\natline{బుద్ధియుక్తో జహాతీహ} \\
\natline{ఉభే సుకృతదుష్కృతే} \\
\natline{తస్మాద్యోగాయ యుజ్యస్వ} \\
\natline{యోగః కర్మసు కౌశలమ్}
\end{tabular}
\end{table}

